
\chapter{Introduction}
%%%
The geographic area covered in the present survey stretches from Europe (including the Mediterranean Islands Malta and Cyprus as well as the regions Anatolia and the Caucasus\is{Caucasus}), over central,\is{Inner Asia} northern,\is{North Asia} and northeastern Asia\is{Northeast Asia} (including the whole of \isi{Siberia}, the adjacent parts of northern Mongolia) to the Islands of the northwestern Pacific Ocean. The language families represented in this area are genealogically categorized by \citet{salminen2007} in his chapter on the endangered languages of “\isi{Europe} and \isi{North Asia}”. By and large, Salminen's inventory of languages will be followed here. However, the present survey strictly follows the geography of northern Eurasia and consequently also includes Siberian Yupik Eskimo,\il{Siberian Yupik Eskimo} Ainu,\il{Ainu} the Sino-Tibetan language Dungan,\il{Dungan} and some Semitic\il{Semitic languages} languages.

\section{The languages of northern Eurasia}
%%%
Adopting Salminen's rather cautious genealogical classification the following families and isolates are considered (roughly from Northeast to Southwest):
%%%
\begin{multicols}{2}
\begin{enumerate}
\item{Eskimo-Aleut}\il{Eskimo-Aleut languages}
\item{Chukotko-Kamchatkan}\il{Chukotko-Kamchatkan languages}\footnote{Salminen describes Chukotkan\il{Chukotkan languages} and Kamchatkan\il{Kamchatkan languages} as two separate language families.}
\item{Nivkh}\il{Nivkh}
\item{Ainu}\il{Ainu}
\item{Japanese}\il{Japanese}
\item{Korean}\il{Korean}
\item{Sino-Tibetan}\il{Sino-Tibetan languages}
\item{Mongolic}\il{Mongolic languages}
\item{Tungusic}\il{Tungusic languages}
\item{Yukaghir}\il{Yukaghir languages}
\item{Yeniseian}\il{Yeniseian languages}
\item{Turkic}\il{Turkic languages}
\item{Nakh-Daghestanian}\il{Nakh-Daghestanian languages}
\item{Abkhaz-Adyghe}\il{Abkhaz-Adyghe languages}
\item{Kartvelian}\il{Kartvelian languages}
\item{Semitic}\il{Semitic languages}
\item{Uralic}\il{Uralic languages}
\item{Indo-European}\il{Indo-European languages}
\item{Basque}\il{Basque}
\end{enumerate}
\end{multicols}
%%%

Even though some of these genealogical units have been assumed to combine to larger stocks (such as Altaic,\il{Altaic languages} North Caucasian\il{North Caucasian languages} and others) the restriction to completely uncontroversial units seems adequate for the present areal typological investigation. This is especially true since an attempt is made to map variation inside genealogical units rather than to evaluate a statistically balanced genealogical sample of languages.

%Andreas Hölzl
%Nikh has several mutually unintelligible "dialects"
%Japanese is better called Japanese-Ryukyuan (or maybe Japonic) as there are at least two main branches of which Japanese is only one; Japanese itself has enough variation to consider it a small language family (e.g., Hachijo); Ryukyuan accordig to one view has the following complex structure: 1 Northern, 1.1 Amami (e.g. Okinoerabu), 1.2 Okinawan (e.g., Shuri), 2 Southern, 2.1 Miyako (e.g. Ogami), 2.2 Macro-Yaeyama, 2.2.1 Yaeyama (e.g. Hateruma), 2.2.2 Yonaguni (e.g. Dunan)
%there are many Mandarin dialects that are spoken at least as far north as Dungan (e.g. Northeast China, Xinjiang) and should thus be included
%the classification of Tungusic given below is somewhat outdated, today most scholars agree in two main branches and four subbranches. in the terminology of Janhunen (2012), these are: 1 Northern Tungusic, 1.1 Ewenic (e.g. Evenki), 1.2 Udegheic (e.g. Udihe), 2 Southern Tungusic, 2.1 Nanaic (e.g. Nanai), 2.2 Jurchenic (e.g. Manchu)
%perhaps better called Yukaghiric as there are still two rather different languages
%Maltese is located as far south as is Amdo Tibetan, Mandarin, Korean, and Japanese
%in Northeast Asia there are several mixed languages and Pidgins that are not easily classified here: Govorka, Chinese Pidgin Russian, Ejnu, Mednyj Aleut

Tungusic is also spoken in northern Siberia (Even, Evenki, formerly Arman) but there are almost no speakers in Mongolia

\section{The language sample}
%%%
All attested adjective attribution marking devices of languages mentioned in the present study are coded in a table in the Appendix.\footnote{The table is derived from \citet{AUTOTYP-NP} where these languages are coded for noun phrase patterns.} This table thus includes a relatively complete list of languages from the northern Eurasian area. At least one representative of each existing taxon is found in that sample. Additionally, several languages from within or outside the area (all of which are mentioned in other chapters of this investigation) or even other languages on which information was easily accessible are coded.

All languages are sorted alphabetically according to their genealogical affiliation. For each of the languages, the attested noun phrase type(s) relevant to adjective attribution marking are listed.

\section{The language maps}
%%%
The language maps have been generated using the data coded in the language sample in the Appendix.

\subsection[Geographic coding]{Data points for geographic coding}
%%%
Each language is displayed as one data point. The corresponding geographic coordinates have either been taken from \citet{walsOnline2013}\is{WALS} or were included using the language coordinates provided by \citet{AUTOTYP}.\is{AUTOTYP} For some languages, which were missing in the mentioned databases, new coordinates had to be defined based on the main geographic location where the respective languages are spoken.

Displaying the distribution of a given feature by means of a borderline around a group of languages~– like in the maps used by typological surveys of the EUROTYP\hyp{}project\footnote{\url{http://www.degruyter.com/view/serial/16329} (Accessed 2016-07-19)}~– was not preferred because these maps might imply the existence of isoglosses around continuous language and dialect areas. A typological survey of non-continuos languages seems rather inadequate for drawing such isoglosses.\footnote{Cf.~also Van Pottelberge's \citeyear{van-pottelberge2001} critique of the “name maps” used by EUROTYP. Furthermore, the EUROTYP language sample is somewhat arbitrary. The western Romance\il{West Romance languages} varieties, for instance, are represented in large number whereas varieties of Balkan Romance\il{Balkan Romance languages} (Megleno-Romanian,\il{Megleno-Romanian} Aromunian,\il{Aromunian} etc.) are missing completely. Also the whole Saamic\il{Saamic languages} branch is represented in the \isi{EUROTYP} sample as one single language only even though Saamic\il{Saamic languages} languages are as diverse as Romance\il{Romance languages} languages.}

\subsection[Type coding]{Data points for type coding}
%%%
In several languages more than one default attribution marking device occurs, for example in \ili{Albanian} (see \S\ref{albanian synchr}) where two lexical classes of adjectives exist: one of them marked for \isi{head\hyp{}driven agreement}, the other simultaneously marked for \isi{head\hyp{}driven agreement} and \isi{attributive nominalization}. In the map's legend, a slash marks the occurrence of multiple basic types in one language: \ili{Albanian} \textit{HDrAgr/Nmlz+HDrAgr}.\footnote{Type abbreviations are explained in the Appendix.}

Parentheses denote secondary types of attribution marking devices with additional semantic content, as in \ili{Chuvash} (see \S\ref{chuvash synchr}), where attributive adjectives are normally juxtaposed but can alternatively be marked for \isi{attributive nominalization} in contrastive focus constructions: \textsc{Chuvash}\il{Chuvash} \textit{Juxt(Nmlz)}.

Square brackets are used for languages where the occurrence of a given type of attribution marking device seems even more restricted or if the device's characteristics remain uncertain due to inadequate data. Consider for example \ili{Turkish} (see \S\ref{turkish synchr}), where \isi{attributive nominalization} occurs as a secondary type but is restricted to \isi{headless noun phrase}s in direct object position (marked for accusative): \textsc{Turkish}\il{Turkish} \textit{Juxt[Nmlz]}. 
Secondary and tertiary types are not coded in the maps. 

\subsection{The maps}
%%%
The maps in Figure~\ref{WorldMap} and Figure~\ref{WorldMapTyp} show the distribution of different adjective attribution marking devices across those world's languages mentioned in the present study. Whereas all types are coded with different colors or shapes in Figure~\ref{WorldMap}, a similar language sample is coded only for the main morpho-syntactic types (\isi{juxtaposition}, agreement, attributive state, \isi{incorporation}) in Figure~\ref{WorldMapTyp}. Note that these world maps do not reflect systematic sampling but are rather the result of random choice due to my work with data coded for the noun phrase structure module of \isi{AUTOTYP} \citep{AUTOTYP-NP}. Note also that the maps show fewer languages from the northern Eurasian area than are actually coded in the language sample in the Appendix.

The other pairs of maps are coded similarly but zoom in on northern Eurasia (Figure~\ref{NEMap} and Figure~\ref{NEMapTyp}), on \isi{North Asia} (Figure~\ref{NAMap} and Figure~\ref{NAMapTyp}) and on Europe (Figure~\ref{EUMap} and Figure~\ref{EUMapTyp}). Whereas the maps of northern Eurasia and \isi{North Asia} show only representatives for the known single taxa, the maps of Europe present a more complete picture. The reason for displaying a deeper resolution in the European map is the easier accessibility of data for almost all existing languages of that area. Displaying a similar deep resolution on the whole northern Eurasian area was not possible due to lack of data for several languages.

In order to present a balanced picture, several European languages are thus not displayed in the larger map of northern Eurasia. When a choice had to be made whether or not to keep a language inside a given taxon, this was always done in favor of diversity rather than uniformity. One taxon can even be represented by more than one language in order to display extraordinary diversity inside that group of closely related languages. Consequently, the northern branch of Germanic\il{North Germanic languages} is represented by Icelandic\il{Icelandic} (with \textit{HDrAgr}), Swedish\il{Swedish} (with \textit{ACAgr+HDrAgr/HDrAgr}) and Västerbotten Swedish\il{Swedish!Västerbotten} (with \textit{Inc/HDrAgr}) (\S\ref{n-germanic synchr}).

The choice to let the maps illustrate the highest possible diversity instead of displaying a genealogically and geographically balanced picture is justified by the general goal of the present investigation, namely the synchronic and diachronic mapping of cross-linguistically attested adjective attribution marking devices in a geographically restricted area. Whereas the mapping of synchronically attested diversity is the aim of the present part, Part~IV (Diachrony) will inspect this diversity form a diachronic perspective.
