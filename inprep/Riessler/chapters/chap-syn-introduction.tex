
\chapter{Introduction}
%%%
The geographic area covered in the present survey stretches from Europe (including the Mediterranean Islands Malta and Cyprus as well as the regions Anatolia and Caucasia), over central, northern, and northeastern Asia (including the whole of Siberia, the adjacent parts of northern Mongolia and Manchuria) to the Islands of the northwestern Pacific Ocean. The language families represented in this area are genealogically categorized by \cite{salminen2007} in his chapter on the endangered languages of “Europe and North Asia”. By and large, Salminen's index of languages will be followed here. However, the present survey strictly follows the geography of northern Eurasia and consequently also includes Siberian Yupic Eskimo, Ainu, the Sinotibetan language Dungan, and some Semitic languages.

\section{The languages of northern Eurasia}
%%%
Adopting Salminen's rather cautious genealogical classification the following families are considered (roughly from Northeast to Southwest):
%%%
\begin{multicols}{2}
\begin{enumerate}
\item{Eskimo-Aleut}
\item{Chukotkan}
\item{Kamchatkan}
\item{Nivkh}
\item{Ainu}
\item{Japanese}
\item{Korean}
\item{Sinotibetan}
\item{Mongolic}
\item{Tungusic}
\item{Yukagir}
\item{Yeniseian}
\item{Turkic}
\item{Nakh-Daghestanian}
\item{Abkhaz-Adyge}
\item{Kartvelian}
\item{Semitic}
\item{Uralic}
\item{Indoeuropean}
\item{Basque}
\end{enumerate}
\end{multicols}
%%%
Even though some of these genealogical units have been assumed to combine to larger stocks (as Altaic, Chukotko-Kamchatkan, North-Caucasian and other) the restriction to uncontroversial units seems adequate for the present areal typological investigation. This is especially true since an attempt is made to map variation inside genealogical units rather than to evaluate a statistically balanced genealogical sample of languages.

\section{The language sample}
%%%
All attested adjective attribution marking devices of languages mentioned in the present study are coded in Table \ref{sample} in the appendix.\footnote{The table is derived from \citet{AUTOTYP-NP} where these languages are coded for noun phrase patterns.} This table thus includes a relatively complete list of languages from the northern Eurasian area. At least one representative of each existing genera is found in that sample. Additionally, several languages from within or outside the area (all of which are mentioned in other chapters of this investigation) or even other languages on which information was easily accessible are coded.

All languages are sorted alphabetically according to their genealogical affiliation. For each of the languages, the attested noun phrase type(s) relevant to adjective attribution marking are listed.

\section{The language maps}
%%%
The language maps have been generated using the interactive reference tool 
for the World Atlas of Language Structures \citep{bibiko2005}. 

\subsection[Geographic coding]{Data points for geographic coding}
%%%
Each language is displayed as one data point. The respective geographic coordinates have either been taken from \cite{WALS} or were included using the language coordinates provided by \cite{AUTOTYP} or on Ljuba Veselinova's website.\footnote{\url{http://www.ling.su.se/staff/ljuba/ 16.02.2014}} For some languages missing in the mentioned databases new coordinates had to be defined based on the main geographic location where the respective languages are spoken.

Displaying the distribution of a given feature by means of a borderline around a group of languages – as in the maps used by typological surveys of the EUROTYP-project\footnote{\url{http://www.eva.mpg.de/lingua/tools-at-lingboard/questionnaire/eurotyp-guidelines/ 16.02.2014}} – was not preferred because these maps might imply the existence of isoglosses around continuous language and dialect areas. A typological survey of non-continous languages seems rather inadequate for drawing such isoglosses.\footnote{Cf.~also Van Pottelberge's \citeyear{van-pottelberge2001} critique of EUROTYP's “name maps”. Furthermore, the Eurotyp sample of languages are somewhat arbitrary. The western Romance varieties, for instance, are represented in large number whereas varieties of Balkan Romance (Megleno-Rumanian, Aromunian, etc.) are missing completely. Also the whole Saamic branch is represented in the Eurotyp sample as one single language only even though Saamic languages are as comparably diverse as Romance languages.}

\subsection[Type coding]{Data points for type coding}
%%%
\todo{Farben markieren verschiedene Typen; Formen markieren Untertypen (secondary types)}
In several languages more than one default attribution marking device occurs, for example in Albanian (cf.~chapter \ref{albanian synchr}) where two lexical classes of adjectives exist: one of them marked for head-driven agreement, the other simultaneously marked for head-driven agreement and attributive nominalization. In the map's legend, a slash marks the occurrence of multiple basic types in one language: \textsc{Albanian} \textit{HDrAgr/Nmlz+HDrAgr}.\footnote{Type abbreviations are explained in Table \ref{sample} in the appendix.}

Parentheses denote secondary types of attribution marking devices with additional semantic content, as Chuvash (cf.~chapter \ref{chuvash synchr}) where attributive adjectives are normally juxtaposed but can alternatively be marked for attributive nominalization in contrastive-focus constructions: \textsc{Chuvash} \textit{Juxt(Nmlz)}.

Square parenthesis will be used for languages where the occurrence of a given type of attribution marking devices seems even more restricted or if the device's characteristics remain uncertain due to inadequate data. Consider for example Turkish (cf.~chapter \ref{turkish synchr}) where attributive nominalization occurs as a secondary type but is restricted to headless noun phrases in direct object position (marked for accusative): \textsc{Turkish} \textit{Juxt[Nmlz]}. 
Secondary and tertiary types are not coded in the maps. 

\subsection{The maps}
%%%
Maps \ref{WorldMap} and \ref{WorldMapTyp} show the distribution of different adjective attribution marking devices across those world's languages mentioned in the present study. Whereas all types are coded with different colors or shapes in Map \ref{WorldMap}, a similar language sample is coded only for the main morpho-syntactic types (juxtaposition, agreement, construct state, incorporation) in Map \ref{WorldMap}. Note that these world maps do not reflect systematic sampling but are rather the result of random choice due to my work with data coded for the noun phrase structure module of AUTOTYP \citep{AUTOTYP-NP}. Note also that the maps show fewer languages from the northern Eurasian area than actually coded in Table \ref{sample}.

The other pairs of maps are coded similarly but zoom in on northern Eurasia (Maps \ref{NEMap} and \ref{NEMapTyp}), on North-Asia (Maps \ref{NAMap} and \ref{NAMapTyp}) and on Europe (Maps \ref{EUMap} and \ref{EUMapTyp}). Whereas the maps of northern Eurasia and North-Asia show only representatives of each genera, the maps of Europe present a more complete picture. The reason for displaying a deeper resolution in the European map is the easier accessibility of data for almost all existing languages of that area. Displaying a similar deep resolution on the whole northern Eurasian area was not possible due to lack of data for several languages.

In order to present a balanced picture, several European languages are thus not displayed in the larger map of northern Eurasia. When a choice had to be made whether or not to keep a language inside a given genera, this was always done in favor of diversity rather than unity. One genera can even be represented by more than one language in order to display extraordinary diversity inside that group of closely related languages. Consequently, the northern branch of Germanic is represented by Icelandic (with \textit{HDrAgr}), Swedish (with \textit{ACAgr+HDrAgr/HDrAgr}) and Västerbotten-Swedish (with \textit{Inc/HDrAgr}) (cf. Section \ref{n-germanic synchr}).

The choice to let the maps illustrate the highest possible diversity instead of displaying a genealogically and geographically balanced picture is legitimated by the general goal of the present investigation, namely the synchronic and diachronic mapping of cross-linguistically attested adjective attribution marking devices in a geographically restricted area. Whereas the mapping of synchronically attested diversity is the aim of the present part, Part \ref{part diachr} will inspect this diversity form a diachronic perspective.
