
\chapter[The languages of northern Eurasia]{Adjective attribution marking in the languages of northern Eurasia}
%%%

The following chapter contains an overall survey of adjective attribution marking devices which occur in the languages of northern Eurasia. For each genealogical unit, both the prototypical and the known minor noun phrase type(s) will be characterized and illustrated with examples. A complete list of adjective attribution marking devices in over 200 single languages considered for the present survey is found in a table starting on page~\pageref{sample1} in the Appendix. The geographic spread of the different noun phrase types is shown on several maps starting on page~\pageref{WorldMap} in the Appendix.

\il{Eskimo-Aleut languages|(}
\il{Eskimo languages|(}
\il{Yupik languages|(}
\il{Central Siberian Yupik|(}
\section{Eskimo-Aleut (Central Siberian Yupik)}
%%%
Whereas most languages of the Eskimo-Aleut family are spoken on islands in the Bering Strait or on the \isi{North America}n continent, a few varieties of the Yupik subbranch of Eskimo can be localized to north-easternmost \isi{Siberia}. But only one of these languages, Central Siberian Yupik, is still spoken \citep[224]{salminen2007}.

In Central Siberian Yupik, only one adjective attribution marking device is attested:
%%%
\begin{itemize}
\item incorporation.
\end{itemize}

\paragraph*{Adjective incorporation in Central Siberian Yupik}
Items that correspond to property-denoting words in other languages (“adjectives”) are phonologically bound nominal roots in Central Siberian Yupik. Adjectival modification is thus expressed by means of polysynthetic morphology and can be characterized as adjective incorporation according to the ontology presented in Part~II (Typology).
%%%
\begin{exe}
\ex {\rm Central Siberian Yupik \citep{de-reuse1994}}
\begin{xlist}
\ex
\gll	qawaagpag\textbf{-rukutaagh-\underline{gh}llag}-Ø\\
	legendary\_big\_bird-huge.\textsc{noun}-big.\textsc{noun}-\textsc{abs}\\
\glt	‘huge big (legendary large) bird’ (54)
\ex	
\gll	mangteghagh\textbf{-\underline{gh}llag}-lgu-uq\\
	house-big.\textsc{noun}-have.\textsc{noun}-\textsc{ind}(3s)\\
\glt	‘He has a big house.’ (55)
\ex	
\gll	mangteghagh\textbf{-\underline{gh}rugllag}-ngllagh-yug-nghit°e-unga\\
	house-big.\textsc{noun}-make.\textsc{noun}-want\_to.\textsc{verb}-\textsc{neg}-\textsc{ind(1s)}\\
\glt	‘I did not want to make a big house.’ (56)
\end{xlist}
\end{exe}
%%%
\il{Yupik languages|)}
\il{Central Siberian Yupik|)}
\il{Eskimo languages|)}
\il{Eskimo-Aleut languages|)}

\il{Chukotko-Kamchatkan languages|(}
\section{Chukotko-Kamchatkan}
Although \cite{salminen2007} describes Chukotkan and Kamchatkan as two independent families, most scholars today agree that they are two branches of one family \citep[see also the comparative dictionary by][]{fortescue2005a}.

\il{Chukotkan languages|(}
\subsection{Chukotkan}
%%%
The Chukotkan branch (aka Chukchi-Koryak)\il{Chukchi-Koryak|see{Chukotkan languages}} of Chukotko-Kamchatkan consists of two sub-branches. The first branch, Chukchi, is represented by only one language, Chukchi proper. The second branch, Koryak-Alutor,\il{Koryak-Alutor languages} is represented by the two languages Alutor\il{Alutor} and Koryak\il{Koryak} proper. A third branch, Kerek,\il{Kerek languages} is probably extinct \citep[253]{salminen2007} and consequently not considered here.

Constituent order inside the noun phrase of Chukotkan languages is strictly head-final. Adjective attribution marking is also similar in all Chukotkan languages. Two types are attested:
%%%
\begin{itemize}
\item incorporation
\item head\hyp{}driven agreement.\is{head\hyp{}driven agreement}
\end{itemize}

\il{Chukchi languages|(}
\subsubsection{Chukchi}
%%%
\paragraph*{Adjective incorporation in Chukchi}
The use of the bound adjective morpheme in the polysynthetic structure (similar to Yupik)\il{Yupik languages} is illustrated in the following examples.\footnote{The vowel -ə- in these and the following examples is epenthetic.}
%%%
\begin{exe}
\ex {\rm Chukchi \citep{skorik1960}}
\begin{xlist}
\ex	
\gll	\textbf{elg-}ə-qoranə\\
	white-ə-deer:\textsc{abs.sg}\\
\glt	‘white reindeer’
\ex
\gll	\textbf{elg-}ə-qorat\\
	white-ə-deer:\textsc{abs.pl}\\
\glt	‘white reindeer (pl.)’
\end{xlist}
\end{exe}
\il{Chukchi languages|)}

\il{Koryak-Alutor languages|(}
\subsubsection{Koryak}
%%%
\paragraph*{Adjective incorporation in Alutor}
Similar to Chukchi,\il{Chukchi} adjective incorporation is the default adjective attribution marking device in Alutor.
%%%
\begin{exe}
\ex {\rm Alutor \citep{nagayama2003}}
\begin{xlist}
\ex
\gll	\textbf{meŋ-}ə-rara-ŋa\\
	big-ə-house-\textsc{abs.sg}\\
\glt	‘big house’
\ex
\gll	\textbf{meŋ-}ə-rara-wwi\\
	big-ə-house-\textsc{abs.pl}\\
\glt	‘big houses’
\end{xlist}
\end{exe}
\il{Koryak-Alutor languages|)}

\il{Chukchi|(}\il{Alutor|(}
\is{head\hyp{}driven agreement|(}
\paragraph*{Head\hyp{}driven agreement in Chukchi and Alutor}
Whereas adjective incorporation is the default and unmarked type of adjective attribution marking in Alutor and Chukchi, several descriptions of the Chukotkan languages mention that adjectives can also occur in an unbound form (for Alutor, see \citealt{nagayama2003}; for Chukchi, see \citealt[103–104, 421–429]{skorik1960} and \citealt[251]{comrie1981}). As unbound morphemes, adjectives take the stative marker \textit{n-} as well as agreement markers for person, number and case.
%%%
\begin{exe}
\ex
\label{chukchi alutor free adj}
\begin{xlist}
\ex {\rm Chukchi \citep{skorik1960}}
\begin{xlist}
\ex
\gll	\textbf{n-ilg-ə-qin-Ø} qoranə\\
	\textsc{stat}-white-ə-\textsc{3sg} deer:\textsc{abs.sg}\\
\glt	‘white reindeer’
\ex
\gll	\textbf{n-ilg-ə-qine-t} qorat\\
	\textsc{stat}-white-ə-3-\textsc{pl} deer:\textsc{abs.pl}\\
\glt	‘white reindeer (pl.)’
\end{xlist}
\ex {\rm Alutor \citep{nagayama2003}}
\begin{xlist}
\ex
\gll	\textbf{n-ə-meŋ-ə-qin} rara-ŋa\\
	\textsc{stat}-ə-big-ə-\textsc{abs:3sg} house-\textsc{abs.sg}\\
\glt	‘big house’
\ex
\gll	\textbf{n-ə-meŋ-ə-laŋ} rara-wwi\\
	\textsc{stat}-ə-big-ə-\textsc{abs:3pl} house-\textsc{abs.pl}\\
\glt	‘big houses’
\end{xlist}
\end{xlist}
\end{exe}
%%%
The number/person/case-agreement suffixes of adjectives and the suffixes marking possessive inflection of nouns belong to one and the same paradigm. Consequently, one could also interpret the Alutor and Chukchi data as another instance of \isi{modifier\hyp{}headed possessor agreement} (as in \ili{Oroch}, described in \S\ref{ModheadAgr}). If so, the examples in \REF{chukchi alutor free adj} should be translated literally as ‘reindeer's whiteness’, ‘house's bigness’. An analysis avoiding syntactic dependency reversal between noun and adjective \citep[cf.][]{malchukov2000}, however, is preferred here for two reasons: the first reason is the constituent order inside the noun phrase. The assumed head shift to a modifier\hyp{}headed possessor agreement construction would violate the otherwise strictly head-final constituent order rule in Alutor and Chukchi. 

The other reason arguing against syntactic head shift between noun and adjective is that in order to use non-incorporating constructions as in the examples in \REF{chukchi alutor free adj}, the adjective is first transformed into a \isi{stative verb} by means of a verbalizing prefix (\textit{-n}, glossed as \textsc{stat} in example \ref{chukchi alutor free adj}).

The verbalizer together with the agreement affix is sometimes glossed as an adjectivizing\is{adjective derivation} circumfix (\textsc{adjz>-\dots-<adjz:agr}), for instance in Nagayama's (\citeyear{nagayama2003}) grammatical description of Alutor. The given noun phrase type should then perhaps be analyzed as attributive state marking (as in Russian,\il{Russian} see \S\S\ref{anti-constr agr}, \ref{russian synchr}). Unlike in Russian, however, the same agreement marking as in attributive constructions shows up on predicates as well.
%%%
\begin{exe}
\ex {\rm Alutor \citep{nagayama2003}}
\begin{xlist}
\ex
\gll	\textbf{n-ə-tur-}iɣəm\\
	\textbf{\textsc{stat}-ə-young-}\textsc{1sg}\\
\glt	‘I'm young’
\end{xlist}
\end{exe}
%%%
Consequently, an analysis of adjective attribution marking in Alutor and Chukchi as belonging to the state-marking type is rejected.

The semantic difference between the two constructions, with adjective incorporation on the one hand and head\hyp{}driven agreement marking on the other hand, is not clear. Whereas adjective incorporation is often described as the main or even only possible type (for Chukchi, see \citealt[37, 101]{kampfe-etal1995}), \citet[288]{kibrik-etal2000} state that this type indicates the corresponding quality or property as referring to background information in Alutor.

\newpage 
The following example from Chukchi, on the other hand, indicates that the non-incorporated adjective is used in an emphasized construction. Sentence (\ref{chukchi free adj}) was elicited by Vladimir Nedjalkov\ia{Nedjalkov, Vladimir} \citep[cited as a personal communication in][330]{rijkhoff2002} in order to find examples of multiple modifiers in one noun phrase, which seems to be avoided by speakers of Chukchi. In sentence (\ref{chukchi inc adj}) with the incorporated adjective, the speaker simply left out the demonstrative when translating into Chukchi.
%%%
\begin{exe}
\ex {\rm Chukchi \citep[Vladimir Nedjalkov, p.c., cit.][330]{rijkhoff2002}}
\begin{xlist}
\ex
\label{chukchi free adj}
\gll	əngena-t ngəroq \textbf{n-ilg-ə-qine-t} qora-t\\
	this-\textsc{pl} three \textsc{stat}-white-ə-3-\textsc{pl} deer-\textsc{pl}\\
\glt	‘these three white reindeer’
\ex
\label{chukchi inc adj}
\gll	ətlon ga-twetcha-twa-len ga-ngəron-\textbf{elg-}ə-qaa-ma\\
	\textsc{3sg} \textsc{pfct}-stand\_up-be-\textsc{3sg} \textsc{com}-white-ə-deer-\textsc{com}\\
\glt	‘He stood next to (these) three white reindeer.’
\end{xlist}
\end{exe}
%%%
\citet[716]{bogoras1922} states that the circum-positioned marker of the unbound adjective “sometimes corresponds to the definite\is{species marking!definite} article or designates an object as referred to before.” The unbound adjective, on the other hand, can only occur in absolutive case which is inherently connected to semantic definiteness\is{species marking!definite} \citep[cf.][207, passim]{dunn1999}.
\il{Chukchi|)}
\il{Alutor|)}
\il{Chukotkan languages|)}
\is{head\hyp{}driven agreement|)}

\il{Kamchatkan languages|(}
\il{Itelmen|(}
\subsection{Kamchatkan}
%%%
The only surviving member of the Kamchatkan branch is Itelmen (aka Western Kamchadal)\il{Western Kamchadal|see{Itelmen}} \citep[224]{salminen2007}.

The only attested type of adjective attribution marking in Itelmen is:\footnote{According to Volodin (\citeyear{volodin1997}), a few adjectives (among them Russian\il{Russian} loan adjectives) occur in \isi{juxtaposition}.}
%%%
\begin{itemize}
\item anti\hyp{}construct state agreement.
\end{itemize}

\paragraph*{Anti\hyp{}construct state agreement in Itelmen}
\label{itelmen synchr}
Constituent order inside the noun phrase of Itelmen is head-final. Adjectives form a class that is clearly syntactically distinguished from nouns: unlike the latter, adjectives are never represented by their root morphemes alone. Unlike verbs, which take TAM\is{TAM marking} markers, adjectives take adjectival morphology and are licensed either by an attributive or predicative\is{predicative marking} (adverbal) suffix (\citealt{volodin1997}, \citealt[54]{georg-etal1999}).
%%%

\newpage 
\begin{exe}
\ex
\label{itelmen ex}
{\rm Itelmen \citep{volodin1997}} 
\begin{xlist}
\ex {\rm Attributive state of adjectives}\\
\gll	thun\textbf{-lah}\\
	dark-\textsc{attr}\\
\glt	‘dark’
\ex {\rm Predicative state of adjectives}\\
\gll	thun\textbf{-k}\\
	dark-\textsc{pred}\\
\glt	‘(is) dark’
\end{xlist}
\end{exe}
%%%
Since attributive adjectives also agree in case (though restricted to instrumental case), the noun phrase type can be characterized as anti\hyp{}construct state agreement, structurally similar to the type found in Russian.\il{Russian} Consider the following example.\footnote{Note that the shape of the state marking suffix \textit{-lan'ļ} ($\leftarrow$ \textit{-lah-ļ}) is the result of a regular morpho-phonological process \citep{georg-etal1999}.}
%%%
\begin{exe}
\ex {\rm Anti\hyp{}construct state agreement in Itelmen \citep{georg-etal1999}}\\
\gll	Kəmma çasit t'-nu-qz-al-kiçen \textbf{teŋ-lan'ļ} \textbf{thalthe-ļ}, min kn-anke t-zapasa-qzo-çen.\\
	\textsc{1sg} now \textsc{1sg}>-eat-\textsc{ipfv-fut-<1sg} good-\textsc{attr:ins} meat-\textsc{ins} \textsc{rel} \textsc{1sg-dat} \textsc{1sg}-keep-\textsc{ipfv-3sg-prtc}\\
\glt	‘Now I will eat the good meat which I kept for you.’
\end{exe}
\il{Itelmen|)}
\il{Kamchatkan languages|)}
\il{Chukotko-Kamchatkan languages|)}

\il{Nivkh|(}
\section{Nivkh}
%%%
Nivkh (aka Gilyak)\il{Gilyak|see {Nivkh}} is an isolated language spoken in the far east of the Eurasian continent on Sakhalin Island in easternmost Russia \citep[222–223]{salminen2007}.

\is{head\hyp{}driven agreement|(}
The only type of adjective attribution marking attested in Nivkh is:
%%%
\begin{itemize}
\item head\hyp{}driven agreement.
\end{itemize}

\paragraph*{Head\hyp{}driven agreement in Nivkh}
Property words in Nivkh are verbal roots. As modifiers in noun phrases these adjectival verbs occur to the left of the head noun in a construction which is sometimes described as a polysynthetic structure (cf.~\citealt[16]{gruzdeva1998}; \citealt[80]{jakobson1971}, quoted by \citealt[138]{rijkhoff2002}). The reason for analyzing adjectives in Nivkh as being incorporated into the modified noun is the phonological boundedness of the constituents evidenced by regular alternations in the initial segments of the noun stem \citep[16]{gruzdeva1998}.
%%%
\begin{exe}
\ex {\rm Nivkh \citep[16]{gruzdeva1998}}
\begin{xlist}
\ex tu ‘lake’
\ex 
\gll	pily-du\\
	be\_big-lake\\
\glt	‘big lake’
\end{xlist}
\end{exe}
%%%
In her sketch grammar of Nivkh, however, \cite{gruzdeva1998} writes adjectival words consistently as morphologically unbound words.\footnote{For instance \textit{čuz pitɣy-Ø} [new book-\textsc{nom}] (19), \textit{kyla n'iɣvn̦} [high man] (33), \textit{pila eri} [big river] (38).}

Interestingly, the phonological stem alternation rules also apply to the plural inflection of nouns and their adjectival attributes by means of reduplication.\is{reduplication} The reduplicated stem of the participle \textit{t'osk̦} in \REF{nivkh redup} ‘destroyed’ is therefore realized as \textit{-zosk̦}.
%%%
\begin{exe}
\ex
\label{nivkh redup}
{\rm Nivkh (Ekaterina Gruzdeva, p.c.)}
\begin{xlist}
\ex 
\gll	tuin \textbf{t'osq-mu} hum-d'\\
	here break.\textsc{ptcp}-boat be-\textsc{ind}\\
\glt	‘there is a destroyed boat here’ 
\ex
\label{nivkh unaltered}
\gll	tuin \textbf{t'osq\textasciitilde zosk̦-mu-ɣu} hum-d'[-ɣu]\\
	here break.\textsc{ptcp}\textasciitilde \textsc{pl}-boat-\textsc{pl} be-\textsc{ind}[-\textsc{pl}]\\
\glt	‘there are destroyed boats here’
\end{xlist}
\end{exe}
%%%
Note that the number agreement of the attributive forms of adjectives by means of reduplication is archaic. According to Ekaterina Gruzdeva (p.c.),\ia{Gruzdeva, Ekaterina} attributive adjectives practically never reduplicate\is{reduplication} any more. Examples of reduplicating adjectives are, however, included in the older grammar by Panfilov (\citeyear{panfilov1965}).
\il{Nivkh|)}
\is{head\hyp{}driven agreement|)}

\il{Ainu|(}
\section{Ainu}
%%%
Ainu is an isolate spoken on Hokkaido Island in northern Japan.
%"isolate" is questionable, cf. Jeju

\is{juxtaposition|(}
The only type of adjective attribution marking attested in Ainu is:
%%%
\begin{itemize}
\item juxtaposition.
\end{itemize}

\paragraph*{Juxtaposition in Ainu}
\label{ainu synchr}
Ainu does not exhibit morphological differences between adjectives and verbs \citep[27]{refsing1986}. Words expressing states (\ref{ainu state}) or properties (\ref{aini qual}) in Ainu are best described as \isi{stative verb}s. They form a subclass of intransitive verbs and are only semantically distinguished from verbs denoting an action \citep[141–142]{refsing1986}. As modifiers of a noun, these property words are juxtaposed to the left.
%%%
\begin{exe}
\il{Ainu!Shizunai}
\ex {\rm Ainu (Shizunai) \citep{refsing1986}}
\begin{xlist}
\ex {\rm “State adjective”}\\
\label{ainu state}
\gll	\textbf{mokor} cep\\
	sleep fish\\
\glt	‘a sleeping fish’ (141)
\ex {\rm “Quality adjective”}\\
\label{aini qual}
\gll	\textbf{pirka} cep\\
	be\_good fish\\
\glt	‘a fine fish’ (142)
\end{xlist}
\end{exe}
\il{Ainu|)}
\is{juxtaposition|)}

\is{juxtaposition|(}
\il{Japanese|(}
\section{Japanese}
%%%
%"isolate" is incorrect, cf. Ryukyuan languages etc.
The noun phrase structure in Japanese, an isolated language, is strictly head-final. Two types of adjective attribution marking devices are attested:
\begin{itemize}
%%%
\item juxtaposition
\item anti\hyp{}construct state marking.
\end{itemize}
\paragraph*{Juxtaposition in Japanese}
Two distinct lexical classes of words describe the state that an entity is in. Verbal adjectives belong to the first class. These adjectives are distinguished from \isi{stative verb}s by the adjectivizer\is{adjective derivation} suffix \textit{-i}. Used as predicates, the adjectivized verbs marked with \textit{-i} follow the noun but do not require any copula. Attributive adjectives, on the other hand, are juxtaposed to the left of the modified noun. 
%%%
\begin{exe}
\ex {\rm Verbal adjectives in Japanese \citep[170]{backhouse1984}}
\begin{xlist}
\ex {\rm Adjective predication}\\
\gll	kono rombun=wa \textbf{naga-i}\\
	this article-\textsc{top} long-\textsc{adjz}\\
\glt	‘This article is long.’
\ex {\rm Adjective attribution}\\
\gll	\textbf{naga-i} rombun\\
	long-\textsc{adjz} article\\
\glt	‘long article’
\end{xlist}
\end{exe}
%%%
Since the adjectivizer\is{adjective derivation} suffix \textit{-i} simply marks \isi{stative verb} roots as (attributive and predicative) adjectives, it is not considered an attribution marking device. Hence, the class of verbal adjectives in Japanese is merely attributed by juxtaposition. Constituent order is crucial for differentiating attributive from predicative adjectives.\is{predicative marking}\footnote{Note that the description if the suffix \textit{-i} as an adjectivizer is simplified here. There is also overlap with \textsc{\isi{tense} marking}, cf.~\textit{rombun-wa naga-i} [\textsc{prs}] ‘the article is long’ versus \textit{rombun-wa naga‑kat-ta} [\textsc{pst}] ‘the article was long’.}
\is{juxtaposition|)}

\paragraph*{Anti\hyp{}construct state in Japanese}
Unlike “verbal adjectives”, which were described in the previous section, the few members of the second adjectival sub-class, i.e., “nominal adjectives” require a special attributive form marked by the invariable attributive suffix \textit{-na}.
%%%
\begin{exe}
\ex {\rm Japanese \citep[72–81]{pustet1989}}
\begin{xlist}
\ex {\rm Attribution: verbal adjective}\\
\gll	\textbf{waka-i} hito\\
	young-\textsc{adjz} person\\
\glt	‘a young person’
\ex {\rm Attribution: nominal adjective}\\
\gll	\textbf{kirei-na} hito\\
	beautiful-\textsc{attr} person\\
\glt	‘a beautiful person’
\end{xlist}
\end{exe}
%%%
Note that the word class boundary between nominal adjectives and nouns in Japanese is not always clear because some words take either the noun attribution marker \textit{-no} (\ref{japan na}) or the adjective attribution marker \textit{-na} (\ref{japan no}) when modifying a noun. The arbitrary behavior of attribution marking of nouns and nominal adjectives in Japanese indicates the continuous nature of these two word classes in this language \citep[79–80]{pustet1989}.
%%%

\newpage 
\begin{exe}
\ex {\rm Japanese \citep[72–81]{pustet1989}}
\begin{xlist}
\ex {\rm Noun attribution}\\
\label{japan na}
\gll	\textbf{wazuka-na} okane\\
	little-\textsc{attr} money\\
\glt	‘little money’
\ex {\rm Adjective attribution}\\
\label{japan no}
\gll	\textbf{wazuka-no} okane\\
	little-\textsc{attr} money\\
\glt	‘little money’
\end{xlist}
\end{exe}
\il{Japanese|)}

\il{Korean|(}
\section{Korean}
%%%
Korean is an isolated language spoken on the Korean peninsula in northeastern Asia.\is{Northeast Asia} The only type of adjective attribution marking attested in Korean is:
%%%
\begin{itemize}
\item anti\hyp{}construct state marking.
\end{itemize}
%%%
Note, however, that Korean does not have a distinct class of adjectives but adjectival notions are expressed by verbs.

\paragraph*{Anti\hyp{}construct state in Korean}
The constituent order in the noun phrase of Korean is strictly head-final. Modifying “property words” are verbs equipped with a special attributive suffix \textit{-(u)n} \citep{martin-etal1969}.
%%%
\begin{exe}
\ex {\rm Korean \citep[61]{chang1996}}
\begin{xlist}
\ex
\begin{xlist}
\ex
\gll	i \textbf{ppalka-n} chayk\\
	this be\_red-\textsc{attr} book\\
\ex	
\gll	i \textbf{ppalka-n} chayk i\\
	this be\_red-\textsc{attr} book \textsc{subj}\\
\glt	‘this red book’
\end{xlist}
\ex
\begin{xlist}
\ex	
\gll	ce \textbf{khu-n} namwu\\
	that be\_big-\textsc{attr} tree\\
\ex
\gll	ce \textbf{khu-n} namwu lul\\
	that be\_big-\textsc{attr} tree \textsc{obj}\\
\glt	‘that big tree’
\end{xlist}
\end{xlist}
\end{exe}
\il{Korean|)}

\il{Sino-Tibetan languages|(}
\il{Dungan|(}
\section{Sino-Tibetan (Dungan)}
\label{sinotibetan synchr}
%%%
The Sino-Tibetan language family is represented in northern Eurasia only by one language, Dungan (aka Dunganese),\il{Dunganese|see{Dungan}} which is a Gansu\il{Gansu languages} variety of Chinese spoken in the Kyrgyz Republic in \isi{Inner Asia} (cf.~\citealt[85]{yuo2003}, \citealt{kalimov1968}).

\is{juxtaposition|(}
Two types of adjective attribution marking are attested in Dungan:
%%%
\begin{itemize}
\item juxtaposition
\item attributive nominalization.\is{attributive nominalization}
\end{itemize}

\paragraph*{Juxtaposition in Dungan}
Adjective attribution marking in the unmarked noun phrase in Dungan is characterized by juxtaposition. Hereby, the adjective either precedes or follows the noun.
%%%
\begin{exe}
\ex {\rm Dungan \citep[480]{kalimov1968}}
\label{dungan juxtap}
\begin{xlist}
\ex 	
\gll	\textbf{da} fonzy\\
	big house\\
\ex
\gll	fonzy \textbf{da}\\
	house big\\
\glt	‘big house’
\end{xlist}
\end{exe}

\is{attributive nominalization|(}
\paragraph*{Attributive nominalization in Dungan}
A second noun phrase type with the adjectival modifier marked by a suffix \textit{-di\textsuperscript{1}} occurs in Dungan as well. Whereas juxtaposition constitutes the general and unmarked type of adjective attribution marking, the attributive suffix \textit{-di\textsuperscript{1}} seems to be much more restricted and occurs for example in connection with a comparative (\ref{dungan compattr}) or negated attribute (\ref{dungan negattr}).
\is{juxtaposition|)}
%%%
\begin{exe}
\ex {\rm Dungan}
\begin{xlist}
\ex {\rm Negated attribute \citep[80]{zevachina2001}}\\
\label{dungan negattr}
\gll	gubuə\textsuperscript{3} bu\textsuperscript{1} \textbf{da\textsuperscript{3}-di\textsuperscript{1}} gun\textsuperscript{1}fu\textsuperscript{1}\\
	went \textsc{neg} big-\textsc{attr} time\\
\glt	‘Not much (lit. ‘not big’) time passed.’
%%%
\ex {\rm Comparative attribute \citep[480]{kalimov1968}}\\
\label{dungan compattr}
\gll	\textbf{da-ščer-di} fonzy\\
	bigger-\textsc{compar}-\textsc{attr} house\\
\glt	‘a somewhat bigr (i.e., different) house’\footnote{Note that the quoted transcriptions of the two authors differ from each other.}
\end{xlist}
\end{exe}	
%%%
The marker \textit{-di\textsuperscript{1}} is clearly cognate with the functionally similar nominalizer \textit{-de} in Mandarin Chinese\il{Mandarin Chinese} (cf.~example \ref{multi mand} in \S\ref{polyfunctionality}). In Dungan, however, \textit{-di\textsuperscript{1}} is sometimes also described as a marker of predicative\is{predicative marking} adjectives, as in \REF{dungan emphpred}.
%%%
\begin{exe}
\ex {\rm Attributive nominalization in Dungan \citep[82]{zevachina2001}}\\
\label{dungan emphpred}
\gll	ž̨y\textsuperscript{3}gə\textsuperscript{1} mə\textsuperscript{1}mə\textsuperscript{2} \textbf{gan\textsuperscript{1}-di\textsuperscript{1}}\\
	this bread stale-\textsc{attr}\\
\glt	‘This bread is \textsc{stale} (i.e., different).’
\end{exe}
%%%
\citet[82]{zevachina2001} labels the function of the marker as an “emphasizing\hyp{}predicative”. But looking at her other examples it becomes obvious that \textit{-di\textsuperscript{1}} does not mark predicative adjectives but rather nominalized attributive adjectives.
%%%
\begin{exe}
\ex {\rm Attributive nominalization in Dungan \citep[82]{zevachina2001}}\\
\gll	ž̨y\textsuperscript{3}gə\textsuperscript{1} fu\textsuperscript{1} bu\textsuperscript{1}cy\textsuperscript{1} \textbf{xun\textsuperscript{1}-di\textsuperscript{1}}, \textbf{zy\textsuperscript{2}-di\textsuperscript{1}}\\
	this book \textsc{neg} red-\textsc{attr} bordeaux-\textsc{attr}\\
\glt	‘This book is not \textsc{red}, but \textsc{bordeaux}.’\\
	(lit. ‘This book is not a red one, but a bordeaux one.’)
\end{exe}
%%%
The nominalizing function of the suffix is also described by \cite{kalimov1968}.
%%%
\begin{exe}
\ex {\rm Attributive nominalization Dungan \citep[484]{kalimov1968}}\\
\label{dungan nmlz}
\gll	\textbf{ščin-di} gǔjdixyn\\
	new-\textsc{attr} expensive\\
\glt	‘The new (one) is expensive.’
\end{exe}
Attributive marking with the suffix \textit{-di\textsuperscript{1}} in Dungan needs to be investigated in more detail, especially in connection to constituent order. The head-initial structure seems to be used in order to emphasize the property denoted by the adjective.

However, according to the descriptions of Dungan taken into account here (i.e., \citealt{kalimov1968} and \citealt{zevachina2001}), the language exhibits two adjective attribution marking devices: \isi{juxtaposition} and attributive nominalization by means of the article \textit{-di\textsuperscript{1}}. While juxtaposition (with the order adjective-noun) seems to be the unmarked type, attributive nominalization is restricted to certain pragmatically marked constructions.
\is{attributive nominalization|)}
\il{Dungan|)}
\il{Sino-Tibetan languages|)}

\il{Mongolic languages|(}
\section{Mongolic}
%Mongolic-nördliches China: Dagurisch, Mongorisch, Bao'an, Donxiang, Oirat
The Mongolic language family consists of five branches (cf.~\citealt[222]{salminen2007}). The core branch, Mongolian, includes the languages Kalmyk,\il{Kalmyk} Khalkha,\il{Khalkha} Khamnigan Mongol,\il{Khamnigan Mongol} and Oyrat\il{Oyrat} (aka Oirat).\il{Oirat|see{Oyrat}} Kalmyk is spoken in easternmost \isi{Europe} (in the Republic of Kalmykia of the Russian Federation). The other Mongolian languages are all spoken in \isi{Inner Asia}, along with Dagur\il{Dagur} which belongs to a satellite branch of the Mongolic family. Languages of the remaining three satellite branches of Mongolic are not considered here since they are all spoken outside the northern Eurasian area.

\is{juxtaposition|(}
With regard to their principal noun phrase structure, all Mongolic languages of northern Eurasia exhibit the inherited \ili{Proto\hyp{}Mongolic} features, including strictly head-final constituent order and juxtaposition of attributive adjectives (“adjectival nouns”) as the only attribution marking device.

Note, however, that adjectives in Mongolic languages do not differ formally from regular nouns but are distinguishable from the latter only by their syntactic behavior and specific derivational patterns (cf.~\citealt[10]{janhunen2003b} for Proto\hyp{}Mongolic\il{Proto\hyp{}Mongolic} and \citealt[161]{svantesson2003} for Khalkha).\il{Khalkha}\footnote{In the two Mongolic languages \ili{Moghol} (spoken in Afghanistan) and Mangghuer\il{Mangghuer} (spoken in China) there is a distinct class of adjectives (cf.~\citealt[252]{weiers2003} for Moghol and \citealt[311]{slater2003} for Mangghuer). However, these languages are not considered since they are spoken outside the northern Eurasian area.}

The only type of adjective attribution marking attested in Mongolic languages of northern Eurasia is:
%%%
\begin{itemize}
\item juxtaposition.
\end{itemize}

\il{Mongolian languages|(}
\subsection{Mongolian}
%%%
\paragraph*{Juxtaposition in Khalkha}\hspace{0.4cm}
The only attested adjective attribution marking device in the languages of the Mongolian branch of Mongolic is juxtaposition, similar to the following example.
%%%
\begin{exe}
\ex {\rm Khalkha \citep{svantesson2003}}
\label{khalkha juxt}
\begin{xlist}
\ex
\gll	sayin nom\\
	good book\\
\glt	‘good book’
\ex 
\gll	sayin nom-uud\\
	good book-\textsc{pl}\\
\glt ‘good books’
\end{xlist}
\end{exe}
\il{Mongolian languages|)}

\il{Monguor languages|(}
\il{Moghol languages|(}
\il{Dagur languages|(}
\subsection{Monguor, Moghol, Dagur}
%%%
The only attested adjective attribution marking device in the languages of the Monguor, Moghol and Dagur branches of Mongolic is juxtaposition \citep{slater2003,weiers2003,tsumagari2003}, similar to example (\ref{khalkha juxt}) from Khalkha Mongolian.
\il{Monguor languages|)}
\il{Moghol languages|)}
\il{Dagur languages|)}
\il{Mongolic languages|)}
\is{juxtaposition|)}

\il{Tungusic languages|(}
\section{Tungusic}
\label{tungusic synchr}
%%%
The Tungusic language family (aka Manchu-Tungus)\il{Manchu-Tungus languages|see{Tungusic languages}} comprises several single languages belonging to the three branches North Tungusic, Amur Tungusic and Manchu, all spoken in southern \isi{Siberia} (Russia), northern Mongolia and northern China.

The constituent order inside the noun phrase in all Tungusic languages is relatively strictly head-final. In several Tungusic languages, attributive adjectives (“adjectival nouns”) are simply juxtaposed with the modified noun. This type is also mentioned as being prototypical of adjective attribution marking devices in Tungusic languages (e.g., \citealt{sunik1968a}; \citealt[133]{kormusin2005}). However, several other types occur as well. The following adjective attribution marking devices are attested in Tungusic:
%%%
\begin{itemize}
\item juxtaposition\is{juxtaposition}
\item head\hyp{}driven agreement\is{head\hyp{}driven agreement}
\item attributive nominalization\is{attributive nominalization}
\item modifier\hyp{}headed possessor agreement.\is{modifier\hyp{}headed possessor agreement}
\end{itemize}

\il{North Tungusic languages|(}
\subsection{North Tungusic}
%%%
Languages belonging to the northern branch of Tungusic are Even (aka Lamut), Evenki (aka Oroqen in China), Negidal and Solon (aka Ewenke in China).

\is{head\hyp{}driven agreement|(}
The major North Tungusic languages, Even and Evenki, deviate from the Tungusic prototype and exhibit head\hyp{}driven agreement as their general type (\citealt[11]{malchukov1995}; \citealt[18]{bulatova-etal1999}). Attributive nominalization and modifier\hyp{}headed possessor agreement occur in these two languages as well, even though these devices are restricted to specially marked noun phrase types.

\il{Even|(}
\paragraph*{Head\hyp{}driven agreement in Even}
According to \citet[20]{malchukov1995}, the occurrence of head\hyp{}driven agreement marking of adjectives in Even is determined by discourse-pragmatic factors: attributes in the rhematic (focus) position always agree with their heads, whereas agreement is optional in non-focus positions \citep[31–32]{malchukov1995}.\footnote{According to \cite[31]{malchukov1995}, regular head\hyp{}driven agreement occurs as the default type of adjective attribution marking only in literary Even and hence in prescriptive grammars. This does not reflect, however, the actual language use.}
%%%
\begin{exe}
\ex
\label{even raising}
{\rm “Attribute raising agreement” in Even \citep[30–31]{malchukov1995}}
\begin{xlist}
\ex
\label{even juxt}
{\rm Juxtaposition}\is{juxtaposition}\\
{\rm (A N-\textsc{number}-\textsc{case})}\\
\gll	\textbf{Eŋi} beji-l-bu emu-re-m.\\
	strong man-\textsc{pl}-\textsc{acc} bring-\textsc{nonfut}-\textsc{1sg}\\

\ex
\label{even numagr}	
{\rm Incomplete head\hyp{}driven agreement}\\
{\rm (A-\textsc{number} N-\textsc{number}-\textsc{case})}\\
\gll	\textbf{Eŋi-l} beji-l-bu emu-re-m.\\
	strong-\textsc{pl} man-\textsc{pl}-\textsc{acc} bring-\textsc{nonfut}-\textsc{1sg}\\

\ex
\label{even agr}	
{\rm Complete head\hyp{}driven agreement}\\
{\rm (A-\textsc{number}-\textsc{case} N-\textsc{number}-\textsc{case})}\\
\gll	\textbf{Eŋi-l-bu} beji-l-bu emu-re-m.\\
	strong-\textsc{pl}-\textsc{acc} man-\textsc{pl}-\textsc{acc} bring-\textsc{nonfut}-\textsc{1sg}\\
\glt	‘I have brought back only strong men.’
\end{xlist}
\end{exe}
%%%
\citet[30–31]{malchukov1995} describes the attributive agreement patterns in Even in a hierarchical way: the adjective modifier can agree in all morphological features of the head-noun (\ref{even agr}) or just in number (\ref{even numagr}). Juxtaposition is also possible but restricted to adjectives in non-focus position (\ref{even juxt}).
\is{head\hyp{}driven agreement|)}

\is{attributive nominalization|(}
\paragraph*{Attributive nominalization in Even (I)}
The “attribute raising agreement” illustrated in the previous section (\S\ref{even raising}) can be extended with a fourth step, specifically with adjective attributes marked by the “restrictive” (i.e., \isi{contrastive focus}) marker \textit{=takan/=teken} (here glossed as a nominalizer).
%%%
\begin{exe}
\ex 
{\rm Even \citep[32]{malchukov1995}}
\begin{xlist}
\ex[]{ 
\gll	\textbf{Eŋi-l-bu=tken} beji-l-bu emu-re-m.\\
	strong-\textsc{pl}-\textsc{acc}=\textsc{nmlz} man-\textsc{pl}-\textsc{acc} bring-\textsc{nonfut}-\textsc{1sg}\\
	}
%%%
\ex[*]{	
\gll	\textbf{Eŋi=tken} beji-l-bu emu-re-m.\\
	strong=\textsc{nmlz} man-\textsc{pl}-\textsc{acc} bring-\textsc{nonfut}-\textsc{1sg}\\
\glt	‘I have brought back only strong men.’
}
\end{xlist}
\end{exe}
%%%
Attributes marked as “restrictive” obligatorily agree with the head noun \citep[32]{malchukov1995}. Noun phrases marked by means of \textit{=takan / =teken} thus resemble the attributive nominalization type, i.e., the attribute is marked as a syntactically complex constituent (i.e., as an embedded complement to the head noun) by means of nominalization.

\paragraph*{Attributive nominalization in Even (II)}
A second attributive nominalization strategy by means of the possessive suffix 3\textsuperscript{rd} person singular (in “determinative”\is{species marking!definite} function; here glossed as a nominalizer) is attested in an investigation of the non-possessive use of the possessive marker in different Turkic and Tungusic languages \citep{benzing1993b}.
%%%
\begin{exe}
\ex 
{\rm Even \citep[17–18 Footnote 58]{benzing1993b}}\\
\gll	hagdiŋata\textbf{-n} orolcemŋā\\
	oldest-\textsc{nmlz} reindeer\_herder\\
\glt	‘the \textsc{oldest} reindeer herder’
\end{exe}
%%%
According to \cite[17–18 Footnote 58]{benzing1993b}, the “determinative”\is{species marking!definite} suffix \textit{-n} ($\Leftarrow$ \textsc{poss:3sg}) can be used as a marker of contrastive focus in Even.
\il{Even|)}
\is{attributive nominalization|)}

\il{Evenki|(}
\is{modifier\hyp{}headed possessor agreement|(}
\paragraph*{Modifier\hyp{}headed possessor agreement in Evenki}
Evenki follows the general Tungusic rule of head-final constituent ordering inside the noun phrase. In constructions emphasizing the property denoted by the attributive adjectives, however, the unmarked adjective-noun order can be reversed. In these constructions, the adjective is obligatorily equipped with the possessive suffix 3\textsuperscript{rd} person (singular or plural).
%%%
\begin{exe}
\ex 
{\rm Evenki \citep[18]{bulatova-etal1999}}
\begin{xlist}
\ex	
\gll	\textbf{aja} bəjə\\
	good man\\
\glt	‘good man’

\newpage 
\ex
\label{evenki poss}
\gll	bi: bəjə \textbf{aja-βa:-n} sa:-m\\
	\textsc{1sg} man good-\textsc{acc}-\textsc{poss:3sg} know-\textsc{1sg}\\
\glt	‘I know the \textsc{good} man’
\end{xlist}
\end{exe}
%%%
According to \citet[18]{bulatova-etal1999}, the phrase final adjective ‘good’ marked with the possessive suffix is used as a true possessive noun in \REF{evenki poss} and they translate the example like this: ‘I know the man's goodness’. This construction, however, is similar to the modifier\hyp{}headed possessor agreement described for \ili{Oroch} (\ref{oroch modhead}) and Udege\il{Udege} (\citealt[485, passim]{nikolaeva-etal2001}).\footnote{Similar modifier\hyp{}headed constructions are found in Even where modifier\hyp{}headed possessor agreement is in fact attested, cf.~\textit{Asatkan \textbf{nood-do-n} haaram.} [girl beautiful-\textsc{acc}-\textsc{poss:3sg} I\_know] \citep[11]{malchukov1995}. But unlike similar modifier\hyp{}headed participles (in possessor agreement constructions) in Even \citep[31]{malchukov1995} and similar modifier\hyp{}headed adjectives in Oroch (\citealt{malchukov2000}, cf.~also example \ref{oroch modhead}) Malchukov translates this example as a true possessive construction with a nominal attribute: ‘I know the girl's beauty’ (but not: ‘I know the beautiful girl’).}
\il{Evenki|)}
\il{North Tungusic languages|)}
\is{modifier\hyp{}headed possessor agreement|)}

\il{Amur Tungusic languages|(}
\subsection{Amur Tungusic}
%%%
The Amur (aka South)\il{South Tungusic|see{Amur Tungusic languages}} branch of Tungusic consists of five languages. According to \citet[223]{salminen2007}, however, it is better to assume two separate subbranches, one of them comprising Udege\il{Udege} and \ili{Oroch} and the other comprising Nanay\il{Nanay} (aka Hejen\il{Hejen|see{Nanay}} in China), \ili{Ulcha} and \ili{Orok} (aka Uilta).\il{Uilta|see{Orok}}

\is{head\hyp{}driven agreement|(}
\il{Oroch-Udege languages|(}
\subsubsection{Oroch-Udege}
%%%
\paragraph*{Head\hyp{}driven agreement in Udege}
Head\hyp{}driven agreement in Udege is restricted to the feature \textsc{number}. Morphologically plural head nouns obligatorily trigger plural marking on the attributive adjective.
%%%
\begin{exe}
\ex 
{\rm Udege \citep[468]{nikolaeva-etal2001}}\\
\gll	uligdig'a\textbf{-ŋku} moxo-ziga bi-si-ti\\
	beautiful\textsc{-pl} cup\textsc{-pl} be\textsc{-pst-3pl}\\
\glt	‘There were beautiful cups.’
\end{exe}
\is{head\hyp{}driven agreement|)}

\il{Oroch|(}
\is{modifier\hyp{}headed possessor agreement|(}
\paragraph*{Modifier\hyp{}headed possessor agreement in Oroch}
Similar to Evenki\il{Evenki} from the northern branch of Tungusic, the Udege-Oroch languages from the Amur branch exhibit modifier\hyp{}headed possessor agreement. Oroch examples for this type of adjective attribution marking have already been discussed in \S\ref{ModheadAgr} but will be repeated here.
%%%
\begin{exe}
\ex 
\label{oroch modhead}
{\rm Oroch (\citealt[207]{avrorin-etal1967}; \citealt[3]{malchukov2000})}
\begin{xlist}
\ex
\gll 	nia	\textbf{aja-ni}\\
	man good-\textsc{poss:3sg}\\
\glt	‘a \textsc{good} man’
\ex
\gll nia-sa \textbf{aja-ti}\\	
	man-\textsc{pl} good-\textsc{poss:3pl}\\
\glt	‘\textsc{good} men’
\end{xlist}
\end{exe}
%%%
Whereas \isi{juxtaposition} is the default type of adjective attribution marking in Oroch, modifier\hyp{}headed possessor agreement occurs only in a special noun phrase type where the adjective is marked for contrastive focus. The special function marked by this construction is to focus on the property denoted by the adjective: ‘a man, a property of whom is “to be good”’ \citep[3]{malchukov2000}. This noun phrase type thus resembles the function of relative clause\is{adnominal modifier!relative clause} formation.\footnote{Note also that a similar construction is found in Even\il{Even} from the Northern Tungusic branch where it is only attested with participles: \textit{Beji-l-bu \textbf{hör-če-wut-ten} emu-re-m.} [man-\textsc{pl}-\textsc{acc} go-\textsc{pfct.ptcp}-\textsc{acc}-\textsc{poss:3pl} bring-\textsc{nonfut}-\textsc{1sg}] ‘I brought back the men who had left’ \citep[31]{malchukov1995}.}
\il{Oroch|)}
\il{Oroch-Udege languages|)}
\is{modifier\hyp{}headed possessor agreement|)}

\il{Nanay-Ulcha-Orok languages|(}
\subsubsection{Nanay-Ulcha-Orok}
%%%
\il{Orok|(}
According to the few grammatical sketches available, the Tungusic languages of the Nanay-Ulcha-Orok branch exhibit \isi{juxtaposition} as the default device for adjective attribution marking, except Orok.

\is{head\hyp{}driven agreement|(}
\paragraph*{Head\hyp{}driven agreement in Orok}
Attributive adjectives in Orok (aka Ulta)\il{Ulta|see{Orok}} show agreement in number but not in case (or other categories) with the modified noun.
%%%
\begin{exe}
\ex 
{\rm Orok \citep[55]{petrova1967}}
\begin{xlist}
\ex
\gll \textit{\textbf{dāi}} \textit{dalu(n)}\\
	big store\\
\glt ‘big store (i.e., warehouse, storehouse)’
\ex 
\gll	\textit{\textbf{dāi-l}} \textit{dalu-l}\\
	big-\textsc{pl} store-\textsc{pl}\\
\glt	‘big stores’
\ex 
\gll	\textit{\textbf{dāi-l}} \textit{dalu-l-tai}\\
	big-\textsc{pl} store-\textsc{pl}-\textsc{loc}\\
\glt	‘in big stores’
\end{xlist}
\end{exe}
\il{Orok|)}
\is{head\hyp{}driven agreement|)}

\il{Ulcha|(}
\is{attributive nominalization|(}
\paragraph*{Attributive nominalization in Ulcha}
According to \citet[36, 52–53]{sunik1985}, adjectives do not “normally” agree with the modified noun in Ulcha. The language is thus characterized by simple \isi{juxtaposition} of attributive adjectives.\footnote{\citet[36]{sunik1985} mentions, however, that a few adjectives sometimes show agreement with the modified noun in case and number (according to the simple or the possessive declension (sic!), i.e., are equipped with a possessive suffix) if they are “derived into nouns”. Unfortunately, he does not provide examples.}

Another adjective attribution marking device mentioned in Sunik's grammar is attributive nominalization by means of the suffix \textit{-d\.uma \textasciitilde-dumE} \citep{sunik1985}.
%%%
\begin{exe}
\ex 
{\rm Ulcha \citep[38]{sunik1985}}
\begin{xlist}
\ex \textit{n'ūči-dumE} {\rm ‘a/the little one (among other people)’}
\ex \textit{ulEn-dumE} {\rm ‘a/the good one (among other people)’}
\end{xlist}
\end{exe}
\il{Ulcha|)}
\il{Nanay-Ulcha-Orok languages|)}
\il{Amur Tungusic languages|)}
\is{attributive nominalization|)}

\il{Manchu languages|(}
\subsection{Manchu}
The two Manchu languages \ili{Manchu} proper and \ili{Sibe} exhibit \isi{juxtaposition} as the default adjective attribution marking device, similarly to the languages from the Nanay-Ulcha-Orok\il{Nanay-Ulcha-Orok languages} branch.
\il{Manchu languages|)}
\il{Tungusic languages|)}

\il{Yukaghir languages|(}
\section{Yukaghir}
\label{yukagir synchr}
%%%
Yukaghir (aka Yukagir)\il{Yukagir|see{Yukaghir languages}} is a small family consisting of the two individual languages Tundra Yukaghir\il{Tundra Yukaghir} and Kolyma Yukaghir\il{Kolyma Yukaghir} (aka Forest Yukaghir)\il{Forest Yukaghir|see{Kolyma Yukaghir}} (\citealt[223]{salminen2007}; \citealt[1–2]{maslova2003a}; \citealt[1]{maslova2003b}).

Noun phrases show strictly head-final constituent order in both Yukaghir languages. True adjective attribution scarcely exists because modifying “property words” in noun phrases are best coded as relative clauses.\is{adnominal modifier!relative clause}
%Martin: But (77) from Korean and (79a) from Dungan are also rel. clauses\\Micha: but there is no special attr-marking here

The following relevant attribution marking types are attested in Yukaghir languages:
%%%
\begin{itemize}
\item incorporation
\item anti\hyp{}construct state marking
	\subitem of “verbal adjectives”
	\subitem of “nominal adjectives”.
\end{itemize}

\il{Kolyma Yukaghir|(}
\is{juxtaposition|(}
\paragraph*{Juxtaposition in Kolyma Yukaghir}
There is no large class of lexical adjectives in Yukaghir. The only true adjectives in both Yukaghir languages belong to two semantic pairs: ‘small’ versus ’big’ and ‘old, ancient’ versus ‘new, fresh; (an)other’. The use of adjectives from the first pair is even restricted to a few lexicalized expressions \citep[70–71]{maslova2003b}. It is hard to categorize these adjectives according to their morpho-syntax. \citet[71]{maslova2003b} glosses the lexicalized expressions with the adjectives ‘small’ and ‘big’ as compounds, like in \textit{čom+parnā} [big+crow] ‘raven’. The adjective ‘new’, on the other hand can not only be used in such compounds but can even be marked additionally by the noun attribution suffix \textit{-d} or by the action nominal suffix \textit{-l} \citep[71]{maslova2003b}.
\is{juxtaposition|)}

\paragraph*{Anti\hyp{}construct state in Kolyma Yukaghir}
With the exception of the very small closed class described in the previous section, there are no adjectives in Kolyma Yukaghir (\citealt[79–112]{krejnovic1982}, \citealt[66–69, 145–147]{maslova2003b}). All other words denoting qualities constitute a subclass of verbs. Used as attributes, these \isi{stative verb}s take the 3\textsuperscript{rd} person singular intransitive suffix \mbox{\textit{-j(e)}}.\footnote{Note that this morpheme takes different phonological shapes as the result of allomorphic alternations.} \citet[66, passim]{maslova2003b} describes the inflected finite verbs, as in \REF{yuk attr} as “special attributive forms”. Syntactically, they have to be analyzed as juxtaposed relative clauses.\is{adnominal modifier!relative clause}
%%%
\begin{exe}
\ex 	
{\rm Kolyma Yukaghir \citep{maslova2003b}}
\begin{xlist}
\ex 
\label{yuk attr}
{\rm Attribution}
\begin{xlist}
\ex
\gll 	\textbf{kellugī-je} šoromo\\
	lazy-\textsc{attr:intr.3sg} person\\
\glt	‘lazy man (lit. ‘man who is lazy’)’ (146)
\ex
\gll	\textbf{kie-s'e} šoromo\\
	come-\textsc{attr:intr.3sg} person\\
\glt	‘man who comes’ (67)
\end{xlist}

\ex 
\label{yuk pred} 
{\rm Predication}\is{predicative marking}
\begin{xlist}
\ex
\gll 	id'ī pen \textbf{omo-s'}\\
	here it good-\textsc{pred:intr.3sg}\\
\glt	‘this is a nice place (lit. ‘here, it is good’)’ (68)
\end{xlist}
\end{xlist}
\end{exe}
%%%
Since verbs take different inflectional suffixes depending on their use as predicates or attributes (i.e., relative clauses, cf.~\ref{yuk attr}, \ref{yuk pred}) the suffix \textit{-j(e)} glossed as \textsc{attr:intr.3sg} can only be analyzed as an anti\hyp{}construct state marker, i.e., it constitutes a dependent\hyp{}marking attribution device which is not connected to noun phrase internal agreement. Even though the marker belongs to the verbal inflection paradigm it is a true licenser of the attributive relationship between a modifying verb phrase (relative clause)\is{adnominal modifier!relative clause} and a noun.

Anti\hyp{}construct state marking in Kolyma Yukaghir does not, however, belong to the domain of true adjective attribution marking but is a relative clause\is{adnominal modifier!relative clause} marking strategy.\footnote{In order to use a verb as modifier inside a noun phrase, the verb can also be nominalized, for example by means of an action nominal marker: \textit{kel-u-l} [come-0-\textsc{nmlz} ‘(a situation of) coming’ \citep[147]{maslova2003b}, \textit{kel-u-l šoromo} [come-0-\textsc{nmlz} person] ‘(a/the) man who came (i.e., (a/the) already arrived man)’ \citep[67]{maslova2003b}. This derivational nominalization of verbs to nominals is not considered to constitute an adjective attribution marking device either.}
\il{Kolyma Yukaghir|)}

\il{Tundra Yukaghir|(}
\paragraph*{Anti\hyp{}construct state in Tundra Yukaghir}
Tundra Yukaghir exhibits an anti\hyp{}construct state marking device of verbs using a relative clause\is{adnominal modifier!relative clause} marking strategy similar to Kolyma Yukaghir\il{Kolyma Yukaghir} \citep[49–50, passim]{maslova2003a}. In her short grammar, \cite{maslova2003a} mentions the occurrence of a second anti\hyp{}construct state marking device and gives the following example:
%%%
\begin{exe}
\ex 	
{\rm Tundra Yukaghir \citep[50]{maslova2003a}}\\
\gll 	lugu-je(\textbf{-d}) apanalā\\
	very\_old-\textsc{attr:intr.3sg}-\textsc{attr} woman\\
\glt	‘very old woman’
\end{exe}
%%%
The use of the marker \textit{-d} is not obligatory and is even restricted to head nouns with vowel-initial stems \citep[50]{maslova2003a}.

Interestingly, the second attribution marking device in Tundra Yukaghir is polyfunctional and regularly serves the licensing of single nouns (\ref{yuk nounattr}) as well as complex noun phrases\is{adnominal modifier!noun} (\ref{yuk npattr}) as attributes.
%%%
\begin{exe}
\ex 
{\rm Tundra Yukaghir \citep{maslova2003a}}
\begin{xlist}
\ex
\label{yuk nounattr}
\gll	iŋli\textbf{-d} igije\\
	breast-\textsc{attr} ropes\\
\glt	‘breast ropes’ (49)

\ex
\label{yuk npattr}
\gll	tude kerewe\textbf{-d} ugurt'e\\
	\textsc{3sg} cow-\textsc{attr} legs\\
\glt	‘the legs of his cow’\footnote{The regular use of the cognate attribution marker \textit{-d} (\textit{\textasciitilde-n}) with nouns and noun phrases as attributes is described for Kolyma Yukaghir as well. The use of the marker as a licenser of adjective attribution, however, seems to be restricted to one adjective, ‘new’ \citep[71]{maslova2003b}.} (44)
\end{xlist}
\end{exe}
\il{Tundra Yukaghir|)}
\il{Yukaghir languages|)}

\il{Yeniseian languages|(}
\il{Ket|(}
\section{Yeniseian}
\label{yeniseian synchr}
%%%
Three branches are posited for the Yeniseian family, but only the Ket language from the northern branch still exists today (\citealt{werner1997a}; \citealt[223]{salminen2007}).

\is{head\hyp{}driven agreement|(}
\is{juxtaposition|(}
The following adjective attribution marking devices are attested in Ket:
%%%
\begin{itemize}
\item juxtaposition
\item head\hyp{}driven agreement
\item attributive nominalization.\is{attributive nominalization}
\end{itemize}

\paragraph*{Juxtaposition and head\hyp{}driven agreement in Ket}
Attributive adjectives in Ket are normally juxtaposed to the left of the noun they modify \citep[38]{vajda2004}. Only a few simple adjective stems describing visible shapes or sizes may optionally take the plural suffix \textit{-ŋ}, as shown in \REF{ket agr}. The other morphological features assigned to the noun phrase, i.e., gender (or class) and case, are not sensitive to syntax in Ket.
%%%
\begin{exe}
\ex 
\label{ket agr}
{\rm Ket \citep[38]{vajda2004}} 
\begin{xlist}
\ex	
\gll	\textbf{qà} quˀŋ\\
	big tent:\textsc{pl}\\
\ex	
\gll	\textbf{qēŋ} quˀŋ\\
	big:\textsc{pl} tent:\textsc{pl}\\
\glt	‘big tents’
\end{xlist}
\end{exe}
%%%
\citet[38]{vajda2004} notes that the optional number agreement marking is “a stylistic device used to emphasize the visual impression created by the quality being described”. This emphasizing construction probably marks \isi{contrastive focus} of the adjective: ‘big tents’ versus ‘\textsc{big} tents’.
\is{juxtaposition|)}
\is{head\hyp{}driven agreement|)}

\is{attributive nominalization|(}
\paragraph*{Attributive nominalization in Ket}
\citet[15, 84–85]{vajda2004} also mentions the nominalizing suffix \textsc{-s} which marks lexical and derived adjectives (\ref{ket adjn nmlz}), noun phrases\is{adnominal modifier!noun} (\ref{ket np nmlz}), and adposition phrases\is{adnominal modifier!adposition phrase} (\ref{ket ap nmlz}) as adnominal modifiers in \isi{headless noun phrase}s.\footnote{Note that the examples (\ref{ket np nmlz} and \ref{ket ap nmlz}) seem to represent phonological compounds. This is evidenced by the phonological reduction in syllable-mediate vowels. The non-nominalized phrases, according to \citet{vajda2005} are \textit{úgda ɔ́lin} ‘a long nose’ and \textit{qō-t-hɯtɯ-ɣa} ‘under the ice [ice-\textsc{gen}-under]’. It is not clear from the description, however, if incorporation is relevant to morpho-syntax as well. But this phenomenon deserves further attention since adjective incorporation is scarcely attested in the world's languages but occurs in a few other non-related branches of the northern Eurasia.}
%%%
\begin{exe}
\ex 
{\rm Ket \citep{vajda2005}}
\begin{xlist}
\ex 
\label{ket adjn nmlz}
{\rm Nominalized adjective}
\begin{xlist}
\ex	
\gll	sîn\textbf{-s}\\
	old-\textsc{nmlz}\\
\glt	‘the old one’
\ex	
\gll	súl-tu\textbf{-s}\\
	blood-\textsc{deriv-nmlz}\\
\glt	‘the bloody one’
\end{xlist}
%%%
\ex	
\label{ket np nmlz}
{\rm Nominalized noun phrase}
\begin{xlist}
\ex
\gll	úgd-ɔ́lin\textbf{-s}\\
	long-nose-\textsc{nmlz}\\
\glt	‘the long-nosed one’
\end{xlist}
%%%
\ex 
\label{ket ap nmlz}
{\rm Nominalized adposition phrase}
\begin{xlist}
\ex
\gll	qó-t-{hɯtɯ-ɣa}\textbf{-s}\\
	ice-\textsc{gen}-under-\textsc{nmlz}\\
\glt	‘the one under the ice’
\end{xlist}
\end{xlist}
\end{exe}
%%%
Grammatical descriptions of Ket (\citealt{vajda2004}, cf.~also \citealt{krukova2007}) only give examples where these nominalized (headless) noun phrases\is{headless noun phrase} are used in apposition, as in the \isi{contrastive focus} construction (\ref{ket contrfoc}). 
%%%
\begin{exe}
\ex 
\label{ket contrfoc}
{\rm Ket \citep{vajda2005}} 
\begin{xlist}
\ex	
{\rm Adjective predication}\\
\gll	bū \textbf{sîn-du} / bū \textbf{sîn-dʌ}\\
	3\textsc{sg} old-\textsc{m.cop} { } 3\textsc{sg} old-\textsc{f.cop}\\
\glt	‘s/he is old’
\ex	
{\rm Contrastive focus construction}\\
\gll	bū \textbf{sîn-s}\\
	3\textsc{sg} old-\textsc{nmlz}\\
\glt	‘s/he is \textsc{old} (i.e., ‘an old one’)’
\end{xlist}
\end{exe}
%%%
The available data does not provide enough evidence for a detailed description and analyses of attributive nominalization by means of the suffix \textit{-s} as a regular attribution marking device in Ket. It it possible that these nominalizations cannot be used as true modifiers of nouns but are restricted to \isi{headless noun phrase}s and are used only in special \isi{contrastive focus} constructions.

There is even evidence against the analysis of nominalization as attributive marking in Ket. Vajda's examples of nominalized adverbials suggest that this contrastive focus marking is used predominantly in copular constructions (as predicates). Since the otherwise regular predicative agreement marking\is{predicative marking} never occurs on these nominalizations \citep[15]{vajda2004} it could also be argued that the nominalizer \textit{-s} constitutes a strategy for secondary predication marking rather than attribution marking.

Attributive nominalization in Ket definitely deserves more attention. The construction might constitute an example of the development of attributive nominalization independent of definiteness marking.\is{species marking!definite}
\il{Ket|)}
\il{Yeniseian languages|)}
\is{attributive nominalization|)}

\il{Turkic languages|(}
\section{Turkic}
%%%
Languages from the Turkic language family are spoken across all of northern Eurasia, including northeastern and southeastern \isi{Europe}, and beyond. The family is divided into two major branches: Bulgar and Common Turkic. Whereas Bulgar Turkic\il{Bulgar Turkic languages} is represented only by one language, the Common Turkic\il{Common Turkic languages} branch can be further divided into nine groups. Seven of these groups have members spoken in northern Eurasia: Oguz,\il{Oguz languages} Karluk,\il{Karluk languages} Kipchak,\il{Kipchak languages} Altay Turkic,\il{Altay Turkic languages} Yenisey Turkic\il{Yenisey Turkic languages} (Khakas),\il{Khakas|see{Yenisey Turkic languages}} Sayan Turkic,\il{Sayan Turkic languages} and Lena Turkic\il{Lena Turkic languages} \citep[221]{salminen2007}.

\is{juxtaposition|(}
All Turkic languages are characterized by strict head-finality in their noun phrase structure. The prototypical adjective attribution marking device in Turkic languages is juxtaposition. This type occurs as the unmarked construction in all Turkic languages. In some Turkic languages, however, an attributive nominalizer marks an attributive adjective in \isi{contrastive focus} constructions. This construction is systematically described (more or less) only for Chuvash\il{Chuvash} from the Bulgar Turkic branch.

The following types of adjective attribution marking are attested:
%%%
\begin{itemize}
\item juxtaposition
\item attributive nominalization.\is{attributive nominalization}
\end{itemize}

\il{Bulgar Turkic languages|(}
\subsection{Bulgar Turkic}
The Bulgar (aka Oghur)\il{Oghur Turkic|see{Bulgar Turkic languages}} subbranch of the Turkic language family is represented only by a single language, Chuvash.

\il{Chuvash|(}
\is{attributive nominalization|(}
\paragraph*{Juxtaposition and attributive nominalization in Chuvash}
\label{chuvash synchr}
Similar to all other Turkic languages, Chuvash exhibits juxtaposition as the default and general adjective attribution marking device (\ref{chuvash juxt}). Besides juxtaposition, an attributive nominalizer is used in \isi{contrastive focus} constructions (\ref{chuvash attr}).
%%%
\begin{exe}
\ex	
{\rm Chuvash \citep{clark1998a}}
\begin{xlist}
\ex 	
\label{chuvash juxt}
{\rm Juxtaposition}\\
\gll	\textbf{χura} χut\\
	black paper\\
\glt	‘black paper’
%%%
\ex	
\label{chuvash attr}
{\rm Attributive nominalization}\\
\gll	\textbf{χur-i} χut\\					 		
	black-\textsc{attr} paper\\
\glt	‘\textsc{black} paper (not of another color)’
\end{xlist}
\end{exe}
%%%
The attributive article \textit{-i} is similar to the possessive suffix 3\textsuperscript{rd} singular. As in other Turkic languages, this article is also obligatorily used in \isi{headless noun phrase}s marked as direct (accusative) objects in Chuvash.
%Contrastive focus, not definiteness (a black one / the black one
%Verschiedene Allomorphieregeln mit Poss3sg
\is{juxtaposition|)}
%%%
\begin{exe}
\ex 
\label{chuvash headless acc}	
{\rm Attributive nominalization in Chuvash \citep[7]{benzing1993b}}\\
\gll	\textbf{χur-i-ne} / \textbf{χĕrl-i-ne} ildem\\
 	black-\textsc{attr}-\textsc{acc} { } red-\textsc{attr}-\textsc{acc} I\_bought\\
\glt 	(Which pen did you buy?) ‘I bought a/the black / red one.’
\end{exe}
%%%
Besides \textit{-i}, a second nominalizer \textit{-sker} is attested in Chuvash. Both formatives are used with similar classes of adjectival and other attributes.
%%%
\begin{exe}
\ex 
{\rm Attributive nominalization in Chuvash \citep{krueger1961}}
\begin{xlist}
\ex 
{\rm Article \#1 \textit{-i} ($\Leftarrow$ \textsc{poss:3sg})}
\begin{xlist}
\ex	
{\rm Attributive adjective}\\
\gll	lajăχχ-i\\
	good-\textsc{attr}\\
\glt	‘which is good / (a/the) good one’

\ex	
{\rm Attributive participle}\\
\gll	vulan-i\\
	read.\textsc{prf}-\textsc{attr}\\
\glt	‘which is read’

\ex	
{\rm Attributive noun}\\
\gll	vărman-t-i\\
	forest-\textsc{loc}-\textsc{attr}\\
\glt	‘which is in the forest’
\end{xlist}

\ex 
{\rm Article \#2 \textit{-sker} (<~Mari\il{Mari languages} \textit{ÿsker})}
\begin{xlist}
\ex	
{\rm Attributive adjective}\\
\gll	lajăχ-sker\\
	good-\textsc{attr}\\
\glt	‘which is good / (a/the) good one’

\ex	
{\rm Attributive participle}\\
\gll	vulană-sker\\
	read.\textsc{prf}-\textsc{attr}\\
\glt	‘which is read’

\ex	
{\rm Attributive noun}\\
\gll	vărman-ta-sker\\
	forest-\textsc{loc}-\textsc{attr}\\
\glt	‘which is in the forest’
\end{xlist}
\end{xlist}
\end{exe}
\il{Chuvash|)}
\il{Bulgar Turkic languages|)}
\is{attributive nominalization|)}

\il{Common Turkic languages|(}
\subsection{Common Turkic}
%%%
\il{Oguz languages|(}
\subsubsection{Oguz}
%%%
\is{juxtaposition|(}
\paragraph*{Juxtaposition in \ili{Azerbaijani}}
Similar to all other Turkic languages, attributive adjectives are simply juxtaposed to the modified noun in Azerbaijani.
%%%
\begin{exe}
\settowidth\jamwidth{[\textbf{high} mountain-\textsc{pl}-\textsc{loc}]}
\ex
\label{azerb juxt}
{\rm Azerbaijani \citep[59–60]{siraliev-etal1971}}
\begin{xlist}
\ex	\textbf{uča} daɣ 		{\rm ‘high mountain’}	\jambox{{\rm [\textbf{high} mountain(\textsc{nom})]}}
\ex	\textbf{uča} daɣ-ɨn						\jambox{{\rm [\textbf{high} mountain-\textsc{gen}]}}
\ex	\textbf{uča} daɣ-da 						\jambox{{\rm [\textbf{high} mountain-\textsc{loc}]}}
\ex	\textbf{uča} daɣ-lar 						\jambox{{\rm [\textbf{high} mountain-\textsc{pl}]}}
\ex	\textbf{uča} daɣ-lar-da 					\jambox{{\rm [\textbf{high} mountain-\textsc{pl}-\textsc{loc}]}}
\ex \dots
\end{xlist}
\end{exe}
\is{juxtaposition|)}

\is{attributive nominalization|(}
\paragraph*{Attributive nominalization in \ili{Turkish}}
\label{turkish synchr}
Similar to other Turkic languages, the attributive nominalization device is used obligatorily in \isi{headless noun phrase}s marked as direct (accusative) objects in Turkish.
%%%
\begin{exe}
\ex 
\label{turkish headless acc}	
{\rm Attributive nominalization in Turkish \citep[7]{benzing1993b}}\\ 
\gll	\textbf{kara-sını} / \textbf{kızıl-ını} aldım\\
 	black-\textsc{attr:acc} { } red-\textsc{attr:acc} I\_bought\\
\glt 	(Which pen did you buy?) ‘I bought a/the black / red one.’
\end{exe}
\il{Oguz languages|)}
 
\il{Karluk languages|(}
\subsubsection{Karluk}
%%%
The default and general adjective attribution marking device in the languages of the Karluk subbranch of Common Turkic is \isi{juxtaposition} and is similar to example (\ref{azerb juxt}) from Azerbaijani.\il{Azerbaijani} Besides juxtaposition, attributive nominalization is also attested.

\paragraph*{Attributive nominalization in \ili{Uigur}} 
The possessive suffix 3\textsuperscript{rd} person singular occurs as an attributive nominalizer in \isi{contrastive focus} constructions in Uigur. This construction is thus similar to example (\ref{chuvash attr}) from \ili{Chuvash} from the Bulgar branch of Turkic.
%%%
\begin{exe}
\ex 
{\rm Uigur \citep[17–18, Footnote 58]{benzing1993b}}\\
\gll	\textbf{uluy-ï} qatun\\
	biggest-\textsc{attr} woman\\
\glt	‘the \textsc{first} wife’
\end{exe}

\paragraph*{Attributive nominalization in \ili{Uzbek}}
Similar to other Turkic languages, the article is also used obligatorily in \isi{headless noun phrase}s marked as direct (accusative) objects in Uzbek. 
%%%
\begin{exe}
\ex 
\label{uzbek headless acc}	
{\rm Attributive nominalization in Uzbek \citep[371]{boeschoten1998}}\\
\gll	{(mėŋȧ qaysisi yarašadi,)} \textbf{qizilim-i}, \textbf{åqim-i}?\\
 	{ } red-\textsc{attr:acc} white-\textsc{attr:acc}\\
\glt 	‘(Which one suits me,) the red one, or the white one?’
\end{exe}
\il{Karluk languages|)}
\is{attributive nominalization|)}

\il{Kipchak languages|(}
\il{Altay Turkic languages|(}
\il{Yenisey Turkic languages|(}
\il{Sayan Turkic languages|(}
\il{Lena Turkic languages|(}
\subsubsection{Kipchak, Altay, Yenisey, Sayan, Lena}
%%%
The default and general adjective attribution marking device in the languages of the Kipchak, Altay, Yenisey (aka Khakas), Sayan and Lena subbranchs of Common Turkic is \textbf{\isi{juxtaposition}} and is similar to example (\ref{azerb juxt}) from Azerbaijani.\il{Azerbaijani}
\il{Kipchak languages|)}
\il{Altay Turkic languages|)}
\il{Yenisey Turkic languages|)}
\il{Sayan Turkic languages|)}
\il{Lena Turkic languages|)}
\il{Common Turkic languages|)}
\il{Turkic languages|)}

\il{Nakh-Daghestanian languages|(}
\section{Nakh-Daghestanian}
%%%
Nakh-Daghestanian is a language family of the Caucasus\is{Caucasus}. It is named after its two main branches: Nakh\il{Nakh languages} and Daghestanian.\il{Daghestanian languages} Whereas Nakh comprises only a few single languages, the Daghestanian branch can be further divided into several subbranches \citep[220, 233]{salminen2007}.

The predominant order of noun phrase constituent in Nakh-Daghestanian languages is adjective-noun. Regarding the morpho-syntactic licensing of adjective attribution, the Nakh-Daghestanian family is characterized by a relatively high diversity of noun phrase types.

The following adjective attribution marking devices are attested:
%%%
\begin{itemize}
\item juxtaposition\is{juxtaposition}
\item head\hyp{}driven agreement marking\is{head\hyp{}driven agreement}
\item anti\hyp{}construct state agreement marking
\item anti\hyp{}construct state marking
\item attributive nominalization.\is{attributive nominalization}
\end{itemize}

\il{Daghestanian languages|(}
\subsection{Daghestanian}
%%%
\il{Avar-Andi-Tsezic languages|(}
\subsubsection{Avar-Andi-Tsezic}
%%%
The Avar-Andi-Tsezic group of Daghestanian is named after three groups of closely related languages: Andi\il{Andi languages} (comprising the languages Akhvakh,\il{Akhvakh} Andi,\il{Andi} Bagvalal,\il{Bagvalal} Botlikh,\il{Botlikh} Chamalal,\il{Chamalal} Godoberi,\il{Godoberi} Karata\il{Karata} and Tindi),\il{Tindi} Tsezic\il{Tsezic languages} (comprising the languages Tsez\il{Tsez} (aka Dido),\il{Dido|see{Tsez}} \ili{Hinuq}, \ili{Khwarshi}, Inkhokvari,\il{Inkhokvari} Bezhta\il{Bezhta} (aka Kapucha)\il{Kapucha|see{Bezhta}} and \ili{Hunzib}. The single language \ili{Avar} forms the third group of Avar-Andi-Tsezic \citep[220, 233]{salminen2007}.

\is{head\hyp{}driven agreement|(}
The prototype of adjective attribution marking in the Avar-Andi-Tsezic languages seems to be head\hyp{}driven agreement, which occurs in all languages of this group.

\il{Godoberi|(}
\paragraph*{Head\hyp{}driven agreement in Godoberi}
The unmarked constituent order in Godoberi is adjective-noun.\footnote{The reversed order marks \isi{contrastive focus} on the adjective: \textit{hac'a χ°aji} [white dog] ‘white dog’, \textit{χ°aji hac'a} [dog white] ‘that very dog (of several others) which is white’ \citep[149]{kazenin1996a}.} Adjectives agree with the head noun in the features \textsc{gender} (if a position for the class-marker is available) and \textsc{number}.
%%%
\begin{exe}
\ex
\settowidth\jamwidth{[\textsc{n.pl}]}
{\rm Godoberi \citep[25]{tatevosov1996a}}
\begin{xlist}
\ex 
{\rm Adjectives taking a gender class prefix}
\begin{xlist}
\ex \textbf{w-oχar} ima 			{\rm ‘old father’}			\jambox{{\rm [\textsc{m}]}}
\ex \textbf{j-aχar} ila				{\rm ‘old mother’}		\jambox{{\rm [\textsc{f}]}}
\ex \textbf{b-aχar} hamaχi			{\rm ‘old donkey’}		\jambox{{\rm [\textsc{n}]}}
\ex \textbf{r-aχar} hamaχi-be 		{\rm ‘old donkeys’}		\jambox{{\rm [\textsc{n.pl}]}}
\end{xlist}

\ex 
{\rm Adjectives taking a gender class suffix}
\begin{xlist}
\ex \textbf{q'arúma-w} ima			{\rm ‘greedy father’}		\jambox{{\rm [\textsc{m}]}}
\ex \textbf{q'aruma-j} ila			{\rm ‘greedy mother’}		\jambox{{\rm [\textsc{f}]}}
\ex \textbf{q'arúma-b} hamaχi 		{\rm ‘greedy donkey’}		\jambox{{\rm [\textsc{n}]}}
\ex \textbf{q'arúma-r} hamaχi-be	{\rm ‘greedy donkeys’}	\jambox{{\rm [\textsc{n.pl}]}}
\end{xlist}
\end{xlist}
\end{exe}
\il{Godoberi|)}
\is{head\hyp{}driven agreement|)}

\il{Tsez|(}
\is{attributive nominalization|(}
\paragraph*{Attributive nominalization in Tsez}
In Tsez, two lexical classes of adjectives have to be distinguished. The members of the first class take gender agreement prefixes. The (few) members of the second class are simply juxtaposed to the modified noun \citep[126]{alekseev-etal2004}.

There is an additional attributive marker: the attributive nominalizing suffix \textit{-ni} which marks attributive adjectives in \isi{headless noun phrase}s and also “restrictive” forms of the adjective.
%%%
\begin{exe}
\ex 
{\rm Tsez \citep{alekseev-etal2004}}
\begin{xlist}
\ex	
{\rm Nominalized headless adjective}
\begin{xlist}
\ex
\gll	igu\textbf{-n}-a:\\
	good-\textsc{attr}-\textsc{erg}\\
\glt	‘a good one’
\ex
\gll	igu\textbf{-ni}-r\\
	good-\textsc{attr}-\textsc{dat}\\
\glt	‘to a good one’
\end{xlist}

\ex	
{\rm “Restrictive” attributive adjective}
\begin{xlist}
\ex	
\label{Tsez restr}
\gll	(eyda) \textbf{eġe-ni} uži dey esiy yoɬ\\
	this little-\textsc{attr} boy \textsc{1:gen} brother:\textsc{nom} be:\textsc{prs}\\
\glt	‘(this) little boy (and not one of the others) is my brother’
\end{xlist}
\end{xlist}
\end{exe}
The content of the “restrictive” (aka “definite”)\is{species marking!definite} form remains somewhat uncertain. The translation of (\ref{Tsez restr}) in the description of \citet[128]{alekseev-etal2004} clearly resembles \isi{contrastive focus} marking (‘the \textsc{little} boy’).
\il{Tsez|)}
\il{Avar-Andi-Tsezic languages|)}
\is{attributive nominalization|)}

\il{Lak|(}
\subsubsection{Lak}
%%%
The Lak subbranch of Daghestanian is formed by one single language: Lak proper.

\is{head\hyp{}driven agreement|(}
\paragraph*{Head\hyp{}driven agreement in Lak}
Constituent order in Lak is adjective-noun. The language exhibits two adjective attribution marking devices. The unmarked and default attribution marking device is head\hyp{}driven agreement which characterizes adjectives derived by means of the adjectivizer\is{adjective derivation} \mbox{\textit{-ssa}}, as in \REF{lak hdragr}. These derived adjectives only agree in gender class. Other morpho-syntactic marking is not applied.
%%%
\begin{exe}
\ex 
\label{lak hdragr}
{\rm Lak \citep[48]{zirkov1955}} 
\begin{xlist}
\ex 
\gll	\textbf{uč-ssa} adimina\\
	fat.\textsc{I}-\textsc{adjz} person(\textsc{I})\\
\glt	‘fat man’

\ex
\gll	\textbf{b-uč-ssa} nic\\
	\textsc{III}-fat-\textsc{adjz} bull\textsc{(III)}\\
\glt	‘fat bull’

\ex
\gll	\textbf{b-uč-ssa} nic-ru\\
	\textsc{III}-fat-\textsc{adjz} bull\textsc{(III)}-\textsc{pl}\\
\glt	‘fat bulls’
\end{xlist}
\end{exe}
%%%
Note that the suffix \textit{-ssa} is a derivational formative rather than a marker of attribution since it occurs on adjectives in attributive and predicative position alike. Predicative adjectives\is{predicative marking} even show similar gender agreement inflection \citep[45–51]{zirkov1955}.
\is{head\hyp{}driven agreement|)}

\paragraph*{Anti\hyp{}construct state agreement in Lak}
While \isi{head\hyp{}driven agreement} marking, as in \REF{lak hdragr}, constitutes the basic and unmarked adjective attribution marking device in Lak, anti\hyp{}construct state agreement marking is restricted to \isi{contrastive focus} constructions.
%%%
\begin{exe}
\ex 
{\rm Lak \citep[45]{zirkov1955}}
\begin{xlist}
\ex
\gll	\textbf{uč-ma} adimina\\
	fat.\textsc{I}-\textsc{attr:I} person\textsc{(I)}\\
\glt	‘\textsc{fat} man’
\ex
\gll	\textbf{b-uč-mur} nic\\
	\textsc{III}-fat-\textsc{attr:III} bull\textsc{(III)}\\
\glt	‘\textsc{fat} bull’
\ex
\gll	\textbf{buč-mi} nic-ru\\
	\textsc{III}-fat-\textsc{attr:pl} bull\textsc{(III)}-\textsc{pl}\\
\glt	‘\textsc{fat} bulls’
\end{xlist}
\end{exe}
%%%
Note that the occurrence of the anti\hyp{}construct state agreement marking suffixes \textit{-ma, -mur, -mi} is restricted to attributive adjectives. Unlike adjectives with the derivational formative \textit{-ssa} with \isi{head\hyp{}driven agreement} marking in number only, adjectives in \isi{contrastive focus} (occurring in the anti\hyp{}construct state agreement noun phrase type) show agreement in number as well \citep[45–51]{zirkov1955}.
\il{Lak|)}

\il{Dargwa|(}
\subsubsection{Dargwa}
%%%
The Dargwa subbranch of Daghestanian has traditionally been described as consisting of one single language (i.e., Dargwa proper) with several sub-varieties \citep[233]{salminen2007}. According to \textcite{korjakov2006a}, Dargwa varieties exhibit fairly diverse grammatical structures and can therefore be described as separate languages.

\is{juxtaposition|(}
\paragraph*{Anti\hyp{}construct state agreement and juxtaposition in Dargwa}
In Dargwa, two adjective attribution marking devices occur. Whereas anti\hyp{}construct state (number) agreement marking (\ref{dargwa anti}) is the default type, juxtaposition (\ref{dargwa juxt}) is restricted to “poetic language” \citep[318]{isaev2004}.
\begin{exe}
\ex 
{\rm Dargwa \citep[318]{isaev2004}}
\begin{xlist}
\ex 
\label{dargwa anti}
{\rm Anti\hyp{}construct state agreement} 
\begin{xlist}
\ex
\gll	\textbf{aɢ-si} ɢali\\
	high-\textsc{attr:sg} house(\textsc{sg})\\
\glt	‘lofty house’
\ex
\gll	\textbf{aɢ-ti} ɢulri\\
	high-\textsc{attr:pl} house:\textsc{pl}\\
\glt	‘lofty houses’
\end{xlist}

\ex
\label{dargwa juxt}
{\rm Juxtaposition}
\begin{xlist}
\ex
\gll	\textbf{aɢ} dubura\\
	high mountain\\
\glt	‘high mountain’
\end{xlist}
\end{xlist}
\end{exe}
\il{Dargwa|)}
\is{juxtaposition|)}

\il{Lezgic languages|(}
\subsubsection{Lezgic}
\label{lezgian synchr}
%%%
The Lezgic subbranch of Daghestanian comprises the languages \ili{Agul}, \ili{Archi}, \ili{Badukh}, \ili{Kryz} (aka Kryts),\il{Kryts|see{Kryz}} \ili{Lezgian}, \ili{Rutul}, \ili{Tabasaran}, \ili{Tsakhur} and \ili{Udi}.

\is{juxtaposition|(}
Adjective-noun is the basic constituent order in the noun phrase of all Lezgic languages. Regarding their adjective attribution marking, the Lezgic languages exhibit the highest degree of diversity. All types found in Nakh-Daghestanian are attested: juxtaposition, \isi{head\hyp{}driven agreement} marking, anti\hyp{}construct state agreement marking, anti\hyp{}construct state marking and attributive nominalization.

\il{Udi|(}
\paragraph*{Juxtaposition in Udi}
The default adjective attribution marking device in Udi is juxtaposition, like in the following (incomplete) paradigm.
%%%
\begin{exe}
\settowidth\jamwidth{[\textsc{gen}]}
\ex 
{\rm Udi \citep[465]{schulze-furhoff1994}}
\begin{xlist}
\ex \textbf{kala} ĝara-Ø	{\rm ‘the old son’}	\jambox{{\rm [\textsc{abs}]}}
\ex \textbf{kala} ĝara-en					\jambox{{\rm [\textsc{erg}]}}
\ex \textbf{kala} ĝara-i					\jambox{{\rm [\textsc{gen}]}}
\ex \dots
\end{xlist}
\end{exe}

\is{head\hyp{}driven agreement|(}
\il{Tabasaran|(}
\paragraph*{Juxtaposition and head\hyp{}driven agreement in Tabasaran}
The default adjective attribution marking device in Tabasaran is juxtaposition, as in Udi.\il{Udi} Only a minor lexical subclass of two adjectives in this language deviate in this respect and show gender and number agreement.
\il{Udi|)}
%%%
\begin{exe}
\ex 
{\rm Tabasaran \citep[50–51]{kurbanov1986}}
\begin{xlist}
\ex 
\gll 	\textbf{uččvu-r} adaš\\
	beautiful-\textsc{I} father\textsc{(I)}\\
\glt	‘beautiful father’

\ex 
\gll	\textbf{uččvu-b} gjajvan\\
	beautiful-\textsc{II} horse\textsc{(II)}\\
\glt	‘beautiful horse’

\ex
\gll	\textbf{uččvu-dar} gjunšjir\\
	beautiful-\textsc{pl} horse:\textsc{pl}\\
\glt	‘beautiful horses’
\end{xlist}
\end{exe}
\il{Tabasaran|)}
\is{juxtaposition|)}

\il{Archi|(}
\paragraph*{Head\hyp{}driven agreement in Archi}
Attributive adjectives in Archi show agreement in gender and number with the modified noun; see the complete agreement paradigm for the adjective ‘good’.
%%%
\begin{exe}
\settowidth\jamwidth{[\textsc{IV sg}]}
\ex 
{\rm Archi \citep{kibrik1994a}}
\begin{xlist}
\ex	hibàt̄u	{\rm ‘good’}	\jambox{{\rm [\textsc{I sg}]}}
\ex	hibàt̄u-r				\jambox{{\rm [\textsc{II sg}]}}
\ex	hibàt̄u-b				\jambox{{\rm [\textsc{III sg}]}}
\ex	hibàt̄u-t				\jambox{{\rm [\textsc{IV sg}]}}
\ex	hibàt̄-ib				\jambox{{\rm [\textsc{pl}]}}
\end{xlist}
\end{exe}
\il{Archi|)}
\is{head\hyp{}driven agreement|)}

\il{Tsakhur|(}
\paragraph*{Anti\hyp{}construct state agreement in Tsakhur}
Adjectives in Tsakhur can be divided into three subclasses according to their choice of attribution marking devices. The first, minor lexical class of adjectives in Tsakhur is characterized by missing inflection. Adjectives belonging to this class are simply juxtaposed to the modified noun \citep[383]{talibov2004}. Members of the two other adjective classes exhibit anti\hyp{}construct state agreement marking.
%%%
\begin{exe}
\ex 
{\rm Tsakhur \citep[382]{talibov2004}}
\begin{xlist}
\ex	
\label{tsakhur gen}
\begin{xlist}
\ex {\rm Gender class I–III}\\
\gll	\textbf{bat'raj-na} jis / dix̌ / balk\textsuperscript{h}an\\
	 beautiful-\textsc{attr:I–III} girl(\textsc{I}) { } son(\textsc{II}) { } horse(\textsc{III})\\
\glt	 ‘beautiful girl / son / horse’
%%%
\ex {\rm Gender class IV}\\
\gll	\textbf{bat'raj-n}	č'alag\\
	beautiful-\textsc{attr:IV} forest(\textsc{IV})\\
\glt	‘beautiful forest’
\end{xlist}
%%%
\ex 
\label{tsakhur attrgen}
\begin{xlist}
\ex {\rm Gender class I}\\
\gll	\textbf{x̌\underline{a}rna} jis\\
	big:\textsc{attr:I–III} mother(\textsc{I})\\
\glt	‘old mother (viz.~grandmother)’
%%%
\ex {\rm Gender class IV}\\
\gll	\textbf{x̌adɨn} balag\\
	big:\textsc{attr:IV} sack(\textsc{IV})\\
\glt	‘big sack’
\end{xlist}
\end{xlist}
\end{exe}
%%%
Whereas the anti\hyp{}construct agreement marker of adjectives from the first group (\ref{tsakhur gen}) is formally identical with the genitive case suffixes of nouns, adjectives from the second group (\ref{tsakhur attrgen}) are equipped with a morphologically complex formative including the genitive suffix and a phonological stem alternation \citep[382]{talibov2004}.
%bei \cite{schulze1997} noch sehr interessante Sachen zum Genitiv-Attributive
\il{Tsakhur|)}

\il{Udi|(}
\is{attributive nominalization|(}
\is{headless noun phrase|(}
\paragraph*{Nominalization in headless noun phrases in Udi}
The default adjective attribution marking device in Udi is \isi{head\hyp{}driven agreement}. In headless noun phrases, however, attributive adjectives are obligatorily nominalized by means of the stem augment \textit{-o-} \textsc{abs} / \textit{-t'-} \textsc{obl}.
%%%
\begin{exe}
\settowidth\jamwidth{[\textsc{nmlz:obl}-\textsc{pl}-\textsc{erg}]}
\ex {\rm Udi \citep[466]{schulze-furhoff1994}}
\begin{xlist}
\ex	kala-o		{\rm ‘the big/old one’}	\jambox{{\rm [\textsc{nmlz.abs}]}}
\ex	kala-o-r						\jambox{{\rm [\textsc{nmlz.abs}-\textsc{pl}]}}
\ex	kala-t'-in						\jambox{{\rm [\textsc{nmlz:obl}-\textsc{erg}]}}
\ex	kala-t'-ĝ-on					\jambox{{\rm [\textsc{nmlz:obl}-\textsc{pl}-\textsc{erg}]}}
\ex	kala-t'-ay						\jambox{{\rm [\textsc{nmlz:obl}-\textsc{gen}]}}
\ex \dots
\end{xlist}
\end{exe}
\il{Udi|)}

\il{Lezgian|(}
\paragraph*{Nominalization in headless noun phrases in Lezgian}
Attributive adjectives in headless noun phrases are nominalized in Lezgian as well. The nominalizing suffix exhibits different forms in the absolute singular case (\textit{-di}), in the oblique cases (\textit{-da}) and in plural (\textit{-bur}).
%%%
\begin{exe}
\settowidth\jamwidth{[\textsc{attr:pl}-\textsc{erg}]}
\ex {\rm Headless adjectives in Lezgian \citep[110]{haspelmath1993}}
\begin{xlist}
\ex	q̃acu-di		{\rm ‘green one’}	\jambox{{\rm [\textsc{attr:sg}]}}
\ex	q̃acu-da						\jambox{{\rm [\textsc{attr:erg.sg}]}}
\ex	q̃acu-da-n						\jambox{{\rm [\textsc{attr}-\textsc{gen}]}}
\ex	q̃acu-bur						\jambox{{\rm [\textsc{attr:pl}]}}
\ex	q̃acu-bur-u					\jambox{{\rm [\textsc{attr:pl}-\textsc{erg}]}}
\ex \dots
\end{xlist}
\end{exe}
%%%

\largerpage
The same attribution marker is also used for the nominalization of noun phrases.
%%%
\begin{exe}
\settowidth\jamwidth{[mother-\textsc{gen}-\textsc{attr}]}
\ex {\rm Nominalized noun phrases in Lezgian \citep[110]{haspelmath1993}}
\begin{xlist}
\ex {\rm Pronoun} 
\begin{xlist}
\ex	zi			{\rm ‘my’}		\jambox{{\rm [\textsc{poss:1sg}]}}
\ex	zi\textbf{-di}	{\rm ‘mine’}	\jambox{{\rm [\textsc{poss:1sg}-\textsc{attr}]}}
\end{xlist}
%%%
\ex {\rm Lexical noun}
\begin{xlist}
\ex	dide.di-n			{\rm ‘mother's’}	\jambox{{\rm [mother-\textsc{gen}]}}
\ex 	dide.di-n\textbf{-di}	{\rm ‘mother's’}	\jambox{{\rm [mother-\textsc{gen}-\textsc{attr}]}}
\end{xlist}
\end{xlist}
\end{exe}
%%%
Even though adjectives without a lexical head in \ili{Udi} and Lezgian are nominalized there is no evidence that these nominalizations serve as attribution marking devices.
\il{Lezgian|)}
\is{attributive nominalization|)}
\is{headless noun phrase|)}

\il{Rutul|(}
\paragraph*{Anti\hyp{}construct state in Rutul}
In Rutul, attributive and predicative\is{predicative marking} adjectives are differentiated by means of two different derivations. Whereas attributive adjectives take an anti\hyp{}construct suffix \textit{-d \textasciitilde-dɨ},\footnote{The allomorph \textit{-dɨ} occurs after consonants \citep[224]{alekseev1994a}.} predicative adjectives take a suffix \textit{-ɨ \textasciitilde-ɨ}\footnote{The allomorph \textit{-ɨ} occurs after dorsal consonants \citep[224]{alekseev1994a}.} or are not marked at all \citep[224]{alekseev1994a}.

Attributive adjectives do not inflect other than by means of anti\hyp{}construct state marking.
%%%
\begin{exe}
\ex {\rm Rutul \citep[237]{alekseev1994a}}
\begin{xlist}
\ex
\gll	\textbf{äkkà-d} dahàr\\
	big-\textsc{attr} stone\\
\glt	‘big stone’
\ex
\gll	\textbf{äkkà-d} dahàr-bɨr\\
	big-\textsc{attr} stone-\textsc{pl}\\
\glt	‘big stones’
\end{xlist}
\end{exe}
%%%
Note that the anti\hyp{}construct state marker \textit{-d \textasciitilde-dɨ} is identical to the genitive case of nouns and thus constitutes a polyfunctional marker \citep{alekseev1994a}.
\il{Rutul|)}
\il{Lezgic languages|)}
\il{Daghestanian languages|)}

\il{Nakh languages|(}
\subsection{Nakh}
%%%
The Nakh branch of Nakh-Daghestanian comprises only three languages: \ili{Bats}, \ili{Ingush} and \ili{Chechen}. The latter two form a common subbranch \citep[220, 233]{salminen2007}.

\is{head\hyp{}driven agreement|(}
The noun phrase structure in all three languages is basically similar. Attributive adjectives precede the modified noun and show head\hyp{}driven agreement. Adjectives in \isi{headless noun phrase}s are additionally marked with an attributive nominalizer.

\il{Chechen-Ingush languages|(}
\subsection{Chechen-Ingush}
\label{ingush synchr}
%%%
\il{Ingush|(}
\paragraph*{Head\hyp{}driven agreement in Ingush}
Attributive adjectives in Ingush agree in case with the modified noun. The adjective agreement paradigm, however, exhibits only a single case distinction of nominative versus oblique.
%%%
\begin{exe}
\settowidth\jamwidth{[\textsc{nom}]}
\ex {\rm Case agreement paradigm in Ingush \citep[99]{nichols1994b}}
\label{ingush agr}
\begin{xlist}
\ex \textbf{joqqa} jurt			{\rm ‘big village’}	\jambox{{\rm [\textsc{nom}]}}
\ex \textbf{joqqa-ča} jurt-a						\jambox{{\rm [\textsc{gen}]}}
\ex \textbf{joqqa-ča} jurt-aa					\jambox{{\rm [\textsc{dat}]}}
\ex \textbf{joqqa-ča} jurt-uo					\jambox{{\rm [\textsc{erg}]}}
\ex \textbf{joqqa-ča} jurt-aca					\jambox{{\rm [\textsc{ins}]}}
\ex \dots
\end{xlist}
\end{exe}
%%%
Some adjectives also show agreement in gender; but only very few adjectives additionally agree in number with the modified noun \citep[99]{nichols1994b}.
\il{Ingush|)}
\is{head\hyp{}driven agreement|)}

\is{attributive nominalization|(}
\il{Chechen|(}
\is{headless noun phrase|(}
\paragraph*{Nominalization in headless noun phrases in Chechen}
Beside \isi{head\hyp{}driven agreement}, Chechen (similar to the other Nakh languages) exhibits attributive nominalization as the regular adjective attribution marking device in headless noun phrases. The formative is a thematic stem extension merged with the case inflection.
%%%
\begin{exe}
\ex {\rm Chechen \citep[29]{nichols1994a}}
\begin{xlist}
\ex
\gll	leqa kert\\
	high fence\\
\glt	‘high fence’
\ex	
\gll	leqa\textbf{-nig}\\
	high-\textsc{attr:nom.sg}\\
\glt	‘the high one’
\end{xlist}
\end{exe}
%%%
Even though adjectives without a lexical head in Chechen are nominalized, there is no evidence that these nominalizations serve as attribution marking devices.
\il{Chechen|)}
\il{Chechen-Ingush languages|)}
\is{headless noun phrase|)}

\il{Bats|(}
\subsection{Bats}
%%%
The noun phrase structure in Bats (aka Tsova-Tush\il{Tsova-Tush|see{Bats}} or Batsbi)\il{Batsbi|see{Bats}} is similar to the structure found in closely related \ili{Chechen} and \ili{Ingush}. Attributive adjectives show \isi{head\hyp{}driven agreement}. Adjectives in \isi{headless noun phrase}s are additionally marked by means of nominalization \citep[172–172]{holisky-etal1994}.
\il{Bats|)}
\il{Nakh languages|)}
\il{Nakh-Daghestanian languages|)}
\is{attributive nominalization|)}

\il{Abkhaz-Adyghe languages|(}
\section{Abkhaz-Adyghe}
%%%
The Abkhaz-Adyghe (aka Northwest Caucasian)\il{Northwest Caucasian|see{Abkhaz-Adyghe languages}} family consists of the two branches Abkhaz and Circassian,\il{Circassian languages} each of which comprises two languages. A third branch, Ubykh,\il{Ubykh languages} is now extinct \citep[220, 233]{salminen2007}. All languages are spoken in the northwestern \isi{Caucasus} region.

\is{head\hyp{}driven agreement|(}
Whereas the adjective-noun constituent order is similar in all Abkhaz-Adyghe languages, the adjective attribution marking devices
%%%
\begin{itemize}
\item head\hyp{}driven agreement (Abkhaz)
\item incorporation (Circassian)
\end{itemize}
occurring in the two branches of this family diverge considerably.

\il{Abkhaz languages}
\subsection{Abkhaz}
%%%
The Abkhaz branch of Abkhaz-Adyghe comprises the two very closely related varieties \ili{Abkhaz} proper and \ili{Abaza}. The constituent order inside the noun phrase of both languages is normally noun-adjective. Only adjectives denoting nationality deviate from this rule and precede the modified noun \citep[222]{comrie1981}.

\il{Abkhaz|(}
\paragraph*{Head\hyp{}driven agreement in Abkhaz}
%WO = noun-adjective %aber nationality = A N %A N auch bei einigen anderen möglich
Attributive adjectives in Abkhaz show number agreement.\footnote{Noun phrases with an attributive adjective following a non-inflected noun in Abkhaz have alternatively been analyzed as polysynthetic constructions (hence adjective incorporation), e.g., by \citet[123]{rijkhoff2002} and \citet{gil2005}.} Note, however, that a plural noun modified by an adjective may remain unmarked \citep[46]{hewitt1989a}. Even though the plural marker may attach only once at the right phrase edge, it is best analyzed as an agreement marker and not a \isi{clitic}. This is evidenced by the fact that the adjective may take the non-human pluralizer even if it modifies a human noun.\footnote{Note that in the closely related language Abaza, plural marking occurs twice but the non-human pluralizer constitutes the obligatory plural agreement marker on adjectives modifying nouns of any gender class \citep[100]{lomtatidze-etal1989}.}
%%%
\begin{exe}
\ex {\rm Abkhaz \citep{hewitt1989a}}
\begin{xlist}
\ex
\gll	a-là(-k°à) \textbf{bzə̀ya-k°a}\\
	\textsc{def}-dog-\textsc{pl:nonhum} good-\textsc{pl:nonhum}\\
\glt	‘the good dogs’
%%%
\ex	
\gll	à-ʒġab(-ċ°a) \textbf{bzə̀ya-k°a} / \textbf{bzə̀ya-ċ°a}\\
	\textsc{def}-girl-\textsc{pl:hum} good-\textsc{pl:nonhum} {} good-\textsc{pl:hum}\\
\glt	‘the good girls’
\end{xlist}
\end{exe}
\il{Abkhaz|)}
\is{head\hyp{}driven agreement|)}

\il{Circassian languages|(}
\subsection{Circassian}
%%%
The Circassian (aka Adyghe\il{Adyghe (branch)|see{Circassian languages}}) branch of Abkhaz-Adyghe comprises the two languages \ili{Adyghe} and \ili{Karbardian}. Both languages exhibit similar noun phrase structures. The constituent order inside the noun phrase is normally noun-adjective. Noun phrases with modifying adjectives in Adyghe and Karbardian are often described as single compound words \citep[222]{comrie1981}.

\il{Karbardian|(}
\paragraph*{Adjective incorporation in Karbardian}
Attributive adjectives in Karbardian (aka Eastern Circassian\il{Eastern Circassian|see{Karbadian}}) occur in a polysynthetic structure to the right of the modified noun. Number and case inflection of the noun phrase is suffixed to the adjective.
%%%
\begin{exe}
\ex {\rm Karbardian \citep[295]{colarusso1989}}
\begin{xlist}
\ex	
\gll	pṡaaṡa\textbf{-daax̂a}-r\\
	girl-beautiful-\textsc{abs}\\
\glt	‘the beautiful girl’
%%%
\ex
\gll	pṡaaṡa\textbf{-daax̂a}-ha-r\\
	girl-beautiful-\textsc{pl}-\textsc{abs}\\
\glt	‘the beautiful girls’
%%%
\ex
\gll	pṡaaṡa\textbf{-daax̂a-c'ək'°}-ər\\
	girl-beautiful-little-\textsc{abs}\\
\glt	‘the small beautiful girl’
\end{xlist}
\end{exe}
\il{Karbardian|)}
\il{Circassian languages|)}
\il{Abkhaz-Adyghe languages|)}

\il{Kartvelian languages|(}
\section{Kartvelian}
\label{kartvelian synchr}
%%%
Kartvelian is a language family comprising the four languages \ili{Georgian}, \ili{Svan}, \ili{Laz} and \ili{Mingrelian} (aka Megrelian\il{Megrelian|see{Mingrelian}} or Iverian).\il{Iverian|see{Mingrelian}} The latter two languages constitute the Zan\il{Zan languages} subbranch inside the family \citep[220]{salminen2007}. Kartvelian languages are all spoken in the southern \ili{Caucasus}, mainly in Georgia but also in adjacent countries.

In the modern Kartvelian languages, the unmarked constituent order of adjectival modifiers and head is noun-final, although the opposite order is also possible \citep[56]{harris1991a}.

\is{head\hyp{}driven agreement|(}
\is{juxtaposition|(}
Three adjective attribution marking types are attested:
%%%
\begin{itemize}
\item juxtaposition
\item head\hyp{}driven agreement
\item appositional head\hyp{}driven agreement.
\end{itemize}
%%%
The inherited Common Kartvelian\il{Common Kartvelian} agreement marking, however, is more or less preserved only in the marked (but inherited) head-initial noun phrase type. In the head-final noun phrase type, on the other hand, the modern Kartvelian languages display a strong tendency to lose head\hyp{}driven agreement. Preposed attributive adjectives in \ili{Mingrelian} and \ili{Laz} are juxtaposed to the head noun as a rule. In Modern \ili{Georgian} and \ili{Svan}, the agreement paradigm of preposed attributive adjectives shows a high degree of syncretism.
\is{juxtaposition|)}

\il{Georgian|(}
\subsection{Georgian}
\label{georgian synchr}
%%%
\paragraph*{Head\hyp{}driven agreement in Georgian}
The only agreement feature in Modern Georgian is \textsc{case}. Note, however, that the adjective agreement paradigm exhibits only three differentiated forms.\footnote{In the marked head-initial constituent order of noun and adjective, which is used in archaic style or for emphasis, case agreement is complete \citep[59]{tuite1998}.}
%%%
\begin{exe}
\settowidth\jamwidth{[\textsc{nom}]}
\ex {\rm Georgian \citep[236]{aronson1991}}
\label{georgian old}
\begin{xlist}
\ex \textbf{ʒvel-i} c'ign-i		{\rm ‘old book’}	\jambox{{\rm [\textsc{nom}]}}
\ex \textbf{ʒvel-ma} c'ign-ma				\jambox{{\rm [\textsc{erg}]}}
\ex \textbf{ʒvel-Ø} c'ign-s					\jambox{{\rm [\textsc{dat}]}}
\ex \textbf{ʒvel-i} c'ign-is					\jambox{{\rm [\textsc{gen}]}}
\ex \textbf{ʒvel-i} c'ign-it					\jambox{{\rm [\textsc{ins}]}}
\ex \textbf{ʒvel-Ø} c'ign-ad					\jambox{{\rm [\textsc{adv}]}}
\ex \dots
\end{xlist}
\end{exe}
\is{head\hyp{}driven agreement|)}

\is{juxtaposition|(}
\paragraph*{Juxtaposition in Georgian}
Whereas the so-called consonantal-stem adjectives like ‘old’ in \REF{georgian old} show \isi{head\hyp{}driven agreement}, there is another lexical class of adjectives (characterized by a stem-final vowel, hence “vocalic-stem adjectives”), the members of which are simply juxtaposed to the modified noun.
%%%
\begin{exe}
\settowidth\jamwidth{[\textsc{nom}]}
\ex {\rm Georgian \citep[236]{aronson1991}}
\begin{xlist}
\ex \textbf{parto} gza	{\rm ‘wide road’}		\jambox{[{\textsc{nom}]}}
\ex \textbf{parto} gza-m				\jambox{\rm[{\textsc{erg}]}}
\ex \textbf{parto} gza-s				\jambox{\rm[{\textsc{dat}]}}
\ex \textbf{parto} gz-is				\jambox{\rm[{\textsc{gen}]}}
\ex \textbf{parto} gz-it				\jambox{\rm[{\textsc{ins}]}}
\ex \textbf{parto} gz-ad				\jambox{\rm[{\textsc{adv}]}}
\ex \dots
\end{xlist}
\end{exe}
\is{juxtaposition|)}

\is{appositional head\hyp{}driven agreement|(}
\paragraph*{Appositional head\hyp{}driven agreement in Georgian}
Appositional modification seems to occur as a secondary type of adjective attribution marking in Georgian. Attributive adjectives are normally preposed and show only limited agreement (\ref{georgian unmarked}). In postposition (marking emphasis), however, the adjective inflects for the full set of cases and numbers (\ref{georgian marked}). This construction thus resembles an independent (headless) noun phrase\is{headless noun phrase} in apposition to the semantic head \citep[652, 677]{testelec1998}. The construction probably marks \isi{contrastive focus} of the adjective.
%%%
\begin{exe}
\ex {\rm Georgian \citep[652]{testelec1998}}
\begin{xlist}
\ex 
\label{georgian unmarked}
\gll	am or \textbf{lamaz} kal-s\\
	that:\textsc{obl} two nice:\textsc{obl} woman-\textsc{dat}\\
\glt	‘to those two nice women’
%%%
\ex 
\label{georgian marked}
\gll	kal-eb-s \textbf{lamaz-eb-s}\\
	woman-\textsc{pl}-\textsc{dat} nice-\textsc{pl}-\textsc{dat}\\
\glt	‘to the \textsc{nice} women’
\end{xlist}
\end{exe}
\is{appositional head\hyp{}driven agreement|)}
\il{Georgian|)}

\is{head\hyp{}driven agreement|(}
\il{Svan|(}
\subsection{Svan}
%%%
\paragraph*{Head\hyp{}driven agreement in Svan}
Attributive adjectives in Svan show limited agreement in case. The paradigm of the agreement marker exhibits only two members: one for nominative and one for the oblique cases.
%%%
\begin{exe}
\ex {\rm Svan \citep[18]{tuite1997}}
\begin{xlist}
\ex	
\gll 	\textbf{luwzer-e}	ma:r-e\\
	diligent-\textsc{nom} man-\textsc{nom}\\
\glt	‘a diligent man’
%%%
\ex	
\gll	\textbf{luwzer-a}	ma:re:m-i 		našdabw\\
	diligent-\textsc{obl} man-\textsc{gen} work\\
\glt	‘(the work) of a diligent man’
\end{xlist}
\end{exe}
%%%
\citet[499]{schmidt1991}, however, describes the tendency in Svan to abolish agreement completely and use an uninflected variant of the attributive adjective in the oblique cases instead.
\il{Svan|)}
\is{head\hyp{}driven agreement|)}

\is{juxtaposition|(}
\il{Zan languages|(}
\subsection{Zan}
%%%
Zan is a subbranch of Kartvelian formed by the two languages \ili{Mingrelian} and \ili{Laz}. The default type of adjective attribution marking in both languages is juxtaposition which occurs obligatorily in the unmarked head-final noun phrase. In the marked head-initial noun phrase, however, attributive adjectives normally agree in number and case with the head noun.

\is{head\hyp{}driven agreement|(}
\il{Mingrelian|(}
\paragraph*{Juxtaposition and head\hyp{}driven agreement in Mingrelian}
The two adjective attribution marking devices occurring in Zan languages are illustrated with Mingrelian examples.
%%%
\begin{exe}
\ex {\rm Mingrelian \citep[361–364]{harris1991b}}
\begin{xlist}
\ex {\rm Juxtaposition}\\
\label{mingrelian juxt}
\gll	\textbf{skvam} cira-en-k\\
	beautiful girl-\textsc{pl}-\textsc{nar}\\
\glt	‘beautiful girl’%Bsp ist konstruiert
%%%
\ex {\rm Head\hyp{}driven agreement}\\
\label{mingrelian agr}
\gll	cira-en-k \textbf{skvam-en-k}\\
	girl-\textsc{pl}-\textsc{nar} beautiful-\textsc{pl}-\textsc{nar}\\
\glt	‘\textsc{beautiful} girl’\footnote{Note that the case marking formative does not obligatorily occur on both constituents in the marked head-initial noun phrase in Mingrelian \citep[363–364]{harris1991b}.}%clitic
\end{xlist}
\end{exe}
\il{Mingrelian|)}
\il{Zan languages|)}
\il{Kartvelian languages|)}
\is{juxtaposition|)}
\is{head\hyp{}driven agreement|)}

\il{Semitic languages|(}
\section{Semitic}
%%%
Semitic languages are only marginally represented in northern Eurasia. The few languages considered here belong either to the Arabic\il{Arabic languages} subbranch of Central Semitic\il{Central Semitic languages} or to Northwest Semitic.\il{Northwest Semitic languages}

\is{head\hyp{}driven agreement|(}
Only one single type of adjective attribution marking is attested in these two branches:
%%%
\begin{itemize}
\item head\hyp{}driven agreement.
\end{itemize}

\il{Northwest Semitic languages|(}
\il{Neo-Aramaic|(}
\subsection{Northwest Semitic}
%%%
Neo-Aramaic (aka Modern Aramaic) is the only language of the northwestern branch of Semitic considered in the present survey. It is spoken in the Middle-East in north-western Iran, Iraq and south-eastern Turkey, but also in adjacent areas of the Caucasus\is{Caucasus} in Azerbaijan, and therefore falls into the geographic area of investigation.

\paragraph*{Head\hyp{}driven agreement in Neo-Aramaic}
Constituent order inside the noun phrase of Neo-Aramaic is noun-adjective. Attributive adjectives agree with the modified noun in gender and number.
%%%
\il{Neo-Aramaic!Kurdistan}
\begin{exe}
\ex {\rm Neo-Aramaic (Kurdistan) \citep{krotkoff1982}}
\begin{xlist}
\ex	ya:la \textbf{zu:ra}	{\rm ‘small boy’}
\ex	bra:ta \textbf{zurta}	{\rm ‘small girl’}
\ex	bnu:ne \textbf{zu:re}	{\rm ‘small kids’}
\end{xlist}
\end{exe}
\il{Neo-Aramaic|)}
\il{Northwest Semitic languages|)}

\il{Central Semitic languages|(}
\subsection{Central Semitic}		
%%%
\il{Arabic languages|(}
\subsubsection{Arabic}
\ili{Cypriot Arabic} (aka Kormakiti)\il{Kormakiti|see{Cypriot Arabic}} and \ili{Maltese} are two Arabic languages of the Central Semitic branch spoken on the Mediterranean islands Cyprus and Malta, and thus belong to \isi{Europe} geographically.

\il{Maltese|(}
\paragraph*{Head\hyp{}driven agreement in Maltese}
The basic and unmarked constituent order in Maltese is noun-adjective. A few adjectives, however, can precede the noun in an emphatic construction \citep[71]{borg-etal1996}.

Adjectives show distinct forms for gender and number in accordance with the morphological features of the modified noun.
%%%
\begin{exe}
\ex {\rm Maltese \citep[328]{aquilina1959}}
\begin{xlist}
\ex
\gll	ra:jel \textbf{sabi:ħ}\\
	man beautiful:\textsc{m:sg}\\
\glt	‘beautiful man’
%%%
\ex
\gll	mara \textbf{sabi:ħa}\\
	woman beautiful:\textsc{f:sg}\\
\glt	‘beautiful woman’
%%%
\ex
\gll	nies \textbf{sbieħ}\\
	people beautiful:\textsc{pl}\\
\glt	‘beautiful people’
\end{xlist}
\end{exe}
%%%
Optionally, the attributive adjective can additionally be marked for definiteness.
%%%
\begin{exe}
\ex {\rm Maltese \citep[330]{aquilina1959}}\\
\gll	il-ktieb \textbf{(il-)qadi:m}\\
	\textsc{def}-book	(\textsc{def-})old\\
\glt	‘the old book’
\end{exe}
%%%
Even though the construction with a repeated definite marker resembles \isi{attributive nominalization}, it is best analyzed as agreement in the \textsc{definite}\is{species marking!definite} value of the feature \textsc{species} \citep[179]{himmelmann1997}. Himmelmann compares the construction in Maltese to Standard \ili{Arabic}, where similar definite (and indefinite) agreement occurs.\is{species marking!indefinite}
%CHECK CECA for interesting similarities between attr. and pred. adjectives
\il{Maltese|)}
\il{Arabic languages|)}
\il{Central Semitic languages|)}
\il{Semitic languages|)}
\is{head\hyp{}driven agreement|)}
				
\il{Uralic languages|(}
\section{Uralic}
\label{uralic synchr}
%%%
The Uralic language family comprises the branches (roughly from West to East) Hungarian,\il{Hungarian} Saamic,\il{Saamic languages} Finnic,\il{Finnic languages} Permic,\il{Permic languages} Mari,\il{Mari languages} Mordvin,\il{Mordvin languages} Khanty,\il{Khanty languages} Mansi,\il{Mansi languages} and Samoyedic\il{Samoyedic languages} \citep[216–218]{salminen2007}. Except for most languages from the Samoyedic subbranch of the family, Uralic languages are all spoken in \isi{Europe}. Uralic is thus one of the major families on the European linguistic map.

The constituent order inside the noun phrase is strictly adjective-initial in all Uralic languages. Similar to Mongolic,\il{Mongolic languages} Turkic\il{Turkic languages} and many other languages of \isi{North Asia}, the prototypical adjective attribution marking device in Uralic languages is \isi{juxtaposition}. This type occurs as the unmarked construction in all Uralic languages with the exception of the two western branches Saamic\il{Saamic languages} and Finnic\il{Finnic languages}, which have abandoned juxtaposition and developed new types.

Secondary adjective attribution marking devices are also attested in languages of the Permic\il{Permic languages} and Mari\il{Mari languages} (and probably also other) branches of Uralic, even though juxtaposition is used in these languages as the default strategy for adjective attribution marking.

The following five adjective attribution marking devices occur in Uralic:
%%%
\begin{itemize}
\item juxtaposition\is{juxtaposition}
\item head\hyp{}driven agreement\is{head\hyp{}driven agreement}
\item anti\hyp{}construct state marking
\item appositional head\hyp{}driven agreement\is{appositional head\hyp{}driven agreement}
\item attributive nominalization.\is{attributive nominalization}
\end{itemize}

\il{Samoyedic languages|(}
\subsection{Samoyedic}
%%%
\il{Enets languages|(}
\il{Forest Enets|(}
\subsubsection{Enets}
The two languages Forest Enets and \ili{Tundra Enets} constitute the Enets branch of Uralic.

\is{juxtaposition|(}
\paragraph*{Juxtaposition in Forest Enets}
In both Enets languages, attributive adjectives are juxtaposed to the modified noun by default.
%%%
\begin{exe}
\ex {\rm Forest Enets  \citep[71]{siegl2013a}}
\label{enets juxt}
\begin{xlist}
\ex 
\gll	\textbf{aga} to\\
	big lake(\textsc{nom:sg})\\
\glt	‘a/the big lake’
%%%
\ex 
\gll	\textbf{aga} to-ʔ\\
	big lake\textsc{-nom:pl}\\
\glt	‘big lakes’
%%%
\ex 
\gll	\textbf{aga} to-xiʔ\\
	big lake\textsc{-nom:du}\\
\glt	‘two big lakes’
%%%%
%\ex 
%\gll	\textbf{aga} to-xun\\
%	big lake\textsc{-loc:sg}\\
%\glt	‘in a/the big lake’
%%%%
%\ex 
%\gll	\textbf{aga} to-xin\\
%	big lake\textsc{-loc:pl}\\
%\glt	‘in a/the big lakes’
\end{xlist}
\end{exe}
\il{Forest Enets|)}
\il{Enets languages|)}
\is{juxtaposition|)}

\il{Nenets languages|(}
\il{Selkup languages|(}
\il{Nganasan|(}
\subsubsection{Nenets, Selkup, Nganasan}
%%%
The two languages \ili{Forest Nenets} and \ili{Tundra Nenets} constitute the Nenets subbranch of Samoyedic. The Selkup branch consists of the three very closely related languages \ili{Northern Selkup}, \ili{Central Selkup} and \ili{Southern Selkup}. The Nganasan branch consists only of one language: Nganasan proper.

Attributive adjectives in Nganasan and the Nenets and Selkup languages are juxtaposed to the modified noun by default, similar to examples (\ref{enets juxt}) from Forest Enets\il{Forest Enets} and (\ref{hung juxt}) from \ili{Hungarian}.
\il{Nenets languages|)}
\il{Selkup languages|)}
\il{Nganasan|)}
\il{Samoyedic languages|)}

\il{Hungarian|(}
\subsection{Hungarian}
%%%
The Hungarian branch of Uralic consists only of one language, i.e., Hungarian proper.\footnote{The outlying dialect Csángó Hungarian\il{Hungarian!Csángó} spoken in Romania is not considered as a distinct language here.}

\is{juxtaposition|(}
\paragraph*{Juxtaposition in Hungarian}
In Hungarian, attributive adjectives are juxtaposed to the modified noun by default.
%%%
\begin{exe}
\ex {\rm Hungarian \citep[41]{hall1938}}
\label{hung juxt}
\begin{xlist}
\ex 
\gll	a \textbf{fekete} szem\\
	\textsc{def} black eye\\
\glt	‘the black eye’
%%%
\ex	
\gll	a \textbf{fekete} szem-ek\\
	\textsc{def} black eye-\textsc{pl}\\
\glt	‘the black eyes’
%%%
\ex
\gll	a \textbf{fekete} szem-ek-nek\\
	\textsc{def} black eye-\textsc{pl}-\textsc{dat}\\
\glt	‘to the black eyes’
%%%
\ex
\gll	a \textbf{fekete} szem-eid\\
	\textsc{def} black eye-\textsc{pl:poss:2sg}\\
\glt	‘your black eyes’
\end{xlist}
\end{exe}
\il{Hungarian|)}
\is{juxtaposition|)}

\il{Khanty languages|(}
\il{Mansi languages|(}
\il{Mari languages|(}
\il{Mordvin languages|(}
\subsection{Khanty, Mansi, Mari, Mordvin}
%%%
The two languages \ili{Northern Khanty} and \ili{Eastern Khanty} constitute the Khanty branch of Uralic. A third language, Southern Khanty,\il{Southern Khanty} is extinct \citep[231]{salminen2007}. The Mansi branch of Uralic consists of the two very closely related languages \ili{Northern Mansi} and \ili{Eastern Mansi}. Two other Mansi languages, \ili{Western Mansi} and \ili{Southern Mansi}, are extinct \citep[231]{salminen2007}. The Mari branch of Uralic is formed by Western Mari\il{Western Mari} (aka Hill Mari)\il{Hill Mari|see{Western Mari}} and Eastern Mari\il{Eastern Mari} (aka Meadow Mari)\il{Meadow Mari|see{Eastern Mari}} \citep[231]{salminen2007}. The Mordvin branch of Uralic is formed by the two closely related languages Erzya\il{Erzya Mordvin} and Moksha \il{Moksha Mordvin} \citep[231]{salminen2007}.

Attributive adjectives in all Khanty, Mansi, Mari and Mordvin languages are juxtaposed to the modified noun by default, similar to examples (\ref{enets juxt}) from \ili{Forest Enets} and (\ref{hung juxt}) from \ili{Hungarian}.
\il{Khanty languages|)}
\il{Mansi languages|)}
\il{Mari languages|)}
\il{Mordvin languages|)}

\il{Permic languages|(}
\subsection{Permic}
%%%
All three Permic languages, Komi-Permyak,\il{Komi-Permyak} Komi-Zyrian\il{Komi-Zyrian} and Udmurt\il{Udmurt} exhibit two distinct types of adjective attribution marking. The default type is juxtaposition, which is the inherited Proto\hyp{}Uralic\il{Proto\hyp{}Uralic} type \citep[80–81]{decsy1990}. However, an \isi{attributive nominalization} device is used in \isi{contrastive focus} constructions as a second type.

\il{Komi-Zyrian|(}
\is{juxtaposition|(}
\paragraph*{Juxtaposition in Komi-Zyrian}
The unmarked sequence of adjective and noun, i.e., juxtaposition, is illustrated by an example from Komi-Zyrian.
%%%
\begin{exe}
\ex {\rm Komi-Zyrian \citep[287]{lytkin1966a}}
\begin{xlist} 
\ex
\gll 		\textbf{bur} 	mort\\
		good	person\\
\glt		‘good person’
%%%
\ex
\gll		\textbf{bur}	mort-jas\\
		good	person-\textsc{pl}\\
\glt		‘good people’
\end{xlist}
\end{exe}
\il{Komi-Zyrian|)}
\is{juxtaposition|)}

\il{Udmurt|(}
\is{attributive nominalization|(}
\paragraph*{Attributive nominalization + appositional head\hyp{}driven agreement in Udmurt}
\label{udmurt synchr}
In Udmurt, an attributive nominalizer homophonous with the 3\textsuperscript{rd} person possessive inflection marker is regularly used as an adjective attribution marking device in \isi{contrastive focus} constructions. Historically, both formatives are similar (cf.~\S\ref{udmurt diachr} in Part~IV Diachrony).
%%%

\newpage 
\begin{exe}
\ex {\rm Udmurt \citep{winkler2001}}
\begin{xlist}
\ex {\rm Juxtaposition (default)}
\begin{xlist}
\ex
\gll	\textbf{badǯ́ym} gurt\\
	big house\\
\glt	‘big house’
%%%
\ex	
\gll	\textbf{badǯ́ym} gurt-jos-y\\
	big house-\textsc{pl}-\textsc{ill}\\
\glt	‘to big house/s’
\end{xlist}
%%%
\ex {\rm Attributive nominalization (contrastive focus)}
\begin{xlist}
\ex
\gll	\textbf{badǯ́ym-ėz} gurt\\
	big-\textsc{attr} house\\
\glt	‘\textsc{big} house’
%%%
\ex	
\gll	\textbf{badǯ́ym-jos-a-z} gurt-jos-y\\
	big-\textsc{pl}-\textsc{ill}-\textsc{attr} house-\textsc{pl}-\textsc{ill}\\
\glt	‘to \textsc{big} house/s’
\end{xlist}
\end{xlist}
\end{exe}
%%%
An adjective equipped with the nominalizer is also marked with (agreeing) case and number suffixes indicating that the nominalized adjective occurs in an attributive appositional construction. Note that the nominalizer also serves as the licenser of adjectival (and other) modification\is{modification marking} in \isi{headless noun phrase}s.
%%%
\begin{exe}
\ex {\rm Nominalization in Udmurt \citep{winkler2001}}
\begin{xlist}
\ex {\rm Adjective}\\
\gll	badǯ́ym\textbf{-ėz}\\
	big-\textsc{attr}\\
\glt	 ‘the big one’
%%%
\ex {\rm Demonstrative}\\
\gll	taiz\textbf{-ėz}\\
 	\textsc{dem:dist}-\textsc{attr}\\
\glt	‘that one over there’
%%%
\ex {\rm Possessor noun phrase}\\
\gll	Ivan-len\textbf{-ėz}\\
	Ivan-\textsc{gen}-\textsc{attr}\\
\glt	‘that one of Ivan's’
\end{xlist}
%%%

\newpage 
\ex {\rm Contrastive focused attribute}
\begin{xlist}
\ex {\rm Demonstrative}\\ 
\label{attr demnmlz}
\gll	taiz\textbf{-ėz} gurt\\
 	\textsc{dem:dist}-\textsc{attr} house\\
\glt	‘\textsc{that} (particular) house over there’
%%%
\ex {\rm Possessor noun phrase}\\ 
\label{attr npnmlz}
\gll	Ivan-len\textbf{-ėz} gurt\\
	Ivan-\textsc{gen}-\textsc{attr} house\\
\glt	‘\textsc{Ivan's} house (and not someone else's)’
\end{xlist}
\end{exe}
%%%
Examples (\ref{attr demnmlz}–\ref{attr npnmlz}) show that attributive nominalization in Udmurt is a true attribution marking device which is polyfunctional and not restricted to \isi{headless noun phrase}s.

\is{species marking!definite|(}
\is{contrastive focus|(}
Note that the attributive article is normally labeled “determinative suffix” (or in similar terms) in the Udmurt (and Uralic) grammatical tradition. This label probably originates from the formative's function as a quasi-definite marker. But “determinative” inflection is obligatory only in the case of differential object marking with the marked versus the unmarked accusative. Note also that the definite-marked accusative suffix, again, is historically identical with the 3\textsuperscript{rd} person possessive suffix.
%%%
\begin{exe}
\ex {\rm Differential object marking in Udmurt \citep[22]{winkler2001}}
\begin{xlist}
\ex
\gll	mon kniga lɨdǯ́-i\\
	\textsc{1sg} book(\textsc{acc}) read-\textsc{1sg.pst}\\
\glt	‘I have read a book.’
%%%
\ex	
\gll	mon (ta) kniga\textbf{-jez} lɨdǯ́-i\\
	\textsc{1sg} this book-\textsc{acc} read-\textsc{1sg.pst}\\
\glt	‘I have read the (i.e., ‘this certain’) book.’
\end{xlist}
\end{exe}
%%%
Note also that in these and similar examples, the concept of definiteness does not always coincide with the use of the differential “in\slash{}definite accusative” marking. According to \citet[21]{winkler2001}, “the marked accusative is used if the object itself is focused, whereas the unmarked is employed if the action itself bears the logical accent.” Accordingly, even such occurrences of the “determinative suffix” thus resemble focus marking rather than definiteness marking.

Even though contrastive focus inflection of nouns (or noun phrases) would be the result of purely morphological (morpho-semantic) assignment, contrastive focus inflection of adnominal adjectives can only be analyzed as a morpho-syntactic feature assigned noun phrase internally. This is evidenced by the agreement pattern: whereas adjectives in non-contrasted (unmarked) constructions are simply juxtaposed to the head noun, contrastive focused adjectives normally show head\hyp{}driven number agreement.\footnote{The different order of morphemes in certain members of contrastive focus inflection paradigms (i.e., number-, case-, and (former) possessive suffix) as compared to the historically similar “regular” possessive inflection \citep[32]{winkler2001} is not of concern here. This phenomenon does, however, provide evidence for the analysis of the contrastive focus marker of adjectives and the possessive marker of nouns as two different formatives from a synchronic point of view.} Agreement marking on the adjective is clearly assigned by syntax, the head noun being the agreement trigger and the attributive adjective (in contrastive focus) being the agreement target.

Attributive marking in contrastive focus constructions in Udmurt (and the other Permic languages) is similar in theory to prototypical anti\hyp{}construct state agreement marking in languages like Russian,\il{Russian} with regard to both synchrony and diachrony. The construction is still analyzed as attributive nominalization because the agreement marking on the nominalized attribute is the indirect result of the attributive appositional construction and the nominalizing and agreement formatives are not fused synchronically.
\is{attributive nominalization|)}

\paragraph*{Appositional head\hyp{}driven agreement in Udmurt}
Note, however, that in Udmurt, number agreement also sometimes occurs without the contrastive focus marker.
%%%
\begin{exe}
\ex {\rm Head\hyp{}driven plural agreement in Udmurt}
\begin{xlist}
\ex 
\gll	badǯ́ym-eš́ gurt-jos\\
	big-\textsc{pl} house-\textsc{pl}\\
\glt	‘\textsc{big} houses’ \citep[40]{winkler2001}
\ex 
\gll	paśkɨt-eś uram-jos\\
	wide-\textsc{pl} street-\textsc{pl}\\
\glt	‘\textsc{wide} streets’ \citep[63]{csucs1990}
\end{xlist}
\end{exe}
%%%
According to \citet[63]{csucs1990}, \isi{head\hyp{}driven agreement} marking in constructions without the “determinative suffix” is the result of analogy. The fact that their use is still restricted to contrastive focus constructions, and is therefore an appositional attribution marking device, is crucial for the analysis as \isi{appositional head\hyp{}driven agreement} (as opposed to true \isi{head\hyp{}driven agreement}).
\is{contrastive focus|)}
\is{species marking!definite|)}
\il{Udmurt|)}
\il{Permic languages|)}

\il{Finnic languages|(}
\subsection{Finnic}
%%%
The Finnic (aka Fennic\il{Fennic languages|see{Finnic languages}} or Baltic Finnic)\il{Baltic Finnic languages|see{Finnic languages}} branch of Uralic comprises the following languages: \ili{Livonian}, \ili{Estonian}, \ili{Votic}, \ili{Finnish}, \ili{Ingrian}, \ili{Karelian}, \ili{Lude} and \ili{Veps}.\footnote{The Võro\il{Estonian!Võro} variety of Estonian, the Meänkieli\il{Finnish!Meänkieli} and Kveeni\il{Finnish!Kveeni} varieties of Finnish, and the Olonets\il{Karelian!Olonets} variety of Karelian are not considered distinct languages here.}

\is{head\hyp{}driven agreement|(}
The Finnic branch is exceptional among Uralic in that all of its member languages regularly exhibit head\hyp{}driven agreement as the regular type of adjective attribution marking.

\il{Finnish|(}
\paragraph*{Head\hyp{}driven agreement in Finnish}
\label{finnish synchr}
The morphological features assigned to the head noun in Finnish are passed on to its adjectival (and other) modifiers. Finnish adjectives thus show a prototypical instance of head\hyp{}driven agreement. 
\begin{exe}
\ex {\rm Finnish (personal knowledge)}
\begin{xlist}
\ex
\gll	\textbf{iso} talo\\
	big house\\
\glt	‘big house’
%%%
\ex
\label{fin num}
 \gll	\textbf{iso-t} talo-t\\
	big-\textsc{pl} house-\textsc{pl}\\
\glt	‘big houses’
%%%
\ex
\label{fin case}
\gll	\textbf{iso-i-ssa}	talo-i-ssa\\
	big-\textsc{pl}-\textsc{iness} house-\textsc{pl}-\textsc{iness}\\
\glt	‘in big houses’
%%%
\ex 	
\label{fin poss}
\gll	\textbf{iso} {(*iso-ni)} talo-ni\\
	big big-\textsc{poss:1sg} house-\textsc{poss:1sg}\\
\glt	‘my big house’
\end{xlist}
\end{exe}
%%%
Note, however, that not all morphological features assign their values to the attributive adjective in Finnish. Whereas number (\ref{fin num}) and case marking (\ref{fin case}) are assigned to the adjective, possessive marking (\ref{fin poss}) is not (as noted earlier in \S\ref{head-driven agreement}).
\il{Finnish|)}
\il{Finnic languages|)}
\is{head\hyp{}driven agreement|)}

\il{Saamic languages|(}
\subsection{Saamic}
\label{saami synchr}
%%%
Saamic languages are spoken on the Scandinavian peninsula in north-central Norway and Sweden as well as in northern Finland and on the Kola peninsula in northwesternmost Russia. Saamic branches further into an eastern and a western subgroup.

The Saamic languages are exceptional among Uralic and the languages of most other families of \isi{Europe} in that they exhibit special attributive marking of adjectives, prototypically expressed by an invariable attributive suffix. In \S\ref{dep-marking state} of Part~II (Typology), this noun phrase type was characterized as \textit{dependent\hyp{}marked attributive state}; the corresponding formative is labeled \textit{anti\hyp{}construct state marker}. Note, however that the regular use of this inflectional category of adjectives and the relevant formatives vary considerably across the different Saamic languages.

\il{East Saamic languages|(}
\subsubsection{East Saamic}
%%%
The four living East Saamic languages Ter,\il{Ter Saami} Kildin,\il{Kildin Saami} Skolt\il{Skolt Saami} and Inari Saami\il{Inari Saami} are spoken on the Kola peninsula in northwesternmost Russia and in the adjacent parts of northern Finland.

\il{Skolt Saami|(}
\paragraph*{Anti\hyp{}construct state in Skolt Saami}\hspace{0.4cm}
Prototypically, the anti\hyp{}construct state marking suffix in Saamic languages has the shape \textit{-(V)s \textasciitilde-(V)s'}.\footnote{The palatalized variant occurs in Ter Saami and Kildin Saami.} The suffix is found in all Saamic languages (\citealt{riesler2006b}; see also \S\ref{saamic diachr} where the origin of attributive state marking in Saamic is dealt with in detail). 

\is{predicative marking|(}
In Skolt Saami, the prototypical pairs of predicative and attributive adjective forms are equipped with the suffixes \textit{-(V)d} \textsc{pred} and \textit{-(V)s} \textsc{attr} respectively, although other suffix pairs occur as well \citep[173–176]{feist2015a}. Whereas the suffix \textit{-(V)d} in \REF{skolt pred} marks the predicative state of the adjective, the suffix \mbox{\textit{-(V)s}} is an attributive state marker. The examples (\ref{skolt attr}) show that the formative is invariable and does not alter its form in a plural or case marked noun phrase.
%%%
\begin{exe}
\ex {\rm Skolt Saami (personal knowledge)}
\begin{xlist}
\ex
\label{skolt pred}
{\rm Predicative}
\begin{xlist}
\gll	tät nijdd lij \textbf{moočč-âd}\\
	this girl is beautiful-\textsc{pred}\\
\glt	‘this girl is beautiful’
\end{xlist}
%%%
\ex
\label{skolt attr}
{\rm Attributive} 
\begin{xlist}
\ex
\gll 	tät lij \textbf{mooʹčč-es} nijdd\\
	this is beautiful-\textsc{attr} girl\\
\glt	‘this is a beautiful girl’
%%%
\ex	
\gll	täk liâ \textbf{mooʹčč-es} niõđ\\
	this are beautiful-\textsc{attr} girl\textbackslash\textsc{pl}\\
\glt	‘these are beautiful girls’
%%%
\ex	
\gll	\textbf{mooʹčč-es} niõđ-i põrtt\\
	beautiful-\textsc{attr} girl-\textsc{gen.pl} house\\
\glt	‘the house of the beautiful girls’
\end{xlist}
\end{xlist}
\end{exe}
%%%
In all Saamic languages, attributive (and predicative) state marking of adjectives is complex and determined by certain lexically defined classes and subclasses of adjectives. Many adjectives are marked only for attributive state but show the unmarked stem form in the predicative form. Consider for instance \textit{neuʹrr} [\textsc{pred}] versus \textit{neeuʹr-es} [\textsc{attr}] ‘bad’, in Skolt Saami. In addition, in the predicative forms of several adjectives, suffixes other than \textit{-(V)d} also occur. Finally, there are a few adjectives which also use the attributive suffix in their predicative forms \citep[cf.][173–176]{feist2015a}.

In fact, a general tendency is noticeable in all Saamic languages: the differentiated morphological marking of predicative and attributive adjectives is being abolished in favor of using the pure or extended stem forms in both syntactic positions. As a result, attributive state marking seems to be in dissolution \citep{riesler2006b}. Several classes of adjectives, however, do not seem to be as affected by the functional spread of the \isi{juxtaposition}al type. In Skolt Saami, the anti\hyp{}construct state marker is even used productively in several derived adjective classes, such as with the abessive adjectivizer.\is{adjective derivation}
%%%
\begin{exe}
\il{Skolt Saami!Notozero}
\ex {\rm Derived adjectives in Skolt Saami (Notozero) \citep[279]{senkevic-g1968}}
\begin{xlist}
\ex {\rm Attributive}\\
\gll	\textbf{päärn-tʹem-es} neezzan\\
	child-\textsc{abess.adjz-attr} woman\\
\glt	‘(a) woman without children’
%%%
\ex {\rm Predicative}\\
\gll 	tät neezzan lij \textbf{päärn-tʹem}\\
	this woman is child-\textsc{abess.adjz}\\
\glt	‘this woman is without children’
\end{xlist}
\end{exe}

\is{juxtaposition|(}
\paragraph*{Juxtaposition in Skolt Saami}
Whereas dependent\hyp{}marked attributive state is the prototypical type of adjective attribution marking in Skolt (as well as in the other Saamic languages), certain adjectives are never inflected in their attributive form, one instance being \textit{nuõrr} ‘young’ \citep[cf.~also][176]{feist2015a}.
%%%
\begin{exe}
\ex {\rm Skolt Saami (personal knowledge)}
\begin{xlist}
\ex {\rm Attributive}
\begin{xlist}
\gll 	tät lij \textbf{nuõrr} nijdd\\
	this is young girl\\
\glt	‘this is a young girl’
%%%
\ex	
\gll	täk liâ \textbf{nuõrr} niõđ\\
	this are young girl\textbackslash\textsc{pl}\\
\glt	‘these are young girls’
\end{xlist}
%%%%%JW: warum zeigst du hier ein PRED-Beispiel? ist das wirklich relevant hier? Außerdem sieht man, dass die PRED-ATTR-Formen unterschiedlich sind (Stufenwechsel) – das wäre auch morphologische Markierung, nicht nur Juxtaposition.
\ex {\rm Predicative}\\
\gll	täk niõđ liâ \textbf{nuõr}\\
	this girl is young\textbackslash\textsc{pred.pl}\\
\glt	‘these girls are young’
\end{xlist}
\end{exe}
%%%
The noun phrase type in which ‘young’ and other members of this adjectival class occur must be characterized as {juxtaposition}. Hence, Skolt Saami exhibits a second, minor adjective attribution marking device in addition to attributive state marking.
\is{predicative marking|)}
\il{Skolt Saami|)}
\il{East Saamic languages|)}
\is{juxtaposition|)}

\il{West Saamic languages|(}
\subsubsection{West Saamic}
The five West Saamic languages are Northern,\il{Northern Saami} Lule,\il{Lule Saami} Pite,\il{Pite Saami} Ume\il{Ume Saami} and \ili{Southern Saami}. They are spoken in northern Norway and Sweden and in the adjacent parts of northern Finland.

\il{Northern Saami|(}
\il{Pite Saami|(}
\is{predicative marking|(}
The default adjective attribution marking device in all West Saamic languages is anti\hyp{}construct state marking, just as in East Saamic.\il{East Saamic languages} Only the few members of a marginal subclass of adjectives are attributed by means of other devices. In general, West Saamic languages are similar to East Saamic in their high degree of irregularity in the morphological marking of attributive adjectives, although grammars of \ili{Northern Saami}, usually taking a rather normative-descriptive approach (e.g., \citealt{nickel1990,sammallahti1998b,svonni2009a}), stress the systemic character of attributive versus predicative marking with the suffix \textit{-(V)s} being the prototypical  formative for attributive morphology.

For another West Saamic language, Pite Saami, and using exclusively corpus data \citet[128–129]{wilbur2014a} argues that the formative \textit{-(V)s} is used much too irregularly to be considered a productive attributive suffix. Because of the considerable inconsistencies in morphological patterns between corresponding attributive and predicative adjectives, \cite[134]{wilbur2014a} generally prefers to analyze these two sets of adjectives simply as semantically and etymologically related, rather than morphologically derivable adjectives. However, even if a large part of these adjectives consists of suppletive pairs, the morpho-syntax of adjectives in Pite Saami shares one important characteristic with the other Saamic languages: whereas attributive adjectives never show morphological agreement, predicative adjectives agree (in \textsc{number}) with the subject noun phrase.
\is{predicative marking|)} 
\il{Pite Saami|)}

\is{head\hyp{}driven agreement|(}
\paragraph*{Head\hyp{}driven agreement in Northern Saami}
For Northern Saami, the default attribution device is anti\hyp{}construct state marking, like in all Saamic languages. A few adjectives, however, regularly show agreement with the head noun in number and case. In Northern Saami, the adjective ‘good’ and sometimes also the adjective ‘bad’ follow this type.
%%%
\begin{exe}
\settowidth\jamwidth{[good-\textsc{gen:pl}(\textasciitilde \textsc{com:pl}) knife-\textsc{com:pl}]}
\ex {\rm Northern Saami \citep[83]{nickel1990}}
\begin{xlist}
\ex	buorre niibi						{\rm ‘good knife’}	\jambox{{\rm [good(\textsc{nom:sg}) knife(\textsc{nom:sg})]}}
\ex	buori niibbi										\jambox{{\rm [good\textbackslash\textsc{gen:sg} knife\textbackslash\textsc{gen:sg}]}}
\ex	buori niibá-i										\jambox{{\rm [good\textbackslash\textsc{gen:sg} knife-\textsc{ill:sg}]}}
\ex	buori niibi-s										\jambox{{\rm [good\textbackslash\textsc{gen:sg} knife-\textsc{loc:sg}]}}
\ex	buri-in niibbi-in										\jambox{{\rm [good-\textsc{com:sg} knife-\textsc{com:sg}]}}
\ex	buori-t niibbi-t										\jambox{{\rm [good-\textsc{nom:pl} knife-\textsc{nom:pl}]}}
\ex	buori-id niibbi-id									\jambox{{\rm [good-\textsc{gen:pl} knife-\textsc{gen:pl}]}}
\ex	buori-id(\textasciitilde ide) niibbi-ide						\jambox{{\rm [good-\textsc{gen:pl}(\textasciitilde \textsc{ill:pl}) knife-\textsc{ill:pl}]}}
\ex	buri-in niibbiin										\jambox{{\rm [good\textbackslash\textsc{loc:pl} knife-\textsc{loc:pl}]}}
\ex	buori-id(\textasciitilde iguin) niibbi-iguin					\jambox{{\rm [good-\textsc{gen:pl}(\textasciitilde \textsc{com:pl}) knife-\textsc{com:pl}]}}
\ex	buorri-n niibi-n										\jambox{{\rm [good-\textsc{ess} knife-\textsc{ess}]}}
\end{xlist}
\end{exe}
%%%
Note that the agreement inflection of the adjective can be characterized as defective because it does not distinguish all single case forms in the paradigm.\is{agreement marking!defective agreement paradigm}
\il{Northern Saami|)}
\il{West Saamic languages|)}
\il{Saamic languages|)}
\il{Uralic languages|)}
\is{head\hyp{}driven agreement|)}

\il{Indo-European languages|(}
\section{Indo-European}
%%%
Indo-European is among the world's language families with the greatest geographic distribution. Most of the European languages belong to this family. But Indo-European languages are spoken as far East as on the \isi{South Asia}n subcontinent. The family can be divided into nine branches \citep[218]{salminen2007}, all of which are represented in the present investigation.

The prototypical adjective attribution marking type in Indo-European is \isi{head\hyp{}driven agreement}. This type is also reconstructed for the \ili{Proto\hyp{}Indo-European} language \citep{decsy1991,watkins1998}. Due to the development of certain secondary types of adjective attribution marking devices, however, divergence is relatively high inside the Indo-European family. Furthermore, in several branches of Indo-European, \isi{head\hyp{}driven agreement} has been lost in favor of various other types of attribution marking (as will be shown in Part~IV Diachrony).

Among the languages of northern Eurasia, the Indo-European family exhibits the highest diversity with regard to the number of possible adjective attribution marking devices. The following types are attested in different Indo-European languages:
%%%
\begin{itemize}
\item juxtaposition\is{juxtaposition}
\item head\hyp{}driven agreement\is{head\hyp{}driven agreement}
\item construct-state marking
\item anti\hyp{}construct state marking
\item anti\hyp{}construct state agreement marking
\item attributive nominalization\is{attributive nominalization}
\item incorporation.
\end{itemize}

\il{Albanian languages|(}
\subsection{Albanian}
\label{albanian synchr}
%%%
The Albanian branch of Indo-European is represented by the two languages Standard \ili{Albanian} and \ili{Arvanitika}.

\il{Albanian|(}
\is{attributive nominalization|(}
\is{head\hyp{}driven agreement|(}
\paragraph*{Attributive nominalization + head\hyp{}driven agreement in Albanian}
In both Albanian languages, adjectives normally follow the head noun and are marked with an article which links its host to the modified noun. Additionally, adjectives are equipped with agreement inflection suffixes co-referencing the \textsc{number}-, \textsc{gender}-, \textsc{case}- and \textsc{species}\is{species marking} values of the head noun. The language thus exhibits an attributive marking device which is a combination of a phonologically free article (historically an attributive nominalizer) and agreement suffixes.
%%%

\newpage 
\begin{exe}
\ex {\rm Standard Albanian \citep[examples from][166–167]{himmelmann1997}}
\label{albanian ex}
\begin{xlist}
\ex
\gll	një	shok 	\textbf{i}	\textbf{mirë}\\
	one:\textsc{m}	friend:\textsc{indef:m} 	\textsc{attr:nom.sg.m}	good:\textsc{nom.sg.m}\\
\glt	‘one good friend’
%%%
\ex	
\gll	shok=u			\textbf{i}			\textbf{mirë}\\
	friend=\textsc{def:nom.sg.m} 	\textsc{attr:nom.sg.m} 	good:\textsc{nom.sg.m}\\
\glt	‘the good friend’
%%%
\ex
\gll	shok=un					\textbf{e}			\textbf{mirë}\\
	friend=\textsc{def:acc.sg.m} 	\textsc{attr:acc.sg.m} 	good:\textsc{acc.sg.m}\\
\glt	‘the good friend (acc.)’
\end{xlist}
\end{exe}
%%%
Note that the circum-positioned agreement marker also occurs with predicative adjectives.\is{predicative marking} 
%%%
\begin{exe}
\ex {\rm Standard Albanian \citep{demiraj1998}}\\
%\begin{xlist}
%\ex Predicative agreement of “article adjectives”
\gll	shok=u është \textbf{i} \textbf{bukur}\\
	friend-\textsc{def:nom.sg.m} be\textsc{.3sg.prs} \textsc{attr:nom.sg.m} pretty:\textsc{nom.sg.m}\\
\glt	‘the friend is pretty’
%\ex Predicative agreement of “simple” adjectives
%\gll	shok=u është \textbf{besnik}\\
%	friend-\textsc{def:nom.sg.m} be\textsc{.3sg.prs} true:\textsc{nom.sg.m}\\
%\glt	‘the friend is faithful’
%\end{xlist}
\end{exe}
%%%
\is{predicative marking|(}
Since adjectives in attributive and predicative position are both equipped with the circumfixed agreement marker the language seems to belong simply to the head\hyp{}driven agreement type. However, true predicative adjectives are not found in Albanian. Instead, attributive adjectives in \isi{headless noun phrase}s are used in predicative position. This is evidenced by case agreement of predicates.
%%%
\begin{exe}
\ex {\rm Standard Albanian \citep{demiraj1998}}
\begin{xlist}
\ex
\gll	Agimi {u kthye} \textbf{i} \textbf{dëshpëruar}\\
	Agimi(\textsc{nom.sg.m}) returned \textsc{attr:nom.sg.m} sorrowful:\textsc{nom.sg.m}\\
\glt	‘Agim returned sorrowfully’
%%%
\ex
\gll	Agimi(\textsc{acc.sg.m}) e pashtë \textbf{të} \textbf{dëshpëruar}\\
	Agimi I saw \textsc{attr:acc.sg.m} sorrowful:\textsc{acc.sg.m}\\
\glt	‘I saw Agimi sorrowful’
\end{xlist}
\end{exe}
%%%
On the other hand, the similar agreement behavior of attributive and predicative adjectives seems to indicate the absence of specific attributive morpho-syntactic marking. However, the attributive article is polyfunctional and can also link other adnominal attributes in addition to adjectives to the modified noun. The analysis of adjective attribution marking in Albanian as belonging to the attributive nominalization type (in combination with head\hyp{}driven agreement) thus seems justified.
%%%
\begin{exe}
\ex {\rm Standard Albanian \citep{demiraj1998}}
\begin{xlist}
\ex
\gll	roman-i 			\textbf{i} 			tretë\\
	novel-\textsc{def:nom.sg.m} \textsc{attr:nom.sg.m} third\\
\glt	‘the third novel’
%%%
\ex	
\gll	libr-i 	\textbf{i} nxënës-it\\
	book(\textsc{m})-\textsc{def:nom.sg.m} \textsc{attr:nom.sg.m} pupil-\textsc{def:gen/dat.sg}\\
\glt	‘the pupil's book’
\end{xlist}
\end{exe}
\is{attributive nominalization|)}

\paragraph*{Head\hyp{}driven agreement in Albanian}
Note, however, that the occurrence of the attributive article is restricted to a lexically defined subclass of adjectives in Albanian: only the so-called “article adjectives” are regularly marked with the article. Other adjectives are marked with head\hyp{}driven agreement affixes alone.
%%%
\begin{exe}
\ex {\rm Standard Albanian \citep[examples from][167]{himmelmann1997}}
\begin{xlist}
\ex
\gll	shok=u					\textbf{besnik}\\
	friend-\textsc{def:nom.sg.m} 	true:\textsc{nom.sg.m}\\
\glt	‘the faithful friend’
%%%
\ex
\gll 	një			shok					\textbf{besnik}\\
	one:\textsc{m}	friend:\textsc{indef:m} 	true:\textsc{nom.sg.m}\\
\glt	‘one faithful friend’
\end{xlist}
\end{exe}
%%%
Again, predicative adjectives behave similar to attributive adjectives.
%%%
\begin{exe}
\ex {\rm Standard Albanian \citep{demiraj1998}}\\
\begin{xlist}
\ex {\rm Predicative agreement of “article adjectives”}\\
\gll	shok=u është \textbf{i} \textbf{bukur}\\
	friend-\textsc{def:nom.sg.m} be\textsc{.3sg.prs} \textsc{attr:nom.sg.m} pretty:\textsc{nom.sg.m}\\
\glt	‘the friend is pretty’
%%%
\ex {\rm Predicative agreement of “simple” adjectives}\\
\gll	shok=u është \textbf{besnik}\\
	friend-\textsc{def:nom.sg.m} be\textsc{.3sg.prs} true:\textsc{nom.sg.m}\\
\glt	‘the friend is faithful’
\end{xlist}
\end{exe}
\il{Albanian|)}
\is{predicative marking|)}
\is{head\hyp{}driven agreement|)}

\il{Arvanitika|(}
\is{attributive nominalization|(}
\is{head\hyp{}driven agreement|(}
\paragraph*{Attributive nominalization + head\hyp{}driven agreement in Arvanitika}
Adjective attribution marking in Arvanitika is very similar to Standard \ili{Albanian}. One adjective class shows head\hyp{}driven agreement marking by means of suffixes. The second adjective class is cognate with the so-called “article adjectives” in Albanian and exhibits attributive nominalization. 
%%%
\begin{exe}
\ex {\rm Arvanitika \citep[303]{sasse1991}}
\begin{xlist}
\ex
\gll	ɲə́ 			djáʎə 			\textbf{i-mírə}\\
	one:\textsc{m} 	boy:\textsc{indef.m} 	\textsc{m}-good:\textsc{m}\\
\glt	‘one good boy’
%%%
\ex
\gll				djáʎi 				\textbf{i-mírə}\\
				boy:\textsc{def.m} 	\textsc{m}-good:\textsc{m}\\
\glt	‘the good boy’
\end{xlist}
\end{exe}
%%%
Unlike in Standard Albanian,\il{Albanian} however, the preposed attributive nominalizer in Arvanitika is a phonologically bound formative. This is evidenced by its phonological behavior in adjective compounds, where the marker remains in its position bound to the adjective stem.
%%%
\begin{exe}
\ex {\rm Arvanitika \citep[304]{sasse1991}}
\label{alb noclitic}
\begin{xlist}
\ex[]{
\gll	\textbf{miso-i-}ngrə́nə / \textbf{miso-tə-}ngrə́nə\\
	half-\textsc{m}-mounted:\textsc{m} { } half-\textsc{acc.m}-mounted:\textsc{m}\\
\glt	‘half-mounted’
}
%%%
\ex[*]{i-miso-ngrə́nə / tə-miso-ngrə́nə}
\end{xlist}
\end{exe}
%%%
Example (\ref{alb noclitic}) shows that the compound degree word \textit{miso-} does not move between the adjective stem and the attributive nominalizer. Consequently, the nominalizer can be characterized as a \isi{clitic} (because it is phonologically bound but morpho-syntactically free) which always attaches on a fixed position, i.e., on the left of the adjective stem.\footnote{Note, however, that the agreement categories \textsc{case/number/gender} are merged into several differentiated morphemes in the suffixed part of the circumfix \citep[124–128]{sasse1991}.}
\il{Arvanitika|)}
\il{Albanian languages|)}
\is{attributive nominalization|)}
\is{head\hyp{}driven agreement|)}

\il{Armenian languages|(}
\il{Eastern Armenian|(}
\subsection{Armenian}
\label{armenian-synch}
%%%
Armenian is a branch consisting only of two closely related varieties, of which only the Eastern Armenian standard language is considered here.

\is{juxtaposition|(}
\paragraph*{Juxtaposition in Eastern Armenian} 
In the unmarked construction, attributive adjectives are unmarked and precede the modified noun.
%%%

\newpage 
\begin{exe}
\ex {\rm Eastern Armenian \citep{ajello1998}}
\begin{xlist}
\ex 
\gll	\textbf{bari} gorc\\
	good work(\textsc{nom.sg})\\
\glt	‘good work’
%%%
\ex 
\gll	\textbf{bari} gorc-s\\
	good work-\textsc{acc.pl}\\
\glt	‘good work (acc.)’
\end{xlist}
\end{exe}
\is{juxtaposition|)}

\is{head\hyp{}driven agreement|(}
\paragraph*{Head\hyp{}driven agreement in Armenian}
A few monosyllabic adjectives show head\hyp{}driven agreement marking in Armenian. 

In theory, however, all adjectives in an emphatic construction can occur in a noun phrase with reversed constituent order. In “emphatic position” \citep[224]{ajello1998}, i.e., in \isi{contrastive focus} attributive adjectives show agreement in case and number as a rule.
%%%
\begin{exe}
\ex {\rm Eastern Armenian \citep[224]{ajello1998}}\\
\gll	bazum gorc-s \textbf{bari-s}\\
	much work-\textsc{acc.pl} good-\textsc{acc.pl}\\
\glt	‘much \textsc{good} work (acc.)’
\end{exe}
\il{Eastern Armenian|)}
\il{Armenian languages|)}
\is{head\hyp{}driven agreement|)}

\il{Indo-Iranian languages|(}
\subsection{Indo-Iranian}
%%%
Indo-Iranian (aka Aryan)\il{Aryan|see{Indo-Iranian languages}} is a major branch within Indo-European. But only a few Indo-Iranian languages belonging to the Iranian\il{Iranian languages} and Indo-Aryan\il{Indo-Aryan languages} subbranches are spoken in northern Eurasia and thus considered here. Most other Indo-Iranian languages are spoken in the \isi{Middle East} and in \isi{South Asia} and hence outside the investigated geographic area. 

\il{Indo-Aryan languages|(}
\il{Romani languages|(}
\subsubsection{Indo-Aryan}
%%%
Indo-Aryan (aka Indic)\il{Indic|see{Indo-Aryan languages}} is a large subbranch of Indo-Iranian, most member languages of which are spoken on the \isi{South Asia}n subcontinent. Outlier languages, spoken in northern Eurasia include \ili{Parya}, a language which was recently discovered in Tajikistan in \isi{Inner Asia} \citep[22]{masica1991}, and the group of Romani languages. Several varieties of Romani are spoken all over \isi{Europe}. Some of them are not mutually intelligible. Rather than being one single language, Romani is thus a group of languages which comprise at least the four subbranches Vlax Romani,\il{Vlax Romani languages} Balkan Romani,\il{Balkan Romani languages} Central Romani\il{Central Romani languages} and North Romani\il{North Romani languages} with several sub-varieties in each of them \citep[2–3]{halwachs-etal2002}.

The default type of adjective attribution marking in Indo-Aryan languages is \isi{head\hyp{}driven agreement} in noun phrases with head-final constituent order \citep[369]{masica1991}. Agreement features in the Romani languages are \textsc{gender} and \textsc{number}, and in most varieties also \textsc{case}. The unmarked constituent order in all varieties of Romani is adjective-noun.

\il{Burgenland Romani|(}
\is{head\hyp{}driven agreement|(}
\paragraph*{Head\hyp{}driven agreement in Burgenland Romani}
\label{romani synchr}
In the Burgenland variety of Romani, adjectives normally show agreement in gender, number and also case with the head noun. Case agreement, however, can be characterized as defective, since all attributive adjectives preceding oblique cases have one similar oblique form.\is{agreement marking!defective agreement paradigm}
%%%
\begin{exe}
\ex {\rm Burgenland Romani \citep[22–23]{halwachs-etal2002}}
\begin{xlist} 
\ex 
\gll	\textbf{bar-o} phral\\
	big-\textsc{nom:m.sg} brother(\textsc{m})\\
\glt	‘big brother’
%%%
\ex
\gll	\textbf{bar-i} phen\\
	big-\textsc{nom:f.sg} sister(\textsc{f})\\
\glt	‘big sister’
\end{xlist}
\end{exe}
\is{head\hyp{}driven agreement|)}

\is{juxtaposition|(}
\paragraph*{Juxtaposition in Burgenland Romani}
A minor lexically defined subclass of adjectives in Burgenland Romani is indeclinable and juxtaposed to the head noun.
%%%
\begin{exe}
\ex {\rm Burgenland Romani \citep[22–23]{halwachs-etal2002}}
\begin{xlist}
\ex 
\gll	\textbf{schukar} phral\\
	beautiful brother(\textsc{m})\\
\glt	‘beautiful brother’
%%%
\ex
\gll	\textbf{schukar} phen\\
	beautiful sister(\textsc{f})\\
\glt	‘beautiful sister’
\end{xlist}
\end{exe}
\il{Burgenland Romani|)}
\is{juxtaposition|)}

\is{attributive nominalization|(}
\paragraph*{Attributive nominalization in Vlax Romani}
\citet{hancock1995} describes the use of a “repeated definite article”\is{species marking!definite} in \isi{contrastive focus} constructions in Vlax Romani. 
%%%

\newpage 
\begin{exe}
\ex {\rm Vlax Romani \citep[30]{hancock1995}}
\begin{xlist}
\ex {\rm Head\hyp{}driven agreement (unmarked construction)}\is{head\hyp{}driven agreement}\\
\gll	o \textbf{baro} raklo\\
	\textsc{def}	big	boy\\
\glt ‘the big boy’
%%%
\ex {\rm Attributive nominalization (emphatic construction)}\\
\gll	o raklo \textbf{o} \textbf{baro}\\
	\textsc{def}	{boy}	\textsc{attr} big\\
\glt	‘the \textsc{big} boy’
\end{xlist}
\end{exe}
\il{Romani languages|)}
\il{Indo-Aryan languages|)}
\is{attributive nominalization|)}

\il{Iranian languages|(}
\subsubsection{Iranian}
\label{iranian synchr}
The second subbranch of Indo-Iranian is formed by Iranian languages, only a few of which are spoken in northern Eurasia.

\is{construct state|(}
A well-known characteristic of noun phrase structure in Iranian languages is the occurrence of the Ezafe construct marking which licenses the attribution of adjectives (and other syntactic classes of modifiers). The Iranian languages surveyed in the present investigation, however, exhibit some diversity in this respect. Attributive construct state marking occurs regularly only in the western Iranian languages Northern Kurdish\il{Northern Kurdish} (aka Kurmanji,\il{Kurmanji|see{Northern Kurdish}} Kirmancî)\il{Kirmancî|see{Northern Kurdish}} and \ili{Tajik}.

\il{Tajik|(}
\paragraph*{Attributive construct state in Tajik}
Tajik follows the Iranian prototype and exhibits a head-marking construct state marking suffix.
%%%
\begin{exe}
\ex {\rm Tajik \citep{rastorgueva1963}}
\begin{xlist}
\ex
\gll	duxtar\textbf{-i} \textbf{xušrūj}\\
	girl-\textsc{attr} beautiful\\
\glt	‘a pretty girl’
%%%
\ex
\gll	duxtar-on\textbf{-i} \textbf{xušrūj}\\
	girl-\textsc{pl}-\textsc{attr} beautiful\\
\glt	‘pretty girls’
\end{xlist}
\end{exe}
\il{Tajik|)}
\is{construct state|)}

\il{Northern Talysh|(}
\paragraph*{Anti\hyp{}construct in Northern Talysh}
\label{talysh synchr}
The constituent order in noun phrases in Northern Talysh is adjective-noun. The language is exceptional among the Iranian (and Indo-European) languages considered here in exhibiting dependent\hyp{}marking anti\hyp{}construct state instead of head-marking \isi{construct state} as the default type of adjective attribution marking.
%%%
\begin{exe}
\ex {\rm Northern Talysh \citep[27]{schulze2000}}%CHECK PAGES
\begin{xlist}
\ex
\gll	\textbf{āğəlmānd-a} odam-on\\
	clever-\textsc{attr} man-\textsc{pl}\\
\glt	‘clever people’
%%%
\ex
\gll	\textbf{yol-a} di\\
	big-\textsc{attr} tree\\
\glt	‘(a) big tree’
\end{xlist}
\end{exe}
\il{Northern Talysh|)}

\il{Ossetic|(}
\is{juxtaposition|(}
\paragraph*{Juxtaposition in Ossetic}
Ossetic is another exceptional language among Iranian, because the language exhibits juxtaposition as the default type of adjective attribution marking.
%%%
\begin{exe}
\ex {\rm Ossetic \citep[12]{abaev1964}}
\label{ossetic attrcomp}
\begin{xlist}
\ex {\rm Simple noun}\\
\gll	færǽt / fǽræt\\
	ax { } ax\textbackslash\textsc{def}\\
\glt	‘axe’ / ‘the axe’
%%%
\ex {\rm Noun phrase with adjectival modifier}\\
\gll	\textbf{cyrg'-}fǽræt / \textbf{cýrg'-}færæt\\
	sharp-axe { } sharp-ax\textbackslash\textsc{def}\\
\glt	‘sharp axe’ / ‘the sharp axe’
\end{xlist}
\end{exe}
%%%
Stress patterns provide evidence for the analysis of Ossetic noun phrase structure as phonological compounds. According to \citet[10]{abaev1964}, “syntactically connected word groups” (such as noun phrases) are marked by single stress. Note that stress, moving from the second to the first syllable marks definiteness\is{species marking!definite} in Ossetic \citep[12]{abaev1964}. There is, however, no evidence that the compounded adjectives are syntactically incorporated.
\is{juxtaposition|)}

Note that attributive \isi{construct state} marking which is cognate with the Ezafe in other Iranian languages occurs in Ossetic as well, but its use is restricted to certain “emphatic”, i.e., \isi{contrastive focus} constructions \citep[467]{thodarson1989}.
\il{Ossetic|)}
\il{Iranian languages|)}
\il{Indo-Iranian languages|)}

\il{Baltic languages|(}
\subsection{Baltic}
\label{baltic synchr}
%%%
\il{East Baltic languages|(}
\subsubsection{East Baltic}
%%%
The Baltic languages form a small branch among Indo-European and are represented in the present survey only by the two languages Lithuanian\il{Lithuanian} and Latvian.\il{Latvian} Both belong to the eastern subbranch of Baltic. All languages from the former western branch\il{West Baltic languages} of Baltic are extinct.

\is{species marking!definite|(}
\is{head\hyp{}driven agreement|(}
Two types of adjective attribution marking occur in modern Baltic languages: head\hyp{}driven agreement and anti\hyp{}construct state agreement. In the descriptive literature on Baltic languages, however, these two noun phrase types are normally not ascribed to syntax, but are described as different agreement declension types determined by the definite or indefinite semantics of the noun phrase.

In \S\ref{anti-constr agr} of Part~II (Typology) I have already argued extensively in favor of a syntactic differentiation of these two agreement marking devices in Baltic (as well as in various Slavic)\il{Slavic languages} languages. Consequently and for the sake of completeness, examples of head\hyp{}driven agreement marking (the so-called indefinite declension) and anti\hyp{}construct state agreement marking (the so-called definite declension) in Latvian\il{Latvian} and Lithuanian\il{Lithuanian} will be repeated in the following paragraphs.

\il{Latvian|(}\il{Lithuanian|(} 
\paragraph*{Head\hyp{}driven agreement in Latvian and Lithuanian} 
Adjectives modifying indefinite nouns show head\hyp{}driven agreement in Latvian and Lithuanian.
%%%
\begin{exe}
\ex 
\begin{xlist}
\ex {\rm Latvian \citep[example from][122]{dahl2015a}}\\
\gll 	\textbf{liel-a} māja\\
	big-\textsc{f.nom.sg} house(\textsc{f})\\
\glt	‘a big house’
%%%

\ex {\rm Lithuanian \citep[13]{bechert1993}}\\
\gll 	\textbf{gẽr-as}			profèsorius\\
	good-\textsc{nom.sg.m} professor(\textsc{m})\\
\glt	‘a good professor’
\end{xlist}
\end{exe}
\is{head\hyp{}driven agreement|)}

\paragraph*{Anti\hyp{}construct state agreement in Latvian and Lithuanian}
Adjectives modifying definite nouns show anti\hyp{}construct state agreement marking in Latvian and Lithuanian.
%%%
\begin{exe}
\ex 
\begin{xlist}	
\ex {\rm Latvian \citep[example from][122]{dahl2015a}}\\
\gll 	\textbf{liel-ā} māja\\
	big-\textsc{attr:f.nom.sg} house(\textsc{f})\\
\glt	‘the big house’
%%%
\newpage 
\ex {\rm Lithuanian \citep[13]{bechert1993}}\\
\gll 	\textbf{ger-àsis}		profèsorius\\
	good-\textsc{attr:nom.sg.m}	professor(\textsc{m})\\
\glt	‘the good professor’
\end{xlist}
\end{exe}
\il{Latvian|)}\il{Lithuanian|)}
\is{species marking!definite|)}
\il{East Baltic languages|)}
\il{Baltic languages|)}

\is{head\hyp{}driven agreement|(}
\il{Celtic languages|(}
\subsection{Celtic}
%%%
The modern Celtic languages belong to two main branches: Gaelic\il{Gaelic languages} and Brittonic.\il{Brittonic languages} By and large, all Celtic languages have preserved the Proto\hyp{}Celtic\il{Proto\hyp{}Celtic} noun phrase structure, including head\hyp{}driven agreement marking on attributive adjectives and noun-adjective constituent order.

\il{Gaelic languages|(}
\subsubsection{Gaelic}
%%%
\il{Scots Gaelic|(}
\paragraph*{Head\hyp{}driven agreement in Scots Gaelic} 
In Scots Gaelic (aka Scottish Gaelic)\il{Scottish Gaelic|see{Scots Gaelic}} adjectives (as well as other modifiers) show agreement in \textsc{gender, number}, and \textsc{case}.
%%%
\begin{exe}
\ex {\rm Scots Gaelic \citep[201]{macauley1992}}
\begin{xlist}
\ex
\gll	an cù \textbf{dubh}\\
	\textsc{def:m} dog(\textsc{m}) black\textbackslash\textsc{m}\\
\glt	‘the black dog’
%%%
\ex
\gll	a' chaora \textbf{dhubh}\\
	\textsc{def:f} sheep(\textsc{f}) black\textbackslash\textsc{f}\\
\glt	‘the black sheep’
\end{xlist}
\end{exe}
%%%
Similar agreement patterns as in Scots Gaelic, with non-linear marking by means of word-initial permutation, are found in Irish \citep[73, 97]{odochartaigh1992}. In the third Gaelic language \ili{Manx}, however, most adjectives are used in an invariable form. Only a certain subclass of monosyllabic adjectives have preserved number agreement in Manx \citep[127]{thomsen1992}.
\il{Scots Gaelic|)}
\il{Gaelic languages|)}

\il{Brittonic languages|(}
\subsubsection{Brittonic}
%%%
The tendency towards a loss of agreement inflection of adjectives is also noticeable in the languages of the Brittonic branch of Celtic. Adjective inflection seems to be most intact in Welsh\il{Welsh} with preserved gender and number agreement \citep[298–299]{thomas1992a}. Breton\il{Breton} and Cornish\il{Cornish} exhibit only agreement in gender (\citealt[405]{ternes1992}; \citealt[355]{thomas1992b}).
\il{Brittonic languages|)}
\il{Celtic languages|)}
\is{head\hyp{}driven agreement|)}

\il{Germanic languages|(}
\subsection{Germanic}
%%%
The modern Germanic languages belong to two branches: North\il{North Germanic languages} and West Germanic.\il{West Germanic languages} The third Germanic subbranch, East Germanic,\il{East Germanic languages} is extinct and is not considered here.

The constituent order of adjective and noun is relatively strictly head-final in all modern Germanic languages.\footnote{The exclusive adjective-initial constituent order in modern Germanic languages is clearly innovative. In documents of all \ili{Old Germanic languages}, the order of adjective and noun was still relatively free \citep[cf.][]{heinrichs1954}.} Most Germanic languages have also preserved the inherited agreement marking on attributive adjectives. But several secondary attributive marking devices have evolved at different stages in the history of Germanic.

The following noun phrase types occur inside the Germanic branch of Indo-European:
\begin{itemize}
%%%
\item{Anti\hyp{}construct state agreement}
\item{Anti\hyp{}construct state agreement + head\hyp{}driven agreement}
\item{Attributive article + \isi{head\hyp{}driven agreement}}
\item{Head\hyp{}driven agreement}
\item{Incorporation.}
\end{itemize}
%%%
Whereas \isi{head\hyp{}driven agreement} and \isi{attributive nominalization} are attested for the earliest stages of Germanic, adjective incorporation is a rather recent innovation (cf.~\S\ref{germanic diachr}).

\il{West Germanic languages|(}
\subsubsection{West Germanic}
\label{w-germanic synchr}
The most common type of adjective attribution marking in West Germanic languages is \isi{head\hyp{}driven agreement}. In most languages of this group, this is the only existing type.

\il{German|(}
\paragraph*{Anti\hyp{}construct state agreement in German}
Attributive adjectives in German show \isi{head\hyp{}driven agreement} according to the features \textsc{gender, number, case} and \textsc{species}.\is{species marking} The complete agreement paradigm was illustrated in Part~II (Typology) (Figure~\ref{german agr} on page~\pageref{german agr}). Note that the adjective agreement paradigm of German exhibits a high degree of syncretism due to merger of originally differentiated formatives. The whole paradigm distinguishes only the four suffixes \textit{-e, -em, -en, -er, -es}.
%%%
\begin{exe}
\ex {\rm Attributive adjectives in German (personal knowledge)}
\begin{xlist}
\ex
\gll	ein \textbf{hoh-es} Haus\\
	\textsc{indef} high-\textsc{indef.n} house(\textsc{n})\\
\glt	‘a high house’
%%%
\ex	
\gll	das \textbf{hoh-e} Haus\\
	\textsc{def} high-\textsc{def.n} house(\textsc{n})\\
\glt	‘the high house’
%%%
\ex	
\gll	\textbf{hoh-e} Häus-er\\
	high-\textsc{pl} house-\textsc{pl}\\
\glt	‘high houses’
%%%
\ex	
\gll	der \textbf{hoh-en} Häus-er\\
	\textsc{def:pl.gen} high-\textsc{def.pl.gen} house-\textsc{pl.gen}\\
\glt	‘of the high houses’
\end{xlist}
\end{exe}
%%%
\is{predicative marking|(}
Attributive and predicative adjectives are morpho-syntactically differentiated in German (and the other West Germanic languages, except \ili{English}): whereas attributive adjectives show \isi{head\hyp{}driven agreement}, predicative adjectives are used in an invariable form. Given the definition of dependent\hyp{}marking attributive state which is applied here (see \S\ref{syntax-morphology-interface}), German thus exhibits anti\hyp{}construct state agreement marking of attributive adjectives.
%%%
\begin{exe}
\ex {\rm Predicative adjectives in German (personal knowledge)}
\begin{xlist}
\ex
\gll	das / ein Haus is \textbf{hoch}\\
	\textsc{def} {} \textsc{indef} house(\textsc{n}) is high\\
\glt	‘a / the house is high’
%%%
\ex	
\gll	(die) Häus-er sind \textbf{hoch}\\
	\textsc{def} house-\textsc{pl} are high\\
\glt	‘(the) houses are high’
\end{xlist}
\end{exe}
\il{German|)}
\is{predicative marking|)}

\il{Yiddish|(}
\is{head\hyp{}driven agreement|(}
\paragraph*{Attributive nominalization + head\hyp{}driven agreement in Yiddish}
\label{yiddish synchr}
The default noun phrase structure in Yiddish is similar to the other West Germanic languages. Head\hyp{}driven agreement occurs as the default type of attribution marking of adjectives. In contrastive focus constructionс, however, adjectives and other modifiers follow the modified noun in an \isi{attributive nominalization} construction.
%%%
\il{Yiddish!Eastern}
\begin{exe}
\ex {\rm Yiddish (Eastern) \citep[96]{jacobs-etal1994}}
\begin{xlist}
\ex {\rm Head\hyp{}driven agreement (unmarked)}
\begin{xlist}
\ex 
\gll	a \textbf{sheyn} meydl\\
	\textsc{indef:f} pretty:\textsc{indef.f} girl\textsc{(f)}\\
\glt	‘a pretty girl’
%%%
\ex
\gll	di \textbf{grine} oygn\\
	\textsc{def:pl} green:\textsc{def.pl} eye:\textsc{pl}\\
\glt	‘the green eyes’
\end{xlist}
%%%
\ex {\rm Attributive nominalization (contrastive focus)}
\begin{xlist}
\ex
\gll	a meydl \textbf{a} \textbf{sheyne}\\
	\textsc{indef:f} girl\textsc{(f)} \textsc{attr:indef.f} pretty:\textsc{attr:indef.f}\\
	%andere Endung, was ist es
\glt	‘a \textsc{pretty} girl’
%%%
\ex
\gll	di oygn \textbf{di} \textbf{grine}\\
	\textsc{def:pl} eye:\textsc{pl} \textsc{attr:def.pl} green:\textsc{def.pl} \\
\glt	‘the \textsc{green} eyes’
\end{xlist}
\end{xlist}
\end{exe}
%umgekehrte Wortfolge deshalb kein normales agreement
\il{Yiddish|)}
\is{head\hyp{}driven agreement|)}

\il{English|(}
\paragraph*{Incorporation in English}
English is the only West Germanic language where \isi{head\hyp{}driven agreement} is missing completely because the original Germanic agreement inflection on adjectives was lost. 
%%%
\begin{exe}
\ex {\rm English (personal knowledge)}
\begin{xlist}
\ex 
\gll	a \textbf{pretty} girl\\
	\textsc{indef} pretty girl\\
%%%
\ex
\gll	the \textbf{pretty} girl\\
	\textsc{def} pretty girl\\
%%%
\ex 
\gll	\textbf{pretty} girl-s\\
	pretty girl-\textsc{pl}\\
\end{xlist}
\end{exe}
%%%
Attributive adjectives cannot, however, occur in \isi{headless noun phrase}s in English but are obligatorily marked with an article used as dummy head.
%%%
\begin{exe}
\ex {\rm English (personal knowledge)}
\begin{xlist}
\ex
\gll	a / the \textbf{smart} \textbf{one}\\
	\textsc{indef} {} \textsc{def} smart \textsc{art}\\
%%%
\ex	
\gll	\textbf{smart} \textbf{one-s}\\
	smart \textsc{art}-\textsc{pl}\\
\end{xlist}
\end{exe}
%%%
The marker \textit{one} in English (originating from the homophonous numeral\is{adnominal modifier!numeral} \textit{one}) is a prototypical instance of an article: it constitutes a phonologically free grammatical word which is the target of agreement. 

Given that attributive adjectives cannot occur other than syntactically bound to a head noun, the regular noun phrase type in English is best analyzed as incorporation. Note that the article is not an attribution marking device in the proper sense. Even though the marker projects a noun phrase by syntactic nominalization, this noun phrase does not modify a higher noun. The nominalization strategy can only be used in noun phrases with an empty lexical head.
%%%
\begin{exe}
\ex {\rm English (personal knowledge)}
\begin{xlist}
\ex[]{
\gll	{}                                a smart girl\\
	{{\upshape [}\textsubscript{NP}} \textsc{indef} \textsubscript{A}smart \textsubscript{N}girl {\upshape ]}\\
	}
%%%
\ex[]{
\gll	{}                               a smart \textbf{one}\\
	{{\upshape [}\textsubscript{NP}} \textsc{indef} \textsubscript{A}smart \textsubscript{HEAD} {\upshape ]}\\
	}
%%%
\ex[*]{
\gll 	{} a smart \textbf{one} {} girl\\
	{{\upshape [}\textsubscript{NP} {\upshape [}\textsubscript{NP}} \textsc{indef} {\textsubscript{A}smart} \textsubscript{HEAD} {\upshape ]} \textsubscript{N}girl {\upshape ]]}\\
	}
\end{xlist}
\end{exe}
%%%
Because attributive adjectives in English are obligatorily bound to a syntactic head and because the nominalizer (“dummy head”) cannot occur in noun phrases modifying a higher head, English exhibits neither true \isi{juxtaposition} nor \isi{attributive nominalization}.
\il{English|)}
\il{West Germanic languages|)}

\il{North Germanic languages|(}
\subsubsection{North Germanic}
\label{n-germanic synchr}
%%%
With regard to existing attribution marking devices, the North Germanic languages exhibit even a higher degree of diversity than West Germanic. This is especially true if major sub-varieties are considered as well. Practically all types attested in West Germanic occur here as well, including adjective incorporation which is otherwise scarcely attested in the languages of northern Eurasia.

\is{head\hyp{}driven agreement|(}
\paragraph*{Head\hyp{}driven agreement in North Germanic} 
Although head\hyp{}driven agreement marking constitutes the prototypical adjective attribution marking device in North Germanic, the adjective agreement paradigms across the different languages reflect the ongoing decline in differentiated categories.

\il{Icelandic|(}
In \textbf{Icelandic}, adjectives inflect for the agreement features \textsc{gender}, \textsc{number}, \textsc{case} and \textsc{species}.\is{species marking} The adjective agreement paradigm of Modern Icelandic (Table~\ref{icelandic agr} in \S\ref{head-driven agreement}) is thus relatively similar to \ili{Old Icelandic} even though the different case endings are already merged in the definite paradigm.
\il{Icelandic|)}

\il{Danish|(}
In \textbf{Danish},\label{danish synchr} there is no agreement feature \textsc{case}, while \textsc{gender} is marked on the attributive adjective only in indefinite noun phrases. In definite\is{species marking!definite} noun phrases, the attributive adjective is marked with an invariable definite agreement suffix (Table~\ref{danish agr paradigm}).
%%%
\begin{table}
\begin{tabular}{l l l l}
\lsptoprule
		& \textsc{utr.sg}	&\textsc{n.sg}	&\textsc{pl}\\
\midrule
\textsc{indef}	&gul	 	&gul-t		&gul-e\\

\textsc{def}	&gul-e	&gul-e		&gul-e\\
\lspbottomrule
\end{tabular}
\caption[Adjective paradigm for Danish]{Agreement paradigm for the adjective ‘yellow’ in Danish (personal knowledge)}
\label{danish agr paradigm}
\end{table}
%%%

\il{Danish!W-Jutlandic|(}
The \textbf{Western Jutlandic} dialect of Danish is most innovative with regard to the decline of agreement features because it has almost completely lost its agreement features and thus resembles \ili{English} (Table~\ref{jutl agr paradigm}).
%%%
\begin{table}
\begin{tabular}{l l l}
\lsptoprule		& \textsc{sg}	&\textsc{pl}\\
\midrule
\textsc{indef}	& gulʔ	 	&gul\\

\textsc{def}	&gul			&gul\\
\lspbottomrule
\end{tabular}
\caption[Adjective paradigm for W-Jutlandic]{Agreement paradigm for the adjective ‘yellow’ in Western Jutlandic (in phonemic transcription) \citep{ringgaard1960}}
\label{jutl agr paradigm}
\end{table}
%%%
\il{Danish!W-Jutlandic|)}
\il{Danish|)}

\il{Swedish|(}
\paragraph*{Anti\hyp{}construct state + head\hyp{}driven agreement in Swedish}
\label{swedish synchr}
Swedish, \ili{Norwegian},\footnote{The two Norwegian standard languages Dano-Norwegian (Norwegian \textit{bokmål}) and New Norwegian (Norwegian \textit{nynorsk}) do not differ in their marking of adjective attribution and they will simply be referred to as Norwegian} and \ili{Faroese} exhibit two adjective attribution marking morphemes simultaneously: an inflectional suffix expressing the agreement features \textsc{gender}, \textsc{number} and \textsc{species}\is{species marking} (but the indefinite utrum gender form of the adjective is always unmarked) plus an article (which again is not found in the indefinite plural form).

\is{species marking!definite|(}
In the (North-)Germanic and typological linguistic tradition, the definite noun phrases with adjectives have most often been characterized as “double definite” (cf.~\citealt{kotcheva1996a}; \citealt{borjars1994}; \citealt{julien2003}; \citealt[354–355]{plank2003}). This makes sense from a historical perspective because the articles (Swedish \textit{den, det, de}) are cognate with the Old Germanic\il{Old Germanic languages} demonstratives which developed into definite markers (cf.~German\il{German} \textit{der, die, das} or English\il{English} \textit{the}). Synchronically, however, the articles in the North Germanic languages with so-called double definiteness (Swedish, both Norwegian\il{Norwegian} languages, Faroese)\il{Faroese} are not definiteness markers. Unlike in West Germanic,\il{West Germanic languages} definiteness is exclusively expressed by an inflectional suffix (Swedish \textit{-(e)n} \textsc{utr}, \textit{-(e)t} \textsc{n}, \textit{-n} \textsc{pl}.)
%"exclusively expressed"?\\if the NP is indef den/det is not used, so den/det has two functions: expressing def and marking attr

Unlike in West Germanic\il{West Germanic languages} languages, where the definite markers are noun phrase markers always attach at the left edge of the phrase, the presence or absence of the cognate articles \textit{den} \textsc{utr}, \textit{det} \textsc{n}, \textit{de(m)} \textsc{pl} in Swedish is determined by the availability of an adjective and not the referential status of the noun phrase. 
%%%
\begin{exe}
\ex {\rm Swedish (personal knowledge)}
\label{swedish np}
\begin{xlist}
\ex
\gll	{(*det)} hus\textbf{-et}\\
	\textsc{attr:def.n} house-\textsc{def:n}\\
\glt	‘the house’
%%%
\ex	
\gll	{*(\textbf{det})} \textbf{hög-a} hus-et\\
	\textsc{attr:def.n} high-\textsc{def.n} house-\textsc{def.n}\\
\glt	‘the high house’
%%%
\ex 
\label{art0}
\gll	{*(\textbf{det})} \textbf{hög-a}\\
	\textsc{attr:def.n} high-\textsc{def.n}\\
\glt	‘the high one’ (about a house)
\end{xlist}
\end{exe}
%%%
Example (\ref{swedish np}) shows how the article can neither attach to a noun nor can an adjectival modifier in a definite noun phrase occur without being marked by the article.\footnote{The expression \textit{det hus} is grammatical only with the homophonous demonstrative \textit{det}, similarly (but restricted to certain regiolects) \textit{det hus-et}. Even the expressions \textit{höga hus-et} is possible for some expression similar to English \textit{White house}. Note also that possessive pronouns\is{adnominal modifier!pronoun} replace the article: \textit{min hög-a hus} [\textsc{poss:1sg} high-\textsc{def.n} house(\textsc{n}] 'my high house’.} Since the definite value of the feature \textsc{species} is always marked by the respective definite inflectional noun suffixes %in \isi{headless noun phrase}s, the definite inflection is absent 
 and since the article only attaches to adjectives, the latter cannot be analyzed as anything but a morpho-syntactic device, i.e., as an adjective attribution marker.

In definite noun phrases, Swedish thus exhibits a circumfixed adjective attribution marking device combined by head\hyp{}driven agreement inflection plus the article. It is plausible that the article developed from an attributive nominalizer. Its use with adjectives in \isi{headless noun phrase}s, as in \REF{art0} resembles attributive nominalization. There is, however, no evidence that the adjective marked by the article is part of a complex constituent (i.e., a \isi{headless noun phrase}) modifying a noun. According to the definition of \isi{attributive nominalization} presented in \S\ref{attr nmlz} of Part~II (Typology), the article in Swedish is thus not a syntactic nominalizer. Its function is the licensing of the attributive state of the adjective along with marking of head\hyp{}driven agreement. Since head\hyp{}driven agreement is additionally marked by inflectional suffixes, the Swedish noun phrase exhibits circum-positioned (i.e., phonologically free and phonologically bound) agreement marking.
\is{species marking!definite|)}

\is{predicative marking|(}
Note that the circum-positioned agreement marker only occurs with attributive adjectives. Predicative adjectives, on the other hand, exhibit “pure” gender and number agreement (\ref{swed pred}). The analysis of adjective attribution marking in Swedish as belonging to anti\hyp{}construct state agreement marking is thus justified.
%%%
\begin{exe}
\ex {\rm Predicative adjectives in Swedish (personal knowledge)}
\label{swed pred}
\begin{xlist}
\ex[]{
	\textit{kåken är \textbf{hög}}	  	\jambox{{\rm ‘the (bad) house is high’} [\textsc{utr}]}
	}
\ex[*]{
	\textit{kåken är \textbf{en hög / den hög-a}}
	}
\ex[]{
	\textit{huset är \textbf{hög-t}} 		\jambox{{\rm ‘the house is high’} [\textsc{n}]}
	}
\ex[*]{
	\textit{huset är \textbf{ett hög-t / det hög-a}}
	}
\ex[]{
	\textit{husen är \textbf{hög-a}} 		\jambox{{\rm ‘the houses are high’} [\textsc{pl}]}
	}
\ex[*]{
	\textit{husen är \textbf{de hög-a}}
	}
\end{xlist}
\end{exe}
%%%
\begin{table}
\begin{tabular}{l l l l l l l}
\lsptoprule
			&\multicolumn{3}{l}{\textsc{indef}}	&\multicolumn{3}{l}{\textsc{def}}\\
\midrule
\textsc{utr.sg}	&en	&\textbf{h{ö}g-Ø}&stuga		&\textbf{den}&\textbf{h{ö}g-a}&stuga-n\\

\textsc{n.sg}	&ett	&\textbf{h{ö}g-t}&hus		&\textbf{det}&\textbf{h{ö}g-a}&hus-et\\

\textsc{pl}		&	&\textbf{h{ö}g-a}&stug-or	&\textbf{de}&\textbf{h{ö}g-a}&stug-or:na\\
\lspbottomrule
\end{tabular}
\caption[Adjective paradigm for Swedish]{Agreement paradigm for the adjective \textit{hög} ‘high’ in Swedish (personal knowledge); \textit{stuga} (\textsc{utr}) ‘cabin’, \textit{hus} (\textsc{n}) ‘house’
}
\end{table}
\il{Swedish|)}
\is{predicative marking|)}
\is{head\hyp{}driven agreement|)}

\il{Swedish!Västerbotten|(}
\paragraph*{Adjective incorporation in Västerbotten Swedish}
\label{bondska synchr}
The dialect spoken in the Västerbotten province in northern Sweden exhibits adjective incorporation as a regular type of adjective attribution marking.
%%%
\begin{exe}
\ex {\rm Västerbotten Swedish \citep[91–92]{holmberg-etal2003}}
\begin{xlist}
\ex
\gll 	\textbf{grann-}kweinn-a\\	
	pretty-woman-\textsc{def}\\
\glt	‘the pretty woman’
%%%
\ex
\gll	en \textbf{grann-}kweinn\\
	\textsc{indef} pretty-woman\\
\glt	‘a pretty woman’
\end{xlist}
\end{exe}
%%%
Adjective incorporation also occurs in several other northern North Germanic dialects of Sweden, Finland and Norway. Whereas adjective incorporation is the default type in Västerbotten Swedish,\footnote{In indefinite noun phrases, however, adjective incorporation is often restricted to monosyllabic adjective stems: \textit{en \textbf{grann-}kweinn} but *\textit{en \textbf{vacker-}kweinn} ‘a \textbf{pretty} woman’. Furthermore, a certain semantic relation between noun and adjective seem to be obligatory: (incorporation) \textit{n \textbf{ny-}bil} ‘a \textbf{new} car (straight from the factory)’, \textit{n \textbf{ny} bil} ‘a \textbf{new} car (new for me)’, (incorporation) *\textit{n \textbf{ny-}hunn} ‘a \textbf{new} dog’, \textit{n \textbf{ny} hunn} ‘a \textbf{new} dog (new for me)’ \citep[91–92]{holmberg-etal2003}.} its occurrence is restricted to definite noun phrases in most other dialects where this type it attested.\is{species marking!definite}

Attributive adjectives cannot occur in indefinite\is{species marking!indefinite} \isi{headless noun phrase}s in Västerbotten Swedish but are obligatorily bound to an article used as dummy head.
%%%
\begin{exe}
\ex {\rm Västerbotten Swedish \citep{holmberg-etal2003,delsing1996b}}
\begin{xlist}
\ex
\gll 	en stor en\\	
	\textsc{indef:m} big(\textsc{m}) \textsc{art:indef:m.sg}\\
%%%
\ex
\gll 	ett stor-t ett\\	
	\textsc{indef:n} big:\textsc{n} \textsc{art:indef:n.sg}\\
\glt	‘a big one’
\end{xlist}
\end{exe}
\il{Swedish!Västerbotten|)}
\il{North Germanic languages|)}
\il{Germanic languages|)}

\il{Hellenic languages|(}
\il{Greek|(}
\subsection{Hellenic}
\label{greek synchr}
%%%
The Hellenic branch of Indo-European is represented by a single language: Modern Greek. 

\is{attributive nominalization|(}
\is{head\hyp{}driven agreement|(}
\paragraph*{Head\hyp{}driven agreement and attributive nominalization + head\hyp{}driven agreement in Greek}
Attributive adjectives in Greek show agreement in \textsc{gender, number} and \textsc{case}.\footnote{A minor class of loan adjectives in Greek belong to a different noun phrase type, \isi{juxtaposition}, because they do not inflect at all \citep{ruge1986}.}

\is{contrastive focus|(}
The unmarked constituent order in Greek is adjective-noun, as in \REF{greek afocus}. The reverse constituent order (noun-adjective), however, is commonly used as well and marks contrastive focus on the attribute, as in \REF{greek afocus}.

\newpage 
%%%
\begin{exe}
\ex {\rm Greek \citep{ruge1986}}
\begin{xlist}
\ex {\rm Head\hyp{}driven agreement}
\label{greek agr}
\begin{xlist}
\ex
\gll	to			\textbf{kokino} 	aftokinito\\
	\textsc{def:m}	red:\textsc{m}	car(\textsc{m})\\
\glt	‘the red car’
\end{xlist}
%%%
\ex {\rm Attributive nominalization}
\label{greek attr}
\begin{xlist}
\ex {\rm Contrastive focus on the attribute}\\
\label{greek afocus}
\gll	to 			aftokinito		\textbf{to}				\textbf{kokino}\\
	\textsc{def:m}	car(\textsc{m})	\textsc{attr:m}	red:\textsc{m}\\
\glt	‘the \textsc{red} car (not the blue one)’
%%%
\ex {\rm Contrastive focus on the noun}\\
\label{greek nfocus}
\gll	\textbf{to}				\textbf{kokino}		to		aftokinito\\
	\textsc{attr:m}	red:\textsc{m}	\textsc{def:m}	car(\textsc{m})\\
\glt	‘the red \textsc{car} (not the buss)’
\end{xlist}
\end{xlist}
\end{exe}
%%%
Note that the noun can move to the contrastive focus position as well, as in \REF{greek nfocus}.

Example (\ref{greek attr}) illustrates the use of the article \textit{to} in two different syntactic functions: whereas \textit{to} \textsc{def} is a determiner marking the noun phrase as definite,\is{species marking!definite} \textit{to} \textsc{attr} is an attributive marker (i.e., a true article) attaching to the adjective noun phrase internally. Attribution of the adjective (in contrastive focus) in \REF{greek afocus} is marked by means of attributive nominalization. The article marks the adjective as phrasal constituent, i.e., as a syntactic complement to the noun.
\is{contrastive focus|)}
\il{Greek|)}
\il{Hellenic languages|)}
\is{attributive nominalization|)}
\is{head\hyp{}driven agreement|)}

\il{Romance languages|(}
\subsection{Romance}
%%%
All Romance languages exhibit \isi{head\hyp{}driven agreement} marking as the main and default adjective attribution marking device. The prototypical agreement features characteristic of most modern Romance languages are \textsc{number} and \textsc{gender}. A third agreement feature, \textsc{case}, was present in earlier stages of Romance but has disappeared in the modern languages.

Three noun phrase types have existed in the Romance branch from its earliest stages:\footnote{A minor class of adjectives belong to a different noun phrase type, 
\isi{juxtaposition}, because they do not inflect at all.}
%%%

\newpage
\begin{itemize}
\item head\hyp{}driven agreement
	\subitem noun-adjective order
	\subitem adjective-noun order
\item attributive nominalization.\is{attributive nominalization}
\end{itemize}
%%%
\is{contrastive focus|(}
\il{Romanian|(}
The unmarked and prototypical noun phrase type in Romance is \isi{head\hyp{}driven agreement} with the adjective following the noun. Besides the basic head-initial constituent order, most Romance languages exhibit a small subgroup of very common adjectives, such as ‘good–bad, young–old, small–large’, which normally precede the head noun (\citealt[146–147]{posner1996}, cf.~also \citealt[340]{silvestri1998}). However, most other adjectives can also precede the noun in the modern Romance languages. This reversed constituent order is regularly determined by semantics-pragmatics in Romanian and is used to give these adjectives a certain emphasis or contrastive focus, as in the following examples from Romanian (\ref{romanian wo}) and Italian (\ref{italian wo}).
%%%
\begin{exe}
\ex
\begin{xlist} 
\ex {\rm Romanian \citep{beyer-etal1987}}\\
\label{romanian wo} 
\begin{xlist}
\ex	
\gll	băiat=ul \textbf{bun}\\
	boy=\textsc{def} good\\
\glt	‘the good boy’
%%%
\ex	
\gll	\textbf{bun}=ul băiat\\
	good=\textsc{def} boy\\
\glt	‘the \textsc{good} (i.e., different) boy’ 
\end{xlist}
%%%
\il{Italian|(}
\ex {\rm Italian \citep[146]{posner1996}}
\label{italian wo}
\begin{xlist}
\ex	
\gll	un vestito \textbf{nuovo}\\
	\textsc{indef} dress new\\
\glt	‘a (brand-)new dress’
%%%
\ex	
\gll	un \textbf{nuovo} vestito\\
	\textsc{indef} new dress\\
\glt	‘a new (i.e., different) dress’
\end{xlist}
\end{xlist}
\end{exe}
%%%
Note that the definite\is{species marking!definite} marker in Romanian is not connected with attribution marking on adjectives. Even though the marker can occur on the attributive adjective which precedes the noun in contrastive use (\ref{romanian wo}), definiteness is a purely morpho-semantic feature in Romanian and is not assigned by syntax (see also \S\ref{syntax-morphology-interface} of Part~I Preliminaries).
\il{Romanian|)}

The common distinction between an “emphatic” adjective preceding a noun and a “descriptive” adjective following a noun goes probably back to the earliest stages of Romance, although it is first attested in Classical Latin\il{Latin!Classical} \citep[146]{posner1996}.
\is{contrastive focus|)}

\is{head\hyp{}driven agreement|(}
\paragraph*{Head\hyp{}driven agreement in Italian} 
In Italian, as in the other Romance languages, the agreement features \textsc{gender} and \textsc{number} are marked on adjectives and on other modifiers within the noun phrase.
%%%
\begin{exe}
\ex {\rm Italian (personal knowledge)}
\begin{xlist}
\ex
\gll	la casa \textbf{alt-a}\\
	\textsc{def:f} house(\textsc{f}) high-\textsc{f}\\
\glt	‘the high house’
%%%
\ex
\gll	le cas-e \textbf{alt-e}\\
	\textsc{def:pl} house-\textsc{pl} high-\textsc{pl}\\
\glt	‘the high houses’
\end{xlist}
\end{exe}
\il{Italian|)}
\is{head\hyp{}driven agreement|)}

\il{Romanian|(}
\is{attributive nominalization|(}

\paragraph*{Attributive nominalization in Romanian}
\label{romanian synchr}
Beside the default type of \isi{head\hyp{}driven agreement} (with either noun-adjective or adjective-noun constituent order), Standard Romanian (aka Daco-Romanian)\il{Daco-Romanian|see{Romanian}} exhibits attributive nominalization as a differentiated third type of adjective attribution marking. The agreement paradigm of the attributive nominalizer (traditionally labeled “adjective article” in the grammatical descriptions of Romanian) is shown in Table~\ref{romanian art}.
%%%

\begin{table}[b]
\begin{tabular}{r c c c}
\lsptoprule		
			&\textsc{f}				&\textsc{n}		&\textsc{m}\\
\midrule
\textsc{sg}		&cea					&\multicolumn{2}{|c}{cel}\\
\midrule
\textsc{pl}		&\multicolumn{2}{c|}{cele}					&cei\\
\lspbottomrule
\end{tabular}
\caption[Article paradigm for Romanian]{Agreement paradigm of the attributive article in Romanian \citep[94]{beyer-etal1987}.
}
\label{romanian art}
\end{table}

%%%
\is{species marking!definite|(}
The use of the non-obligatory attributive marker emphasizes the adjective following a noun (\citealt[94]{beyer-etal1987}, \citealt[148]{posner1996}). But it is also regularly used to mark definite \isi{headless noun phrase}s, as in the following example.
%%%
\begin{exe}
\ex {\rm Romanian \citep[94]{beyer-etal1987}}\\
\gll Punct-e=le \textbf{cele} \textbf{negr-e} se disting mai bine decât \textbf{cele} \textbf{cenuşi-i}.\\
dot-\textsc{pl}=\textsc{def.m.pl} \textsc{att:m.pl} black-\textsc{pl} \textsc{refl.3sg} distinguish \textsc{compar} well than \textsc{att:m.pl} grey-\textsc{pl}\\
\glt ‘The black dots distinguish themselves better than the grey ones.’
\end{exe}
%%%
The content of this marker, besides licensing of the attributive relation, is not clearly defined in descriptions of Romanian. The article seems to regularly mark definite headless adjectives\is{headless noun phrase} and superlative adjectives. \citet[141]{kramsky1972} compares the function of the article with that of the definite marker and describes the function of the attributive article in Romanian as a “deictic reactualizer” because it has a referential function but can co-occur with the definite marker (\ref{rum def comp}). Note, however, that the definite marker is absent in a noun phrase with reversed constituent order marking \isi{contrastive focus} (\ref{rum def sup}).
\is{species marking!definite|)}
%%%
\begin{exe}
\ex {\rm Romanian \citep[93–94]{beyer-etal1987}}
\begin{xlist}
\ex
\label{rum def comp}
\gll	poet=ul \textbf{cel} \textbf{mai} \textbf{mare}\\
	poet(\textsc{m})=\textsc{def.m} \textsc{att:m.sg} \textsc{super} great\\
\glt	‘the greatest poet’
%%%
\ex
\label{rum def sup}
\gll	\textbf{cel} \textbf{mai} \textbf{mare} poet\\
	\textsc{att:m.sg} \textsc{super} great poet(\textsc{m})\\
\glt	‘the \textsc{greatest} poet’
\end{xlist}
\end{exe}
\il{Romanian|)}
\il{Romance languages|)}
\is{attributive nominalization|)}

\il{Slavic languages|(}
\subsection{Slavic}
\label{slavic synchr}
Slavic (aka Slavonic)\il{Slavonic|see{Slavic languages}} forms a branch inside the Indo-European family. All Slavic languages are spoken in \isi{Europe}, except Russian, which is also spoken in \isi{North Asia}.

The prototypical type of adjective attribution marking is \isi{head\hyp{}driven agreement}. The prototypical agreement features characteristic of Slavic languages are \textsc{number}, \textsc{gender} and \textsc{case}. In the closely related South Slavic languages \ili{Bulgarian} and \ili{Macedonian} however, case inflection of nouns and adjectives has been lost.

Beside \isi{head\hyp{}driven agreement}, anti\hyp{}construct state agreement arose in Slavic languages as a secondary type of adjective attribution marking. The opposition between head\hyp{}driven and anti\hyp{}construct state agreement can be traced back to all Old Slavic\il{Old Slavic languages} languages and already existed in the oldest Slavic manuscripts, the best documented of which are from \ili{Old Bulgarian} (aka Old Church Slavonic).\il{Old Church Slavonic|see{Old Bulgarian}} To a certain extent, this state of development is still reflected in South Slavic.\il{South Slavic languages} In most other modern Slavic languages, however the opposition between the two types was lost by abolishing one or the other type.

Basically, the modern Slavic languages belong to three types and exhibit the following three attribution marking devices:
%%%
\begin{itemize}
\item exclusively \isi{head\hyp{}driven agreement}
\item exclusively anti\hyp{}construct state agreement
\item simultaneously \isi{head\hyp{}driven agreement} and anti\hyp{}construct state agreement
\item attributive nominalization.\is{attributive nominalization}
\end{itemize}
%%%
Constituent order in Slavic can be described as basically adjective-noun, although there is much variation across the single languages. The reversed order of constituents is often possible but in some languages it is restricted to “emphasized” constructions or poetic language. 

\il{West Slavic languages|(}
\is{head\hyp{}driven agreement|(}
\subsubsection{West Slavic}
%%%
All West Slavic languages exhibit head\hyp{}driven agreement as the exclusive type of adjective attribution marking.

\il{Lower Sorbian|(}
\paragraph*{Head\hyp{}driven agreement in Lower Sorbian}
Lower Sorbian exemplifies a Slavic language with head\hyp{}driven agreement as the exclusive type of adjective attribution marking. Attributive adjectives in Lower Sorbian show agreement in gender, number and case. 
%%%
\begin{exe}
\ex {\rm Lower Sorbian \citep{janas1976}}
\begin{xlist}
\ex
\gll	\textbf{dobr-y} cłowjek\\
	good-\textsc{nom.sg.m} person(\textsc{m})\\
\glt	‘good person’
%%%
\ex
\gll	k \textbf{dobr-emu} cłowjek-oju\\
	to good-\textsc{dat.sg.m} person-\textsc{dat:sg.m}\\
\glt	‘to a/the good person’
\ex
\gll	\textbf{dobr-e} cłowjek-y\\
	good-\textsc{nom.pl} person-\textsc{nom:pl}\\
\glt	‘good people’
\end{xlist}
\end{exe}
\il{Lower Sorbian|)}
\il{West Slavic languages|)}
\is{head\hyp{}driven agreement|)}

\il{East Slavic languages|(}
\subsubsection{East Slavic}
\is{predicative marking|(}
All three East Slavic languages Belorussian,\il{Belorussian} Russian\il{Russian} and Ukrainian\il{Ukrainian} exhibit anti\hyp{}construct state agreement marking. There is, however, a tendency to merge attributive (“long”) and predicative (“short”) adjective agreement declension classes, yielding pure \isi{head\hyp{}driven agreement} as in West Slavic.\il{West Slavic languages}

\il{Russian|(}
\paragraph*{Anti\hyp{}construct state agreement in Russian}
\label{russian synchr}
In Russian, attributive as well as predicative adjectives show agreement in \textsc{gender} and \textsc{number}. Attributive adjectives agree additionally in \textsc{case}. The agreement suffixes of the attributive and predicative paradigms, however, have different forms.\footnote{This is true for the stylistically marked “short form adjectives”, see in more detail \S\ref{anti-constr agr}.}
%%%
\begin{exe}
\ex {\rm Russian (personal knowledge)}
\label{ru agr}
\begin{xlist}
\ex {\rm Attribution}
\label{ru attr}
\begin{xlist}
\ex	\textbf{krasiv-yj}			mal'čik\\
	beautiful-\textsc{attr:m.nom} 	boy(\textsc{f})
\glt	‘a handsome boy’
%%%
\ex	
\gll	\textbf{krasiv-ogo}			mal'čik-a\\
	beautiful-\textsc{attr:m.gen}	boy-\textsc{m.gen}\\
\glt	‘of a handsome boy’
%%%
\ex
\gll 	\textbf{krasiv-aja} 			devuška\\
	beautiful-\textsc{attr:f.nom}	girl(\textsc{f})\\
\glt	 ‘a pretty girl’
\end{xlist}
%%%
\ex {\rm Predication}
\begin{xlist}
\ex
\gll 	Etot 			mal'čik		\textbf{krasiv}\\
	\textsc{dem:m} boy(\textsc{m}) 	beautiful:\textsc{m}\\
\glt	 ‘this boy is handsome’
%%%
\ex	
\gll	Eta 			devuška		\textbf{krasiv-a}\\
	\textsc{dem:f} tower(\textsc{f}) 	high-\textsc{f}\\
\glt	‘this girl is pretty’
\end{xlist}
\end{xlist}
\end{exe}
%%%
The agreement suffixes of attributive and predicative adjectives clearly belong to different paradigms (cf.~Table~\ref{Russian adj agr paradigm}). The so-called long agreement suffixes (\ref{ru attr}) mark the values of the morphological agreement features. Simultaneously, they license the (morpho-syntactic) attributive relation inside the noun phrase (cf.~also the discussion in \S\ref{anti-constr agr}). 
%%%
\begin{table}[h]
\begin{tabular}{l c c c c}
\lsptoprule			
			&\textsc{m}	&\textsc{f}		&\textsc{n}	&\textsc{pl}\\
\midrule
\textsc{attr}	&–yj/–ój		&–aja/–ája	&–oje/–óje	&–yje/–\'yje\\
\textsc{pred}	&{Ø}		&–a			&–o			&–y/–i\\
\lspbottomrule
\end{tabular}
\caption[Adjective paradigm for Russian]{Attributive and predicative adjective declension in Russian (personal knowledge) for nominative case}
\label{Russian adj agr paradigm}
\end{table}
\is{predicative marking|)}
\il{Russian|)}
\il{East Slavic languages|)}

\il{South Slavic languages|(}
\is{head\hyp{}driven agreement|(}
\subsubsection{South Slavic}
\label{s-slavic synchr}
%%%
All South Slavic languages exhibit head\hyp{}driven agreement marking as the default type of adjective attribution marking. In Serbo-Croatian\il{Serbo-Croatian} (aka Bosnian-Croatian-Montenegrin-Serbian\il{Bosnian-Croatian-Montenegrin-Serbian|see{Serbo-Croatian}}) and Slovenian,\il{Slovenian} anti\hyp{}construct state agreement marking occurs as a secondary type. Even \isi{attributive nominalization} is attested in Slovenian.

\il{Bulgarian|(}
\paragraph*{Head\hyp{}driven agreement in Bulgarian}
Attributive adjectives in Bulgarian show agreement in the features \textsc{gender} and \textsc{number}.
%%%
\begin{exe}
\ex {\rm Bulgarian (personal knowledge)}\footnote{The stem allomorph with inserted -\textit{ă}- in \textsc{m.sg} is the result of a phonological process. The stem allomorph with the extension -\textit{ij}- is morpho-phonological and triggered by the definite marker. Note that -\textit{ij}- is a reflex of the \ili{Old Bulgarian} anti\hyp{}construct state agreement marker.}
%%%
\begin{xlist}
\ex {\rm Indefinite noun phrase}
\begin{xlist}
\ex
\gll	\textbf{dobăr} i \textbf{vesel} măž\\
	good:\textsc{m} and cheerful.\textsc{m} man(\textsc{m})\\
\glt	‘good and cheerful man’
%%%
\ex
\gll	\textbf{dobr-a} i \textbf{vesel-a} žena\\
	good-\textsc{f} and cheerful-\textsc{f} woman(\textsc{f})\\
\glt	‘good and cheerful woman’
%%%
\ex
\gll	\textbf{dobr-i} i \textbf{vesel-i} žen-i\\
	good-\textsc{pl} and cheerful-\textsc{pl} woman-\textsc{f.pl}\\
\glt	‘good and cheerful women’
\end{xlist}
%%%
\ex {\rm Definite noun phrase}\is{species marking!definite}
\begin{xlist}
\ex
\gll	\textbf{dobr-ij}=ăt i \textbf{vesel-ij}=ăt măž\\
	good:\textsc{m}=\textsc{def.m} and cheerful:\textsc{m}=\textsc{def.m} man(\textsc{m})\\
\glt	‘the good and cheerful man’
%%%
\ex
\gll	\textbf{dobr-a}=ta i \textbf{vesel-a}=ta žena\\
	good-\textsc{f}=\textsc{def.f} and cheerful-\textsc{f}=\textsc{def.f} woman(\textsc{f})\\
\glt	‘the good and cheerful woman’
%%%
\ex
\gll	\textbf{dobr-i}=te i \textbf{vesel-i}=te žen-i\\
	good-\textsc{pl}=\textsc{def.pl} and cheerful-\textsc{pl}=\textsc{def.pl} woman-\textsc{pl}\\
\glt	‘the good and cheerful women’
\end{xlist}
\end{xlist}
\end{exe}
\il{Bulgarian|)}
\is{head\hyp{}driven agreement|)}

\il{Serbo-Croatian|(}
\paragraph*{Anti\hyp{}construct state agreement in Serbo-Croatian}
\label{serbian synchr}
Serbian (similar to the other varieties of Serbo-Croatian) exemplifies a Slavic language which exhibits both \isi{head\hyp{}driven agreement} and anti\hyp{}construct state agreement in different functions. \isi{head\hyp{}driven agreement} constitutes the basic type of adjective attribution marking in Serbian. Most adjectives, however, have “double forms” \citep[179–180]{kramsky1972}. Consider the following example.
%%%
\begin{exe}
\ex {\rm Serbian \citep[59]{zlatic1997}}
\begin{xlist}
\ex {\rm Indefinite noun phrase (“pure” \isi{head\hyp{}driven agreement})}\\
\gll	\textbf{dobar}, \textbf{veseo} čovek\\
	good:\textsc{m} cheerful:\textsc{m} person(\textsc{m})\\
\glt	‘a good, cheerful person’
%%%
\ex {\rm Definite noun phrase (anti\hyp{}construct state agreement)}\\
\gll	\textbf{dobr-i}, \textbf{vesel-i} čovek\\
	good-\textsc{attr:m} cheerful-\textsc{attr:m} man(\textsc{m})\\
\glt	‘the \textsc{good}, \textsc{cheerful} person’
\end{xlist}
\end{exe}
%%%
\is{species marking!definite|(}
\is{species marking!indefinite|(}
Anti\hyp{}construct state agreement marking (“long form agreement”) in Serbo\hyp{}Croatian is sometimes described as a definite marker on the adjective (e.g., by \citealt[18–19]{kordic1997}). However, the short-form adjective can also be used in a noun phrase marked as definite, for instance by a demonstrative pronoun (\ref{serbian short-def}). And the “long form” adjective can also be used in a noun phrase marked as indefinite, for instance by the indefinite article (\ref{serbian indef}). 
%%%
\begin{exe}
\ex {\rm Serbian \citep{marusic-etal2007}}
\begin{xlist}
\ex {\rm Definite noun phrase with “pure” \isi{head\hyp{}driven agreement}}\\
\label{serbian short-def} 
\gll	ovaj \textbf{dobar}, \textbf{veseo} \v{c}ovek\\
	\textsc{dem:m} good:\textsc{m} cheerful:\textsc{m} person(\textsc{m})\\
\glt	‘this good, cheerful man’
%%%
\ex {\rm Indefinite noun phrase with anti\hyp{}construct state agreement}\\
\label{serbian indef}
\gll	Treba mi \textbf{jedan} \textbf{crven-i} kaput.\\
	need.\textsc{3sg} \textsc{1sg.dat} \textsc{indef:m} red-\textsc{attr:m} coat(\textsc{m})\\
\glt (in a store with red coats on display)\\‘I need a \textsc{red} coat (viz.~one of those red coats).’
\end{xlist}
\end{exe}
%%%
\is{species marking!indefinite|)}
The examples with “short form” adjectives in definite contexts and “long form” adjectives in indefinite contexts provides the best evidence against the analyses of the two different adjective agreement suffixes as markers of the category \textsc{species} of the head noun. 

Rather than as a definite marker, the long-form adjective agreement suffixes in Serbian are best analyzed as anti\hyp{}construct state agreement markers used in special contrastive focus constructions.\footnote{Note even that school grammars of Serbian sometimes explain the rules for the use of the two adjective declensions with the help of the the questions “what sort?” (requires the “short form”) and “which one?” (requires the “long form”) \citep[327]{browne1993}.}
\is{species marking!definite|)}
\il{Serbo-Croatian|)}

\il{Slovenian|(}
\paragraph*{Anti\hyp{}construct state agreement in Slovenian}
\label{slovenian synchr}
In theory, Slovenian (aka Slovene)\il{Slovene|see{Slovenian}} is identical to \ili{Serbo-Croatian} in exhibiting \isi{head\hyp{}driven agreement} marking and anti\hyp{}construct state agreement marking as two separate devices for adjective attribution.
%%%
\is{contrastive focus|(}
\begin{exe}
\ex {\rm Slovenian \citep[410]{priestly1993}}
\label{slov longshort}
\begin{xlist}
\ex {\rm “Short form” adjective (\isi{head\hyp{}driven agreement})}
\begin{xlist}
\ex
\gll 	\textbf{nȍv} pə̏s\\
	new:\textsc{nom.m.sg} dog\textsc{(m)}\\
\glt	‘new dog’
%%%
\ex	
\gll	en \textbf{nȍv} pə̏s\\
	\textsc{indef:m.sg} new:\textsc{nom.m.sg} dog\textsc{(m)}\\
\glt	‘a new dog’
\end{xlist}
%%%
\ex {\rm “Long form” adjective (anti\hyp{}construct state agreement)}
\begin{xlist}
\ex	
\gll	\textbf{nóvi} pə̏s\\
	new:\textsc{attr:nom.m.sg} dog\textsc{(m)}\\
\glt	‘\textsc{new} dog’
%%%
\ex
\gll	{ta} \textbf{nóvi} pə̏s\\
	{\textsc{attr}} new:\textsc{attr:nom.m.sg} dog\textsc{(m)}\\
\glt	 ‘the \textsc{new} dog’
\end{xlist}
\end{xlist}
\end{exe}
%According to its morphological type, the non-concatenative anti\hyp{}construct state agreement marking device in Slovenian seems to be exceptional among northern Eurasian languages because it constitutes a tonal distinction between short high tone (\textit{nȍv}) and long low tone \textit{nóvi}. %\footnote{The historical explanation, however, is straightforward and looks much less exotic...} 
Note, however, that the use of morphologically differentiated adjectives for \isi{head\hyp{}driven agreement} versus anti\hyp{}construct state agreement in Slovenian is very restricted and is found more or less only with masculine adjectives in nominative singular \citep[410–411]{priestly1993}.

\is{species marking!definite|(}
Similar to Serbo-Croatian,\il{Serbo-Croatian} anti\hyp{}construct state agreement marking in Slovenian is sometimes described as a definite marker on the adjective (e.g., by \citealt[411]{priestly1993}). Semantic definiteness in Slovenian, however, is not marked obligatorily (cf.~example \ref{slov longshort}). Furthermore, the analysis of the anti\hyp{}construct state agreement as a definite marker can be rejected completely because examples are found in which this marker also occurs in overtly marked indefinite noun phrases.
%%%
\ea
{\rm Slovenian \citep{marusic-etal2007}}\\
\gll 	rabi mi \textbf{en} \textbf{rde\v{c}i} pla\v{s}\v{c}\\
	need.\textsc{3sg} \textsc{1sg.dat} \textsc{indef:m} red:\textsc{attr:m} coat(\textsc{m})\\
\glt (in a store with red coats on display)\\‘I need a \textsc{red} coat (viz.~one of those red coats).’\footnote{Cf.~the similar construction with concatenative anti\hyp{}construct state agreement marking in Serbian\il{Serbo-Croatian!Serbian} in \REF{serbian indef}.}
\z
%%%
Anti\hyp{}construct agreement marking are thus analyzed as attribution marking device with the additional content of contrastive focus rather than as a detached definite marker.
\is{species marking!definite|)}

\is{attributive nominalization|(}
\is{head\hyp{}driven agreement|(}
\paragraph*{Attributive nominalization + head\hyp{}driven agreement in Slovenian}
Besides head\hyp{}driven agreement and anti\hyp{}construct state agreement, adjectives in (colloquial) Slovenian can also be marked by means of an attributive article.
%%%
\begin{exe}
\ex {\rm Slovenian \citep{marusic-etal2007}}
\label{slovenian art}
\begin{xlist}
\ex {\rm Indefinite noun phrase}\\
\gll	Lihkar je mim prdirkal en \textbf{ta} \textbf{hiter} avto.\\
	just\_now \textsc{aux} by sped \textsc{indef:n} \textsc{attr} fast:n car(\textsc{n})\\
\glt	‘Some \textsc{fast} car has just sped by (viz.~one of the fast type of cars has just sped by).’ 
%%%
\ex {\rm Definite noun phrase}\\
\label{slovenian def}
\gll 	ta \textbf{ta} \textbf{zelen} \textbf{ta} \textbf{debel} svin\v{c}nik\\
	\textsc{dem} \textsc{attr} green\textsc{:m} \textsc{attr} thick\textsc{:m} pencil\\
\glt 	‘this \textsc{green}, \textsc{thick} pencil’
\end{xlist}
\end{exe}
%%%
The attributive article \textit{ta} in Slovenian is homophonous with the demonstrative determiner (from which it originates historically), but \REF{slovenian def} with the double use of \textit{ta} on stacked adjectives and after the determiner clearly shows that these markers serve two different functions: whereas \textit{ta} \textsc{dem} is a determiner marking the noun phrase for special local deictic species \textit{ta} \textsc{attr} is an attributive marker (i.e., a true article) attaching to the adnominal adjective. Attribution of the adjective in contrastive focus in \REF{slovenian art} is marked by means of attributive nominalization (in combination with head\hyp{}driven agreement).

According to \cite{marusic-etal2007,marusic-etal2007b}, the article \textit{ta} gives the adjective a classifying reading and the construction \textit{ta}+A:\textsc{attr} can be compared to a “reduced relative clause”,\is{adnominal modifier!relative clause} hence a syntactic complement to the noun.
\is{contrastive focus|)}
\il{Slovenian|)}
\il{South Slavic languages|)}
\il{Slavic languages|)}
\il{Indo-European languages|)}
\is{attributive nominalization|)}
\is{head\hyp{}driven agreement|)}

\il{Basque|(}
\section{Basque}
%%%
Basque is a language isolate spoken in the Basque country in northeastern Spain and in adjacent parts of France in southwestern \isi{Europe}. 

\is{juxtaposition|(}
\paragraph*{Juxtaposition in Basque}
Attributive adjectives are juxtaposed to the right of the noun they modify.
%%%
\ea
\label{basque juxtap}
{\rm Basque \citep[81]{saltarelli1988}}\\
\gll	gona \textbf{gorri} \textbf{estu}-ak\\
	skirt red tight-\textsc{def.pl.abs}\\
\glt	‘the tight red skirts’
\z
%%%
Note that the features \textsc{species},\is{species marking} \textsc{number}, and \textsc{case} in \REF{basque juxtap} are not assigned to the adjective through agreement. The corresponding portmanteau suffixes marking the values of these morphological features always attach to right edge of the phrase in Basque. Consequently, they always attach to the attributive adjective if one is present \citep[171]{hualde-etal2003}
\is{juxtaposition|)}
\il{Basque|)}
