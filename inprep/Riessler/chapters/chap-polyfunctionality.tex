%\documentclass{LSP/langsci}
%
%\usepackage{localmetadata}
%\usepackage{localpackages}
%\usepackage{localhyphenation}
%\usepackage{localcommands}
%\usepackage{authorindex}
%\bibliography{localbibliography}

%\begin{document}

\chapter[Polyfunctionality]{Excursus: Polyfunctionality of attribution marking devices} \label{polyfunctionality}
%%%
In a typological survey, noun phrases with adjectival modifiers can be examined from different perspectives. In the previous chapter, noun phrases with attributive adjectives were described according to their syntactic, morpho-syntactic, and/or morpho-semantico-syntactic structure. But noun phrase types of a given language can also be defined on a polyfunctionality scale with regard to the class of attributed elements beyond adjective attribution: Attributive adjectives and other adnominal modifiers (such as demonstratives, possessor nouns, adpositional phrases, clauses, etc.) may or may not be used in similar noun phrase structures.

Moreover, polyfunctionality is also relevant in languages where one and the same device is used as a nominal modification marker beyond attribution: for modification inside an adjective phrase (licensing, for instance, a degree word as modifier of an adjective) or as a modification marker inside an adpositional phrase (licensing, for instance, an adposition as determined by a noun phrase). \textit{Attribution marker} should thus be understood as a term denoting a subset of \textit{modification markers} relevant to nominal phrase structure in general.

Finally, the polyfunctionality concerns even the semantic content (or function) of certain devices beyond modification marking. 

In the present chapter, polyfunctionality of adjective attribution marking devices will be illustrated with examples from a few languages.

\section{Polyfunctionality of modification markers}
%%%
\todo{check zu possessor compounding Dahl 2004. The growth and maintenance of linguistic complexity. Studies in language companion series, 71. Amsterdam: John Benjamins.}
In many languages, more than one class of attributes belong to one and the same noun phrase type. Some languages exhibit even highly polyfunctional noun phrase types and use one and the same device for licensing verbs, nouns, adjectives and even other syntactic classes as attributive modifiers inside noun phrases.

In example (\ref{multi mand}) from Mandarin Chinese, the anti-construct state marker \textit{de} illustrates a highly polyfunctional attribution marking device. It licenses adjectival (\ref{mandarin adj}), nominal (\ref{mandarin noun}) and verbal attributes (\ref{mandarin rel}).\footnote{Note, however, that the attributive marker is not obligatory. In noun phrases with pronominal and adjectival attributes, it can also be omitted. If \textit{de} is used with adjectives, a certain clarifying or delineating stress is put on the denoted property, e.g.~\textit{hóng hūa} [red flower] ‘a red flower’, \textit{hóng de hūa} [RED \textsc{attr} flower] ‘a flower that is red (and not of a different color)’ \citep[119–123]{li-etal1981}.}
%%%
\begin{exe}
\ex
\langinfo{Mandarin Chinese}{Sinotibetan}{\citealt{li-etal1981}} \label{multi mand}
\begin{xlist}
\ex	{\rm Noun (possessor) attribute}\\
\gll	Zhāngsān 	\textbf{de} 	shū\\
	Zhangsang 	{\textsc{attr}} 	book\\
\glt	‘Zhangsang's book’\label{mandarin noun}
\ex	{\rm Adjectival attribute}\\
\gll	xīn 		(\textbf{de}) 	shū\\
	new	 	({\textsc{attr}}) 	book\\
\glt	‘new book’\label{mandarin adj}
\ex	{\rm Verbal (relative clause) attribute}\\
\gll	wŏ zuótiān 	măi 	\textbf{de} 	shū\\
	1\textsc{1sg} 	buy	yesterday 	{\textsc{attr}} 	book\\
\glt	‘the book I bought yesterday’\label{mandarin rel}
\end{xlist}
\end{exe}
%%%
In Minangkabau, an Austronesian language spoken on Sumatra in Indonesia, juxtaposition is polyfunctional to a similar degree.
%%%
\begin{exe}
\ex 
\langinfo{Minangkabau}{Austronesian}{\citealt[3–4]{gil2005}} \label{multi minangkabau}
\begin{xlist}
\ex {\rm Noun (possessor) attribute}\\
\gll	batiak Kairil\\
	papaya Kairil\\
\glt	‘Kairil's papaya’
\ex {\rm Adjectival attribute}\\
\gll	batiak kuniang\\
	papaya yellow\\
\glt	‘a/the yellow papaya’
\ex {\rm Verbal (relative clause) attribute}\\
\gll	batiak Kairil bali\\
	papaya Kairil buy\\
\glt	‘a/the papaya that Kairil bought’
\end{xlist}
\end{exe}
%%%
Tagalog is another language with a polyfunctional attribution marker. The Tagalog linker, however, is less polyfunctional than juxtaposition in Minangkabau or anti-construct state marking in Mandarin Chinese. It marks only verbal and adjectival attributes.\footnote{Note that that the constituent order of attribute and head noun is free in Tagalog: The relative clause and the adjective can also occur in a head-initial phrase type. In this case, the linker \textit{=ng} attaches phonologically to the noun (\citealt[1]{gil2005}; \citealt[160, 162]{himmelmann1997}).}
%%%
\begin{exe}
\ex 
\langinfo{Tagalog}{Austronesian}{\citealt[6]{gil2005}} \label{multi tagalog}
\begin{xlist}
\ex {\rm Adjectival attribute}\\
\gll	pula\textbf{=ng} mangga\\
	red{=\textsc{attr}} mango\\
\glt	‘red mango’
\ex {\rm Verbal (relative clause) attribute}\\
\gll	binili ni Jojo\textbf{=ng} mangga\\
	bought \textsc{pers.gen} Jojo{=\textsc{attr}} mango\\
\glt	‘mango that Jojo bought’
\end{xlist}
\end{exe}
%%%
Highly polyfunctional attribution marking by means of a head-marking construct suffix is found even in Persian.\footnote{Note the consistent glossing \textsc{mod} instead of \textsc{attr}. The Persian construct marker licenses modification beyond attribution.}
%%%
\begin{exe}
\ex 
\langinfo{Persian}{Indoeuropean}{\citealt{mahootian1997}}\label{multi persian}
\begin{xlist}
\ex {\rm Modification inside an adpositional phrase}\\
\gll	tu\textbf{-ye} ašpæzxune\\
	in\textbf{-\textsc{mod}} kitchen\\
\glt ‘in the kitchen’
\ex {\rm Nominal attribution}
\begin{xlist}
\ex {\rm Noun (non-possessor) attribute}\\
\gll 	ængoštær\textbf{-e} ælmas\\
	ring-\textsc{mod} diamond\\
\glt 	‘diamond ring’
\ex {\rm Noun (possessor) attribute}\\
\gll	ængoštær\textbf{-e} pedær\\
	ring-\textsc{mod} father\\
\glt	‘father's ring’
\end{xlist}
\ex {\rm Adjectival attribute}\\
\gll	ælmas\textbf{-e} bozorg\\
	diamond-\textsc{mod} big\\
\glt	‘a big diamond’
\ex {\rm Adpositional attribute}\\
\gll	miz\textbf{-e} tu{-ye} ašpæzxune\\
	table-\textsc{mod} in{-\textsc{mod}} kitchen\\
\glt ‘the table in the kitchen’
\ex {\rm (Infinite) verbal attribute}\\
\gll	væqt\textbf{-e} ræftæn\\
	time-\textsc{mod} to\_go\\
\glt	‘time to go’
\end{xlist}
\end{exe}
%%%
While nominal, adjectival, adpositional and (infinite) verbal attributes are marked by the same device, finite verbal attributes (relative clauses) never occur in a similar noun phrase type in Persian.

In Västerbotten-Swedish, a language of the northern Eurasian area under investigation, attribution marking by means of adjective incorporation is also considered to be polyfunctional (cf.~Sections \ref{attr incorporation} and \ref{bondska synchr}). Beside adjective attribution, the device marks attribution of (human) possessors.
%%%
\begin{exe}
\ex 
\langinfo{Västerbotten-Swedish}{Indoeuropean}{\citealt[5]{gil2005}} \label{multi bondska}
\begin{xlist}
\ex {\rm Noun (human possessor) attribute}\\
\gll	Pelle-äpple\\
	Pelle-apple\\
\glt	‘Pelle's apple’
\ex {\rm Adjectival attribute}\\
%\begin{xlist}
%\ex (lexical adjective)
\gll	rö-äpple\\
	red-apple\\
\glt	‘red apple’
%\ex (past participle)
%\gll	(stick-ad)-tröja\\
%	(knitt-\textsc{ptcp:pst})-sweater\\
%\glt	‘knitted sweater’
%\end{xlist}
\end{xlist}
\end{exe}
%%%
\todo{bessere Referenz zu VB-Schwedisch??}
Gil (\citeyear{gil2005}) surveyed the polyfunctionality of attribution markers licensing possessor nouns, adjectives and relative clauses in a world-wide sample of languages. According to the number of morpho-syntactically differentiated classes of attributes Gil grouped the languages of his sample into the following types:
%%%
\begin{itemize}
\item \textbf{Weakly differentiating languages} using polyfunctional devices for attribution of all three syntactic categories, as in Mandarin (\ref{multi mand}) and Minangkabau (\ref{multi minangkabau})
\item \textbf{Moderately differentiating languages} using polyfunctional devices for attribution of two syntactic categories, for instance:
	\begin{itemize}
	\item adjectives and relative clauses, as in Tagalog (\ref{multi tagalog})
	\item possessor nouns and adjectives, as in Västerbotten-Swedish (\ref{multi bondska}) and Persian (\ref{multi persian})
	\end{itemize}
\item \textbf{Highly differentiating languages} are not polyfunctional at all, as in German where the three syntactic classes are marked differently.
\end{itemize}
%%%
In Gil's sample, Europe and adjacent parts of Asia and Africa stand out as an area with predominantly non-polyfunctional languages, while almost all languages of Southeast Asia are of low differentiation \citep[8]{gil2005}.

Northern Eurasian languages of the “moderately differentiating” type included in Gil's sample are Japanese and Västerbotten-Swedish (with polyfunctional attribution marking of possessor nouns and adjectives) as well as Ainu, Nivkh and Tatar (with polyfunctional attribution marking of adjectives and relative clauses).\footnote{Note that English is not coded as “moderately differentiating” by \citet{gil2005}, although juxtaposition can be polyfunctionally used as a device for attribution of adjectives and relative clauses (with reverse constituent order though: \textit{The woman I saw.})} No languages of the “weakly differentiated” type are known to occur in the northern Eurasian area. 
%%%
\begin{figure} \label{multi abcd}
\parbox[b]{0.20\textwidth}{
\begin{center}{\sc Mandarin},\\{\sc Minangkabau}\\
\medskip
\begin{tabular}{| l |}
\hline
\\
\hline
\hline
\\
\hline
{\sc attr}$_{Rel}$\\
\hline
{\sc attr}$_{A}$\\
\hline
{\sc attr}$_{N}$\\
\hline
\end{tabular}
\end{center}
}
\parbox[b]{0.20\textwidth}{
\begin{center}{\sc Tagalog}\\
\bigskip
\begin{tabular}{| l |}
\hline
\\
\hline
\hline
\\
\hline
{\sc attr}$_{Rel}$\\
\hline
{\sc attr}$_{A}$\\
\hline
\\
\hline
\end{tabular}
\end{center}
}
\parbox[b]{0.20\textwidth}{
\begin{center}{\sc Västerbotten-}\\{\sc Swedish}\\
\medskip
\begin{tabular}{| l |}
\hline
\\
\hline
\hline
\\
\hline
\\
\hline
{\sc attr}$_{A}$\\
\hline
{\sc attr}$_{N}$\\
\hline
\end{tabular}
\end{center}
}
\parbox[b]{0.20\textwidth}{
\begin{center}{\sc Persian}\\
\bigskip
\begin{tabular}{| l |}
\hline
{\sc mod}$_{NP}$\\
\hline
\hline
{\sc attr}$_{AdP}$\\
\hline
\\
\hline
{\sc attr}$_{A}$\\
\hline
{\sc attr}$_{N}$\\
\hline
\end{tabular}
\end{center}
}
\caption[Functional map for modification marking]{Functional maps for modification markers: the anti-construct state marking in \textsc{Mandarin Chinese} and juxtaposition in \textsc{Minangkabau}, the linker in \textsc{Tagalog}, adjective incorporation in \textsc{Västerbotten-Swedish} and construct state marking in {\sc Persian}}
\end{figure}
%%%
\figref{multi abcd} illustrates the polyfunctionality of modification markers in the languages mentioned in this chapter.\footnote{Cf.~\cite{haspelmath2003}, for a systematic and historiographic description of functional (or semantic) maps.} The true attributive functions of the marker, i.e.~licensing of adpositional, verbal, and adjectival attributes, are found in the middle cells of the left column in \figref{multi abcd}. The cell extending upwards shows the additional function of the marker as licensee of modification above the noun phrase level (i.e.~inside an adposition phrase).% or below the noun phrase level (i.e.~inside an adjective phrase).

The order of {\sc attr}$_{Rel}$ through {\sc attr}$_{N}$ in these functional maps corresponds to the hierarchical alignment of polyfunctional attribution marking suggested by Bingfu Lu and Zhenglin Qu.\footnote{Lu's and Qu's hierarchy, cited from a LingTyp posting (“The alignment of modification coding”, LingTyp Item \#2580, 6 May 2009, 01:36, \url{http://listserv.linguistlist.org/cgi-bin/wa?A2=ind0905A&L=LINGTYP&P=R146}) is based on a similar hierarchy for Austronesian languages by Foley (\citeyear{foley1980}). Note that Foley's hierarchy is proposed to be cross-linguistically valid and even includes two more syntactic classes than considered here: Determiner > Numeral > Noun > Adjective > Verb.}
\todo{gibt es eine Publikation inzwischen?}
%%%
\begin{exe}
\ex	Noun < Adjective < Verb
\end{exe}
%%%
The hierarchy is to be read as follows: The highest category of attributive modifiers are verbs (i.e.~relative and other attributive clauses), the next lower categories are adjectives and nouns. If one attributive category is marked with a polyfunctional attribution marker, the less bounded category adjacent to the left side in the hierarchy should be marked by the same device, too.
%%%
\begin{figure}
\parbox[b]{\textwidth}{
\begin{center}
\begin{tabular}{| l || c | c |}
\cline{1-1}
\\
\hline
{\sc attr}$_{Rel}$ & {\sc nmlz} & {\sc foc}\\
\hline
{\sc attr}$_{A}$\\
\cline{1-1}
{\sc attr}$_{N}$\\
\cline{1-1}
\end{tabular}
\end{center}
}
\caption[Functional map for modification marking]{Functional map for the modification marker \textit{ve} in \textsc{Lahu}}
\label{lahu funcmap}
\end{figure}

\section[Polyfunctionality and additional content]{Polyfunctionality of modification markers and additional content}
%%%
Polyfunctional modification marking devices with semantic content (or function) beyond attribution are also attested in several languages. Lahu is an example of a Southeast Asian language of the “weakly differentiating” type according to Gil's (\citeyear{gil2005}) classification. Syntactically similar to Mandarin Chinese, Lahu exhibits an anti-construct state marker \textit{ve} that licenses adjectival (\ref{lahu adj}), nominal (\ref{lahu noun}) and verbal attributes (\ref{lahu rel}). In addition, the marker \textit{ve} in Lahu is used as nominalizer (\ref{lahu compl}) and as focus marker (\ref{lahu focus}).\footnote{See \citealt{bickel1999} on the “Standard Sino-Tibetan Nominalization pattern” which in some languages include even additional content beyond attribution, nominalization and focus.}
%%%
\begin{exe}
\ex
\langinfo{Lahu}{Sinotibetan}{\citealt{matisoff1973}}
\begin{xlist}
\ex	{\rm Attribution}
\begin{xlist}
\ex	{\rm Adjectival attribute}\\
\gll	dàʔ	\textbf{ve}	ŋâʔ\\
	pretty	\textsc{attr}	bird\\
\glt	‘pretty birds’ (194)\label{lahu adj}
\ex	{\rm Noun (possessor) attribute}\\
\gll	Càl\^{ɔ}	\textbf{ve}	\`{ɔ}ha\\
	Jalaw	\textsc{attr}	picture\\
\glt	‘Jalaw's picture’ (141)\label{lahu noun}
\ex	{\rm Verbal (relative clause) attribute}\\
\gll	c\'{ɔ}	câ	\textbf{ve}	ŋâʔ\\
	boil	eat	\textsc{attr}	bird\\
\glt	‘birds one boils to eat’ (194)\label{lahu rel}
\end{xlist}
\ex	{\rm Additional semantic content}
\begin{xlist}
\ex {\rm Nominalization (of a complement clause)}\\
\gll	n\`{ɔ}	qôʔ \textbf{ve}	thàʔ	ŋà mâ	na ɣa	qôʔ-ma!\\
	you	say \textsc{nmlz}	\textsc{acc} I	\textsc{neg} understand	be\_able	\textsc{interj}\\
\glt	‘I can't catch what you're saying!’ (157)\label{lahu compl}
\ex	{\rm Focusing (of a clause)}\\
\gll	mâ		qay	\textbf{ve}\\
	\textsc{neg}	go	\textsc{foc}\\
\glt	‘I am certainly not going.’ (362)\label{lahu focus}
\end{xlist}
\end{xlist}
\end{exe}
%%%
The functions of the marker \textit{ve} in Lahu can also be summarized in a functional map, see \figref{lahu funcmap}. The true attributive functions of the marker, i.e.~licensing of verbal, nominal and adjectival attributes, are found in the cells of the left column in \figref{lahu funcmap}. The cells extending to the right show the additional content of the attributive marker, i.e.~as a nominalizer and focus marker of a clause.

\section{Conclusion}
%%%
From a purely synchronic point of view, polyfunctionality of adjective attribution marking devices seems less relevant to the area under investigation, northern Eurasia. Most languages of the area exhibit highly differentiated attribution marking devices. Languages of the “moderately differentiating” type are rare; no languages of the “weakly differentiated” type are known to occur in the northern Eurasian area at all.

However, polyfunctionality can perhaps indicate historical change if additional semantic content of attribution marking devices across related languages is taken into consideration. The topic of polyfunctional attribution markers across languages of one family will thus be taken up again in Part~\ref{part synchr} of this study.
