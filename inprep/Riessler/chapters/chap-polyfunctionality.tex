
\chapter[Polyfunctionality]{Excursus: Polyfunctionality of attribution marking devices}
\label{polyfunctionality}
%%%
\is{modification marking|(}
In a typological survey, noun phrases with adjectival modifiers can be examined from different perspectives. In the previous chapter, noun phrases with attributive adjectives were described according to their syntactic, morpho-syntactic, and/or morpho-semantico-syntactic structure. But noun phrase types of a given language can also be defined with respect to polyfunctionality and regarding to the class of attributed elements beyond adjective attribution: attributive adjectives may or may not be used in similar noun phrase structures like other adnominal modifiers (such as demonstratives,\is{adnominal modifier!demonstrative} adposition phrases,\is{adnominal modifier!adposition phrase} clauses,\is{adnominal modifier!relative clause} etc.).

Moreover, polyfunctionality is also relevant in languages where one and the same device is used as a nominal modification marker beyond attribution: for modification inside an adjective phrase (licensing, for instance, a degree word as modifier of an adjective) or as a modification marker inside an adposition phrase (licensing, for instance, an adposition as determined by a noun phrase). \textit{Attribution marker} should thus be understood as a term denoting a subset of \textit{modification markers} relevant to nominal phrase structure in general.

Finally, the polyfunctionality concerns even the semantic content (or function) of certain devices beyond modification marking. 

In the present chapter, polyfunctionality of adjective attribution marking devices will be illustrated with examples from a few languages.

\section{Polyfunctionality of modification markers}
%%%
%perhaps more on possessor compounding in Dahl 2004. The growth and maintenance of linguistic complexity. Studies in language companion series, 71. Amsterdam: John Benjamins.
In many languages, more than one class of attributes belong to one and the same noun phrase type. Some languages exhibit even highly polyfunctional noun phrase types and use one and the same device for licensing verbs, nouns, adjectives and even other syntactic classes as attributive modifiers inside noun phrases.
\newpage 

In example (\ref{multi mand}) from Mandarin Chinese,\il{Mandarin Chinese} the anti\hyp{}construct state marker \textit{de} illustrates a highly polyfunctional attribution marking device. It licenses adjectival (\ref{mandarin adj}), nominal (\ref{mandarin noun}) and verbal attributes (\ref{mandarin rel}).\footnote{Note, however, that the attributive marker is not always obligatory. In noun phrases with pronominal and adjectival attributes, it can also be omitted. If \textit{de} is used with adjectives, a certain clarifying or delineating focus or stress – resembling \isi{contrastive focus} marking – is put on the denoted property, like in \textit{hóng hūa} [red flower] ‘a red flower’, \textit{hóng de hūa} [\textsc{red} \textsc{attr} flower] ‘a flower that is red (and not of a different color)’ \citep[119–123]{li-etal1981}.}
%Andreas Hölzl: it may be interesting to notice that there is an additional use of de for modifying verbs and that there are three different characters for de depending on its use: 的, 得, and 地
%%%
\begin{exe}
\ex
\label{multi mand}
\langinfo{Mandarin Chinese}{Sino-Tibetan}{\citealt{li-etal1981}}
\begin{xlist}
\ex	{\rm Noun (possessor) attribute}\\
\label{mandarin noun}
\gll	Zhāngsān 	\textbf{de} 	shū\\
	Zhangsang 	\textsc{attr} 	book\\
\glt	‘Zhangsang's book’
%%%
\ex	{\rm Adjectival attribute}\\
\label{mandarin adj}
\gll	xīn 		(\textbf{de}) 	shū\\
	new	 	(\textsc{attr}) 	book\\
\glt	‘new book’
%%%
\ex	{\rm Verbal (relative clause) attribute}\\
\label{mandarin rel}
\gll	wŏ 			zuótiān 	măi 		\textbf{de} 	shū\\
	\textsc{1sg} 	buy		yesterday 	\textsc{attr} 	book\\
\glt	‘the book I bought yesterday’
\end{xlist}
\end{exe}
%%%

In Minangkabau,\il{Minangkabau} an Austronesian language spoken on Sumatra in Indonesia, \isi{juxtaposition} is polyfunctional to a similar degree.
%%%
\begin{exe}
\ex
\label{multi minangkabau}
\langinfo{Minangkabau}{Austronesian}{\citealt[3–4]{gil2005}}
\begin{xlist}
\ex {\rm Noun (possessor) attribute}\\
\gll	batiak Kairil\\
	papaya Kairil\\
\glt	‘Kairil's papaya’
\ex {\rm Adjectival attribute}\\
\gll	batiak kuniang\\
	papaya yellow\\
\glt	‘a/the yellow papaya’
\ex {\rm Verbal (relative clause) attribute}\\
\gll	batiak Kairil bali\\
	papaya Kairil buy\\
\glt	‘a/the papaya that Kairil bought’
\end{xlist}
\end{exe}
%%%

Tagalog\il{Tagalog} is another language with a polyfunctional attribution marker. The Tagalog\il{Tagalog} \isi{linker}, however, is less polyfunctional than \isi{juxtaposition} in Minangkabau\il{Minangkabau} or anti\hyp{}construct state marking in Mandarin Chinese.\il{Mandarin Chinese} It marks only verbal\is{adnominal modifier!verb} and adjectival attributes.\footnote{Note that the constituent order of attribute and head noun is free in Tagalog: the relative clause and the adjective can also occur in a head-initial phrase type. In this case, the linker \textit{=ng} attaches phonologically to the noun (\citealt[1]{gil2005}; \citealt[160, 162]{himmelmann1997}).}\il{Tagalog}
%%%
\begin{exe}
\ex
\label{multi tagalog}
\langinfo{Tagalog}{Austronesian}{\citealt[6]{gil2005}}
\begin{xlist}
\ex {\rm Adjectival attribute}\\
\gll	pula\textbf{=ng} mangga\\
	red{=\textsc{attr}} mango\\
\glt	‘red mango’
\ex {\rm Verbal (relative clause) attribute}\\
\gll	binili ni Jojo\textbf{=ng} mangga\\
	bought \textsc{pers.gen} Jojo{=\textsc{attr}} mango\\
\glt	‘a/the mango that Jojo bought’
\end{xlist}
\end{exe}
%%%

\il{Persian|(}
Highly polyfunctional attribution marking by means of a head-marking construct suffix is found even in Persian.\footnote{Note the consistent glossing \textsc{mod} instead of \textsc{attr}. The Persian construct marker licenses modification beyond attribution.}
%%%
\begin{exe}
\ex 
\label{multi persian}
\langinfo{Persian}{Indo-European}{\citealt{mahootian1997}}
\begin{xlist}
\ex {\rm Adposition phrase}\\
\gll	tu\textbf{-ye} ašpæzxune\\
	in\textbf{-\textsc{mod}} kitchen\\
\glt ‘in the kitchen’
%%%
\ex {\rm Nominal attribution}
\begin{xlist}
\ex {\rm Noun (non-possessor) attribute}\\
\gll 	ængoštær\textbf{-e} ælmas\\
	ring-\textsc{mod} diamond\\
\glt 	‘diamond ring’
\ex {\rm Noun (possessor) attribute}\\
\gll	ængoštær\textbf{-e} pedær\\
	ring-\textsc{mod} father\\
\glt	‘father's ring’
\end{xlist}
%%%
\ex {\rm Adjectival attribute}\\
\gll	ælmas\textbf{-e} bozorg\\
	diamond-\textsc{mod} big\\
\glt	‘a big diamond’
%%%
\ex {\rm Adpositional attribute}\\
\gll	miz\textbf{-e} tu{-ye} ašpæzxune\\
	table-\textsc{mod} in{-\textsc{mod}} kitchen\\
\glt ‘the table in the kitchen’
\ex {\rm (Infinite) verbal attribute}\\
\gll	væqt\textbf{-e} ræftæn\\
	time-\textsc{mod} to\_go\\
\glt	‘time to go’
\end{xlist}
\end{exe}
%%%

While the same device marks nominal, adjectival, adpositional\is{adnominal modifier!adposition phrase} and (infinite) verbal\is{adnominal modifier!verb} attributes, finite verbal\is{adnominal modifier!verb} attributes (relative clauses)\is{adnominal modifier!relative clause} never occur in a similar noun phrase type in Persian.
\il{Persian|)}

\il{Swedish!Västerbotten|(}
In Västerbotten Swedish, a language variety of the northern Eurasian area under investigation, attribution marking by means of adjective incorporation is also considered to be polyfunctional (see \S\S\ref{attr incorporation}, \ref{bondska synchr}). Beside adjective attribution, the device marks attribution of (human) possessors.\is{adnominal modifier!possessor noun}
%%%
\begin{exe}
\ex 
\label{multi bondska}
{\rm Västerbotten Swedish (Indo-European; \citealt[examples from][5]{gil2005})}
\begin{xlist}
\ex {\rm Noun (human possessor) attribute}\\
\gll	Pelle-äpple\\
	Pelle-apple\\
\glt	‘Pelle's apple’
\ex {\rm Adjectival attribute}\\
%\begin{xlist}
\gll	rö-äpple\\
	red-apple\\
\glt	‘red apple’
%\ex (past participle)
%\gll	(stick-ad)-tröja\\
%	(knitt-\textsc{ptcp:pst})-sweater\\
%\glt	‘knitted sweater’
%\end{xlist}
\end{xlist}
\end{exe}
%%%
\il{Swedish!Västerbotten|)}

%bessere Referenz zu VB-Schwedisch?, participle in VB-Swedish?
\is{adnominal modifier!possessor noun|(}
\is{adnominal modifier!relative clause|(}
\citet{gil2005} surveyed the polyfunctionality of attribution markers licensing possessor nouns, adjectives and relative clauses in a world-wide sample of languages. According to the number of morpho-syntactically differentiated classes of attributes Gil grouped the languages of his sample into the following types:

%%%
\begin{itemize}
\item \textbf{Weakly differentiating languages} using polyfunctional devices for attribution of all three syntactic categories, as in Mandarin Chinese\il{Mandarin Chinese} (\ref{multi mand}) and Minangkabau\il{Minangkabau} (\ref{multi minangkabau})
\item \textbf{Moderately differentiating languages} using polyfunctional devices for attribution of two syntactic categories, for instance:
	\begin{itemize}
	\item adjectives and relative clauses, as in Tagalog\il{Tagalog} (\ref{multi tagalog})
	\item possessor nouns and adjectives, as in Västerbotten Swedish\il{Swedish!Västerbotten} (\ref{multi bondska}) and Persian\il{Persian} (\ref{multi persian})
	\end{itemize}
\item \textbf{Highly differentiating languages} are not polyfunctional at all, as in German\il{German} where the three syntactic classes are marked differently.
\end{itemize}
%%%

In Gil's\ia{Gil, David} sample, Europe\is{Europe} and adjacent parts of Asia\is{Asia} and Africa\is{Africa} stand out as an area with predominantly non-polyfunctional languages, while almost all languages of Southeast Asia\il{Southeast Asian languages} are of low differentiation \citep[8]{gil2005}.

%%%
\begin{figure}[t]
\parbox[b]{0.20\textwidth}{
\begin{center}\textsc{Mandarin},\\\textsc{Minangkabau}\\
\medskip
\begin{tabular}{| l |}
\hline
\\
\hline
\hline
\\
\hline
\textsc{attr}\textsubscript{Rel}\\
\hline
\textsc{attr}\textsubscript{A}\\
\hline
\textsc{attr}\textsubscript{N}\\
\hline
\end{tabular}
\end{center}
}
\parbox[b]{0.20\textwidth}{
\begin{center}\textsc{Tagalog}\\
\bigskip
\begin{tabular}{| l |}
\hline
\\
\hline
\hline
\\
\hline
\textsc{attr}\textsubscript{Rel}\\
\hline
\textsc{attr}\textsubscript{A}\\
\hline
\\
\hline
\end{tabular}
\end{center}
}
\parbox[b]{0.20\textwidth}{
\begin{center}\textsc{Västerbotten }\\\textsc{Swedish}\\
\medskip
\begin{tabular}{| l |}
\hline
\\
\hline
\hline
\\
\hline
\\
\hline
\textsc{attr}\textsubscript{A}\\
\hline
\textsc{attr}\textsubscript{N}\\
\hline
\end{tabular}
\end{center}
}
\parbox[b]{0.20\textwidth}{
\begin{center}\textsc{Persian}\\
\bigskip
\begin{tabular}{| l |}
\hline
\textsc{mod}\textsubscript{NP}\\
\hline
\hline
\textsc{attr}\textsubscript{AdP}\\
\hline
\\
\hline
\textsc{attr}\textsubscript{A}\\
\hline
\textsc{attr}\textsubscript{N}\\
\hline
\end{tabular}
\end{center}
}
\caption[Functional map for modification marking]{Functional maps for modification markers: the anti\hyp{}construct state marking in \textsc{Mandarin Chinese} and juxtaposition in \textsc{Minangkabau}, the linker in \textsc{Tagalog}, adjective incorporation in \textsc{Västerbotten Swedish} and construct state marking in \textsc{Persian}
\label{multi abcd}
\il{Mandarin Chinese}\il{Minangkabau}\il{Tagalog}\il{Swedish!Västerbotten}\il{Persian}
}
\end{figure}
%%%

Northern Eurasian languages of the “moderately differentiating” type included in Gil's\ia{Gil, David} sample are Japanese\il{Japanese} and Västerbotten Swedish\il{Swedish!Västerbotten} (with polyfunctional attribution marking of possessor nouns and adjectives) as well as Ainu,\il{Ainu} Nivkh\il{Nivkh} and Tatar\il{Tatar} (with polyfunctional attribution marking of adnominal adjectives and relative clauses).\footnote{Note that English\il{English} is not coded as “moderately differentiating” by \citet{gil2005}, although \isi{juxtaposition} can be used polyfunctionally as a device for attribution of adjectives and relative clauses (with reverse constituent order though: \textit{The woman I saw.})} No languages of the “weakly differentiated” type are known to occur in the northern Eurasian area. 

\largerpage[2]
Figure~\ref{multi abcd} illustrates the polyfunctionality of modification markers in the languages mentioned in this chapter.\footnote{\citep[Cf.][]{haspelmath2003}, for a systematic and historiographic description of functional (or semantic) maps.} The true attributive functions of the marker, i.e., licensing of adpositional, verbal, and adjectival attributes, are found in the middle cells of the left column in Figure~\ref{multi abcd}. The cell extending upwards shows the additional function of the marker as licenser of modification above the noun phrase level (i.e., inside an adposition phrase).%or below the noun phrase level (i.e., inside an adjective phrase).

The order of \textsc{attr}\textsubscript{Rel} through \textsc{attr}\textsubscript{N} in these functional maps corresponds to the hierarchical alignment of polyfunctional attribution marking suggested for Austronesian languages by \citet{foley1980}.\footnote{Note that Foley's hierarchy is proposed to be cross-linguistically valid and even includes two more syntactic classes than considered here: Determiner > Numeral > Noun > Adjective > Verb.}
\is{adnominal modifier!noun}
\is{adnominal modifier!numeral}
\is{adnominal modifier!verb}
%%%
\begin{exe}
\ex	{\rm Noun < Adjective < Verb}
\end{exe}
%%%

The hierarchy is to be read as follows: the highest category of attributive modifiers are verbs (i.e., relative and other attributive clauses), the next lower categories are adjectives and nouns. If one attributive category is marked with a polyfunctional attribution marker, all categories to the left side in the hierarchy should be marked by the same device, too.

\begin{figure}
\parbox[b]{\textwidth}{
\begin{center}
\begin{tabular}{| l || c | c |}
\cline{1-1}
\\
\hline
\textsc{attr}\textsubscript{Rel} & \textsc{nmlz} & \textsc{foc}\\
\hline
\textsc{attr}\textsubscript{A}\\
\cline{1-1}
\textsc{attr}\textsubscript{N}\\
\cline{1-1}
\end{tabular}
\end{center}
}
\caption[Functional map for modification marking]{Functional map for the modification marker \textit{ve} in \textsc{Lahu}\il{Lahu}}
\label{lahu funcmap}
\end{figure}
\is{adnominal modifier!possessor noun|)}
\is{adnominal modifier!relative clause|)}

\section[Polyfunctionality and additional content]{Polyfunctionality of modification markers and additional content}
%%%

\largerpage[-1]
Polyfunctional modification marking devices with semantic content (or function) beyond attribution are also attested in several languages. Lahu\il{Lahu} is an example of a Southeast Asian language\il{Southeast Asian languages} of the “weakly differentiating” type according to Gil's (\citeyear{gil2005}) classification. Syntactically similar to Mandarin Chinese,\il{Mandarin Chinese} Lahu\il{Lahu} exhibits an anti\hyp{}construct state marker \textit{ve} that licenses adjectival (\ref{lahu adj}), nominal\is{adnominal modifier!noun} (\ref{lahu noun}) and verbal\is{adnominal modifier!verb} attributes (\ref{lahu rel}). In addition, the marker \textit{ve} in Lahu\il{Lahu} is used as a nominalizer\is{attributive nominalization} (\ref{lahu compl}) and as a focus marker (\ref{lahu focus}).\footnote{See \citealt{bickel1999} on the “Standard Sino-Tibetan\il{Sino-Tibetan languages} Nominalization pattern” (which in some languages include even additional content beyond attribution, nominalization and focus.}
%%%
\begin{exe}
\ex
\langinfo{Lahu}{Sino-Tibetan}{\citealt{matisoff1973}}
\begin{xlist}
\ex	{\rm Attribution}
\begin{xlist}
\ex	{\rm Adjectival attribute}\\
\label{lahu adj}
\gll	dàʔ	\textbf{ve}	ŋâʔ\\
	pretty	\textsc{attr}	bird\\
\glt	‘pretty birds’ (194)
%%%
\ex	{\rm Noun (possessor) attribute}\\
\label{lahu noun}
\gll	Càl\^{ɔ}	\textbf{ve}	\`{ɔ}ha\\
	Jalaw	\textsc{attr}	picture\\
\glt	‘Jalaw's picture’ (141)
%%%
\ex	{\rm Verbal (relative clause) attribute}\\
\label{lahu rel}
\gll	c\'{ɔ}	câ	\textbf{ve}	ŋâʔ\\
	boil	eat	\textsc{attr}	bird\\
\glt	‘birds one boils to eat’ (194)
\end{xlist}
%%%
\ex	{\rm Additional semantic content}
\begin{xlist}
\ex {\rm Nominalization (of a complement clause)}\\
\label{lahu compl}
\gll	n\`{ɔ}	qôʔ \textbf{ve}	thàʔ	ŋà mâ	na ɣa	qôʔ-ma!\\
	you	say \textsc{nmlz}	\textsc{acc} I	\textsc{neg} understand	be\_able	\textsc{interj}\\
\glt	‘I can't catch what you're saying!’ (157)
%%%
\ex	{\rm Focusing (of a clause)}\\
\label{lahu focus}
\gll	mâ		qay	\textbf{ve}\\
	\textsc{neg}	go	\textsc{foc}\\
\glt	‘I am certainly not going.’ (362)
\end{xlist}
\end{xlist}
\end{exe}
%%%

The functions of the marker \textit{ve} in Lahu\il{Lahu} can also be summarized in a functional map, see Figure~\ref{lahu funcmap}. The true attributive functions of the marker, i.e., licensing of verbal,\is{adnominal modifier!verb} nominal\is{adnominal modifier!noun} and adjectival attributes, are found in the cells of the left column in Figure~\ref{lahu funcmap}. The cells extending to the right show the additional content of the attributive marker, i.e., as a nominalizer\is{attributive nominalization} and focus marker of a clause.

\section{Conclusion}
%%%
From a purely synchronic point of view, polyfunctionality of adjective attribution marking devices seems less relevant to the area under investigation, northern Eurasia. Most languages of the area exhibit highly differentiated attribution marking devices. Languages of the “moderately differentiating” type are rare; no languages of the “weakly differentiated” type are known to occur in the northern Eurasian area at all.

However, polyfunctionality can indicate historical change if additional semantic content of attribution marking devices across related languages is taken into consideration. The topic of polyfunctional attribution markers across languages of one family will thus be taken up again in Part~III (Synchrony) of this study.
\is{modification marking|)}
