
\chapter{Areal typology in the Circum-Baltic area}\label{circumbaltic}
%%%
\il{Germanic languages|(}\il{Baltic languages|(}\il{Slavic languages|(}\il{Finnic languages|(}\il{Saamic languages|(}
\is{Circum-Baltic area|(}
\is{buffer zone|(}
The Circum-Baltic area can be defined geographically as the drainage area of the Baltic Sea. Autochthon languages belonging to this area are mostly from the Germanic, Baltic and Slavic branches of Indo-European\il{Indo-European languages} as well as from the Finnic and Saamic branches of Uralic.\il{Uralic languages} Several authors have tried to establish a Circum-Baltic Linguistic Area (\isi{Sprachbund}) based on shared linguistic features across member languages of this area (for instance \citealt{koptjevskaja-tamm2006}).

\cite{riesler2006a} described areality in the morpho-syntax of noun phrase structure in the Circum-Baltic languages. It is conspicuous that both the languages of the two Uralic\il{Uralic languages} branches of the area and the languages of the three contacting Indo-European\il{Indo-European languages} branches have innovated adjective attribution marking devices which deviate from the prototypes of their respective families.

Saamic innovated anti\hyp{}construct state marking and Finnic innovated \isi{head\hyp{}driven agreement}. The prototype of adjective attribution marking in Uralic,\il{Uralic languages} however, is \isi{juxtaposition}. Except in Saamic and Finnic, juxtaposition occurs in all Uralic\il{Uralic languages} languages as the default adjective attribution marking device (see \S\ref{uralic synchr}) and is also reconstructed for Proto\hyp{}Uralic\il{Proto\hyp{}Uralic} (\citealt[66, 81]{decsy1990}; \citealt[32]{janhunen1981}).

Head\hyp{}driven agreement\is{head\hyp{}driven agreement} is the prototype of adjective attribution marking in Indo-European\il{Indo-European languages} and is also the type reconstructed for Proto\hyp{}Indo-European\il{Proto\hyp{}Indo-European} \citep{decsy1991,watkins1998}. In Germanic, Baltic and Slavic, however, a secondary type evolved from \isi{attributive nominalization}. Consequently, several modern languages of these branches exhibit anti\hyp{}construct state agreement marking as a default or secondary device.

All five Circum-Baltic branches (Germanic, Baltic, Slavic, Finnic, Saamic) of the “buffer zone” have thus undergone change and innovated adjective attribution marking devices which deviate from the prototypes of their respective families:
%%%
\begin{itemize}
\item	\textbf{Finnic:}
\subitem Juxtaposition ≠ \isi{head\hyp{}driven agreement}\is{juxtaposition}
\item	\textbf{Saamic:}
\subitem Juxtaposition ≠ Anti\hyp{}construct state
\item	\textbf{Germanic, Baltic, Slavic:}
\subitem Head\hyp{}driven agreement ≠ Anti\hyp{}construct state agreement
\end{itemize}
%%%
The developments in Saamic and in the three Indo-European\il{Indo-European languages} branches can even be connected to each other in structural terms: the innovative anti\hyp{}construct state (agreement) marking in these languages evolved from an attributive apposition construction marked by means of attributive nominalizers.\is{attributive nominalization} The principal \isi{grammaticalization} paths are thus similar:
%%%
\begin{exe}
\ex {\upshape [}\textsubscript{\rm NP} {\upshape [}\textsubscript{\rm NP'} \textsubscript{\rm A}big \textsubscript{\rm HEAD}Ø-\textsc{nmlz}{\upshape ]} \textsubscript{\rm N}house{\upshape ]} $\Rightarrow$ {\upshape [}\textsubscript{\rm NP} \textsubscript{\rm A}big-\textsc{attr} \textsubscript{\rm N}house{\upshape ]}
\end{exe}
%%%
\is{grammaticalization!grammaticalization area|(}
\il{Proto\hyp{}Germanic|(}\il{Proto\hyp{}Baltic\slash{}Slavic|(}\il{Proto\hyp{}Finnic|(}\il{Proto\hyp{}Saamic|(}
Therefore, \citet[271]{riesler2006a} described the result of this areal innovation as a “grammaticalization area” \citep{heine-etal2005}, i.e., a linguistic area of geographically neighboring languages in which similar processes of grammatical changes took place as the result of language contact. According to \citet{heine-etal2005}, a \textit{model language} must affect at least two different \textit{replica languages} in a grammaticalization area. In the case described here, a pre-proto-stage of either Germanic or Baltic\slash{}Slavic could probably be the “model” since \isi{attributive nominalization} by means of cognate markers evolved in several other branches of Indo-European.\il{Indo-European languages} But even Uralic\il{Uralic languages} influence should be considered. Possible model and replica languages of the area are thus:
%%%
\begin{itemize}
\item	\begin{center}Proto\hyp{}Baltic\slash{}Slavic <~\textbf{Pre-Proto\hyp{}Germanic} > Proto\hyp{}Saamic\end{center}
\item	\begin{center}Proto\hyp{}Germanic <~\textbf{Pre-Proto\hyp{}Baltic\slash{}Slavic} > Proto\hyp{}Saamic\end{center}
\item \begin{center}Proto\hyp{}Baltic\slash{}Slavic <~\textbf{Pre-Proto\hyp{}Saamic} > Proto\hyp{}Germanic\end{center}
\end{itemize}
\il{Proto\hyp{}Germanic|)}\il{Proto\hyp{}Baltic\slash{}Slavic|)}\il{Proto\hyp{}Finnic|)}\il{Proto\hyp{}Saamic|)}
%%%
Given the high age and the cognate constructions and formatives in other Indo-European\il{Indo-European languages} branches (mostly Iranian)\il{Iranian languages} and considering other attested Baltic contact influence on Saamic,\footnote{See, for instance, \citealt{riesler2009} for lexical borrowings.} it seems most plausible to locate the core of the grammaticalization area in the Baltic\slash{}Slavic groups of Indo-European.\il{Indo-European languages} Saamic and Germanic have probably borrowed the model of \isi{attributive nominalization} but realized the construction with their own inherited morpho-syntactic means.
\is{grammaticalization!grammaticalization area|)}

Nonetheless the vast geographic spread of cognate constructions among several Indo-European,\il{Indo-European languages} Uralic,\il{Uralic languages} Turkic\il{Turkic languages} and even Tungusic\il{Tungusic languages} branches makes it also possible to assume a source outside both Indo-European\il{Indo-European languages} and Uralic\il{Uralic languages} and a development preceding the proto-stages of these language families.
\is{buffer zone|)}
\is{Circum-Baltic area|)}
\il{Germanic languages|)}\il{Baltic languages|)}\il{Slavic languages|)}\il{Finnic languages|)}\il{Saamic languages|)}
