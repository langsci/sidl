
\chapter{Areal typology in the Circum-Baltic area}\label{circumbaltic}
The Circum-Baltic area can be defined geographically as the drainage area of the Baltic Sea. Autochthon languages belonging to this area are mostly from the Germanic, Baltic and Slavic branches of Indoeuropean as well as from the Finnic and Saamic branches of Uralic. Several authors have tried to establish a Circum-Baltic Linguistic Area (Sprachbund) based on shared linguistic features across member languages of this area (for instance \citealt{koptjevskaja-tamm2006}).

\cite{riesler2006a} described areality in the morpho-syntax of noun phrase structure in the Circum-Baltic languages. It is conspicuous that both the languages of the two Uralic branches of the area and the languages of the three contacting Indoeuropean branches have innovated adjective attribution marking devices which deviate from the prototypes of their respective families.

Saamic innovated anti-construct state marking and Finnic innovated head-driven agreement. The prototype of adjective attribution marking in Uralic, however, is juxtaposition. Except in Saamic and Finnic, juxtaposition occurs in all Uralic languages as the default adjective attribution marking device (cf.~Section \ref{uralic synchr}) and is also reconstructed for proto-Uralic (\citealt[66, 81]{decsy1990}; \citealt[32]{janhunen1981}).

The prototype of adjective attribution marking in Indoeuropean is head-driven agreement and is also the type reconstructed for proto-Indoeuropean \citep{decsy1991,watkins1998}. In Germanic, Baltic and Slavic, however, a secondary type evolved from attributive nominalization. Consequently, several modern languages of these branches exhibit anti-construct state agreement marking as a default or secondary device.

All five Circum-Baltic branches (Germanic, Baltic, Slavic, Finnic, Saamic) of the “buffer zone” have thus innovated adjective attribution marking devices which deviate from the prototypes of their respective families:

\begin{itemize}
\item	\textbf{Finnic:}
\subitem Juxtaposition ≠ Head-driven agreement
\item	\textbf{Saamic:}
\subitem Juxtaposition ≠ Anti-construct state
\item	\textbf{Germanic, Baltic, Slavic:}
\subitem Head-driven agreement ≠ Anti-construct state agreement
\end{itemize}
The developments in Saamic and in the three Indoeuropean branches can even be connected to each other in structural terms: The innovative anti-construct state (agreement) marking in these languages evolved from an attributive apposition construction marked by means of attributive nominalizers. The principal grammaticalization paths are thus similar:
\begin{exe}
\ex $[_{NP}$ $[_{NP'}$ $_{A}$big $_{HEAD}$Ø-\textsc{nmlz}$]$ $_{N}$house$]$ $\Rightarrow$ $[_{NP}$ $_{A}$big-\textsc{attr} $_{N}$house$]$
\end{exe}
Therefore, \citet[271]{riesler2006a} described the result of this areal innovation as a “grammaticalization area” \citep{heine-etal2005}, i.e.~a linguistic area of geographically neighboring languages in which similar processes of grammatical changes took place as the result of language contact. According to \citet{heine-etal2005}, a \textit{model language} must affect at least two different \textit{replica languages} in a grammaticalization area. In the case described here, a pre-proto-stage of either Germanic or Balto-Slavic could probably be the “model” since attributive nominalization by means of cognate markers evolved in several other branches of Indoeuropean. But even Uralic influence should be considered. Possible model and replica languages of the area are thus:
\begin{itemize}
\item	\begin{center}proto-Baltic/Slavic < \textbf{pre-proto-Germanic} > proto-Saamic\end{center}
\item	\begin{center}proto-Germanic < \textbf{pre-proto-Baltic/Slavic} > proto-Saamic\end{center}
\item \begin{center}proto-Baltic/Slavic < \textbf{pre-proto-Saamic} > proto-Germanic\end{center}
\end{itemize}
Given the high age and the cognate constructions and formatives in other Indoeuropean branches (mostly Iranian) and considering other attested Baltic contact influence on Saamic,\footnote{Cf. for instance \citealt{riesler2009} for lexical borrowings.} it seems most plausible to locate the core of the grammaticalization area in the Balto-Slavic groups of Indoeuropean. Saamic and Germanic have probably borrowed the model of attributive nominalization but realized the construction with their own inherited morpho-syntactic means.

Nonetheless the vast geographic spread of cognate constructions among several Indoeuropean, Uralic, Turkic and even Tungusic branches makes it also possible to assume a source outside both Indoeuropean and Uralic and a development preceding the proto-stages of these language families.
