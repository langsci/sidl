
\chapter{Results and conclusions}

These concluding sections summarize the essence of this study, provide an overview of the main findings and address a few questions relevant for a future investigation towards a general description of noun phrase structures.

\section{Aims and content}
The aims of this study were: (1) a synchronic-typological description of adjective attribution marking devices in northern Eurasia, i.e.~typologizing geographically relevant languages according to their syntactic and morpho-syntactic kinds of adjective attribution marking, (2) a synchronic survey of the geographic distribution of the attested kinds of adjective attribution marking devices across the northern Eurasian languages, and (3) a diachronic description and functional reconstruction of hitherto undescribed patterns in the evolution of adjective attribution marking in the Indoeuropean and Uralic languages of the Circum-Baltic area of northern Europe.

\paragraph{(1)} As the main result of the {\bf synchronic-typological description}, an ontological classification of attested syntactic and morpho-syntactic types of adjective attribution marking devices was developed. For the purpose of comparison and achieving stringent classification standards, even interesting devices attested in languages outside the area were taken into consideration. 

Central typological parameters for the morpho-syntactic description of noun phrase structure are \textit{syntactic source} (i.e.~the central syntactic operation which licenses attribution and belongs primarily either to agreement marking or to government), \textit{syntactic pattern} (i.e.~devices projecting embedded noun phrases, devices projecting simple adjective phrases, or incorporation) and \textit{syntactic locus} of the respective formatives (on-head, on-dependent, floating).

The following overview lists all known devices (one single device, which is not attested in the northern Eurasian area, is given in parentheses).
\begin{itemize}
\item{Juxtaposition} (as in Komi)
\item{Incorporation} (as in Chukchi)
\item{Construct state} (as in Kurmanji)
\item{Anti-construct state} (as in Saami)
\item{Attributive nominalization} (as in Udmurt)
\item{Attributive article} (as in Yiddish)
\item{Anti-construct state agreement} (as in Russian)
\item{Head-driven agreement} (as in Finnish)
\item{Apposed head-driven agreement} (as in Georgian)
\item{Modifier-headed possessor agreement} (as in Saliba)
\item{(Linker)} (as in Tagalog [attested only outside northern Eurasia])
\end{itemize}
A more detailed overview of the attested types including definitions and an ontological cross-classification is presented in Tables \ref{tabledefontology} and \ref{syntaxontology} on pages \pageref{tabledefontology}–\pageref{syntaxontology} and in Figure \ref{tree ontology} on page \pageref{tree ontology}.

\paragraph{(2)} The {\bf synchronic survey} showed that the most common types of adjective attribution marking devices are head-driven agreement (the Indoeuropean prototype which characterizes most parts of the European linguistic map) and juxtaposition (the prototype in Uralic, Turkic and Mongolic which monotonously characterizes larger parts of North Asia). Modifier-headed possessor agreement is the least common type in northern Eurasia since it is known to occur only in Tungusic. The Mongolic and Turkic families of North Asia exhibit a very low degree of diversity in regard to their adjective attribution marking devices. A relatively high degree of diversity characterizes several branches of Indoeuropean (especially Germanic and Indo-Iranian) and Uralic (especially Saamic). Typological diversity is thus predominantly found in peripheral subareas of Northern Eurasia where different language families meet, for instance in the Circum-Baltic area in northernmost Europe and in Inner Asia (cf.~Section \ref{areality}).

\paragraph{(3)} The {\bf diachronic description} revealed a re-occurring pattern of attributive nominalizers developing further into attributive state markers and various other types of attribution marking devices in different languages of the area and during different periods of time. These structurally similar diachronic paths, which had not yet been systematically investigated from a cross-linguistic perspective, were reconstructed in detail for Balto-Slavic (Section \ref{slavic diachr}), Germanic (Section \ref{germanic diachr}) and Saamic (Section \ref{saamic diachr}). In the three Indoeuropean branches Baltic, Slavic and Germanic, anti-construct state agreement marking evolved from attributive nominalization. Anti-construct state marking arose in the Saamic branch of Uralic as the result of a structurally similar development from attributive nominalization. The developments in the geographically adjacent but genealogically unrelated languages present evidence for areality across Circum-Baltic languages (Section \ref{circumbaltic}).\\

\noindent The thesis also provides an extensive appendix including a list of 234 languages sorted by their genealogical affiliation and coded for attested noun phrase types (Table \ref{sample} starting on page \pageref{sample}) as well as a collection of maps illustrating the spread of attested noun phrase types across a world sample of languages (Maps \ref{WorldMap} and \ref{WorldMapTyp} on pages \pageref{WorldMap}–\pageref{WorldMapTyp}), across all northern Eurasian genera (Maps \ref{NEMap} through \ref{NAMapTyp} on pages \pageref{NEMap}–\pageref{NAMapTyp}) and across Europe languages (Maps \ref{EUMap} and \ref{EUMapTyp} on pages \pageref{EUMap}–\pageref{EUMapTyp}).

\section{Innovative findings}

The study presents the first systematic description and mapping of all attested adjective attribution marking devices in the languages of northern Eurasia. It also provides the first complete ontology of adjective attribution marking devices based on syntactic and morpho-syntactic noun phrase types found in northern Eurasian languages. The geographic spread of different adjective attribution marking devices across the main genera of all northern Eurasian language families is surveyed and mapped similar in a way similar to the Eurotyp-surveys\footnote{Cf.~\url{http://www-uilots.let.uu.nl/ltrc/eurotyp/}} but covering a larger area. 

The present study has a strong diachronic component. Synchronic typological research certainly sheds light on the evolution of language; nevertheless, linguistic typology can scarcely be considered a historical discipline per se since the applied method is most often exclusively a synchronic comparison of linguistic data. The present investigation, however, achieved a historical reconstruction of adjective attribution marking in several languages by using the historical-comparative method in combination with synchronic typology. By applying this innovative methodological approach a new hypothesis about the origin of secondary adjective attribution marking devices in Germanic, Balto-Slavic and Saamic can be proposed. 

The three most important results of this study are (1) the discovery that \textsc{state} has to be included in the inventory of morpho-syntactic features, (2) the finding that adjectival modifiers can be phrasally embedded constituents, and (3) the diachronic attestation of contrastive-focus constructions with phrasally embedded adjectival modifiers as a common source of innovative adjective attribution marking devices in the northern Eurasian languages.

\subsection{The morpho-syntactic feature \textsc{state}}
Morpho-syntax is commonly understood as phrase internal morphology, i.e. morphology assigned by syntax. The inventory of morpho-syntactic features thus excludes true morphological features which are assigned to phrasal constituents from (phrase internal) syntax. Prototypical examples of morphological features not assigned by noun phrase internal syntax are inflectional class of a noun (an inherent feature), definiteness marking of a noun (a feature assigned by semantics) or accusative marking of a noun phrase in object position inside a verb phrase (a morpho-syntactic feature assigned inside a verb phrase). The most typical morpho-syntactic features in noun phrase syntax are assigned by agreement triggered by one constituent, for instance adjective agreement in definiteness or in accusative case. If agreement of dependent constituents is triggered by a head noun the relevant feature has first to be assigned to the head from outside: either by semantics (e.g.~definiteness) or by noun phrase external syntax (e.g.~accusative). 

Feature inventories (like the inventory presented by \citealt{kibort2010a}), however, do not yet include instances of morphological marking triggered not by constituents but by the syntactic structure as such. The present study provides an important contribution to the general typology of morpho-syntax by complementing the known inventory of morpho-syntactic features with truly morpho-syntactic devices, such as the well-known “construct state” in Persian. The state marker in Persian is not the result of either agreement or government but is assigned by syntax alone.

State markers (glossed in the following examples with the value \textsc{mod} “modification”) can occur with different loci, i.e.~on-head (\ref{hmstate}), on-dependent (\ref{dmstate}) or floating (\ref{floatingstate}).
\newpage
\begin{exe}
\ex
\begin{xlist}
\ex \textsc{Persian} head-marking state \label{hmstate}
\gll xane-ye bozorg\\
	house-\textsc{mod} big\\
\glt 	‘a/the large house’ (cf.~page \pageref{persian constr state}, elsewhere)
\ex	\textsc{Kildin Saami} dependent-marking state \label{dmstate}
\gll 	ēl'l'-es' 		pērrht.\\
	high-\textsc{mod}	house\\
\glt	‘a/the high house’ (cf.~page \pageref{kildin attr.adj.sg}, elsewhere)
\ex	\textsc{Tagalog} floating state \label{floatingstate}
\gll maganda-ng bahay / bahay na maganda\\
	beautiful-\textsc{mod} house {} house \textsc{attr} beautiful\\
\glt	‘a/the beautiful house’ (cf.~page \pageref{tagalog linker}, elsewhere)
\end{xlist}
\end{exe}
State marking is not restricted to noun phrase structure. In the following example, a state marker (glossed as a “modification marker”) marks a noun phrase as the dependent constituent inside an adpositional phrase.
\begin{exe}
\ex \textsc{Kildin Saami} dependent-marking state
\gll 	pērht		al'n\\
	house\textbackslash\textsc{mod}	on\\
\glt 	‘on a/the house’ (cf.~page \pageref{state ap kildin}, elsewhere)
\end{exe}

\subsection{Embedded adjectival modifiers: synchrony} \label{embeddedsynchr}
It is common knowledge that noun phrases can contain simple modifiers (like simple nouns: \textit{stone house} or adjective phrases: \textit{a big house}), embedded phrasal modifiers, i.e.~modifiers which are projected as complex noun phrases themselves (like an adnominal possessor noun phrase: \textit{John's sister's house}), or complex modifiers which are projected higher than noun phrases (like an adnominal adpositional phrase: \textit{house in the village} or an adnominal relative clause: \textit{a house which is huge}). It was demonstrated in the present analysis that even adjectival modifiers can constitute embedded noun phrases and occur in attributive apposition constructions, as in Udmurt:
%%!!??check
%\begin{exe}
%\ex Embedded adjectival attribute in \textsc{Udmurt} 
%\gll	badǯ́ym-ėz gurt\\
%	[$_{NP}$ [$_{NP'}$ $_{A}$big-\textsc{nmlz}] $_{N}$house]\\
%\glt	‘a/the LARGE house’ (cf.~page \pageref{udmurt synchr}, elsewhere)
%\end{exe}
Unexpected agreement features provided evidence for the embedded adjectival modifier in Udmurt (as well as in other languages). Such attributive apposition constructions are syntactically similar to the well-known nominalizations in South-East-Asian languages (see, e.g., \citealt{bickel1999} on the “Standard Sino-Tibetan nominalization pattern”). In the northern Eurasian area such constructions with embedded modifiers are especially  common in contrastive-focus constructions and as the diachronic source of several other adjective attribution marking devices (cf.~also Section \ref{embeddeddiachr}). 

Consequently, the syntactic ontology of adjective attribution marking presented in this study includes the phrasal projection of the attribution marking device as a central parameter with three values:
\begin{itemize}
\item embedded modifier
\item simple modifier
\item incorporated modifier
\end{itemize}
These parameters are applicable in a typology of general noun phrase syntax (including modifiers which are not adjectives and modifiers which are not simple constituents). Consider Table  \ref{ontologyderived} (derived from Table \ref{syntaxontology} on page \pageref{syntaxontology}) which includes a phrasally embedded attribute (like the juxtaposed relative clause in Minangkabau, example \ref{juxtrel}), a simple attribute (like the juxtaposed adjective in Komi, example \ref{juxta}) and an incorporated attribute (like the incorporated possessor in Västerbotten-Swedish, example \ref{juxtposs}).
\begin{exe}
\ex
\begin{xlist}
\ex \textsc{Minangkabau} juxtaposed relative clause \citep[3–4]{gil2005} \label{juxtrel}
\gll batiak Kairil bali\\
	papaya Kairil buy\\
\glt	‘the papaya Kairil bought’
\ex \textsc{Komi-Zyrian} juxtaposed adjective \citep[287]{lytkin1966a} \label{juxta}
\gll		bur	mort-jas\\
		good	person-\textsc{pl}\\
\glt		‘good people’
\ex \textsc{Västerbotten-Swedish} possessor noun incorporation \citep[3–4]{gil2005} \label{juxtposs}
\gll	Pelle-äpple\\
	Pelle-apple\\
\glt	‘Pelle's apple’
\end{xlist}
\end{exe}

\begin{table}
\begin{center}
\begin{footnotesize}
\begin{tabular}{c | c | c}
\hline
\hline
Phrasally embedded	attribute	&Simple attribute			&Incorporated attribute\\
\hline	
“juxtaposed relative clause”	&“adjective juxtaposition”		&“possessor incorporation”\\
\hline
%%!!??check
%[$_{NP}$ [$_{Rel}$ NP V] N]		&[$_{NP}$ A N]	&[$_{NP}$ $_{PSR}$N $_{PSD}$N]\\
\hline
\hline
\end{tabular}
\caption[Ontology of general noun phrase structure]{Ontology of general noun phrase structure (derived from Table \ref{syntaxontology} on page \pageref{syntaxontology} and restricted to morphologically unmarked attribution marking devices, i.e.~phrasally embedded, simple and incorporated attributes)}\label{ontologyderived}
\end{footnotesize}
\end{center}
\end{table}

\subsection{Embedded adjectival modifiers: diachrony}\label{embeddeddiachr}
Adjectival modifiers which are embedded as a noun phrase projection are common cross-linguistically in contrastive-focus constructions (cf.~also Section \ref{embeddedsynchr}), as in Udmurt:
%%!!??check
%\begin{exe}
%\ex \textsc{Udmurt} juxtaposed simple and embedded adjectival attribute 
%\begin{xlist}
%\ex	Juxtaposition (default)
%\gll	{}		badǯ́ym gurt\\
%	[$_{NP}$ $_{A}$big $_{N}$house]\\
%\glt	‘a/the large house’
%\ex	Attributive nominalization (contrastive focus)
%\gll	{}		{}			badǯ́ym-ėz gurt\\
%	[$_{NP}$ [$_{NP'}$ $_{A}$big-\textsc{nmlz}] $_{N}$house]\\
%\glt	‘a/the LARGE house’ (cf.~page \pageref{udmurt synchr}, elsewhere)
%\end{xlist}
%\end{exe}
In Udmurt, as in other languages where attributive nominalization is attested in constructions with contrastive-focused adjectives, focus  always takes scope over a whole noun phrase (but not over an adjective phrase). This explains why the adjective phrase has to be nominalized and occurs in an attributive appositional construction (i.e.~embedded as noun phrase with an empty head).

%\begin{exe}
%\ex
%\begin{xlist}
%!!??check
%\ex	Simple noun in contrastive focus\\[$_{NP}$ N]$_{focus}$
%\ex	Noun phrase with adjectival modifier in contrastive focus\\[$_{NP}$ A N]$_{focus}$
%\ex	Embedded adjectival modifier in contrastive focus\\[$_{NP}$ [$_{NP'}$ A]$_{focus}$ N]
%\ex	Simple adjectival modifier in contrastive focus (impossible)\\{*}[$_{NP}$ A_{focus}$ N]
%\end{xlist}
%\end{exe}

This synchronic finding is directly connected to the diachronic evidence for attributive apposition because attributive nominalization is a major (and chronologically re-occurring) diachronic source for the grammaticalization of new adjective attribution marking devices in different languages of the area.

The ultimate etymological source of attributive state marking formatives are prototypically local or person deictic markers (which also tend to be reanalyzed as markers of definiteness, cf.~Figure \ref{ie-ural funcmap} on page \pageref{ie-ural funcmap}). These markers are initially used as attributive nominalizers in contrastive-focus constructions and later reanalyzed either as anti-construct state markers or anti-construct state agreement markers:
%\begin{exe}
%\ex
%\begin{xlist}
%!!??check
%\ex $[_{NP}$ $[_{NP'}$ A-\textsc{nmlz}$]$_{focus} N$]$ $\Rightarrow$ $[_{NP}$ A-\textsc{attr} N$]$
%\ex $[_{NP}$ $[_{NP'}$ A-\textsc{nmlz:agr}$]$_{focus} N$]$ $\Rightarrow$ $[_{NP}$ A-\textsc{attr:agr} N$]$
%\end{xlist}
%\end{exe}

%\section{Other findings}
%
%\paragraph{Information structure and the evolution of attribution marking} 
%Cross-linguistic data show how relevant information structure is for the description of noun phrase syntax: In Indoeuropean, Uralic, Turkic, Tungusic and Kartvelian, secondary adjective attribution marking devices occur in contrastive-focus constructions. Since contrastive focus has scope over a whole noun phrase (but not over an adjective phrase) in all attested cases, the adjective is used in an attributive appositional construction, i.e. in an embedded noun phrase.
%
%Information structure is also relevant to diachronic noun phrase syntax because in several languages of northern Eurasia new primary devices were innovated from attributive appositional constructions. A typical grammaticalization path starts with attributive nominalization used as a secondary device in contrastive-focus constructions. The original emphatic construction with a phrasally embedded adjective is later reanalyzed as a default attribution marking device (either as anti-construct state or as anti-construct state agreement).
%
%Such a development started for instance in proto-Baltic, proto-Slavic and proto-Germanic where attributive nominalization arose as a secondary adjective attribution marking device (alongside the original head-driven agreement) in contrastive-focus constructions and developed further into anti-construct state agreement:
%%!!??check
%%\begin{exe}
%%\ex $[_{NP}$ $[_{NP'}$ A-\textsc{nmlz:agr}$]$ N$]$ $\Rightarrow$ $[_{NP}$ A-\textsc{attr:agr} N$]$
%%\end{exe}
%The etymological source of anti-construct state agreement markers in the Indoeuropean branches are local-deictic markers (demonstratives).
%
%Similarly, in proto-Saamic attributive nominalization arose as a secondary adjective attribution marking device (in addition to the original juxtaposition) in contrastive-focus constructions and developed further into anti-construct state:
%%%!!??check
%%\begin{exe}
%%\ex $[_{NP}$ $[_{NP}$ A-\textsc{nmlz}$]$ N$]$ $\Rightarrow$ $[_{NP}$ A-\textsc{attr} N$]$
%%\end{exe}
%The etymological source of anti-construct state marking in Saamic is a person-deictic marker (possessive suffix).
%
%Even proto-Finnic head-driven agreement likely originated in a contrastive-focus construction, specifically from appositive head-driven agreement which was reanalyzed as the default adjective attribution marking device under Indoeuropean influence:
%%%!!??check
%%\begin{exe}
%%\ex $[_{NP}$ $[_{NP'}$ A-\textsc{agr}$]$ N$]$ $\Rightarrow$ $[_{NP}$ A-\textsc{agr} N$]$
%%\end{exe}
%
%\paragraph{Attributive nominalization and definiteness marking}
%Data from Saamic and from other Uralic and Turkic languages in which attributive nominalizers originate from the possessive suffix 3\textsuperscript{rd} person singular contradict Himmelmann's (\citeyear[220–221]{himmelmann1997}) assumption that a functional convergence between attributive nominalizers and definiteness markers with a person-deictic or a local-deictic etymological source is unlikely to occur. 
%
%The data is, however, in accordance with Himmelmann's (\citeyear[220–221]{himmelmann1997}) assumption about the functional extension of deictic elements to attributive and definite markers if one acknowledges that definite markers with a local-deictic etymological source can evolve from attribution markers (but not vice versa), as in Indoeuropean:
%\begin{exe}
%\ex \textsc{dem $\Rightarrow$ nmlz ($\Rightarrow$ def)}
%\end{exe}
%By contrast, in the Uralic and Turkic languages, in which the etymological source is a person-deictic marker, attribution markers evolved from definite markers:
%\begin{exe}
%\ex \textsc{poss $\Rightarrow$ def $\Rightarrow$ nmlz}
%\end{exe}
%This finding implies an implicational universal: \textit{Possessive markers develop to attributive nominalizers only in languages in which similar possessive markers are already used as markers of (quasi-) definiteness} (cf.~Universal \ref{universal} on page \pageref{universal}).
%
%\paragraph{“Displaced” definiteness marking on adjectives}
%Synchronic and diachronic data from the languages analyzed in the present study provide clear evidence against the existence of “displaced” definiteness marking on attributive adjectives (as proposed, e.g., for Baltic languages or for Amharic; cf. \citealt[115]{dahl2007}). The primary function of the respective markers is always the licensing of adjective attribution (by means of attributive nominalization in contrastive-focus constructions). Even though there is a functional overlapping between attributive nominalization and definiteness marking from a diachronic perspective, the grammaticalization of definiteness marking is secondary in all attested cases.
%
%\paragraph{The northern European “buffer zone”}
%The Circum-Baltic branches Baltic, Germanic, Slavic (Indoeuropean), Saamic (Uralic) and possibly also Finnic (Uralic) constitute a “buffer zone” (similar to Stilo's \citeyear{stilo2005} notion of this term) between the Indoeuropean and Uralic prototypes of noun phrase structure.
%
%The Circum-Baltic “buffer zone” is the result of areal grammaticalization processes (similar to Heine's / Kuteva's \citeyear{heine-etal2005}  “grammaticalization area” ) in which new adjective attribution marking devices were grammaticalized from original attributive appositional constructions marking contrastive focus on the adjective. The developments are most likely the result of contact-induced changes and originate in proto-Baltic.
%
%\section{Prospects for future research}
%\paragraph{General noun phrase structure} The focus of the present study lies on noun phrases with adjectival modifiers, but taking a look at noun phrases with other modifiers (using, for instance, the AUTOTYP database of \citealt{AUTOTYP-NP}) suggests that the central morpho-syntactic parameters for the typologization of adjective attribution marking (i.e.~\textit{source, pattern} and \textit{locus}, see above) can be applied to a syntactic description of noun phrase structure in general. However, a systematic description of general noun phrase structure, including noun phrases with all possible kinds of adnominal modifiers (demonstratives, numerals, relative clauses, etc.) and performed on a world-wide sample of languages will most likely reveal several new noun phrase types and morpho-syntactic parameters. To illustrate this, one new parameter will be described below.   
%
%In AUTOTYP, several languages are coded in which the head-dependent relation in noun phrases has shifted in the sense that the semantic dependent shares at least some of the syntactic properties of the head. This resembles the type of modifier-headed possessor agreement found in Oroch or Saliba adjectives described in this study (cf.~also \citealt{malchukov2000} for a typology of “dependency reversal in noun-attributive constructions” and \citealt{ross1998}, who surveyed this type in Oceanic languages). Another prototypical example of such a modifier-headed noun phrase is found in Wari'. 
%\begin{exe}
%\ex \label{ex1} \textsc{Wari'} (<Chapacura-Wanham; \citealt{everett-etal1997})
%  \gll	mam mao 'in-on		ca	    mixem  pucun	wom-u\\
%  	with  go:\textsc{sg}  \textsc{1sg:real-3sg.m}	\textsc{real}  black    \textsc{poss:3sg.m}	cotton-\textsc{poss:1sg}\\
%  \glt ‘I went with my dirty clothes’ (lit. ‘with my cotton's blackness’)
%\end{exe}
%In the ontology presented in the present study, modifier-headed possessor agreement has been described as a device which is assigned by dependent-driven agreement (i.e.~possessor agreement) and which is also phrasally embedded (because the attribute takes the slot of the possessed noun phrase). The shifted head-dependent relation, however, was not included as a parameter in the ontological cross-classification because modifier-headed possessor agreement was the only type of modifier-headed noun phrases relevant for adjective attribution marking.
%
%The shifted head-dependent relation, however, can be relevant for the typologization of general noun phrase structure. In fact, several different types of modifier-headed noun phrases are attested with other kinds of modifiers, for instance in Russian and several other European languages in which numerals higher than one require special case marking on the head noun.
%\begin{exe}
%\ex \textsc{Russian}\label{russianheadstand}
%\begin{xlist}
%\ex \label{rushead}
%\gll tri mal'čik-a\\
%  	three		boy-\textsc{gen.sg.m}\\
%  \glt ‘three boys’
%\ex \label{rusgen}
%  \gll	kniga mal'čik-a\\
%  	book	 boy-\textsc{gen.sg.m}\\
%  \trans ‘the boy's book’
%\end{xlist}
%\end{exe}
%The noun ‘boy’ in the Russian construction with the numeral ‘three’ is marked with genitive case (\ref{rushead}). Consequently, this construction is syntactically equivalent to the genitive marked possessive noun phrase (\ref{rusgen}). The use of the (dependent-marking) possessor case in noun phrases with numeral modifiers suggests that the numeral is the syntactic head and the noun is the modifier. Since agreement is not involved in the assignment of the attribution marker, the type found in Russian is clearly distinguished from the above mentioned modifier-headed possessor agreement in Wari' and should therefore be labeled \textit{modifier-headed case} (cf.~\citealt{AUTOTYP-NP}).
%
%\paragraph{Polyfunctionality} In a typological survey of noun phrase structures, all types attested in a single language have to be coded if they are distinguished by a formal characteristic, such as a distinct marker, a distinct constituent order, a general marker with a distinct function, etc. Thus, this survey automatically accounts for the polyfunctionality of attribution marking if one and the same device is used with a similar function but for at least two different kinds of modifiers.
%
%A survey of polyfunctional attribution markers in a world-wide sample of languages has already been presented by \citet{gil2005} (see also Section \ref{polyfunctionality}). Gil's typology, however, is restricted to noun phrases with three different kinds of modifiers: possessor nouns, adjectives and relative clauses. A more thorough investigation of all kinds of multifunctional noun phrase markers in a restricted area (such as northern Europe) could trace the sub-areal distributions of various multifunctional types across certain genera. Together with a description of known evolutionary paths of attribution marking, such a survey would also help to develop a theory that accounts for polyfunctionality from both a diachronic and a synchronic perspective.
