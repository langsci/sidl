
\chapter{Noun phrases and adjectival modifiers}
%%%

\section{Noun phrases}
%%%
A noun phrase is a referential syntactic unit which can serve as subject, object or oblique argument of a verb or as predicative complement of a nominal sentence. Furthermore, a noun phrase can be used in adverbial and adnominal functions. According to common syntactic models, the head determines the category of the phrase and governs the dependent constituent(s) in the phrase \citep[cf.][57]{nichols1986}. Consequently, the head of a noun phrase is a noun (or a pronoun). Dependent constituents in noun phrases, also called “attributes”, narrow the denotation, i.e.~modify the head noun descriptively. Typical modifiers in noun phrases are “nominal attributes” (or noun phrases), “adjectival attributes“ (or adjective phrases), “adpositional attributes” (or adposition phrases)\is{adnominal modifier!adposition phrase} and “clausal attributes“ (or relative clauses),\is{adnominal modifier!relative clause} as in the following example.\footnote{Possible syntactic dependencies between modifying constituents inside this noun phrase are ignored in this illustrating example.}
%%%
\ea 
$[_\textrm{NP} [_\textrm{PSR}$ her$] [_\textrm{AP}$ brand new$]$ house $[_\textrm{AdP}$ over there$] [_\textrm{Rel}$ which is big$] ]$
\z
%%%
Noun phrases can thus contain simple modifiers, like nouns or adjectives, or more complex types of modifiers which are complex phrases themselves: for instance (possessor) noun phrases (\textit{my}), adjective phrases (\textit{brand new}), adposition phrases\is{adnominal modifier!adposition phrase} (\textit{in the village}) or relative clauses\is{adnominal modifier!relative clause} (\textit{which was expensive}).

\section{Adjectival modifiers}
%%%
This book presents a cross-linguistic comparison of “adjectival attributes”, or \emph{attributive adjectives}. It investigates the syntactic and morphosyntactic behavior of adjectives inside noun phrases, in particular how they are formally licensed as dependent constituents in noun phrases.

The notion “adjective” needs some clarification because adjectives do not constitute a universal syntactic category. Whereas in some languages adjectives form a distinct word class, in other languages adjectives may not be clearly distinguishable from other parts of speech and constitute a flexible category together with nouns or with verbs. In a third group of languages, adjectives do not exist as a distinct word class et all.

For the survey of languages considered in this investigation, the term \emph{adjective} had thus to be defined in a purely semantic sense, as words with a lexical meaning referring to properties or qualities such as ‘high’, ‘beautiful’, ‘red’, etc. “Qualifying modifiers” \citep[100, elsewhere]{rijkhoff2002} in this broad sense are all lexical elements specifying properties of their referents. This definition excludes possessive pronouns,\is{adnominal modifier!possessor pronoun} demonstratives,\is{adnominal modifier!demonstrative} numerals,\is{adnominal modifier!numeral} and words meaning ‘other’, all of which may behave syntactically like adjectival modifiers in several languages. On the other hand, the semantic definition of adjectives includes adjectival nouns and adjectival verbs (cf.~“nouny” and “verby” adjectives in \citet[25–34, elsewhere]{wetzer1996}) and even qualifying modifiers which are true verbs or true nouns in some languages.

Even though adjectives do not constitute a universal syntactic category, almost all languages seem to exhibit some type of modifier construction in the noun phrase to specify qualitative properties. Hixkaryana,\il{Hixkaryana} a Carib language spoken in Brazil, however, has been mentioned as a counterexample because qualitative properties are only expressed in predicative constructions (\citealt[37, 131]{derbyshire1979}; \citealt[138]{rijkhoff2002}).
%%%
\begin{figure}
\begin{tabular}{|| ll || c | c | c ||}
\hline\hline
Type~1 languages &(Flexible)&\multicolumn{3}{c||}{V / N / A}\\
\hline
Type~2 languages&(Flexible)&V&\multicolumn{2}{c||}{N / A}\\
\hline
Type~3 languages&(Differentiated)&V&N&A\\
\hline
Type~4 languages&(Rigid)&V&\multicolumn{2}{c||}{N}\\
\hline
Type~5 languages&(Rigid)&\multicolumn{3}{c||}{V}\\
\hline\hline
\end{tabular}
\caption[Parts-of-speech systems]{Parts-of-speech systems \citep[based on][]{hengeveld-etal2004}}
\label{hengeveld adj}
\end{figure}
%%%
If a language does not exhibit a distinct class of adjectives, inherent properties of the referent are most often expressed by other lexical means, for example by a relative clause\is{adnominal modifier!relative clause} (headed by a finite stative\is{stative verb} or descriptive verb) used as an adnominal modifier or by a qualifying noun phrase (headed by an abstract, property marking noun) as adnominal modifiers \citep[cf.][100]{rijkhoff2002}.

Similar to \citet{hengeveld-etal2004}, the present study is based on the characterization of adjectives as semantic predicates which can be used as modifiers of nouns without further (derivational) operations. A typology of parts-of-speech systems is illustrated in Figure~\ref{hengeveld adj}.

In the “flexible” language types 1–2 in Figure~\ref{hengeveld adj}, certain classes of lexemes can occur in more than one function (as verbs/nouns/adjectives in Type~1 or as nouns/adjectives in Type~2). In the “differentiated” type of languages, on the other hand, the various classes of lexemes are strictly divided according to their function and constitute a tripartite system of lexeme classes with verbs\slash{}nouns\slash{}adjectives (Type~3). The “rigid” types of languages exhibit either a bipartite system with verbs/nouns (Type~4) or a system exhibiting only one class of lexemes: verbs (Type~5).\footnote{The classification of \citet{hengeveld-etal2004} has seven types because the authors also include manner adverbs as a distinct class. According to the original classification, Type~3 in Table~\ref{hengeveld adj} should thus be divided further yielding the three subtypes V–N–A/Adv (flexible), V–N–A–Adv (rigid) and V–N–A (rigid).}

Most northern Eurasian languages belong to a type of language which exhibits a distinct class of adjectives, whether flexible or rigid (and whether this class is open or closed and counts only very few lexemes). Languages spoken on the European subcontinent predominantly belong to Type~3 and exhibit adjectives as a distinct major class. Most Indo-European\il{Indo-European languages} languages of northern Eurasia belong to this type, but also Basque\il{Basque}, the Uralic\il{Uralic languages} languages of Europe and most languages belonging to one of the three Caucasian\il{Caucasian languages} language families.

Type~2 languages with a flexible class of “noun-adjectives” are also well represented in northern Eurasia. In practically all Mongolic\il{Mongolic languages}, Tungusic\il{Tungusic languages} and Turkic\il{Turkic languages} languages, for example, there is usually no sharp distinction between adjectives and nouns (\citealt[122–123]{rijkhoff2002}; \citealt[9]{poppe1964}).

Type~4 languages lacking a flexible or distinct class of adjectives are represented, for example, by Ainu\il{Ainu}, Korean\il{Korean} and Nivkh\il{Nivkh}. In these languages, verbs are normally employed as qualifying adnominal modifiers.

Languages of Type~1 (with a flexible class of “verb-adjectives”) or 5 (exhibiting exclusively verbs) are not represented in the northern Eurasian area.

\section{Syntax of adjectival modification}
%%%
The present book deals with noun phrases in which adjectives occur as attributes. Predicative adjectives are not dealt with systematically,\footnote{A typology of adjective predication is \citet{wetzer1996}.} although in some cases attributive and predicative adjectives will be contrasted to each other, especially if the languages in question code them differently. The main question to answer with my investigation is how different languages license the syntactic position of adjectival modifiers inside noun phrases, i.e.~what grammatical devices are used for the encoding of the syntactic relationship between an adjectival dependent and its head noun. 

\subsection{Noun phrase internal syntax}
%%%
The syntactic relationship between noun phrase constituents can be encoded by means of purely syntactic structures, i.e.~simply stringing together constituents, or by adding syntactic or morphological devices. 

\il{English|(}
The adjective can take up the modifier slot in the noun phrase without further syntactic or morphological marking taking place inside the noun phrase. Such syntactic licensing means that the relationship between dependent and head is encoded purely structurally in terms of designated positions. An instance of purely syntactic licensing are noun phrases with adjectival modifiers in English. The adjective obligatorily precedes the noun but is not marked otherwise.
%%%
\ea 
\langinfo{English}{Indo-European}{personal knowledge}\\
large houses
\z
%%%
An example of a syntactic device is the dummy head \textit{one} in English which occurs obligatorily in noun phrases without lexical heads.
%%%
\ea
\langinfo{English}{Indo-European}{personal knowledge}\\
\ea
\gll 	a large one\\
   	\textsc{indef} large $_\textrm{HEAD}$\textsc{:sg}\\
\glt ‘a large one’
\ex
\gll	large ones\\
    	large $_\textrm{HEAD}$\textsc{:pl}\\
\glt ‘large ones’
\z
\z
%%%
The dummy head \textit{one} is a noun phrase constituent itself, hence a true syntactic attribution marking device, even though morphology is also involved in this syntactic structure because \textit{one} is inflected for number. The difference between covert and overt syntactic attribution marking devices can also be illustrated with different relative clauses in English.
%%%
\ea
\langinfo{English}{Indo-European}{personal knowledge}\\
\ea $[_\textrm{NP}$ the house $[_\textrm{REL}$ I built$] ]$ \label{engrelap}
\ex $[_\textrm{NP}$ the house $[_\textrm{REL}$ that I built$] ]$ \label{engrel1}
\ea $[_\textrm{NP}$ the man $[_\textrm{REL}$ who$_\textrm{nom}$ built a house$] ]$ \label{engrel2a}
\ex $[_\textrm{NP}$ the man $[_\textrm{REL}$ whose$_\textrm{gen}$ house was built$] ]$ \label{engrel2b}
\z
\z
\z
%%%
Whereas (\ref{engrelap}) exemplifies a covert syntactic device because the relative clause is simply juxtaposed, (\ref{engrel1}) is an overt syntactic device because the the relative clause is marked by an invariable formative. In (\ref{engrel2a}+\ref{engrel2b}), the relativizer \textit{who} is also an overt syntactic device. But in the marking of this relative clause construction, morphology is involved too because the relativizer inflects for case according to the semantic role of the relativized noun.

\il{German|(}
Morphological attribution marking devices are either overt (linear or else) morphemes bound to constituents or covert morphological processes, like incorporation.\footnote{Morphological attribution marking devices can also attach to complex constituents, as the possessor marking \isi{clitic}s in English or Swedish\il{Swedish} which attach to noun phrases: Swedish\il{Swedish} $[_\textrm{NP} [_\textrm{NP}$ kungen$]$=s rike$]$ the\_king=\textsc{poss} empire ‘the empire of the king’, $[_\textrm{NP} [_\textrm{NP}$ kungen av Sverige$]$=s rike$]$ the\_king of Sweden=\textsc{poss} empire ‘the empire of the King of Sweden’.} A prototypical instance of a morphological adjective attribution marking device is agreement inflection, as in German.
\il{English|)}
%%%
\ea
\langinfo{German}{Indo-European}{personal knowledge}\\
\gll	groß-e Häus-er\\
	big-\textsc{pl} house-\textsc{pl}\\
\glt	‘large houses’
\z
%%%
Agreement inflection of attributive adjectives in German is a morphological device, it exists only because syntax requires it,  hence a morphosyntactic device. Other morphological marking in German occurs on syntactic units or on constituents of syntactic units without belonging to morphosyntax. For instance, the plural inflection on the head noun (\textit{Häus-er}) or the inflectional circumfix yielding a participle (\textit{ge-bau-t}) in (\ref{german morphology}) belongs exclusively to the level of (inflectional and derivational) morphology but not to syntax.
%%%
\ea\label{german morphology}
\langinfo{German}{Indo-European}{personal knowledge}\\
\gll	\textbf{ge-bau-t}-e Häus-er\\
	\textbf{\textsc{ptcp}-build-\textsc{ptcp}}-\textsc{pl} house-\textsc{pl}\\
\glt	‘built houses’
\z
%%%
Note that adjectives have been characterized as predicates which can be used as modifiers of nouns without further (derivational) operations. Consequently, the German participle stem \textit{gebaut} ($\leftarrow$ \textit{bauen + ge- … -t}) is an adjective in this broad sense. Syntactically, the participle behaves like a true adjective and takes similar attribution marking. The attribution marking device (i.e.~the agreement inflection) attaches to the participle stem as such (marked with parentheses in \ref{german morphology}). The participle inflection of the verb root \textit{bau-} yielding this new stem does not belong to the sphere of syntax. Similarly, category-changing derivational morphology in other languages yielding, for example, a \isi{stative verb} or a participle function, is not considered to be morphological licensing of adjectival modification.

\subsection{Headless noun phrases}
%%%
Adjectives as well as various other modifiers can also occur in noun phrases without a noun. Normally, this is the case with adjectives in elliptical constructions or adjectives which are made to nouns by means of a derivational process (“substantivized”). In many languages, noun phrases with and without an overtly expressed head noun exhibit a similar phrase structure, as in the following examples from German.
%%%
\ea \label{germ headhadless}
\langinfo{German}{Indo-European}{personal knowledge}\\
\z
\parbox[t]{2.4in}{a.~\textit{ganz neue Häuser}}
\parbox[t]{2.3in}{b.~\textit{ganz neue} (viz.~\textit{Häuser})}\\

\parbox[t]{2.4in}{~\Tree 	
[.NP 
	[.AP	[.Deg	[.ganz very ] ] 
		[.A 		[.neu-e new-\textsc{agr} ] ] ]
	[.N	[  		[.Häus-er house-\textsc{infl} ] ] ] ] 
}
\parbox[t]{2.3in}{~\Tree 
[.NP 
	[.AP	[.Deg 	[.ganz very ] ] 
		[.A 		[.neu-e new-\textsc{agr} ] ] ] 
	[.N 	[		[	[.Ø ] ] ] ] ]
}
\todo{LATEX: check spacing after trees}

\parbox[t]{2.4in}{~\Tree
[.NP 
	[.AP	[.Deg	[.ganz very ] ] 
		[.A 		[.neu-e new-\textsc{agr} ] ] ]
	[.N	[  		[.Häus-er house-\textsc{infl} ] ] ] ] 
}
\parbox[t]{2.3in}{~\Tree
[.NP 
	[.AP	[.Deg 	[.ganz very ] ] 
		[.A 		[.neu-e new-\textsc{agr} ] ] ] 
	[.N 	[		[ 		[.Ø ] ] ] ] ]
}
\todo{LATEX: check spacing after trees}

\noindent The syntactic structure of the two examples in (\ref{germ headhadless}) is principally identical except for the missing head noun ‘house’ with its morphological plural marking in the second structure. The attributive adjective ‘new’ is marked for the same morphosyntactic agreement features in both examples. Even though the adjective in the headless phrase is semantically a noun and used referentially, it is still syntactically the modifier of the (elliptic) noun ‘house’. The syntactic status of the modifier as head of an adjective phrase is indicated by its ability to take dependents, such as the degree word ‘very’. German thus allows the syntactic head position to remain empty in elliptical constructions.
\il{German|)}

\il{Kildin Saami|(}
In other languages, accepting an empty head position in the (elliptical) noun phrase seems less straightforward. In Kildin Saami, for example, nouns and adjectives share identical inflection paradigms. As modifiers of nouns, however, adjectives are not inflected but are simply juxtaposed,\footnote{This is true only for one class of adjectives. Other adjective classes show different morphosyntactic behavior, see \S~\ref{saami synchr} below.} as in (\ref{kildin uninfl nom}) and (\ref{kildin uninfl loc}). Only when attributive adjectives occur in elliptical noun phrases are they inflected identically to nouns, as in (\ref{kildin infl nom}) and (\ref{kildin infl loc}).\footnote{The stem alternation in the adjective \textit{odt : od-} is due to a regular morpho-phonological process.}
%%%
\ea
\langinfo{Kildin Saami}{Uralic}{personal knowledge}\\
\ea \label{kildin uninfl nom}
\gll	čofta odt pērrht\\
	very new house(\textsc{nom:sg})\\
\glt	‘a very new house’
\ex \label{kildin uninfl loc}
\gll	čofta odt pērht-es't\\
	very new house-\textsc{loc:sg}\\
\glt	‘in a very new house’
\ex \label{kildin infl nom}
\gll	čofta odt 				\rm{(viz.~}pērrht\rm{)}\\
	very new(\textsc{nom:sg})\\
\glt 	‘a very new one’
\ex \label{kildin infl loc}
\gll	čofta od-es't 			\rm{(viz.~}pērht-es't\rm{)}\\
	very new-\textsc{loc:sg}\\
\glt 	‘in a very new one’
\z
\z
%%%
If the elliptical construction in Kildin Saami is analyzed as having an empty syntactic head position, as in German,\il{German} an explanation for the different behavior of the (nominal) case inflection is needed. Unlike in German,\il{German} where (nominal) inflection is always bound to the noun, inflection in Kildin Saami can occur bound to nouns or adjectives. Case marking in Kildin Saami could thus be analyzed as \isi{clitic} and bound to the whole noun phrase and hence showing up on the rightmost phrase constituent.
%%%
\begin{exe}
\ex 
\langinfo{Kildin Saami}{Uralic}{personal knowledge}
\end{exe}
\parbox[t]{.45\textwidth}{
a.~\Tree 
[.{NP} 
	[.{AP}	[.{Deg}	[.{čofta} very ] ] 
			[.{A}		[.{odt} {new} ] ] ] 
	[.{N} 		[		[.{pērrht} {house} ] ] ] ]
}
\parbox[t]{.45\textwidth}{
b.~\Tree 
[.{NP} 
	[.{AP} 	[.{Deg} 	[.{čofta} very ] ] 
			[.{A} 		[.{odt} {new} ] ] 
	] 
	[.{N} 		[		[.{pērht} {house} ] ] ]
	[.{CASE}	[		[.{=es't} {=\textsc{infl}} ] ] ]
]
}

\parbox[t]{.45\textwidth}{
c.~\Tree 
[.{NP} 
	[.{AP} 	[.{Deg} 	[.{čofta} very ] ] 
			[.{A} 		[.{odt} {new} ] ] ] 
	[.{N} 		[		[.{Ø}		[.{Ø} ] ] ] ] ]
}
\parbox[t]{.45\textwidth}{
d.~\Tree 
[.{NP} 
	[.{AP} 	[.{Deg} 	[.{čofta} very ] ] 
			[.{A} 		[.{od} {new} ] ] 
	] 
	[.{N} 		[		[.{Ø}		[.{Ø} ] ] ] ]
	[.{CASE}	[		[.{=es't} {=\textsc{infl}} ] ] ]
]
}
\il{Kildin Saami|)}

\il{English|(}
\noindent Another type of language in which elliptical noun phrases behave differently is exemplified by English. In elliptic constructions, attributive adjectives are obligatorily marked with the marker \textit{one}. This marker is exclusively used in headless noun phrases with adjectival (and some other) modifiers. It never occurs if the head noun is overtly expressed. 
%%%
\begin{exe}
\ex 
\langinfo{English}{Indo-European}{personal knowledge}
\end{exe}
\parbox[t]{.45\textwidth}{a. very new houses}
\parbox[t]{.45\textwidth}{b. very new ones (viz.~houses)}\\

\parbox[t]{.45\textwidth}{
\begin{tikzpicture}
\tikzset{level 1+/.style={level distance=2\baselineskip}}
\tikzset{frontier/.style={distance from root=6\baselineskip}}
\Tree 
[.{NP} 
	[.{AP} 
		[.{Deg} {very} ] 
		[.{A} {new} ] 
	] 
	[.{N} [.{house-s} {house-\textsc{infl}} ] ] 
]
\end{tikzpicture}
}
\parbox[t]{.45\textwidth}{
\begin{tikzpicture}
\tikzset{level 1+/.style={level distance=2\baselineskip}}
\tikzset{frontier/.style={distance from root=6\baselineskip}}
\Tree [.{NP} [.{AP} [.{Deg} {very} ] [.{A} {new} ] ] [.{N} {one-\textsc{infl}} ] ]
\end{tikzpicture}
}
\todo{LATEX: solid line\\check spacing after trees}

\noindent Being a grammatical word, hence a constituent in the phrase structure, the marker \textit{one} in English is sometimes described as “dummy head” \citep[cf., e.g.][23]{rijkhoff2002} replacing the noun at the syntactic head position. Consequently, it could be argued that the syntactic head position is never empty in English.
\il{English|)}

\subsection{Appositional modification} \label{apposition}
%%%
\is{modification marking!appositional|(}
Apposition\footnote{Note the different meaning of “juxtaposition”, which is defined as a distinct functional type in \S~\ref{juxtaposition}.} is commonly described as a sequence of two (or more) co-referential constituents on the same syntactic level and hence with the same syntactic function, as in the following expression.
%%%
\ea
$(_\textrm{np} [_\textrm{NP}$ Alma and Iva$] [_\textrm{NP}$ my daughters$] )$ are in this picture. \label{almaiva}
\z\todo{np -> NP?\\Micha: no! I have explained in the text what "np" vs "NP" means here}
%%%
Syntactically, the two independent noun phrases \textit{Alma and Iva, my daughters} together serve as one argument phrase in (\ref{almaiva}).\footnote{The notation of the appositional unit in round brackets is borrowed from \citet[21]{rijkhoff2002}.} In other words, apposition can be defined as a single semantic phrase which consists of several independent syntactic phrases which together serve one syntactic function.

\emph{Appositional modification} differs from true apposition in that the apposed constituent phrase is semantically and syntactically dependent on the other constituent phrase. Similar to the definition presented in \citet[22]{rijkhoff2002}, appositional (noun) modification is here understood as a construction in which the dependent constituent is not part of the (integral) phrase headed by the modified noun. Semantically, the appositional modifier is headed by the modified noun. Syntactically, however, the appositional modifier has an empty head which is co-referential with the head noun of the apposed noun phrase.

\il{Georgian|(}
Appositional modification seems to occur as a secondary marked type of adjective attribution marking in several languages, for instance in Georgian. Attributive adjectives are normally preposed and show only limited agreement (see~\ref{georgian unmarked1}). In postposition (marking emphasis), however, the adjective inflects for the full set of cases and numbers (\ref{georgian marked1}). This construction thus resembles an independent (headless) noun phrase in apposition to the semantic head \citep[652, 677]{testelec1998}; cf.~also \S~\ref{georgian synchr} below.
%%%
\ea
\langinfo{Georgian}{Kartvelian}{\citealt[652]{testelec1998}}\\
\ea \label{georgian unmarked1}
\gll	am or \textbf{lamaz} kal-s\\
	that:\textsc{obl} two nice:\textsc{obl} woman-\textsc{dat}\\
\glt	‘to those two nice women’
\ex \label{georgian marked1}
\gll	kal-eb-s \textbf{lamaz-eb-s}\\
	woman-\textsc{pl}-\textsc{dat} nice-\textsc{pl}-\textsc{dat}\\
\glt	‘to the NICE women’
\z
\z
%%%
\il{Bulgarian|(}
Even without differentiated attribution marking, constituent order change between attribute and head can indicate apposition, as in Bulgarian. Note that the constituent order in noun phrases of Bulgarian is strictly head-final. In poetic language, however, it is possible to move the adjective after the noun.
%%%
\ea
\langinfo{Bulgarian}{Indo-European}{personal knowledge}\\
\ea
\gll	tezi \textbf{golem-i} gradove\\
	these big-\textsc{pl} towns\\
\glt	‘these big towns’
\ex	\label{bulgarian marked}
\gll	tezi gradove \textbf{golem-i}\\
	these towns big-\textsc{pl}\\
\glt	‘these big towns’
\z
\z
%%%
It seems impossible to prove that Bulgarian presents an example of appositional modification. The emphasized noun phrase in (\ref{bulgarian marked}) could simply be analyzed as integral noun phrase differentiated from other non-emphasized noun phrases by constituent order. Georgian, however, is different from Bulgarian. The emphasized noun phrase in (\ref{georgian marked1}) exhibits different morphosyntactic marking due to the additional agreement features (Georgian) and is very likely to be analyzed as an attributive appositional construction.
\il{Bulgarian|)}\il{Georgian|)}

Evidence for appositional modification as a syntactically distinguished noun phrase type is also found in constructions were the apposed headless noun phrase is overtly marked by means of \isi{attributive nominalization} (see \S \ref{attr nmlz}). Attributive nominalization can be illustrated with the epithet construction in German.
%%%
\ea 
\langinfo{German}{Indo-European}{personal knowledge}\\
$[_\textrm{NP}$ Friedrich $[_\textrm{NP}$ der Gro{ß}e$] ]$ \textrm{‘Frederick the Great’}
\z
\is{modification marking!appositional|)}