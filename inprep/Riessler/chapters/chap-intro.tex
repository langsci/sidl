
\chapter{Introduction}
%%%
\subsection*{Aim}\todo[inline]{should this be section 1.1, etc? Micha: if you prefer, but I personally would rather like to keep it like this}
%%%
The aim of this investigation is to typologize adjective attribution marking devices in the languages of northern Eurasia. Agreement and construct state marking are commonly known morphological devices for the licensing of adjectival modifiers; an example of a purely syntactic device is juxtaposition.  

The main parts of this thesis include an ontological classification of all attested devices in the geographic area of investigation and a survey of adjective attribution marking devices occurring across the northern Eurasian language families. Finally, several attested scenarios for the evolution of adjective attribution marking devices in languages of northern Eurasia are discussed.

\subsection*{Question}
%%%
The most central questions dealt with in this investigation regard the formal licensing of the syntactic relation between a head noun and its adjectival dependent inside a noun phrase:
%%%
\begin{itemize}
\item What syntactic, morphological or other adjective attribution marking devices are available in languages? 
\item How can these devices be systematically described and typologized? 
\item How is the occurrence of the different types distributed geographically? 
\item How does attribution marking arise and diffuse across languages?
\end{itemize}

\subsection*{Method}
%%%
The present study is the result of empirical research based on data from grammatical descriptions on the investigated languages. It follows a data-driven, bottom-up and framework-neutral approach (\citealt[cf.][]{haspelmath2010} and also the method of “Autotypology”\is{Autotypology} following \citealt{bickel-etal2002} and \citealt{bickel2007}).

The method of sampling and mapping of data is inspired by the \emph{AUTOTYP}\footnote{Cf.~\url{http://www.spw.uzh.ch/autotyp/} 16.02.2014}\is{Autotypology!AUTOTYP project} and \emph{Eurotyp}\footnote{Cf.~\url{http://www.degruyter.com/view/serial/16329} 16.02.2014}\is{EUROTYP project} research programs as well as the \emph{WALS} project \citep{walsOnline2013}.\is{WALS database} The approach presented here is closer to \emph{Eurotyp}\is{EUROTYP project} than to \emph{WALS}\is{WALS database} or \emph{AUTOTYP}\is{Autotypology!AUTOTYP project} in coding as many different genera from the geographic area of investigation as possible.

\subsection*{Content}
%%%
The book is divided into four main parts. In Part~\ref{part:theory} “Theoretical preliminaries”, a few basic comparative concepts relevant to a framework-neutral description of a noun phrase and its constituents are introduced. This part also discusses the syntax-morphology interface in noun-phrase structure which is of central importance for the present study.

Part~\ref{part typ} “Typology” presents a general ontology of adjective attribution marking devices based on data from northern Eurasian and other languages.

In Part~\ref{part synchr} “Synchrony”, a synchronic-typological survey of noun phrase structure with attributive adjectives in northern Eurasia is presented and exemplified with data from all genera of the area.

The book's last Part~\ref{part diachr} “Diachrony” is devoted to the evolution of adjective attribution marking devices. It describes several different paths of evolving and abolishing adjective attribution marking devices in northern Eurasian languages.
