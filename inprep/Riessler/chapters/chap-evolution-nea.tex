%??Section 11.1.2.1 on the evolution of anti-construct state agreement marking in Baltic and Slavic striked me as highly speculative; it does not include a single example from real texts from the earlier stages of Baltic or Slavic! The same observation pertains to the following section on Germanic. In my view, this largely invalidates the whole discussion. In particular, the empirical justification for the following claim on p. 248: “The three most important results of this study are ... (3) the diachronic attestation of contrastive-focus constructions with phrasally embedded adjectival modifiers as a common source of innovative adjective attribution marking devices in the northern Eurasian languages” — is not provided. %??By the way, on the history of “long” versus “short” adjectives in Russian, see works by Karin Larsen (2005, 2006, 2007).

\is{grammaticalization|(}
\chapter[The evolution of attribution marking]{The evolution of attribution marking in northern Eurasian languages}
%%%
Attribution marking devices were typologized in Part~II (Typology) and their geographic distribution across the genealogical entities of northern Eurasia was presented in Part~III (Synchrony). The present, diachronic part focuses on linguistic changes which led to the emergence of the attested synchronic diversity within the northern Eurasian area.

Not all attested changes are investigated in equal depth in each genealogical unit. Special focus lies on the grammaticalization of attributive markers from attributive nominalizers in the Saamic \il{Saamic languages} and Finnic\il{Finnic languages} branches of Uralic as well as in the Baltic,\il{Baltic languages} Slavic\il{Slavic languages} and Germanic\il{Germanic languages} branches of Indo-European. Different types of adjective attribution marking have been grammaticalized from attributive nominalizers\is{attributive nominalization} in different languages of the area and during different periods of time. Up to now, these diachronic patterns have not been systematically investigated from a cross-linguistic perspective.

The parallel evolution of attributive nominalizers and other adjective attribution marking devices is interesting not only from a general typological perspective. The linguistic interference zone between Uralic and Indo-European in northeastern Europe exhibits a relatively high degree of diversity from a synchronic point of view (see \S\ref{areality}). Consequently, it appears that the synchronically and diachronically attested developments have to be described in areal linguistic terms and provide further evidence for establishing a Northern European \textit{Sprachbund}.\is{Sprachbund}

\is{attributive nominalization|(}
\section[Attributive nominalizers]{The emergence of attributive nominalizers}
%%%
Attributive nominalization as a special subtype of dependent marking attributive state (see \S\ref{attr nmlz}) is not synchronically attested as a default licenser of the attributive connection of adjectives in any language of northern Eurasia. However, in several languages of the area, attributive constructions with nominalizers constitute a special type of noun phrases characterized earlier as attributive apposition. A typical example is \ili{Udmurt} (Uralic) where an adjectival attribute equipped with an article is marked for contrastive focus (see \S\ref{udmurt synchr}).

The only two Northern Eurasian languages exhibiting attributive nominalization as a default attribution marking device synchronically are \ili{Albanian} proper and \ili{Arvanitika} from the Albanian\il{Albanian languages} branch (Indo-European). The marker, however, is used only in a circumfixed\is{position!circumfixed} construction together with the inherited \isi{head\hyp{}driven agreement}.

Attributive nominalizers are also documented in historical stages of several Indo-European branches, such as Baltic,\il{Baltic languages} Slavic\il{Slavic languages} and Germanic.\il{Germanic languages} But even here, these markers are not the default devices. Instead, attributive articles compete with other attributive markers and are restricted to emphatically marked noun phrases. In several of these \ili{Indo-European languages}, however, the articles have evolved into new default types of attribution marking. A prototypical example of attribution marking originating from an attributive article is anti\hyp{}construct state agreement marking in \ili{Russian} (see \S\ref{russian synchr}). In other languages, the former attributive article is still traceable as a secondary type of attribution marking, as in the modern \ili{Baltic languages}. Here, the attributive article also evolved into an anti\hyp{}construct state agreement marker but it is still restricted to a semantically defined subset of noun phrases (see \S\ref{baltic synchr}). 

\is{attributive article|(}
\ia{Himmelmann, Nikolaus|(}
The synchrony and diachrony of attributive articles have also been dealt with in a cross-linguistic investigation of grammaticalized adnominal D(eictic) elements by \cite{himmelmann1997}. Himmelmann assumes that attributive articles (“linking articles” in his terminology) originally occurred in appositional nominal expressions. These “linking constructions” are characterized as complex noun phrases in which the attribute occurs as a syntactically independent nominal expression. The “linking article” (i.e., \textit{attributive article} in terms of the present typology) serves as a nominalizer and licenses the attribute as a syntagma of its own \cite[188]{himmelmann1997}.
\is{attributive article|)}

The diachronic data from several Indo-European,\il{Indo-European languages} Uralic\il{Uralic languages} and \ili{Turkic languages} presented in the following sections support Himmelmann's conclusions about a common source of attributive marking originating from pronouns or other deictic elements used as attributive nominalizers.
\ia{Himmelmann, Nikolaus|)}

\il{Uralic languages|(}
\il{Turkic languages|(}
\is{juxtaposition|(}
\subsection{Attributive nominalizers in Uralic and Turkic}
\label{uralic-turkic diachr}
%%%
Juxtaposition has been the prototype of adjective attributive marking in all Uralic and Turkic languages since the proto-stages of these languages (cf.~\citealt[80–81]{decsy1990} for Uralic and \citealt[75–76]{decsy1998} for Turkic). However, as the result of a secondary development in some branches of Uralic and Turkic, an attributive nominalizer grammaticalized. Synchronically, it occurs as minor attribution marking device in specially marked noun phrase types in several languages of these two families.%{decsy1998}sagt eigentlich nichts so richtig zu Kongruenz

In the Saamic\il{Saamic languages} and Finnic\il{Finnic languages} branches of Uralic, juxtaposition has been replaced completely by new adjective attribution marking devices. In \ili{Proto\hyp{}Saamic} the prototypical attributive connector of adjectives was probably anti\hyp{}construct state marking. A comparison of synchronic evidence across modern Saamic languages makes this reconstruction very likely \citep{riesler2006b}. However, the modern \ili{Saamic languages} show a strong tendency to abandon the anti\hyp{}construct state marker and re-introduce the morphologically unmarked adjective attribution marking device juxtaposition. In \ili{Proto\hyp{}Finnic}, the original Uralic type has also been lost and has now been replaced by \isi{head\hyp{}driven agreement} marking of attributive adjectives. In \S\ref{Finnic diachr} and \ref{saamic diachr}, the emergence of agreement in Finnic and anti\hyp{}construct state marking in Saamic will be explored and described as a possible result of the grammaticalization of attributive nominalizers.
\is{juxtaposition|)}

\il{Udmurt|(}
Since the emergence of attributive nominalizers in Udmurt (and other modern Uralic languages) probably reflects structurally similar stages of development as those assumed for \ili{Proto\hyp{}Saamic} and \ili{Proto\hyp{}Finnic}, the Udmurt case will be described in depth in the following sections.

\subsubsection{The contrastive focus marker in Udmurt}
\label{udmurt diachr}
%%%
Synchronic data from Udmurt illustrates the emergence of an attributive article and might even indicate how this attribution marker has been generalized as an anti\hyp{}construct state marker. 

The use of the 3\textsuperscript{rd} person possessive suffix as a contrastive focus marker in Udmurt was exemplified in \S\ref{udmurt synchr} on the synchrony of attribution marking in Permic.\il{Permic languages} In the following sections, the etymological source and the evolution of this contrastive focus construction will be illustrated with the help of further examples.

As in several other Uralic languages, the possessive suffix 3\textsuperscript{rd} person singular in Udmurt is often used as a definite-like marker. Grammatical descriptions of Udmurt use different terms to define the function of this formative, for example as “determinative” \citep{kelmakov-etal1999}, “contrastive-deictic” \citep{alatyrev1970}, “anaphorical-emphasizing” \citep{kiekbaev1965}, or simply “definite” \citep{winkler2001}. The suffix is characterized in the following as “quasi-definite” since Udmurt (like most other Uralic languages) has no morphologized feature \textsc{species}. The use of the marker is obviously determined by the referential status of the noun phrase, but it does not occur obligatorily in definite noun phrases. Since the rules for definiteness marking are not the subject of the present investigation, the formative in definite-like constructions will simply be referred to as \textit{determinative suffix}, which is also consistent with some of the grammatical descriptions mentioned above (e.g., \citealt{kelmakov-etal1999}).\is{species marking!definite}

Besides its function as a possessive marker, the 3\textsuperscript{rd} person singular possessive suffix occurs not only in quasi-definite noun phrases but is even used as an (attributive) nominalizer and as a marker of contrastive focus on adjectives. From a synchronic point of view, the functions of \textsc{poss:3sg} in the different non-possessive uses are probably better analyzed as belonging to different grammatical categories. Consequently, different glosses (such as \textsc{poss, def, nmlz, contr}) should be applied. However, in order to illustrate the similar historical source of the synchronically differentiated grammatical meanings one and the same gloss (i.e., \textsc{poss:3sg}) is used in the following examples.
%%%
\begin{exe}
\ex {\rm Possessive and non-possessive functions of (historical) \textsc{poss:3sg}}
\begin{xlist}
\ex {\rm Possessive marking}
\label{udmurt possmarking}
\begin{xlist}
\ex	
\gll	gurt\textbf{-ėz}\\
	house-\textsc{poss:3sg}\\
\glt	‘her/his/its house’
\ex	
\gll	gurt-jos-a\textbf{-z}\\
	house-\textsc{pl}-\textsc{ill}-\textsc{poss:3sg}\\
\glt	‘into her/his/its houses’
\end{xlist}
\ex {\rm “Determinative” marking}
\begin{xlist}
\ex	
\gll	gurt\textbf{-ėz}\\
	house-\textsc{poss:3sg}\\
\glt	‘this house’
\ex	
\gll	gurt-jos-a\textbf{-z}\\
	house-\textsc{pl}-\textsc{ill}-\textsc{poss:3sg}\\
\glt	‘into these houses’
\end{xlist}
\ex	{\rm Attributive nominalization}
\label{udmurt diachr nomzr}
\begin{xlist}
\ex {\rm Demonstrative}\\
\label{udmurt diachr dem-nomzr}
\gll	ta\textbf{-iz} / so\textbf{-iz}\\
 	\textsc{dem:prox}-\textsc{poss:3sg} {} \textsc{dem:dist}-\textsc{poss:3sg}\\
\glt	‘this one over here’ / ‘that one over there’
%%%
\ex 	{\rm Possessor noun phrase}\\
\label{udmurt diachr gen-nomzr}
\gll	Ivan-len\textbf{-ėz}\\
	Ivan-\textsc{gen}-\textsc{poss:3sg}\\
\glt	 ‘the one of Ivan (Ivan's)’
\ex 	{\rm Adjective}\\
\label{udmurt diachr adj-nomzr}
\gll	badǯ́ym\textbf{-ėz}\\
	big-\textsc{poss:3sg}\\
\glt	 ‘the big one’
\end{xlist}
\ex {\rm Contrastive focus marking}
\label{udmurt diachr contr}
\begin{xlist}
\ex	
\gll	badǯ́ym\textbf{-ėz} gurt\\
	big-\textsc{poss:3sg} house\\
\glt	‘a/the \textsc{large} house’
\ex	
\gll	badǯ́ym-jos-a\textbf{-z} gurt-jos-y\\
	big-\textsc{pl}-\textsc{ill}-\textsc{poss:3sg} house-\textsc{pl}-\textsc{ill}\\
\glt	‘into (the) \textsc{large} houses’
\end{xlist}
\end{xlist}
\end{exe}
%%%
The use of the suffix \textit{-ėz} as marker of contrastive focus is obviously connected to its other non-possessive functions. The order of examples (\ref{udmurt possmarking}–\ref{udmurt diachr contr}) probably reflects the functional expansion of the original possessive marker to a “determinative” marker on noun phrases and a contrastive focus marker on adjectives. The clue for understanding this development is the use of the suffix \textit{-ėz} as an attributive nominalizer in \isi{headless noun phrase}s, as shown in (\ref{udmurt diachr nomzr}). Here, the determinative suffix is used as a true attributive nominalizer to mark a demonstrative (\ref{udmurt diachr dem-nomzr}), a possessor noun\is{adnominal modifier!possessor noun} (\ref{udmurt diachr gen-nomzr}) or an adjective (\ref{udmurt diachr adj-nomzr}) as modifiers by projecting a full (headless) noun phrase. Note however that headless adjectives, demonstratives,\is{adnominal modifier!demonstrative} and noun possessors (in genitive)\is{adnominal modifier!possessor noun} are not obligatorily marked by means of attributive nominalization in Udmurt. The marker is used in order to emphasize the property denoted by the attribute and to contrast it to other properties of the same set.

The emphasizing function of the determinative suffix, finally, is the link to its use as contrastive focus marker on adjectives. It seems clear that these contrastive focus constructions originate from appositional constructions of nouns with emphasized headless attributes.\is{headless noun phrase}\footnote{The zero-morpheme (equipped with the nominalizer Ø-\textsc{nmlz}) in (\ref{udmurt apposition}) is only presented for a better illustration of the empty head position to which the (nominalized) adjective moves in this appositional noun phrase.}
%%%
\begin{exe}
\ex {\upshape [}\textsubscript{NP} {\upshape [}\textsubscript{NP'} \textsubscript{A}big \textsubscript{HEAD}Ø-\textsc{nmlz}{\upshape ]} \textsubscript{N}house{\upshape ]}
\label{udmurt apposition}
\end{exe}
%%%
The agreement patterns in noun phrases with attributes in contrastive focus provide the best evidence for this assumption. In their default use, attributive adjectives (as well as other modifiers) do not show agreement with the head noun. However, when the attribute is marked for contrastive focus (by means of the attributive nominalizer \textsc{attr} $\Leftarrow$ \textsc{poss:3sg}), case and number marking spread to the adjective.
%%%
\begin{exe}
\ex {\rm Juxtaposition\is{juxtaposition} versus anti\hyp{}construct state agreement marking (i.e., in contrastive focus) \citep{kelmakov-etal1999,winkler2001}}
\begin{xlist}
\ex	{\rm Adjective attribute}
\begin{xlist}
\ex 	
\gll	badǯ́ym / badǯ́ym\textbf{-ėz} gurt\\
	big {} big-\textsc{attr} house\\
\glt	‘large house’ : ‘\textsc{large} house’
\ex	
\gll	badǯ́ym / badǯ́ym\textbf{-jos-a-z} gurt-jos-y\\
	big {} big-\textsc{pl}-\textsc{ill}-\textsc{attr} house-\textsc{pl}-\textsc{ill}\\
\glt	‘to (the) large houses’ : ‘to (the) \textsc{large} houses’
\end{xlist}
%%%
\ex	{\rm Possessor noun attribute}\is{adnominal modifier!possessor noun}\footnote{Note that the cross-referencing possessive agreement marker does not occur with a genitive construction in contrastive focus \citep[81]{kelmakov-etal1999}.}
\begin{xlist}
\ex	
\gll	Ivan-len / Ivan-len\textbf{-ėz} gurt-ėz\\
	Ivan-\textsc{gen} {} Ivan-\textsc{gen}-\textsc{attr} house-\textsc{poss:3sg}\\
\glt	‘Ivan's house’ : ‘\textsc{Ivan's} house’
\ex	
\gll	Ivan-len / Ivan\textbf{-jos-a-z-len} gurt-jos-a-z\\
	Ivan-\textsc{gen} {} Ivan-\textsc{pl}-\textsc{ill}-\textsc{attr}-\textsc{gen} house-\textsc{pl}-\textsc{ill}-\textsc{poss:3sg}\\
\glt	‘to Ivan's houses’ : ‘to \textsc{Ivan's} houses’
\end{xlist}
\ex	{\rm Demonstrative attribute}
\label{udmurt det dem}
\begin{xlist}
\ex	
\gll	so / so\textbf{-iz} gurt\\
 	\textsc{dem:dist} {} \textsc{dem:dist}-\textsc{attr} house\\
\glt	‘that house’ : ‘\textsc{that} house’
\ex 	
\gll	ta / ta\textbf{-os-a-z} gurt-jos-y\\
	\textsc{dem:prox} {} \textsc{dem:prox}-\textsc{pl}-\textsc{ill}-\textsc{attr} house-\textsc{pl}-\textsc{ill}\\
\glt	 ‘to these houses’ : ‘to \textsc{these} houses’
\end{xlist}
\end{xlist}
\end{exe}
%POSS als Kongruenzkategorie
%%%
Following the intuition of the authors of grammatical descriptions of Udmurt, however, one could also analyze these constructions as true noun phrases with a syntactic structure as in (\ref{udmurt notapposition}) (as opposed to \ref{udmurt apposition}) where the original nominalizer of the attribute in the \isi{headless noun phrase} became a dependent marking attributive construct device linking the attribute in contrastive focus to the semantic head ‘house’ in the noun phrase.
%%%
\begin{exe}
\ex[?]{
{\upshape [}\textsubscript{\rm NP} \textsubscript{\rm A}big-\textsc{contr} \textsubscript{\rm HEAD}house{\upshape ]}
}
\label{udmurt notapposition}
\end{exe}
%%%
Even if head\hyp{}driven number and case agreement is involved in attribution marking of adjectives in contrastive focus, Udmurt is better analyzed as a language exhibiting an attributive appositional construction rather than an anti\hyp{}construct state agreement marking. The agreement and anti\hyp{}construct state marking formatives are not fused and agreement marking occurs only indirectly as the result of the nominalization of the appositional headless adjective.\is{headless noun phrase}
\il{Udmurt|)}

\subsubsection[Possessive suffixes as attributive nominalizers]{Possessive suffixes as attributive nominalizers in other Uralic and in Turkic languages}
%%%
Non-possessive uses of 3\textsuperscript{rd} person singular possessive suffixes similar to Udmurt are well attested in several Uralic and Turkic languages.\footnote{In several languages, even 2\textsuperscript{rd} person singular possessive occurs in the same function.} In descriptions of these languages, the marker is often characterized as “emphatic-definite” or simply “definite” (cf.~\citealt[148]{tauli1966}; \citealt{kunnap2004}). But obviously this is greatly oversimplified. It is especially unclear what it would mean to mark an adjectival modifier as “definite”.\is{species marking!definite}

Besides in Udmurt, the use of the (historical) 3\textsuperscript{rd} person singular possessive suffix as a marker of contrastive focus is similarly regular (though less systematically described) in the other \ili{Permic languages} (cf.~\citealt[67]{serebrennikov1963}).%Verweis ist nicht so gut

\il{Mari languages|(}
In the Mari languages, which belong to the Volgaic\il{Volgaic languages} branch of Uralic, the possessive suffix is also commonly used as a determinative suffix for nouns (cf.~\citealt[75–76]{alhoniemi1993}). The regular use of the formative to derive a certain set of “determinative” or contrastive focused demonstratives and quantifiers in Mari (similar to the \ili{Udmurt} example (\ref{udmurt det dem}) on page~\pageref{udmurt det dem}) gives at least some evidence that the Mari languages have (or had) an attributive nominalizer in contrastive focus constructions as well.\footnote{The homophonous focus \isi{clitic} \textit{=že} in Eastern Mari (\textit{təi=že kuze ilaš tüŋalat?} ‘And how are \textsc{you} going to live?’ \citealt[80]{alhoniemi1993}) is most likely not cognate with the 3\textsuperscript{rd} person singular possessive suffix but borrowed from the formally and functionally similar marker focus marker in \ili{Russian}.}%anhand der deklinationsformen zeigen, dass es poss ist und nicht =že, z.B. weil kasus danach folgt
%%%
\begin{exe}
\ex \langinfo{Eastern Mari}{Uralic}{\citealt{alhoniemi1993}}
\begin{xlist}
\ex {\rm “Short” demonstratives (i.e., unmarked)} 
\begin{xlist}
\ex tide {\rm ‘this’ /} tudo {\rm ‘that’ (82)}
\end{xlist}
%%%
\ex {\rm “Long” demonstratives (i.e., in contrastive focus)}
\begin{xlist}
\ex tide\textbf{-že} {\rm ‘this one’ /} tudo\textbf{-že} {\rm ‘that one’ (82)}%Ist S. 82 (und 80) oben richtig?
\end{xlist}
%%%
\ex {\rm Quantifiers in contrastive focus}\\
\gll	Tə̂nar\textbf{-žə̂}-m mə̂j nalam, Tə̂nar\textbf{-žə̂}-m tə̂j.\\
	so.much-\textsc{poss:3sg}-\textsc{acc} I take, so.much-\textsc{poss:3sg}-\textsc{acc} you\\
\glt	‘So much I will take, so much you.’ (76)
\end{xlist}
\end{exe}
%%%
\il{Chuvash|(}
A similar use of the (historical) 3\textsuperscript{rd} person singular possessive suffix as a marker of contrastive focus in the Turkic language Chuvash has been shown in \S\ref{chuvash synchr}. Interestingly, the Turkic language Chuvash and the Uralic languages Eastern\il{Eastern Mari} and \ili{Western Mari} and \ili{Udmurt} are among the core members of the \isi{Volga-Kama area}.\footnote{Other core members of the Volga-Kama \isi{Sprachbund} area are the Turkic languages \ili{Tatar} and \ili{Bashkir}. The Uralic languages Mordvin\il{Mordvin languages} and \ili{Komi-Permyak} are considered peripheral members \citep{helimski2005}.} The languages of this linguistic area show linguistic convergence on several levels of their grammars. In all Uralic and Turkic languages of that area, at least the “emphatic-definite” use of the 3\textsuperscript{rd} person singular possessive suffix is attested. Thus, it cannot be ruled out that the evolving attributive nominalizer in Chuvash, \ili{Udmurt} and the Mari languages has been borrowed in either direction.\is{species marking!definite}%ÿsker erwähnen?
\il{Mari languages|)}

\il{Tungusic languages|(}
The phenomenon might even reflect a much older and more widespread feature of a larger subarea of northern Eurasia including at least Tungusic. As demonstrated in the synchronic \S\ref{tungusic synchr} on Tungusic, similar constructions with the 3\textsuperscript{rd} person singular possessive suffix also seem to regularly occur in this family. Even in other languages of the area, examples of the use of the 3\textsuperscript{rd} person singular possessive suffix as an attributive nominalizer (though not on adjectives) are attested. Example (\ref{mongolian nmlz}) illustrates the use of the 3\textsuperscript{rd} person singular possessive suffix as an attributive nominalizer of pronouns in Khalkha Mongolian.
%%%
\begin{exe}
\ex {\rm Attributive nominalization in \langinfo{Khalkha}{Mongolic}{\citealt[6]{pavlov1985}}}
\label{mongolian nmlz}
\begin{xlist}
\ex	olan ‘much’ – olan\textbf{-ki} ‘what is in majority; the largest part’
\ex	numaj ‘much’ – numajj\textbf{-i} ‘what is in majority; the largest part’
\end{xlist}
\end{exe}
%%%
Not also that the (historical) 3\textsuperscript{rd} person singular possessive suffix occurs in practically all Turkic languages in lexicalized local and temporal attributes. 
%%%
\begin{exe}
\ex {\rm Attributive nominalization in \langinfo{Chuvash}{Turkic}{\citealt[67–68]{benzing1963}}}
\begin{xlist}
\ex
\gll	śul-χi\\
	year-\textsc{loc:poss:3sg}\\
\glt	‘yearly, annual’ (originally ‘what is in a year’)
\ex
\gll	yal-t-i\\
	village-\textsc{loc-poss:3sg}\\
\glt	‘local’ (originally ‘what is in a village’)
\ex
\gll	kil-t-i\\
	home-\textsc{loc-poss:3sg}\\
\glt	‘domestic’ (originally ‘what is in the home’)
\end{xlist}
\end{exe}
%%%
It remains unclear whether the evolution of attributive nominalization and contrastive focus marking of attributive adjectives occurs independently in certain branches or areal groupings across Indo-European, Uralic, Turkic and Tungusic or goes back to a general northern Eurasian areal tendency.
\il{Tungusic languages|)}
\il{Chuvash|)}
\il{Uralic languages|)}
\il{Turkic languages|)}

\il{Indo-European languages|(}
\subsection{Attributive nominalizers in Indo-European}
\label{ie diachr}
%%%

\is{species marking!definite|(}
\il{Baltic languages|(}
\il{Slavic languages|(}
\subsubsection[Baltic and Slavic]{Attributive articles and the emergence of anti\hyp{}construct state agreement marking in Baltic and Slavic}
\label{slavic diachr}
%%%
%The choice of one of these markers versus the other is normally connected to the definiteness or indefiniteness of the noun phrase in the Old Slavic languages. 
%In \ili{Old Bulgarian} attested definite noun phrases in which the adjective is marked with the short-suffix
%restrictions determined by semantic of the noun, referential status of the whole NP, but most clearly by the semantic of the adjective: So bilden Beziehungsadjective generell selten Langformen, die Possessivadjective im besonderen
%kommen ausschließlich in der Kurzform vor
%Another exception are nominalized adjectives in \isi{headless noun phrase}s which occur most often in the long form. (Mendoza)
%It is thus questionable whether or not the original function... really was marking of definiteness
%%Def marker: regular expression of anaphoric definiteness of a noun phrase; has no further functions

\ili{Russian} is the only Slavic language exhibiting anti\hyp{}construct state agreement marking as the default and only type of attributive connection of adjectives (\textit{xorošij} \textsc{attr:nom.m.sg} ‘good’ versus \textit{xoroš} \textsc{pred:nom.m.sg}, see also \S\ref{russian synchr}). The Russian construction where attributive adjectives are obligatorily equipped with special anti\hyp{}construct state agreement suffixes resembles a construction in the closely related Baltic languages. In the latter, however, the occurrence of anti\hyp{}construct state agreement marking is usually described as being restricted to definite noun phrases. The competition between complex attributive agreement and “pure” agreement marking was already characteristic of \ili{Old Baltic languages} (cf.~\ili{Lithuanian} \textit{geràsis} versus \textit{g{\~e}ras}, \ili{Latvian} \textit{labais} versus \textit{labs} ‘good’) and \ili{Old Slavic languages} (cf.~\ili{Old Bulgarian} \textit{dobrъjь} versus \textit{dobrъ} ‘good’). Old Slavic and \ili{Old Baltic languages} are thus similar to modern \ili{Lithuanian} and modern \ili{Latvian} in exhibiting two types of adjective attribution marking suffixes in different functions.

In the Slavic and Indo-European linguistic traditions, adjectives equipped with anti\hyp{}construct state agreement marking are normally referred to as “long-form adjectives” (contrasted to “short-form adjectives”). Other commonly used terms for the anti\hyp{}construct state agreement markers are “pronominal, complex” or “compound” agreement suffixes. Analogically, the two inflectional paradigms of long- versus short-form adjectives equipped with number, gender, and case agreement values are normally labeled in a similar way as “long-form, pronominal, complex, or compound” versus “short-form” adjective declension. Obviously, these terms describe the form or the origin of the formative rather than its function and are rather useless for a typological comparison.

Similar to the modern Baltic languages, the markers are sometimes also labeled “definite” agreement suffixes in Old Slavic.\il{Old Slavic languages} As will be shown below, the notion of “definiteness” does not exactly cover the functionality of the marker in Old Slavic either.

The corresponding attributive constructions in modern Slavic and Baltic languages have already been dealt with in the synchronic part of this investigation (especially \S\S\ref{slavic synchr}, \ref{baltic synchr}). In the present chapter, the origin and development of anti\hyp{}construct state agreement marking in Baltic and Slavic along two possible grammaticalization paths (see~\ref{2paths} below) will be discussed. It will be argued that these constructions have arisen from attributive articles which originally marked contrastive focus of the attribute rather than from nominal relative constructions. Before dealing with the syntactic evolution of the attributive constructions in Slavic and Baltic, however, the etymology of the formative (which is similar for both scenarios) will be sketched in the following short section.

\subsubsection{Etymology of the formative} 
Whereas the “pure” agreement declension (of the so-called short-forms) of adjectives continues the \ili{Proto\hyp{}Indo\hyp{}European} default type of adjective attribution marking, the anti\hyp{}construct (long-form) agreement suffixes, as in \ili{Lithuanian} \textit{geràs-is žmõgus}, \ili{Latvian} \textit{laba-is cilvēks}, or \ili{Old Bulgarian} \textit{dobrъ-jь človekъ} ‘the good person’, arose as a result of a phonological merger between the short-form agreement suffixes of the adjective and a pronominal stem reconstructed as \ili{Proto\hyp{}Baltic\slash{}Slavic} \textit{*-jĭ/jь-}.

This pronominal part of the long-form agreement suffix likely goes back to a pronominal stem reconstructed as \ili{Proto\hyp{}Indo-European} \textit{*i̭o-} \citep[61]{wissemann1958}. The anti\hyp{}construct state agreement marker in Baltic\slash{}Slavic could thus be cognate with relative markers in other Indo-European languages, such as Old Indo-Aryan\il{Old Indo-Aryan languages} \textit{yá-h}, Old Iranian\il{Old Iranian languages} \textit{yō}, or \ili{Ancient Greek} \textit{hós} \cite[53]{heinrichs1954}.

An alternative etymology has been suggested by Mikkola (\citeyear[52]{mikkola1950}; %verweis falsch
 see also \citealt[102]{leskien1871}; \citealt[164–165]{leskien1919}; \citealt[19ff.]{wijk1935}). Mikkola believes that \ili{Proto\hyp{}Baltic\slash{}Slavic} \textit{*-jь-} was an anaphoric marker which goes back to the 3\textsuperscript{rd} person singular pronoun (cf.~\ili{Lithuanian} \textit{jìs}, \textit{jõ} \textsc{3sg:gen} or \ili{Old Bulgarian} \textit{jь}, \textit{jego} \textsc{3sg:gen}). The phonological merger of Indo-European \textit{*is} \textsc{3sg.m} with \textit{\textit{*i̭os}} \textsc{m} ‘which’ in Baltic\slash{}Slavic \cite[21 Footnote 8]{schmidt1959} makes this explanation possible from the point of view of sound correspondence.

The terminus post quem of the innovative attribution marking in Baltic and Slavic can be determined relatively easily. Different phonological and morphological developments of the long-form agreement suffixes in Baltic and Slavic imply that the phonological merger of adjective and the formative \textit{*-jь-} took place independently in Old Slavic\il{Old Slavic languages} and Old Baltic\il{Old Baltic languages} \citep[64–65]{koch1992}. 

It is not certain whether the Baltic and Slavic branches of Indo-European go back to a common proto-form or \ili{Proto\hyp{}Baltic\slash{}Slavic} have to be reconstructed as independent Indo-European daughter languages. If the latter case proves to be right, the rise of anti\hyp{}construct state agreement marking could be parallel, but due to contact in \ili{Proto\hyp{}Baltic\slash{}Slavic} (as stated, for example, by \citealt[77]{pohl1980}). Since the reconstruction of proto-languages is not an aim of this investigation and since the developments in Baltic and Slavic are similar from a chronological, functional and (Indo-European) etymological point of view, discussing the rise of anti\hyp{}construct state agreement marking in Baltic and Slavic together in the same section makes perfect sense.

\subsubsection{Evolution of the construction} 
It is commonly assumed that the function of the long-form suffix on the adjective in Old Baltic\il{Old Baltic languages} and Old Slavic\il{Old Slavic languages} was to mark the noun phrase as definite. This opinion is repeated by practically all authors of comparative grammars and reference books of the Baltic\slash{}Slavic languages as well as in works dealing specifically with adjectives and noun phrase syntax of these languages (cf.~\citealt[211]{mendoza2004} with references).

Definite nouns, however, are not obligatorily modified by long-form adjectives in Old Slavic.\il{Old Slavic languages} Furthermore, nominalized (headless) adjectives\is{headless noun phrase} are normally equipped with long-form suffixes, regardless of the referential status of the noun phrase as definite or indefinite. The analysis of the long-form adjective suffix as definite marker might thus not be as straightforward as it appears in the reference books. 

\citet[214–215]{mendoza2004} connects the original distribution of long- versus short-forms to contrastive focus marking, i.e., the restrictive versus non\hyp{}restrictive semantics of the attribute, instead of the referential status of the modified noun. In a similar way argues \citet{tolstoj1957}, who sees the main function of the long-form adjectives likewise in setting a certain property of a referent apart from properties of the rest of similar referents. 

The later re-interpretation of such “restrictive” (i.e., contrastive focus) expressions as definite and even the generalization of the original restrictive adjective marker to a marker of anaphoric reference of the modified noun seems functionally plausible. There is no indication, however, that the long-form agreement suffixes morphologized to true definite markers in the \ili{Old Slavic languages}. Even in the modern stages of the South Slavic languages \ili{Slovenian} and \ili{Serbo-Croatian}, where remnants of the two different adjective inflections still occur, the so-called definite (long-form) declension of adjectives is semantically restricted to certain adjectival subclasses (see \S\ref{s-slavic synchr}).

Furthermore, in \ili{Bulgarian} and \ili{Macedonian}, which are the only modern Slavic languages exhibiting a fully morphologized category \textsc{species}, the corresponding definite marking does not originate from the long-form adjectives. This is true despite the fact that the long-form agreement marking in \ili{Old Bulgarian} (i.e., the ancestor language of Modern \ili{Bulgarian} and Modern \ili{Macedonian}) is attested to have almost grammaticalized as a marker of anaphoric reference of the noun phrase.

Note also that even the morphological status of the so-called definite adjectives in the modern Baltic languages has been doubted. It has sometimes been argued that the long-form adjective in \ili{Lithuanian} might convey emphasis rather than definiteness, at least in certain expressions (cf.~\citealt[181–182]{kramsky1972}).

Even though the suffixes marking long-form agreement in Old Baltic\il{Old Baltic languages} and Old Slavic\il{Old Slavic languages} show some functional extension to markers of anaphoric reference or even definiteness of the noun phrase, this development is secondary. The original function of the long-form agreement suffixes was to mark an adjectival attribute in an emphatic or contrastive focus construction. Consequently, the suffix \textit{*-jь-} in \ili{Proto\hyp{}Baltic\slash{}Slavic} has to be analyzed as an attribution marker on the adjective rather than as a marker of definiteness of the modified noun.
\is{species marking!definite|)}

Leaving the question about the further development of the anti\hyp{}construct state agreement marker \textit{*-jь-} in different Baltic and Slavic languages aside, two opposing theories about its original function and the assumed functional developments of the anti\hyp{}construct state agreement marker in Baltic and Slavic will be discussed in the following sections:
%%%
\begin{itemize}
\item \textbf{Scenario 1:} The formative \textsc{attr} arose from a relative pronoun, hence:\\
\textsc{dem $\Rightarrow$ rel $\Rightarrow$ attr}
\item \textbf{Scenario 2:} The formative \textsc{attr} arose from an attributive article, hence:\\
\textsc{dem $\Rightarrow$ nmlz $\Rightarrow$ attr}
\label{2paths}
\end{itemize}

\subsubsection{Scenario 1: Nominal relative constructions in \ili{Proto\hyp{}Baltic\slash{}Slavic}}
According to the first theory, the attributive marker in Baltic and Slavic originates from a relative pronoun. This theory seems to be widely accepted since Delbrück's and Brugmann's (cf.~\citealt[432–433]{delbruck1893}; \citealt[331, 344]{brugmann-etal1916}) statements on the question. Their argumentation has been taken up and augmented with new data by \citet{schmidt1959}, \citet{koch1992,koch1999} and others. Koch argues that a reflex of the \ili{Proto\hyp{}Indo-European} relative pronoun \textit{*(h)i̭o-} is attested as an attributive marker of adjectival, possessive,\is{adnominal modifier!possessor noun} and adverbial modifiers\is{adnominal modifier!adverbial phrase} of nouns in \ili{Proto\hyp{}Baltic\slash{}Slavic}. He describes the constructions in which these attributes occur as “nominal relative constructions” \cite[470, passim]{koch1999}.

The most substantial part in Koch's argumentation seems to be the similar use of cognate relative pronouns as polyfunctional markers in relative constructions as attested in Old Iranian\il{Old Iranian languages} and \ili{Old Indo-Aryan languages}.
%%%
\il{Old Persian|(}
\begin{exe}
\ex {\rm Ezafe in \langinfo{Old Persian}{Indo-European}{\citealt{meillet1931}, here cited after \citealt[4]{samvelian2007b}}}
\label{ez oldpersian}
\begin{xlist}
\ex	{\upshape [}kāra {\upshape [}\textbf{hya} manā{\upshape ]]}
\glt	‘my army’ (lit. ‘army which is mine’)
%%%
\ex	{\upshape [}kāsaka {\upshape [}\textbf{hya} kapautaka{\upshape ]]}
\glt	‘the blue stone’ (lit. ‘stone which is blue’)
\ex	vivānam jatā utā avam {\upshape [}kāram {\upshape [}\textbf{hya} dārayavahaus xšāyaθiyhyā{\upshape ]]}
\glt	‘Beat Vivâna and his army which declares itself as a proponent of the king Darius.’
\end{xlist}
\end{exe}
%%%
Koch's (\citeyear[53, passim]{koch1992}) main arguments for the old age of the relative function of \textit{*(h)i̭o-} in \ili{Proto\hyp{}Indo-European} are found in attested cognate markers. In several Indo-European languages, the historical \textit{*(h)i̭o-} pronoun marks similar relative constructions as in the Old Persian examples (\ref{ez oldpersian}). Koch does not disprove, however, the assumption that the relative function of the pronoun derives from the deictic-anaphorical marking by means of a demonstrative. In fact, the Old Persian examples (\ref{ez oldpersian}) clearly show verb-less relative constructions linked to the head noun with an attributive article.
%According to Benveniste %GET CITE FROM Koch1992p57 (cf.~even \cite[PAGE]{lehmann1984}) are not originate leitet sich der idg. Relativsatz nicht aus satzartigen Vorgängern ab sondern aus verblosen Nominalsyntagmen%
%%%%%two relative pronouns reconstructed for the Indo-European proto-language: \textit{*(h)i̭o-} and \textit{*k\textsuperscript{ṷ}i-/*k\textsuperscript{ṷ}o-}
%%The relative function of the second pronoun \textit{*k\textsuperscript{ṷ}i-/*k\textsuperscript{ṷ}o-} clearly derives from its indefinite or interrogative meaning (KOCH, cf.~German \textit{welch-} English \textit{which} \textsc{interrog, rel}). 
\il{Old Persian|)}

Furthermore, it is not certain whether the old pronoun (or article) \textit{*(h)i̭o-} was inherited into \ili{Proto\hyp{}Baltic\slash{}Slavic}. The pronominal stem is attested in Baltic or Slavic only as the base of some derived connectors \cite[56]{heinrichs1954}. Even though the etymological pronoun seems to be preserved in the stem of the \ili{Old Bulgarian} relative marker \textit{jь-že}, the function of this marker is clearly yielded by the emphatic particle \textit{-že} \cite[56]{heinrichs1954}. %Zitat falsch!!
 The old relative pronoun seems to be completely lost in Old Baltic\il{Old Baltic languages} where different relative markers occur (as in \ili{Lithuanian} \textit{ku\~rs} $\Leftarrow$ \textit{kurìs}, \ili{Latvian} \textit{kuŕš} noted by \citealt[15]{schmidt1959}).

\citet[468, 470]{koch1999} dates the original relative construction back to an early \ili{Pre-Proto\hyp{}Baltic\slash{}Slavic} age. According to him, the relative pronoun did not agree in case with the head noun in the inherited Indo-European relative construction (\ref{koch rel}). Such morpho-syntactic behavior would in fact be expected from a true relative pronoun. But according to Koch's reconstruction (\ref{koch nomzr}), case agreement between a head noun and a relative pronoun was already present in \ili{Proto\hyp{}Baltic\slash{}Slavic}. Finally, the long-form agreement inflection arose independently as a result of the phonological merger of the adjective and the original pronoun in Old Baltic\il{Old Baltic languages} and Old Slavic\il{Old Slavic languages} (\ref{koch attr}). Most crucial in this reconstruction is the fact that the assumed original relative pronoun has obviously never marked a true relative clause construction in \ili{Proto\hyp{}Baltic\slash{}Slavic}.
%%%
\begin{exe}
\ex
\label{koch rel}
\begin{xlist}
\ex	{\rm Nominal relative constructions in Pre-Proto\hyp{}Baltic\slash{}Slavic \citep[468]{koch1999}}\footnote{The example is glossed in accordance to Koch; a translation is missing in the source.}\\
\glll	*dråugås gīvås jås {\rm /} *dråugåm gīvås jås\\
	friend:\textsc{nom} good:\textsc{nom} \textsc{rel:nom} / friend:\textsc{acc} good:\textsc{nom} \textsc{rel:nom}\\
	N\textsubscript{nom} A\textsubscript{nom} \textsc{rel}\textsubscript{nom} { } N\textsubscript{acc} A\textsubscript{nom} \textsc{rel}\textsubscript{nom}\\
%%%
\ex	{\rm \ili{Proto\hyp{}Baltic\slash{}Slavic} attributive article}\\
\label{koch nomzr}
\glll	*dråugås gīvås-jås {\rm /} *dråugåm gīvåm-jåm\\
	friend:\textsc{nom} good-\textsc{nmlz:nom} / friend:\textsc{acc} good-\textsc{nmlz:acc}\\
	N\textsubscript{nom} A\textsubscript{nom}-\textsc{nmlz}\textsubscript{nom} { } N\textsubscript{acc} A\textsubscript{acc}-\textsc{nmlz}\textsubscript{acc}\\
%%%
\ex	{\rm Old Baltic/Old Slavic anti\hyp{}construct state agreement marking}\\
\label{koch attr}
\glll	*dråugås gīvå-jås {\rm /} *dråugåm gīvå-jåm\\
	friend:\textsc{nom} good-\textsc{attr:nom} / friend:\textsc{acc} good-\textsc{attr:nom}\\
	N\textsubscript{nom} A-\textsc{attr}\textsubscript{nom} { } N\textsubscript{acc} A-\textsc{attr}\textsubscript{acc}\\
\end{xlist}
\end{exe}
%%%
This assumed development presupposes the transition of original “nominal relative constructions” in \ili{Pre-Proto\hyp{}Baltic\slash{}Slavic} (step 1) to a construction with an attributive article (\textsc{nmlz}) in \ili{Proto\hyp{}Baltic\slash{}Slavic} as an intermediate step (2). The anti\hyp{}construct (“long-form”, i.e., \textsc{attr}) agreement marking arose as a last step (3) in Old Baltic\il{Old Baltic languages} and Old Slavic.\il{Old Slavic languages}
%%%
\begin{itemize}
\item Stage 1 [\textsubscript{NP} \textsubscript{HEAD}N {\upshape [}\textsubscript{ATTRIBUTE(CLAUSE)} A\textsubscript{[+agr]} \textsc{rel}\textsubscript{[-agr]}{\upshape ]]}
\item Stage 2 [\textsubscript{NP} \textsubscript{HEAD}N {\upshape [}\textsubscript{ATTRIBUTE(NP')} A\textsubscript{[+agr]}-\textsc{nmlz}\textsubscript{[+agr]}{\upshape ]]}\label{koch constit nomzr}
\item Stage 3 [\textsubscript{NP} \textsubscript{HEAD}N \textsubscript{ATTRIBUTE(A)}A-\textsc{attr}\textsubscript{[+agr]}{\upshape ]}
\end{itemize}
%%%
Koch's reconstruction gives no conclusive arguments for the existence of “nominal relative constructions” marked with a relative pronoun \textit{*(h)i̭o-} in \ili{Pre-Proto\hyp{}Baltic\slash{}Slavic}. Theoretically, the attributive nominalization construction (step 2) could be much older and be the primary one in Indo-European. The corresponding “nominal relative constructions” in Indo-Aryan\il{Indo-Aryan languages} and Iranian\il{Iranian languages} might just as well originate from attributive nominalization constructions. The Indo-European relative pronoun \textit{*(h)i̭o-} would than go back to a deictic pronoun, probably \textit{*i-} ($\Rightarrow$ \ili{Latin}, \ili{Gothic} \textit{is} \textsc{dem}) which was used as attributive article as early as in \ili{Proto\hyp{}Indo-European}.

%also The question of constituent order is left open in Koch's reconstruction. In the examples (\ref{koch rel}–\ref{koch nomzr}) the attribute is marked by a postponed pronoun and follows the noun. In the Old Iranian and Old Indo-Aryan languages, in which the assumed cognate relative pronoun is attested, 
%the attribute also follows the noun but the relative marker occurs between the constituents.
%\begin{exe}
%\ex \textsc{Old Persian} Samvelian
%\ex \textsc{Modern Persian}
%\end{exe}
%Already in Old Slavic and Old Baltic the adjective predominantly preceded the noun. aber nachgestellte adjective belegt besonders in emphatischem ausdruck noch heute. The constituent order change from Indo-European NA to Baltic\slash{}Slavic AN unproblematisch

\subsubsection{Scenario 2: Attributive nominalizing constructions in \ili{Proto\hyp{}Baltic\slash{}Slavic}} 
According to the second idea about the emergence of the long-form adjectives in Baltic\slash{}Slavic, the attributive marker was originally an article. One opponent of the “relative” theory is van Wijk, who believes 
%%%
\begin{quote}
[\dots] dass wir fürs Slavische vollständig auskommen ohne die Annahme relativer Pronominalformen vom idg. Stamme \textit{i̭e/i̭o-}, und dass dasselbe für das Baltische gilt. \citep[28]{wijk1935}%!!check translations of all quotes
\end{quote}
%%%
%%\cite{otrebski1968} sieht in –j- der pronominalen Flexionsformen die Fortsetzung einer “hervorhebenden Partikel”
Leaving open whether an attributive article or a relative pronoun constitutes the ultimate origin of the anti\hyp{}construct state agreement in \ili{Pre-Proto\hyp{}Baltic\slash{}Slavic}, Koch's reconstruction would in fact be compatible with Wijk's “article theory”. The attribute nominalizing construction with the pronominal marker \textit{*-jь-} as attributive article in \ili{Proto\hyp{}Baltic\slash{}Slavic} is clearly reflected in step 2 of Koch's reconstruction (examples \ref{koch nomzr} and \ref{koch constit nomzr}). The final step 3 in which the attributive nominalizer becomes an anti\hyp{}construct state marker is completely similar to the development assumed by \cite{wijk1935}.

The most plausible functional explanation of the grammaticalization of the pronominal marker \textit{*-jь-} into an attributive article is formulated by Wissemann (\citeyear{wissemann1958}). He argues that the original function of the anti\hyp{}construct (“long-form”) agreement suffixes was that of a “Gelenkspartikel”\is{attributive article} \citep[76]{wissemann1958}, i.e., an \textit{attributive article} or \textit{attributive nominalizer} in terms of the present study. Wissemann also shows that the function as anaphoric (“quasi-definite”) noun phrase marker is secondary.\is{species marking!definite}

Another argument in favor of the attributive nominalizing function of the \ili{Proto\hyp{}Baltic\slash{}Slavic} attributive article \textit{*-jь-} can be found in its polyfunctional use with different types of attributes. Besides marking the attributive connection of (emphasized) adjectives and participles, the article also served to mark some non-adjectival (and originally non-agreeing) attributes, such as adverbial phrases\is{adnominal modifier!adverbial phrase} and noun phrases marked with genitive.\is{adnominal modifier!possessor noun}

\citet[467–468]{koch1999} gives a list of lexicalized attributive expressions in which \textit{*-jь-} occurs as an attributive marker. These examples of frozen nominalizations present evidence of the original attributive nominalizing function of the \ili{Proto\hyp{}Baltic\slash{}Slavic} article. 
%%%
\begin{exe}
\ex
\begin{xlist}
\ex {\rm Attribution of adverbial phrases}
\begin{xlist}
\ex {\rm \ili{Old Bulgarian}}\\
	utrějь {\rm ‘tomorrow- (attr.)’ $\leftarrow$} (j)utrě {\rm ‘morning’}
\ex {\rm \ili{Old Bulgarian}}\\
	vьnějь {\rm ‘outside (attr.)’ $\leftarrow$} vьně {\rm ‘(on the) outside’}\\
	bezumajь {\rm ‘ignorant’ $\leftarrow$} bez uma {\rm ‘without mind’}
\ex {\rm \ili{Old Bulgarian}}\\
	nabožijo̜jь {\rm ‘pleasing to God (attr.)’ $\leftarrow$} na božijo̜ {\rm ‘pleasing to God’}
\end{xlist}
%%%
\ex {\rm Attribution of noun phrases in genitive (attested only in Baltic)}
\begin{xlist}
\ex {\rm \ili{Lithuanian}}\\
	di\~evojis {\rm ‘god-like (attr.)’ $\leftarrow$} di\~evo {\rm \textsc{gen.sg} $\leftarrow$} di\~evas {\rm \textsc{nom.sg} ‘God’}
\ex {\rm \ili{Lithuanian}}\\
	pači\~u̜jis {\rm ‘belonging to (attr.)’ $\leftarrow$} pači\~u̜ {\rm \textsc{gen.pl} $\leftarrow$} pàts {\rm \textsc{nom.pl} ‘self’}
\end{xlist}
\end{xlist}
\end{exe}
%%%
Against his own suggestion that in Baltic\slash{}Slavic anti\hyp{}construct state agreement marking originates from nominal relative constructions, in other words:
%%%
\begin{itemize}
\item \textbf{Scenario 2:} \textsc{dem $\Rightarrow$ nmlz $\Rightarrow$ attr}
\end{itemize}
Koch's examples provide the best arguments for the opposite assumption that attributive nominalizing constructions are the source of that marker.

\is{species marking!definite|(}
\il{Germanic languages|(}
\subsubsection[Germanic]{Attributive nominalizers and the emergence of anti\hyp{}construct state agreement marking in Germanic}
\label{germanic diachr}
%%%
As in the Baltic\slash{}Slavic languages, the emergence of attributive nominalizers in Germanic is functionally connected to the rise of definiteness marking. In Modern Baltic and some South Slavic languages, the occurrence of anti\hyp{}construct state agreement marking is restricted to (semantically) definite noun phrases. This functional devision between “true” \isi{head\hyp{}driven agreement} and anti\hyp{}construct state agreement marking was already characteristic of all Old Baltic\il{Old Baltic languages} and \ili{Old Slavic languages}. 

As in the \ili{Proto\hyp{}Baltic\slash{}Slavic} languages, a secondary inflectional paradigm of adjectives was innovated in \ili{Proto\hyp{}Germanic}. This so-called weak adjective declension has often been described as the first definite marking device in Germanic (e.g., by \citealt{heinrichs1954} and \citealt[170]{ringe2006}) because its use was restricted to (semantically) definite noun phrases. Semantic definiteness, however, was never marked obligatorily in any of the \ili{Old Germanic languages}. Even though demonstrative pronouns were sometimes used in semantically definite phrases, definite markers had not yet been grammaticalized in Old Germanic varieties. Examples from Old Germanic text sources show that the use of both demonstratives and “weak adjectives” in definite phrases was optional (cf.~\citealt{philippi1997}; \citealt{heinrichs1954}).
\il{Baltic languages|)}
\il{Slavic languages|)}

Only the modern Germanic languages exhibit true definite markers and thus a grammaticalized feature \textsc{species}. But the so-called definite articles of modern Germanic languages originate from etymological sources which were different from the older anti\hyp{}construct state agreement marking suffixes. Following \citet[267–268]{riesler2006a}, the rise of the Germanic “weak” adjective declension is here explained as a result of attributive nominalization. 
%%%
\begin{exe}
\ex {\rm “Strong” and “weak” agreement in Proto\hyp{}Germanic \citep[169]{ringe2006}}
\begin{xlist}
\ex {\rm Head\hyp{}driven (“strong”) agreement}\\
\gll *k\textsuperscript{w}ik\textsuperscript{w}a-\\
	quick:\textsc{m.sg.nom-}\\
%%%
\ex {\rm Anti\hyp{}construct state (“weak”) agreement}\\
\gll *k\textsuperscript{w}ik\textsuperscript{w}a\textbf{-n-}\\
	quick:\textsc{m.sg.nom}\textbf{\textsc{-nmlz-}}\\
\glt	‘quick’
\end{xlist}
\end{exe}
%%%
The \ili{Pre-Proto\hyp{}Germanic} formative marking “weak” agreement is sometimes described as an “individualizing” or “nominalizing” suffix of nominals (i.e., adjectives and, perhaps, nouns as well). These functions are reflected in (nick-) names, such as \ili{Ancient Greek} \textit{ágáthōn} ‘the Good’ ($\leftarrow$ \textit{ágáthós} ‘good’) or \ili{Latin} \textit{Catō} ‘the Shrewd’ ($\leftarrow$ \textit{catus} ‘shrewd’) which are also derived from nouns equipped with the cognate suffix \textit{*-n-} \citep[170]{ringe2006}.\footnote{Names such as Latin \textit{Marcus Catō, Ovidius Nasō} are interpreted as ‘Marcus the cunning’ and ‘Ovidius the nose’ \citep[6–7]{nocentini1996}.}

Some scholars have reconstructed a pronominal stem extension \textit{*-en-/-on-} as the origin of the suffix (for example \citealt[52]{mikkola1950} and \citealt[67]{heinrichs1954}). Others express their doubt about the pronominal origin of this marker (for example \citealt[21 Footnote 6]{schmidt1959}). But even without a definitely reconstructed etymology of the formative, the construction clearly shows similarities with the attributive nominalization of adjectives in \ili{Proto\hyp{}Baltic\slash{}Slavic}. It thus seems relatively safe to follow Mikkola (\citeyear{mikkola1950}) and Heinrichs ({\citeyear{heinrichs1954}) in assuming that the weak adjective declension in Germanic goes back to a construction with an attributive nominalizer.

\citet[170]{ringe2006} finds it “reasonable to hypothesize that the \textit{n-}stem suffix of the weak adjective paradigm was originally a definite article”. But this hypothesis must be rejected because the marker was never obligatory in definite contexts. Similar to Baltic\il{Baltic languages} and Slavic,\il{Slavic languages} it seems much more plausible to assume that the article was never a true definiteness marker. It can rather be assumed that the clue for understanding the origin of the “weak” adjective declension in Germanic is the nominalizing function of the \textit{article},\is{attributive article} which originally marked an (emphatically-contrasted) adjective as an appositional attribute.\is{contrastive focus}
\is{species marking!definite|)}

The rise of anti\hyp{}construct state agreement marking of attributive adjectives in Germanic thus followed a similar grammaticalization path as in Baltic\il{Baltic languages} and Slavic.\il{Slavic languages}\footnote{The zero-morpheme (equipped with the nominalizer Ø-\textsc{nmlz}) in (\ref{germanic gram1}) and following examples is only presented for a better illustration of the empty head position to which the (nominalized) adjective moves in the appositional noun phrase.}
%%%
\begin{exe}
\ex
{\rm Grammaticalization of anti\hyp{}construct state agreement in Germanic}
\label{germanic gram1} 
\begin{xlist}
\ex
\label{germanic1}
	{\rm Stage 1}
\begin{xlist}
\ex	{\rm Agreement marking (default)}\\
{\upshape [}\textsubscript{\rm NP} \textsubscript{\rm A}big-\textsc{agr} \textsubscript{\rm N}house{\upshape ]}
%%%
\ex	{\rm Attributive apposition (emphatic)}\\
\label{germanic art1}
{\upshape [}\textsubscript{\rm NP} {\upshape [}\textsubscript{\rm NP'} \textsubscript{\rm A}big \textsubscript{\rm HEAD}Ø-\textsc{nmlz}{\upshape ]} \textsubscript{\rm N}house{\upshape ]]}
\end{xlist}
%%%
\ex
\label{germanic2}
	{\rm Stage 2} 
\begin{xlist}
\ex	{\rm Agreement marking (default)}\\
{\upshape [}\textsubscript{\rm NP} \textsubscript{\rm A}big-\textsc{agr} \textsubscript{\rm N}house{\upshape ]}
%%%
\ex	{\rm Agreement marking (emphatic)}\\
\label{germanic ACAgr}
{\upshape [}\textsubscript{\rm NP} \textsubscript{\rm A}big-\textsc{agr:contr} \textsubscript{\rm N}house{\upshape ]}
\end{xlist}
%%%
\ex
\label{germanic3}
	{\rm Stage 3}
\begin{xlist}
\ex 	{\rm Agreement marking (default)}\\
{\upshape [}\textsubscript{\rm NP} \textsubscript{\rm A}big-\textsc{agr:attr} \textsubscript{\rm N}house{\upshape ]}
\end{xlist}
\end{xlist}
\end{exe}
%%%
During Stage 1 (\ref{germanic1}), the attributive nominalizer (i.e., the pronominal stem extension \textit{*-en-/-on-}) competed with the default adjective attribution marking device (i.e., the inherited Indo-European \isi{head\hyp{}driven agreement}) but was restricted only to emphatic attributive appositional constructions. This stage can be dated back to \ili{Proto\hyp{}Germanic} at the latest. In all \ili{Old Germanic languages}, the original attributive appositional construction is reanalyzed\is{re-analysis} as a true noun phrase in which the former attributive nominalizer marks an adjective in contrastive focus. The secondary attribution marking device still competed with the default adjective attribution marking device (i.e., \isi{head\hyp{}driven agreement} during Stage 2 (\ref{germanic2}). The competition between the two different adjective attribution marking devices was dissolved during Stage 3 (\ref{germanic3}). This stage is reflected by the modern \ili{West Germanic languages} where only one type of adjective attribution marking occurs. Due to the fact that agreement inflection of adjectives in modern \ili{West Germanic languages} (except in \ili{English}) only marks attributive but not predicative adjectives,\is{predicative marking} this adjective attribution marking device has been characterized as anti\hyp{}construct state agreement (see \S\ref{w-germanic synchr}).

\is{species marking!definite|(}
\subsection[Definite noun phrases in Germanic]{Excursus: Definite noun phrases in Germanic}
%%%
In the previous section, it was shown that the grammaticalization of the feature \textsc{species} (definiteness) in Germanic is a relatively recent phenomenon which is not directly connected to the rise of attributive nominalization and anti\hyp{}construct state agreement marking (so-called “weak” or “definite” agreement). Even though anti\hyp{}construct state agreement usually occurred in semantically definite noun phrases, true definite markers evolved much later.

The etymological source of the definite markers were local-deictic (demonstrative) pronouns: \ili{Proto\hyp{}Germanic} \textit{*sa, *sō, *þat}, in North Germanic additionally also \textit{en, enn, et} \citep[15]{heinrichs1954}. Interestingly, the evolving definite markers from the first set of \ili{Proto\hyp{}Germanic} demonstratives were also first used as attribution markers of adjectives \citep{gamillscheg1937, nocentini1996}. Later, the use of the articles was extended from appositional (nominalized) adjectives to whole noun phrases \citep[63]{philippi1997}. If the grammaticalization path illustrated in (\ref{germanic gram1}) is extended with one more stage, the evolution of definiteness marking in Germanic can be included as well. Note that the additional developments in the grammaticalization path (\ref{germanic gram2}) are also partly connected to adjective attribution.
%%%
\begin{exe}
\il{West Germanic languages}
\ex {\rm Grammaticalization of definiteness marking in West Germanic}
\label{germanic gram2}
\begin{xlist}
\ex {\rm Stage 3}
\begin{xlist}
\ex {\rm Agreement marking (default)}\\
{\upshape [}\textsubscript{\rm NP} \textsubscript{\rm A}big-\textsc{agr:attr} \textsubscript{\rm N}house{\upshape ]}
%%%
\ex {\rm Attributive apposition (emphatic)}\\
\label{germanic art2}
{\upshape [}\textsubscript{\rm NP} {\upshape [}\textsubscript{\rm NP'} \textsubscript{\rm ART}the \textsubscript{\rm A}big-\textsc{agr:attr} \textsubscript{\rm HEAD}Ø{\upshape ]} \textsubscript{\rm N}house{\upshape ]}
\end{xlist}
\ex {\rm Stage 4}
\begin{xlist}
\ex {\rm Definiteness marking}\\
\label{germanic def}
{\upshape [}\textsubscript{\rm NP} \textsubscript{\rm DEF}the \textsubscript{\rm A}big-\textsc{agr:attr} \textsubscript{\rm N}house{\upshape ]}
\end{xlist}
\end{xlist}
\end{exe}
%%%
Note that an attributive apposition construction for marking emphasis occurs twice in the illustrated grammaticalization path (\ref{germanic gram2}). In Stage 1 (\ref{germanic art1}), the attributive nominalizer is the pronominal stem extension \textit{*-en-/-on-} which becomes the anti-constract state agreement marker in the following stage (\ref{germanic ACAgr}). The second attributive nominalizer in Stage 3 (\ref{germanic art2}) is the demonstrative pronoun which becomes the definite marker in the following stage (\ref{germanic def}). These two attributive nominalizers have different etymological sources and attach to different positions inside the noun phrase but they are functional equivalents.

Stage 4 in example \REF{germanic gram2} did not fully affect North Germanic.\il{North Germanic languages} Instead, the \ili{Old North Germanic languages} (Old East\il{Old East Norse} and \ili{Old West Norse}) grammaticalized definite markers from the demonstratives \textit{en, enn, et} \citep[15]{heinrichs1954}. These markers are the complete morpho-syntactic opposites of West Germanic:\il{West Germanic languages} Unlike the West Germanic preposed and free form definite marker, all modern North Germanic\il{North Germanic languages} standard languages exhibit a postposed definite noun inflection. The different morpho-syntactic realization of the general Germanic tendency towards grammaticalization of definiteness is best explained as contact-induced change due to Saamic\il{Saamic languages} influence in North Germanic \citep{kusmenko2008}.
%%%
\begin{exe}
\ex {\rm Grammaticalization of definiteness marking in Germanic}
\label{germanic gram3}
\begin{xlist}
\ex {\rm Stage 4}
\begin{xlist}
\ex {\rm Definiteness marking (West Germanic)}\\
{\upshape [}\textsubscript{\rm NP} \textsubscript{\rm DEF}the \textsubscript{\rm A}big-\textsc{agr:attr} \textsubscript{\rm N}house{\upshape ]}
%%%
\ex {\rm Definiteness marking (North Germanic)}\\
\label{ngermanic def}
{\upshape [}\textsubscript{\rm NP} \textsubscript{\rm ATTR:AGR}the\textsubscript{\rm agr:attr} \textsubscript{\rm A}big-\textsc{agr:attr} \textsubscript{\rm N}house-\textsc{def}{\upshape ]}
\end{xlist}
\end{xlist}
\end{exe}
%%%
Note that in North Germanic Stage 4 (\ref{ngermanic def}) the former preposed nominalizer (article) did not grammaticalize into a true definite marker like in West Germanic but into an anti\hyp{}construct state agreement marker. The noun phrase structure is thus different from Stage 3 (\ref{germanic art2}) because the attributive apposition of a the nominalized headless adjective\is{headless noun phrase} is lost and the semantic head of the overall noun phrase is syntactically reunited with its adjectival modifier.

Synchronic data from different North Germanic\il{North Germanic languages} varieties reflect intermediate stages in the evolution of definite noun phrase structure. This cross-linguistic variation is most likely the result of competing grammaticalization of a preposed article and a postposed definite inflection \citep{dahl2003}. 

As with all modern \ili{West Germanic languages},\footnote{In English, the noun phrase structure is similar in theory, with the exception of adjectives in headless noun phrases which are obligatorily nominalized: \textit{the good \textbf{one}}; see also \S\ref{w-germanic synchr}.} the Western Jutlandic\il{Danish!W-Jutlandic} dialect of Danish exhibits phrasal definite marking by means of a phonologically free and preposed definite article.
%%%
\begin{exe}
\il{Danish!W-Jutlandic}
\ex {\rm W-Jutlandic}\footnote{The examples are constructed according to \citet{lund1932}, cf.~also \citet[121–122]{delsing1993} and \citet{dahl2003}.}
\begin{xlist}
\ex de korn {\rm [\textsc{def} corn]}
\ex de god (et) {\rm [\textsc{def} good:\textsc{agr} (\textsc{nmlz:agr})]}
\ex de god korn {\rm [\textsc{def} good:\textsc{agr} corn]}
\end{xlist}
\end{exe}
%%%
In several of the northernmost North Germanic\il{North Germanic languages} varieties, definiteness is also marked phrasally but by means of a phonologically bound and postposed formative. Consequently, the phrasal definite marker attaches as suffix to definite nouns and definite headless adjectives\is{headless noun phrase} alike. Note also that adjectives are incorporated into (or compounded with) the head noun. 
%%%
\begin{exe}
\il{Swedish!Västerbotten}
\ex {\rm Västerbotten Swedish}\footnote{The examples are constructed according to \citet{astrom1893}, cf.~also Delsing (\citeyear[122–123]{delsing1993}) and \cite{dahl2003}.}
\begin{xlist}
\ex korn-e {\rm [corn-\textsc{def}]}
\ex god-e {\rm [good-\textsc{def}]}
\ex god-korn-e {\rm [good-corn-\textsc{def}]}
\end{xlist}
\end{exe}
%%%
\il{Swedish|(}
In the North Germanic languages \ili{Norwegian}\footnote{New- and Dano Norwegian\il{Norwegian!New Norwegian}\il{Norwegian!Dano Norwegian}} and Swedish as well as in \ili{Faroese}, the definite marker is an inflectional suffix as in the Västerbotten dialect of Swedish,\il{Swedish!Västerbotten} i.e., phonologically bound and postposed. The formative is, however, exclusively a noun marker and does not show up on adjectives in definite \isi{headless noun phrase}s. The latter are not overtly marked as definite but show circum-positioned definite agreement marking by means of a preposed attributive article and definite agreement inflection.
%%%
\begin{exe}
\il{Swedish}
\ex {\rm Swedish (personal knowledge)}
\begin{xlist}
\ex[]{
	korn-et {\rm [corn-\textsc{def}]}
	}
\ex[]{
	det god-a korn-et {\rm [\textsc{nmlz:agr} good-\textsc{agr} corn-\textsc{def}]}
	}
\ex[]{
	det god-a {\rm [\textsc{nmlz:agr} good-\textsc{agr}]}
	}
\ex[*]{
	det korn-et
	}
\end{xlist}
\end{exe}
%%%
\il{Danish|(}
\il{Icelandic|(}
In Danish and (colloquial) Icelandic, the definite marker has two allomorphs: an inflectional noun suffix similar to Swedish (i.e., a phonologically bound and postposed) and a definite article similar to the \ili{West Germanic languages} (i.e., phonologically free and preposed). Interestingly, the allomorphy of the definite marker in Danish and Icelandic is triggered by the part-of-speech membership of the host: whereas the bound allomorph selects for nouns, the free form selects for adjectives.
\il{Swedish|)}
%%%
\begin{exe}
\ex {\rm Danish (personal knowledge)}
\begin{xlist}
\ex[]{
	korn-et {\rm [corn-\textsc{def}]}
	}
\ex[]{
	det god-e korn {\rm [\textsc{def} good-\textsc{agr} corn]}
	}
\ex[]{
	det god-e {\rm [\textsc{def} good-\textsc{agr}]}
	}
\ex[*]{
	det god-e korn-et
	}
\end{xlist}
\end{exe}
%%%
\begin{table}
\begin{tabular}{lccc}
\lsptoprule
			&\textsc{utr}	&\textsc{n}		&\textsc{pl}\\
\midrule
\textsc{def}	&-en [den]		&-et [det]			&-{Ø} [de]\\
\lspbottomrule
\end{tabular}
\caption[Paradigm of \textsc{def} in Danish]{Paradigm of the definite marker in Danish (personal knowledge). Note that the choice whether the suffix or the free from constitute the base morpheme or the allomorph seems arbitrary.}
\label{danish defallomorph}
\end{table}
\il{Danish|)}
%%%
\begin{exe}
\ex {\rm Icelandic (personal knowledge)}
\begin{xlist}
\ex[]{
	korn-ið {\rm [corn-\textsc{def}]}
	}
\ex[]{
	hið goð-a {\rm [\textsc{def} good-\textsc{agr}]}
	}
\ex[]{
	hið goð-a korn {\rm [\textsc{def} good-\textsc{agr} corn]}
	}
\ex[*]{
	hið goð-a korn-ið {\rm [\textsc{def} good-\textsc{agr} corn-\textsc{def}]}
	}
\end{xlist}
\end{exe}
\il{Icelandic|)}

\il{North Germanic languages|(}
\is{buffer zone|(}
\subsubsection{“Double definiteness” and a “buffer zone” in North Germanic}
\label{buffer}
The geographic distribution of different morpho-syntactic types of definiteness marking across North Germanic reveals interesting areal patterns. The occurrence of adjective incorporation coincides with the area of the missing preposed article. Both features are characteristic of the northeastern periphery of North Germanic (\citealt{delsing1996b}, cf.~also \citealt{riesler2001a,riesler2002a}). The structural connection between adjective incorporation and the missing preposed article is obvious: the construction with the compounded (incorporated) adjective in definite noun phrases substitutes the corresponding construction with the preposed article in those dialects where a preposed article has not (yet) been developed from the former demonstrative. The northeastern North Germanic data thus reflects an early Stage 3 in the illustrated grammaticalization path (\ref{germanic3}).

The northeastern North Germanic dialect area constitutes the innovation center of the grammaticalization of a (suffixed) inflectional category \textsc{species} (definiteness). The southwestern North Germanic dialects, located geographically at the very opposite periphery, exhibit a structurally reversed picture of northeastern North Germanic which is in its direction of evolution almost identical to the situation in West Germanic.\il{West Germanic languages}

Dahl describes the phrasal definite markers in southwestern and northeastern North Germanic dialects as the result of structurally and geographically opposed processes of grammatical changes.
%%%
\begin{quote}
[T]he variation we can see in the attributive constructions is the result of the competition between them about the same territory. \citep[147]{dahl2003}
\end{quote}
%%%
The “competition” between northeastern and southwestern grammaticalization tendencies in Germanic is not restricted to definite marking. Several grammatical categories which developed as the result of common Germanic (or even Indo-European) tendencies, have grammaticalized into non-fusional (analytic) constructions in West Germanic\il{West Germanic languages} but into concatenate (synthetic) constructions in North Germanic. Language contact with neighboring \ili{Uralic languages} would offer the most plausible explanation for the structurally differentiated developments inside the Germanic branch. Consequently, Kusmenko (\citeyear{kusmenko2008}) proposed a model for explaining the morphological fusion of definiteness and other North Germanic innovative categories as the result of interference features during the language shift of the assimilated Saami of Mediaeval Scandinavia.

A direct connection between language contact and the rise of adjective incorporation and the missing preposed adjective article in northeastern North Germanic varieties was also suggested by \cite{riesler2001a,riesler2002a}. But even if this idea cannot be proven correct the historical connection between missing preposed adjective articles, adjective incorporation and the morpho-syntactic type of definiteness marking (i.e., morphologically fused and postposed) in the northeastern North Germanic dialect area is obvious. Saamic\il{Saamic languages} influence (causing the morphological fusion of postposed definiteness marking) would thus at least be an indirect trigger of these areal grammaticalization phenomena in North Germanic which can be described as a “buffer zone” \citep{stilo2005}.\footnote{Stilo created the term for a similar language area between competing grammaticalization tendencies due to contact induced-changes in the Southern Caucasus.\is{Caucasus} The parallel between Stilo's “buffer zone” and Dahl's (\citeyear{dahl2003}) “competing” morpho-syntactic types in North Germanic languages was first mentioned to the author by Tania Kuteva (p.c.).\ia{Kuteva, Tania} But neither Dahl nor Kuteva drew contact linguistic implications in the North Germanic case. The idea about the North Germanic “buffer zone” as an indirect result of contact-induced changes was first mentioned by \cite{riesler2006a}.}

%!!One summerizing sentence on the cyclic nature of grammaticalization in Germanic (and Baltic).
\begin{sidewaystable}
\begin{tabular}[t]{l l l l l l l l l l}
\lsptoprule
&	&Proto\hyp{}& &Old-&	&Modern&&\\
&	&Germanic&	&Germanic&	&Germanic&&\\
\midrule
\textsc{dem1}&$\Rightarrow$&\textsc{art1}&$\Rightarrow$&\textsc{attr}&$\Rightarrow$&\textsc{agr}&$\Rightarrow$&Ø&English, (W-Jutlandic)\\
\textsc{dem1}&$\Rightarrow$&\textsc{art1}&$\Rightarrow$&\textsc{attr}&$\Rightarrow$&\textsc{agr}&&&W+N-Germanic\\
\\
&&&		&\textsc{dem2}&$\Rightarrow$&\textsc{art2}&$\Rightarrow$&\textsc{def1}&W(+N)-Germanic\\
&&&		&\textsc{dem2}&$\Rightarrow$&\textsc{art2}&&&N-Germanic\\
&&&		&\textsc{dem2}&&&&&Västerbotten Swedish\\
\\
&&&		&\textsc{dem3}&$\Rightarrow$&\textsc{def2}&&&N-Germanic\\
\lspbottomrule
\end{tabular}
\caption[Article grammaticalization cycle in Germanic]{Article grammaticalization cycle in Germanic languages (adapted from \citealt[272]{riesler2006a}).}
\end{sidewaystable}
\is{species marking!definite|)}
\il{North Germanic languages|)}
\il{Germanic languages|)}
\il{Indo-European languages|)}
\is{buffer zone|)}

\subsection[Attributive nominalization and anti\hyp{}construct state]{Attributive nominalization and the grammaticalization of anti\hyp{}construct state (agreement) marking}
%%%
The previous sections described how anti\hyp{}construct state agreement marking arose in the Baltic,\il{Baltic languages} Slavic\il{Slavic languages} and Germanic\il{Germanic languages} branches of Indo-European. Structurally similar developments were also described for \ili{Udmurt} from the Permic branch of Uralic, in \ili{Chuvash} and other so-called \ili{Uralo-Altaic languages} in \S\ref{uralic-turkic diachr}. 

The emergence of attributive nominalizers such as secondary attribution markers seem to reflect a general tendency in several branches of the Indo-European,\il{Indo-European languages} Uralic\il{Uralic languages} and Turkic\il{Turkic languages} language families. The etymological source of the attributive nominalizer in all of these languages is either a local deictic determiner or the 3\textsuperscript{rd} person possessive marker with “determinative” functions.

\il{Lezgic languages|(}
Synchronic data from several languages of the Lezgic (Daghestanian) branch of Nakh-Daghestanian (see \S\ref{lezgian synchr}) seem to reflect a similar grammaticalization path from deictics to attributive nominalizers. Most Lezgic languages sampled for the present study have \isi{juxtaposition} as the default adjective attribution marking device. Attributive nominalization also occurs in most languages of this branch but is restricted to \isi{headless noun phrase}s. The attributive nominalizer is a stem augment \textit{-tV- / -dV-} which could be connected historically to the deictic pronouns occurring with similar shapes in these languages. In \ili{Budukh}, the cognate suffix \textit{-ti} is not used as an attributive nominalizer but to emphasize “a high degree of quality”, cf.~\textit{godak} ‘short’ : \textit{godak-ti} ‘very short’ \citep[267]{alekseev1994b}. In \ili{Rutul}, the cognate marker \textit{-d} is used as an anti\hyp{}construct state marker on attributive adjectives as the default \citep[224]{alekseev1994a}. A different but nevertheless related function of the cognate marker is attested in \ili{Archi} where the suffix \textit{-t̄u} derives adjectives from nouns, adverbs and postpositions \citep[318]{kibrik1994b}.

\ia{Himmelmann, Nikolaus|(}
The data from Lezgic deserves further investigation, but it suggests a pattern where the dependend-marking attributive state evolves from attributive nominalization. It is also very obvious that the attributive nominalizers in Uralic\il{Uralic languages} and Turkic\il{Turkic languages} have evolved along a similar grammaticalization path as the one described for several Indo-European\il{Indo-European languages} (and other) languages by \cite{himmelmann1997}. Important differences between Himmelman's “linking articles”\is{attributive article}  and the attributive nominalizers described here, however, are (1) the origin of the Uralic and Turkic nominalizers from person-deictic rather than from local-deictic markers and (2) the inflectional use of the markers in Uralic and Turkic as compared to their original adnominal use in Indo-European.
\il{Lezgic languages|)}

\is{species marking!definite|(}
The data from Uralic\il{Uralic languages} and Turkic\il{Turkic languages} is especially interesting, since it contradicts Himmelmann's (\citeyear[220–221]{himmelmann1997}) assumption that a functional convergence between attributive nominalizers with a person-deictic or a local-deictic etymological source is unlikely to occur. Of central importance to Himmelmann's analyses is the “anamnestic” use of the deictic markers from which the articles are grammaticalized. According to Himmelmann, the use of “D(eictic) elements” in order to refer to properties the speaker believes to be well-known for her/his interlocutor is the most relevant precondition for their further grammaticalization into articles and definite markers. Whereas the “anamnestic” use is inherent in (local-deictic) demonstratives, the same is not true for (person-deictic) possessive markers. The further grammaticalization of demonstratives into functional determinative elements (like articles and definiteness markers in several \ili{Indo-European languages}) is accompanied by a functional extension of an original “anamnestic” to an associative-anaphoric use of the markers. This is in contrast to the further grammaticalization of possessive markers into functional determinative elements (like attributive articles and quasi-definiteness markers in certain \ili{Uralic languages}) which is accompanied by a functional extension from an original associative-anaphoric to “anamnestic” use.
%%%
\begin{quote}
D-Elemente breiten sich von pragmatisch-definiten Kontexten auf semantisch\hyp{}definite aus, während Possessivpronomina sich umgekehrt von einem semantisch\hyp{}definiten Kontext auf einen bzw. mehrere pragmatisch\hyp{}definite Kontexte ausdehnen. \citep[221]{himmelmann1997}
\end{quote}
%!!Translation; check also the other quotes
%%%
Himmelmann's thesis regarding the opposite functional extension of person-deictics might still be valid and compatible with the Uralic and Turkic data. In those Uralic\il{Uralic languages} and \ili{Turkic languages} with attested attributive nominalization, the definite (or quasi-definite) function of the possessive marker is also always present. It can therefore be assumed that the definite (or quasi-definite) use of the marker obligatorily occurs as an intermediate step during the grammaticalization of possessive markers to attributive nominalizers.
\ia{Himmelmann, Nikolaus|)}
\is{species marking!definite|)}
%%%
\il{Uralic languages}
\il{Turkic languages}
\begin{itemize}
\item \textbf{Person-deictic source} (Uralic, Turkic)\\
	\textsc{poss $\Rightarrow$ def $\Rightarrow$ nmlz}
\end{itemize}
%%%
In the Indo-European languages with attributive articles such an intermediate step is probably not necessary.
\il{Indo-European languages}
\begin{itemize}
\item \textbf{Local-deictic source} (Indo-European)\\
	\textsc{dem ($\Rightarrow$ def) $\Rightarrow$ nmlz}
\end{itemize}
%%%
In fact, in the West Germanic\il{West Germanic languages} and \ili{South Slavic languages}, definite markers evolve from attributive nominalizers but not vice versa.\is{species marking!definite}
%%%
\il{West Germanic languages}
\il{South Slavic languages}
\begin{itemize}
\item \textbf{Local-deictic source (West Germanic, South Slavic)}\\
	\textsc{dem $\Rightarrow$ nmlz ($\Rightarrow$ def)}
\end{itemize}
%%%
This observation will be taken up again. If the tentative observation on the languages with “grammaticalized person-deictic elements” (i.e., possessive markers as attributive nominalizers) proves right it would imply the following implicational universal:
%%%
\begin{exe}
\label{universal}
\ex {\rm \textbf{Implicational universal}}\\
	\textit{Possessive markers develop into attributive nominalizers only in languages in which similar possessive markers are already used as markers of (quasi-) definiteness.}
\end{exe}
%%%
Whereas the etymology and the evolution of attribution markers in Indo\hyp{}European has been described (more or less systematically) by different authors, much less has been written about the emergence of attribution markers in different Uralic and Turkic languages. The emergence of anti\hyp{}construct state marking in Saamic, which has not been described at all, appears to be especially interesting in this respect.
\is{attributive nominalization|)}

\il{Saamic languages|(}
\section[Anti\hyp{}construct state in Saamic]{The emergence of anti\hyp{}construct state marking in Saamic}
\label{saamic diachr}
%%%
In \S\ref{udmurt diachr}, it was shown that the contrastive focus marker in \ili{Udmurt} most likely evolved from an attributive article. \cite{riesler2006b} suggested the idea that a similar construction was the ultimate source of anti\hyp{}construct state marking in the languages of the relatively closely related Saamic branch of Uralic. Since this theory about the rise of attribution marking in Saamic is based on a controversial idea, it calls for a relatively detailed discussion which will be presented in the following sections.

In \S\ref{saami synchr}, it was shown that adjectives in all Saamic languages are normally marked morpho-syntactically by means of differentiated attributive and predicative state markers. Even though the system of attributive and \isi{predicative marking} is highly irregular in the Saamic languages, it can be shown that the attributive forms of adjectives are prototypically marked with a suffix (\ili{Northern Saami}) \textit{-s}. This suffix constitutes a prototypical example of an anti\hyp{}construct state marker, i.e., a dependent marking attributive morpheme.

The origin of anti\hyp{}construct state marking in Saamic is controversial. The suffix \textit{-s} is definitely not inherited from \ili{Proto\hyp{}Uralic}. It is probably not borrowed from any of the known current or historical contact languages of Saamic either. Considering this as well as the fact that Saamic is a rare instance among the Northern-Eurasian languages in exhibiting anti\hyp{}construct state marking on adjectives, relatively little attention has been paid to explaining its origin.

\subsection{State of research}
%%%
The different proposed theories which explain the origin of the anti\hyp{}construct state marker on adjectives in Saamic can be subsumed as follows:
%%%
\begin{enumerate}
\item Grammatical borrowing from Indo-European
\item Functional extension of an adjective derivational marker
\item Grammaticalization from an attributive nominalizer\is{attributive nominalization}
\end{enumerate}
%%%
The idea about a grammaticalization from an attributive nominalizer presented by Nielsen (\citeyear{nielsen1933}) and Atányi (\citeyear{atanyi1942,atanyi1943}) is the only contribution to the subject spelled out in certain detail. Interestingly enough, the idea has been rejected as “hardly convincing” (my translation) in a one-sentence-statement in Korhonen's (\citeyear{korhonen-m1981}) historical grammar of Saami. Korhonen's judgement that the origin of the attributive suffix in Saamic is still unclear \cite[246]{korhonen-m1981} seems to reflect the state of research up to today. Neither of the three hypotheses mentioned above has been discussed seriously in Saami or Uralic historical linguistics.\footnote{An exception is a short article by \cite{sarv-m2001} who presents the different ideas but does not come to conclusive results.}. All proposed hypothesis will be evaluated.

\subsubsection{Loan adjectives} 
Trond Trosterud\ia{Trosterud Trond} (p.c.) has suggested that the attributive suffix in Saamic origins from an ending typical of \ili{Proto\hyp{}Germanic} loan adjectives in Saami. The Saamic suffix \textit{-s} would then reflex the (pre-rhotacism) form of the \ili{Proto\hyp{}Germanic} case suffix \textit{-R} for masculine nominative singular which was adopted into \ili{Proto\hyp{}Saamic} together with loan adjectives. According to this hypothesis (which is not discussed in any publication so far) the adjective ending \mbox{\textit{-s}} occurred originally on Germanic loan adjectives but was later generalized and used with inherited adjectives as well. In fact, a considerable number of Germanic\il{Germanic languages} loan adjectives with the corresponding ending \textit{-s} <~\ili{Proto\hyp{}North Germanic} \textit{-R} \textsc{m.nom.sg} is attested in Saamic, for instance:
%%%
\begin{itemize}
\item \ili{Northern Saami} \textit{smáves} ‘small’ $\Leftarrow$ \ili{Proto\hyp{}Saamic} \textit{*smāv̀e̮} <~\ili{Proto\hyp{}North Germanic}; cf.~\ili{Old Norse} \textit{smalr} \textsc{m} (or a more recent North Germanic\il{North Germanic languages} borrowing; cf.~\ili{Swedish} \textit{sm\aa}; \citealt[263]{sammallahti1998b})
\item \ili{Lule Saami} \textit{riukas} ‘far-reaching’ <~\ili{Proto\hyp{}North Germanic}, cf.~\ili{Old Norse} \textit{drùgr}, \ili{Norwegian} \textit{drjug} \cite[267]{qvigstad1893}
\item \ili{Lule Saami} \textit{lines} ‘soft, yielding, mild’ <~\ili{Proto\hyp{}North Germanic}, cf.~\ili{Old Norse} \textit{linr}, \ili{Norwegian} \textit{lin} \cite[218]{qvigstad1893}
\item \ili{Northern Saami} \textit{luov\.{o}s $\sim$ luovus} ‘loose, not tied’ $\Leftarrow$ \ili{Proto\hyp{}Saamic} \textit{*luovōs $\sim$ *luove̮s} <~\ili{Proto\hyp{}North Germanic} \textit{*lauss} \textsc{m} (where the suffix \textit{-R} is assimilated into /s/) \cite[264]{sammallahti1998b}
\item \ili{Northern Saami} \textit{suohtas} ‘fun, nice’ $\Leftarrow$ \ili{Proto\hyp{}Saamic} \textit{*suohte̮s} <~\ili{Proto\hyp{}Germanic} \textit{*swōtu-} \cite[264]{sammallahti1998b}, cf.~\ili{Old Norse} \textit{*søtr} \textsc{m} 
\item \ili{Northern Saami} \textit{viiddis} ‘wide, extensive’ $\Leftarrow$ \ili{Proto\hyp{}Saamic} \textit{*vij{\dh}ēs} <~\ili{Proto\hyp{}North Germanic} \cite[148–149]{lehtiranta1989}, cf.~\ili{Old Norse} \textit{v\'i{\dh}r} \textsc{m}
\end{itemize}
%%%
The sound change of \ili{Proto\hyp{}Germanic} \textit{*-z} $\Rightarrow$ \ili{Proto\hyp{}North Germanic} \textit{-R} ($\Rightarrow$ \ili{Common North Germanic} \textit{-r}) took place around 500 AD. The hypothesis of the loan origin of the Saamic attributive suffix presupposes that the corresponding suffix in Germanic\il{Germanic languages} had a sound value [-z] (or ?[-s]). The exact sound value of \textit{-R}, however, is not at all certain. What is commonly accepted is that the sound was phonologically distinguished from /r/ \citep{skold1954}.

From the point of view of its etymology, the adjective ending \textit{-s} is identical to the ending \textit{-s} of some borrowed \ili{Proto\hyp{}Germanic} nouns, such as \ili{Proto\hyp{}Saamic} \textit{*vālās}, cf.~\ili{Northern Saami} \textit{fàlis} ‘whale’ <~\ili{Proto\hyp{}North Germanic}, cf.~\ili{Old Norse} \textit{hvalr}, cf.~\ili{Norwegian} \textit{hval} (\citealt[144]{qvigstad1893}; \citealt[144–145]{lehtiranta1989}) or \ili{Proto\hyp{}Saamic} \textit{*kāllēs}, cf.~\ili{Northern Saami} \textit{gállis} ‘old man’ <~\ili{Proto\hyp{}Germanic} \textit{*karilaz} \textsc{m} \cite[44–45]{lehtiranta1989}. The ending \textit{-s} in bisyllabic nominals is thus an indicator that the word in question might belong to the layer of \ili{Proto\hyp{}North Germanic} borrowings in Saamic.

\is{predicative marking|(}
In many instances of Germanic loan adjectives the ending \textit{-s}, however, marks only the predicative and not the attributive form, consider (from the list above):
%%%
\begin{itemize}
\item \ili{Northern Saami} \textit{smávva} [small.\textsc{attr}] $\leftarrow$ \textit{smáves} ‘small’
\item \ili{Lule Saami} \textit{riuka} [far-reaching.\textsc{attr}] $\leftarrow$ \textit{riukas} ‘far-reaching’
\item \ili{Lule Saami} \textit{littna} [soft.\textsc{attr}] $\leftarrow$ \textit{lines} ‘soft’
\end{itemize}
%%%
Other loan adjectives have identical forms with the ending \textit{-s} in both predicative and attributive function:
%%%
\begin{itemize}
\item \ili{Northern Saami} \textit{luov\.{o}s $\sim$ luovus} ‘loose'
\item \ili{Northern Saami} \textit{suohtas} ‘fun, nice'
\item \ili{Northern Saami} \textit{viiddis} ‘wide, extensive'
\end {itemize}
%%%
It is unclear whether the Germanic\il{Germanic languages} loan adjectives ending in \textit{-s} regularly occurred in both attributive and predicative positions already in \ili{Proto\hyp{}Saamic}, or the ending \textit{-s} expanded from predicative to attributive forms, or vice versa.

The relatively regular occurrence of the ending \textit{-s} in the predicative forms suggests that the corresponding Germanic\il{Germanic languages} loan adjectives also ending in \textit{-s} were originally used to denote predicates rather than attributes. This seems reasonable from the point of view of the morpho-semantics of the borrowed Germanic\il{Germanic languages} adjectives as well. The ending \textit{-R} ($\Leftarrow$ \textit{*z}) marks masculine nominals only in the so-called strong declension and thus occurred more likely on predicative adjectives which normally denote temporary properties. Attributive adjectives in Germanic,\il{Germanic languages} by contrast, could be marked either by means of \isi{head\hyp{}driven agreement} (“strong declension”) or anti\hyp{}construct state agreement (“weak declension”) depending on the semantic or referential status of the attribute. An adjective denoting a permanent property was normally marked with the anti\hyp{}construct state agreement suffix (see \S\ref{germanic diachr}).

Consequently, the Saamic ending \textit{-s} could have been borrowed exclusively from “strong” adjectives in masculine nominative singular, the only form which had the ending \textit{-R} ($\Leftarrow$ \textit{*z}) in \ili{Proto\hyp{}North Germanic}. It is thus doubtful that just the borrowed forms with \textit{-s} have been generalized as attributive forms by bilingual speakers in the assumed Saamic-Germanic language contact situation.\footnote{There is no doubt that language contact between speakers of Proto\hyp{}Saamic and Proto\hyp{}North Germanic took place; cf.~\citealt{kusmenko2008}. It is, however, rather irrelevant to the case described here which contact scenario has to be assumed: borrowing proper or shift-induced interference in the Saamic L2 of original Germanic speakers.} It should thus be assumed that the Germanic loan etymology of certain adjectives in Saamic does not provide a clue for the origin of the attributive suffix. 

Another problem in the hypothesis of the Germanic origin of the Saamic adjective ending \textit{-s} might be the class of inherited Saamic adjectives which also have the ending \textit{-s} when used predicatively. Consider the following examples:
%%%
\begin{itemize}
\item \ili{Northern Saami} \textit{báhkas} ‘hot’ $\leftarrow$ \textit{báhkka} [hot.\textsc{attr}] $\Leftarrow$ \ili{Proto\hyp{}Saamic} \textit{*pāhke̮s} $\Leftarrow$ \ili{Pre-Proto\hyp{}Saamic} \textit{*pakka-} 'hot; cold’; cf.~Finnish \textit{pakkanen} ‘frost’ \citep[230]{sammallahti1998b}
\item \ili{Northern Saami} \textit{garas} ‘hard’ $\leftarrow$ \textit{garra} [hard.\textsc{attr}] $\Leftarrow$ \ili{Proto\hyp{}Saamic} \textit{*ke̮\`re̮-} $\Leftarrow$ \ili{Pre-Proto\hyp{}Saamic} \textit{*kiri-}; cf.~Finnish \textit{kireä} ‘tight, tense’ \citep[242]{sammallahti1998b}
\item \ili{Northern Saami} \textit{o{\dj}as} ‘new’ $\leftarrow$ \textit{o{\dj}{\dj}a} [new.\textsc{attr}] $\Leftarrow$ \ili{Proto\hyp{}Saamic} \textit{*o\`{\dh}e̮-} \citep[258]{sammallahti1998b}.
\end{itemize}
%%%
Since the most typical \ili{Proto\hyp{}Saamic} root can be reconstructed as an open bisyllabic,\footnote{Cf.~the list of reconstructed Proto\hyp{}Saamic lexemes in \cite{lehtiranta1989}.} the ending \textit{-s} of these predicative adjectives could not have belonged to the root originally. The ending-less attributive forms in the examples above would then reflect the original adjective roots, characterized as bisyllabics with an open second syllable. According to the \ili{Proto\hyp{}Saamic} morpho-phonological rules, the stem consonant center exhibits the strong grade before an open second syllable, unlike the predicative forms which have a closed second syllable ending in \textit{-s} and show the weak grade of the consonant center.

The same morpho-phonological rule applies to loan adjectives with ending-less attributive forms (like ‘small’ in \ili{Northern Saami}: \textit{smávva} [small:\textsc{attr}] $\leftarrow$ \textit{smáves}). If one adopts the idea of \textit{-s} originally being a Germanic\il{Germanic languages} case suffix, the attributive forms of the loan adjectives in Saamic can only be derived from the strong-declension forms of Germanic\il{Germanic languages} predicative adjectives and not from attributive adjectives.

In the case of the inherited Saamic adjectives, however, it is usually assumed that the predicative ending \textit{-s} is derivational (see also the following paragraph). This assumption presupposes the ending-less (attributive) adjective being the base form from which the predicative form is derived by means of the derivational ending \textit{-s}. %But this would likewise fit the morpho-phonological rules with weak consonant stem grades in the attributive forms.
\is{predicative marking|)}

\subsubsection{Locative adjective derivation}
According to \citet[96]{bergsland1946}, the origin of the attributive suffix \textit{-s} in Saamic is identical with that of the synchronically homophonous adjective derivational suffix \textit{-s} originating from a lative case marker. Cognate formatives deriving adjectives from nouns occur in other \ili{Uralic languages}, like \ili{Hungarian} \textit{erős} ‘powerful, strong’ ($\leftarrow$ \textit{erő} ‘power, strength’), \textit{kékes} ‘bluish’ ($\leftarrow$ \textit{kék} ‘blue’).

The development of local case expressions to adjectives is semantically plausible and could in principle be adopted for Saamic. Probably, the local case suffix was first used as adverbalizer of nominal stems and became a true adjectivizer at a later stage, hence:\is{adjective derivation}
%%%
\begin{itemize}
\item \textsc{lative case} $\Rightarrow$ \textsc{adverbalizer} $\Rightarrow$ \textsc{adjectivizer}
\end{itemize}
%%%
The intermediate stage in the assumed development from a local case expression to an adjective is reflected in place adverbs like \ili{Northern Saami} \textit{guhkás} ‘(going) far’ $\Leftarrow$ \ili{Proto\hyp{}Saamic} \textit{*kuhkā-se̮} \cite[246]{sammallahti1998b} and probably also in other adverbal derivations, like the collective numbers on \textit{-s}, cf.~\ili{Northern Saami} \textit{golmmas} ‘a group of three’ $\leftarrow$ \textit{golbma} ‘three’.

\is{predicative marking|(}
Since predicative adjectives are not subject of this investigation, the the observation is sufficient that both the assumed (inherited) locative derivation and the assumed suffix borrowing are possible scenarios which do not necessarily exclude each other. As a result of these developments, a lexically defined subclass of adjectives with predicative forms on \textit{-s} arose in \ili{Common Saamic} (or earlier). The marker of this class of adjectives, the ending \textit{-s}, is either:
%%%
\begin{itemize}
\item borrowed from <~\ili{Proto\hyp{}North Germanic} \textit{-R} \textsc{m.nom.sg}
\item derived (historically) from $\Leftarrow$ \textsc{lative case}, %gib die Etymologie vom LATIVE
\item the result of merger of both developments.
\end{itemize}
%%%
\noindent The adjective class characterized by predicative forms on \textit{-s} (which has more or less regular ending-less attributive forms) is clearly identifiable in all modern Saamic languages.

Bergsland's (\citeyear[96]{bergsland1946}) suggestion that the similar ending \textit{-s} in the attributive forms of certain adjectives goes back to the Uralic lative case suffix as well is relevant to the present investigation. Deduced from his statement that the attributive suffix \textit{-s} is “originally a Finno-Volgaic lative suffix” Sammallahti (\citeyear[71]{sammallahti1998b}) agrees with Bergslands explanation. Also \cite{judakin1997} argues in this direction.

The adjective ending \textit{-s}, which is the basis for Bergsland's and Sammallahti's argumentation, marks the predicative form of some adjectives and the attributive form of others. There are only a few adjectives which have the ending \textit{-s} in both predicative and attributive forms. Neither Bergsland\ia{Bergsland, Knut} nor Sammallahti\ia{Sammallahti, Pekka} discuss the question as to whether the assumed lative derivation originally occurred: a) on predicative adjectives, b) on attributive adjectives, or c) on both forms simultaneously.
 
A cross-comparison of cognate forms of attributive and predicative adjectives in different Saamic languages suggests that adjectives with similar predicative and attributive forms with \textit{-s} form a minor class which very likely arose as the result of a secondary development.

Cross-comparison can also provide evidence for separate etymologies of two homophonous predicative and attributive endings \textit{-s}. The locative derivational suffix can only be the source of this suffix \textit{-s} which is homophonous on predicative and attributive adjectives in modern West Saamic languages. The original attributive adjective suffix, however, should be reconstructed as a (phonetically palatalized) suffix *[-sVʲ\textsubscript{[+front]}] preceding a front vowel. In the easternmost \ili{Kola Saami languages}, the attributive suffix \textit{-s'} has a palatalized coda and is clearly distinct from the non-palatalized \textit{-s} on predicative adjectives as well as from the (cognate) lative adverbalizer \textit{-s}.
%%%
\il{Kildin Saami}
\il{Northern Saami}
\begin{exe}
\settowidth\jamwidth{(Northern Saami)}
\ex 
\begin{xlist}
\ex	{\rm Adjective stem ‘long (pred.)’}\\
	\textit{guhkki}	\jambox{ {\rm Northern Saami} }
	\textit{kuhk'}	\jambox{ {\rm Kildin Saami} }
%%%
\ex 	{\rm Adverb ‘(going) far’}\\
	{\rm (adverbalizer suffix (non-palatalized) $\Leftarrow$ \textit{*-s})}\\
	\textit{guhkás}	\jambox{ {\rm Northern Saami} }
	\textit{kugkas}	\jambox{ {\rm Kildin Saami} }
%%%
\ex 	{\rm Attributive form ‘long (attr.)’}\\
	{\rm (attributive suffix (palatalized) $\Leftarrow$ \textit{*-s'})}\\
	\textit{guhkes}	\jambox{ {\rm Northern Saami} }
	\textit{kugk'es'}	\jambox{ {\rm Kildin Saami} }
\end{xlist} 	
\end{exe}
\is{predicative marking|)}

\is{attributive nominalization|(}
\subsubsection{Attributive nominalization}
A different hypothesis about the origin of the attributive forms in Saamic has been proposed by Joszéf Budenz (\citeyear{budenz1870}; according to \citealt{atanyi1942,atanyi1943}) who believed that the suffix \textit{-s} represents the original possessive suffix 3\textsuperscript{rd} person singular. Budenz does not give any evidence specifically for Saami. He simply assumes that the determinative function of the possessive suffix, a similar use of which he observed in different Uralic\il{Uralic languages} and \ili{Turkic languages} (see \S\ref{uralic-turkic diachr}), caused the development in Saami. Budenz' idea was taken up specifically for Saamic by Atányi (\citeyear{atanyi1942}, \citeyear{atanyi1943}). Atányi also refers to Nielsen (\citeyear{nielsen1933}, reprinted in \citealt{nielsen1945b}), who had a similar idea (probably independently of Budenz, who he does not refer to).

This hypothesis on the origin of the attributive forms in Saamic perfectly accounts for the different phonological shapes of the (historical) adjectivizer\is{adjective derivation} \mbox{\textit{*-s}} and the attributive suffix \textit{-s} ($\Rightarrow$ E-Saamic \textit{-s'}). According to this theory, recently taken up again by \cite{riesler2006b}, the attributive suffix \textit{-s/-s'} reflects an old 3\textsuperscript{rd} person singular possessive suffix which was used as an attributive article on contrastive-emphasized adjectives.

The reconstructed \ili{Proto\hyp{}Saamic} forms of the possessive marker \textit{*-sē} \cite[73]{sammallahti1998b} versus the adjectivizer\is{adjective derivation} \textit{*-se̮} are consistent with the synchronic findings. The different phonological form of the two suffixes (/-s\textsuperscript{j}/ versus /-s/) in the \ili{Kola Saami languages} and the phonological merger of both suffixes (non-palatalized /-s/) in the western Saamic languages can be accounted for by a regular sound law: in the \ili{Kola Saami languages} the apocope of etymologically front vowels (\textit{*i, *e}) is reflected by the palatalization of the consonant preceding the lost vowel. Apocope of non-front vowels (like \textit{*-se̮}) did not affect the quality of the consonant. This sound law does not apply to the western Saamic languages which do not exhibit (phonological) palatalization and consequently consonants preceding etymologically front and back vowels are non-palatalized.
%%%
\begin{exe}
\settowidth\jamwidth{(Northern Saami)}
\ex
\begin{xlist}
\ex {\rm ‘guest’ (not possessed)}
\begin{xlist}
\ex[*]{kuasse		\jambox{ {\rm \ili{Proto\hyp{}Saamic}} }}
\ex[]{kuss'			\jambox{ {\rm \ili{Kildin Saami}} }}
\ex[]{guossi		\jambox{ {\rm \ili{Northern Saami}} }}
\end{xlist}
%%%
\ex {\rm ‘her/his/its guest’ (marked with \textsc{poss:3sg} suffix)}
\begin{xlist}
\ex[*]{kuasse-sē	\jambox{ {\rm \ili{Proto\hyp{}Saamic}} }}
\ex[]{kuss'es'		\jambox{ {\rm \ili{Kildin Saami}} }}
\ex[]{guossis		\jambox{ {\rm \ili{Northern Saami}} }}
\end{xlist}
\end{xlist}
\end{exe}
%%%
Beside the overall irregularity in the attributive marking in all Saamic languages (see \S\ref{saami synchr}), the different morpho-phonological behavior of the nominal stems which \textsc{poss:3sg} and \textsc{attr} attach to appears to be an argument against this reconstruction. 
%%%
\il{Kildin Saami}
\il{Northern Saami}
\begin{exe}
\settowidth\jamwidth{(Northern Saami)}
\ex {\rm Strong (\textsc{str}) and weak (\textsc{wk}) consonant grade in adjectives and nouns}
\begin{xlist}
\ex 
\glll 	kugk'(\textsc{wk})-es' suhk(\textsc{str}) 		\jambox{ {\rm Kildin Saami} }\\
	guhke(\textsc{wk})-s suohkku(\textsc{str}) 		\jambox{ {\rm Northern Saami} }\\
	long-\textsc{attr} stocking\\
\glt	‘the long stocking’
\ex 
\glll	suhk(\textsc{str})-es' lī kuhk'(\textsc{str}) 		\jambox{ {\rm Kildin Saami} }\\
	suohkku(\textsc{str})-s lea guhkki(\textsc{str}) 	\jambox{ {\rm Northern Saami} }\\
	stocking-\textsc{poss:3sg} is long.\textsc{pred.}\\
\glt	‘her stocking is long’
\ex
\glll	kugk'(\textsc{wk})-es' sugk(\textsc{wk})-es't 	\jambox{ {\rm Kildin Saami} }\\
	guhke(\textsc{wk})-s suohku(\textsc{wk})-s 	\jambox{ {\rm Northern Saami} }\\
	long-\textsc{attr} stocking-\textsc{loc.sg}\\
\glt	‘in the long stocking’
\end{xlist}
\end{exe}
%%%
A noun marked for possession is in the strong consonant grade. An adjective marked for attribution is always in the weak grade. In the example above, the strong grade of the consonant (orthographically represented as \textit{hk} in Kildin Saami and \textit{hkk} in \ili{Northern Saami}) occurs in the nominative case of the bare or possessive marked noun (\textit{suhk/suohkku, suhkes'/suohkkus}) as well as in the predicative form\is{predicative marking} of the adjective (\textit{kuhk'/kuhkki}). The attributive form of the adjective (\textit{kugk'/guhkes}) and the noun stem hosting the locative suffix (\textit{sugkes't/suohkus}) are in the strong grade.

Historically, consonant gradation was a purely phonological process where the strong consonant grade always occurred before the open final syllable of a disyllabic word. The stem consonant was phonetically shortened when the final open syllable was closed due to inflectional processes. Consonant gradation was later morphologized due to phonological attrition and the loss of certain inflectional suffixes.

From a synchronic point of view, the consonant gradation rules account for the weak consonant grade in the attributive form of the adjective but not for the strong grade in the noun with possessive marking. The \ili{Northern Saami} words \textit{suohkku} ‘stocking’ and \textit{guhkki} ‘long (pred.)’ have open second syllables hence strong consonant stems (here a consonant cluster, the first part of which is a geminate /\=CC/). The second syllable in both forms is closed: \textit{suohkkus} /suoh:.ku-s/ marked with the possessive suffix and \textit{guhkis} /kuh.ki-s/ marked with the attributive suffix. However, the consonant stem of the noun \textit{suohkkus} remains strong (/\=CC/) even before the syllable closing suffix, whereas the geminate part of the cluster is shortened (/CC/) in the adjective \textit{guhkis}.

It is important to note that the possessive suffix is reconstructed as \ili{Proto\hyp{}Saamic} \textit{*-sē} \cite[73]{sammallahti1998b} and thus originally had a different syllable structure. The formative obviously did not close the second syllable in \ili{Proto\hyp{}Saamic}, as in **/kuh:.ke.-sē/ and **/suoh:.ku.-sē/.\footnote{Note that these invented examples in simplified transcriptions serve the purpose of illustration (and are hence marked with **). The stem of the adjective ‘long’ is reconstructed as Proto\hyp{}Saamic \textit{*ku\`{h}kē} \cite[246]{sammallahti1998b}. The noun ‘stocking’ is a loan word (cf.~\ili{Swedish} (dialectal) \textit{sokk}, \ili{Finnish} \textit{sukka}) and might not be reconstructable for Proto\hyp{}Saamic.} From a diachronic point of view, the consonant gradation rules would thus account for the strong consonant grade in the noun marked with a possessive suffix but not for the weak grade in the attributive adjective.
 
Two possible explanations could explain the different consonant grades in the noun and the adjective marked by means of \textit{-s} $\Leftarrow$ \textit{*-sē}.
%%%
\begin{itemize}
\item Following Nielsen (\citeyear{nielsen1945b}), the possessive marker in its function as attributive nominalizer was originally attached to a genitive (i.e., weak stem) form of the adjective. The weak consonant stem was thus triggered by the genitive suffix, reconstructed as \ili{Pre-Proto\hyp{}Saamic} \textit{*-n} $\Rightarrow$ \ili{Proto\hyp{}Saamic} \mbox{\textit{*-Ø}} \cite[65]{sammallahti1998b} and preceding the attributive marker. The date of the morphologization of stem gradation would not be relevant for this explanation.
\item The other possible explanation presupposes a relatively late date for the morphologization of stem gradation, i.e., not earlier than the apocope of the possessive marker's final vowel (\textit{-s $\Leftarrow$ *-sē}). If the possessive marker was not a true suffix but a phonological word on its own by the time stem gradation was morphologized in Saamic, the marker would have remained outside the phonological domains of its host word and would not have been able to trigger stem gradation on the latter. 
\end{itemize}
%%%
Since genitive (or “possessor case”) marking on attributive adjectives is attested in other northern Eurasian languages, as in both Yukaghir\il{Yukaghir languages} (see \S\ref{yukagir synchr}) and in \ili{Lezgic languages} (see \S\ref{lezgian synchr}), Nielsen's assumption that the 3\textsuperscript{rd} singular possessive marker was originally attached to an attributive form of adjectives (or other nominals) in genitive is principally possible.

Yet there is no evidence that genitive attribution marking on adjectives ever occurred regularly in Saamic or even in other \ili{Uralic languages}.\footnote{The “defective” agreement paradigm\is{agreement marking!defective agreement paradigm} of pronouns (and even sometimes adjectives) with the genitive singular form in all cases except nominative singular can scarcely be connected to Nielsen's idea. As an anti\hyp{}construct state marker, the “genitive” should occur through the whole paradigm including in nominative singular.} Furthermore, the functional side of the assumed development, in which an adjective marked by two attributive markers (genitive+attributive nominalizer) simultaneously, would also need some further clarification.

The second hypothesis assuming that the possessive marker never triggered stem gradation, could also account for the weak consonant grade in adjectives (remember that the weak grade seemed to contradict the stem gradation rules from a historical point of view). In certain aspects, the possessive marker behaves like a free pronoun rather than like an affix: the possessive marker shows pronominal agreement (and hosts the agreement suffixes which co-reference the number of the possessor) but the marker itself is hosted by an inflected noun (marked for number and case of the possessed). Note also that the possessive inflection is morpho-syntactically different from case and number inflection in the closely related \ili{Finnic languages}. Only the latter features trigger noun phrase internal agreement.

Only the 3\textsuperscript{rd} person singular possessive marker was used as an attributive nominalizer. Since this marker was hosted by uninflected adjectives, it is reasonable to assume that at one point the nominalizing possessive marker behaved differently from true possessive markers. The attributive nominalizer might thus have become a true phonologically bound formative earlier than the homophonous possessive marker. As a result of the apocope of the suffix-final vowel, the second syllable in the attributive form was closed:
%%%
\begin{exe}
\ex 
\gll	**/kuh:.ke.-sē/ $\Rightarrow$ **/kuh.ke-s/\\
	long-\textsc{poss:3sg} {} long-\textsc{attr}\\
\end{exe}
%%%
Subsequently, the stem gradation rules were applied regularly and yielded the short consonant grade of the adjective stem equipped with the affixal attributive marker. The noun equipped with the possessive marker, however, kept its open second syllable even after the apocope. The non-affixal possessive suffix – as a phonological word of its own – remained outside the phonological domain of stem gradation.
%%%
\begin{exe}
\ex 
\gll	**/suoh:.ku.=sē/ $\Rightarrow$ **/suoh:.ku.=s/\\
	stocking=\textsc{poss:3sg} {} stocking=\textsc{poss:3sg}\\
\end{exe}
\is{attributive nominalization|)}

\subsection{The origin of anti\hyp{}construct state in Saamic}
%%%
Synchronic data from related \ili{Uralic languages} provide good evidence in favor of the assumed grammaticalization path from possessive to anti\hyp{}construct state marking in Saami.
%%%
\begin{itemize}
\item \textsc{possessive} (\textsc{3sg}) $\Rightarrow$ \textsc{attributive nominalization} $\Rightarrow$ \textsc{anti}-\textsc{construct}
\end{itemize}
%%%
The first step of this development, i.e., the use of the possessive marker as an attributive article, is attested in the Permic languages \ili{Komi-Zyrian} and \ili{Udmurt}. Note also that the possessive marker in \ili{Udmurt} shows different morphological behavior depending on its function as a true possessive or as an attributive article. For more detail see the respective sections on the synchrony (\S\ref{udmurt synchr}) and diachrony (\S\ref{udmurt diachr}) of attribution marking in \ili{Udmurt}.

The \ili{Permic languages} are closely related to Saamic and, theoretically, the rise of attributive marking in these two branches of Uralic could go back to a common \ili{Proto\hyp{}Uralic} construction. True evidence to prove such a common development at a relatively early time is, however, missing. Quite the contrary, it could be objected that the innovation of a new type of attribution marking is currently under way in the \ili{Permic languages} whereas the innovation in Saamic took place 2000 years ago and is obviously loosing ground today in favor of the re-introduced type \isi{juxtaposition}.

But the comparison with the related \ili{Permic languages} makes sense from a purely typological perspective. Assuming that the possessive marker already had a “determinative” function in \ili{Proto\hyp{}Uralic} (as stated, for instance, by \citealt[32]{janhunen1981}; \citealt[66, 81]{decsy1990}; \citealt{kunnap2004}) and that this function is still present in most of the modern Uralic languages, the existence of an attributive nominalizer in Permic indisputably proves that the proposed origin of the attribution marker in Saamic is functionally plausible \citep{riesler2006b}.\is{attributive nominalization}

Furthermore, the nominalizing function of the (person-deictic) marker of possession is attested not only in several \ili{Uralic languages} but also in \ili{Turkic languages}. And, finally, a typologically similar grammaticalization path of a (local-deictic) demonstrative to an attributive article is also attested in \ili{Indo-European languages} of the area.

In all mentioned Turkic, Uralic and Indo-European languages where the development of attributive nominalizers is attested, this innovative type of attribution marking originally co-occurred with another, inherited type. The use of contrastive pairs of attributes marked with or without the anti\hyp{}construct state marker in modern Saamic languages provides good evidence for a similar development in earlier stages of Saami.\is{attributive nominalization}

Several grammatical descriptions of \ili{Northern Saami} give examples of such contrastive pairs of attributes with different meanings. Nielsen describes the difference between forms with and forms without an attributive suffix as a difference in “modality” of the attributive relation \cite[203]{nielsen1945b}. Most examples, however, do not display true adjectives but rather attributive forms of present participles. If the property denoted by the participle is stressed or emphasized as belonging permanently to the referent of the modified noun the participles are often equipped with the attributive suffix.
%%%
\begin{exe}
\il{Northern Saami}
\ex
\begin{xlist}
\ex {\rm Northern Saami \citep[204]{nielsen1945b}}
\begin{xlist}
\ex
\gll 	juhhki olmmoš\\
	drinking person\\
\glt	 ‘drinking person’
%%%
\ex	
\gll	juhkke\textbf{-s} olmmoš\\
	drinking-\textsc{attr} person\\
\glt	‘alkoholic (i.e., a person addicted to drinking)’
\end{xlist}
%%%
\ex {\rm Northern Saami \citep[282]{bartens1989}}
\begin{xlist}
\ex 	
\gll	šaddi soahki – soahki lea šaddi\\
 	growing birch – birch is growing\\
\glt	‘growing birch’ – ‘(a/the) birch is growing’
%%%
\ex
\gll	Goa{\dj}i duohkin lea šaddi\textbf{-s} soahki.\\
	hut behind is growing-\textsc{attr} birch\\
\glt	‘There is a fast growing birch behind the hut.’
\end{xlist}
\end{xlist}
%Reading people
\end{exe}
%%%
Besides participles, there are even contrastive pairs of attributive adjectives or nouns which distinguish temporal versus permanent (or otherwise emphasized) properties.
%%%
\begin{exe}
\il{Northern Saami}
\ex
\begin{xlist}
\ex {\rm Northern Saami \citep[48]{bergsland1976}}
\begin{xlist}
\ex
\gll	arve-dálki\\
	rain-weather\\
\glt	‘rain-weather’
%%
\ex
\gll	arvve\textbf{-s} dálki\\
	rain-\textsc{attr} weather\\
\glt	‘wet weather (i.e., weather full of rain)’
\end{xlist}
\end{xlist}
\end{exe}
%%%
It must be emphasized that these adjectives equipped with the attributive suffix are additionally marked as denoting permanent or “definite” properties. This is exactly consistent with the reconstructed meaning of the so-called weak adjective forms in \ili{Proto\hyp{}Germanic} or the so-called long adjective forms in \ili{Proto\hyp{}Baltic\slash{}Slavic} (see \S\ref{slavic diachr}). The semantics of the regular and productive contrastive focus constructions in \ili{Chuvash} and \ili{Udmurt} (which are often described as “emphatic” or “definite” as well, see \S\S\ref{chuvash synchr}, \ref{udmurt synchr}) also show a perfect parallel to Saamic.\is{species marking!definite}

It is thus most likely that the Saamic anti\hyp{}construct state marker originates from a construction in which the possessive marker 3\textsuperscript{rd} person singular was used as attributive nominalizer in appositional noun phrases similar to the contrastive focus construction attested in Modern \ili{Udmurt} and in several other Uralic and non-Uralic languages of northern Eurasia.\is{attributive nominalization}

Whereas the unmarked noun phrase type in \ili{Proto\hyp{}Saamic} was characterized by \isi{juxtaposition}, the attributive article was used to mark a construction with an adjective in contrastive focus. The emphatic construction later became generalized as the default marker of the attributive connection.\footnote{The zero-morpheme (equipped with the nominalizer Ø-\textsc{nmlz}) in (\ref{saami gram}) and following examples is only presented for a better illustration of the empty head position to which the (nominalized) adjective moves in the appositional noun phrase.}
%%%
\begin{exe}
\ex {\rm Grammaticalization of anti\hyp{}construct state marking in Saamic} 
\label{saami gram}
\begin{xlist}
\ex {\rm Stage 1: \ili{Pre-Proto\hyp{}Saamic}}
\begin{xlist}
\ex {\rm Juxtaposition}\\
	{\upshape [}\textsubscript{\rm NP} \textsubscript{\rm A}long \textsubscript{\rm N}stocking{\upshape ]}
\end{xlist}
%%%
\ex {\rm Stage 2a: \ili{Proto\hyp{}Saamic}}
\begin{xlist}
\ex {\rm Juxtaposition (default)}\\
	{\upshape [}\textsubscript{\rm NP} \textsubscript{\rm A}long \textsubscript{\rm N}stocking{\upshape ]}
%%%
\ex {\rm Attributive apposition (emphatic)}\\
	{\upshape [}\textsubscript{\rm NP} {\upshape [}\textsubscript{\rm NP'} \textsubscript{\rm A}long \textsubscript{\rm HEAD}Ø-\textsc{nmlz}{\upshape ]} \textsubscript{\rm N}stocking{\upshape ]}
\end{xlist}
\ex {\rm Stage 3: modern Saamic languages}
\begin{xlist}
\ex {\rm Anti\hyp{}construct state marking}\\
	{\upshape [}\textsubscript{\rm NP} \textsubscript{\rm A}long-\textsc{attr} \textsubscript{\rm N}stocking{\upshape ]}
\end{xlist}
\end{xlist}
\end{exe}
%%%
The irregularities in the use of attributive forms within and across the modern Saamic languages are the result of recent developments. Originally, the attributive form was generated regularly and productively. A cross-comparison of adjectives in different Saamic languages clearly shows that adjectives with deleted \textit{-s/-s'} in one Saamic language exhibit the suffix in another language. Consider, for example, \ili{Northern Saami} \textit{uhca} but \ili{Lule Saami} \textit{ucces} ‘small’ or \ili{Northern Saami} \textit{seakka} but \ili{Kildin Saami} \textit{sie{\ng}{\ng}kes'} ‘thin’ (for more examples see \citealt{riesler2006b}).

\is{predicative marking|(}
It is most likely that neither the predicative forms (ending in \textit{-d} or \textit{-s}) nor the attributive form (ending in \textit{-s/-s'}) reflect inherited stems in Saami. Both are complex forms which are derived from either nominal or verbal stems by means of different suffixes. The predicative forms with \textit{-s} evolved from derivations by means of an old lative case suffix. Germanic\il{Germanic languages} loan adjectives with the homophonous (Germanic) ending \textit{-s} ($\Leftarrow$ \ili{Proto\hyp{}Germanic} \textit{-R}) where integrated into the class of these predicative “lative-derivations”. The attributive suffix \textit{-s/-s'}, on the other hand, originates from the possessive marker 3\textsuperscript{rd} person singular which was originally used as an attributive nominalizer (i.e., attributive article) in contrastive focus constructions. The suffix was later generalized as the default attributive state marker.\is{attributive nominalization}

The merger of predicative and attributive forms of some adjectives observed in modern Saamic languages does not contradict the proposed reconstruction of the original attributive marking. It does, however, reflect another diachronic path of adjective attribution marking: namely the collapsing of an originally regular and productive construction and the innovation of a new type. Interestingly, this secondary development in modern stages of Saamic will most likely result in the renewed introduction of \isi{juxtaposition}, i.e., the original Uralic\il{Uralic languages} prototype of adjective attribution marking.
\il{Saamic languages|)}
\is{predicative marking|)}

\il{Finnic languages|(}
\is{head\hyp{}driven agreement|(}
\section[Agreement in Finnic]{The emergence of agreement in Finnic}
\label{Finnic diachr}
%%%
The languages of the Finnic branch spoken in the northwestern periphery of Uralic are exceptional within this family because they exhibit head\hyp{}driven agreement as the default type of attribution marking of adjectives.
%%%
\begin{exe}
\ex {\rm Finnish (personal knowledge)}
\begin{xlist}
\ex
\gll	iso talo\\
	big house\\
\glt	‘large house’
\ex	
\gll	iso-t talo-t\\
	big-\textsc{pl} house-\textsc{pl}\\
\glt	‘large houses’
\ex	
\gll	iso-i-ssa	talo-i-ssa\\
	big-\textsc{pl}-\textsc{iness} house-\textsc{pl}-\textsc{iness}\\
\glt	‘in large houses’
\end{xlist}
\end{exe}
%%%
%Note, however, that not all morphological features assign their values to the attributive adjective in Finnish. Whereas number (\ref{fin num}) and case marking (\ref{fin case}) is assigned to the adjective, possessive marking (\ref{fin poss}) is not.
%\footnote{\cite[212]{mark1979} mentions also the missing agreement category \textsc{possessive} in Finnish (as in the other Finnic languages) Finnish \textit{sininen kukka-ni} [blue flower-\textsc{poss:1sg}] ‘my blue flower'. believes that the possessive declension of nouns is younger than the agreement of adjectives}
There is no doubt that agreement marking replaced \isi{juxtaposition} at a certain point during the linguistic development from \ili{Proto\hyp{}Uralic} to \ili{Proto\hyp{}Finnic}.

In several \ili{Uralic languages}, irregular agreement of pronominal modifiers and even some adjectives and adjective-like modifiers are attested (cf.~examples in \citealt{honti1997} and \citealt[288–295]{stolz2015a}). This might indicate a connection to the fully developed agreement marking of adjectives in Finnic. It is, however, unclear whether the incomplete and irregular agreement phenomena in Saamic\il{Saamic languages} and other closely related \ili{Uralic languages} reflect a stage of development at which agreement marking was more widespread – in at least the Finnic and Saamic branches – or agreement marking is due to a more recent innovation which became completely enforced only in the Finnic branch. 

The rise of agreement marking on attributive adjectives, pronouns,\is{adnominal modifier!pronoun} and numerals\is{adnominal modifier!numeral} in Finnic is usually regarded as a result of language contact with \ili{Indo-European languages} from the Germanic\il{Germanic languages} and/or Baltic\il{Baltic languages} groups (cf.~\citealt[25]{tauli1955}; \citealt{hajdu1996}; see also \citealt[288–295]{stolz2015a}). Indeed, the high amount of Germanic and Baltic loanwords in Finnic languages indicate intimate contacts between speakers of Uralic and Indo-European languages in that area. In order to prove the hypothesis that agreement marking arose as a result of influence from Indo-European languages, however, one has to reconstruct concrete mechanisms behind this profound contact-induced language change. The idea that agreement marking is a borrowed model might not be as straightforward as it appears. Even though many \ili{Uralic languages} under strong \ili{Russian} influence seem to have borrowed many more grammatical features than Finnic did under Germanic and Baltic influence, none of these languages shows any trace of borrowed Russian agreement marking.

\is{juxtaposition|(}
\il{Hungarian|(}
In a short article, \citet{mark1979} presents a contact-independent explanation of the innovative head\hyp{}driven agreement marking in Finnic. His explanation is based on the observation that nominalized adjectives in apposition to nouns in Hungarian (as well as in other \ili{Uralic languages}) show agreement triggered by the semantic head of the elliptic noun phrase.
%%%
\begin{exe}
\ex {\rm Hungarian \citep[209]{mark1979}}
\label{hung ap}
\begin{xlist}
\ex {\rm Juxtaposition (no agreement marking)}
\begin{xlist}
\ex \textit{őreg postást} 		{\rm [A N\textsubscript{\rm nom.sg}] ‘the old postman’}
\ex \textit{őreg postások} 		{\rm [A N\textsubscript{\rm nom.pl}] ‘the old postmen’}
\end{xlist}
\ex {\rm Apposition (agreement marking)}
\begin{xlist}
\ex \textit{postást, őreg\textbf{et}} 	{\rm [[N\textsubscript{\rm nom.sg}] [A\textsubscript{nom.sg}]] ‘a postman, an old one’}
\ex \textit{postások, őreg\textbf{ek}} 	{\rm [[N\textsubscript{nom.sg}] [A\textsubscript{nom.sg}]] ‘postmen, old ones’}
\end{xlist}
\end{xlist}
\end{exe}
%%%
Similar ideas about a possible contact-independent origin of head\hyp{}driven agreement in Finnic have also been put forward, for example by \cite{ravila1941} and \cite{papp1962}. In theory, the rise of agreement marking as a result of generalization of an originally emphasized adjective in apposition seems plausible. Language contact with agreement-marking languages could still have been a catalyst. 

In \ili{Hungarian}, the attributive appositions described by Márk are post\hyp{}positioned while attributive adjectives in \ili{Finnish} still precede the noun. A comparison to attributive apposition by means of nominalization in \ili{Udmurt} seems more promising. In \S\ref{udmurt synchr} on the synchrony of attributive marking in Udmurt, it has been demonstrated how case and number agreement marking occurs in the contrastive focus construction with attributive adjectives and pronouns.
%%%
\begin{exe}
\ex {\rm Udmurt \citep{winkler2001}}
\label{udmurt ap}
\begin{xlist}
\ex 	{\rm Juxtaposition (no agreement marking)}
\begin{xlist}
\ex	\textit{badǯ́ym gurt} 				{\rm [A N\textsubscript{nom:sg}] ‘large house’}
\ex	\textit{badǯ́ym gurtjos} 			{\rm [A N\textsubscript{nom:pl}] ‘large houses’}
\ex	\textit{badǯ́ym gurtjosy} 			{\rm [A N\textsubscript{pl:ill}] ‘to (the) large houses’}
\end{xlist}
%%%
\ex	{\rm Attributive apposition (agreement marking)}
\begin{xlist}
\ex	\textit{badǯ́ym\textbf{ėz} gurt} 		{\rm [[A\textsubscript{contr}] [N]] ‘\textsc{large} house’}
\ex	\textit{badǯ́ym\textbf{josyz} gurtjos} 	{\rm [[A\textsubscript{contr:pl}] [N\textsubscript{pl}]] ‘\textsc{large} houses’}
\ex	\textit{badǯ́ym\textbf{josaz} gurtjosy} 	{\rm [[A\textsubscript{contr:pl:ill}] [N\textsubscript{pl:ill}]] ‘to \textsc{large} houses’}
\end{xlist}
\end{xlist}
\end{exe}
%%
In both Hungarian and \ili{Udmurt} examples (\ref{hung ap}) and (\ref{udmurt ap}), the agreement morphology is syntactically spread from the (semantic) head noun to the adjectival modifier only in appositional noun phrases (with the modifier in contrastive focus). In Udmurt, there is an additional morpheme available, i.e., the attributive nominalizer \textit{-(ė)z} ($\Leftarrow$ \textsc{poss:3sg}). In the Hungarian example, the emphasized construction is only marked by the duplicated number and case agreement (in combination with changed constituent order).\is{attributive nominalization}
\il{Hungarian|)}
\is{juxtaposition|)}

\il{Permic languages|(}
Attributive apposition in contrastive focus constructions is without a doubt innovative in \ili{Udmurt}. Since all members of the Permic group show similar constructions, the development could be dated back to \ili{Proto\hyp{}Permic} and would thus have a time depth comparable to the innovation of head\hyp{}driven agreement in Finnic. Since head\hyp{}driven agreement is also involved in \ili{Udmurt} anti\hyp{}construct state marking (namely as a “relict” of the appositional structure in which the attribute in contrastive focus originally occurred), the Permic and Finnic innovations could be structural parallels. Modern Finnic languages, however, do not provide any evidence that an attributive nominalizer was ever used as a marker of appositional attribution. The agreement markings thus seems to be the primary innovation assumedly caused by contact with “agreeing” \ili{Indo-European languages}. Regardless of contact influence being involved or not, the innovative head\hyp{}driven agreement marking in Finnic could still have been used in an appositional construction originally. Note also that in Udmurt, number agreement sometimes (irregularly) occurs even in constructions without the contrastive focus marker.
%%%
\begin{exe}
\ex {\rm Head\hyp{}driven plural agreement in Udmurt \citep{winkler2001}}\\
\gll	badǯ́ym-jos gurt-jos\\
	big-\textsc{pl} house-\textsc{pl}\\
\glt	‘\textsc{large} houses’
\end{exe}
%%%
Note even that a similar innovation of head\hyp{}driven agreement in contrastive focus constructions is attested not only for Permic languages but also occurs irregularly in other Uralic branches (cf.~\citealt[136–138, 142]{honti1997} for Mari\il{Mari languages} and Nenets;\il{Nenets languages} \citealt[177]{siegl2013a} for \ili{Tundra Enets}).

To conclude these tentative considerations, it cannot be ruled out that the rise of head\hyp{}driven agreement marking in Finnic and anti\hyp{}construct state agreement in \ili{Udmurt} are both results of original attributive apposition constructions. For Finnic, however, this idea remains highly speculative unless one can find evidence for the occurrence of an attributive nominalizer such as the marker in Modern \ili{Udmurt} or in \ili{Proto\hyp{}Saamic}.\is{attributive nominalization}

Whereas anti\hyp{}construct state agreement marking in \ili{Udmurt} (and other Permic languages) only substitutes for the default marker in contrastive focused constructions, Finnic and Saamic\il{Saamic languages} have completely lost Uralic\il{Uralic languages} \isi{juxtaposition} as the default adjective attribution marking device and innovated completely new morpho-syntactic devices. It must also be noted that the Finnic and Saamic innovations took place in two closely related and geographically adjacent branches of Uralic. Moreover, the developments are of similar age. And finally, non-related but geographically adjacent languages (Baltic,\il{Baltic languages} Germanic,\il{Germanic languages} Slavic\il{Slavic languages}) show structurally similar developments.
\il{Permic languages|)}
\il{Finnic languages|)}
\is{head\hyp{}driven agreement|)}

\section{Other attested scenarios of grammaticalization}
%%%
The previous sections dealt with the rise of adjective attribution marking devices in a few branches of Indo-European,\il{Indo-European languages} Uralic\il{Uralic languages} and Turkic.\il{Turkic languages} The synchronic data from the synchronic survey in Part~III (Synchrony), however, present evidence of several diachronic scenarios. Only a few of them will be sketched in the following sections.

\is{attributive article|(}
\is{species marking!definite|(}
\subsection[Articles, definiteness and adjective attribution]{Articles, definiteness and the evolution of adjective attribution marking in Indo-European}
%%%
The rise of attributive articles and their (partial or complete) further development to definite markers in Baltic,\il{Baltic languages} Slavic\il{Slavic languages} and Germanic,\il{Germanic languages} as described above, took place on functionally and chronologically parallel paths in various other \ili{Indo-European languages} of Europe. This has been observed by several scholars (cf.~\citealt{brugmann-etal1916}; \citealt{gamillscheg1937}; \citealt{heinrichs1954} and, more recently, \citealt{nocentini1996}, \citealt{philippi1997}; \citealt{himmelmann1997}). It is not clear whether these parallel developments across western-Indo-European branches can be explained in terms of areal typology, i.e., as the result of linguistic contacts, or whether they are inherited from a common ancestor language. Independent developments, though theoretically possible, seem rather unlikely given the close genealogical and areal connection between the languages in question.

In those western branches of the Indo-European family where definite markers have evolved, cognate formatives are also usually attested as adjective attribution markers. The attributive article in \ili{Rumanian}, for instance (see \S\ref{rumanian synchr}), is also attested in \ili{Latin} and other \ili{Romance languages}, cf.~\ili{Latin} \textit{Cato ille maior, Babylon illa magna}.\footnote{Cf.~the secondary attributive articles in Germanic languages in similar constructions: \ili{English} \textit{Philip the Fair}, \ili{German} \textit{Friedrich der Große} which is also cognate (and homophonous) with the definite marker. The Germanic constructions have been dealt with in more detail in \S\ref{attr nmlz}.} The suffixed definite marker in \ili{Rumanian} evolved from this attributive article (\citealt{gamillscheg1937}; \citealt[5]{nocentini1996}). Note also that the attributive article in Romance is polyfunctional and can mark adjectival, genitival and prepositional attributes as well as relative clauses.

In the two \ili{Albanian languages} (see \S\ref{albanian synchr}), the attributive article \textit{i} \textsc{nom}, \textit{e/të} \textsc{acc} and \textit{të} \textsc{obl} and the definite suffix \textit{-i} \textsc{nom}, \textit{-in/-në} \textsc{acc} and \textit{it} \textsc{obl} most likely have the same etymological source, i.e., Indo-European *\textit{-to} (cf.~\citet[165]{himmelmann1997} with references), which is also the etymological source of the definite marker \textit{to} and the homophonous attributive article in Ancient Greek (see \S\ref{greek synchr} for the corresponding constructions in Modern \ili{Greek}).

\is{attributive nominalization|(}
Indo-European *\textit{-to} is the etymological source of secondary attributive articles in \ili{Slavic languages} as well. The use of this marker in attributive apposition constructions is already well-attested in \ili{Old East Slavic} documents.
%%%
\begin{exe}
\ex {\rm Attributive nominalization in Old E-Slavic (Indo-European)}
\begin{xlist}
\ex
\gll	[\dots] sъ usmъ galiiei-sk\textbf{-ymъ}\\
	{ } with Jesus:\textsc{com} Galilee-\textsc{adjz}-\textsc{nmlz:instr}\\
\glt 	‘[\dots] with Jesus the Galilaen’ \citep[Matthew 26, cit.][214]{mendoza2004}
\ex 
\gll	vъ sarefto̜ sidonъ-sk\textbf{-o̜jo̜}\\
	to Sarepta:\textsc{prepos} Sidonia-\textsc{adjz}-\textsc{nmlz:acc}\\
\glt 	‘to Sarepta in Sidonia’ \citep[Luke 4, cit.][214]{mendoza2004}
\end{xlist}
\end{exe}
%does it show the attributive marker *-to, or the marker *-jis??
%%%
In \ili{Bulgarian}, the former attributive nominalizer grammaticalized into a true definite marker. In an analogous manner (but much later in time), reflexes of the \ili{Proto\hyp{}Baltic\slash{}Slavic} pronoun \textit{*tъ} \textsc{m} developed into definite suffixes in northern Russian dialects\il{Russian!Northern} (cf.~\citealt
{leinonen2006a}).\footnote{Whereas \ili{Komi-Zyrian} (Uralic) influence triggered the suffixation of these anaphoric markers in northern Russian dialects \citep
{leinonen2006a}, a typologically similar grammaticalization process due to Turkic\ili{Turkic languages} influence is behind the chronologically much older suffixation of definite marking in Bulgarian \citep[114–122]{kusmenko2008}.}

\ia{Dahl, Östen|(}
Dahl (\citeyear[149–152]{dahl2003}; see also \citealt[122–123]{dahl2015a}) shows that in some languages definite noun phrases with attributive adjectives (or other adnominal modifiers) show special behavior. He compares the “displaced”\footnote{The term “displaced” is not used by Dahl but adopted from \citet[114–116]{melcuk2006}.} definite marking with “long form” adjectives in the \ili{Baltic languages} with, among others, the demonstrative \textit{ille} linking postponed adjectives to proper nouns in \ili{Latin} constructions like \textit{Babylon illa magna} \cite[150]{dahl2003}. But due to its function and syntactic behavior the attributive article in Romance\il{Romance languages} can clearly be distinguished from definite markers \citep[329]{gamillscheg1937}. As it was demonstrated for the \ili{Baltic languages} (see \S\ref{anti-constr agr}), the so-called “long form” inflection (i.e., anti\hyp{}construct state agreement inflection) of adjectives is not a true definiteness marker.

\il{Amharic|(}
Dahl also gives examples of languages in which “displaced” definiteness markers (or “quasi-definiteness markers”) evolved from other sources than local\hyp{}deictic pronouns, as in Amharic where an attributive nominalizer grammaticalized from a (person-deictic) possessive marker in contrastive focus construction.
\ia{Dahl, Östen|)}
%%%
\begin{exe}
\ex	\langinfo{Amharic}{Afro-Asiatic}{\citealt{hudson1997}}
\begin{xlist}
\ex	{\rm Default construction}
\begin{xlist}
\ex
\gll	təlləq bet\\
	large	house\\								
\glt	‘(a) large house’
%%%
\ex	
\label{amharic ambiguous}
\gll	təlləq bet\textbf{-u}\\
	large house-\textsc{poss:3sg}\\
\glt	(1) ‘his large house’ (if the owner has only one house, which is large); (2) ‘the large house’
%%%
\ex
\gll	təlləq bet-e\\
	large house-\textsc{poss:1sg}\\
\glt	‘my large house’
\end{xlist}
%%%
\ex	{\rm Contrastive focus construction}
\begin{xlist}
\ex	
\gll	təlləq\textbf{-u} bet\\
	large-\textsc{?def} house\\
\glt	‘(a/the) \textsc{large} house’
%%%
\ex	
\label{amharic nonambiguous}
\gll	təlləq\textbf{-u} bet\textbf{-u}\\
	large-\textsc{?def} house-\textsc{poss:3sg}\\
\glt	‘his \textsc{large} house’ (if the owner has more than one house but the expression is referring to the large one)
%%%
\ex
\gll	təlləq\textbf{-u} bet-e\\
	large-\textsc{?def} house-\textsc{poss:1sg}\\
\glt	‘my \textsc{large} house’
\end{xlist}
\end{xlist}
\end{exe}
%%
The suffix \textit{-u} [\textsc{m}] used for emphasizing the adjective in Amharic is homophonous with the definite noun marker and with the 3\textsuperscript{rd} singular possessive marker. Note that the possessive and the definite suffixes of nouns (or noun phrases) are mutually exclusive \citep[463]{hudson1997}. Hence, the examples in (\ref{amharic ambiguous}) are ambiguous; they could have a possessive or a definite reading. The “emphasizing” adjective suffix \textit{-u} [\textsc{m}], however, does not co-occur with the definite suffix. Therefore, the reading of the examples in (\ref{amharic nonambiguous}) is not ambiguous.

Consequently, the suffix \textit{-u} [\textsc{m}] in Amharic should be analyzed as an adjective attribution marker rather than as a “detached” marker of definiteness. 
%%%
\begin{exe}
\ex	\langinfo{Amharic}{Afro-Asiatic}{\citealt{hudson1997}}
\begin{xlist}
\ex	{\rm Attributive nominalization (contrastive focus)}
\begin{xlist}
\ex
\gll	təlləq\textbf{-u} bet\\
	large-\textsc{attr} house(\textsc{m})\\
\glt	‘(a/the) \textsc{large} house’
%%%
\ex	
\gll	qonjo\textbf{-wa} dəmmät\\
	pretty-\textsc{attr:f} cat(\textsc{f})\\
\glt	‘(a/the) beautiful cat’
\end{xlist}
%%%
\ex	{\rm Attributive nominalization (\isi{headless noun phrase})}
\begin{xlist}
\ex
\gll	təlləq\textbf{-u}\\
	large-\textsc{attr:m}\\
\glt	‘(a/the) big one’
%%%
\ex	
\gll	qonjo-wa\\
	pretty-\textsc{attr:f}\\
\glt	‘(a/the) pretty one’
\end{xlist}
\end{xlist}
\end{exe}
%%%
Contrastive focus marking on adjectives in Amharic is thus very similar to the marking found in \ili{Udmurt}. In both languages, attributive apposition is marked by means of attributive nominalization. The respective formatives in both languages originate from (person-deictic) possessor markers.

Consistently, data from northern Eurasian languages and Amharic do not provide evidence for the existence of “displaced” definiteness markers. From a diachronic perspective, however, there is much evidence for a functional overlapping between attributive nominalization and definiteness marking. In all \ili{Indo-European languages} dealt with so far, adjective attribution is the primary function. The former local-deictic marker in these languages always grammaticalizes into an attributive nominalizer first. The further development into true markers of definiteness comes only after this stage.
\il{Amharic|)}
\is{attributive article|)}
\is{species marking!definite|)}
\is{attributive nominalization|)}

\il{Iranian languages|(}
\subsection[Head-marking attributive construct state]{The emergence of head-marking attributive construct state in Iranian}
\label{iranian diachr}
%%%
As shown in \S\ref{iranian synchr}, several Iranian languages of the northern Eurasian area exhibit a head-marking attributive construct state device as a licenser of adjective attribution. The Iranian construct state marker (aka \textit{Ezafe}) originates from the Old Iranian\il{Old Iranian languages} relative particle \textit{-hya}, which has undergone a process of grammaticalization, to end up as a part of nominal morphology in the modern Iranian languages \citep{haider-etal1984,samvelian2007b}. Since the \ili{Old Persian} relative particle \textit{-hya} itself originates from a demonstrative, the emergence of construct state marking in Iranian and anti\hyp{}construct state marking in other \ili{Indo-European languages} follow a similar path. Originally, \textit{-hya} was a grammatical word marking the phrase or clause on its right as a syntactic modifier of the noun on its left \citep{haider-etal1984}. Syntactically, the marker was an attributive article hosted by the attribute. In Baltic\ili{Baltic languages} and Slavic,\ili{Slavic languages} the article developed further into an anti\hyp{}construct state agreement marker (see \S\ref{slavic diachr}). In Iranian, however, the article attached phonologically to the head noun. According to \citet[3]{samvelian2007} this conflict between opposite directions of phonological and syntactic alignments was later resolved by the \isi{re-analysis} of the article as a head-marking inflectional affix. As the result of this grammaticalization, syntactic and phonological attachments were alined to each other.
\il{Iranian languages|)}

\is{juxtaposition|(}
\subsection[Innovation of juxtaposition]{Innovation of juxtaposition}
%%%
Two scenarios are attested where juxtaposition has been innovated: either by loss of agreement marking or by loss of anti\hyp{}construct state marking.

\subsubsection{Loss of agreement marking}
\is{head\hyp{}driven agreement|(}
\il{Common Kartvelian|(}
Head\hyp{}driven agreement (in number and case) of adjectival modifiers following the head noun can be reconstructed for Common Kartvelian. In \ili{Old Georgian}, this pattern is more or less preserved. In modern \ili{Kartvelian languages}, however, the unmarked constituent order of adjectival modifiers and head is noun-final, although the opposite order is possible as well \citep[56]{harris1991a}. As shown in \S\ref{kartvelian synchr} of Part~III (Synchrony), the agreement features of Common Kartvelian are more or less preserved only in the marked (but inherited) head-initial noun phrase type. In the head-final noun phrase type, on the other hand, modern \ili{Kartvelian languages} display a strong tendency to lose head\hyp{}driven agreement. Preposed attributive adjectives in \ili{Mingrelian} and \ili{Laz} are juxtaposed to the head noun as a rule. In Modern \ili{Georgian} and \ili{Svan}, the agreement paradigm of preposed attributive adjectives shows a high degree of syncretism (cf.~\citealt[56]{harris1991a}; \citealt[56–60, passim]{tuite1998}).
\il{Common Kartvelian|)}

Two other non-related languages of the \isi{Southern Caucasus}, Armenian and \ili{Ossetic} have lost noun phrase internal agreement too \citep[]272–281]{stolz2015a}.\footnote{The innovation of juxtaposition in the Eastern Armenian standard language is not complete, though. There is a small class of adjectives which are marked by means of head-driven agreement, see\S\ref{armenian-synch}.} According to \cite[109]{johanson2002a}, Turkic contact influence is the explanation for the loss of agreement in Armenian. 

Interestingly, the loss of adjective agreement marking in Armenian and Kartvelian is connected to the shift of the default constituent order. Note, however, that juxtaposition can also be innovated without constituent order shift, as in \ili{English} where the change is a result of the complete loss of the agreement inflection during the course of time from Middle\il{Middle English} to Modern \ili{English}. 
\is{juxtaposition|)}
\is{head\hyp{}driven agreement|)}

\subsubsection{Loss of anti\hyp{}construct state marking}
\ili{Saamic languages} present another evidence of a language change in which \isi{juxtaposition} replaces an original morpho-syntactic device. The original anti\hyp{}construct state marking, which is itself innovative in \ili{Proto\hyp{}Saamic} (see \S\ref{saamic diachr}) is in dissolution in modern Saamic languages as the result of the merger of attributive and predicative adjective forms which were originally distinguished from one another.\is{predicative marking}

%??Germanic, Slavic

\begin{figure}
\parbox[b]{0.5\textwidth}{
\begin{center}Indo-European\\
\medskip
\begin{tabular}{| m{1.4cm} || m{.9cm} | m{1.1cm} | m{.7cm} |}
\cline{1-1}
\textsc{mod}\textsubscript{NP}\\
\cline{1-1}
\textsc{attr}\textsubscript{AdP}\\
\hline
 & \textsc{nmlz} & \textsc{contr} & \textsc{def}\\
\hline
\textsc{attr}\textsubscript{A}\\
\cline{1-1}
\textsc{attr}\textsubscript{N}\\
\cline{1-1}
\end{tabular}
\end{center}
}
\parbox[b]{0.5\textwidth}{
\begin{center}Uralic\\
\medskip
\begin{tabular}{| m{1.4cm} || m{.9cm} | m{1.1cm} | m{.7cm} |}
\cline{1-1}
\\
\cline{1-1}
\\
\hline
 & \textsc{nmlz} & \textsc{contr} & \textsc{def}\\
\hline
\textsc{attr}\textsubscript{A}\\
\cline{1-1}
\\
\cline{1-1}
\end{tabular}
\end{center}
}

\parbox[b]{0.5\textwidth}{
\begin{center}Turkic\\
\medskip
\begin{tabular}{| m{1.4cm} || m{.9cm} | m{1.1cm} | m{.7cm} |}
\cline{1-1}
\\
\cline{1-1}
\\
\hline
 & \textsc{nmlz} & \textsc{contr} & \textsc{def}\\
\hline
\\
\cline{1-1}
\\
\cline{1-1}
\end{tabular}
\end{center}
}
\parbox[b]{0.5\textwidth}{
\begin{center}Tungusic\\
\medskip
\begin{tabular}{| m{1.4cm} || m{.9cm} | m{1.1cm} | m{.7cm} |}
\cline{1-1}
\\
\cline{1-1}
\\
\hline
 & \textsc{nmlz} & \textsc{contr} & \\
\hline
\textsc{attr}\textsubscript{A}\\
\cline{1-1}
\\
\cline{1-1}
\end{tabular}
\end{center}
}
\caption[Functional map of cognate devices]{Functional map of markers cognate with the Old Iranian\il{Old Iranian languages} “relative particle” \textit{-hya} (across Indo-European languages) and the possessive suffixes 3\textsuperscript{rd} person singular (across Uralic,\il{Uralic languages} Turkic\il{Turkic languages} and \ili{Tungusic languages})}
\label{ie-ural funcmap}
\end{figure}

\section{Diachronic polyfunctionality}
%%%
\is{species marking!definite|(}
In \S\ref{polyfunctionality}, a few examples of polyfunctional adjective attribution marking devices were presented. It was shown, however, that the polyfunctionality parameter is less relevant to northern Eurasian languages because most languages of the area exhibit highly differentiated attribution marking devices. Polyfunctionality might, however, indicate a historical dimension if additional semantics of attribution marking devices is taken into consideration and if the languages of a whole genera are compared to each other. For instance, construct state marking of adjectives and other modifiers, as attested especially in Indo-European\il{Indo-European languages} varieties (but also in Turkic\il{Turkic languages} and Uralic),\il{Uralic languages} seems to be inherently tied to the evolution of \isi{attributive nominalization}, contrastive focus and even definiteness marking in several languages. Figure~\ref{ie-ural funcmap} shows functional maps similar to the one in Figures \ref{multi abcd} and \ref{lahu funcmap} in \S\ref{polyfunctionality} but with scope over cognate markers in whole language families.

The polyfunctionality of the \ili{Persian} Ezafe \textit{-(y)e} was described in \S\ref{polyfunctionality}. This construct state marker licenses nominal (\textsc{attr}\textsubscript{N}), adjectival (\textsc{attr}\textsubscript{A}) and adpositional (\textsc{attr}\textsubscript{AdP}) attributes as well as modification\is{modification marking!} within an adposition phrase (\textsc{mod}\textsubscript{NP}). The cognate formative in the closely related Iranian language \ili{Northern Kurdish} is even connected to definiteness marking (\textsc{def}) (\citealt{schroder2002}; cf.~also Table~\ref{ez kirmanji paradigm} on page~\pageref{ez kirmanji paradigm}). In Old Iranian\il{Old Iranian languages}, Old Baltic\il{Old Baltic languages} and \ili{Old Slavic languages}, a cognate marker was used as an attributive nominalizer (\textsc{nmlz}, or as a “relative particle” marking non-verbal attributes; see \S\ref{iranian diachr} and \ref{slavic diachr}). The marker's further grammaticalization into an anti\hyp{}construct state agreement marker in Baltic and Slavic is connected to contrastive focus marking (\textsc{contr}).

The marker described in the functional map for Uralic is the possessive suffix 3\textsuperscript{rd} person singular, which is used as a quasi-definite marker (\textsc{def}) in a variety of modern \ili{Uralic languages}. In \ili{Udmurt} the original possessive suffix is regularly used as a nominalizer (\textsc{nmlz}) and has grammaticalized into a marker of contrastive focus of adjectives (\textsc{contr}) (see \S\ref{udmurt diachr}). In Saamic,\il{Saamic languages} finally, the cognate marker has grammaticalized into an anti\hyp{}construct state marker (\textsc{attr}\textsubscript{A}).

Turkic\il{Turkic languages} is similar to Uralic\il{Uralic languages} but without evidence for the grammaticalization of the possessive suffix 3\textsuperscript{rd} person singular to a true adjective attribution marker. In Tungusic,\il{Tungusic languages} finally, there is no evidence for definiteness marking but the possessive suffix 3\textsuperscript{rd} person singular is used as \isi{dependent\hyp{}driven agreement} marker in \il{Even} (\textsc{attr}\textsubscript{A}).

These diachronic functional maps demonstrate general synchronic paths of attribution marking devices and give the impression that nominalization and appositional attribution play an important role in the further development of the respective markers as attribution marking devices.\is{attributive nominalization}
\is{grammaticalization|)}
\is{species marking!definite|)}
