
\chapter[Areal uniformity and diversity]{Areal uniformity and diversity in northern Eurasia}\label{areality}
In the previous chapter, the prototypical and the known minor noun phrase types occurring in the languages of northern Eurasia were characterized and illustrated with examples. This survey thus provides an overall picture of the degree of typological uniformity or divergence with regard to adjective attribution marking marking within both the whole area and each genealogical unit.

\section{Attested attribution marking devices}
Fourteen (simple and combined) types of adjective attribution marking devices are attested in the languages of northern Eurasia:
\begin{itemize}
\item Anti-construct state (as in Saami)
\item Anti-construct state + head-driven agreement (“double agreement”, as in Swedish)
\item Anti-construct state + construct state (“double-construct state”, as in Northern Saami)
\item Anti-construct state agreement (as in Russian)
\item Appositive head-driven agreement (as in Georgian)
\item Attributive article (as in Yiddish)
\item Attributive article + head-driven agreement (“double agreement”, as in Albanian)
\item Incorporation (as in Chukchi)
\item Attributive nominalization (as in Dungan)
\item Attributive nominalization (as in Udmurt)
\item Construct state (as in Kurmanji)
\item Juxtaposition (as in Komi)
\item Head-driven agreement (as in Finnish)
\item Modifier-headed possessor agreement (as in Oroch)
\end{itemize}
Only one type attested in the world-wide sample (Table \ref{sample}) does not occur in the northern Eurasian area: the floating construct state marker ({\it linker}) found, for instance, in Tagalog (Austronesian). 

The Indoeuropean family has the largest absolute number of attested adjective attribution marking devices (nine), followed by Nakh-Daghestanian and Uralic (five each), and Kartvelian and Tungusic (four each). The Mongolic family has the lowest possible number with only one attested device, just as with Kamchatkan and the isolates Ainu, Basque, Korean and Nivkh.

The most rare types are:  (1) modifier-headed possessor agreement, which is attested only as a secondary device in a few Tungusic languages, and (2) the combined construct device (i.e.~“double-construct state”), which is attested only marginally in one single language, Northern Saami (Uralic). Attributive nominalization (as an article combined and with head-driven agreement) is also very rare. This type occurs as the primary device only in the Albanian branch of the Indoeuropean family, but it is also attested as a secondary or tertiary device in other languages. Head-marking construct state is also relatively uncommon in the northern Eurasian area as it is attested only in Iranian languages (Indoeuropean).

The most common type is juxtaposition, followed by head-driven agreement.

\section{Prototypes of attribution marking devices}
Several language families of northern Eurasia exhibit clear prototypes of adjective attribution marking devices: All Mongolic and Turkic languages have juxtaposition as the default device, as is the case for the languages of most branches of Uralic as well. Head-driven agreement occurs as another prototype in many branches of the Indoeuropean family. Even though the attested deviation from the prototype is much higher in Indoeuropean than in Mongolic, Turkic and Uralic, head-driven agreement marking can be shown to occur prototypically in most Indoeuropean genera.

For Abkhaz-Adyge, Chukotkan, Kartvelian, Nakh-Daghestanian and Tungusic families, synchronic prototypes are not very easy to find because a predominant type does not occur inside these families. The other language families of northern Eurasia are either isolates (Nivkh, Ainu, Japanese, Korean, Basque) or they exhibit rather shallow genealogical diversity (Kamchatkan, Yukagir, Yeniseian). Together with a few other families, predominantly spoken outside the investigated area (Eskimo-Aleut, Sino-Tibetan, Semitic), these families are excluded from generalizations about prototypes. 

Larger language families with strikingly high diversity in regard to the attested absolute number of adjective attribution marking devices are Indoeuropean, Nakh-Dagestanian, Uralic and Tungusic. A strikingly high degree of unity is found in Mongolic and Turkic.

\section[Diachronic implications]{Diachronic implications of uniformity and diversity inside and across genera}
Measuring the degree of diversity (or unity) from a synchronic point of view may help identify diachronic processes. A significantly high degree of diversity inside a given genus as compared to its proto-stage is likely to manifest pervasive linguistic changes and the innovation of new types. Similarly, the synchronic attestation of a high degree of unity inside a given genus indicates the inheritance of original types without significant innovations. 

A genus is defined as a group of related languages which go back to a reconstructed (or documented) proto-form. The East-Saamic languages, for instance, form a group of sister languages which derived from proto-East-Saamic. Proto-East-Saamic is derived together with its Saamic sister languages from a more distant proto-stage, i.e.~proto-Saamic which again is derived together with its Uralic sister languages from proto-Uralic. Since the proto-stages of languages are normally reconstructed as single languages it can be assumed that most of them had only one single type of adjective attribution marking (similar to the prevailing number of languages spoken today, cf.~the sample in Table \ref{sample} in the appendix). It is self-evident that daughter languages which descend from a proto-language will either inherit the original adjective attribution marking devices, innovate secondary (or tertiary etc.) devices or replace the original devices with new ones. The proto-Saamic daughter language of proto-Uralic, for instance, has replaced the original Uralic juxtaposition with anti-construct state marking (cf.~Section \ref{saamic diachr}). The proto-Slavic daughter language of proto-Indoeuropean inherited the original Indoeuropean head-driven agreement marking but innovated a secondary type, i.e.~anti-construct state agreement marking (cf.~Section \ref{slavic diachr}). All modern Mongolic languages, by contrast, exhibit juxtaposition uniformly and have obviously inherited this device from their proto-languages (proto-Dagur, proto-Moghol, proto-Mongolian etc.) which in turn must have inherited juxtaposition from proto-Mongolic. A comparison of synchronically attested diversity inside and across genera might thus have diachronic implications.

The simple statistics in Table \ref{diversity} tries to scope the degree of diversity in the investigated families of northern Eurasia. Column 1 lists all families, branches and subbranches in alphabetical order. Isolates and genera with only one member language are not included in the table, and neither are genera which are not spoken predominantly in northern Eurasia, with only two exceptions: the Iranian and Indo-Aryan subbranches within the Indoeuropean family. Since the highest possible diversity is of interest here, the number of all attested devices devices (including secondary and tertiary types restricted to special noun phrase types) are counted. 

The second column in Table \ref{diversity} (“Units (abs.)”) gives the absolute number of coded languages from each genus. The third column (“Types (abs.)”) gives the absolute number of attested types. The next two columns 4 and 5 present ratio figures. The first of them  (“Ratio (gen.)”) results from dividing the number of attested types in the given genus by the number of types attested for the higher branch:\medskip

$Diversity_{genus} = \frac{Types_{genus}}{Types_{family}}$.\medskip

\noindent For instance, West-Saamic has a ratio of $1.00$ because it exhibits all 4 types attested in the whole Saamic branch. The Saamic branch as such has a ratio of $1.25$ because 4 types are found in Saamic compared to 5 attested for the whole Uralic family. Similarly, South-Slavic has also a ratio of $1.00$ because it exhibits all 3 types attested in Slavic. But the Slavic branch as such has a higher ratio of $3.00$ (meaning a lesser degree of diversity) because only 3 types are attested in this branch out of 9 types for the whole Indoeuropean family. 

The last ratio figures (“Ratio (lgs.)”) result from dividing the overall number of languages by the number of attested types in the given genus:\medskip

$Diversity_{languages} = \frac{Languages_{genus}}{Types_{genus}}$.\medskip

\noindent For instance, 5 West-Saamic languages are coded for 4 different types, resulting in a ratio of $1.25$. For the whole Saami branch all together 9 languages are coded for 5 types, resulting in a somewhat higher ratio figure of $1.80$. South-Slavic has the ratio of $1.33$ because the 4 South-Slavic languages are coded for 3 types; Slavic, however, has $4.33$ because 13 Slavic languages are coded for only 3 different types.

\begin{footnotesize}
\begin{longtable}[h]{l l l || c || c | c | c || c c c | c}
\hline
\hline
\multicolumn{3}{r||}{1}			&2		&3		&4		&5		&6\\
\hline
\multicolumn{3}{r||}{\textit{Family}}	&Units	&Types	&Ratio	&Ratio	&Diversity\\
&\multicolumn{2}{l||}{Main branch}	&(abs.)	&(abs.)	&(gen.)	&(lgs.)	&Value\\
\hline
\multicolumn{3}{r||}{\textit{Abkhaz-Adyge}}&\textit{4}&\textit{2}&–&\textit{2.00}	&low\\
&\multicolumn{2}{l||}{Abkhaz}		&2		&1		&2.00	&2.00	&–\\
&\multicolumn{2}{l||}{Adyge}		&2		&1		&2.00	&2.00	&–\\
\hline
\multicolumn{3}{r||}{\textit{Chukotkan}}&\textit{3}&\textit{2}&–&\textit{1.50}		&–\\
&\multicolumn{2}{l||}{Chukchi}		&1		&2		&1.00	&0.50	&–\\
&\multicolumn{2}{l||}{Koryak}		&2		&2		&1.00	&1.00	&–\\
\hline

\multicolumn{3}{r||}{\textit{Indoeuropean}}&\textit{65}&\textit{9}&–&\textit{7.22}	&low\\
&\multicolumn{2}{l||}{Albanian}		&2		&2		&4.50	&1.00	&–\\
&\multicolumn{2}{l||}{Armenian}		&1		&2		&4.50	&0.50	&–\\
&\multicolumn{2}{l||}{Baltic}		&2		&2		&4.50	&1.00	&–\\
&\multicolumn{2}{l||}{Celtic}		&6		&2		&4.50	&3.00	&low\\
&&Brittonic					&3 		&1		&2.00	&3.00	&–\\
&&Gaelic						&3 		&2		&1.00	&1.50	&–\\
&\multicolumn{2}{l||}{Germanic}		&14		&5		&1.29	&2.00	&high\\
&&N-Germanic					&6		&4		&1.25	&2.25	&mid\\
&&W-Germanic 				&8		&3		&1.67	&2.67	&mid\\
&\multicolumn{2}{l||}{Hellenic}		&1		&2		&4.50	&0.50	&–\\
&\multicolumn{2}{l||}{Indo-Iranian}	&14		&7		&1.23	&2.00	&high\\		
&&Indo-Aryan					&6		&3		&2.33	&2.00	&mid\\
&&Iranian						&8		&6		&1.67	&1.33	&high\\
&\multicolumn{2}{l||}{Romance}		&10		&2		&4.50	&5.00	&low\\
&&E-Romance					&1		&2		&1.00	&0.50	&–\\
&&Italo-W-Romance				&7		&1		&2.00	&7.00	&low\\
&&S-Romance					&2		&1		&2.00	&2.00	&–\\	
&\multicolumn{2}{l||}{Slavic}		&13		&3		&3.00	&4.33	&low\\
&&E-Slavic					&3		&2		&1.50	&1.50	&–\\
&&S-Slavic					&4		&3		&1.00	&1.33	&mid\\
&&W-Slavic					&6		&1		&3.00	&6.00	&very low\\
\hline
\multicolumn{3}{r||}{\textit{Kartvelian}}&\textit{4}	&\textit{3}	&–	&\textit{1.33}	&mid\\
&\multicolumn{2}{l||}{Georgian}		&2		&3		&1.00	&0.67	&–\\
&\multicolumn{2}{l||}{Svan}		&1		&2		&2.00	&0.50	&–\\
&\multicolumn{2}{l||}{Zan}			&2		&2		&2.00	&1.00	&–\\
\hline
\multicolumn{3}{r||}{\textit{Mongolic}}&\textit{6}&\textit{1}&–&\textit{6.00}		&very low\\
&\multicolumn{2}{l||}{Dagur}		&1		&1		&1.00	&1.00	&–\\
&\multicolumn{2}{l||}{Moghol}		&1		&1		&1.00	&1.00	&–\\
&\multicolumn{2}{l||}{Mongolian}	&5		&1		&1.00	&5.00	&very low\\
\hline
\multicolumn{3}{r||}{\textit{Nakh-Daghestanian}}&\textit{28}&\textit{5}&–&\textit{5.60}&low\\
&\multicolumn{2}{l||}{Daghestanian}	&25	&5	&1.00	&5.00			&mid\\
&&Avar-Andic-Tsezic			&13	&4	&1.25	&3.25			&mid\\
&&Dargwa					&1	&2	&2.50	&0.50			&–\\
&&Lak						&1	&2	&2.50	&0.50			&–\\
&&Lezgian					&10	&4	&1.25	&2.25			&mid\\
&\multicolumn{2}{l||}{Nakh}		&3	&3	&1.67	&1.00			&–\\
&&Bats						&1	&2	&1.50	&0.50			&–\\
&&Chechen-Ingush				&2	&2	&1.50	&1.00			&–\\
\hline
\multicolumn{3}{r||}{\textit{Tungusic}}&\textit{10}&\textit{4}&–&\textit{2.25}		&mid\\
&\multicolumn{2}{l||}{Amur Tungusic}	&5	&4	&1.00	&1.25			&high\\
&&Nanay-Ulcha-Orok			&3	&3	&1.33	&1.00			&–\\
&&Oroch-Udege				&2	&3	&1.33	&0.67			&–\\
&\multicolumn{2}{l||}{Manchu}		&1	&1	&4.00	&1.00			&–\\
&\multicolumn{2}{l||}{N-Tungusic}	&4	&4	&1.00	&1.00			&high\\
\hline
\multicolumn{3}{r||}{\textit{Turkic}}&\textit{22}&\textit{2}&–&\textit{11.00}		&very low\\
&\multicolumn{2}{l||}{Bulgar}		&1 	&2	&1.00	&0.50			&–\\
&\multicolumn{2}{l||}{Common Turkic}&21 	&2	&1.00	&10.50			&low\\
&&Altay						&2	&1	&2.00	&2.00			&–\\
&&Karluk						&2	&2	&1.00	&1.00			&–\\
&&Kipchak					&8	&1	&2.00	&8.00			&very low\\
&&Lena						&2	&1	&2.00	&2.00			&–\\
&&Oguz						&4	&2	&1.00	&2.00			&low\\
&&Yenisey					&2	&1	&2.00	&2.00			&–\\
\hline
\multicolumn{3}{r||}{\textit{Uralic}}	&\textit{32}&\textit{5}&–&\textit{6.40}		&low\\
&\multicolumn{2}{l||}{Finnic}		&7	&1	&5.00	&7.00			&low\\
&\multicolumn{2}{l||}{Hungarian}	&1	&1	&4.00	&1.00	&–\\
&\multicolumn{2}{l||}{Khanty}		&1	&1	&4.00	&1.00&–\\
&\multicolumn{2}{l||}{Mansi}		&1	&1	&4.00	&1.00&–\\
&\multicolumn{2}{l||}{Mari}			&2	&2	&2.00	&1.00&–\\
&\multicolumn{2}{l||}{Mordvin}		&2	&1	&4.00	&2.00&–\\
&\multicolumn{2}{l||}{Permic}		&3	&3	&1.33	&1.00&–\\
&\multicolumn{2}{l||}{Saamic}		&9	&4	&1.25	&1.80			&high\\
&&E-Saamic					&4	&3	&1.33	&1.33			&high\\
&&W-Saamic					&5	&4	&1.00	&1.25			&high\\
&\multicolumn{2}{l||}{Samoyedic}	&4	&1	&5.00	&4.00			&low\\
&&Enets						&1	&1	&2.00	&1.00&–\\
&&Nenets						&1	&1	&2.00	&1.00&–\\
&&Nganasan					&1	&1	&2.00	&1.00&–\\
&&Selkup						&1	&1	&2.00	&1.00&–\\
\hline
\multicolumn{3}{r||}{\textit{Yukagir}}	&\textit{2}&\textit{2}&–&\textit{1.00}&–\\
&\multicolumn{2}{l||}{Yukagir}		&2	&2	&1.00	&1.00&–\\
\hline
\hline
\caption[Number and ratio of attested types per genealogical unit]{Number and ratio of attested types per genealogical unit: absolute number of types (column 3), ratio against the generally attested number of types in the respective higher branch or family (column 4, higher numbers mean less diversity), ratio against the number of coded languages (column 5, higher numbers mean less diversity) and a diversity value tested for statistical significance (column 6, only for genera with more than 3 languages).}\label{diversity}
\end{longtable}
\end{footnotesize}
The absolute number of types shows directly which families or branches inside families exhibit more types than other comparable genera. The first ratio in column 4 (against the number of types in the genus) indicates where the more diverse or the more uniform branches are located inside a primary genus (i.e.~inside a family or higher branch). These ratio figures can be used for a comparison of languages inside families or between comparable genera across families because the figures result from dividing the absolute number of  attested adjective attribution marking devices in a given family by the number of devices attested in a given sub-genus (i.e.~branch or subbranch). East-Saamic (Uralic) with a ratio of $1.33$, for instance, seems just as diverse as South-Slavic (Indoeuropean). The proto-stages of both genera have comparable time depth (approx. 1000 AD), both genera have four members and they both exhibit three attested types of adjective attribution marking devices. The number of three attested types in the two branches can than be checked against the overall number of types attested in the respective families: 4 types are attested in Uralic, 9 types are attested in Indoeuropean. As compared to Uralic, the Saamic branch with a ratio of $2.25$ is thus much more diverse (exhibiting almost all types attested for the whole family) than South-Slavic languages within Indoeuropean with a ratio of $4.33$ (exhibiting less than half of the generally attested types in the whole family).

The second ratio in column 5 (against the number of coded languages) relativizes the first two figures statistically. It seems much more likely that a higher number of coded languages results in a higher number of detected devices. The second ratio can thus serve to test the degree of diversity (in column 3 and 4) for statistical significance. 

The simple statistics presented in Table  \ref{diversity} can perhaps illustrate the degree of diversity at least in these cases where the two ratio figures (against the number of coded types and the number of coded languages) and the degree of diversity in absolute numbers coincide to a certain degree. The significant values in column 6 (“Diversity”) are labeled impressionistically as {\it  very low, low, mid-low, mid-high, high}. A hyphen marks these cases where a significant value cannot be found because the genera in question has too few members (less than 4). Note that a value {\it very high} is not found. This classification, however, does not mark diversity in absolute terms but the deviation from the average value of the whole sample. The Turkic family, for instance,  can be shown to have a very low diversity and several of its branches have clearly low (or low in direction) diversity level as well. For the Mongolic family, a very low value has been calculated. Whereas a low value has even been calculated for the whole Nakh-Daghestanian family, the Daghestanian branch as well as two of its subbranches have a relatively high diversity value. The same is true for Uralic, which has low diversity value as a family and in several of its branches, while one branch, Saamic, has a high value. Tungusic has an middle diversity value but two of its branches are clearly higher diverse. For Indoeuropean, finally, a significant value has not been found, but high values are calculated for Indo-Iranian and Germanic.

The general picture coincides thus partly with what is known about areal distribution and spread of other linguistic features \citep[cf., e.g.,][]{nichols1992}: Less diversity (higher numbers) is found in the inner parts of North-Asia (Mongolic, Turkic) whereas languages in the northern Eurasian periphery, especially in south-easternmost Europe (Caucasus) but also in Northeastern-Europe (Circum-Baltic) and in north-easternmost Asia (Pacific Rim), seem to exhibit a higher degree of diversity (lower numbers) in respect to the morpho-syntax of adjective attribution.

Event though the figures in Table \ref{diversity} illustrate exclusively synchronic findings and the applied statistics is rather impressionistic, it stands to reason that they reflect historical developments (i.e.~language changes) in certain parts of the area. Note that the underlying sample is not balanced and thus perhaps not easily applicable for statistical analyses. However, this is an exploratory study; detailed statistical investigations can be left for future research.

The massive innovations in several neighboring genera or in larger geographic sub-areas attested synchronically may even point to contact-induced changes in areal hotbeds of innovation. In Part \ref{part diachr}, some light will be shed on diachronic variation and on the evolution of highly diverse adjective attribution marking inside language families of northern Eurasia.
