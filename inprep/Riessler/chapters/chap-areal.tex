
\chapter[Areal uniformity and diversity]{Areal uniformity and diversity in northern Eurasia}\label{areality}
%%%
In the previous chapter, the prototypical and the known minor noun phrase types occurring in the languages of northern Eurasia were characterized and illustrated with examples. This survey thus provides an overall picture of the degree of typological uniformity or divergence with regard to adjective attribution marking marking within both the whole area and each genealogical unit.

\section{Attested attribution marking devices}
%%%
13 (simple and combined) types of adjective attribution marking devices are attested in the languages of northern Eurasia:
%%%
\begin{enumerate}
\item Anti\hyp{}construct state\\as in \ili{Kildin Saami}\is{anti\hyp{}construct state}
\item Anti\hyp{}construct state + \isi{head\hyp{}driven agreement} (“double agreement”)\\as in \ili{Swedish}\is{anti\hyp{}construct state}
\item Anti\hyp{}construct state + \isi{construct state} (“double\hyp{}construct state”)\\as in \ili{Northern Saami}\is{anti\hyp{}construct state}
\item Anti\hyp{}construct state agreement\\as in \ili{Russian}\is{anti\hyp{}construct state agreement}
\item Appositional head\hyp{}driven agreement\\as in \ili{Georgian}\is{appositional head\hyp{}driven agreement}
\item Attributive article\\as in \ili{Yiddish}\is{attributive article}
\item Attributive article + \isi{head\hyp{}driven agreement} (“double agreement”)\\as in \ili{Albanian}\is{attributive article}
\item Attributive nominalization\\as in \ili{Udmurt}\is{attributive nominalization}
\item Construct state\\as in \ili{Northern Kurdish}\is{attributive nominalization}\is{construct state}
\item Incorporation\\as in \ili{Chukchi}\is{incorporation}
\item Juxtaposition\\as in \ili{Komi-Zyrian}\is{juxtaposition}
\item Head\hyp{}driven agreement\\as in \ili{Finnish}\is{head\hyp{}driven agreement}
\item Modifier\hyp{}headed possessor agreement\\as in \ili{Oroch}\is{modifier\hyp{}headed possessor agreement}
\end{enumerate}
%%%
Only one type attested in the world-wide sample (see the appendix) does not occur in the northern Eurasian area: the floating construct state marker (\textit{linker}) found, for instance, in \ili{Tagalog} (Austronesian). \is{linker}

The Indo-European\il{Indo-European languages} family has the largest absolute number of attested adjective attribution marking devices (nine). It is followed by Nakh-Daghestanian\il{Nakh-Daghestanian languages} and Uralic\il{Uralic languages} (five each) and Kartvelian\il{Kartvelian languages} and Tungusic\il{Tungusic languages} (four each). The Mongolic\il{Mongolic languages} family has the lowest possible number with only one attested device, just as with Kamchatkan\il{Kamchatkan languages} and the isolates Ainu,\il{Ainu} Basque,\il{Basque} Korean\il{Korean} and Nivkh.\il{Nivkh}

The most rare types are: (1) \isi{modifier\hyp{}headed possessor agreement}, which is attested only as a secondary device in a few \ili{Tungusic languages}, and (2) the combined construct device (i.e.~“double\hyp{}construct state”), which is attested only marginally in one single language, \ili{Northern Saami} (Uralic). Attributive nominalization\is{attributive nominalization} combined  with \isi{head\hyp{}driven agreement} is also very rare. This type occurs as the primary device only in the \ili{Albanian languages} (Indo-European), but it is also attested as a secondary or tertiary device in a few other languages. Head-marking construct state is also relatively uncommon in the northern Eurasian area as it is attested only in \ili{Iranian languages} (Indo-European).

The most common type is \isi{juxtaposition}, followed by \isi{head\hyp{}driven agreement}.

\section{Prototypes of attribution marking devices}
%%%
Several language families of northern Eurasia exhibit clear prototypes of adjective attribution marking devices: All Mongolic\il{Mongolic languages} and Turkic\il{Turkic languages} languages have \isi{juxtaposition} as the default device, as is the case for the languages of most branches of Uralic\il{Uralic languages} as well. Head\hyp{}driven agreement occurs as another prototype in many branches of the Indo-European\il{Indo-European languages} family. Even though the attested deviation from the prototype is much higher in Indo-European\il{Indo-European languages} than in Mongolic,\il{Mongolic languages} Turkic\il{Turkic languages} and Uralic,\il{Uralic languages} \isi{head\hyp{}driven agreement} marking can be shown to occur prototypically in most Indo-European\il{Indo-European languages} genera.

For the Abkhaz-Adyghe,\il{Abkhaz-Adyghe languages} Chukotkan,\il{Chukotkan languages} Kartvelian,\il{Kartvelian languages} Nakh-Daghestanian\il{Nakh-Daghestanian languages} and Tungusic\il{Tungusic languages} families, synchronic prototypes are not very easy to find because a predominant type does not occur inside these families. The other language families of northern Eurasia are either isolates (Nivkh,\il{Nivkh} Ainu,\il{Ainu} Japanese, Korean,\il{Korean} Basque)\il{Basque} or they exhibit rather shallow genealogical diversity (Kamchatkan,\il{Kamchatkan languages} Yukaghir, Yeniseian). Together with a few other families, predominantly spoken outside the investigated area (Eskimo-Aleut,\il{Eskimo-Aleut languages} Sino-Tibetan,\il{Sino-Tibetan languages} Semitic),\il{Semitic languages} these families are excluded from generalizations about prototypes. 

Larger language families representing a strikingly high diversity in regard to the attested absolute number of adjective attribution marking devices are Indo-European,\il{Indo-European languages} Nakh-Daghestanian,\il{Nakh-Daghestanian languages} Uralic\il{Uralic languages} and Tungusic.\il{Tungusic languages} A strikingly high degree of unity is found in Mongolic\il{Mongolic languages} and Turkic.\il{Turkic languages}

\section[Diachronic implications]{Diachronic implications of uniformity and diversity inside and across genera}
%%%
Measuring the degree of diversity (or unity) from a synchronic point of view may help identify diachronic processes. A very high degree of diversity inside a given genus as compared to its proto-stage is likely to manifest pervasive linguistic changes and the innovation of new types. Similarly, the synchronic attestation of a high degree of unity inside a given genus indicates the inheritance of original types without significant innovations. 

A genus is defined as a group of related languages which go back to a common reconstructed (or documented) language, ultimately the the proto-form of a language family. The East Saamic\il{West Saamic languages} languages, for instance, form a group of sister languages which derived from Proto\hyp{}East-Saamic.\il{Proto\hyp{}East Saamic} Proto\hyp{}East-Saamic\il{Proto\hyp{}East Saamic} is derived together with its Saamic\il{Saamic languages} sister languages from a more distant proto-stage, i.e.~Proto\hyp{}Saamic\il{Proto\hyp{}Saamic} which again is derived together with its Uralic\il{Uralic languages} sister languages from Proto\hyp{}Uralic.\il{Proto\hyp{}Uralic} Since the proto-stages of languages are normally reconstructed as single languages it can be assumed that most of them had only one single type of adjective attribution marking (similar to the prevailing number of languages spoken today, cf.~the sample in Table~\ref{sample1} in the appendix). Daughter languages which descend from a proto-language will either inherit the original adjective attribution marking devices, innovate secondary (or tertiary etc.) devices or replace the original devices with new ones. The Proto\hyp{}Saamic\il{Proto\hyp{}Saamic} daughter language of Proto\hyp{}Uralic,\il{Proto\hyp{}Uralic} for instance, has replaced the original Uralic\il{Uralic languages} \isi{juxtaposition} with anti\hyp{}construct state marking (see \S~\ref{saamic diachr}). The \ili{Proto\hyp{}Baltic\slash{}Slavic} daughter languages of Proto\hyp{}Indo-European\il{Proto\hyp{}Indo-European} inherited the original Indo-European\il{Indo-European languages} \isi{head\hyp{}driven agreement} marking but innovated a secondary type, i.e.~anti\hyp{}construct state agreement marking (see \S~\ref{slavic diachr}). All modern Mongolic\il{Mongolic languages} languages, by contrast, exhibit juxtaposition uniformly and have obviously inherited this device from their proto-languages (Proto\hyp{}Dagur,\il{Proto\hyp{}Dagur} Proto\hyp{}Moghol,\il{Proto\hyp{}Moghol} \ili{Proto\hyp{}Mongolic}, etc.) which in turn must have inherited juxtaposition from Proto\hyp{}Mongolic.\il{Proto\hyp{}Mongolic} A comparison of synchronically attested diversity inside and across genera might thus have diachronic implications.

The simple statistics in Table~\ref{diversity} tries to illustrate the degree of diversity in the investigated families of northern Eurasia. Column 1 lists all families, branches and subbranches in alphabetical order. Isolates and genera with only one member language are not included in the table, and neither are genera which are not spoken predominantly in northern Eurasia, with only two exceptions: the Iranian\il{Iranian languages} and Indo-Aryan\il{Indo-Aryan languages} subbranches within the Indo-European\il{Indo-European languages} family. Since the highest possible diversity is of interest here, the number of all attested devices devices (including secondary and tertiary types restricted to special noun phrase types) are counted. 

The second column in Table~\ref{diversity} (“Units (abs.)”) gives the number of coded languages from each genus. The third column (“Types (abs.)”) gives the absolute number of attested types. The next two columns 4 and 5 present ratio figures. The first of them (“Ratio (gen.)”) results from dividing the number of attested types in the given genus by the number of types attested for the higher branch:\medskip

$\text{Diversity}_{\text{genus}} = \frac{\text{Types}_{\text{genus}}}{\text{Types}_{\text{family}}}$.\medskip

\noindent For instance, West-Saamic\il{West Saamic languages} has a ratio of $1.00$ because it exhibits all four types attested in the whole Saamic\il{Saamic languages} branch. The Saamic\il{Saamic languages} branch as such has a ratio of $1.25$ because four types are found in Saamic\il{Saamic languages} compared to five types attested for the whole Uralic\il{Uralic languages} family. Similarly, South Slavic\il{South Slavic languages} has also a ratio of $1.00$ because it exhibits all three types attested in Slavic.\il{Slavic languages} But the Slavic\il{Slavic languages} branch as such has a higher ratio of $3.00$ (meaning a lesser degree of diversity) because only three types are attested in this branch out of nine types for the whole Indo-European\il{Indo-European languages} family. 

The last ratio figures (“Ratio (lgs.)”) result from dividing the overall number of languages by the number of attested types in the given genus:\medskip

$\text{Diversity}_{\text{languages}} = \frac{\text{Languages}_{\text{genus}}}{\text{Types}_{\text{genus}}}$.\medskip

\noindent For instance, five West-Saamic\il{West Saamic languages} languages are coded for four different types, resulting in a ratio of $1.25$. For the whole Saamic\il{Saamic languages} branch alltogether nine languages are coded for five types, resulting in a somewhat higher ratio figure of $1.80$. South Slavic\il{South Slavic languages} has the ratio of $1.33$ because the four South Slavic\il{South Slavic languages} languages are coded for three types; Slavic,\il{Slavic languages} however, has $4.33$ because 13 Slavic\il{Slavic languages} languages are coded for only three different types.

\begin{longtable}[h]{llS[detect-all]S[detect-all]S[detect-all]S[detect-all]c}
\lsptoprule
\multicolumn{2}{r}{1}			&\multicolumn{1}{c}{2}		&\multicolumn{1}{c}{3}		&\multicolumn{1}{c}{4}		&\multicolumn{1}{c}{5}		&\multicolumn{1}{c}{6}\\
\midrule
\multicolumn{2}{r}{\textit{Family}}	&\multicolumn{1}{c}{Languages}	&\multicolumn{1}{c}{Types}	&\multicolumn{1}{c}{Ratio}	&\multicolumn{1}{c}{Ratio}	&\multicolumn{1}{c}{Diversity}\\
\multicolumn{2}{l}{Main branch}	&\multicolumn{1}{c}{(abs.)}	&\multicolumn{1}{c}{(abs.)}	&\multicolumn{1}{c}{(gen.)}	&\multicolumn{1}{c}{(lgs.)}	&\multicolumn{1}{c}{value}\\
\midrule\endfirsthead
\midrule
\multicolumn{2}{r}{\textit{Family}}	&\multicolumn{1}{c}{Languages}	&\multicolumn{1}{c}{Types}	&\multicolumn{1}{c}{Ratio}	&\multicolumn{1}{c}{Ratio}	&\multicolumn{1}{c}{Diversity}\\
\multicolumn{2}{l}{Main branch}	&\multicolumn{1}{c}{(abs.)}	&\multicolumn{1}{c}{(abs.)}	&\multicolumn{1}{c}{(gen.)}	&\multicolumn{1}{c}{(lgs.)}	&\multicolumn{1}{c}{value}\\
\midrule\endhead
\multicolumn{2}{r}{\textit{Abkhaz-Adyghe}}&\textit{4}&\textit{2}&–&\textit{2.00}	&low\il{Abkhaz-Adyghe languages}\\
\midrule
\multicolumn{2}{l}{Abkhaz}		&2		&1		&2.00	&2.00	&–\il{Abkhaz languages}\\
\multicolumn{2}{l}{Circassian}		&2		&1		&2.00	&2.00	&–\il{Circassian languages}\\
\midrule
\multicolumn{2}{r}{\textit{Chukotkan}}&\textit{3}&\textit{2}&–&\textit{1.50}		&–\il{Chukotkan languages}\\
\midrule
\multicolumn{2}{l}{Chukchi}		&1		&2		&1.00	&0.50	&–\il{Chukchi languages}\\
\multicolumn{2}{l}{Koryak-Alutor}	&2		&2		&1.00	&1.00	&–\il{Koryak-Alutor languages}\\
\midrule
\multicolumn{2}{r}{\textit{Indo-European}}&\textit{65}&\textit{9}&–&\textit{7.22}	&low\il{Indo-European languages}\\\midrule
\multicolumn{2}{l}{Albanian}		&2		&2		&4.50	&1.00	&–\il{Albanian languages}\\
\multicolumn{2}{l}{Armenian}		&1		&2		&4.50	&0.50	&–\il{Armenian languages}\\
\multicolumn{2}{l}{Baltic}		&2		&2		&4.50	&1.00	&–\il{Baltic languages}\\
\multicolumn{2}{l}{Celtic}		&6		&2		&4.50	&3.00	&low\il{Celtic languages}\\
&Brittonic					&3 		&1		&2.00	&3.00	&–\il{Brittonic languages}\\
&Gaelic						&3 		&2		&1.00	&1.50	&–\il{Gaelic languages}\\
\multicolumn{2}{l}{Germanic}		&14		&5		&1.29	&2.00	&high\il{Germanic languages}\\
&N-Germanic					&6		&4		&1.25	&2.25	&mid\il{North Germanic languages}\\
&W-Germanic 				&8		&3		&1.67	&2.67	&mid\il{West Germanic languages}\\
\multicolumn{2}{l}{Hellenic}		&1		&2		&4.50	&0.50	&–\il{Hellenic languages}\\
\multicolumn{2}{l}{Indo-Iranian}	&14		&7		&1.23	&2.00	&high\il{Indo-Iranian languages}\\		
&Indo-Aryan					&6		&3		&2.33	&2.00	&mid\il{Indo-Aryan languages}\\
&Iranian						&8		&6		&1.67	&1.33	&high\il{Iranian languages}\\
\multicolumn{2}{l}{Romance}		&10		&2		&4.50	&5.00	&low\il{Romance languages}\\
&E-Romance					&1		&2		&1.00	&0.50	&–\il{East Romance languages}\\
&Italo-W-Romance				&7		&1		&2.00	&7.00	&low\il{Italo-West Romance languages}\\
&S-Romance					&2		&1		&2.00	&2.00	&–\il{South Romance languages}\\	
\multicolumn{2}{l}{Slavic}		&13		&3		&3.00	&4.33	&low\il{Slavic languages}\\
&E-Slavic					&3		&2		&1.50	&1.50	&–\il{East Slavic languages}\\
&S-Slavic					&4		&3		&1.00	&1.33	&mid\il{South Slavic languages}\\
&W-Slavic					&6		&1		&3.00	&6.00	&very low\il{West Slavic languages}\\
\midrule
\multicolumn{2}{r}{\textit{Kartvelian}}&\textit{4}	&\textit{3}	&–	&\textit{1.33}	&mid\il{Kartvelian languages}\\\midrule
\multicolumn{2}{l}{Georgian}		&2		&3		&1.00	&0.67	&–\il{Georgian languages}\\
\multicolumn{2}{l}{Svan}		&1		&2		&2.00	&0.50	&–\il{Svan languages}\\
\multicolumn{2}{l}{Zan}			&2		&2		&2.00	&1.00	&–\il{Zan languages}\\
\midrule
\multicolumn{2}{r}{\textit{Mongolic}}&\textit{6}&\textit{1}&–&\textit{6.00}		&very low\il{Mongolic languages}\\\midrule
\multicolumn{2}{l}{Dagur}		&1		&1		&1.00	&1.00	&–\il{Dagur languages}\\
\multicolumn{2}{l}{Moghol}		&1		&1		&1.00	&1.00	&–\il{Moghol languages}\\
\multicolumn{2}{l}{Mongolian}	&5		&1		&1.00	&5.00	&very low\il{Mongolian languages}\\
\midrule
\multicolumn{2}{r}{\textit{Nakh-Daghestanian}}&\textit{28}&\textit{5}&–&\textit{5.60}&low\il{Nakh-Daghestanian languages}\\\midrule
\multicolumn{2}{l}{Daghestanian}	&25	&5	&1.00	&5.00			&mid\il{Daghestanian languages}\\
&Avar-Andi-Tsezic			&13	&4	&1.25	&3.25			&mid\il{Avar-Andi-Tsezic languages}\\
&Dargwa					&1	&2	&2.50	&0.50			&–\il{Dargwa languages}\\
&Lak						&1	&2	&2.50	&0.50			&–\il{Lak languages}\\
&Lezgian					&10	&4	&1.25	&2.25			&mid\il{Lezgic languages}\\
\multicolumn{2}{l}{Nakh}		&3	&3	&1.67	&1.00			&–\il{Nakh languages}\\
&Bats						&1	&2	&1.50	&0.50			&–\il{Bats languages}\\
&Chechen-Ingush				&2	&2	&1.50	&1.00			&–\il{Chechen-Ingush languages}\\
\midrule
\multicolumn{2}{r}{\textit{Tungusic}}&\textit{10}&\textit{4}&–&\textit{2.25}		&mid\il{Tungusic languages}\\\midrule
\multicolumn{2}{l}{Amur Tungusic}	&5	&4	&1.00	&1.25			&high\il{Amur Tungusic languages}\\
&Nanay-Ulcha-Orok			&3	&3	&1.33	&1.00			&–\il{Nanay-Ulcha-Orok languages}\\
&Oroch-Udege				&2	&3	&1.33	&0.67			&–\il{Oroch-Udege languages}\\
\multicolumn{2}{l}{Manchu}		&1	&1	&4.00	&1.00			&–\il{Manchu languages}\\
\multicolumn{2}{l}{N-Tungusic}	&4	&4	&1.00	&1.00			&high\il{North Tungusic languages}\\
\midrule
\multicolumn{2}{r}{\textit{Turkic}}&\itshape 22&\textit{2}&–&\textit{11.00}		&very low\il{Turkic languages}\\\midrule
\multicolumn{2}{l}{Bulgar}		&1 	&2	&1.00	&0.50			&–\il{Bulgar Turkic languages}\\
\multicolumn{2}{l}{Common Turkic}&21 	&2	&1.00	&10.50			&low\il{Common Turkic languages}\\
&Altay						&2	&1	&2.00	&2.00			&–\il{Altay Turkic languages}\\
&Karluk						&2	&2	&1.00	&1.00			&–\il{Karluk languages}\\
&Kipchak					&8	&1	&2.00	&8.00			&very low\il{Kipchak languages}\\
&Lena						&2	&1	&2.00	&2.00			&–\il{Lena Turkic languages}\\
&Oguz						&4	&2	&1.00	&2.00			&low\il{Oguz languages}\\
&Yenisey					&2	&1	&2.00	&2.00			&–\il{Yenisey Turkic languages}\\
\midrule
\multicolumn{2}{r}{\textit{Uralic}}	&\textit{32}&\textit{5}&–&\textit{6.40}		&low\il{Uralic languages}\\\midrule
\multicolumn{2}{l}{Finnic}		&7	&1	&5.00	&7.00			&low\il{Finnic languages}\\
\multicolumn{2}{l}{Hungarian}	&1	&1	&4.00	&1.00			&–\il{Hungarian}\\
\multicolumn{2}{l}{Khanty}		&1	&1	&4.00	&1.00			&–\il{Khanty languages}\\
\multicolumn{2}{l}{Mansi}		&1	&1	&4.00	&1.00			&–\il{Mansi languages}\\
\multicolumn{2}{l}{Mari}			&2	&2	&2.00	&1.00			&–\il{Mari languages}\\
\multicolumn{2}{l}{Mordvin}		&2	&1	&4.00	&2.00			&–\il{Mordvin languages}\\
\multicolumn{2}{l}{Permic}		&3	&3	&1.33	&1.00			&–\il{Permic languages}\\
\multicolumn{2}{l}{Saamic}		&9	&4	&1.25	&1.80			&high\il{Saamic languages}\\
&E-Saamic					&4	&3	&1.33	&1.33			&high\il{East Saamic languages}\\
&W-Saamic					&5	&4	&1.00	&1.25			&high\il{West Saamic languages}\\
\multicolumn{2}{l}{Samoyedic}	&4	&1	&5.00	&4.00			&low\il{Samoyedic languages}\\
&Enets						&1	&1	&2.00	&1.00			&–\il{Enets languages}\\
&Nenets						&1	&1	&2.00	&1.00			&–\il{Nenets languages}\\
&Nganasan					&1	&1	&2.00	&1.00			&–\il{Nganasan}\\
&Selkup						&1	&1	&2.00	&1.00			&–\il{Selkup languages}\\
\midrule
\multicolumn{2}{r}{\textit{Yukaghir}}	&\textit{2}&\textit{2}&–&\textit{1.00}		&–\il{Yukaghir languages}\\
\lspbottomrule
\caption[Number and ratio of attested types per genealogical unit]{Number and ratio of attested types per genealogical unit: absolute number of types (column 3), ratio against the generally attested number of types in the respective higher branch or family (column 4, higher numbers mean less diversity), ratio against the number of coded languages (column 5, higher numbers mean less diversity) and a diversity value tested for statistical significance (column 6, only for genera with more than three languages).}
\label{diversity}
\end{longtable}
% \todo{kursivgesetzte Zahlen noch nicht optimal ausgerichtet.}

The absolute number of types shows directly which families or branches inside families exhibit more types than other comparable genera. The first ratio in column 4 (against the number of types in the genus) indicates where the more diverse or the more uniform branches are located inside a primary genus (i.e.~inside a family or higher branch). These ratio figures can be used for a comparison of languages inside families or between comparable genera across families because the figures result from dividing the absolute number of attested adjective attribution marking devices in a given family by the number of devices attested in a given sub-genus (i.e.~branch or subbranch). East Saamic\il{West Saamic languages} (Uralic) with a ratio of $1.33$, for instance, seems just as diverse as South Slavic\il{South Slavic languages} (Indo-European). The proto-stages of both genera have comparable time depth (approx. 1000 AD), both genera have four members and they both exhibit three attested types of adjective attribution marking devices. The number of three attested types in the two branches can than be checked against the overall number of types attested in the respective families: four types are attested in Uralic,\il{Uralic languages} nine types are attested in Indo-European.\il{Indo-European languages} As compared to Uralic,\il{Uralic languages} the Saamic\il{Saamic languages} branch with a ratio of $2.25$ is thus much more diverse (exhibiting almost all types attested for the whole family) than South Slavic languages within Indo-European\il{Indo-European languages} with a ratio of $4.33$ (exhibiting less than half of the generally attested types in the whole family).

The second ratio in column 5 (against the number of coded languages) relativizes the first two figures statistically. It seems much more likely that a higher number of coded languages results in a higher number of detected devices. The second ratio can thus serve to test the degree of diversity (in column 3 and 4) for statistical significance. 

The simple statistics presented in Table \ref{diversity} can perhaps illustrate the degree of diversity at least in those cases where the two ratio figures (against the number of coded types and the number of coded languages) and the degree of diversity in absolute numbers coincide to a certain degree. The significant values in column 6 (“Diversity”) are labeled impressionistically as \textit{very low, low, mid-low, mid-high, high}. A hyphen marks those cases where a significant value cannot be found because the genus in question has too few members (less than four). Note that a value \textit{very high} is not found. This classification, however, does not mark diversity in absolute terms but the deviation from the average value of the whole sample. The Turkic\il{Turkic languages} family, for instance, can be shown to have a very low diversity and several of its branches have clearly low diversity level as well. For the Mongolic\il{Mongolic languages} family, a very low value has been calculated. Whereas a low value has even been calculated for the whole Nakh-Daghestanian\il{Nakh-Daghestanian languages} family, the Daghestanian\il{Daghestanian languages} branch as well as two of its subbranches have a relatively high diversity value. The same is true for Uralic,\il{Uralic languages} which has a low diversity value as a family and in several of its branches, while one branch, Saamic,\il{Saamic languages} has a high value. Tungusic\il{Tungusic languages} has a middle diversity value but two of its branches are clearly more highly diverse. For Indo-European,\il{Indo-European languages} finally, a significant value has not been found, but high values are calculated for Indo-Iranian\il{Indo-Iranian languages} and Germanic.\il{Germanic languages}

The general picture coincides thus partly with what is known about areal distribution and spread of other linguistic features \citep[cf., e.g.,][]{nichols1992}: Less diversity (higher numbers) is found in the inner parts of \isi{North Asia} (Mongolic,\il{Mongolic languages} Turkic)\il{Turkic languages} whereas languages in the northern Eurasian periphery, especially in south-easternmost Europe (\isi{Caucasus}) but also in north-easternmost Europe (Circum-Baltic)\is{Circum-Baltic area} and in north-easternmost Asia (Pacific Rim),\is{Pacific Rim area} seem to exhibit a higher degree of diversity (lower numbers) in respect to the morpho-syntax of adjective attribution.

Even though the figures in Table~\ref{diversity} illustrate exclusively synchronic findings and the applied statistics is rather impressionistic, it stands to reason that they reflect historical developments (i.e.~language changes) in certain parts of the area. Note that the underlying sample is not balanced and thus perhaps not easily applicable for statistical analyses. However, this is an exploratory study; detailed statistical investigations can be left for future research.

The massive innovations in several neighboring genera or in larger geographic sub-areas attested synchronically may even point to contact-induced changes in areal hotbeds of innovation. In Part~IV Diachrony, some light will be shed on diachronic variation and on the evolution of highly diverse adjective attribution marking inside language families of northern Eurasia.
