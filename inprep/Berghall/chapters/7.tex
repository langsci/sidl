%7

\chapter{Sentence types}
%\hypertarget{RefHeading22681935131865}
{}

The basic speech acts are mostly expressed by the functional sentence types typical of them: a statement by a declarative sentence, a question by an interrogative sentence and a command by an imperative sentence.  

\section{Statements}
%\hypertarget{RefHeading22701935131865}
{}
The declarative sentence, used to make a statement/assertion, is the unmarked sentence type, default in narrative, descriptive and procedural texts and common in other text types as well. The final verb has full tense and person/number marking. The intonation pattern in declarative sentences is falling (\sectref{sec:2.1.3.2}). 

\section{Questions} \label{sec:7.2}
%\hypertarget{RefHeading22721935131865}
{}
The basic function of questions, or interrogative sentences, is to request either information or some action from the addressee(s). Rhetorical questions have other functions as well. Structurally the two basic types are non-polar, or content questions and polar, or yes-no questions. Echo questions and confirmation questions are modifications of these. 

\subsection{Non-polar questions} \label{sec:7.2.1}
%\hypertarget{RefHeading22741935131865}
{}
Non-polar questions, or content questions, require the use of question words (\sectref{sec:3.7.1}). There is no question-word fronting: a question word occupies the same position that the questioned element would have in a statement.\footnote{This is typical of Papuan SOV languages \citep[294]{Reesink1987}.} The intonation is falling like in a statement, but the stressed syllable of the question word has a slightly higher pitch than the words before and/or after it (\sectref{sec:2.1.3.2}).

Any argument or peripheral in a clause can be questioned, as well their constituents. 

\ea%x1167
\label{ex:7:x1167}
\gll (\textstyleEmphasizedVernacularWords{Mua})  \textstyleEmphasizedVernacularWords{naareke}  koora  ku-am-ika-i-ya? \\
  (man)  who.\textsc{cf}  house  build-\textsc{ss}.\textsc{sim}-be-Np-3s    \\
\glt `Who is building a house?'
\z





\ea%x1166
\label{ex:7:x1166}
\gll Muuka  nain  \textstyleEmphasizedVernacularWords{maa}  \textstyleEmphasizedVernacularWords{mauwa}  enim-i-non? \\
son  that1  thing  what  eat-Np-\textsc{fu}.3s  \\
\glt `What will the son eat?'
\z





\ea%x1164
\label{ex:7:x1164}
\gll No  muuka  wiipa  \textstyleEmphasizedVernacularWords{kamin}  (nefar  ik-ua)? \\
2s.\textsc{unm}  son  daughter  how.many  (2s.\textsc{dat}  be-\textsc{pa}.3s)\\
\glt `How many children (lit. son daughter) do you have?'
\z





\ea%x1165
\label{ex:7:x1165}
\gll Mua  napuma  \textstyleEmphasizedVernacularWords{moram}  owowa  p-ikiw-i-yan? \\
man  sick/body  why  village  Bpx-go-Np-\textsc{fu}.1p  \\
\glt `Why should we take the body to the village?'
\z





\ea%x1378
\label{ex:7:x1378}
\gll Mukuna  aw-o-k  nain  \textstyleEmphasizedVernacularWords{kamenap}  umuk-i-yen? \\
fire  burn-\textsc{pa}-3s  that1  how  extinguish-Np-\textsc{fu}.1p      \\
\glt `How could we extinguish the fire that was burning?'
\z





\ea%x1173
\label{ex:7:x1173}
\gll Maa  nain  epa  \textstyleEmphasizedVernacularWords{kain=pa}  imenar-i-non? \\
thing  that1  place  which=\textsc{loc}  appear-Np-\textsc{fu}.3s      \\
\glt `Where (lit: in which place) will that thing appear?'
\z





\ea%x1174
\label{ex:7:x1174}
\gll Wi  \textstyleEmphasizedVernacularWords{kaakew}  mua=ke  uf-e-mik? \\
3p.\textsc{unm}  which.village  man=\textsc{cf}  dance-\textsc{pa}-1/3p      \\
\glt `The men of which village danced?'
\z





The kind of ambiguity between a subject and an object that Usan has, which arises from the fronting of a topicalized element\footnote{A \textit{theme} in my terminology here.} \citep[294]{Reesink1987}, is not possible in Mauwake. This is because the question words take the contrastive focus marker \nobreakdash-\textstyleStyleVernacularWordsItalic{ke}  when functioning as a subject. Because of elision, and the merging of the contrastive focus marker with the question word, the word for `who' in Mauwake actually has a contrasted/nominative \REF{ex:7:x1167}, \REF{ex:7:x1171} and an accusative form \REF{ex:7:x1170}. The object is fronted in \REF{ex:7:x1171} as a theme; in \REF{ex:7:x1172} the object is not fronted.

\ea%x1170
\label{ex:7:x1170}
\gll [Mua  nain]\textsubscript{S}  [\textstyleEmphasizedVernacularWords{naarew}]\textsubscript{O}  aruf-a-k? \\
man  that1  who(\textsc{acc})  hit-\textsc{pa}-3s      \\
\glt `Who did that man hit?'
\z





\ea%x1171
\label{ex:7:x1171}
\gll [Mua  nain]\textsubscript{O}  [\textstyleEmphasizedVernacularWords{naareke}]\textsubscript{S}  aruf-a-k? \\
man  that1  who.\textsc{cf}  hit-\textsc{pa}-3s      \\
\glt `Who hit that man?'
\z





\ea%x1172
\label{ex:7:x1172}
\gll [\textstyleEmphasizedVernacularWords{(Mua)}  \textstyleEmphasizedVernacularWords{naareke}]\textsubscript{S}  [mua  nain]\textsubscript{O}  aruf-a-k? \\
(man)  who.\textsc{cf}  man  that1  hit-\textsc{pa}-3s      \\
\glt `Who hit that man?'
\z





It is most common to have the question in a main clause, but medial clauses also easily allow non-polar questions. The scope of the question word only extends to the clause which contains it. In \REF{ex:7:x1168} the fact that the people ran away is not questioned.

\ea%x1168
\label{ex:7:x1168}
\gll \textstyleEmphasizedVernacularWords{Mua}  \textstyleEmphasizedVernacularWords{naareke}  \textstyleEmphasizedVernacularWords{wia}  \textstyleEmphasizedVernacularWords{aruf-eya}  baurar-e-mik? \\
man  who.\textsc{cf}  3p.\textsc{acc}  hit-2/3s.\textsc{ds}  run.away-\textsc{pa}-1/3p      \\
\glt `Who hit them (so that) they ran away?'
\z





A constituent in a complement clause (with a nominalized verb) can be questioned, but not in a relative clause. 

\ea%x1175
\label{ex:7:x1175}
\gll \textstyleEmphasizedVernacularWords{Ama}  \textstyleEmphasizedVernacularWords{kamin}  \textstyleEmphasizedVernacularWords{ikiw-owa}  ma-e-mik? \\
sun/time  how.much  go-\textsc{nmz}  say-\textsc{pa}-1/3p      \\
\glt `At what time did they say to go?'
\z





\ea%x1176
\label{ex:7:x1176}
\gll \textstyleEmphasizedVernacularWords{*Wi}  \textstyleEmphasizedVernacularWords{iikamin}  \textstyleEmphasizedVernacularWords{ekap-e-mik}  \textstyleEmphasizedVernacularWords{nain}  wia  uruf-a-n? \\
3p.\textsc{unm}  when  come-\textsc{pa}-1/3p  that1  3p.\textsc{acc}  see-\textsc{pa}-2s \\
\glt
\z



Multiple constituents in the same clause can be questioned with a question word. This is not common, but the following elicited sentences are considered completely natural. 

\ea%x1177
\label{ex:7:x1177}
\gll Emeria  \textstyleEmphasizedVernacularWords{naareke}  ama  \textstyleEmphasizedVernacularWords{kamin=pa}  ekap-o-k? \\
woman  who.\textsc{cf}  sun/time  how.much=\textsc{loc}  come-\textsc{pa}-3s      \\
\glt `Who (woman) came at what time?'
\z





\ea%x1178
\label{ex:7:x1178}
\gll Mua  \textstyleEmphasizedVernacularWords{kain=ke}  emeria  \textstyleEmphasizedVernacularWords{kain}  aaw-o-k? \\
man  which=\textsc{cf}  woman  which  take-\textsc{pa}-3s      \\
\glt `Which man took/married which woman?'
\z





When there is a lot of hesitation in the question, the question clitic -\textstyleStyleVernacularWordsItalic{i}, which normally marks a polar question, is added to the end of the question. This is the same form that the echo questions have (7.2.3). 

\ea%x1187
\label{ex:7:x1187}
\gll Auwa  efa  amukar-e-k  nain  yo  \textstyleEmphasizedVernacularWords{kamenap} ar-i-nen=\textstyleEmphasizedVernacularWords{i}?\\
father  1s.\textsc{acc}  scold-\textsc{pa}-3s  that  1s.\textsc{unm}  how   become-Np-\textsc{fu}.1s=\textsc{qm}   \\
\glt `(I wonder) what will happen to me because father scolded me?'	
\z


%%%% 2014-08-11-14-36






\subsection{Polar questions} \label{sec:7.2.2}
%\hypertarget{RefHeading22761935131865}
{}
Polar questions\footnote{Also called nexus questions, or yes-no questions.} expect either confirmation or negation of the questioned proposition. According to \citet[63]{Wurm1982}, a polar question in \textstyleAcronymallcaps{TNG} languages is often marked by an affix which is part of the verb complex. In Mauwake it is coded by the question clitic -\textstyleStyleVernacularWordsItalic{i}  (\sectref{sec:3.12.8}) and slightly rising intonation (\sectref{sec:2.1.3.2}), both occurring sentence-finally.  Because Mauwake is an \textstyleAcronymallcaps{SOV} language, the clitic most often attaches itself to a verb \REF{ex:7:x1179}, but it can attach to another word class as well, when there is no final verb:

\ea%x1179
\label{ex:7:x1179}
\gll Ni  nain  me=ko  uruf-a-man\textstyleEmphasizedVernacularWords{=i}? \\
2p.\textsc{unm}  that1  not=\textsc{nf}  see-\textsc{pa}-2p=\textsc{qm}   \\
\glt `Didn't you see that?'
\z





\ea%x1180
\label{ex:7:x1180}
\gll Nos=\textstyleEmphasizedVernacularWords{i}? \\
2s.\textsc{fc}=\textsc{qm}      \\
\glt `You?'
\z





\ea%x1181
\label{ex:7:x1181}
\gll Maa  nain  eliwa=\textstyleEmphasizedVernacularWords{ki}? \\
thing  that1  good=\textsc{cf}.\textsc{qm}      \\
\glt `Is that thing good?'
\z





When the polar question is in the negative, a one-word answer may be ambiguous. Traditionally the answer either affirmed or negated the affirmative or negative \textstyleEmphasizedWords{\textsc{polarity}} of the question, but because of the influence of Tok Pisin and English, Mauwake is changing so that the answer tends to either affirm or negate the \textstyleEmphasizedWords{\textsc{verb}} (\sectref{sec:6.2.4}).

Alternative questions can be closed or open.\footnote{\citet{Haspelmath2007} calls only the former an alternative (or disjunctive) question, and the latter a question with standard disjunction.} The former give two, or sometimes more, alternatives, one of which has to be chosen; the latter also allow the possibility that none of the alternatives is chosen. The two types differ in Mauwake as to what the last alternative is like.

The non-final alternatives in a closed question take the question marker -\textstyleStyleVernacularWordsItalic{i}. The final alternative, usually preceded by the disjunctive coordinator \textstyleStyleVernacularWordsItalic{e} `or' (\sectref{sec:3.11.2}), may be just a negation particle \textstyleStyleVernacularWordsItalic{weetak} or \textstyleStyleVernacularWordsItalic{wia} \REF{ex:7:x1182}, a full statement \REF{ex:7:x1183}, or an elliptical clause with only the questioned item \REF{ex:7:x1184}.

\ea%x1182
\label{ex:7:x1182}
\gll Yo  emeria=ko  efar  uruf-a-man=\textstyleEmphasizedVernacularWords{i}  \textstyleEmphasizedVernacularWords{e}  weetak? \\
1s.\textsc{unm}  woman=\textsc{nf}  1s.\textsc{dat}  see-\textsc{pa}-2p=\textsc{qm}  or  no      \\
\glt `Did you see my wife or not?'
\z





\ea%x1183
\label{ex:7:x1183}
\gll Nain  kema  suuw-i-man=\textstyleEmphasizedVernacularWords{i}  \textstyleEmphasizedVernacularWords{e}  kema  irin-ar-e-man? \\
that1  liver  push-Np-\textsc{pr}.2p=\textsc{qm}  or  liver  stuck-\textsc{inch}-\textsc{pa}-2p      \\
\glt `Do you remember (lit: push the liver) that, or have you forgotten (lit: liver is stuck) it?'
\z




\ea%x1184
\label{ex:7:x1184}
\gll No  Matukar  ikiw-i-nan=\textstyleEmphasizedVernacularWords{i}  Dylup=\textstyleEmphasizedVernacularWords{i}  \textstyleEmphasizedVernacularWords{e}  Sarang? \\
2s.\textsc{unm}  Matukar  go-Np-\textsc{fu}.2s=\textsc{qm}  Dylup=\textsc{qm}  or  Sarang      \\
\glt `Will you go to Matukar, Dylup, or Sarang?'
\z


When the alternative question is open, the question marker -\textstyleStyleVernacularWordsItalic{i}  marks not only the non-final alternatives but also the final one. 

\ea%x1185
\label{ex:7:x1185}
\gll Matukar  ikiw-i-nan=\textstyleEmphasizedVernacularWords{i}  \textstyleEmphasizedVernacularWords{e}  Dylup  ikiw-i-nan=\textstyleEmphasizedVernacularWords{i}? \\
Matukar  go-Np-\textsc{fu}.2s=\textsc{qm}  or  Dylup  go-Np-\textsc{fu}.2s=\textsc{qm}      \\
\glt `Will you go to Matukar or Dylup (or perhaps neither)?'
\z





\ea%x1186
\label{ex:7:x1186}
\gll Mukuna=ko  wu-a-man=\textstyleEmphasizedVernacularWords{i}  \textstyleEmphasizedVernacularWords{e}  mua=ko  wia  uruf-a-man=\textstyleEmphasizedVernacularWords{i}? \\
fire=\textsc{nf}  put-\textsc{pa}-2p=\textsc{qm}  or  man=\textsc{nf}  3.\textsc{acc}  see-\textsc{pa}-2p=\textsc{qm}      \\
\glt `Did you light a fire or did you feel (that there was) a man?'
\z





An alternative question is left open also when the last alternative is replaced with the question word \textstyleStyleVernacularWordsItalic{kamenion} '(or) what?'/ `(or) how is it?':

\ea%x1350
\label{ex:7:x1350}
\gll Maa  en-owa=ko  p-ekap-e-mik=\textstyleEmphasizedVernacularWords{i}  \textstyleEmphasizedVernacularWords{kamenion}? \\
food  eat-\textsc{nmz}=\textsc{nf}  \textsc{bp}x-come-\textsc{pa}-1/3p=\textsc{qm}  or.what      \\
\glt `Did they bring food, or what?'
\z





\ea%x1351
\label{ex:7:x1351}
\gll Beel(a)-al-i-non=\textstyleEmphasizedVernacularWords{i}  \textstyleEmphasizedVernacularWords{kamenion},  naap  uruf-am-ik-ua. \\
rotten-\textsc{inch}-Np-\textsc{fu}.3s=\textsc{qm}  or.what  thus  see-\textsc{ss}.\textsc{sim}-be-\textsc{pa}.3s      \\
\glt `He was watching whether it would rot or what would happen.'
\z





Leading questions are another subtype of polar questions. The person asking wants to guide the answer in a certain direction. This is done in Mauwake by adding the epistemic modal adverb clitic -\textstyleStyleVernacularWordsItalic{yon} `perhaps' to the predicate of the question clause. The slightly rising intonation in the question distinguishes it from a statement. 

\ea%x1349
\label{ex:7:x1349}
\gll Me  ikiw-o-k=\textstyleEmphasizedVernacularWords{yon}? \\
not  go-\textsc{pa}-3s-perhaps      \\
\glt `He didn't go, did he?'
\z




\subsection{Echo questions} \label{sec:7.2.3}
%\hypertarget{RefHeading22781935131865}
{}
Echo questions are used when an original statement or question is questioned, either because it was not properly heard in the first place or because the addressee has some doubts about it. Structurally all echo questions are polar questions.

Echo question of a statement is like a normal polar question, except that the questioned element receives an extra stess.

\ea%x1189
\label{ex:7:x1189}
\gll A:Paapa  Goroka  ikiw-i-non.  -  B:  Gor\'oka  ikiw-i-non=i? \\
A:elder.sibling  Goroka  go-Np-\textsc{fu}.3p  -  B:  Goroka  go-Np-\textsc{fu}.3p=\textsc{qm}      \\
\glt `A: Big brother is going to Goroka. B: Is he going to \textstyleEmphasizedWords{\textsc{Goroka}}?'
\z





When the validity of a non-polar question \REF{ex:7:x1190} is questioned, the question clitic is attached directly to the end of the question already containing a question word \REF{ex:7:x1191}. 

\ea%x1190
\label{ex:7:x1190}
\gll A:  Mua  naarew  wia  maak-e-k?  \\
A:  man  who  3.\textsc{acc}  tell-\textsc{pa}-3s      \\
\glt `Who did he tell?'
\z





\ea%x1191
\label{ex:7:x1191}
\gll B:  Mua  naarew  wia  maak-e-k=\textstyleEmphasizedVernacularWords{i}? \\
B:  man  who  3.\textsc{acc}  tell-\textsc{pa}-3s=\textsc{qm}      \\
\glt `Who did he tell???'
\z
But if the addressee wants to check if (s)he heard correctly, the echoed question is made into a complement of a sentence-final utterance verb, which gets a question clitic attached to it \REF{ex:7:x1192}.

\ea%x1192
\label{ex:7:x1192}
\gll B:  Mua  naarew  wia  maak-e-k  \textstyleEmphasizedVernacularWords{na-i-n=i}? \\
B:  man  who  3.\textsc{acc}  tell-\textsc{pa}-3s  say-Np-\textsc{pr}.2s=\textsc{qm}      \\
\glt `Are you asking who he told?'
\z





Since polar questions already have a clause-final question clitic, an echo question cannot be formed by adding the same clitic a second time. Instead, the original question is made into a complement of the utterance verb \textstyleStyleVernacularWordsItalic{ma}- `say' or \textstyleStyleVernacularWordsItalic{na}- `say/think'. 

\ea%x1193
\label{ex:7:x1193}
\gll A:  Nain  eliwa=ki?  B:  Nain  eliwa-ki  \textstyleEmphasizedVernacularWords{ma-e-n=i}? \\
A:  that1  good=\textsc{cf}.\textsc{qm}  B:  that1  good=\textsc{cf}.\textsc{qm}  say-\textsc{pa}-2s=\textsc{qm}      \\
\glt `A: Is that good?  B: Did you ask if that is good?'
\z





\subsection{Confirmation questions}
%\hypertarget{RefHeading22801935131865}
{}
Confirmation questions are mainly used in argumentation. The question word \textstyleStyleVernacularWordsItalic{naap}\textstyleStyleVernacularWordsItalic{-i}  `(is it) like that?' is tagged to a statement, which may be preceded by another question. 

\ea%x1188
\label{ex:7:x1188}
\gll Ni  kema  maneka  naap  efa  wu-i-man=i, yo  eliw  nia  saliw-i-nen,  \textstyleEmphasizedVernacularWords{naap=i}? \\
2p.\textsc{unm}  liver  big  thus  1s.\textsc{acc}  put-Np-\textsc{pr}.2p=\textsc{qm}  1s.\textsc{unm}  well  2p.\textsc{acc}  heal-Np-\textsc{fu}.1s  thus=\textsc{qm}    \\
\glt `Do you believe about me that I can heal you, is that so?'
\z









\subsection{Indirect questions}
%\hypertarget{RefHeading22821935131865}
{}
Indirect questions are a subgroup of complement clauses and are discussed under \textstyleEmphasizedWords{\textsc{Indirect speech}} in  \sectref{sec:8.3.2.1.2}.

\ea%x1203
\label{ex:7:x1203}
\gll [Yo  maa  mauwa  uruf-a-m]  efa  na-e-k. \\
1s.\textsc{unm}  thing  what  see-\textsc{pa}-1s  1s.\textsc{acc}  say-\textsc{pa}-3s      \\
\glt `He asked me what I saw.'
\z





\ea%x1204
\label{ex:7:x1204}
\gll [Kamin  wu-a-mik(-yon)],  yo  me  wiar amis-ar-e-m. \\
how.much  put-\textsc{pa}-1/3p-perhaps  1s.\textsc{unm}  not  3.\textsc{dat} knowledge-\textsc{inch}-\textsc{pa}-1s     \\
\glt `I don't know how much they put.'
\z




\subsection{Rhetorical questions}
%\hypertarget{RefHeading22841935131865}
{}
Traditionally the Mauwake speakers lived in a society where everyone more or less knew everybody's business and there was not much need for eliciting information by asking questions. Consequently, many questions in normal speech are rhetorical in nature. The question form may be used to emphasise the opposite of what is said, or sometimes just to prompt the addressee to think more clearly, but very often rhetorical questions have an element of reproach or assigning blame as well.

\ea%x1205
\label{ex:7:x1205}
\gll Maamuma  kaaneke  ika-eya  ni-i-yan? \\
 money  where.\textsc{cf}  be-2/3s.\textsc{ds}  give.you-Np-\textsc{fu}.1p     \\
\glt `Where would we have that kind of money to give you? (=We do not have money to give you.)'
\z





\ea%x1206
\label{ex:7:x1206}
\gll Yo  anane  niam=iya  ika-i-nen=i? \\
1s.\textsc{unm}  always  2p.\textsc{refl}=\textsc{com}  be-Np-\textsc{fu}.1s=\textsc{qm}      \\
\glt `Will I be with you forever? (= I will not.)'
\z





\ea%x1202
\label{ex:7:x1202}
\gll No  moram  naap  om-em-ika-i-n? \\
2s.\textsc{unm}  why  thus  cry-\textsc{ss}.\textsc{sim}-be-Np-\textsc{pr}.2s      \\
\glt `Why are you crying like that? (=You should not cry like that.)'
\z





\ea%x1207
\label{ex:7:x1207}
\gll Mua  naareke  nia  maak-eya  ekap-e-man? \\
man  who.\textsc{cf}  2p.\textsc{acc}  say-2/3p.\textsc{ds}  come-\textsc{pa}-2p      \\
\glt `Who told you to come? (=You shouldn't have come)'
\z





Implied reproach or accusation is particularly common with  questions including the word \textstyleStyleVernacularWordsItalic{moram} `why?', but it is not limited to them. Especially accusations of theft are couched in neutral-looking questions \REF{ex:7:x1208}.

\ea%x1208
\label{ex:7:x1208}
\gll Aa  muuka,  no  moram  naap  yia  on-a-n? \\
oh  son  2s.\textsc{unm}  why  thus  1s.\textsc{acc}  do-\textsc{pa}-2s      \\
\glt `Oh son, why did you do this to us?'
\z





\ea%x1209
\label{ex:7:x1209}
\gll Yo  seewa  gelemuta  uruma  or-o-k  nain  uruf-a-man=i? \\
 1s.\textsc{unm}  rat  small  valley  descend-\textsc{pa}-3s  that1  see-\textsc{pa}-2p=\textsc{qm}     \\
\glt `Have you seen my ``little rat'' (pig) that went down to the valley? (implying: I have no doubt that you have stolen my pig.)'
\z




Because questions are so easily understood as reproaches or accusations, real questions are often preceded by a preamble to prevent this interpretation.

\ea%x1210
\label{ex:7:x1210}
\gll [Ama  arow=pa  mauw-owa  weeser-eya]  maa  mauwa  on-a-man? \\
 sun  three=\textsc{loc}  work-\textsc{nmz}  finish-2/3s.\textsc{ds}  thing  what  do-\textsc{pa}-2p     \\
\glt `After your work finished at three, what did you do?'
\z





\ea%x1211
\label{ex:7:x1211}
\gll [Yo  oram  nefa  nokar-i-yem],  soomia=ko  efar uruf-a-n=i? \\
1s.\textsc{unm}  just  2s.\textsc{acc}  ask-Np-\textsc{pr}.1s  spoon=\textsc{nf}  1s.\textsc{dat}  see-\textsc{pa}-2s=\textsc{qm}    \\
\glt `I'm just asking: have you seen my spoon?'
\z









\ea%x1355
\label{ex:7:x1355}
\gll Anane  maneka  ewur  me  urup-i-n  nain  moram? \\
 always  big  quickly  not  ascend-Np-\textsc{pr}.2s  that1  why     \\
\glt `What is the reason why you never come up quickly?'
\z





\subsection{Answers to questions} \label{sec:7.2.7}
%\hypertarget{RefHeading22861935131865}
{}
Apart from rhetorical questions, a verbal answer is often expected. An affirmative answer to a polar question \REF{ex:7:x1216} may be just an affirmative interjection \REF{ex:7:x1217} or the verb from the question by itself or preceded by the interjection \REF{ex:7:x1218}. A negative answer must have at least one negator, whether only a negative interjection (\sectref{sec:6.2.3}), or any of the other negators, or both \REF{ex:7:x1220}. Less commonly the answer may also be a full statement with or without a preceding affirmation \REF{ex:7:x1219}  or negation.

\ea%x1216
\label{ex:7:x1216}
\gll No  uurika  owow  maneka  ikiw-i-nan=i? \\
2s.\textsc{unm}  tomorrow  village  big  go-Np-\textsc{fu}.2s=\textsc{qm}      \\
\glt `Are you going to town tomorrow?'
\z




\ea%x1217
\label{ex:7:x1217}
\gll Ae/Oo. \\
 yes     \\
\glt `Yes.'
\z





\ea%x1218
\label{ex:7:x1218}
\gll (Ae,)  ikiw-i-nen. \\
yes  go-Np-\textsc{fu}.1s      \\
\glt `(Yes,) I am going.'
\z





\ea%x1220
\label{ex:7:x1220}
\gll (Weetak,)  me  ikiw-i-nen. \\
no  not  go-Np-\textsc{fu}.1s      \\
\glt `(No,) I am not going.'
\z





\ea%x1219
\label{ex:7:x1219}
\gll (Ae,)  yo  uurika  owow  maneka  ikiw-i-nen. \\
 yes  1s.\textsc{unm}  tomorrow  village  big  go-Np-\textsc{fu}.1s     \\
\glt `(Yes,) I'll go to town tomorrow.'
\z





The reply to a non-polar question most typically includes an answer to the questioned item and often the verb of the original question too.

\ea%x1221
\label{ex:7:x1221}
\gll Maa  sira  kamenap  nain  en-em-ik-e-man? \\
thing/food  kind  how  that1  eat-\textsc{ss}.\textsc{sim}-be-\textsc{pa}-2p      \\
\glt `What kind of food did you eat?'
\z





\ea%x1222
\label{ex:7:x1222}
\gll Wi  mia  kia  en-owa  nain  (en-em-ik-e-mik). \\
3p.\textsc{unm}  body  white  eat-\textsc{nmz}  that1  (eat-\textsc{ss}.\textsc{sim}-be-\textsc{pa}-1/3p)      \\
\glt `(We ate) the white people's food.'
\z





If the speaker wants to negate the presupposition in the question, (s)he begins with a negator, and then goes on to answer the question itself \REF{ex:7:x1224}.

\ea%x1223
\label{ex:7:x1223}
\gll Neremena  kamenap  nefa  on-a-k? \\
2s/p.nephew  how  2s.\textsc{acc}  do-\textsc{pa}-3s      \\
\glt `What did your nephew do to you?'
\z





\ea%x1224
\label{ex:7:x1224}
\gll \textstyleEmphasizedVernacularWords{Weetak},  yo  mauw-a-m  ne  o  me  efa  uruf-a-k.\\
no  1s.\textsc{unm}  work-\textsc{pa}-1s  \textsc{add}  3s.\textsc{unm}  not  1s.\textsc{acc}  see-\textsc{pa}-3s     \\
\glt `I worked but he did not even look at me.' (Implying: Your presupposition is wrong; he did not do anything indecent to me.)
\z

 
If the question or statement itself is negative, a one-word answer is ambiguous in present-day usage, and a full clause is needed to disambiguate it. Traditionally an answer to a question affirmed or negated the affirmative or negative \textstyleEmphasizedWords{\textsc{polarity}} of the question or statement:

\ea%x1151
\label{ex:7:x1151}
\gll O  aakun-owa  marew=yon.  -\textstyleEmphasizedVernacularWords{Wia},  aakun-owa  wiar  ik-ua.\\
3s.\textsc{unm}  talk-\textsc{nmz}  no(ne)-perhaps  -no  talk-\textsc{nmz}  3.\textsc{dat} be-\textsc{pa}.3s     \\
\glt `Perhaps he doesn't have anything to say. --No, he \textstyleEmphasizedWords{\textsc{does}} have something to say.'
\z









\ea%x1117
\label{ex:7:x1117}
\gll Auwa  me  ekap-o-k=i?  -\textstyleEmphasizedVernacularWords{Weetak}  (ekap-o-k). \\
 1s/p.father  not  come-\textsc{pa}-3s=\textsc{qm}  -no  (come-\textsc{pa}-3s)     \\
\glt `Didn't father come? --Yes (he \textstyleEmphasizedWords{\textsc{did}}).'
\z
But Mauwake is changing to become more like English\footnote{A similar change is taking place in Tok Pisin, and it is likely that this is causing the development in Mauwake too.} in that the negative answer stands for a negative statement regardless of the polarity of the question or statement that it is a reply to: 

\ea%x1118
\label{ex:7:x1118}
\gll Auwa  me  ekap-o-k=i?  -\textstyleEmphasizedVernacularWords{Weetak}  (me  ekap-o-k). \\
 1s/p.father  not  come-\textsc{pa}-3s=\textsc{qm}  -no  (not  come-\textsc{pa}-3s)     \\
\glt `Didn't father come? --No (he didn't).'
\z





\section{Commands} \label{sec:7.3}
%\hypertarget{RefHeading22881935131865}
{}
The simple imperative is the default way of expressing a command in Mauwake. It shows in the verb inflection (\sectref{sec:3.8.3.3.2}). In a prohibition the verbal negator \textstyleStyleVernacularWordsItalic{me} `not' precedes the simple imperative \REF{ex:7:x1075}.

\ea%x1072
\label{ex:7:x1072}
\gll Ni  Medebur  \textstyleEmphasizedVernacularWords{karu-eka},  \textstyleEmphasizedVernacularWords{baurar-eka}. \\
2p.\textsc{unm}  Medebur  run-\textsc{imp}.2p  flee-\textsc{imp}.2p      \\
\glt `Run(pl.) to Medebur, flee.'
\z





\ea%x1075
\label{ex:7:x1075}
\gll Momora,  no  naap  \textstyleEmphasizedVernacularWords{me} \textstyleEmphasizedVernacularWords{ma-e}. \\
fool  2s.\textsc{unm}  thus  not  say-\textsc{imp}.2s      \\
\glt `Fool, don't say like that.'
\z





The simple imperative can be strengthened with the intensity adverb \textstyleStyleVernacularWordsItalic{akena} `very, truly' following the verb.

\ea%x1073
\label{ex:7:x1073}
\gll Ni  sira  samora  \textstyleEmphasizedVernacularWords{piipu-eka  akena}.\\ 
2p.\textsc{unm}  habit  bad  leave-\textsc{imp}.2p  truly\\
\glt `Really get rid of your bad habits.'
\z





Another way to intensify it is with the clause-final interjection \textstyleStyleVernacularWordsItalic{nom} `\textstyleEmphasizedWords{\textsc{please}}!', which is only used when a person has already been told to do something at least once and has not complied.

\ea%x1074
\label{ex:7:x1074}
\gll \textstyleEmphasizedVernacularWords{Pootin-e},  \textstyleEmphasizedVernacularWords{nom}! \\
 stop.crying-\textsc{imp}.2s  please     \\
\glt `Stop crying, \textstyleEmphasizedWords{\textsc{please}}!'
\z





The imperative marking on verbs shows only in the finite forms. When a command or request is in a medial clause, and the final clause verb is in the indicative mood and future tense, there is nothing in the medial verb to indicate the mood. 

\ea%x1076
\label{ex:7:x1076}
\gll No  opaimika  pon  aaw-o-n  nain  \textstyleEmphasizedVernacularWords{ma-eya} i  miim-i-yen.\\
2s.\textsc{unm}  talk  turtle  get-\textsc{pa}-3s  that1  tell-2/3s.\textsc{ds} 1p.\textsc{unm}  hear-Np-\textsc{fu}.1p     \\
\glt `Tell us about your catching a turtle, and we'll listen.' (Or: `You will tell us about your catching a turtle and we'll listen.')
\z


This type of clause combination has given rise to a softer, less direct command, which is given with a medial different-subject form of a verb; the final clause is left out altogether.\footnote{This fairly common usage of a medial verb form in Papuan languages is probably the origin of the use of \textit{pastaim} `first' in Tok Pisin commands, e.g. \textit{Kam pastaim} `Come!'}  This form is particularly common when commands are given to children. 

\ea%x1084
\label{ex:7:x1084}
\gll P-ekap-\textstyleEmphasizedVernacularWords{eya}! \\
Bpx-come-2/3s.\textsc{ds}      \\
\glt `Bring it!'
\z

The imperative of the final clause may have an influence on the medial clause(s) so that they, too, are interpreted as belonging within the scope of the command. This happens very easily with same-subject medial verbs \REF{ex:7:x1082}; it is also possible but much less likely when the subject changes \REF{ex:7:x1364}\footnote{This example may also be interpreted to have two commands, a ``soft'' one, expressed with a medial verb, and a regular one.}. \REF{ex:7:x1846} is ambiguous: in the situation where was said, the medial clause was not in the scope of the final clause imperative; in some other situation it could be. When the medial verb has a first person form, imperative interpretation is not possible \REF{ex:7:x1083}. 

\ea%x1082
\label{ex:7:x1082}
\gll Emeria  manina  \textstyleEmphasizedVernacularWords{ikiw-ep}  en-owa  \textstyleEmphasizedVernacularWords{nop-ap}  \textstyleEmphasizedVernacularWords{or-eka}. \\
woman  garden  go-\textsc{ss}.\textsc{seq}  eat-\textsc{nmz}  search-\textsc{ss}.\textsc{seq}  descend-\textsc{imp}.2p      \\
\glt `Women, go to the garden, look for food and come down.'
\z





\ea%x1364
\label{ex:7:x1364}
\gll Mua  emeria  wia  \textstyleEmphasizedVernacularWords{maak-eya}  me  efa  \textstyleEmphasizedVernacularWords{enim-uk}. \\
man  woman  3p.\textsc{acc}  tell-2/3s.\textsc{ds}  not  2s.\textsc{acc}  eat-\textsc{imp}.3p      \\
\glt `Tell the people and let them not eat me.'
\z





\ea%x1846
\label{ex:7:x1846}
\gll Feeke  wiar  \textstyleEmphasizedVernacularWords{ik-ok}  kiiriw  mua  wiar  \textstyleEmphasizedVernacularWords{urup-e}. \\
here.\textsc{cf}  3.\textsc{dat}  be-\textsc{ss}  again  man  3.\textsc{dat}  ascend-\textsc{imp}.2s      \\
\glt `Having been here with him (=your brother), go up to your husband again.'
\z





\ea%x1083
\label{ex:7:x1083}
\gll I  or-op  ununa  \textstyleEmphasizedVernacularWords{anum-amkun}  \textstyleEmphasizedVernacularWords{ma-eka},  ``{\dots''} \\
1p.\textsc{unm}  descend  slit.gong  beat-1s/p.\textsc{ds}  say-\textsc{imp}.2p      \\
\glt `When we go down and beat the slit gong, say, ``{\dots}'' '
\z



A special feature in Mauwake commands is that they occur with a pronominal  subject more often than statements do (\sectref{sec:3.5.2.1}, 3.5.11). 

Although a command is usually directed towards one or more people in the second person, it can also be directed towards self as part of a group of two \REF{ex:7:x1157} or more \REF{ex:7:x1158}, or towards a third person in singular \REF{ex:7:x1159} or plural \REF{ex:7:x1160}. 

\ea%x1157
\label{ex:7:x1157}
\gll Aria,  i  owowa=ko  \textstyleEmphasizedVernacularWords{or-u}. \\
 alright  1p.\textsc{unm}  village=\textsc{nf}  descend-\textsc{imp}.1d     \\
\glt `Alright, let's go down to the village.'
\z





\ea%x1158
\label{ex:7:x1158}
\gll Ikiw-ep=ko  wia  \textstyleEmphasizedVernacularWords{uruf-ikua}. \\
  go-\textsc{ss}.\textsc{seq}=\textsc{nf}  3p.\textsc{acc}  see-\textsc{imp}.1p    \\
\glt `Let's go and see them.'
\z





\ea%x1159
\label{ex:7:x1159}
\gll Womokowa  me  wia  \textstyleEmphasizedVernacularWords{maak-inok}. \\
 3s/p.brother  not  3p.\textsc{acc}  tell-\textsc{imp}.3s     \\
\glt `Let her not talk to her brothers.'
\z





\ea%x1160
\label{ex:7:x1160}
\gll Ona  mua  owawiya  ek-ap  uruf-am-ik-ok  \textstyleEmphasizedVernacularWords{ep-am-ika-uk.}\\
3s.\textsc{gen}  man  with  go-\textsc{ss}.\textsc{seq}  see-\textsc{ss}.\textsc{sim}-be-\textsc{ss}  come-\textsc{ss}.\textsc{sim}-be-\textsc{imp}.3p    \\
\glt `Let her with her husband keep going, seeing him and coming back.'
\z









Imperatives cannot have tense distinctions, but aspectual distinctions are possible. The continuous aspect form is used for habitual in \REF{ex:7:x1896} and for continuous aspect in \REF{ex:7:x1162}. Completive aspect is used in \REF{ex:7:x1161} and stative in \REF{ex:7:x1162}.

\ea%x1896
\label{ex:7:x1896}
\gll Sira naap \textstyleEmphasizedVernacularWords{on-am-ik-eka}. \\
custom thus do-\textsc{ss}.\textsc{sim}-be-\textsc{imp}.2p      \\
\glt `freetranslation'
\z





\ea%x1161
\label{ex:7:x1161}
\gll Aakisa  naap  \textstyleEmphasizedVernacularWords{on-ap-pu-e}. \\
now  thus  do-\textsc{ss}.\textsc{seq}-\textsc{cmpl}-\textsc{imp}.2s      \\
\glt `Now do that.'
\z





\ea%x1162
\label{ex:7:x1162}
\gll No  me  mokoka  \textstyleEmphasizedVernacularWords{opar-ep-ik-e}. \\
2s.\textsc{unm}  not  eye  close-\textsc{ss}.\textsc{seq}-be-\textsc{imp}.2s      \\
\glt `Don't have/keep your eyes closed.'
\z





The second person future tense form is also used for a command, but this is not very common. It is used in a specific situation, not for giving generic commands or rules. The sentence \REF{ex:7:x1080} was said to a person who was suspected of lying, and in \REF{ex:7:x1081} parents give instructions to their daughter how to mourn.

\ea%x1080
\label{ex:7:x1080}
\gll No  \textstyleEmphasizedVernacularWords{me}  \textstyleEmphasizedVernacularWords{sail-i-nan}! \\
 2s.\textsc{unm}  not  lie-Np-\textsc{fu}.2s     \\
\glt `Don't lie!'
\z





\ea%x1081
\label{ex:7:x1081}
\gll Naap  ma-emi  \textstyleEmphasizedVernacularWords{om-em-ika-i-nan}  na. \\
thus  say-\textsc{ss}.\textsc{sim}  cry-\textsc{ss}.\textsc{sim}-be-Np-\textsc{fu}.2s  \textsc{intj}      \\
\glt `Say like that and wail.'
\z






%%% Local Variables: 
%%% mode: latex
%%% TeX-master: "../main"
%%% End: