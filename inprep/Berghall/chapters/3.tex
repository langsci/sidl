%3

\chapter{Morphology}\label{sec:3}
%\hypertarget{RefHeading19221935131865}
{}
\section{Introduction}\label{sec:3:1}
%\hypertarget{RefHeading19241935131865}
{}
A grammatical word in Mauwake is defined on the basis of the following main criteria quoted from \citet[12--14]{Dixon2010b}:

A grammatical word
\begin{itemize}
\item has as its base one or more lexical roots to which morphological processes apply;
\item has a conventionalized coherence and meaning.
\end{itemize}
When a grammatical word involves compounding or affixation, its component grammatical elements 

\begin{itemize}
\item always occur together;
\item generally occur in a fixed order
\end{itemize}

The following supplementary criteria also apply. A word only allows one inflectional affix of any one type (ibid. 15). Also in derivation recursiveness is blocked except in the case of causatives (ibid. 16--17). Even here the recursion is more ostensible than real, as it does not add another argument into the clause (\sectref{sec:3.8.2.3.1}). Person/number suffixes act as word-final boundary markers in finite verbs (ibid. 17). Many words, especially those belonging to the major word classes, ``may constitute a complete utterance'' (ibid. 19) by themselves. 

The boundaries of the grammatical and phonological words coincide, except in the case of clitics. Grammatically a clitic is a word but phonologically it is bound to the preceding word.

The classes of nouns, adjectives, personal pronouns, quantifiers, verbs and adverbials can be reasonably clearly defined both morpho-syntactically and semantically. The classes of question words and deictics include words with heterogeneous syntactic behaviour; question words have semantic and functional, and some morphological similarities as a group, whereas the category of deictics is based on strong morphological and semantic similarities. Connectives share the function of conjoining elements on the same level. As ``functor words'' postpositions and especially clitics are dependent on the preceding phrase. Interjections are different from all the other word classes in that they operate outside the normal syntax and often constitute a whole expression by themselves.

Nouns are naturally the largest category, but verbs are morphologically the most complex and interesting word class.

Although the great majority of the words in Mauwake can be assigned to one of the categories above, there is some indeterminacy with regard to some words that seem to belong to two or more word classes and the meanings which are clearly related.\footnote{In Austronesian languages it is common to have pre-categorial stems that may combine with affixation belonging to various word classes; only the whole word may be assigned to a particular word class.} They are not homonyms, since they are semantically related. Some transitive verbs have been derived by zero derivation from nouns and adjectives, and even from adverbs (\sectref{sec:3.8.2.2.1}, 3.8.4.4.3). Nominalized verbs (\sectref{sec:3.2.6.1}) function as nouns or adjectives. At the end of section 3.2.2 there is a list of words that are originally nouns but have become adjectives as well. Some non-numeral quantifiers (\sectref{sec:3.4.2}) also function as intensity adverbs (\sectref{sec:3.9.2}). Besides these, there are individual words that function in more than one word class; these are mentioned where they occur.

\section{Nouns}\label{sec:3:2}
%\hypertarget{RefHeading19261935131865}
{}
\subsection{General discussion}\label{sec:3:2:1}
%\hypertarget{RefHeading19281935131865}
{}
Although the traditional semantic definition of the noun as the ``name of a person, place or thing'' is not valid as a basis for assigning members to the class, it still gives a good general description of the prototypical members of the class in Mauwake. In \citegen[63]{Frawley1992} words, ``when the traditional definition is reversed, the definition turns out to be true. Nouns are not always persons, places or things, but persons, places and things always turn out to be nouns.''.\footnote{See also \citet[117]{Sapir1921}, \citet[60]{Jespersen1924}, \citet[449]{Lyons1977} and \citet[7]{Schachter1985}.} Recognizing the semantic motivation of the class does not eliminate the need to define the class by its formal or functional properties.

No good morphological definition of nouns is possible in Mauwake, as there is no inflection for number \REF{ex:3:}, gender or class,\footnote{Gender or class systems are widespread in Papuan languages \citep[77]{Foley1986}. Especially in the TNG languages a covert system is common \citep[58]{Wurm1982}, where the noun class determines what existential verb is used with each noun.} or case, in the noun itself. Especially the lack of plural marking is typical of the nouns in Trans-New Guinea languages \citep[36]{Wurm1982}. The glosses in the following example indicate a singular/plural alternative in the nouns, but the singular form in the glosses of other examples is to be understood as neutral regarding the number. 

\ea%x1
\label{ex:3:x1}
\gll siowa wiawi\\
dog(s) father(s)\\
\glt`The dog's/dogs' owner(s)'
\z

Nouns are usually monomorphemic, with the exception of a small group of inalienably possessed nouns (\sectref{sec:3.2.4}), nouns derived from verbs (\sectref{sec:3.2.6.1}), reduplicated nouns (\sectref{sec:3.2.6.2}) and compound nouns (\sectref{sec:3.2.5}). The division into \textstyleEmphasizedWords{count} and \textstyleEmphasizedWords{mass} nouns is not very noticeable. It is mainly shown in the choice between the quantifiers \textstyleStyleVernacularWordsItalic{unowa} `many' and \textstyleStyleVernacularWordsItalic{maneka} `big, much', and to some extent in verb agreement morphology (\sectref{sec:3.4}).

The syntactic function provides the best criterion for defining a noun in Mauwake. Nouns function mainly as the head of a noun phrase, often the head being the only element in the \textstyleAcronymallcaps{NP}.\footnote{Sometimes an adjective, a quantifier or a genitive pronoun looks like a head of a NP, but those cases are elliptical, and the head noun is recoverable.} They can also function as a qualifier or, more rarely, as a modifier in a \textstyleAcronymallcaps{NP}. In \REF{ex:3:x2} \textstyleAcronymallcaps{NP}s, in this case manifested by just nouns, function as subject and object.

\ea%x2
\label{ex:3:x2}
\gll Emeria=ke iwera fiirim-i-mik.\\
 woman=\textsc{cf} coconut gather-Np-\textsc{pr}.1/3p\\
\glt`(The) women gather coconuts.'
\z

Hopper and \citet[710]{Thompson1984} also maintain that {``from the discourse point of view, nouns function to introduce participants and `props' and to deploy them''}\footnote{Actually this is the function of a NP rather than a noun.}{.} This is true in Mauwake as well, but it is not used as a criterion for defining the nouns.

\subsection{Nouns and adjectives: one or two word classes?}\label{sec:3:2:2}
%\hypertarget{RefHeading19301935131865}
{}
Since adjectives in Mauwake are phonologically, morphologically and syntactically very similar to nouns, the question must be asked whether the two form just one class of nominals or whether they belong to two separate word classes. In the following discussion they are treated on a semantic basis as if they were separate classes, i.e. certain words are called nouns and others adjectives, but a final conclusion as to their status is not drawn until the end of the section.

A \textstyleEmphasizedWords{\textsc{phonologically}} interesting feature common to nouns and adjectives is that the majority of both end in the vowel /a/.\footnote{In the other word classes words ending in /a/ do occur but they are very infrequent.} Inside noun phrases this vowel, when unstressed, is usually elided preceding a vowel and often also preceding a consonant. In cases like \REF{ex:3:x3}, where there are two or more possible places for elision, the vowel most easily drops at the end of an adjective preceding an intensifier. Elision is also acceptable in two or more sites within one \textstyleAcronymallcaps{NP} \REF{ex:3:x3}, \REF{ex:3:x4}. 

\ea%x3
\label{ex:3:x3}
\gll koora eliw(a) akena  {\upshape also:} koor(a) eliw(a) akena \\
house good very\\
\glt`a very good house'
\z

\ea%x4
\label{ex:3:x4}
\gll koor(a) kemena manek(a) akena nain \\
house inside big very that1\\
\glt`the very big room'
\z

\textstyleEmphasizedWords{\textsc{Morphologically}} nouns and adjectives resemble each other in that they lack inflection. There is no number, case, or gender marking in the adjectives, nor is there any inflection for comparison. (For comparison of adjectives, see \sectref{sec:6.5}). 

Both nouns \REF{ex:3:x7} and adjectives \REF{ex:3:x8} may be derived from verbs with the nominaliser suffix \nobreakdash-\textstyleStyleVernacularWordsItalic{owa}.

\ea%x7
\label{ex:3:x7}
\gll mua \textstyleEmphasizedVernacularWords{soop-owa} sira \\
man bury-	extsc{nmz} custom\\
\glt`the burial custom (lit: the custom of burying men)'
\z

\ea%x8
\label{ex:3:x8}
\gll Emi \textstyleEmphasizedVernacularWords{kekan-owa} nain puuk-a-mik. \\
taboo be.strong-\textsc{nmz} that1 cut-\textsc{pa}-1/3p\\
\glt`They broke the strong taboo rule.'
\z

Verbs can be derived from both adjectives and nouns by zero verb formation \REF{ex:3:}, \REF{ex:3:} or by the inchoative verbaliser \nobreakdash-\textstyleStyleVernacularWordsItalic{ar} (\textstyleParagraphCharChar{\stepcounter{nx}{\thenx}}). (See \sectref{sec:3.8.2.2} for these processes and more examples.)

\ea%x482
\label{ex:3:x482}
\gll Miiw-aasa samor-a-k. \\
land-canoe bad-\textsc{pa}-3s\\
\glt`He broke/ruined the car.'
\z

\ea%x484
\label{ex:3:x484}
\gll Iwer(a) ififa palis-i-ya. \\
coconut dry pair.of.coconuts-Np-\textsc{pr}.3s\\
\glt`He is tying dry coconuts into pairs.'
\z

\ea%x483
\label{ex:3:x483}
\gll Miiw-aasa samor-ar-e-k. \\
land-canoe bad-\textsc{inch}-\textsc{pa}-3s\\
\glt`The car broke.'
\z

A clear morphological \textstyleEmphasizedWords{difference} between nouns and adjectives is that adverbs may be formed from some adjectives by deleting the word-final /a/, but they cannot be formed from nouns in the same way.

\ea%x19
\label{ex:3:x19}
\gll samora {{\textgreater} samor} \\
 \\
\glt`bad' `badly'
\z

\textstyleEmphasizedWords{\textsc{Syntactically}} there are a few similarities between nouns and adjectives. Both can function as a modifier following the head noun in a \textstyleAcronymallcaps{NP}, although adjectives \REF{ex:3:x9} are much more common in this position. In \citegen[161]{HopperEtAl1985} terms, it is nouns whose categorial status has been reduced, i.e. nouns that are not fully individuated in the discourse \REF{ex:3:x10}, that can function in this modifier position.

\ea%x9
\label{ex:3:x9}
\gll aasa \textstyleEmphasizedVernacularWords{awona} fain \\
canoe old this\\
\glt`this old canoe'
\z

\ea%x10
\label{ex:3:x10}
\gll mua \textstyleEmphasizedVernacularWords{sira} \textstyleEmphasizedVernacularWords{eliwa} \\
man manner good\\
\glt`a well-mannered man (=a good man)'
\z

The intensifier \textstyleStyleVernacularWordsItalic{akena} `real(ly), very' can also modify both adjectives \REF{ex:3:} and nouns \REF{ex:3:}.

\ea%x11
\label{ex:3:x11}
\gll mua \textstyleEmphasizedVernacularWords{akena} \\
man real/true\\
\glt`a real man'
\z

Complete or partial reduplication of adjectives is a common strategy for indicating plurality in Austronesian languages \citep[62]{Wurm1982}, and it also occurs to some extent in many Papuan languages, including Mauwake. Reduplication is a more productive process in the adjectives \REF{ex:3:x12}, \REF{ex:3:x481}, but it is possible for a few nouns too \REF{ex:3:x13}, \REF{ex:3:x1859} (\sectref{sec:3.2.6.2}). 

\ea%x12
\label{ex:3:x12}
\gll ifa \textstyleEmphasizedVernacularWords{samo-samora} \\
snake \textsc{rdp}-bad\\
\glt`bad snakes'
\z

\ea%x481
\label{ex:3:x481}
\gll Maa \textstyleEmphasizedVernacularWords{ele-eliwa} sesek-a-mik. \\
thing/food \textsc{rdp}-good sell-\textsc{pa}-1/3p\\
\glt`They sold good foods (different kinds).'
\z

\ea%x13
\label{ex:3:x13}
\gll \textstyleEmphasizedVernacularWords{Owow-owowa} ikiw-e-mik. \\
\textsc{rdp}-village go-\textsc{pa}-1/3p\\
\glt`They went to many villages.'
\z

\ea%x1859
\label{ex:3:x1859}
\gll \textstyleEmphasizedVernacularWords{sira-sira} \\
custom-custom\\
\glt`many customs', `different kinds'
\z

The syntactic \textstyleEmphasizedWords{\textsc{differences}} between nouns and adjectives are as follows. Adjectives do not function as the head of a noun phrase. The cases where they would seem to do so are in fact cases of ellipsis, and the head noun must be recoverable from the context, either linguistic or extra-linguistic. 

\ea%x14
\label{ex:3:x14}
\gll {\O} awona nain p-ekap-e! \\
{\O} old that1 BPx-come-\textsc{imp}.2s\\
\glt`Bring the old one!'
\z

Only a noun may occur as a qualifier in a noun phrase, preceding the head noun \REF{ex:3:}. In some of these cases it is difficult to decide whether they are really \textstyleAcronymallcaps{NP}s with a qualifier and a head noun, or compound nouns. But if the latter is the case, then the restriction applies that an adjective cannot be the first element in a compound noun.

\ea%x15
\label{ex:3:x15}
\gll \textstyleEmphasizedVernacularWords{mera} eka \\
fish water\\
\glt`fish soup'
\z

\ea%x16
\label{ex:3:x16}
\gll [[\textstyleEmphasizedVernacularWords{mera} \textstyleEmphasizedVernacularWords{eka}] \textstyleEmphasizedVernacularWords{en-owa}] sira \\
fish water eat-\textsc{nmz} custom\\
\glt`the custom of eating fish soup'
\z

An adjective cannot be the only element following a genitive pronoun, but a noun can. Even in elliptical expressions an adjective following a genitive pronoun is not very acceptable \REF{ex:3:x17}. 

\ea%x17
\label{ex:3:x17}
\gll ?Yiena {\O } \textstyleEmphasizedVernacularWords{awona} nain p-ekap-e! \\
1p.\textsc{gen} {\O} old that1 BPx-come-\textsc{imp}.2s\\
\glt`Bring our old one(s)!'
\z

An exception to this rule is the adjective \textstyleStyleVernacularWordsItalic{maneka} `big'. The expression \textstyleStyleVernacularWordsItalic{yiena Maneka} `our Lord' (literally: our Big one), is probably formed following Tok Pisin \textstyleForeignWords{Bikpela bilong yumi.}\footnote{Non-\textsc{pr}ototypical adjectives are discussed later in this section; `big' is a prototypical adjective, so its use in a typically nominal position is an exception.} 

\ea%x105
\label{ex:3:x105}
\gll wi Amerika \textstyleEmphasizedVernacularWords{maneka}, unuma Magerka \\
3p.\textsc{unm} America big name MacArthur\\
\glt`the leader of the Americans, whose name was MacArthur'
\z

Only an adjective functions as the head of an adjective phrase. In that position it may be modified by intensity adverbs (\sectref{sec:3.9.2}). Of these, \textstyleStyleVernacularWordsItalic{lawisiw} `rather' does not modify nouns at all \REF{ex:3:x18}; \textstyleStyleVernacularWordsItalic{akena} `very' and \textstyleStyleVernacularWordsItalic{pepek} `enough' may modify nouns as well; \textstyleStyleVernacularWordsItalic{wenup} `very'can do that too, but as a noun modifier it has a somewhat restricted use and a different meaning, `many'.

\ea%x18
\label{ex:3:x18}
\gll Mera nain \textstyleEmphasizedVernacularWords{lawisiw} \textstyleEmphasizedVernacularWords{maneka} \textstyleEmphasizedVernacularWords{akena}. \\
fish that1 rather big very\\
\glt`That fish is rather huge.'
\z

What further obscures the area of nouns and adjectives is the fact that there are a number of words that sometimes function like nouns \REF{ex:3:x20}, sometimes like adjectives \REF{ex:3:x21}, and also semantically could be like either.

\ea%x20
\label{ex:3:x20}
\gll \textstyleEmphasizedVernacularWords{Pina} maneka kamenap? \\
weight big what.like\\
\glt`What is the weight like?', `How big is the weight?'
\z

\ea%x21
\label{ex:3:x21}
\gll Maa nain lawisiw \textstyleEmphasizedVernacularWords{pina}. \\
thing that1 rather heavy\\
\glt`The thing is rather heavy.
\z

The prototype view offers a plausible solution for the problem. Starting from the study of basic colour terms \citep{BerlinEtAl1969} it has been applied to other areas of semantics and also to linguistic categorization (e.g.\citealt{Wierzbicka1986,Taylor1989} and \citealt{Frawley1992}). The main idea that categories have more central, or focal, members as well as more marginal members was also recognized by \citet{Crystal1967} in his description of English word classes. The prototype approach allows for stability as well as flexibility \citep[53]{Taylor1989}, both of which are needed in an attempt to describe a human language.

If prototypical linguistic categories are focal, or optimal, instances on a continuum \citep[321]{Seiler1978} and maximally distinct from one another \citep[709]{HopperEtAl1984}%Thompson
, what are prototypical nouns like as opposed to prototypical adjectives? According to \citet{Wierzbicka1986}, noun indicates \textstyleEmphasizedWords{\textsc{categor}}\textstyleEmphasizedWords{\textsc{ization}}: most prototypical nouns identify a certain kind of person, thing or animal. Relative \textstyleEmphasizedWords{\textsc{temporal stability}} is for Giv\'on what characterizes nouns, and the most prototypical nouns denote concrete, physical, compact entities \citeyear[151]{Givon1984}. Instead of time stability, \citet[66]{Frawley1992} claims it is relative \textstyleEmphasizedWords{\textsc{atemporality}}\textstyleEmphasizedWords{} that makes an entity an entity. Adjectives, or property concepts, indicate \textstyleEmphasizedWords{\textsc{description}}, and they denote single properties unlike nouns which denote a cluster of properties \citep{Wierzbicka1986}.

In Mauwake, a prototypical noun occurs as a head in a \textstyleAcronymallcaps{NP}, as a pre-modifier or, less frequently, as a post-modifier in a \textstyleAcronymallcaps{NP}, or as any element in a compound noun. It does not occur as the head in an \textstyleAcronymallcaps{AP}. It can be modified by adjectives or genitive pronouns but not by the intensity adverbs \textstyleStyleVernacularWordsItalic{lawisiw} `rather' and \textstyleStyleVernacularWordsItalic{wenup} `very'. Prototypical \textstyleEmphasizedWords{\textsc{adjectives}}\textstyleEmphasizedWords{ }occur\textstyleEmphasizedWords{ }as the head of an adjective phrase. They do not pre-modify nouns or function as the first element in a compound noun.

It turns out that in Mauwake the most prototypical nouns include names of concrete \textstyleEmphasizedWords{\textsc{non}}-human rather than human objects, when one would expect words referring to human beings to be nouns \textstyleEmphasizedWords{\textsc{par excellence}} (see \citealt[192]{Taylor1989}). Some human nouns may be used as post-modifiers in a \textstyleAcronymallcaps{NP}: from the cluster of properties denoted by the noun one has been picked out, and the noun is used like an adjective \REF{ex:3:x23}, \REF{ex:3:x24}. The adjectival use of \textstyleStyleVernacularWordsItalic{mua} `man' in \REF{ex:3:x23} is particularly interesting, because the adjectives \textstyleStyleVernacularWordsItalic{morena} `male' and \textstyleStyleVernacularWordsItalic{suwina} `female' are used for animals.

\ea%x23
\label{ex:3:x23}
\gll labuel(a) mua \\
pawpaw man\\
\glt`male pawpaw'
\z

\ea%x24
\label{ex:3:x24}
\gll donki takira \\
donkey young.person\\
\glt`young donkey'
\z

The less prototypical status of human nouns also shows in words like \textstyleStyleVernacularWordsItalic{apura} `widow' and \textstyleStyleVernacularWordsItalic{oosa} `widower' which may occur by themselves as heads of a \textstyleAcronymallcaps{NP}, but which are most typically used as post-modifiers of \textstyleStyleVernacularWordsItalic{emeria} `woman' and \textstyleStyleVernacularWordsItalic{mua} `man', respectively.\footnote{Other words in this group are \textstyleFootnoteBaseChar{\textit{muupera}}\textbf{\textit{} }`visitor, guest' and especially \textstyleFootnoteBaseChar{\textit{weria}}, which as a human noun only occurs in the combination \textstyleFootnoteBaseChar{\textit{mua weria}}, `uncle/ male cross cousin/ nephew'. The \textstyleFootnoteBaseChar{\textit{mua weria'}}s are responsible for burying a dead person and dispensing of his/her belongings (\sectref{sec:1.3.6}).} As age in human beings tends to be to be treated as a crucial determinant of \textstyleEmphasizedWords{\textsc{kind}}, even languages with large adjective classes often have special nouns for referring to old persons \citep[368]{Wierzbicka1986}. In Mauwake, adjectives that indicate age in humans are non-\textsc{pr}ototypical, more noun-like than most adjectives: both \textstyleStyleVernacularWordsItalic{iperowa} `middle-aged' and \textstyleStyleVernacularWordsItalic{panewowa} `old' are used as the head of a \textstyleAcronymallcaps{NP} besides the typical adjectival use.

\ea%x25
\label{ex:3:x25}
\gll \textstyleEmphasizedVernacularWords{Iperowa} opora wiar miim-i-yen. \\
middle-aged talk 3.\textsc{dat} hear-Np-\textsc{fu}.1p\\
\glt`We will listen to the talk of the middle-aged (men).'
\z

According to \citet[56]{Dixon1977}, if a language has adjectives at all, words expressing age, dimension, value and colour are likely to belong to the adjective class, however small the class. The most prototypical adjectives in Mauwake belong to these groups, with the exception of adjectives denoting human age, discussed above. In the group of adjectives denoting either physical property or human propensity, some are ambiguous as to their basic category: \textstyleStyleVernacularWordsItalic{anima} is both `blade' and `sharp', and \textstyleStyleVernacularWordsItalic{pina} both `weight, burden' and `heavy'. Different groups of adjectives, as well as the use of adjectives, are discussed below in Section 3.3.

With the rules given above it is fairly straightforward to distinguish the nouns and adjectives in Mauwake. But a small group remains that seems to have a membership in both classes. Originally they are are nouns that have now been employed as adjectives as well. The claim is based on the fact that the noun category is the more basic and universally recognized, whereas the existence of the adjective category is disputed in some languages; and in Mauwake the noun class is clearly established, large, and more easily definable. Also, there are at least two nouns in Mauwake that currently seem to be in the process of becoming regular adjectives: the meaning of the phrase stays the same with the pre-modifying noun and the post-modifying adjective. 

\ea%x107
\label{ex:3:x107}
\gll \textstyleEmphasizedVernacularWords{napum(a)} mua\\
sickness man\\
\glt`a sick man'
\z

\ea%x108
\label{ex:3:x108}
\gll mua \textstyleEmphasizedVernacularWords{napuma} \\
man sick\\
\glt`a sick man', also: `human (lit: man's) sickness'
\z

\ea%x1822
\label{ex:3:x1822}
\gll \textstyleEmphasizedVernacularWords{wadol(a)} opora \\
lie/false talk\\
\glt`a lie'
\z

\ea%x1823
\label{ex:3:x1823}
\gll opor(a) \textstyleEmphasizedVernacularWords{wadola} \\
talk lie/false\\
\glt`a lie'
\z

\tabref{tab:3:nounadj} gives a list of the most common of the words functioning both as nouns and as adjectives.

\begin{table}
 \caption{Words functioning both as nouns and adjectives.}
\label{tab:3:nounadj}

\begin{tabular}{lll} 
\mytoprule
anima &`blade, point, edge' &`sharp'\\
afila &`grease' &`greasy, sweet'\\
foma &`ashes' &`grey'\\
ikina &`smell' &`smelly'\\
irauwa &`hole' &`deep'\\
makena &`true' &`truth, essential nature'\\
napuma &`sickness, corpse' &`sick'\\
pina &`weight, burden, guilt' &`heavy'\\
siisia &`design, pattern' &`spotted, patterned'\\
tumina &`dirt' &`dirty'\\
wadola &`lie' &`false, fake'\\
\mybottomrule
\end{tabular} 

\end{table}
\subsection{Common vs. proper nouns}\label{sec:3:2:3}
%\hypertarget{RefHeading19321935131865}
{}
There is very little difference between common and proper nouns in Mauwake, and it can be questioned whether the two should be grouped separately as is traditionally often done in language descriptions. Proper nouns are sometimes classified separately because they are said to be unable to have modifiers \citep[152]{Roberts1987}, and in practice, they usually occur without any modifiers. This is related to the fact that they normally only have a referent, but no intension. In most of the cases where a proper noun is modified, ``it lacks a unique reference and is being used as a common noun'' \citep[59]{VanValinEtAl1997}:

 (\stepcounter{nx}{\thenx}x26) I mean the old and cranky Joe Smith, not the younger one. 

The most common type of a proper noun is a name of a \textstyleEmphasizedWords{\textsc{person}}. A proper noun may also become a true common noun, when one or more of the qualities of a person are used to characterise some other being \citep[66]{Jespersen1924}. For example, the name of a well-known expatriate, Jooren, was borrowed by Mauwake speakers to mean `a stingy shopkeeper' (that is, one who does not sell things on credit and does not give discount to relatives). 

In Mauwake proper names can be modified without difficulty, especially by the demonstrative \textstyleStyleVernacularWordsItalic{nain} `that', but also by adjectives. In a culture where there are several namesakes, and surnames are rarely used, modifiers are occasionally needed to distinguish between people \REF{ex:3:x27}.

\ea%x27
\label{ex:3:x27}
\gll \textstyleEmphasizedVernacularWords{Adek} \textstyleEmphasizedVernacularWords{panewowa} \textstyleEmphasizedVernacularWords{nain} ma-i-yem. \\
Adek old that1 say-Np-\textsc{pr}.1s\\
\glt`I am talking about the \textstyleEmphasizedWords{old} Adek.'
\z

But even proper names that have a unique reference and do not need to be distinguished from any other referent can be modified:

\ea%x106
\label{ex:3:x106}
\gll \textstyleEmphasizedVernacularWords{Dabe} \textstyleEmphasizedVernacularWords{fain} uuw-ow(a) mua=ke. \\
Dabe this work-\textsc{nmz} man=\textsc{cf}\\
\glt`Dabe here is a hard worker.'
\z

In this case the behaviour of proper names is similar to that of the personal pronouns, which also have unique reference, but can be modified nevertheless. {Van Valin and LaPolla} (ibid. 59--60) note that languages may vary in how freely they allow proper nouns and pronouns to take modifiers.

Name taboos influence the use of personal names in several ways. A person is given many different names: at least one from each parents' side (as in-laws may not mention each others' names), a baptismal name, and possibly others as well. These names are used by different people. Name taboos may be avoided by calling someone by a teknonym like `Sarak's father', or by calling a wife by the husband's name when she is with the in-laws and the husband is not around. Nicknames, often referring to physical properties, are also very common: \textstyleStyleVernacularWordsItalic{buburia} `bald', \textstyleStyleVernacularWordsItalic{mua kuuma} `lame' (literally `stick-man'). The term `namesake' is very common and even used of people who have been named after different names of the same person. Two boys, Yoli and Wangali, were called namesakes of each other, as they were both named after the same ancestor. 

Perhaps the most characteristic feature of personal names is \textstyleEmphasizedWords{\textsc{discourse-\textsc{pr}agmatic}}: in a text their token frequency is very low. Especially the main participant, once (s)he has been mentioned by name -- if (s)he ever is -- (s)he is then usually referred to by other means: a \textstyleAcronymallcaps{NP}, pronoun, or just person marking on the verb. 

Besides the names of people, \textstyleEmphasizedWords{\textsc{place names}} form another large group of proper names. In Mauwake, the proper name often modifies a generic noun: \textstyleStyleVernacularWordsItalic{Moro} (\textstyleStyleVernacularWordsItalic{owowa}) `Moro (village), \textstyleStyleVernacularWordsItalic{Siburten} (\textstyleStyleVernacularWordsItalic{ema}) `Siburten (mountain/hill)', \textstyleStyleVernacularWordsItalic{Nemuru} (\textstyleStyleVernacularWordsItalic{eka}) `Nemuru (river)' (\sectref{sec:4.1}). 

The place name is also used when the inhabitants are referred to. When reference is made to an individual or a select group, the place name is used as a qualifier in the noun phrase: 

\ea%x421
\label{ex:3:x421}
\gll \textstyleEmphasizedVernacularWords{Amiten}  mua oko ekap-o-k. \\
Amiten man other come-\textsc{pa}-3s\\
\glt`A man from Amiten came.'
\z

When the whole group is referred to, a plural pronoun is added to the place name:

\ea%x422
\label{ex:3:x422}
\gll \textstyleEmphasizedVernacularWords{I} \textstyleEmphasizedVernacularWords{Moro=ke} uf-e-mik. \\
1p.\textsc{unm} Moro=\textsc{cf} dance-\textsc{pa}-1/3p\\
\glt`We Moro people danced.'
\z

\ea%x423
\label{ex:3:x423}
\gll \textstyleEmphasizedVernacularWords{(Wi)} \textstyleEmphasizedVernacularWords{Lasen} \textstyleEmphasizedVernacularWords{wia} nokar-e-k.\footnotemark{} \\
3p.\textsc{unm} Lasen 3p.\textsc{acc} ask-\textsc{pa}-3s\\
\glt`He asked the Lasen people'
\z

\footnotetext{The optional initial pronoun \textit{wi} is part of the object here, not a subject pronoun.}

\subsection{Alienable and inalienable possession}\label{sec:3:2:4}
%\hypertarget{RefHeading19341935131865}
{}
The Austronesian languages in Melanesia tend to have very elaborate semantically based possessive systems that indicate the relationship between the ``possessor'' and the ``possession'': kin relation, body part, food etc. Inalienable possession is indicated by affixation on the noun, alienable possession by a separate possessive pronoun. Because of this, the simpler inalienable possession marking also evident in many \textstyleAcronymallcaps{TNG} languages could easily be attributed to influence from Austronesian languages. But \citet[28]{Ross1996} claims it is likely that even Proto \textstyleAcronymallcaps{TNG} had inalienable nouns before there was any contact with Austronesian languages.\footnote{On the time frames of TNG occupation and Austronesian migration, see e.g. \citet[39--41]{Ross2005}.} In Mauwake the division into alienably and inalienably possessed nouns is along the lines of kinship terms (see \sectref{sec:1.3.6} for a kinship chart). Most kin terms obligatorily indicate who the {\textquotedbl}possessor{\textquotedbl} is:

\begin{table}
\caption{Please provide a caption}
\label{}
\begin{tabular}{lllll} 
\mytoprule
 &1s/p &2s/p &3s/p &possessor\\
\midrule
a. &auwa &niawi &wiawi &`father'\\
b. &aite &niena &onak &`mother'\\
c. &paapa &neepe &weepe &`elder sibling'\\
d. &(y)aamun &niamun &wiamun &`younger sibling'\\
e. &yaaya &nie &wie &`uncle'\\
f. &paapan &noopan &woopan &`aunt'\\
g. &kae &neke &weke &`grandfather'\\
h. &kome &nokome &wokome &`grandmother'\\
i. &eremena &neremena &weremena &`nephew, niece'\\
j. &emar, yomar &nomar &womar &`(cross-)cousin'\\
k. &yomokowa &nomokowa &womokowa &`brother'\footnote{Among siblings, age is more important than sex: \textstyleFootnoteBaseChar{\textit{paapa}} and \textstyleFootnoteBaseChar{\textit{aamun}} are used very frequently and for siblings of either gender. When the gender is in focus, \textstyleFootnoteBaseChar{\textit{yomokowa}} is used for `my brother' and \textstyleFootnoteBaseChar{\textit{ekera}} for `my sister' especially by siblings of the opposite sex.}\\
l. &(y)ekera &nekera &wekera &`sister'\\
m. &(y)emi &nemi &wemi &`(man's) brother-in-law'\\
n. &epua &nepua &wepua &`(woman's) brother-in-law\footnote{A woman calls her elder sister's husband \textit{auwa} `father', but the other brothers-in-law are \textit{epua}.}\\
o. &yomora &nomora &womora &`sister-in-law'\\
p. &yopariw &nopariw &wopariw &`husband's brother's wife'\\
q. &yamekua &namekua &wamekua &`daughter-in-law'\footnote{Some in-law relations are non-symmetrical: even though there are special terms for sons- and daughters-in-law, \textstyleFootnoteBaseChar{\textit{auwa}} `(my) father' and \textstyleFootnoteBaseChar{\textit{aite}} `(my) mother' are used for `(my) mother-in-law' and `(my) father-in-law'.}\\
r. &yar &nar &war &`son-in-law'\\
s. &yookati &nookati &wookati &`co-wife'\footnote{This term dates back to the time when polygamy was practiced; it was used for the wives of the same man.}\\
t. &yomawa &nomawa &womawa &`namesake'\\
\mybottomrule 
\end{tabular}
\end{table}

The possessive prefixes \textstyleStyleVernacularWordsItalic{y}-, \textstyleStyleVernacularWordsItalic{n}- and \textstyleStyleVernacularWordsItalic{w}- in the inalienably possessed nouns developed from the first, second, and third person pronouns. These prefixes are in the process of merging with the root. The terms in (a-j) above are somewhat more lexicalized than the ones in (k-s): the first person prefix is mostly lost, and in some cases there is suppletion in the stem. These are some of the socially most important and frequently used kinship terms. The frequent use probably accounts for the omission of the possession prefix in the first person: these terms are used more as terms of address, whereas the other kinship nouns are only needed as terms of reference. Also, there is a tendency to drop the first person prefix before the front vowel /e/ regardless of the closeness of the kinship relation.

The ``possessors'' are differentiated as first, second or third person but not as single vs. plural. An unmarked \REF{ex:3:x1311} or a genitive \REF{ex:3:x28}, \REF{ex:3:x1312} pronoun may be used to either make this number distinction or to emphasise the kin relationship, when the relationship is used as a term of reference rather than as a term of address.

\ea%x1311
\label{ex:3:x1311}
\gll Kuuten \textstyleEmphasizedVernacularWords{wiawi} iperowa, \textstyleEmphasizedVernacularWords{yo} \textstyleEmphasizedVernacularWords{auwa} kapa=ke. \\
Kuuten 3s/p.father firstborn 1s.\textsc{unm} 1s/p.father lastborn=\textsc{cf}\\
\glt`Kuuten's father was the firstborn, my father the lastborn.'\footnote{Both of these fathers could be called \textit{auwa} `my/our father(s)' by the two men.}
\z

\ea%x28
\label{ex:3:x28}
\gll Aakisa \textstyleEmphasizedVernacularWords{yena}  \textstyleEmphasizedVernacularWords{auwa} kapa fain=ke yia uruf-i-ya. \\
now 1s.\textsc{gen} 1s/p.father lastborn this=\textsc{cf} 1p.\textsc{acc} see-Np-\textsc{pr}.3s\\
\glt`Now this lastborn of my ``fathers'' watches over us.'
\z

\ea%x1312
\label{ex:3:x1312}
\gll Sa, a \textstyleEmphasizedVernacularWords{nena} \textstyleEmphasizedVernacularWords{nie=ke}, \textstyleEmphasizedVernacularWords{nena} \textstyleEmphasizedVernacularWords{nepua=ke,} niawi=ke.\\
\textsc{intj} \textsc{intj} 2s.\textsc{gen} 2s/p.uncle=\textsc{cf} 2s.\textsc{gen} 2s/p.brother-in-law 2s/p.father\\
\glt`(Don't you understand,) those are \textit{your} uncle(-in-law), \textit{your} brother-in-law and father(-in-law).'
\z

When a neutral, ``non-possessed'', kinship term is needed, the first person form is used. This is interesting, as the third person singular is typically considered the neutral, or unmarked, form. The terms `(my) mother' and `(my) father' are also used as respectful terms of address for almost any stranger regardless of age, or for anyone whose status in the kinship system is uncertain.\footnote{I have been addressed as {\textquotedbl}mother{\textquotedbl} by an old man who temporarily forgot what my status according to their kinship system was - I was actually his granddaughter!}

Four alienably possessed nouns, namely those for `man', `woman', `boy' and `girl', have been taken into the kinship system for terms of some nuclear family members: 

\begin{table}
\caption{Please provide a caption}
\label{}
\begin{tabular}{ll}
\mytoprule
mua &`man, husband'\\
emeria &`woman, wife'\\
muuka &`boy, child, son'\\
wiipa &`girl, daughter'\\
\mybottomrule
\end{tabular}
\end{table}


Also the term \textstyleStyleVernacularWordsItalic{nembesir} `ancestor (beyond grandparents)' or `descendant (beyond grandchildren)' is an alienably possessed noun, possibly because relatives so far removed in time are considered less relevant. It is used both for males and females. But the term for `namesake', \textstyleStyleVernacularWordsItalic{yomawa}, is included in the inalienably possessed kinship terms, as a child is named after some relatives, and the namesake relation forms an additional bond between them.

\subsection{Noun compounding}\label{sec:3:2:5}
%\hypertarget{RefHeading19361935131865}
{}
The distinction between compound nouns and noun phrases is a problematic area in many languages, including Mauwake. Both are formed by combining independent elements into larger units, and their form and meaning are largely based on the form and meaning of those elements \citep[40]{Anderson1985a}. Phonological, morphological, syntactic as well as semantic criteria have been called upon to differentiate between compounds and phrases.

In many languages, ``word accent'' \citep[204]{Lyons1968}, i.e. stress and/or pitch, helps to distinguish compounds. In Mandarin Chinese, contrastive stress can only fall on the ``stress center'' of a word, including compounds \citep[41]{Anderson1985a}. In Finnish, the primary stress is on the first, and only on the first, syllable of even very long compound words like \textstyleForeignWords{kuluttajansuoja-asiamiesverkostokysymys} `the question of consumer ombudsman network', but even in Finnish there are unclear cases like \textstyleForeignWords{valveillaolo} vs. \textstyleForeignWords{valveilla olo}\textstyleEmphasizedWords{} `being awake'. In the latter, the varying writing convention reflects the ambiguity. 

Linguists differ in their views about the importance of stress placement in interpreting English compounds. \citet[228]{Bloomfield1935} and \citet[41]{Anderson1985a} consider it criterial, and so do \citet[1330]{QuirkEtAl1989}, although more cautiously. \citet[120]{Lees1968} takes it as one premise for his study of compounds while admitting that the case is not very well substantiated. Others, like \citet[31]{Jespersen1933}, \citet{Downing1977} and \citet{Bauer1983} do not consider a single primary stress essential for compounds. According to \citet[105]{Bauer1983}, \citeauthor{Lyons1968}' \citeyear[20]{Lyons1968} criteria for judging ``wordness'' in English, i.e. positional mobility and uninterruptability (or internal stability) do not distinguish between single- and double-stressed compounds.

Morphology may place constraints on compounding. In English, the genitive is common in phrases but rare in compounds: duck's egg vs. duck-egg (Anderson{ 1985a}:41).\footnote{But note also women's lib(eration), a compound.} In Finnish, the first part of a compound is often in the nominative or genitive case, whereas the other cases are infrequent in this position. In German, certain elements may serve as morphological ``glue'' between the parts of a compound (ibid. 42).

The two criteria for wordness by \citet[202]{Lyons1968} mentioned above are syntactic in nature: a word, hence also a compound, is moved as one unit, and cannot be interrupted by other words as a phrase often can. These criteria do not apply to all, and only, compound words, but they are useful in trying to establish the difference between compounds and phrases in a given language. \citet[232]{Bloomfield1933} adds another one: a member of a compound generally cannot serve as a constituent in a syntactic construction. One can say \textstyleEmphasizedWords{a very black bird} but not \textstyleEmphasizedWords{* a very blackbird}.

The semantic interpretation of phrases is generally quite compositional: the meaning of the whole can be deduced from the meanings of the words. Compounds are more heterogeneous in their interpretation: some are compositional, whereas others involve special interpretive principles not applicable to phrases. Also, compounds as words are subject to changes of meaning, so many compounds may have meanings that are only vaguely or metaphorically related to that which is predicted on the basis of the parts (Anderson{ 1985a}:42). Knowledge of the pragmatics of the situation may be needed for the interpretation of many compound words (Bauer{ 1983}:58). The more fully lexicalized the compounds are, the more the meaning of the whole may deviate from the meaning of the parts. The same compound word may also be fully lexicalized in a certain context, and still be open for other interpretations in other contexts (Andrew Pawley, p. c.). 

While there are languages where it is easy to distinguish between compound nouns and noun phrases, in others there is an intermediate area between the two. Thus \citet[810]{Downing1977} doubts that the dividing line is always well-defined, and  \citet[1569]{QuirkEtAl1989} suggest the concept of ``partial compounding'' to account for the formal and semantic gradience between compounds and phrases in English. Bringing a historical viewpoint to the question, citing developments in English both from phrase to compound and from compound to phrase, \citet[102]{Jespersen1924} offers a very liberal view: ``it is of no consequence whether we reckon [the] doubtful cases as one word or two words, for ... a word group (like a single word) may be either primary or an adjunct or a subjunct''. 

None of the criteria mentioned above can be easily applied in Mauwake. \textstyleEmphasizedWords{\textsc{Semantically}} there is a continuum between fully compositional noun phrases and fully lexicalized compounds. But \citet[227]{Bloomfield1933} warns that the greater specialization in meaning in the compound words as against phrases should not be used as a criterion, as ``we cannot gauge meanings accurately enough, and many a phrase is as specialized in meaning as any compound''. This warning is all the more relevant when one studies a language not one's own. 

The basic \textstyleEmphasizedWords{\textsc{stress pattern}} of noun phrases and compounds is similar, as one of the modifiers usually receives the phrase stress rather than the head noun \REF{ex:3:x29}, \REF{ex:3:x30}. Likewise, in compound nouns the modifying formative receives the main stress and the main formative is only weakly stressed \REF{ex:3:x31}, \REF{ex:3:x32}: the ``stress centre'' \citep[45]{Anderson1985a} is on another element than the head. 

\ea%x29
\label{ex:3:x29}
\gll yo 'auwa aasa\footnotemark{} \\
1s.\textsc{unm} 1s/p.father canoe\\
\glt`my father's canoe'
\z

\footnotetext{In the examples \REF{ex:3:}-\REF{ex:3:} only the phrase stress is marked by ' preceding the stressed syllable.}

\ea%x30
\label{ex:3:x30}
\gll aas(a) ge'lemuta \\
canoe small\\
\glt`a small canoe'
\z

\ea%x31
\label{ex:3:x31}
\gll 'miiw(a)-aasa\footnotemark{} \\
land-canoe\\
\glt`vehicle, car'
\z
\footnotetext{In Mauwake orthography, the parts of a compound word are usually written separately to help the new readers to identify the parts; \textit{miiw-aasa} `vehicle' is one of the exceptions.}

\ea%x32
\label{ex:3:x32}
\gll enow(a) ge'lemuta{\footnotemark} \\
food/meal small\\
\glt`feast'
\z
\footnotetext{\textit{Enow gelemuta} is not used with its literal meaning `small meal'.}
However, the head noun in a \textstyleAcronymallcaps{NP} may receive the phrase stress if it is emphasized for contrast, clarification or some other reason, whereas the stress centre in a compound stays the same. 

Since there is hardly any \textstyleEmphasizedWords{\textsc{morphology}} in nouns and noun phrases, one would not expect to find much help here in distinguishing between compounds and phrases. But there is a minor factor that is relevant in this respect: a phrase containing a noun and an adjective can be pluralized by adjectival reduplication when the adjective allows reduplication \REF{ex:3:x33}, whereas a compound noun with a similar structure usually cannot \REF{ex:3:x34}, even if it is possible in some rare cases \REF{ex:3:x35}. 

\ea%x33
\label{ex:3:x33}
\gll maa gelemuti-tik \\
thing small-\textsc{rdp}\\
%\glt`small things' %does not exist in original document
\z

\ea%x34
\label{ex:3:x34}
\gll *enow(a) gelemuti-tik \\
food/meal small\\
\glt `feast'
\z


\ea%x35
\label{ex:3:x35}
\gll owow(a) mane-maneka \\
village \textsc{rdp}-big\\
\glt`towns', `big villages'
\z

Uninterruptibility is more typical of compounds than phrases. The noun phrase \textstyleStyleVernacularWordsItalic{owow maneka} means `a big village', as a compound it means `a town/city'. As a phrase it is interruptible \REF{ex:3:x1768}, as a compound it is not.

\ea%x1768
\label{ex:3:x1768}
\gll owowa lawisiw maneka \\
village rather big\\
\glt`a rather big village'
\z

Likewise, as a compound \textstyleStyleVernacularWordsItalic{kae sira} 'ancestral custom' (literally: `grandfather's custom') is uninterruptible. When a genitive pronoun is inserted between the two parts, the meaning cannot be `ancestral custom':

\ea%x1860
\label{ex:3:x1860}
\gll kae ona sira \\
grandfather 3s.\textsc{gen} custom\\
\glt`grandfather's custom/habit'
\z

In Mauwake word combinations are treated as compounds if they 1) have a specialized meaning, 2) have a stress centre not affected by contrastive stress, and 3) tend to be uninterruptible. However, this distinction is very tentative in some cases. Some examples are provided where the same combination may be either a compound noun or a noun phrase.

Morphologically there are four compound noun types in Mauwake: \textstyleAcronymallcaps{N}+\textstyleAcronymallcaps{N,} \textstyleAcronymallcaps{V}\textsubscript{NMZ}\textstyleAcronymallcaps{} +\textstyleAcronymallcaps{N, N+V}\textsubscript{NMZ}\textstyleAcronymallcaps{ } and \textstyleAcronymallcaps{N}+\textstyleAcronymallcaps{ADJ}. Syntactically these correspond to a head noun with a nominal pre- or post-modifier in a \textstyleAcronymallcaps{NP} or a head noun with an adjective post-modifier in the \textstyleAcronymallcaps{NP}. In most compound nouns the last noun is the head. But in generic-specific compounds as well as the \textstyleAcronymallcaps{N}+\textstyleAcronymallcaps{ADJ} and\textstyleAcronymallcaps{} \textsc{N+V}\textsubscript{NMZ} compounds the first part is the main element and the scope of its meaning is restricted by the second part. In coordinate compounds the two parts are equally important.

On the basis of the semantic relations between the parts the \textstyleAcronymallcaps{N}+\textstyleAcronymallcaps{N} compounds can be divided into a few main groups. In the first one the relationship can be said to be characterized by \textstyleEmphasizedWords{\textsc{origin}} understood very widely, e.g. in the sense of place of origin \REF{ex:3:}, source \REF{ex:3:}, or ``possession'' \REF{ex:3:}, \REF{ex:3:}. 

\ea%x37
\label{ex:3:x37}
\gll piip(a) mera \\
seaweed fish\\
\glt`rainbow fish'
\z

\ea%x40
\label{ex:3:x40}
\gll emeria napuma \\
woman sick(ness)\\
\glt`menstruation'
\z

\ea%x41
\label{ex:3:x41}
\gll ibiamun sama \\
dove ladder\\
\glt`cross-beam (in a roof)'
\z

The compound noun \REF{ex:3:} has the stress centre on the first part, but the noun phrase \textstyleStyleVernacularWordsItalic{emeria napuma}, with the phrase stress on \textstyleStyleVernacularWordsItalic{napuma}, may be used to mean either `a sick woman', or more commonly `a (dead) woman's body', a euphemistic expression. 

The second relationship is a \textstyleEmphasizedWords{\textsc{whole-\textsc{pa}rt}} relationship: the first element states the whole, the second its part.

\ea%x42
\label{ex:3:x42}
\gll mokok(a) oposia \\
eye meat\\
\glt`pupil (of the eye)'
\z

\ea%x43
\label{ex:3:x43}
\gll ekek(a) muuna \\
branch joint/projection\\
\glt`bud'
\z

The third relationship is that of \textstyleEmphasizedWords{\textsc{container}}. As a compound \textstyleStyleVernacularWordsItalic{muuk(a) sia} \REF{ex:3:x45} has the stress centre on the first word, in a noun phrase \REF{ex:3:x1770} the phrase stress may also be on the second item if it is emphasized; a third person singular genitive pronoun may be added between the parts as well. Example \REF{ex:3:x46} is an extended compound: \textstyleStyleVernacularWordsItalic{iinan aasa} is a ``sky canoe'', or vehicle, for flying in the sky, and \textstyleStyleVernacularWordsItalic{iinan aasa epa} a place for those vehicles.

\ea%x45
\label{ex:3:x45}
\gll muuk(a) sia \\
son netbag\\
\glt`womb', `pouch (of a marsupial) 
\z 
\todo{check whether \REF{ex:x1770} was a subexample}

\ea%x1770
\label{ex:3:x1770}
\gll muuk(a) sia \\
son netbag\\
\glt`a son's/child's netbag (used for carrying the baby)'
\z

\ea%x46
\label{ex:3:x46}
\gll iinan aasa epa \\
sky canoe place\\
\glt`airstrip, airport'
\z

As was mentioned above, the \textstyleEmphasizedWords{\textsc{generic-specific}} relationship is different in that the modifying part follows rather than precedes the main part. In this respect these compounds resemble phrases where the head noun has an adjective rather than a noun modifier. A particularly common word for the first part in these compounds is the maximally generic word in Mauwake, \textstyleStyleVernacularWordsItalic{maa} `\textstyleFreeTranslationChar{thing'}\REF{ex:3:}.\footnote{The scope of meaning for \textstyleFootnoteBaseChar{\textit{maa}} is like that of `thing\textit{'} in its widest sense in English.}

\ea%x47
\label{ex:3:x47}
\gll mera nepa \\
fish bird\\
\glt`eagle ray'
\z

\ea%x48
\label{ex:3:x48}
\gll oon(a) tiretira \\
bone horizontal.cane (in roof structure)\\
\glt`rib'
\z

\ea%x49
\label{ex:3:x49}
\gll maa pela \\
thing leaf\\
\glt`(edible) greens'
\z

There are two compound types with nominalized verbs. When the nominalized verb follows the other noun, it behaves like an adjective and receives the phrase stress.

\ea%x1521
\label{ex:3:x1521}
\gll maa en-owa \\
thing/food eat-\textsc{nmz}\\
\glt`food'
\z

\ea%x1522
\label{ex:3:x1522}
\gll emer(a) ik-owa \\
sago roast-\textsc{nmz}\\
\glt`bread, roasted sago'
\z

A compound type where the nominalized verb precedes the other noun is more common than the one above. When the second part is a human noun, it usually has to be the \textstyleEmphasizedWords{\textsc{agent}} of the verb \REF{ex:3:}, but when the noun is non-human, it is harder to find a common denominator for the semantic relationships between the parts in different compounds. Quite often the meaning centers around function, purpose or ``typical'' action, place, time etc.

\ea%x52
\label{ex:3:x52}
\gll uuw-ow(a) mua \\
work-\textsc{nmz} man\\
\glt`worker'
\z

\ea%x53
\label{ex:3:x53}
\gll in-ow(a) koora \\
sleep-\textsc{nmz} house\\
\glt`bedroom'
\z

\ea%x54
\label{ex:3:x54}
\gll om-ow(a) eka \\
cry-\textsc{nmz} water\\
\glt`tear'
\z

This compound type particularly easily allows compounds with more than two roots: 

\ea%x55
\label{ex:3:x55}
\gll ikemik(a) kaik-ow(a) mua \\
wound tie-\textsc{nmz} man\\
\glt`doctor'
\z

\ea%x56
\label{ex:3:x56}
\gll emer(a) en-ow(a) mua \\
sago eat-\textsc{nmz} man\\
\glt`a Sepik man (lit: a sago eater)'\footnote{Sepik province is known for its main staple, sago starch.}
\z 

\todo{what happened to examples 57-59}

\ea%x60
\label{ex:3:x60}
\gll ama urup-ow(a) (epa/kame) \\
sun rise-\textsc{nmz} place/side\\
\glt`east'
\z

In the example \REF{ex:3:x61} the main noun \textstyleStyleVernacularWordsItalic{epa}/\textstyleStyleVernacularWordsItalic{kame} can be dropped, and this happens in some other compounds as well:

\ea%x61
\label{ex:3:x61}
\gll epir(a) suruk-ow(a) (tetelka) \\
plate wipe-\textsc{nmz} finger\\
\glt`forefinger' 
\z  

The \textstyleEmphasizedWords{\textsc{coordinate}} compounds are different from the other compounds in that neither of the parts modifies the other. The meaning of the whole is derived from the combined meaning of the two terms. Also, there is no stress centre: both parts of the compound are stressed equally. The number of these compounds is small.

\ea%x50
\label{ex:3:x50}
\gll emeria mua \\
woman man\\
\glt`people'
\z

\ea%x51
\label{ex:3:x51}
\gll muuka wiipa \\
son daughter\\
\glt`children'
\z

The \textstyleAcronymallcaps{N}+\textstyleAcronymallcaps{ADJ} compounds are as hard to distinguish from phrases as some of the other groups mentioned above. Again the uninterruptibility and lexicalized meaning are the main criteria. If the adjective \textstyleStyleVernacularWordsItalic{sepa} `black' is added between the two words in \REF{ex:3:}, the meaning changes into `a small black man'.

\ea%x57
\label{ex:3:x57}
\gll mua gelemuta \\
man small\\
\glt`a little boy'
\z

\ea%x58
\label{ex:3:x58}
\gll mia yoowa \\
body/skin hot\\
\glt`fever'
\z

\ea%x59
\label{ex:3:x59}
\gll maa samora \\
thing bad\\
\glt`mosquito'
\z

Compounding is a productive process in Mauwake, and it is the most common language-internal means used for adding new lexical items to the language. 

\subsection{Derived nouns}\label{sec:3:2:6}
%\hypertarget{RefHeading19381935131865}
{}
In this section I will discuss derivations where the \textstyleEmphasizedWords{\textsc{end result}} is a noun. There are only two of these: nouns made out of verbs, and noun reduplications. 

\subsubsection{Action nominals}\label{sec:3:z:y:x}
%\hypertarget{RefHeading19401935131865}
{}
The process of nominalizing verbs is a straightforward and fully productive process of adding the nominalizing suffix -\textstyleStyleVernacularWordsItalic{owa} to the verb stem. The nominalized verbs most commonly function as nouns, sometimes also as adjectives \REF{ex:3:}.\footnote{In the Mauwake dictionary some of these nominalized forms have their own entry as if they were fully lexicalized as nouns, but this is to some extent a concession to other languages, where separate nouns may be required for the action nominals and more lexicalized deverbal nouns (for the distinction, see \citealt[193]{Ylikoski2003}). In Mauwake it is often difficult to establish which of the nominalizations are lexicalized.}

\ea%x62
\label{ex:3:x62}
\gll uf-\textstyleEmphasizedVernacularWords{owa} \\
dance-\textsc{nmz}\\
\glt`(the act of) dancing', `(traditional) dance'
\z

\ea%x63
\label{ex:3:x63}
\gll irak-\textstyleEmphasizedVernacularWords{owa} \\
fight-\textsc{nmz}\\
\glt`fighting', `fight/war' 
\z 

\ea%x1231
\label{ex:3:x1231}
\gll Fiirim-\textstyleEmphasizedVernacularWords{owa}=pa opaimika aakun-e-mik. \\
gather-\textsc{nmz}=\textsc{loc} talk talk-\textsc{pa}-1/3p\\
\glt`In the meeting we talked.'
\z

\ea%x1247
\label{ex:3:x1247}
\gll Amina puk\textstyleEmphasizedVernacularWords{-owa} eliw(a) marewa=ke. \\
pot break-\textsc{nmz} good none=\textsc{cf}\\
\glt`The pot is broken (and) not good' or: `The broken pot is not good.'
\z

Action nominals function like any regular nouns in Mauwake. They can be, for example, a head \REF{ex:3:x64} or a qualifier \REF{ex:3:x65} in a \textstyleAcronymallcaps{NP}, and a first \REF{ex:3:x66} or last element \REF{ex:3:x67} in a compound noun.

\ea%x64
\label{ex:3:x64}
\gll Siowa \textstyleEmphasizedVernacularWords{alu-owa} miim-ap ekap-o-k. \\
dog make.noise-\textsc{nmz} hear-\textsc{ss}.\textsc{seq} come-\textsc{pa}-3s\\
\glt`He heard the dog's noise and came' or: `The dog heard noise and came.'
\z

\ea%x65
\label{ex:3:x65}
\gll Irak-owa kerer-owa epa weeser-em-ik-eya {\dots}{\footnotemark} \\
fight-\textsc{nmz} appear-\textsc{nmz} time finish-\textsc{ss}.\textsc{sim}-be-2/3s.\textsc{ds}\\
\glt`As the time of the war was getting close{\dots}' (Lit: `As the war-appearing time was coming to an end{\dots}')
\z 

\footnotetext{\textit{Kererowa} is both the head of \textit{irakowa kererowa} and part of the qualifier phrase in \textit{irakowa kererowa epa.} }

\ea%x66
\label{ex:3:x66}
\gll Oram \textstyleEmphasizedVernacularWords{niir-ow(a)} opora ma-e-m. \\
just laugh-\textsc{nmz} talk say-\textsc{pa}-1s\\
\glt`I just said it as a joke.'
\z

\ea%x67
\label{ex:3:x67}
\gll Kaul \textstyleEmphasizedVernacularWords{wafur-owa}  mera \textstyleEmphasizedVernacularWords{aaw-owa} eliw. \\
hook throw-\textsc{nmz} fish get-\textsc{nmz} all.right\\
\glt`As for throwing a hook, it is a good way of catching fish.' (Lit: `Hook-throwing is all right for fish-catching.') 
\z 

The following expressions form an interesting pair, as \REF{ex:3:x424} is a \textstyleAcronymallcaps{NP} with a nominalized verb as a head, and \REF{ex:3:x425} is a compound noun with a nominalized verb as the first part.

\ea%x424
\label{ex:3:x424}
\gll mua aakun-\textstyleEmphasizedVernacularWords{owa} \\
man talk-\textsc{nmz}\\
\glt`talk(ing) of man/people', `people's talk'
\z

\ea%x425
\label{ex:3:x425}
\gll aakun-\textstyleEmphasizedVernacularWords{ow}(\textstyleEmphasizedVernacularWords{a}) mua \\
talk-\textsc{nmz} man\\
\glt`a talker', `a spokesman'
\z

Action nominals keep their verb-like property of being able to take the same arguments and peripherals as the verb serving as the root of the noun. The result is a nominalized clause, which functions like a noun phrase. This is discussed further in \sectref{sec:5.7} and \sectref{sec:8.3.2}.

\citet[334--342]{ComrieEtAl2007} list various kinds of other nominalization possibilities,\footnote{\citet[500]{Givon1999} calls all of these \textit{lexical nominalizations}, and Ylikoski calls them \textit{deverbal nouns} \citeyear[193]{Ylikoski} to distinguish them from action nominals.} but in Mauwake the corresponding expressions are compound nouns or noun phrases consisting of the nominalized verb (or clause) plus another noun, rather than simple nominalizations. 

\ea%x1232
\label{ex:3:x1232}
\gll ikemika kaik-\textstyleEmphasizedVernacularWords{ow(a)} mua \\
wound tie-\textsc{nmz} man\\
\glt`doctor, nurse'
\z

\ea%x1233
\label{ex:3:x1233}
\gll maa eneka teek-\textstyleEmphasizedVernacularWords{ow(a)} (maa)\footnotemark{} \\
thing tooth open-\textsc{nmz} (thing)\\
\glt`can opener'
\z

\footnotetext{\textit{Maa eneka} is a compound referring to edible animals; the very generic noun \textit{maa} `thing' may be omitted from the end.}

\subsubsection{Noun reduplication}\label{sec:3:z:y:x}
%\hypertarget{RefHeading19421935131865}
{}
Reduplication of nouns to denote plurality is a very marginal process in Mauwake, whereas reduplication of verbs (\sectref{sec:3.8.2.4.1}) is much more frequent, and that of adjectives (\sectref{sec:3.3}) also more common. Usually the whole noun is reduplicated; final /a/ is deleted in the reduplicated part of words that are longer than two syllables \REF{ex:3:}. 

\ea%x68
\label{ex:3:x68}
\gll Dabuel \textstyleEmphasizedVernacularWords{poka-poka} nain=iw biiris on-am-ik-e-mik. \\
pawpaw \textsc{rdp}-trunk that1=\textsc{inst} bridge make-\textsc{ss}.\textsc{sim}-be-\textsc{pa}-1/3p\\
\glt`They kept making the bridge with pawpaw trunks.'
\z

\ea%x69
\label{ex:3:x69}
\gll Waaya pa-ep \textstyleEmphasizedVernacularWords{kio-kiowa} naap uup-e-mik. \\
pig butcher-\textsc{ss}.\textsc{seq} \textsc{rdp}-piece thus cook-\textsc{pa}-1/3p\\
\glt`We butchered the pig and cooked the pieces like that.'
\z

\ea%x426
\label{ex:3:x426}
\gll \textstyleEmphasizedVernacularWords{Owow-owowa} ikiw-e-mik. \\
\textsc{rdp}-village go-\textsc{pa}-1/3p\\
\glt`We went to several villages.'
\z

\section{Adjectives}\label{sec:3:3}
%\hypertarget{RefHeading19441935131865}
{}
The existence of noun and verb as universal categories is generally acknowledged, but the status of adjectives is less clear. There is considerable variation among languages as to what belongs to the adjective class, and sometimes a question is posed whether the class exists at all. But when there is a class of adjectives, the following tendencies emerge: languages that have a small class of adjectives show a lot of similarity in what kinds of concepts they express through this class; and similarly, in languages where the adjective class is large the semantic content of the class is fairly constant \citep[20]{Dixon1977}. Semantically it is somewhat of an in-between category sharing similarities with both nouns and verbs \citep[447]{Lyons1977}. Nouns ``connote the possession of a complex of qualities, and [adjectives] the possession of one single quality'' (\citealt[81]{Jespersen1924}; see also \citealt[362]{Wierzbicka1986}). Nouns have reference, adjectives do not \citep[77]{HakulinenEtAl1979}%Karlsson
. Instead of categorizing like nouns do, adjectives describe \citep[357]{Wierzbicka1986}. They may also code transitory states, and in \citegen[52]{Givon1984} time-stability scale they occupy the middle area between nouns and verbs.\footnote{But see \citegen{Thompson1988} criticism on Giv\'on's placing of adjectives on the time-stability scale.} 

The morphological and syntactic coding of ``property concepts'' reflects their semantically ambivalent status: especially in languages which have either no adjectives or only a small adjective class, the concepts are usually expressed via verbs and/or nouns, sometimes by other means \citep[20]{Dixon1977}.

The adjective class in Mauwake is a relatively small open class when compared with nouns and verbs. But compared with some other Papuan languages \citep[50--51]{Dixon1977} it is a fairly large class: the number of non-derived adjectives currently in the dictionary is about 80.\footnote{Usan also has a relatively large adjective inventory (\textstyleFootnoteBaseChar{Reesink 1987}:63).} The morphological and syntactic similarities and differences between nouns and adjectives were discussed above in \sectref{sec:3.2.2}. Adjectives do not inflect at all. 

A prototypical adjective functions as the head of an adjective phrase\footnote{Often the head is the sole constituent of the adjective phrase.} (\sectref{sec:4.2}) and may be modified by different intensity adverbs (\sectref{sec:3.9.2}), including the pre-modifier \textstyleStyleVernacularWordsItalic{lawisiw} `rather' \REF{ex:3:} and various post-modifiers \REF{ex:3:}. 

\ea%x70
\label{ex:3:x70}
\gll Nomokowa \textstyleEmphasizedVernacularWords{maala} war-e-k. \\
tree long cut-\textsc{pa}-3s\\
\glt`He cut a tall tree.'
\z

\ea%x71
\label{ex:3:x71}
\gll Waaya me \textstyleEmphasizedVernacularWords{maneka}, muuka, \textstyleEmphasizedVernacularWords{kia} \textstyleEmphasizedVernacularWords{gelemuta}. \\
pig not big son white small\\
\glt`The pig was not big, it was a piglet, white (and) small.'
\z

\ea%x72
\label{ex:3:x72}
\gll Malol \textstyleEmphasizedVernacularWords{lawisiw} \textstyleEmphasizedVernacularWords{yoowa}. \\
open.sea rather hard\\
\glt`(Fishing in the) open sea is rather hard.'
\z

\ea%x73
\label{ex:3:x73}
\gll Koora nain \textstyleEmphasizedVernacularWords{maneka} \textstyleEmphasizedVernacularWords{wenup}. \\
house that1 big very\\
\glt`That house is very big.'
\z

Only the following adjectives have been found to be non-scalar:

\begin{table}
\caption{Please provide a caption}
\label{}
\begin{tabular}{ll}
\mytoprule
morena &`male'\\
suwina &`female'\\
emi &`taboo(ed)'\\
enuma\footnote{\textit{Enuma} also means `new' and `green'.} &`alive'\\
\mybottomrule
\end{tabular}
\end{table}

The typical adjectives in Mauwake are all non-derived, and among them are all those listed by \citet[23]{Dixon1977} as the most common adjectives cross-linguistically: large, small, long, short, old, new, good, bad, black, white and red.

Of the various adjective groups mentioned by \citet{Dixon1977}, those of \textstyleEmphasizedWords{\textsc{age}} and value are quite small in Mauwake. Only two of the age adjectives are non-derived, the other two are derived:

\begin{table}
\caption{Please provide a caption}
\label{}
\begin{tabular}{llcll}
\mytoprule
awona &`old' && panewowa &`old'\\
enuma &`new' && iperowa &`middle-aged, elder'\\ 
\mybottomrule
\end{tabular}
\end{table}

 The adjective \textstyleStyleVernacularWordsItalic{awona} `old' refers to the age of things, not people; when used of people, the meaning is `previous' \REF{ex:3:}. Correspondingly, its antonym \textstyleStyleVernacularWordsItalic{enuma} `new' refers to age of things or recency in humans \REF{ex:3:}. The adjective referring to age in people, \textstyleStyleVernacularWordsItalic{panewowa} `old'\footnote{\textstyleFootnoteBaseChar{\textit{Panewowa}} is derived from the verb \textstyleFootnoteBaseChar{\textit{pan}}- `grow old'.} does not have any adjective as an antonym; the noun \textstyleStyleVernacularWordsItalic{takira} `youth' is used instead. \textstyleStyleVernacularWordsItalic{Panewowa} `old' and \textstyleStyleVernacularWordsItalic{iperowa} `middle-aged' do not indicate age only, but social status as well: it is the middle-aged men, rather than young or old, that have most power and make the important decisions in the community. \textstyleStyleVernacularWordsItalic{Iperowa} is also used for older siblings when the age of siblings is compared.

\ea%x74
\label{ex:3:x74}
\gll Emeria \textstyleEmphasizedVernacularWords{panewowa} nain Kait emeria \textstyleEmphasizedVernacularWords{awona}=ke. \\
woman old that1 Kait woman old=\textsc{cf}\\
\glt`The old woman is Kait's old (=previous) wife.'
\z

\ea%x75
\label{ex:3:x75}
\gll Ona mua \textstyleEmphasizedVernacularWords{enuma} iiriw pani-e-k. \\
3s.\textsc{gen} man new already grow.old-\textsc{pa}-3s\\
\glt`Her new husband is (already) old.'
\z

\textstyleEmphasizedWords{\textsc{Value}} adjectives are the following: 

\begin{table}
\caption{Please provide a caption}
\label{}
\begin{tabular}{llcll}
\mytoprule
eliwa &`good' &- &samora &`bad'\\
makena &`true' &- &wadola &`false'\\
emi &`taboo, forbidden'\\
\mybottomrule
\end{tabular}
\end{table}

\ea%x1760
\label{ex:3:x1760}
\gll Inasin opaimika \textstyleEmphasizedVernacularWords{eliwa} me yia maak-e-mik. \\
spirit talk good not 1p.\textsc{acc} tell-\textsc{pa}-1/3p\\
\glt`They did not speak good Tok Pisin (lit: spirit talk) to us.'
\z

\ea%x1759
\label{ex:3:x1759}
\gll Iiriw sira nain \textstyleEmphasizedVernacularWords{emi} maneka wiar ik-ua. \\
earlier custom that1 forbidden big 3.\textsc{dat} be-\textsc{pa}.3s\\
\glt`Earlier that custom was completely forbidden to them.'
\z

The list of \textstyleEmphasizedWords{\textsc{colour}} terms is also very limited; only the first three terms in the list are purely colour terms, all the others have their origin elsewhere:

\begin{table}
\caption{Please provide a caption}
\label{} 
\begin{tabular}{llcl}
\mytoprule
sepa &`black'&&\\
kia &`white'&&\\
oka &`red', `brown'&&\\
enuma &`green' &{\textless} &`new'\\
ligam &`yellow' &{\textless} &`turmeric'\\
ekapina &`blue' &{\textless} &`shrub sp. (used for blue dye)'\\
foma &`grey' &{\textless} &`ashes'\footnote{cf. \cite[4]{BerlinEtAl1969}.}\\
\mybottomrule
\end{tabular}

\end{table}

\ea%x1753
\label{ex:3:x1753}
\gll Aalbok mia \textstyleEmphasizedVernacularWords{sepa}  \textstyleEmphasizedVernacularWords{akena} kerer-e-k. \\
black.cuckoo.shrike body black very become-\textsc{pa}-3s\\
\glt`The body of the black cuckoo-shrike became very black.'
\z

\ea%x109
\label{ex:3:x109}
\gll Konima nain \textstyleEmphasizedVernacularWords{sepa} \textstyleEmphasizedVernacularWords{kia}. \\
cloth that1 black white\\
\glt`The cloth is black-and-white.'
\z

\ea%x1754
\label{ex:3:x1754}
\gll Mia afif(a) \textstyleEmphasizedVernacularWords{oka}, \textstyleEmphasizedVernacularWords{oka} gelemuta. \\
body hair red, red small\\
\glt`The feathers were red, (it was) red and small.'
\z

\ea%x1755
\label{ex:3:x1755}
\gll Komora nain \textstyleEmphasizedVernacularWords{kia} \textstyleEmphasizedVernacularWords{ne} \textstyleEmphasizedVernacularWords{maneka} wenup. \\
cuscus that1 white \textsc{add} big very\\
\glt`That cuscus is/was white and very big.'
\z

In \REF{ex:3:} the dimensional adjective for `small' may follow directly after the colour adjective, whereas the adjective \textstyleStyleVernacularWordsItalic{maneka} `big' needs a connective between the two adjectives in \REF{ex:3:}, because \textstyleStyleVernacularWordsItalic{maneka} is used as an intensifier when immediately following a colour term, and \textstyleStyleVernacularWordsItalic{kia maneka} would mean `completely white'.

The darkness of a colour is expressed through the adjectives \textstyleStyleVernacularWordsItalic{sepa} `black' and \textstyleStyleVernacularWordsItalic{kia} `white' used as modifiers of the main colour adjective \REF{ex:3:x110}.

\ea%x110
\label{ex:3:x110}
\gll ifa \textstyleEmphasizedVernacularWords{enuma} \textstyleEmphasizedVernacularWords{lawisiw} \textstyleEmphasizedVernacularWords{sepa} \\
leaf new/green rather black\\
\glt`a dark green leaf'
\z

Among the adjectives denoting \textstyleEmphasizedWords{\textsc{dimension}} there are a number of terms describing various kinds of thinness and thickness, as well as shortness. 

\begin{table}
\caption{Please provide a caption}
\label{} 
\begin{tabular}{llcll}
\mytoprule
maneka &`large' &- &gelemuta &`small'\\
maala &`long' &- &iiwa &`short'\\
kuruma &`thick' &- &gawela &`thin'\\
fula(kia) &`fat' &- &bebeta &`slim, skinny'\\
teena &`thin'&&&\\
komosia &`small, short'&&&\\
\mybottomrule
\end{tabular}
\end{table}

\ea%x1756
\label{ex:3:x1756}
\gll Epa dabela=pa mia suuw-owa \textstyleEmphasizedVernacularWords{gawela} suuw-ap mia fulil-i-nan.\\
place cold=\textsc{loc} body push-\textsc{nmz} thin push-\textsc{ss}.\textsc{seq} body feel.cold-Np-\textsc{fu}.2s\\
\glt`When you wear thin clothes (mia suuwowa) in a cold place you will feel cold.' 
\z

\ea%x76
\label{ex:3:x76}
\gll Owor(a) ara \textstyleEmphasizedVernacularWords{teena}  nain ku-i-non. \\
betelnut.palm trunk thin that1 break-Np-\textsc{fu}.3s\\
\glt`The thin betelnut palm trunk will break.'
\z

\ea%x77
\label{ex:3:x77}
\gll Epa dabel-al-eya mia suuw-owa \textstyleEmphasizedVernacularWords{kuruma}  wu-e. \\
place cold-\textsc{inch}-2/3s.\textsc{ds} body push-\textsc{nmz} thick put-\textsc{imp}.2s\\
\glt`When it gets cold, put thick clothes on.'
\z

The group of adjectives denoting \textstyleEmphasizedWords{\textsc{physical property}} is larger than any of the other groups and includes several antonym pairs. The list below is just a sample:

\begin{table}
\caption{Please provide a caption}
\label{} 
\begin{tabular}{llcll}
\mytoprule
yoowa &`hot, hard' &- &dabela &`cold'\\
supuka &`wet' &- &ififa &`dry'\\
pina &`heavy' &- &efefa &`light'\\
kaken &`straight' &- &meka &`crooked'\\
melina &`clear' &- &wiwisa &`murky'\\
anima &`sharp' &- &duduwa &`blunt'\\
dubila &`slippery, smooth'&&&\\
itita &`soft'&&&\\
masia &`bitter (taste)'&&&\\
siina &`tight'&&&\\
\mybottomrule
\end{tabular}
\end{table}

\ea%x78
\label{ex:3:x78}
\gll Iwera \textstyleEmphasizedVernacularWords{ififa} ora-eya fiirim-i-mik. \\
coconut dry descend-2/3s.\textsc{ds} gather-Np-\textsc{pr}.1/3p\\
\glt`When the dry coconuts drop we gather them'.
\z

\ea%x1758
\label{ex:3:x1758}
\gll {\dots}epia foma lawisiw \textstyleEmphasizedVernacularWords{yoowa} ik-ua. \\
fire(wood) ashes rather hot be-\textsc{pa}.3s\\
\glt`{\dots} the ashes were rather hot.'
\z

\textstyleEmphasizedWords{\textsc{Human propensity}} adjectives is the second largest group. 


\begin{table}
\caption{Please provide a caption}
\label{} 
\begin{tabular}{llcll}
\mytoprule
lebuma &`lazy' &- &topia &`diligent'\\
asia &`wild' &- &memela &`tame'\\
lebuma &`lazy'&&&\\
momora &`foolish'&&&\\
popora &`quiet'&&&\\
yamunsia &`stingy'&&&\\
\mybottomrule
\end{tabular}
\end{table}

\ea%x1757
\label{ex:3:x1757}
\gll Takira=ke keker op-ap \textstyleEmphasizedVernacularWords{popor(a)} maneka ik-e-mik. \\
boy=\textsc{cf} fear hold-\textsc{ss}.\textsc{seq} quiet big be-\textsc{pa}-1/3p\\
\glt`The boys were afraid and very quiet.'
\z

\ea%x1418
\label{ex:3:x1418}
\gll Mua \textstyleEmphasizedVernacularWords{lebuma} nain emeria me wi-i-mik. \\
man lazy that1 woman not give.them-Np-\textsc{pr}.1/3p\\
\glt`We do not give wives to lazy men.'
\z

Although Mauwake has a considerable inventory of adjectives for a Papuan language, in actual use they are rather infrequent.\footnote{Their frequency in the text material is about 1.5\% of all the words.} Especially physical property and human propensity are frequently expressed through verbs which have been verbalized from adjectives. A true adjective is a more likely candidate to indicate a stable or essential quality of the head noun \REF{ex:3:}, whereas the verbalized form is used for more temporary characteristics \REF{ex:3:}-\REF{ex:3:}.

\ea%x1419
\label{ex:3:x1419}
\gll Sama=pa or-owa nain eliw, nain ikoka or-op or-op or-op \textstyleEmphasizedVernacularWords{lebum(a)-ar-i-nan}, epasia akena.\\
stairs=\textsc{loc} descend-\textsc{nmz} that1 well that1 later descend-\textsc{ss}.\textsc{seq} descend-\textsc{ss}.\textsc{seq} descend-\textsc{ss}.\textsc{seq} lazy-\textsc{inch}-Np-\textsc{fu}.2s far very\\
\glt`Descending on the stairs is all right, but later when you have gone down and down and down you will be lazy/tired, (as) it is very far.'
\z

\ea%x79
\label{ex:3:x79}
\gll Moma \textstyleEmphasizedVernacularWords{kasu(a)-ar-eya}  me enim-i-mik. \\
taro hard-\textsc{inch}-2/3s.\textsc{ds} not eat-Np-\textsc{pr}.1/3p\\
\glt`We don't eat hard taro.' (Lit: `When taro is hardened, we don't eat it.')
\z

\ea%x80
\label{ex:3:x80}
\gll \textstyleEmphasizedVernacularWords{Yamunsi(a)-ar-iwkin} me wia nokar-e-m. \\
stingy-\textsc{inch}-2/3p.\textsc{ds} not 3p.\textsc{acc} ask-\textsc{pa}-1s\\
\glt`They were (being) stingy, (so) I didn't ask them.'
\z

\textsc{Speed} is expressed through adverbs or verbs rather than adjectives.

\textstyleEmphasizedWords{\textsc{Comparison}} of adjectives is an area where there is very little differentiation in many Papuan languages, including Mauwake.\footnote{See \citet[134--135]{Roberts1987}; \citet[68]{Reesink1987}; \citet[63--64]{Hardin2002}. \citet[268]{Haiman1980} reports only three or four true adjectives for Hua, and does not mention comparison.} Intensifiers are used for this function, as well as the verb \textstyleStyleVernacularWordsItalic{nomak-} `overcome, surpass'. 

\ea%x81
\label{ex:3:x81}
\gll Poka fain maala, nain \textstyleEmphasizedVernacularWords{nomak-e-k}, ne oko nain \textstyleEmphasizedVernacularWords{maala} \textstyleEmphasizedVernacularWords{akena}.\\
stilt this long that1 surpass-\textsc{pa}-3s \textsc{add} other that1 long very\\
\glt`This stilt is longer than that, and/but the other one is the longest (lit: very long).'
\z

Two adjectives can also be compared by contrasting them: 

\ea%x441
\label{ex:3:x441}
\gll Nomokow(a) kakawa fain \textstyleEmphasizedVernacularWords{iiwa}, oko \textstyleEmphasizedVernacularWords{maala} puuk-a-n. \\
tree part this short other long cut-\textsc{pa}-2s\\
\glt`You cut this plank shorter than the other one.' (Lit: `You cut this plank short, the other long.')
\z

Adjectives denoting size form a scale of three: \textstyleStyleVernacularWordsItalic{gelemuta} `small', \textstyleStyleVernacularWordsItalic{manisiri} `biggish', \textstyleStyleVernacularWordsItalic{maneka} `big'. Usually, if three degrees of comparison are needed, it is possible to express them periphrastically, but that is seldom necessary. Comparison as a functional domain is discussed in \sectref{sec:6.5}. 

Like nouns, adjectives can also be \textstyleEmphasizedWords{\textsc{reduplicated}} for plural (\sectref{sec:2.3.3.2}). Reduplication of adjectives is not very common, but it is more frequent than that of nouns. 

\ea%x85
\label{ex:3:x85}
\gll Maa eneka kes \textstyleEmphasizedVernacularWords{mane-maneka} oram iw-e-mik. \\
thing tooth case \textsc{rdp}-big just give.him-\textsc{pa}-1/3p\\
\glt`They just gave him big cases of meat tins.'
\z

The adjective \textstyleStyleVernacularWordsItalic{gelemuta} `small' has several reduplicated forms: \textstyleStyleVernacularWordsItalic{gelemuti-tik}\textstyleEmphasizedVernacularWords{,} \textstyleStyleVernacularWordsItalic{gelemutu-mut}\textstyleEmphasizedVernacularWords{,} \textstyleStyleVernacularWordsItalic{gele-gelemuti-tik}\textstyleEmphasizedVernacularWords{.} 

\ea%x486
\label{ex:3:x486}
\gll Waaya \textstyleEmphasizedVernacularWords{gelemutu-mut} pu-puuk-e. \\
pig small-\textsc{rdp} \textsc{rdp}-cut-\textsc{imp}.2s\\
\glt`Cut the pig into small pieces.'
\z

Occasionally reduplication can be used for an intensifying function as well. The noun modified by the reduplicated adjective in \REF{ex:3:x485} is either singular or plural, in \REF{ex:3:x86} it is definitely singular.

\ea%x485
\label{ex:3:x485}
\gll Biiris eliwa me on-a-mik, \textstyleEmphasizedVernacularWords{damo-damola}=ko. \\
bridge good not make-\textsc{pa}-1/3p \textsc{rdp}-bad=\textsc{nf}\\
\glt`They didn't make a good bridge (but) very bad.' (or: `{\dots}good bridges but bad.')
\z

\ea%x86
\label{ex:3:x86}
\gll {\dots ifa}=ke keraw-a-k, mamepaperuma \textstyleEmphasizedVernacularWords{gele-gelemuti-tik} nain=ke. \\
{\dots}snake=\textsc{cf} bite-\textsc{pa}-3s death.adder \textsc{rdp}-small-\textsc{rdp} that1=\textsc{cf}\\
\glt`{\dots} a snake bit him, a very small death adder.'
\z

\textstyleEmphasizedWords{\textsc{New adjectives}} are derived from verbs with the nominalizing suffix \nobreakdash-\textstyleStyleVernacularWordsItalic{owa}. This is not a very productive process.

\begin{table}
\caption{Please provide a caption}
\label{} 
\begin{tabular}{llcll}
\mytoprule
kekanowa &`strong' &{{\textless}} &kekan- &`be strong'\\
panewowa &`old' &{{\textless}} &pan- &`become old'\\
kainowa &`high (voice)' &{{\textless}} &kain- &`be high (voice)'\\
bolonowa &`slack' &{{\textless}} &bolon- &`be slack'\\
\mybottomrule
\end{tabular}
\end{table}


\ea%x1766
\label{ex:3:x1766}
\gll No mua samora, mua emin(a) \textstyleEmphasizedVernacularWords{kekan}\textstyleEmphasizedVernacularWords{-}\textstyleEmphasizedVernacularWords{owa} nefa na-i-kuan.\\
2s.\textsc{unm} man bad man occiput be.strong-\textsc{nmz} 2s.\textsc{acc} say-Np-\textsc{fu}.3p\\
\glt`They will call you a bad man, a pig-headed (lit: strong occiput) man.'
\z

\ea%x1765
\label{ex:3:x1765}
\gll Someka aw-i-ya nain iwakara \textstyleEmphasizedVernacularWords{kain-owa} maneka aw-i-ya.\\
song weave-Np-\textsc{pr}.3s that1 neck be.high-\textsc{nmz} big weave-Np-\textsc{pr}.3s\\
\glt`When (s)he sings, (s)he sings with a very high voice.'
\z

\ea%x1767
\label{ex:3:x1767}
\gll Makera \textstyleEmphasizedVernacularWords{saawirin}\textstyleEmphasizedVernacularWords{-}\textstyleEmphasizedVernacularWords{owa} kaik-a-m. \\
cane surround-\textsc{nmz} tie-\textsc{pa}-1s\\
\glt`I tied the cane round.'
\z

Adjectives can be made into verbs by zero verb formation (\sectref{sec:3.8.2.2.1}) or by the inchoative verbaliser \nobreakdash-\textit{ar} (\sectref{sec:3.8.2.2.2}).

\section{Quantifiers}\label{sec:3:4}
%\hypertarget{RefHeading19461935131865}
{}
Quantifiers are a small closed class of words. The group can be divided into numeral and non-numeral quantifiers. The reasons for treating them as a group of their own, separate from adjectives, are the following. Their position is after the adjectives in a \textstyleAcronymallcaps{NP}.\footnote{Actually it is the Quantifier Phrase that comes after the Adjective Phrase, but usually the phrases consist of only one word, a quantifier in the former and an adjective in the latter.} Some of the numerals consist of a phrase or even a clause, but they still function as a single unit. And semantically quantifiers are quite different from adjectives.

A quantifier is the only obligatory element in a quantifier phrase (\textstyleAcronymallcaps{QP,} \sectref{sec:4.3}). These are used as post-modifiers in a \textstyleAcronymallcaps{NP}, where their position is between an adjective phrase (\textstyleAcronymallcaps{AP}) and demonstrative \REF{ex:3:x87}, or by themselves as a non-verbal predicate \REF{ex:3:x442}. 

\ea%x87
\label{ex:3:x87}
\gll I koora maneka \textstyleEmphasizedVernacularWords{kuisow} nain yiar aw-o-k. \\
1p.\textsc{unm} house big one that1 1p.\textsc{dat} burn-\textsc{pa}-3s\\
\glt`That one big house of ours burned.'
\z

\ea%x442
\label{ex:3:x442}
\gll Mua iperowa \textstyleEmphasizedVernacularWords{arow} \textstyleEmphasizedVernacularWords{muutiw.} \\
man middle-aged three only.\\
\glt`There are/were only three middle-aged men.' (Lit: `The middle-aged men (are) only three.') 
\z

The numerals, especially \textit{erup} `two', may be added to a pronoun to quantify it: the numeral occurs following a reflexive (or occasionally unmarked) form of the pronoun, but the pronoun is used like an unmarked pronoun.

\ea%x89
\label{ex:3:x89}
\gll Ne \textstyleEmphasizedVernacularWords{wiam} \textstyleEmphasizedVernacularWords{erup} pun epa neeke or-o-mik. \\
\textsc{add} 3p.\textsc{refl} two too place there.\textsc{cf} descend-\textsc{pa}-1/3p\\
\glt`And the two of them too went down there.'
\z

\subsection{Numerals}\label{sec:3:y:x}
%\hypertarget{RefHeading19481935131865}
{}
The traditional counting system in Mauwake is quinary, i.e. based on five\footnote{In New Guinea languages, there are counting systems based on two, five ten and twenty, as well as systems that use different body parts as tallies. All of these systems are present in the Madang area languages as well \citep{Lean1991}.}, and counting is gestured using the fingers.\footnote{To count, the fingers are bent down one by one, starting from the little finger of the right hand, and proceeding towards the thumb, then on to the little finger of the other hand etc.} 

\begin{table}
\caption{Please provide a caption}
\label{} 
\begin{tabular}{ll}
\mytoprule
kuisow &`one'\\
erup &`two'\\
arow &`three'\\
erepam &`four'\\
ikur / wapen inawiya &`five' / `a hand sleeps'\\
(ikur) okai(wi)=pa kuisow &`six' (lit: `(five) one on/from the other side')\\
(ikur) okai(wi)=pa erup &`seven'\\
(ikur) okai(wi)=pa arow &`eight'\\
(ikur) okai(wi)=pa erepam &`nine'\\
iimeka kuisow / okaipa okaipa inek &`ten' / `both sides sleep'\\
\mybottomrule
\end{tabular}
\end{table}


\ea%x90
\label{ex:3:x90}
\gll Uura ama \textstyleEmphasizedVernacularWords{ikur} \textstyleEmphasizedVernacularWords{okai}(\textstyleEmphasizedVernacularWords{wi})\textstyleEmphasizedVernacularWords{=pa} \textstyleEmphasizedVernacularWords{arow} naap in-e-mik. \\
night sun five other.side=\textsc{loc} three thus sleep-\textsc{pa}-1/3p\\
\glt`In the evening we slept at around eight o'clock.'
\z

Nowadays the borrowed Tok Pisin numerals have largely superseded the vernacular numerals, especially those indicating numbers ten and above \REF{ex:3:x91}. There are no terms for `hundred', `thousand' or bigger numbers in the vernacular system.

\ea%x91
\label{ex:3:x91}
\gll Mokoma \textstyleEmphasizedVernacularWords{ten} \textstyleEmphasizedVernacularWords{arow} aaw-o-k. \\
year ten three get-\textsc{pa}-3s\\
\glt`He became 30 years old.'
\z

Numerals can be modified with the intensity adverbs \textstyleStyleVernacularWordsItalic{kakeniw} `\textstyleFreeTranslationChar{correctly, exactly}', \textstyleStyleVernacularWordsItalic{akena} `really, truly' or \textstyleStyleVernacularWordsItalic{muutiw} `only'.

\ea%x443
\label{ex:3:x443}
\gll \textstyleEmphasizedVernacularWords{Erepam} \textstyleEmphasizedVernacularWords{kaken}\textstyleEmphasizedVernacularWords{=iw} mik-a-mik. \\
four straight-\textsc{isol} spear-\textsc{pa}-1/3p\\
\glt`We speared exactly four.'
\z

\ea%x661
\label{ex:3:x661}
\gll Mua \textstyleEmphasizedVernacularWords{arow} \textstyleEmphasizedVernacularWords{akena} epa nain iimar-e-mik. \\
man three truly place that stand.up-\textsc{pa}-1/3p\\
\glt`Exactly three men stood at that place.'
\z

When the number is somewhat uncertain and the disjunctive connective \textstyleStyleVernacularWordsItalic{e} `\textstyleFreeTranslationChar{or}' and/or the question marker -\textstyleStyleVernacularWordsItalic{i} is used, either the smaller or the bigger number may be mentioned first.

\ea%x1416
\label{ex:3:x1416}
\gll Waaya maneka wiowa \textstyleEmphasizedVernacularWords{erup=i} \textstyleEmphasizedVernacularWords{e} \textstyleEmphasizedVernacularWords{arow} naap mik-iwkin um-i-ya.\\
pig big spear two=\textsc{qm} or three thus hit-2/3p.\textsc{ds} die-Np-\textsc{pr}.3s\\
\glt`When a big pig is hit with two or three spears it dies.'
\z

\ea%x92
\label{ex:3:x92}
\gll Mua wiam \textstyleEmphasizedVernacularWords{ikur=i} \textstyleEmphasizedVernacularWords{erepam} naap wia aaw-e-mik. \\
man 3p.\textsc{refl} five=\textsc{qm} four thus 3p.\textsc{acc} get-\textsc{pa}-1/3p\\
\glt`They took/got those four or five men.'
\z

Repetition \REF{ex:3:x93} or reduplication \REF{ex:3:x94} of the numerals indicates manner: `so and so many \textstyleEmphasizedWords{\textsc{at a time}}'. The reduplicated form of \textstyleStyleVernacularWordsItalic{kuisow} `\textstyleFreeTranslationChar{one}', \textstyleStyleVernacularWordsItalic{kui-kuisow}, has two meanings: `one by one' and `a few'. 

\ea%x93
\label{ex:3:x93}
\gll Naap \textstyleEmphasizedVernacularWords{kuisow} \textstyleEmphasizedVernacularWords{kuisow} aaw-ikiw-e-mik. \\
thus one one get-go-\textsc{pa}-1/3p\\
\glt`They kept getting them one at a time as they went.
\z

\ea%x94
\label{ex:3:x94}
\gll Waaya merena \textstyleEmphasizedVernacularWords{ere-erup} kaik-ap{\dots} \\
pig leg \textsc{rdp}-two tie-\textsc{ss}.\textsc{seq}\\
\glt`I tied the pig's legs two and two together and {\dots}'
\z

Money is counted using different nouns indicating certain amounts:

\begin{table}
\caption{Please provide a caption}
\label{} 
\begin{tabular}{ll}
\mytoprule
maamuma ({\textless} maa mumua) &'10 toea', also generic `money', lit: `seed'\\
fuluwa &`1 kina', lit: `hole' (the coin has a hole)\\
ifa &`2 kina', lit: `leaf'\\
ifa oka &`5 kina', lit: `red leaf'\\
kuuma &`10 kina', lit: `stick'\footnote{From a stick of tobacco, used for payment in the colonial days.} \\
\mybottomrule
\end{tabular}
\end{table}


\ea%x97
\label{ex:3:x97}
\gll \textstyleEmphasizedVernacularWords{Kuuma} \textstyleEmphasizedVernacularWords{kuisow} \textstyleEmphasizedVernacularWords{ifa} \textstyleEmphasizedVernacularWords{erup} naap yia sesenar-e-mik. \\
stick one leaf two thus 1p.\textsc{acc} buy-\textsc{pa}-1/3p\\
\glt`They paid to us (lit: bought us for) 14 kina.' 
\z

Mauwake has no separate words for \textstyleEmphasizedWords{ordinal} numbers. To indicate numerical order, various structures are employed. In many cases the cardinal numbers can be used:

\ea%x96
\label{ex:3:x96}
\gll Mua arow epa nain iimar-e-mik, yos=ke \textstyleEmphasizedVernacularWords{erepam}. \\
man three place that1 stand-\textsc{pa}-1/3p 1s.\textsc{fc}=\textsc{cf} four\\
\glt`Three men were standing there, and I was the fourth.'
\z

\ea%x428
\label{ex:3:x428}
\gll Koora tuun-e: \textstyleEmphasizedVernacularWords{kuisow} \textstyleEmphasizedVernacularWords{iki}(\textstyleEmphasizedVernacularWords{w})\textstyleEmphasizedVernacularWords{-}(\textstyleEmphasizedVernacularWords{e})\textstyleEmphasizedVernacularWords{p} \textstyleEmphasizedVernacularWords{erepam}, ne \textstyleEmphasizedVernacularWords{oko}  nain ona koora.\\
house count-\textsc{imp}.2s one go-\textsc{ss}.\textsc{seq} four \textsc{add} other that1 3s.\textsc{gen} house\\
\glt`His house is the fifth one' (Lit: `Count the houses: one to four, and the other/next is his house.')
\z

In the case of time units, cardinal numbers are combined with the verb \textstyleStyleVernacularWordsItalic{ikiw}- `go':

\ea%x427
\label{ex:3:x427}
\gll \textstyleEmphasizedVernacularWords{Fofa} \textstyleEmphasizedVernacularWords{okai}(\textstyleEmphasizedVernacularWords{wi})\textstyleEmphasizedVernacularWords{=pa} \textstyleEmphasizedVernacularWords{arow} \textstyleEmphasizedVernacularWords{ikiw-eya} ekap-i-non. \\
day other.side=\textsc{loc} three go-2/3s.\textsc{ds} come-Np-\textsc{fu}.3s\\
\glt`He will come on the ninth day.' (Lit: `When eight days have gone he will come').
\z

Order can also be indicated through verbs like \textstyleStyleVernacularWordsItalic{murar}- and \textstyleStyleVernacularWordsItalic{ook}- `follow'. 

\ea%x98
\label{ex:3:x98}
\gll Wi Ulingan=ke nomak-e-mik. Ne i Moro \textstyleEmphasizedVernacularWords{murar-e-mik}.\\
3p.\textsc{unm} Ulingan=\textsc{cf} win-\textsc{pa}-1/3p \textsc{add} 1p.\textsc{unm} Moro follow-\textsc{pa}-1/3p\\
\glt`The Ulingan people/team won. And (we from) Moro came second.'
\z

Numbers are \textstyleEmphasizedWords{not} used when listing one's children. The terms \textstyleStyleVernacularWordsItalic{iperowa} `firstborn', \textstyleStyleVernacularWordsItalic{ookap onarowa} `following' (used repeatedly, if necessary) and \textstyleStyleVernacularWordsItalic{kapa} `lastborn' are employed for that.

\subsection{Non-numeral quantifiers}\label{sec:3:y:x}
%\hypertarget{RefHeading19501935131865}
{}
Some non-numeral quantifiers can only be used with either count or mass nouns, others occur with both. Those that can be used with both are:

\begin{table}
\caption{Please provide a caption}
\label{} 
\begin{tabular}{ll}
\mytoprule
senam &`too much/too many'\\
unowiya &`all' (from: \textstyleStyleVernacularWordsxiiptItalic{unowa} `\textstyleFreeTranslationChar{many}' plus comitative clitic =\textstyleStyleVernacularWordsxiiptItalic{iya})\\
iiwawun &`all/altogether'\\
\mybottomrule
\end{tabular}
\end{table}


\ea%x665
\label{ex:3:x665}
\gll Moma \textstyleEmphasizedVernacularWords{senam} en-e-mik. \\
taro too.much eat-\textsc{pa}-1/3p\\
\glt`We ate too much taro.'
\z

\ea%x666
\label{ex:3:x666}
\gll Nomokowa \textstyleEmphasizedVernacularWords{senam} war-e-man. \\
tree too.many cut.\textsc{pa}-2p\\
\glt`You cut too many trees.'
\z

\ea%x99
\label{ex:3:x99}
\gll Yagin eka=pa \textstyleEmphasizedVernacularWords{unow=iya} nan yaki-e-mik. \\
Yagin water=\textsc{loc} many=\textsc{com} there bathe-\textsc{pa}-1/3p\\
\glt`We all bathed there at Yagin together.'
\z

The following are only used with \textstyleEmphasizedWords{\textsc{count}} nouns:

\begin{table}
\caption{Please provide a caption}
\label{} 
\begin{tabular}{ll}
\mytoprule
papako\footnote{\textit{Papako} is actually a plural indefinite `other'(\sectref{sec:3.7.2}), but it has a secondary function as a quantifier.} &`other/\textstyleFreeTranslationChar{some/a few}'\\
unowa &`\textstyleFreeTranslationChar{many}'\\
unow onaiya &`all' (from \textstyleStyleVernacularWordsItalic{unowa} plus \textstyleStyleVernacularWordsItalic{onaiya} `together with')\\
wenup &`\textstyleFreeTranslationChar{lots of}'\\
\mybottomrule
\end{tabular}
\end{table}


\ea%x100
\label{ex:3:x100}
\gll Mua \textstyleEmphasizedVernacularWords{unowa}, emeria \textstyleEmphasizedVernacularWords{papako} um-e-mik. \\
man many woman some die-\textsc{pa}-1/3p\\
\glt`Many men and some women died.'
\z

\ea%x667
\label{ex:3:x667}
\gll Ipia saana=pa iina \textstyleEmphasizedVernacularWords{wenup}. \\
rain season=\textsc{loc} mosquito lots.of\\
\glt`In the rainy season there are lots of mosquitoes.'
\z

Both \textstyleStyleVernacularWordsItalic{wenup} and \textstyleStyleVernacularWordsItalic{unowa} can be intensified with \textstyleStyleVernacularWordsItalic{akena} `very'; \textstyleStyleVernacularWordsItalic{unowa} may also be intensified with \textstyleStyleVernacularWordsItalic{wenup} `\textstyleFreeTranslationChar{very}' \REF{ex:3:}; or with \textstyleStyleVernacularWordsItalic{maneka} (lit: `\textstyleFreeTranslationChar{big}') that gives it the meaning `\textstyleFreeTranslationChar{all}' \REF{ex:3:}.

\ea%x809
\label{ex:3:x809}
\gll Siipepe kokora maroka \textstyleEmphasizedVernacularWords{wenup} \textstyleEmphasizedVernacularWords{akena} ika-i-ya. \\
Siipepe riverbed prawn lots.of very be-Np-\textsc{pr}.3s\\
\glt`There are lots of prawns in the Siipepe riverbed.'
\z

\ea%x101
\label{ex:3:x101}
\gll Iinan aasa nepa saarik, \textstyleEmphasizedVernacularWords{unow(a)} \textstyleEmphasizedVernacularWords{akena/wenup}. \\
sky canoe bird like many very\\
\glt`The planes were like birds, very many.'
\z

\ea%x102
\label{ex:3:x102}
\gll Emeria \textstyleEmphasizedVernacularWords{unow}(\textstyleEmphasizedVernacularWords{a}) \textstyleEmphasizedVernacularWords{maneka} sosora bee-beela a-e-mik. \\
woman many big grass.skirt \textsc{rdp}-rotten tie-\textsc{pa}-1/3p\\
\glt`All the women put on rotten grass skirts.'
\z

The negation of the universal quantifier is discussed below in \sectref{sec:6.2.2}.

The following quantifiers only occur with \textstyleEmphasizedWords{\textsc{mass}} nouns:

\begin{table}
\caption{Please provide a caption}
\label{} 
\begin{tabular}{ll}
\mytoprule
maneka &`\textstyleFreeTranslationChar{a lot/much}' (lit: `big')\\
gelemuta &`\textstyleFreeTranslationChar{little}'\\
lawiliw &`\textstyleFreeTranslationChar{somewhat/a little}'\\
\mybottomrule
\end{tabular}
\end{table}


\ea%x103
\label{ex:3:x103}
\gll Eka yoowa=pa aaya \textstyleEmphasizedVernacularWords{maneka/gelemuta} wu-e. \\
water hot=\textsc{loc} sugar big/little put-\textsc{imp}.2s\\
\glt`Put a lot of/a little sugar in the tea.'
\z

The following non-numeral quantifiers also function as degree/intensity adverbs, modifying a verb: \textstyleStyleVernacularWordsItalic{iiwawun}, \textstyleStyleVernacularWordsItalic{lawiliw}, \textstyleStyleVernacularWordsItalic{senam} and \textstyleStyleVernacularWordsItalic{wenup} (\sectref{sec:3.9.2}).

\ea
\gll Yos=ke \textstyleEmphasizedVernacularWords{lawiliw} asip-i-yem.\\
1s=\textsc{cf} somewhat help-Np-\textsc{pr}.1s\\
\glt`I am helping her somewhat/a little.'
\z

\ea%x512
\label{ex:3:x512}
\gll Iperowa=ke \textstyleEmphasizedVernacularWords{senam} kekan-e-mik. \\
middle.aged=\textsc{cf} too.much be.strong-\textsc{pa}-1/3p\\
\glt`The middle-aged men were very strong (in their opinion).'
\z

\ea%x513
\label{ex:3:x513}
\gll Waaya mik-amkun \textstyleEmphasizedVernacularWords{iiwawun} um-o-k. \\
pig spear-1s/p.\textsc{ds} altogether die-\textsc{pa}-3s\\
\glt`When I speared the pig it died completely.'
\z

\textstyleEmphasizedWords{\textsc{Fractions}} are hard to express in Mauwake.\footnote{I have not seen fractions treated in grammars of Papuan languages, but know from discussions with colleagues that translating fractions is a major problem not only in Mauwake but in other Papuan languages as well.} The noun \textstyleStyleVernacularWordsItalic{enakiwa} `half' is sometimes also used for unspecified `part', and \textstyleStyleVernacularWordsItalic{okaiwi} `one/other side' can be used for `half', when a clearly bounded entity is divided in half \REF{ex:3:x104}. I have not found other terms indicating fractions. Longer expressions are needed for them e.g. `divide into ten parts and take one part'.

\ea%x104
\label{ex:3:x104}
\gll Yabuela \textstyleEmphasizedVernacularWords{okaiwi} enak-e. \\
pawpaw one.side feed.me-\textsc{imp}.2s\\
\glt`Give me half of the pawpaw to eat.'
\z

\section{Pronouns}\label{sec:3:5}\footnotemark{}
%\hypertarget{RefHeading19521935131865}
{}
\footnotetext{Most of the material in this section has been published in my earlier paper \citep{Jarvinen1991}.}
\subsection{Introduction}\label{sec:3:y:x}
%\hypertarget{RefHeading19541935131865}
{}
Pronouns are a closed class of words. According to traditional grammar, pronouns can substitute for nouns, but actually they substitute for full noun phrases. 

Pronouns in Mauwake only include personal pronouns. Demonstratives, which are like pronouns in some respects, are discussed under deictics (\sectref{sec:3.6.2}). The indefinites, which are used as modifiers in a noun phrase, are closely related to question words and are treated in \sectref{sec:3.7.2}.

In principle all the pronouns in Mauwake are used for humans only. In legends also spirits can be referred to by these pronouns since they sometimes act like humans and can take human form. There is no third person singular pronoun for non-humans. 

\citet{Wurm1982} posited three typological sets of personal pronouns for Papuan languages, and mentioned Madang province as an area where set III is particularly widespread. The basic forms of Wurm's set III pronouns are:

\begin{table}
\caption{Please provide a caption}
\label{} 
\begin{tabular}{lll}
\mytoprule
 &singular &plural\\
\midrule
1 &da\~{ta\~{}ya} &ki\~{ti}\\
2 &na &nik\\
3 &nu\footnote{The third person plural form is not included in Wurm's typology because of gaps in the material and greater variability than in the other person forms.} &\citep[40--42]{Wurm1982}\\
\mybottomrule
\end{tabular}
\end{table}


In all the three pronoun sets fronting of vowels often goes together with plurality (ibid. 78), the non-singular forms in Papuan languages being derived from the singular forms \citep[361]{Franklin1979}. 

With more data and after more rigorous and detailed work on the \textstyleAcronymallcaps{TNG} pronouns, \citet[5]{Ross1995} gives the following as reconstructions of Proto Madang and Proto Croisilles free pronouns:

\begin{table}
\begin{tabular}{lcccccc}
\mytoprule
 & 1s & 2s & 3s & 1p & 2p & 3p\\ 
\midrule
\midrule
\parbox{1.1cm}{Proto Madang} & *ya & *na & *ua/*nu & *i- & *ni-/*ta- & {\dots}\\ 
\\
\parbox{1.1cm}{Proto Croisilles} & *ya & *na/*ni & *ua/*nu & *i[ge]/*i[na] & *ni[ge] & *ua[ge]/*ua[na]\\
\mybottomrule
\mybottomrule
\end{tabular}
\caption{Proto Madang and Proto Croisilles free pronouns}
\label{tab:8}
\end{table}

For different functions in the clause Papuan languages often have one or two classes, or functional sets, of pronouns with or without prepositions or suffixes to mark the appropriate cases. Amele \citep{Roberts1987}, Maia \citep[71]{Hardin2002}, Hua \citep[215]{Haiman1980}, Waskia \citep[53]{RossEtAl1978}%Paol
 and Bargam \citep[29]{Hepner2002} have only one basic set each, to which postpositions or suffixes are added. Usan \citep{Reesink1987} and Siroi \citep{Wells1979} each have a nominative and a possessive set. Most Finisterre-Huon languages have different sets for regular and emphatic pronouns \citep{McElhanon1973}.

Person is the more basic category than number in the pronoun systems of Papuan languages \citep[69]{Foley1986}. As for number, it is most common just to have a two-way distinction between singular and plural, but dual forms are also quite widespread in \textstyleAcronymallcaps{TNG} languages, and trial forms are found in some areas as well. An inclusive-exclusive distinction in the first person plural form is not common \citep[60]{Wurm1982} like it is in Austronesian languages, but according to \citet[56]{Ross2005} it has probably been an areal feature for a long time, even before the Austronesians arrived. 

Morphological resemblance between free pronouns and some verbal affixes, most commonly subject markers, is fairly widespread in Papuan languages \citep{Franklin1979}. It is not unusual to find that verbal affixes, e.g. object markers, make fewer person/number distinctions than free pronouns \citep[67]{Foley1986}.

In the following respects Mauwake manifests general typological features of \textstyleAcronymallcaps{TNG} Madang pronouns. There is no gender or noun class system that would be indicated through concord and marking with nouns and/or pronouns. Also, the morphology is suffixal rather than prefixal. There is no inclusive-exclusive distinction. Possession is marked through suffixation on the personal pronouns \citep[40--42]{Wurm1982}. 

The basic unmarked pronouns in Mauwake reflect the Proto Croisilles forms rather closely, apart from the third person plural form \textstyleStyleVernacularWordsItalic{wi}, which \citet[23]{Ross1996} mentions as an innovation *\textstyleStyleVernacularWordsItalic{u-i}- shared by the Kumil languages and the neighbouring Kaukombar languages. The ending -\textstyleStyleVernacularWordsItalic{fa} in the first and second person singular accusative pronouns is an innovation in the Kumil languages only. 

Some features in the Mauwake pronoun system not typical of Papuan languages are the existence of dative pronouns and also their use as possessives, and the distinction between the unmarked pronouns and the focal pronouns. 

The personal pronoun system in Mauwake is very regular, including the first, second and third persons both in singular and plural. Normally the plural form can also be used for dual; the dual number is only marked in one group, and there by adding a numeral rather than through affixation (\sectref{sec:3.5.8}). Since dual number does not occur in verb person marking either, apart from the first person imperative form, it is not very significant in the category of number. Spatial deixis is not marked in the personal pronoun system in Mauwake. The case is marked to some extent. \tabref{tab:9} lists the personal pronouns in Mauwake:

\begin{table}  
\begin{tabular*}{.7\textwidth}{@{\extracolsep{\fill}}lllllll}
\mytoprule
 & \multicolumn{2}{c}{{Free}}
 & \textsc{acc} & \textsc{gen} & \textsc{dat} & \textsc{isol} \\
& \textsc{unm} & {Focal}  \\
\midrule
1s & yo & yo-s & efa & y-ena & efa-r & ya-isow        \\
2s & no & no-s & nefa & n-ena & nefa-r & na-isow      \\
3s & (w)o & (w)o-s & {\O} & o-na & wi-ar & wa-isow    \\
1p & (y)i & (y)i-s & yia & yi-ena & yi-ar & (y)i-isow \\
2p & ni & ni-s & nia & ni-ena & ni-ar & ni-isow       \\
3p & wi & wi-s & wia & wi-ena & wi-ar & wi-isow       \\
\mybottomrule
\end{tabular*}

\begin{tabular*}{.7\textwidth}{@{\extracolsep{\fill}}llll}
\mytoprule
& \textsc{restr} & \textsc{refl} & \textsc{com}\\
\midrule
 1s & yena-iw/yos-iw & y-ame   & efa-m-iya\\
 2s & nena-iw/nos-iw & n-ame & nefa-m-iya\\
 3s & ona-iw/os-iw & w-ame & wama-iya\\
 1p & yien-iw/is-iw & yi-am & yiam-iya\\
 2p & nien-iw/nis-iw & ni-am & niam-iya\\
 3p & wien-iw/wis-iw & wi-am & wiam-iya\\
\mybottomrule
\end{tabular*}

\caption{Personal pronouns}
\label{tab:9}
\end{table}

Mauwake is a so-called pro-drop language, and a complete sentence can consist of a verb alone. The person of the subject is marked fully in the final verbs and partially in the medial verbs, so that besides the pragmatic clues there are also grammatical means for tracing the participants. But the pronouns are not completely optional: their use is rather strictly dictated by textual factors.

It is a fairly common feature in languages that pronouns can either modify a noun in a \textstyleAcronymallcaps{NP} or replace a full \textstyleAcronymallcaps{NP}, but cannot be the head of a \textstyleAcronymallcaps{NP} taking modifiers \citep[e.g.][]{ HakulinenEtAl1979,Saari1985,Roberts1987}. In Mauwake the personal pronouns usually occur without modifiers, but they \textstyleEmphasizedWords{\textsc{can}} be modified by a demonstrative, provided there is no collocational clash between the demonstrative and the personal pronoun. 

\ea%x530
\label{ex:3:x530}
\gll \textstyleEmphasizedVernacularWords{Ni} \textstyleEmphasizedVernacularWords{fain=ke} ekap-eka! \\
2s.\textsc{unm} this=\textsc{cf} come-\textsc{imp}.2p\\
\glt`You here (or: This group of you), come!'
\z

\ea%x531
\label{ex:3:x531}
\gll \textstyleEmphasizedVernacularWords{O} \textstyleEmphasizedVernacularWords{nain} fan me ik-ua. \\
3s.\textsc{unm} that1 here not be-\textsc{pa}.3s\\
\glt`He is not here.'
\z

A pronoun copy after a full \textstyleAcronymallcaps{NP} is hardly ever used in Mauwake for the subject. The rare example \REF{ex:3:x683} is from a hortatory text and may show rhetoric style:

\ea%x683
\label{ex:3:x683}
\gll Maneka fain [wie \textstyleEmphasizedVernacularWords{wi}] eliw wiar op-i-kuan. \\
big this uncle 3p.\textsc{unm} well 3.\textsc{dat} grab-NpFU.3p\\
\glt`These big ones the uncles may take from her.'
\z

The example \REF{ex:3:x532} is not a case of a genuine pronoun copy, since the genitive pronoun \textstyleStyleVernacularWordsItalic{wiena} adds the emphasizing meaning `themselves':

\ea%x532
\label{ex:3:x532}
\gll \textstyleEmphasizedVernacularWords{Wi}  iperowa \textstyleEmphasizedVernacularWords{wi-ena} ekap-e-mik. \\
3p.\textsc{unm} middle.aged 3p-\textsc{gen} come-\textsc{pa}-1/3p\\
\glt`The middle-aged (people) themselves came.' 
\z

For a pronoun copy of the genitive in a possessive \textstyleAcronymallcaps{NP}, see \sectref{sec:4.1.1}.

Pronouns as deictic elements are discussed in \sectref{sec:6.3.1}.

\subsection{Free pronouns}\label{sec:3:y:x}
%\hypertarget{RefHeading19561935131865}
{}
There are two sets of free pronouns: the unmarked pronouns, and the slightly longer focal pronouns.

\subsubsection{Unmarked pronouns}\label{sec:3:z:y:x}
%\hypertarget{RefHeading19581935131865}
{}

\begin{table}[h]
 \caption{Unmarked pronouns.}
\label{tab:3:unmarkedpronouns}

\begin{tabular}{lll}
\mytoprule
 &singular &plural\\
\midrule
1 &yo &(y)i\\
2 &no &ni\\
3 &(w)o &wi\\
\mybottomrule 
\end{tabular}
\end{table}

The unmarked pronouns are as given in \tabref{tab:3:unmarkedpronouns}.
The main use of the unmarked pronouns is as subjects. In narratives only the person marking on the verb, rather than a pronoun, is used for an established, continuing subject/topic (\sectref{sec:9.2.2}). Especially third person unmarked pronouns marking the subject are quite rare in narrative texts; first person pronouns are relatively much more common \citep[79--80]{Jarvinen1991}. 

%  \stepcounter{nx}{\thenx}x533) 
\ea 
\gll Irak-owa=ke kerer-eya \textstyleEmphasizedVernacularWords{wi}  puk-omak-e-mik.\\
fight-\textsc{nmz}=\textsc{cf} appear-2/3p.\textsc{ds} 3p.\textsc{unm} disperse-\textsc{distr}.\textsc{pl}-\textsc{pa}-1/3p\\
\glt`When the fight started they (many) dispersed.'
\z

\ea%x534
\label{ex:3:x534}
\gll \textstyleEmphasizedVernacularWords{O} koora=pa naap ik-ok um-o-k. \\
3s.\textsc{unm} house=\textsc{loc} thus be-SS die-\textsc{pa}-3s\\
\glt`She was like that in the house and died.'
\z

\ea%x1803
\label{ex:3:x1803}
\gll Bogia=pa nan wu-ap \textstyleEmphasizedVernacularWords{i} kiiriw ekap-e-mik. \\
Bogia=\textsc{loc} there put-\textsc{ss}.\textsc{seq} 1p.\textsc{unm} again come-\textsc{pa}-1/3p\\
\glt`We buried his body (lit: put it) there in Bogia and came (back) again.'
\z

However, with imperative verbs the subject pronoun is common (\sectref{sec:3.5.11}, \sectref{sec:3.8.3.3.2}, \sectref{sec:7.3}). In this Mauwake provides an interesting exception to a very strong cross-linguistic tendency of dropping subject pronouns in imperative clauses (\citealt[80]{Givon1979}; \citealt[173--174]{SadockEtAl1985}). In this position the pronoun is usually unstressed, unless it is contrasted with the subject of another clause coordinated with the imperative clause \REF{ex:3:}. 

\ea%x1771
\label{ex:3:x1771}
\gll ``\textstyleEmphasizedVernacularWords{No} me baurar-e,'' naap maak-e-k. \\
2s.\textsc{unm} not flee-\textsc{imp}.2s thus tell-\textsc{pa}-3s\\
\glt` ``Don't run away,'' he told her.'
\z

\ea%x1772
\label{ex:3:x1772}
\gll \textstyleEmphasizedVernacularWords{I} or-u. \\
1p.\textsc{unm} descend-\textsc{imp}.1d\\
\glt`Let's go down.'
\z

\ea%x1780
\label{ex:3:x1780}
\gll \textstyleEmphasizedVernacularWords{No} feeke ik-e, yo Amerika wia akup-ikiw-i-yem.\\
2s.\textsc{unm} here.\textsc{cf} be-\textsc{imp}.2s 1s.\textsc{unm} America 3p.\textsc{acc} search-go-Np-\textsc{pr}.1s\\
\glt`You stay here, I will go searching the Americans.'
\z

In an imperative clause the subject pronoun may also be used appositionally with a \textstyleAcronymallcaps{NP} that has vocative function, to address a person \REF{ex:3:x627}. 

\ea%x627
\label{ex:3:x627}
\gll Muuka, \textstyleEmphasizedVernacularWords{no} aakisa emeria aaw-e! \\
son 2s.\textsc{unm} now woman take-\textsc{imp}.2s\\
\glt`Son, take a wife now!' (i.e. It is time for you to get married.)
\z

There are some cases where the imperative clauses tend not to have a subject pronoun. When the clause has a theme (\sectref{sec:9.1}) different from the subject, and especially when the theme is another pronoun \REF{ex:3:}, the imperative subject is blocked:

\ea%x1773
\label{ex:3:x1773}
\gll A, ifera\textsubscript{T}H feeke un-eka. \\
ah salt.water here.\textsc{cf} draw/fetch-\textsc{imp}.2p\\
\glt`Ah, fetch the salt water (right) here.'
\z

\ea%x1774
\label{ex:3:x1774}
\gll Yo\textsubscript{TH, TP} momor me yook-e. \\
1s.\textsc{unm} foolishly not follow.me-\textsc{imp}.2s\\
\glt`Don't be foolish and follow me.' (Lit: `Don't follow me foolishly.')
\z

When an imperative final clause is preceded by a different-subject medial clause, it does not have a subject pronoun either:

\ea%x1775
\label{ex:3:x1775}
\gll Nefa war-iwkin naap ma-e. \\
2s.\textsc{acc} shoot-2/3p.\textsc{ds} thus say-\textsc{imp}.2s\\
\glt`When they shoot you, (then) say like that.'
\z

A sentence-initial subject pronoun is quite common, when one or more same-subject medial clauses precede the imperative final clause and the scope of the imperative extends backwards over all the verbs:

\ea%x628
\label{ex:3:x628}
\gll \textstyleEmphasizedVernacularWords{Ni} ikiw-ep moma perek-eka! \\
2p.\textsc{unm} go-\textsc{ss}.\textsc{seq} taro pull.out-\textsc{imp}.2p\\
\glt`Go and pull out (i.e. harvest) taro!'
\z

The only example in the text data of a subject pronoun repeated in the final clause is a case where the medial clause is subordinated with the topic marker -\textstyleStyleVernacularWordsItalic{na}: 

\xbox{1.1\textwidth}{
\ea%x1776
\label{ex:3:x1776}
\gll \textstyleEmphasizedVernacularWords{Ni} uf-ep-na \textstyleEmphasizedVernacularWords{ni} maadara me iirar-eka {\dots}\\
2p.\textsc{unm} dance-\textsc{ss}.\textsc{seq}=\textsc{tp} 2s.\textsc{unm} forehead.ornament not remove-\textsc{imp}.2p\\
\glt`If/when you have danced, do not remove the forehead ornaments {\dots}' 
\z
}

When the level of politeness is reduced, the subject pronoun is less common. Some acceptable reasons for this are urgency \REF{ex:3:x1777}, or speech by an official that is expected to be brusque \REF{ex:3:x1778}. Example \REF{ex:3:x1779} is from a situation where the behaviour of some men has been offensive to their wives, and when the men return home and give a blunt command, their wives react to this additional insult by repeating the command and then stating their own grievance and their revenge. 

\ea%x1777
\label{ex:3:x1777}
\gll Karu-eka, ikoka Yaapan ir-ami {\dots} \\
run-\textsc{imp}.2p later Japan come-\textsc{ss}.\textsc{sim}\\
\glt`Run, later the Japanese will come and {\dots}'
\z

\ea%x1778
\label{ex:3:x1778}
\gll ...amia mua=ke ma-e-mik, ``Nainiw owowa ikiw-eka.'' \\
bow man=\textsc{cf} say-\textsc{pa}-1/3p again village go-\textsc{imp}.2p\\
\glt`{\dots}the policemen said, ``Go back to the village.'' '
\z

\ea%x1779
\label{ex:3:x1779}
\gll Ekap-emi wia maak-e-mik, ``Maa iiw-eka.'' ```Maa iiw-eka.' Nis=ke sira oko on-ami...''\\
come-\textsc{ss}.\textsc{sim} 3p.\textsc{acc} tell-\textsc{pa}-1/3p food dish.out-\textsc{imp}.2p food dish.out-\textsc{imp}.2p 2p.\textsc{fc}=\textsc{cf} custom other do-\textsc{ss}.\textsc{sim}\\
\glt`They came and told them, ``Dish out the food.'' `` `Dish out the food!' You acted offensively (lit: did another custom) and{\dots}'' '
\z

There is some tendency to have a pronominal form to occupy the sentence-initial theme position (\sectref{sec:9.1}), especially when the pronoun refers to the main participant of the sentence. In some cases this results in the restructuring of the sentence so that a medial clause appears in the middle of the finite clause, instead of coming before it as would be more normal. In \REF{ex:3:x539} and \REF{ex:3:x540} the medial clauses are enclosed in square brackets. 

\ea%x539
\label{ex:3:x539}
\gll \textstyleEmphasizedVernacularWords{Yo} [eka yoowa Magidar=ke kirip-ap yi-eya] en-e-m.\\
1s.\textsc{unm} water hot Magidar=\textsc{cf} mix-\textsc{ss}.\textsc{seq} give.me-2/3s.\textsc{ds} eat-\textsc{pa}-1s\\
\glt`Magidar made tea and gave it to me, and I drank it.'
\z

\ea%x540
\label{ex:3:x540}
\gll \textstyleEmphasizedVernacularWords{No} [um-eya] or-o-n. \\
2s.\textsc{unm} die-2/3s.\textsc{ds} descend-\textsc{pa}-2s\\
\glt`After he died you went down.'
\z

Sentence-initial unmarked pronouns are also used when they are not subjects but rather mark a pronoun with other than subject function as the theme. The first person pronoun in particular is placed in the theme position very frequently, the second person less so and the third person least of all.

\ea%x535
\label{ex:3:x535}
\gll \textstyleEmphasizedVernacularWords{Yo} \textstyleEmphasizedVernacularWords{efa} uruf-e! \\
1s.\textsc{unm} 1s.\textsc{acc} look-\textsc{imp}.2s\\
\glt`Look at me!'
\z

\ea%x536
\label{ex:3:x536}
\gll \textstyleEmphasizedVernacularWords{I} \textstyleEmphasizedVernacularWords{yiena} mua opora \textstyleEmphasizedVernacularWords{yia} asip-owa ekap-e-mik nain \\
1p.\textsc{unm} 1p.\textsc{gen} man talk 1p.\textsc{acc} help-\textsc{nmz} come-\textsc{pa}-1/3p that\\
\glt`Our men who have come to help us with the language {\dots}'
\z

Especially in spoken language the unmarked pronouns may also be used, instead of genitive pronouns, to indicate possession. This is most commonly done with kinship terms and body parts, sometimes with other nouns\footnote{The following list covers most of them: \textstyleFootnoteBaseChar{\textit{opora}} `talk, speech', \textit{opaimika} `mouth, speech', \textit{unuma} `name', \textstyleFootnoteBaseChar{\textit{koora}} `house, home', \textit{manina} `garden', \textit{siowa} `dog' and \textstyleFootnoteBaseChar{\textit{amina}} `saucepan'.} too, referring to things closely associated with a person. This usage can be seen as a kind of widening of the range of inalienably possessed nouns beyond the kinship terms (\sectref{sec:3.2.4}) to other nouns that would be inalienably possessed in related languages or some other languages in the area. 

\ea%x537
\label{ex:3:x537}
\gll \textstyleEmphasizedVernacularWords{Yo} auwa nan ik-ua. \\
1s.\textsc{unm} 1s/p.father there be-\textsc{pa}.3s\\
\glt`My father is there.'
\z

\ea%x538
\label{ex:3:x538}
\gll Ikoka Yaapan=ke \textstyleEmphasizedVernacularWords{ni}  umakuna nia puuk-i-kuan. \\
Later Japan=\textsc{cf} 2p.\textsc{unm} neck 2p.\textsc{acc} cut-Np-\textsc{fu}.3s\\
\glt`Later the Japanese will cut your necks.'
\z

\ea%x1804
\label{ex:3:x1804}
\gll Aria, \textstyleEmphasizedVernacularWords{yo} opora muut nan-e-k. \\
alright 1s.\textsc{unm} talk only there-\textsc{pa}-3s\\
\glt`Alright, there is my talk.'
\z

The third person plural unmarked pronoun is used to pluralise a noun phrase \REF{ex:3:x625}. It is also often used with a place name to refer to the inhabitants of the place collectively \REF{ex:3:x626}.

\ea%x625
\label{ex:3:x625}
\gll \textstyleEmphasizedVernacularWords{Wi} sawur nain=ke kuura puuk-a-mik. \\
3p.\textsc{unm} spirit that1=\textsc{cf} fly cut-\textsc{pa}-3s\\
\glt`Those spirits changed into flies.'
\z

\ea%x626
\label{ex:3:x626}
\gll \textstyleEmphasizedVernacularWords{Wi} Lasen=ke kuum-e-mik. \\
3p.\textsc{unm} Lasen=\textsc{cf} burn-\textsc{pa}-1/3p\\
\glt`The Lasen people burned it.' (Or: `It was the Lasen people who burned it.')
\z

The neutral focus marker -\textstyleStyleVernacularWordsItalic{ko} attaches itself to the unmarked pronoun rather than the focal pronoun. I do not know the reason for this.\footnote{Kwan Poh San suggests as a possible reason that as the irrealis focus does not give as strong an emphasis as the contrastive focus, it also attaches itself to a less emphasized form of the pronoun (p.c.).}

\ea%x547
\label{ex:3:x547}
\gll Waaya en-e-man nain \textstyleEmphasizedVernacularWords{yo=ko} me uruf-a-m. \\
pig eat-\textsc{pa}-2p that1 1s.\textsc{unm}=\textsc{nf} not see-\textsc{pa}-1s \\
\glt`I didn't (get to even) see the pig that you ate.'
\z

The unmarked pronouns are used as the basic form for focal, genitive, reflexive-reciprocal and isolative pronouns.

\subsubsection{Focal pronouns}\label{sec:3:z:y:x}
%\hypertarget{RefHeading19601935131865}
{}
The focal pronouns are similar to the unmarked pronouns but have final \textstyleStyleVernacularWordsItalic{-s: yos, nos, (w)os, (y)is, nis, wis.} These pronouns are never used for a neutral, non-focused subject. They are used in isolation and in lists \REF{ex:3:x541}, as well as with the topic marker -\textstyleStyleVernacularWordsItalic{na} \REF{ex:3:x542}, the contrastive focus marker -\textstyleStyleVernacularWordsItalic{ke} \REF{ex:3:x543}, the question marker -\textstyleStyleVernacularWordsItalic{i} \REF{ex:3:x544} and the adverb \textstyleStyleVernacularWordsItalic{pun} `also' \REF{ex:3:x546}. With the limiter -\textstyleStyleVernacularWordsItalic{iw} \REF{ex:3:x545} the focal pronoun forms one of the two kinds of restrictive pronoun. (See \sectref{sec:3.5.7}.)

\ea%x541
\label{ex:3:x541}
\gll \textstyleEmphasizedVernacularWords{Yos}, yena emeria, ne Yoli gelemuta {\dots} \\
1s.\textsc{fc} 1s.\textsc{gen} woman \textsc{add} Yoli little\\
\glt`I, my wife and little Yoli {\dots}'
\z

\ea%x542
\label{ex:3:x542}
\gll \textstyleEmphasizedVernacularWords{Nos}=na? \\
2s.\textsc{fc}=\textsc{tp}\\
\glt`What about you?'
\z

\ea%x543
\label{ex:3:x543}
\gll \textstyleEmphasizedVernacularWords{Is}=ke me kuum-e-mik. \\
1p.\textsc{fc}=\textsc{cf} not burn-\textsc{pa}-1/3p\\
\glt`\textstyleEmphasizedWords{\textsc{We}} didn't burn it.'
\z

\ea%x544
\label{ex:3:x544}
\gll \textstyleEmphasizedVernacularWords{Yos}=i? \\
1sg.\textsc{fc}=\textsc{qm}\\
\glt`I?' 
\z

\ea%x546
\label{ex:3:x546}
\gll \textstyleEmphasizedVernacularWords{Os} pun opora kuisow naap=iw ma-e-k. \\
3s.\textsc{fc} also talk one thus=\textsc{lim} say-\textsc{pa}-3s\\
\glt`\textstyleEmphasizedWords{\textsc{He}} also said the same thing.'
\z

\ea%x545
\label{ex:3:x545}
\gll Anane \textstyleEmphasizedVernacularWords{nos=iw} nefa maak-i-ya. \\
always 2s.\textsc{fc}=\textsc{lim} 2s.\textsc{acc} tell-Np-\textsc{pr}.3s\\
\glt`He always talks to you only.'
\z

When the subject of an imperative clause is contrasted with some other possible subject, the focal pronoun with contrastive focus clitic is employed: 

\ea%x629
\label{ex:3:x629}
\gll \textstyleEmphasizedVernacularWords{Nos=ke} ikiw-e! \\
2s.\textsc{fc}=\textsc{cf} go-\textsc{imp}.2s\\
\glt`\textstyleEmphasizedWords{\textsc{You}} go (not someone else)!'
\z

\subsection{Accusative pronouns}\label{sec:3:y:x}
%\hypertarget{RefHeading19621935131865}
{}
The accusative pronouns may have been derived from the unmarked pronouns, but because at present there is little similarity between the singular forms of the two sets, the accusative pronouns are treated as a set of their own. Their main use is to mark the syntactic object of a clause, which is typically the semantic patient but with a few verbs may be a recipient (\sectref{sec:5.2}). The plural forms are also used for the beneficiary, as the beneficiary suffix -\textstyleStyleVernacularWordsItalic{a} in the verb (\sectref{sec:3.8.3.1}) does not distinguish between singular and plural. The accusative pronouns serve as a basis for some other pronoun forms with different functions as well. The form of the accusative pronouns is reflected very closely in the plural stems of the object cross-referencing verbs but not in the singular stems (\sectref{sec:3.8.4.2.4}). 

The accusative pronouns are:

\begin{table}
\caption{Please provide a caption}
\label{} 
\begin{tabular}{lll}
\mytoprule
 &singular &plural\\
\midrule
1 &efa &yia\\
2 &nefa &nia\\
3 &{\O} (zero) &wia\\
\mybottomrule
\end{tabular}
\end{table}


Only objects that are [+human] are marked with the pronoun. As there is no other case marking in \textstyleAcronymallcaps{NP}s, except for oblique case marking like locative and instrument for [\nobreakdash-human] \textstyleAcronymallcaps{NP}s, the accusative pronouns provide some of this case marking, when the object is a [+human] \textstyleAcronymallcaps{NP}. Much of the time there is no overt pronoun, as the third person singular form is zero.\footnote{Zero pronoun for the third person singular is not exceptional cross-lingustically (\citealt[278]{Lyons1968}; \citealt[66]{Foley1986}; \citealt[166]{Givon1976}), and in Papuan languages it is common especially for the object pronoun. All the 25 Northern Adelbert Range languages compared by \citet[9,160]{ZGraggen1980} have zero as object pronoun or object marking on the verb for the third person singular form.} 

The position of the accusative pronouns in Mauwake is immediately preceding the verb. This is probably the main reason why \citet{ZGraggen1971} treats them as verbal prefixes. Likewise, \citet[108]{Reesink1987} states that Usan has object prefixes, even if they have a rather loose status and can be detached from the verb. But I consider the object pronouns in Mauwake independent words, as they all have two syllables and follow the normal stress pattern of the language. They are, however, very closely bound to the verb, and it seems that a cliticization process is going on.\footnote{In \citet{Jarvinen1991} I discussed this question whether Mauwake pronouns are full words, clitics or affixes, at some length.} 

The accusative pronouns are used for encoding semantic patient \REF{ex:3:x548}, or recipient \REF{ex:3:x550}, both of which are syntactic objects (\sectref{sec:5.2}, \sectref{sec:5.3}). 

\ea%x548
\label{ex:3:x548}
\gll Irakowa=pa \textstyleEmphasizedVernacularWords{wia} war-e-mik. \\
fight=\textsc{loc} 3p.\textsc{acc} kill-\textsc{pa}-1/3p\\
\glt`In the fight they killed them.'
\z

\ea%x550
\label{ex:3:x550}
\gll Opora nain \textstyleEmphasizedVernacularWords{efa} maak-e-k. \\
talk that 1s.\textsc{acc} tell-\textsc{pa}-3s\\
\glt`He told me the story.' 
\z

The plural forms of the accusative pronouns are used together with the beneficiary form in the verb to disambiguate between the persons \REF{ex:3:x549} (\sectref{sec:3.8.3.1}).

\ea%x549
\label{ex:3:x549}
\gll Aite maa \textstyleEmphasizedVernacularWords{yia} p-or-om-a-k. \\
mother food 1p.\textsc{acc} Bpx-descend-\textsc{ben}-\textsc{bnfy}2.\textsc{pa}-3s\\
\glt`Mother brought food down for us.'
\z

The only grammatical difference between the semantic roles of patient and beneficiary is shown in the verb, which can incorporate the benefactive suffix; and between patient and recipient there is no syntactic or morphological difference. The following hierarchy is followed: if there is a recipient not incorporated in the verb root,\footnote{Verbs like `give' and `feed' incorporate the recipient object in the verb root itself (\sectref{sec:3.8.4.2.4}).} the accusative pronoun refers to it \REF{ex:3:}, if there is no recipient but a plural beneficiary, the pronoun refers to the latter \REF{ex:3:}. And if there is neither recipient nor beneficiary, the accusative pronoun refers to the patient \REF{ex:3:}.\footnote{Cf. a rather similar hierarchy for the distributive suffix in verbs (\sectref{sec:3.8.2.3.2})} 

Transitive verbs in Mauwake usually require an overt object, and verbs like `teach', `tell', `ask', which can take two objects, require the presence of at least the human object, whether patient \REF{ex:3:}, or recipient \REF{ex:3:}. In \REF{ex:3:} the pronoun \textstyleStyleVernacularWordsItalic{wia} `3p.\textsc{acc}' may be definite or indefinite, hence the alternative free translations.

\ea%x552
\label{ex:3:x552}
\gll \textstyleEmphasizedVernacularWords{Nefa} nokar-i-yem. \\
2s.\textsc{acc} ask-Np-\textsc{pr}.1s\\
\glt`I'm asking you.'
\z

\ea%x551
\label{ex:3:x551}
\gll Inglis \textstyleEmphasizedVernacularWords{wia} ofakow-i-ya. \\
English 3p.\textsc{acc} teach-Np-\textsc{pr}.3s\\
\glt`(S)he teaches them English.' (Or: `(S)he teaches English.')
\z

In rare cases the human object may be left out:

\ea%x553
\label{ex:3:x553}
\gll Oram nokar-i-yem. \\
just ask-Np-\textsc{pr}.1s\\
\glt`I'm just asking.' (Asking nobody in particular, or for no particular reason.)
\z

Transitive verbs with [+human] objects require pronouns even when the object is mentioned as a noun or a noun phrase.

\ea%x554
\label{ex:3:x554}
\gll Emeria \textstyleEmphasizedVernacularWords{wia} amukar-e-k. \\
woman 3p.\textsc{acc} scold-\textsc{pa}-3s\\
\glt`He/she scolded the women.'
\z

\ea%x555
\label{ex:3:x555}
\gll Emeria \textstyleEmphasizedVernacularWords{nia} amukar-e-k. \\
woman 2p.\textsc{acc} scold-\textsc{pa}-3s\\
\glt`He/she scolded (you) women.'
\z

Since the third person singular form is zero, all the cases with [+human] object noun without overt object pronoun by default indicate the third person singular \REF{ex:3:}. Because there is no number or case distinction in the nouns for the arguments of the verb, without this indication by pronouns it would often be ambiguous whether the \textstyleAcronymallcaps{NP} was subject or object, or whether the object was singular or plural. 

\ea%x556
\label{ex:3:x556}
\gll Emeria amukar-e-k. \\
woman scold-\textsc{pa}-3s\\
\glt`He scolded his wife.'
\z

In theory, the example \REF{ex:3:} could also mean `The woman scolded him/her' but in practice it does not. For when the subject is old/established information it is usually left out rather than marked by a \textstyleAcronymallcaps{NP}, and when it is new information, it is marked by the contrastive focus marker -\textstyleStyleVernacularWordsItalic{ke}.\footnote{To have the meaning `He/she scolded a/the \textstyleEmphasizedWords{woman'}, the noun would be followed by the non-numeral quantifier \textstyleFootnoteBaseChar{\textit{oko}} `a, a certain' or the demonstrative \textstyleFootnoteBaseChar{\textit{nain}} `that'.}

It must be clearly indicated whether the speaker or addressee is included in the object \REF{ex:3:}, \REF{ex:3:}, \REF{ex:3:}.

\ea%x557
\label{ex:3:x557}
\gll Mua \textstyleEmphasizedVernacularWords{yia} aaw-i-kuan. \\
man 1p.\textsc{acc} take-Np-\textsc{fu}.3p\\
\glt`They will take (us) men.'
\z

\citet[52--53]{Reesink1987} mentions that Usan, another Pihom Stock language, has object prefixes, but a free pronoun can also occupy the object position in the third person singular. This is not the case in Mauwake; in \REF{ex:3:x1354} the free pronoun \textstyleStyleVernacularWordsItalic{o} is a re-activated topic (\sectref{sec:9.2.3}). The negative clause \REF{ex:3:x1353} shows that the position of the free pronoun is not directly preceding the verb. The clauses \REF{ex:3:x684}, \REF{ex:3:x560} have a similar structure with pronouns in non-third person marking a theme. When the pronoun is fronted as a theme (\sectref{sec:9.1}), it is this unmarked pronoun that is used in the theme position. 

\ea%x1354
\label{ex:3:x1354}
\gll Wi teeria papako \textstyleEmphasizedVernacularWords{o} {\O} asip-a-mik... \\
3p.\textsc{unm} group other 3s.\textsc{unm} {\O} help-\textsc{pa}-1/3p\\
\glt`Another group helped him{\dots}' (Or: `He was helped by another group{\dots}')
\z

\ea%x1353
\label{ex:3:x1353}
\gll \textstyleEmphasizedVernacularWords{O} me {\O aaw-e-mik.} \\
3s.\textsc{unm} not {\O} take-\textsc{pa}-1/3p\\
\glt`They did not take/choose him.' (Or: `He was not taken by them.')
\z

\ea%x684
\label{ex:3:x684}
\gll \textstyleEmphasizedVernacularWords{Yo} me \textstyleEmphasizedVernacularWords{efa} aaw-e-mik. \\
1s.\textsc{unm} not 1s.\textsc{acc} take-\textsc{pa}-1/3p\\
\glt`They didn't take/choose me.' (Or: `I wasn't taken by them.')
\z

\ea%x560
\label{ex:3:x560}
\gll \textstyleEmphasizedVernacularWords{Yo} \textstyleEmphasizedVernacularWords{efa} aaw-e-mik. \\
1s.\textsc{unm} 1s.\textsc{acc} take-\textsc{pa}-1/3p\\
\glt`They took/chose me.' (Or: `I was taken by them.')
\z

There is one instance where the free third person singular pronoun does occur after the negator and immediately preceding the verb, just like accusative pronouns. This is when there is constituent negation (\sectref{sec:6.2.2}.) on the object, which then also receives clausal stress \REF{ex:3:x561} (\sectref{sec:2.1.3.1}, 9.2.3). Here it is the negator that moves to precede the constituent it negates. The same process is also seen in \REF{ex:3:x562} where the negator has moved in front of the whole object \textstyleAcronymallcaps{NP}. 

\ea%x561
\label{ex:3:x561}
\gll Me \textstyleEmphasizedVernacularWords{o} uruf-a-m. \\
not 3s.\textsc{unm} see-\textsc{pa}-1s\\
\glt`It wasn't him/her that I saw.'
\z

\ea%x562
\label{ex:3:x562}
\gll Me \textstyleEmphasizedVernacularWords{wi} owow mua \textstyleEmphasizedVernacularWords{wia} arew-a-mik{\dots} \\
not 3p.\textsc{unm} village man 3p.\textsc{acc} wait-\textsc{pa}-1/3p\\
\glt`It wasn't the village people that we waited for{\dots}'
\z

There are situations where it is impossible to determine whether the unmarked third person singular pronoun is marking a topic/subject or an object fronted as a theme (\sectref{sec:9.1}). The context would be needed to disambiguate between the slightly different meanings of \REF{ex:3:x563}, which do not come out well in the English translation. The first meaning is likely if the context mentions some other people seeing something; the second meaning is more probable elsewhere. 

\ea%x563
\label{ex:3:x563}
\gll \textstyleEmphasizedVernacularWords{O} me uruf-a-k. \\
3s.\textsc{unm} not see-\textsc{pa}-3s\\
\glt`\textit{(S)he} didn't see him/her/it.' (Or: `(S)he didn't see \textit{him/her}.')
\z

There are a few verbs in Mauwake that cross-reference the patient or recipient object in the verb root (\sectref{sec:3.8.4.2.4}). These verbs do not allow a separate accusative pronoun for the function that is already expressed by the verb root \REF{ex:3:x564}, \REF{ex:3:x1526}, but it is possible to have a separate accusative pronoun for the patient when the verb cross-references the recipient rather than the patient \REF{ex:3:x565}.

\ea%x564
\label{ex:3:x564}
\gll Ipia=ke \textstyleEmphasizedVernacularWords{yiar-eya} ekap-e-mik. \\
rain=\textsc{cf} hit.us-2/3s.\textsc{ds} come-\textsc{pa}-1/3p\\
\glt`The rain hit us and we/they came.'
\z

\ea%x1526
\label{ex:3:x1526}
\gll Yomar, no uurika \textstyleEmphasizedVernacularWords{yook}\textstyleEmphasizedVernacularWords{-}\textstyleEmphasizedVernacularWords{ap} urup-e. \\
friend 2s.\textsc{unm} tomorrow follow.me-\textsc{ss}.\textsc{seq} ascend-\textsc{imp}.2s\\
\glt`Friend, follow me up tomorrow.'
\z

\ea%x565
\label{ex:3:x565}
\gll Iiriw \textstyleEmphasizedVernacularWords{nefa} \textstyleEmphasizedVernacularWords{wi-e-mik}. \\
already 2s.\textsc{acc} give.them-\textsc{pa}-1/3p\\
\glt`We have already given you to them.'
\z

When other verbs require both a [+human] recipient and a [+human] patient, it is encoded as a clause chain. The first verb then takes one of the arguments and the second the other.

\ea%x566
\label{ex:3:x566}
\gll Uuriw \textstyleEmphasizedVernacularWords{wia} aaw-ep \textstyleEmphasizedVernacularWords{nia} p-ekap-om-i-yen. \\
morning 3p.\textsc{acc} take-\textsc{ss}.\textsc{seq} 2p.\textsc{acc} Bpx-come-\textsc{ben}-Np-\textsc{fu}.1p\\
\glt`In the morning we will bring them (people) to you.'
\z

\subsection{Genitive pronouns}\label{sec:3:y:x}
%\hypertarget{RefHeading19641935131865}
{}
Since possession can be expressed by means of three different kinds of personal pronouns in Mauwake, I call the function \textstyleEmphasizedWords{\textsc{possessive}} and the different grammatical forms \textstyleEmphasizedWords{\textsc{genitive}}, \textstyleEmphasizedWords{\textsc{dative}}\textstyleDefinedWords{} and \textstyleEmphasizedWords{\textsc{unmarked pronoun}}. All these forms have other functions besides possessive, as has already been shown for the unmarked pronoun. 

The genitive pronouns are derived from the unmarked pronouns by the ending \nobreakdash-\textstyleStyleVernacularWordsItalic{ena}:\footnote{This ending is probably related to the specifier -\textit{ena.}}

\begin{table}
\caption{Please provide a caption}
\label{} 
\begin{tabular}{lll}
\mytoprule
 &singular &plural\\
\midrule
1 &y-ena &yi-ena\\
2 &n-ena &ni-ena\\
3 &o-na &wi-ena\\
\mybottomrule
\end{tabular}
\end{table}


The main function of the genitive pronoun is to indicate the possessor in a \textstyleAcronymallcaps{NP,} and the main strategy for expressing the possessor in a \textstyleAcronymallcaps{NP} is to use either the genitive pronoun or a possessive noun phrase. Unlike most other modifiers of the noun, the genitive pronoun precedes the head noun. This is in accord with \citegen[202]{Givons1984} implicational hierarchy of conformity to basic word order, as well as Dryer's (2007a:62) statement about word order correlations. In Mauwake only the nominal and genitive modifiers and noun complements, which are also at the top of Giv\'on's \citeyear{Givon1984} hierarchy, precede the head noun in the \textstyleAcronymallcaps{NP}s; all the other modifiers follow the head noun.

The genitive pronoun is used when the possessor is coreferential with the subject,\footnote{It does not have to be used when the possessive relationship is clear from the context; see (234)} and its meaning is often close to English `own'.

\ea%x1805
\label{ex:3:x1805}
\gll Sawur emeria nain=ke \textstyleEmphasizedVernacularWords{ona} soma mua nain ifakim-o-k. \\
spirit woman that1=\textsc{cf} 3s.\textsc{gen} lover man that1 kill-\textsc{pa}-3s\\
\glt`The spirit woman killed her (own) lover.'
\z

\ea%x1806
\label{ex:3:x1806}
\gll Mua me wia imen-ap=na \textstyleEmphasizedVernacularWords{niena} maa=ke ... \\
man not 3p.\textsc{acc} find-\textsc{ss}.\textsc{seq}=\textsc{tp} 2p.\textsc{gen} thing=\textsc{cf}\\
\glt`If you don't find the men, it's your (own) business {\dots}'
\z

\ea%x567
\label{ex:3:x567}
\gll \textstyleEmphasizedVernacularWords{Niena} unuma maifa feeke siisim-eka. \\
2p.\textsc{gen} name paper here.\textsc{cf} write-\textsc{imp}.2p\\
\glt`Write your names on the paper here/ on this paper.'
\z

In descriptive or equative clauses genitive pronouns can modify both the subject \textstyleAcronymallcaps{NP} \REF{ex:3:x568} and the non-verbal predicate \textstyleAcronymallcaps{NP} \REF{ex:3:x569}, whereas the dative pronouns can modify neither. 

\ea%x568
\label{ex:3:x568}
\gll \textstyleEmphasizedVernacularWords{Yena} koora maneka wenup. \\
1s.\textsc{gen} house big very\\
\glt`My house is very big.'
\z

\ea%x569
\label{ex:3:x569}
\gll Mua fain me \textstyleEmphasizedVernacularWords{nena} niawi akena=ke. \\
man this not 2s.\textsc{gen} 2s/p.father true=\textsc{cf}\\
\glt`This man is not your real father.'
\z

It is possible for a genitive pronoun to co-occur with a dative pronoun to modify the same noun which is not coreferential with the subject. (See \sectref{sec:3.5.5} for a further discussion on the differences between genitive and dative possessives.)

\ea%x570
\label{ex:3:x570}
\gll \textstyleEmphasizedVernacularWords{Yena} koora \textstyleEmphasizedVernacularWords{efar} aw-o-k. \\
1s.\textsc{gen} house 1s.\textsc{dat} burn-\textsc{pa}-3s\\
\glt`My house burned.'
\z

Even when the possessor is expressed by a noun or \textstyleAcronymallcaps{NP}, the genitive pronoun is sometimes explicit, \textstyleParagraphContinuationChar{occurring either between the possessor and the possessed} \textstyleAcronymallcaps{NP}\textstyleParagraphContinuationChar{ (}\textstyleParagraphContinuationChar{\stepcounter{nx}{\thenx}}\textstyleParagraphContinuationChar{) or, quite frequently, preceding both (}\textstyleParagraphContinuationChar{\stepcounter{nx}{\thenx}}\textstyleParagraphContinuationChar{).} 

\ea%x573
\label{ex:3:x573}
\gll Om-em-ik-eya sawur emeria \textstyleEmphasizedVernacularWords{ona} wiawi onak=ke ekap-emi maak-e-mik{\dots}
\\
cry-\textsc{ss}.\textsc{sim}-be-2/3s.\textsc{ds} spirit woman 3s.\textsc{gen} 3s/p.father 3s/p.mother=\textsc{cf} come-\textsc{ss}.\textsc{sim} tell-\textsc{pa}-1/3p\\
\glt`While she was crying, the spirit woman's father and mother came and told her, {\dots}'
\z

\ea%x574
\label{ex:3:x574}
\gll \textstyleEmphasizedVernacularWords{Wiena} mia kia maa=iw on-a-mik. \\
3p.\textsc{gen} skin white thing=\textsc{inst} do-\textsc{pa}-1/3p\\
\glt`They did it with the Europeans' things.'
\z

The reason for this addition of a pronoun may be the lack of case marking in nouns, which makes the processing of possessed \textstyleAcronymallcaps{NP}s more difficult when there are modifying nouns in the \textstyleAcronymallcaps{NP}. But it is also quite common for a possessive \textstyleAcronymallcaps{NP} to occur without a genitive pronoun. 

\ea%x575
\label{ex:3:x575}
\gll Mua oko miira inawera=pa uruf-ap ma-i-mik, {\dots} \\
man other face dream=\textsc{loc} see-\textsc{ss}.\textsc{seq} say-Np-\textsc{pr}.1/3p\\
\glt`When we see another man's face in a dream we say, {\dots}'
\z

The third person singular possessive pronoun provides an exception to the rule that the personal pronouns are only used for the humans. However, the cases where \textstyleStyleVernacularWordsItalic{ona} `3s. possessive' refers to a non-human possessor are few and seem to require the connotation `own':

\ea%x1808
\label{ex:3:x1808}
\gll {\dots}\textstyleEmphasizedVernacularWords{ona} pia=pa nan karu-emi {\dots} \\
3s.\textsc{gen} bamboo=\textsc{loc} there run-\textsc{ss}.\textsc{sim} \\
\glt`{\dots}it (molten copper) runs there in its pipe (lit:bamboo) and {\dots}'
\z

In those instances where the possessed \textstyleAcronymallcaps{NP} in the predicative position lacks an overt head noun, three different strategies may be used. I have not observed any difference in meaning. The genitive pronoun may occur by itself, without a head noun, which can either be deleted completely \REF{ex:3:x576} or substituted by \textstyleStyleVernacularWordsItalic{nain} `that' \REF{ex:3:x577}, or the \textstyleAcronymallcaps{NP} can be expressed by a possessive phrase \REF{ex:3:x578} (\sectref{sec:4.4}). In all these instances the head noun occurs earlier in the same sentence, or occasionally in the preceding sentence. 

\ea%x576
\label{ex:3:x576}
\gll Ikiwosa \textstyleEmphasizedVernacularWords{yena}, wapena \textstyleEmphasizedVernacularWords{yena}{\dots} \\
head 1s.\textsc{gen}, hand 1s.\textsc{gen}\\
\glt`The head is mine (to eat), the hands are mine{\dots}'
\z

\ea%x577
\label{ex:3:x577}
\gll Fikera pun \textstyleEmphasizedVernacularWords{wiena} nain=ke. \\
kunai.grass too 3p.\textsc{gen} that1=\textsc{cf}\\
\glt`The kunai grass is theirs, too.'
\z

\ea%x578
\label{ex:3:x578}
\gll Maa nain \textstyleEmphasizedVernacularWords{yo/yena} \textstyleEmphasizedVernacularWords{efarik}. \\
thing that1 1s.\textsc{unm}/1s.\textsc{gen} 1s.\textsc{dat}\\
\glt`That thing is mine.'
\z

Like possessives in many other languages, the genitive pronoun may function as the subject of a nominalized clause (\sectref{sec:5.7}). The unmarked pronoun is used in the same position too; I have not found any difference in their use.

\ea%x571
\label{ex:3:x571}
\gll \textstyleEmphasizedVernacularWords{Yiena} owow maneka ikiw-owa nain ma-i-yem. \\
1p.\textsc{gen} village big go-\textsc{nmz} that1 say-Np-\textsc{pr}.1s\\
\glt`I'm telling about our going to town.'
\z

As ordinary main clause subjects the genitive pronouns are more emphatic than the unmarked pronouns.\footnote{Usan \citep[55]{Reesink1984}, Siroi \citep[20]{Wells1979} and Maia \citep[73]{Hardin2002} also use the same pronoun forms for possessive and emphatic pronouns, whereas Waskia \citep{RossEtAl1978} does not.} The pronunciation reflects the emphasis too: these pronouns receive a stronger stress than the unmarked pronouns when used as a subject.

\ea%x572
\label{ex:3:x572}
\gll Aasa enuma \textstyleEmphasizedVernacularWords{yena} me suuw-i-yem. \\
canoe new 1s.\textsc{gen} not push-Np-\textsc{pr}.1s\\
\glt`I don't take a new canoe down myself.'
\z

The following example has two identical genitive pronouns, the first one functioning as an emphatic subject pronoun and the second one as a possessive pronoun:

\ea%x686
\label{ex:3:x686}
\gll \textstyleEmphasizedVernacularWords{Yiena} iisow, \textstyleEmphasizedVernacularWords{yiena} garanga muutiw aaw-ep uup-ep en-e-mik.\\
1p.\textsc{gen} 1p.ISOL 1p.\textsc{gen} family only take-\textsc{ss}.\textsc{seq} cook-\textsc{ss}.\textsc{seq} eat-\textsc{pa}-1/3p\\
\glt`Only our family by ourselves (lit: we ourselves we only, our family only) took it, cooked and ate it.'
\z

A genitive pronoun is also possible as the subject of a relative clause, when the subject is emphatic:

\ea%x1809
\label{ex:3:x1809}
\gll Wi teeria papako o asip-a-mik, [\textstyleEmphasizedVernacularWords{ona} eka sesenar-ep wienak-e-k nain]\textsubscript{RC}.
\\
3p.\textsc{unm} group other 3s.\textsc{unm} help-\textsc{pa}-1/3p 3s.\textsc{gen} water buy-\textsc{ss}.\textsc{seq} feed.them-\textsc{pa}-3s that1\\
\glt`Another group helped him, those for whom \textit{he} had bought and given beer.'
\z

When the limiting clitic -\textstyleStyleVernacularWordsItalic{iw} `only' is added to the genitive pronoun, the result is a restrictive pronoun (\sectref{sec:3.5.7}):

\ea%x604
\label{ex:3:x604}
\gll Yo me nia maak-i-nen, \textstyleEmphasizedVernacularWords{nien=iw} ma-eka. \\
1s.\textsc{unm} not 2p.\textsc{acc} tell-Np-\textsc{fu}.1s 2p.\textsc{gen}=\textsc{lim} say-\textsc{imp}.2p\\
\glt`I will not tell you (what to do); discuss it on your own (among yourselves/as a group).'
\z

\subsection{Dative pronouns}\label{sec:3:y:x}
%\hypertarget{RefHeading19661935131865}
{}
The dative case is typically associated with the semantic function of goal. The pronouns called dative in Mauwake do sometimes function as goals, but mostly they have a locative or source function. So the term here is to be understood more as a [\textstyleEmphasizedWords{\textsc{+}}human]\textstyleEmphasizedWords{ }\textstyleEmphasizedWords{\textsc{locative}}, which includes not only locative but goal and source as well. The dative pronouns have also grammaticalized as possessives to form possessive predicate construction (\sectref{sec:5.5.2}) and as attributive possessives to indicate that the possessor is non-coreferential with the subject. 

The dative pronouns are formed by adding -r to the accusative pronouns, with the exception of third person singular, which is identical with the plural.\footnote{The third person singular form probably used to be \textstyleFootnoteBaseChar{\textit{wo-ar}}, which is still currently used by a few people.}

\begin{table}
\caption{Please provide a caption}
\label{} 
\begin{tabular}{lll}
\mytoprule
 &singular &plural\\
\midrule
1 &efa-r &yia-r\\
2 &nefa-r &nia-r\\
3 &wia-r &wia-r\\
\mybottomrule
\end{tabular}
\end{table}


The syntactic function of a dative pronoun may be clausal (a locative adverbial phrase, see \sectref{sec:4.6.1}), or \textstyleAcronymallcaps{NP}-internal (a possessive modifier, see \sectref{sec:4.1.1}). Regardless of its function, the dative pronoun is always in immediately preverbal position. 

The semantic function of a dative pronoun is related to the verb of the clause. With motion verbs it has goal function: 

\ea%x1781
\label{ex:3:x1781}
\gll Pok-ap ika-iwkin mua \textstyleEmphasizedVernacularWords{wiar} ekap-e-mik. \\
sit-\textsc{ss}.\textsc{seq} be-2/3p.\textsc{ds} man 3.\textsc{dat} come-\textsc{pa}-1/3p\\
\glt`They were sitting and (their) husbands came to them.'
\z

\ea%x580
\label{ex:3:x580}
\gll Mia kokas-owa=ke \textstyleEmphasizedVernacularWords{wiar} kerer-e-k. \\
skin itch-\textsc{nmz}=\textsc{cf} 3.\textsc{dat} appear/arrive-\textsc{pa}-3s\\
\glt`Her skin started to itch.' (Lit: `Skin itch appeared to her.)' 
\z

With stative verbs the pronouns indicate location. (Note that the free translation needs to use a comitative expression, since English does not have a [+human] locative expression equivalent to the Mauwake dative.) 

\ea%x1782
\label{ex:3:x1782}
\gll Feeke \textstyleEmphasizedVernacularWords{wiar} ik-ok kiiriw mua wiar urup-e. \\
here.\textsc{cf} 3.\textsc{dat} be-SS again man 3.\textsc{dat} ascend-\textsc{imp}.2s\\
\glt`Having been here with him, go (back) to your husband again.'
\z

\ea%x1783
\label{ex:3:x1783}
\gll Wi sawur nain ir-ami fan \textstyleEmphasizedVernacularWords{yiar} pok-a-mik.{\footnotemark} \\
3p.\textsc{unm} spirit that1 go.east-\textsc{ss}.\textsc{sim} here 1p.\textsc{dat} sit-\textsc{pa}-1/3p\\
\glt`The spirits, going eastward, sat here with us.'
\z

\footnotetext{Although the most natural free translation is `{\dots}with us', comitative connotation should not be read into the Mauwake text; this is a locative.}

With verbs that indicate receiving something (take, get, buy, etc.) the dative has the semantic function of source:

\ea%x579
\label{ex:3:x579}
\gll Yo emeria Lasen=pa \textstyleEmphasizedVernacularWords{wiar} aaw-e-m. \\
1s.\textsc{unm} woman Lasen=\textsc{loc} 3.\textsc{dat} get/take-\textsc{pa}-1s\\
\glt`I got (my) wife from (the) Lasen (people).'
\z

\ea%x1784
\label{ex:3:x1784}
\gll Kuisow akena ika-eya yos=ke \textstyleEmphasizedVernacularWords{wiar} sesenar-ep aaw-e-m. \\
one very be-2/3s.\textsc{ds} 1s.\textsc{fc}=\textsc{cf} 3.\textsc{dat} buy-\textsc{ss}.\textsc{seq} get-\textsc{pa}-1s\\
\glt`There was only one and (it was) I (who) bought it from them.'
\z

\ea%x1785
\label{ex:3:x1785}
\gll Mua oko=ke waaya nain mik-ap \textstyleEmphasizedVernacularWords{nefar} aaw-i-non. \\
man other=\textsc{cf} pig that1 spear-\textsc{ss}.\textsc{seq} 2s.\textsc{dat} get/take-Np-\textsc{fu}.3s\\
\glt`Another man will spear the pig and take it from you.'
\z

The ``source'' can also be more abstract. I have observed this use only with verbs indicating hearing or speaking.

\ea%x1786
\label{ex:3:x1786}
\gll Naap \textstyleEmphasizedVernacularWords{wiar} miim-a-m. \\
thus 3.\textsc{dat} hear-\textsc{pa}-1s\\
\glt`I heard thus about him/her/them.'
\z

A locative phrase referring to a village or village area including its inhabitants is commonly used with a dative pronoun as well, otherwise it refers to just the location rather than the inhabitants. The pronoun may be used with towns or bigger areas as well, but the bigger the location, the less probable the pronoun is. In \REF{ex:3:} the people ran away to the people in the Bogia area, whereas in \REF{ex:3:} the people of Bogia town may not have been involved in the burial at all. In \REF{ex:3:} the speaker was going to the Highlands, not in order to meet the Highlanders but to work in a location there. 

\ea%x586
\label{ex:3:x586}
\gll Lasen \textstyleEmphasizedVernacularWords{wiar} ek-a-mik. \\
Lasen 3.\textsc{dat} go.east-\textsc{pa}-1/3p\\
\glt`We went to Lasen (village).'
\z

\ea%x587
\label{ex:3:x587}
\gll Baurar-ep Bogia kame \textstyleEmphasizedVernacularWords{wiar} ikiw-e-mik. \\
run.away-\textsc{ss}.\textsc{seq} Bogia area 3.\textsc{dat} go-\textsc{pa}-1/3p\\
\glt`They ran away to the Bogia area.'
\z

\ea%x1802
\label{ex:3:x1802}
\gll P-ikiw-ep Bogia=pa nan wu-a-mik. \\
Bpx-go-\textsc{ss}.\textsc{seq} Bogia=\textsc{loc} there put-\textsc{pa}-1/3p\\
\glt`We/They took it (a body) and buried it in Bogia.'
\z

\ea%x1800
\label{ex:3:x1800}
\gll Uuriw iinan aasa aaw-ep Epa Dabela urup-e-mik. \\
morning sky canoe take-\textsc{ss}.\textsc{seq} place cold ascend-\textsc{pa}-1/3p\\
\glt`In the morning we took an airplane and went up to the Highlands.'
\z

Cross-linguistically a \textstyleEmphasizedWords{\textsc{possessive predicate}} construction, a `have' construction, has often been derived from a locative or a goal/dative construction, plus a verb of existence \citep[50--61]{Heine1997}. In the possessive predicates in Mauwake the dative pronoun precedes the verb \textstyleStyleVernacularWordsItalic{ik}\textstyleStyleVernacularWordsItalic{-} `be'. 

\ea%x1788
\label{ex:3:x1788}
\gll I sira naap \textbf{yiar} ik-ua. \\
1p.\textsc{unm} custom thus 1p.\textsc{dat} be-\textsc{pa}.3s\\
\glt`We have a custom like that.' (Lit: `A custom like that is to us.)'
\z

The possessive predicate construction is discussed in more detail in \sectref{sec:5.5.2}. 

The same dative pronoun has also grammaticalized as a possessive attribute in a noun phrase , but here it is the semantic function of [+\textstyleEmphasizedWords{\textsc{human}}] \textstyleEmphasizedWords{\textsc{source}} that is behind the development. Conceptually the structures `\textstyleAcronymallcaps{X} took \textstyleAcronymallcaps{Y} from me' and `\textstyleAcronymallcaps{X} took my \textstyleAcronymallcaps{Y}' are very close. In \REF{ex:3:x581} \textstyleStyleVernacularWordsItalic{efar} can mean either `my' or `from me'. 

\ea%x581
\label{ex:3:x581}
\gll Nos=ke anane urema \textstyleEmphasizedVernacularWords{efar} ikum-ar-i-n. \\
2s.\textsc{fc}=\textsc{cf} always bandicoot 1s.\textsc{dat} illicitly-\textsc{inch}-Np-\textsc{pr}.2s\\
\glt`You always steal bandicoots from me / my bandicoots.'
\z

That it is difficult to distinguish between the roles of possessor and source is not unusual.\footnote{Sometimes it is hard to distinguish even between a possessor and a goal. In the following sentence \textit{efar} could also mean `to my place/house', with the head noun deleted: \textit{Yo me efar ekap-e}! [1s.\textsc{unm} not 1s.\textsc{dat} come-\textsc{imp}.2s] `Don't come to me!' } \citet[133]{Heine1997} mentions that early in the grammaticalization process ``these expressions can simultaneously be interpreted with reference to either their non-possessive source or to possession.'' In the following examples the source interpretation is not possible. The example \REF{ex:3:} describes a situation in future when the speaker will already be dead and his son is made to lose his inheritance.

\ea%x1861
\label{ex:3:x1861}
\gll A, yo aamun nan \textstyleEmphasizedVernacularWords{efar} ik-ua. \\
ah 1s.\textsc{unm} younger.sibling there 1s.\textsc{dat} be-\textsc{pa}.3s\\
\glt`Ah, there is my younger brother.'
\z

\ea%x1862
\label{ex:3:x1862}
\gll Ikoka yena yeepa muuka=ke yo muuka \textstyleEmphasizedVernacularWords{efar} iirar-ep maak-i-non {\dots}\\
later 1s.\textsc{gen} elder.sibling son=\textsc{cf} 1s.\textsc{unm} son 1s.\textsc{dat} remove-\textsc{ss}.\textsc{seq} tell-Np-\textsc{fu}.3s\\
\glt`Later my elder brother's son will remove/displace/drive away my son and tell him, {\dots}'
\z

This grammaticalization probably started with the verbs denoting taking and getting, but it is only a short step from there to interpreting the dative as a possessor with other verbs as well, especially as it is likely that the dative pronoun was already earlier established in the possessive predicate structure. 

\ea%x1789
\label{ex:3:x1789}
\gll Owowa \textstyleEmphasizedVernacularWords{yiar} kuuf-owa ekap-e-mik. \\
village 1p.\textsc{dat} see-\textsc{nmz} come-\textsc{pa}-1/3p\\
\glt`They came to see our village.'
\z

\ea%x1791
\label{ex:3:x1791}
\gll Auwa afura \textstyleEmphasizedVernacularWords{wiar} akim-ap=ko uruf-e. \\
1s/p.father lime 3.\textsc{dat} try-\textsc{ss}.\textsc{seq}=\textsc{nf} see-\textsc{imp}.2s\\
\glt`Try father's lime and see (what it is like).'
\z

\ea%x1790
\label{ex:3:x1790}
\gll Ikiwosa \textstyleEmphasizedVernacularWords{wiar} pepekim-ep kaik-a-m. \\
head 3.\textsc{dat} measure-\textsc{ss}.\textsc{seq} tie-\textsc{pa}-1s\\
\glt`I measured her head and tied it (a cane).'
\z

\ea%x1795
\label{ex:3:x1795}
\gll No me emeria \textstyleEmphasizedVernacularWords{nefar} maak-i-mik. \\
2s.\textsc{unm} not woman 2s.\textsc{dat} tell-Np-\textsc{pr}.1/3p\\
\glt`We are not telling/talking to your wife.'
\z

Although the possessive is often associated with malefactive overtones as in \REF{ex:3:} and \REF{ex:3:}, this is not part of its meaning \REF{ex:3:}, \REF{ex:3:}. 

\ea%x1787
\label{ex:3:x1787}
\gll Buburia koora \textstyleEmphasizedVernacularWords{wiar} aw-o-k. \\
bald house 3.\textsc{dat} burn-\textsc{pa}-3s \\
\glt`The bald man's house burned (on him).'
\z

\ea%x1792
\label{ex:3:x1792}
\gll Irak-emi amina \textstyleEmphasizedVernacularWords{wiar} fo-fook-omak-e-mik. \\
fight-\textsc{ss}.\textsc{sim} pot 3.\textsc{dat} \textsc{rdp}-split-\textsc{distr}/\textsc{pl}-\textsc{pa}-1/3p\\
\glt`They\textsubscript{i} fought and split their\textsubscript{j} pots.'
\z

But since Mauwake already had genitive pronouns to indicate possession, why did another possessive strategy develop? The answer may lie in the original source function of the dative pronoun. The referent of the participant with the source function is normally another than the referent of the clausal subject, and it is this feature of non-coreferentiality with the subject that became the distinctive feature for the new possessive. 

The dative possessive construction is particularly useful for disambiguating between the subject and the possessor, if both of them are in third person. The following two pairs of examples show this clearly. The corresponding English sentences are ambiguous, whereas the Mauwake sentences are not: 

\ea%x1797
\label{ex:3:x1797}
\gll Yena eremena=ke \textstyleEmphasizedVernacularWords{ona}  siowa aruf-eya kepura ku-o-k. \\
1s.\textsc{gen} nephew=\textsc{cf} 3s.\textsc{gen} dog hit-2/3s.\textsc{ds} leg break-\textsc{pa}-3s\\
\glt`My nephew\textsubscript{i} hit his\textsubscript{i} dog and its leg broke.'
\z

\ea%x1796
\label{ex:3:x1796}
\gll Yena eremena=ke siowa \textstyleEmphasizedVernacularWords{wiar} aruf-eya kepura ku-o-k. \\
1s.\textsc{gen} nephew=\textsc{cf} dog 3.\textsc{dat} hit-2/3s.\textsc{ds} leg break-\textsc{pa}-3s\\
\glt`My nephew\textsubscript{i} hit his/her\textsubscript{j} dog and its leg broke.'
\z

\ea%x1798
\label{ex:3:x1798}
\gll Wis=ke wiawi maak-e-mik. \\
3p.\textsc{fc}=\textsc{cf} 3s/p.father tell-\textsc{pa}-1/3p\\
\glt`(It was) they\textsubscript{i} (who) told their\textsubscript{i} father.'
\z

\ea%x1799
\label{ex:3:x1799}
\gll Wis=ke wiawi \textstyleEmphasizedVernacularWords{wiar} maak-e-mik. \\
3p.\textsc{fc}=\textsc{cf} 3s/p.father 3.\textsc{dat} tell-\textsc{pa}-1/3p\\
\glt`(It was) they\textsubscript{i} (who) told their\textsubscript{j} father.'
\z

Currently the dative possessive has to be used when the possessor is non-coreferential with the subject or recipient of the clause.

\ea%x588
\label{ex:3:x588}
\gll Marasin nain=ke kema \textstyleEmphasizedVernacularWords{wiar} iw-a-k. \\
medicine that1=\textsc{cf} liver 3.\textsc{dat} go-\textsc{pa}-3s\\
\glt`The medicine went into his liver.'
\z

\ea%x853
\label{ex:3:x853}
\gll Wiowa nain o wapena=pa \textstyleEmphasizedVernacularWords{wiar}  ku-o-k. \\
spear that1 3s.\textsc{unm} hand=\textsc{loc} 3.\textsc{dat} break-\textsc{pa}-3s\\
\glt`The spear broke in his hand.'
\z

\ea%x1794
\label{ex:3:x1794}
\gll Pina ... \textstyleEmphasizedVernacularWords{nefar} kaken-ami welaw-i-kuan. \\
guilt {\dots} 2s.\textsc{dat} straighten-\textsc{ss}.\textsc{sim} finish-Np-\textsc{fu}.3p\\
\glt`They will straighten your(sg) {\dots} guilt and finish it.'
\z

It follows from the non-coreferentiality restriction that a possessed \textstyleAcronymallcaps{NP} with the possessive pronoun in the dative cannot be the subject of a clause. 

In the possessor function the dative pronoun does not co-occur with the accusative pronoun in the same clause \REF{ex:3:x584}. In the rare occasion where there would be rivalry for the position immediately preceding the verb, the accusative is chosen \REF{ex:3:x583} rather than the dative \REF{ex:3:x1928}. 

\ea%x584
\label{ex:3:x584}
\gll *Yena muuka erup \textstyleEmphasizedVernacularWords{efar} \textstyleEmphasizedVernacularWords{wia} aaw-o-k. \\
1s.\textsc{gen} son two 1s.\textsc{dat} 3p.\textsc{acc} take-\textsc{pa}-3s\\
\glt % 
\z

\ea%x583
\label{ex:3:x583}
\gll Yena muuka erup \textstyleEmphasizedVernacularWords{wia} aaw-o-k. \\
1s.\textsc{gen} son two 3p.\textsc{acc} take-\textsc{pa}-3s\\
\glt`He took my two sons.'
\z

\ea%x1928
\label{ex:3:x1928}
\gll ?Yena muuka erup \textstyleEmphasizedVernacularWords{efar} aaw-o-k. \\
1s.\textsc{gen} son two 3.\textsc{dat} take-\textsc{pa}-3s\\
\glt %
\z

But if the dative pronoun has the semantic role of goal, it may co-occur with an accusative pronoun; in this case it precedes the accusative pronoun.

\ea%x1576
\label{ex:3:x1576}
\gll O \textstyleEmphasizedVernacularWords{wiar} \textstyleEmphasizedVernacularWords{nefa} sesek-i-yem. \\
3s.\textsc{unm} 3.\textsc{dat} 2s.\textsc{acc} send-Np-\textsc{pr}.1s\\
\glt`I am sending you to him.'
\z

The use of the genitive possessive pronoun is much less restricted. Besides being employed where the possessor is coreferential with the subject \REF{ex:3:x589} or recipient \REF{ex:3:x590}, it can also be used when a possessed \textstyleAcronymallcaps{NP} is the subject or non-verbal predicate of a descriptive or equative clause \REF{ex:3:x591}. 

\ea%x589
\label{ex:3:x589}
\gll Eema=ke \textstyleEmphasizedVernacularWords{ona} kolos Garamin iw-o-k. \\
Eema=\textsc{cf} 3s.\textsc{gen} dress Garamin give.him/her-\textsc{pa}-3s\\
\glt`Eema\textsubscript{i} gave her\textsubscript{i} dress to Garamin.'
\z

\ea%x590
\label{ex:3:x590}
\gll Eema=ke Garamin \textstyleEmphasizedVernacularWords{ona} kolos iw-o-k. \\
Eema=\textsc{cf} Garamin 3s.\textsc{gen} dress give.him/her-\textsc{pa}-3s\\
\glt`Eema\textsubscript{i} gave Garamin\textsubscript{j} her\textsubscript{j} dress.'
\z

\ea%x591
\label{ex:3:x591}
\gll \textstyleEmphasizedVernacularWords{Yena} koora maneka wenup. \\
1s.\textsc{gen} house big very\\
\glt`My house is very big.'
\z

The genitive or unmarked pronoun may co-occur together with the dative pronoun referring to the same person, thus emphasizing the possessive function of the dative \REF{ex:3:x1863}, \REF{ex:3:x593}.

\ea%x1863
\label{ex:3:x1863}
\gll \textstyleEmphasizedVernacularWords{Yo} emeria \textstyleEmphasizedVernacularWords{efar} uruf-a-man=i e wia? \\
1s.\textsc{unm} woman 1s.\textsc{dat} see-\textsc{pa}-2p=\textsc{qm} or no\\
\glt`Have you seen my wife or not?'
\z

\ea%x593
\label{ex:3:x593}
\gll \textstyleEmphasizedVernacularWords{Ona} koora=pa \textstyleEmphasizedVernacularWords{wiar} wu-a-mik. \\
3s.\textsc{gen} house=\textsc{loc} 3.\textsc{dat} put-\textsc{pa}-1/3p\\
\glt`They put it in his (own) house.'
\z

Example \REF{ex:3:x594} shows how the genitive and dative possessives, in \textstyleEmphasizedWords{\textsc{different}} person forms, can modify the same noun. The dative pronoun can here be interpreted either as a possessive `your (wives)' or as a source `(wives) from you'.

\ea%x594
\label{ex:3:x594}
\gll Emeria ikoka Yaapan \textstyleEmphasizedVernacularWords{wiena} \textstyleEmphasizedVernacularWords{niar} aaw-i-kuan. \\
woman later Japanese 3p.\textsc{gen} 2p.\textsc{dat} take-Np-\textsc{fu}.3p\\
\glt`Later the Japanese will take your wives as their own.'
\z

In the following example, where there are several possessive \textstyleAcronymallcaps{NP}s, the two genitive pronouns both refer to the man who is identified in the preceding text. In the second clause the possessor is a modifier in the subject \textstyleAcronymallcaps{NP}, so it has to be in the genitive. The subject in the third clause is the lover's spirit, and because only one dative possessive is possible in one clause, here it is naturally assigned to the man's wife whose things were thrown around, and the man is referred to by a genitive possessive. In this case the genitive possessive also underlines the fact that one of the women was the man's own wife. The clauses are separated by brackets.

\ea%x1318
\label{ex:3:x1318}
\gll [Ikiw-ep-ik-eya] [\textstyleEmphasizedVernacularWords{ona} soma emeria nain kukusa nain=ke ekap-ep] [\textstyleEmphasizedVernacularWords{ona} emeria nain maa \textstyleEmphasizedVernacularWords{wiar} wafufur-eya] [naap maak-e-k,] {\dots}\\
go-\textsc{ss}.\textsc{seq}-be-2/3s.\textsc{ds} 3s.\textsc{gen} lover woman that1 spirit that1=\textsc{cf} come-\textsc{ss}.\textsc{seq} 3s.\textsc{gen} woman that1 thing 3.\textsc{dat} throw.around-2/3s.\textsc{ds} thus tell-\textsc{pa}-3s\\
\glt`When he\textsubscript{i} was gone, his\textsubscript{i} lover-woman's\textsubscript{j} spirit came and threw around his\textsubscript{i} (own) wife's\textsubscript{k} things, and she\textsubscript{k} told her like this, {\dots}'
\z

Dative pronouns also have a longer form, with the suffix -\textstyleStyleVernacularWordsItalic{ik}: \textstyleStyleVernacularWordsItalic{efarik}, \textstyleStyleVernacularWordsItalic{nefarik} etc. The pronoun is a contracted form of the `have' construction, with just the root left of the verb \textstyleStyleVernacularWordsItalic{ik}- `be', which has been suffixed to the pronoun. In natural text the frequency of these pronouns is extremely low. They have to be used when the dative pronoun is clause final \REF{ex:3:}-\REF{ex:3:}, as the regular dative pronoun only occurs pre-verbally. The longer form is often accompanied by either the genitive pronoun \REF{ex:3:} or the unmarked pronoun \REF{ex:3:}, which suggests that it is more emphasized than the simple dative.

\ea%x597
\label{ex:3:x597}
\gll Miiw ara gelemuta nain \textstyleEmphasizedVernacularWords{yiena} \textstyleEmphasizedVernacularWords{yiarik}. \\
land piece small that1 1p.\textsc{gen} 1p.\textsc{dat}\\
\glt`That small piece of ground is ours.'
\z

\ea%x598
\label{ex:3:x598}
\gll Wiawi=ke amap-or-o-k=i, weke \textstyleEmphasizedVernacularWords{wiarik}? \\
3s/p.father=\textsc{cf} Bpx-descend-\textsc{pa}-3s=\textsc{qm} 3s/p.grandfather 3.\textsc{dat}\\
\glt`Did her father take her down to her grandfather?'
\z

The long dative with a ``receive'' type verb in the following example can be traced back to \textstyleStyleVernacularWordsxiiptItalic{niar ikeya} `you had it, and{\dots}':

\ea%x596
\label{ex:3:x596}
\gll Yo mesa up-owa fain \textstyleEmphasizedVernacularWords{ni} \textstyleEmphasizedVernacularWords{niarik} aaw-ep isak-e-m.\\
1s.\textsc{unm} winged.bean plant-\textsc{nmz} this 2p.\textsc{unm} 2p.\textsc{dat} get-\textsc{ss}.\textsc{seq} plant-\textsc{pa}-1s\\
\glt`I got these winged bean seeds from you and planted them.'
\z

\subsection{Isolative pronouns}\label{sec:3:y:x}
%\hypertarget{RefHeading19681935131865}
{}
The isolative pronoun forms are based on the unmarked pronouns. The ending \nobreakdash-\textstyleStyleVernacularWordsItalic{isow}\textstyleStyleVernacularWordsItalic{,} which the numeral \textstyleStyleVernacularWordsItalic{kuisow} `one' shares with these pronouns, may be an earlier morpheme possibly meaning `alone'. The meaning of the isolative pronouns is roughly `X alone' or `by -self'. In the singular forms the vowel /o/ is replaced by /a/, since /oi/ is not a permissible vowel sequence in Mauwake. 

\begin{table}
\caption{Please provide a caption}
\label{} 
\begin{tabular}{lll}
\mytoprule
&singular &plural\\
\midrule
1 &ya-isow &(y)i-isow\\
2 &na-isow &ni-isow\\
3 &wa-isow &wi-isow\\
\mybottomrule
\end{tabular}
\end{table}


When an isolative pronoun functions as a subject, which is \textstyleEmphasizedWords{\textsc{not}} theme (\sectref{sec:9.1}), it is alone \REF{ex:3:x599}; but more commonly it is both theme and subject, and is preceded by the unmarked pronoun also showing the case marking overtly \REF{ex:3:x600}. 

\ea%x599
\label{ex:3:x599}
\gll Manina \textstyleEmphasizedVernacularWords{waisow} mauw-ap neeke wu-a-k. \\
garden 3s.ISOL work-\textsc{ss}.\textsc{seq} there.\textsc{cf} put-\textsc{pa}-3s\\
\glt`He made his garden alone/by himself and left it there.'
\z

\ea%x600
\label{ex:3:x600}
\gll \textstyleEmphasizedVernacularWords{No} \textstyleEmphasizedVernacularWords{naisow} or-op kaul wafur-e. \\
2s.\textsc{unm} 2s.ISOL descend-\textsc{ss}.\textsc{seq} hook throw-\textsc{imp}.2s\\
\glt`Go down alone/by yourself and do fishing (lit: throw the hook).'
\z

The example \REF{ex:3:x601} has an accusative pronoun to show the case and an initial unmarked pronoun \textstyleStyleVernacularWordsxiiptItalic{yo} `I' to mark the object as theme.

\ea%x601
\label{ex:3:x601}
\gll \textstyleEmphasizedVernacularWords{Yo} \textstyleEmphasizedVernacularWords{yaisow} me \textstyleEmphasizedVernacularWords{efa} keraw-a-k. \\
1s.\textsc{unm} 1s.ISOL not 1s.\textsc{acc} bite-\textsc{pa}-3s\\
\glt`It didn't bite only me.' (Or: `It wasn't only me that it bit.')
\z

When the isolative pronoun is preceded by the genitive/emphatic pronoun it is intensified:

\ea%x1813
\label{ex:3:x1813}
\gll Aakisa mua iperowa nain \textstyleEmphasizedVernacularWords{ona} \textstyleEmphasizedVernacularWords{waisow} soor owowa=pa ika-i-ya.\\
now man middle.aged that1 3s.\textsc{gen} 3s.ISOL jungle village=\textsc{loc} be-Np-\textsc{pr}.3s\\
\glt`Now that middle-aged man is staying all by himself in a jungle hamlet.'
\z

In the plural the meaning is `\textstyleEmphasizedWords{\textsc{only}} we/you/they (as a \textstyleEmphasizedWords{\textsc{group}})'. 

\ea%x602
\label{ex:3:x602}
\gll Wi feeke ika-uk, \textstyleEmphasizedVernacularWords{i} \textstyleEmphasizedVernacularWords{iisow} ikiw-i-yen. \\
3p.\textsc{unm} here.\textsc{cf} be-\textsc{imp}.3p 1p.\textsc{unm} 1p.ISOL go-Np-\textsc{fu}.1p\\
\glt`Let them stay here, only we will go.'
\z

When the first syllable of a plural isolative pronoun is reduplicated, the pronoun refers to \textstyleEmphasizedWords{\textsc{individuals}} in the group:

\ea%x603
\label{ex:3:x603}
\gll \textstyleEmphasizedVernacularWords{Ii-iisow} pok-ap opora siisim-ep weeser-eya unow=iya aakun-e-mik.\\
\textsc{rdp}-1p.ISOL sit-\textsc{ss}.\textsc{seq} talk write-\textsc{ss}.\textsc{seq} finish-2/3s.\textsc{ds} many=\textsc{com} talk-\textsc{pa}-1/3p\\
\glt`We sat and wrote separately, and then talked together.'
\z

Although the pronouns in \REF{ex:3:} and \REF{ex:3:} sound rather similar, there is a stress difference between them. In the former, \textstyleStyleVernacularWordsItalic{iisow} gets stronger stress than \textstyleStyleVernacularWordsItalic{i}, in the latter the first syllable of the reduplicated word is stressed.

\subsection{Restrictive pronouns}\label{sec:3:y:x}
%\hypertarget{RefHeading19701935131865}
{}
The restrictive pronouns are formed by adding the limiting clitic -\textstyleStyleVernacularWordsItalic{iw} `only' either to a genitive pronoun or to a focal pronoun (\sectref{sec:3.12.6}). When it is added to a genitive pronoun it means `on one's own':

\ea%x605
\label{ex:3:x605}
\gll No \textstyleEmphasizedVernacularWords{nena=iw} ma-i-n=i? \\
2s.\textsc{unm} 2s.\textsc{gen}=\textsc{lim} say-Np-\textsc{pr}.2s=\textsc{qm}\\
\glt`Do you say it on your own?' (i.e. `Did you think of it yourself?')
\z

\ea%x606
\label{ex:3:x606}
\gll O=ko me efa maak-e-k, \textstyleEmphasizedVernacularWords{yena=iw} amis-ar-e-m.\\
3s.\textsc{unm}=\textsc{nf} not 1s.\textsc{acc} tell-\textsc{pa}-3s 1s.\textsc{gen}=\textsc{lim} knowledge-\textsc{inch}-\textsc{pa}-1s\\
\glt`He/she didn't tell me, I learned it on my own.'
\z

\ea%x607
\label{ex:3:x607}
\gll \textstyleEmphasizedVernacularWords{Yien=iw} ikiw-ik-ua. \\
1p.\textsc{gen}=\textsc{lim} go-be-\textsc{pa}.3s\\
\glt`Let's go on our own (as a group, or one by one).'
\z

When the limiting clitic is added to the focal form of the free pronoun it adds the meaning of exclusiveness to the pronoun:

\ea%x608
\label{ex:3:x608}
\gll Anane \textstyleEmphasizedVernacularWords{nos=iw} nefa maak-i-ya. \\
always 2s.\textsc{fc}=\textsc{lim} 2s.\textsc{acc} tell-Np-\textsc{pr}.3s\\
\glt`He always talks to you only.'
\z

\ea%x609
\label{ex:3:x609}
\gll Wi anane \textstyleEmphasizedVernacularWords{is=iw} yiam=iya irak-i-mik. \\
3p.\textsc{unm} always 1p.\textsc{fc}=\textsc{lim} 1p.\textsc{refl}=\textsc{com} fight-Np-\textsc{pr}.1/3p\\
\glt`They always fight with us only.'
\z

\subsection{Reflexive-reciprocal pronouns}\label{sec:3:y:x}
%\hypertarget{RefHeading19721935131865}
{}
The reflexive-reciprocal pronouns have the unmarked pronouns as their basis, but the derivative suffix is slightly different for singular and plural. They are as follows:

\begin{table}
\caption{Please provide a caption}
\label{} 
\begin{tabular}{lll}
\mytoprule
 &singular &plural\\
\midrule
1 &y-ame\footnote{In the coastal dialect the singular suffix is -\textit{ama}.} &yi-am\\
2 &n-ame &ni-am\\
3 &w-ame &wi-am\\
\mybottomrule
\end{tabular}
\end{table}


The singular forms are used as reflexives only \REF{ex:3:x610}, \REF{ex:3:x1864}, the plural forms both as reflexives \REF{ex:3:x611}, \REF{ex:3:x1865} and as reciprocals \REF{ex:3:x612}, \REF{ex:3:x1866}.

\ea%x610
\label{ex:3:x610}
\gll Naap on-ap \textstyleEmphasizedVernacularWords{yame} amukar-e-m. \\
thus do-\textsc{ss}.\textsc{seq} 1s.\textsc{refl} scold-\textsc{pa}-1s\\
\glt`Having done so I scolded myself (i.e. was angry at myself).'
\z

\ea%x1864
\label{ex:3:x1864}
\gll Iinan akena ikiw-ep \textstyleEmphasizedVernacularWords{wame} pipilim-ep aakun-em-ika-i-non.\\
on.top very go-\textsc{ss}.\textsc{seq} 3s.\textsc{refl} hide-\textsc{ss}.\textsc{seq} speak-\textsc{ss}.\textsc{sim}-be-Np-\textsc{fu}.3s\\
\glt`It (= a bird) will go very high up and hide itself and keep making its calls.'
\z

\ea%x611
\label{ex:3:x611}
\gll \textstyleEmphasizedVernacularWords{Niam} tuun-ap teeria erup wu-eka. \\
2p.\textsc{refl} count-\textsc{ss}.\textsc{seq} group two put-\textsc{imp}.2p\\
\glt`Count yourselves and form two groups.'
\z

\ea%x1865
\label{ex:3:x1865}
\gll Nainiw sande uura \textstyleEmphasizedVernacularWords{yiam} fiirim-e-mik. \\
again Sunday night 1p.\textsc{refl} gather-\textsc{pa}-1/3p\\
\glt`Again on Sunday night we gathered.'
\z

\ea%x612
\label{ex:3:x612}
\gll \textstyleEmphasizedVernacularWords{Wiam} fook-ap irak-e-mik. \\
3p.\textsc{refl} split-\textsc{ss}.\textsc{seq} fight-\textsc{pa}-1/3p\\
\glt`They split from each other and fought.'
\z

\ea%x1866
\label{ex:3:x1866}
\gll Sarir-ap {\dots } \textstyleEmphasizedVernacularWords{yiam} far-i-mik. \\
surround-\textsc{ss}.\textsc{seq} {\dots} 1p.\textsc{refl} call-Np-\textsc{pr}.1/3p\\
\glt`We surround (the fish) {\dots} and call each other.'
\z

In many contexts only the reflexive or the reciprocal interpretation is natural. But a potential ambiguity in some contexts is resolved by adding a genitive pronoun to mark the reflexive \REF{ex:3:x614} and an unmarked or restrictive pronoun to mark the reciprocal pronoun \REF{ex:3:x613}.

\ea%x614
\label{ex:3:x614}
\gll \textstyleEmphasizedVernacularWords{Niena} \textstyleEmphasizedVernacularWords{niam} kookal-eka. \\
2p.\textsc{gen} 2p.\textsc{refl} like-\textsc{imp}.2p\\
\glt`Like/love yourselves.'
\z

\ea%x613
\label{ex:3:x613}
\gll \textstyleEmphasizedVernacularWords{Ni/nieniw} \textstyleEmphasizedVernacularWords{niam} kookal-eka. \\
2p.\textsc{unm}/2p.LIM 2p.\textsc{refl} like-\textsc{imp}.2p\\
\glt`Like/love each other.'
\z

The reflexives are not very frequent in Mauwake, because they seem to be fairly strongly connected with [+Control]. If one hurts oneself unintentionally, the cause(r) or instrument occupies the subject position instead of the person hurt. Thus, \REF{ex:3:x617} is a semantically appropriate equivalent for the English clause `I cut myself with a knife':

\ea%x617
\label{ex:3:x617}
\gll Fura=ke efa puuk-a-k. \\
knife=\textsc{cf} 1s.\textsc{acc} cut-\textsc{pa}-3s\\
\glt`A knife cut me.'
\z

But a reflexive pronoun is used especially in expressions involving \textstyleParagraphChari{body parts} when one does something to oneself, and the instrument is not known or mentioned \REF{ex:3:x618}. In corresponding expressions English often uses possessive rather than reflexive pronouns.

\ea%x618
\label{ex:3:x618}
\gll Merena \textstyleEmphasizedVernacularWords{yame} puuk-a-m. \\
leg 1s.\textsc{refl} cut-\textsc{pa}-1s\\
\glt`I cut my leg.' (Or: `I cut myself in the leg.')
\z

The plural forms of the reflexive pronouns have another, quite different use: when they are followed by numerals, especially by `two' or `three', they function as dual/trial etc. forms for the personal pronouns. They are considered to be in the nominative case when not followed by other pronoun forms \REF{ex:3:x615}. Other cases need to be shown by appropriate additional pronouns \REF{ex:3:x616}.

\ea%x615
\label{ex:3:x615}
\gll \textstyleEmphasizedVernacularWords{Yiam} \textstyleEmphasizedVernacularWords{arow} nain miim-ap soran-e-mik. \\
1p.\textsc{refl} three that1 hear-\textsc{ss}.\textsc{seq} be.startled-\textsc{pa}-1/3p\\
\glt`The three of us heard that and were startled.'
\z

\ea%x616
\label{ex:3:x616}
\gll Amia mua=ke \textstyleEmphasizedVernacularWords{wiam} \textstyleEmphasizedVernacularWords{erup} nain \textstyleEmphasizedVernacularWords{wia} nokar-e-k, {\dots} \\
bow man=\textsc{cf} 3p.\textsc{refl} two that1 3p.\textsc{acc} ask-\textsc{pa}-3s\\
\glt`The policeman asked those two {\dots}'
\z

\subsection{Comitative pronouns}\label{sec:3:y:x}
%\hypertarget{RefHeading19741935131865}
{}
The comitative set is a mixture as far as the basic forms are concerned. The first and second person singular forms have accusative pronouns, all the others have the reflexive pronouns as their roots. The ending is the comitative clitic -\textstyleStyleVernacularWordsItalic{iya} (\sectref{sec:3.12.1}), which can also be added to nouns and is one of several ways of expressing accompaniment in Mauwake. The first and second person singular forms have a transition consonant -\textstyleStyleVernacularWordsItalic{m}- preceding the comitative clitic.

\begin{table}
\caption{Please provide a caption}
\label{} 
\begin{tabular}{lll}
\mytoprule
 &singular &plural\\
\midrule
1 &efa-m-iya &yiam-iya\\
2 &nefa-m-iya &niam-iya\\
3 &wama-iya &wiam-iya\\
\mybottomrule
\end{tabular}
\end{table}


\ea%x619
\label{ex:3:x619}
\gll Lasen mua emeria \textstyleEmphasizedVernacularWords{wiam=iya} me aakun-e-mik. \\
Lasen man woman 3p.\textsc{refl}=\textsc{com} not talk-\textsc{pa}-1/3p\\
\glt`We didn't talk with the Lasen people.'
\z

\ea%x620
\label{ex:3:x620}
\gll Liisa Poh San ikos \textstyleEmphasizedVernacularWords{yiam=iya} soomar-emi {\dots} \\
Liisa Poh San with 1p.\textsc{refl}=\textsc{com} walk-\textsc{ss}.\textsc{sim}\\
\glt`Liisa and Poh San walked with us and {\dots}'
\z

\subsection{Primary and secondary reference of personal pronouns}\label{sec:3:y:x}
%\hypertarget{RefHeading19761935131865}
{}
Typically pronouns refer to the persons the form indicates: first person singular to the speaker, second person singular to the addressee etc. Besides this primary, or default, reference some pronouns may also have a secondary reference, if the person and/or number of the referent(s) is different from that indicated by the pronoun.

In Mauwake both the first and second person singular forms as well as the third person plural marking on verbs can be used for non-specific, or generic, reference. They occur particularly in explanations of customs or general principles, and in examples. The sentences are usually in the future tense and therefore hypothetical. In these texts the second person singular pronoun and the third person verb marking can alternate quite freely. Example \REF{ex:3:x621} is from a text describing the adoption process in general, and example \REF{ex:3:x622} was said to a person who does not even have a spirit name to call upon, nor does know how to spear pigs. Here the pronouns have acquired a non-deictic role: their correct interpretation does not depend on the non-linguistic context \citep[260]{AndersonEtAl1985}%Keenan
.

\ea%x621
\label{ex:3:x621}
\gll \textstyleEmphasizedVernacularWords{Yo} muuka kookal-ep \textstyleEmphasizedVernacularWords{yena} samapora wia maak-i-nen.\\
1s.\textsc{unm} son like-\textsc{ss}.\textsc{seq} 1s.\textsc{gen} clan 3p.\textsc{acc} tell-Np-\textsc{fu}.1s\\
\glt`When I like to have a son/child I will tell my clan.' (Or: `When \textit{one} wants a child he will tell his own clan.')
\z

\ea%x622
\label{ex:3:x622}
\gll \textstyleEmphasizedVernacularWords{No} waaya mik-ap inasina unuma me unuf-i-nan=na mua oko=ke nainiw mik-ap \textstyleEmphasizedVernacularWords{nefar} aaw-i-non.\\
2s.\textsc{unm} pig spear-\textsc{ss}.\textsc{seq} spirit name not call-Np-\textsc{fu}.2s=\textsc{tp} man other=\textsc{cf} again spear-\textsc{ss}.\textsc{seq} 2s.\textsc{dat} take-Np-\textsc{fu}.3s\\
\glt`If you spear a pig and don't call your spirit name, another man will spear it again and take it from you.' (Or: `If \textit{one} spears a pig{\dots}')
\z

When a maximally generic object is needed for a transitive verb, or when there is no overt object available, the first person plural accusative form is used. 

\ea%x623
\label{ex:3:x623}
\gll Ifa nain=ke \textstyleEmphasizedVernacularWords{yia} keraw-i-ya. \\
snake that1=\textsc{cf} 1p.\textsc{acc} bite-Np-\textsc{pr}.3s\\
\glt`That snake bites.'
\z

\ea%x624
\label{ex:3:x624}
\gll Marasin fain \textstyleEmphasizedVernacularWords{yia} girin-i-ya. \\
medicine this 1p.\textsc{acc} smart-Np-\textsc{pr}.3s\\
\glt`This medicine smarts.'
\z

\subsection{Use of personal pronouns in text}\label{sec:3:y:x}
%\hypertarget{RefHeading19781935131865}
{}
In Mauwake it is possible to leave the subject pronoun out, as the person and number of the subject are marked on the verb suffix. And this is not only possible but very common: approximately only 6\% of all the clauses in narrative and descriptive texts have a pronominal subject, compared to about 30\% of the clauses having a subject \textstyleAcronymallcaps{NP} of any kind. As the other arguments are not marked on the verb, except for a two-way distinction for beneficiary (\sectref{sec:3.7.3.1}), other than subject pronouns need to be used for them if there is no full \textstyleAcronymallcaps{NP}, and they are often employed even when there is a \textstyleAcronymallcaps{NP}.

The frequency of subject pronouns depends on whether the person referred to is first, second or third, and on the type of text as well. The first person, both in singular and plural, is commonly referred to with a pronoun, instead of just a verb suffix. Second person pronouns are very frequent in hortatory texts and are used somewhat in conversations. Most narratives in the data have their main participants in third person, but pronouns are used to refer to them quite rarely. 

A pronoun may be used for the second mention of a newly established topic (\sectref{sec:9.2.2}). In particular when an important participant has been introduced by a proper name, in the next sentence (s)he can be referred to by a personal pronoun. 

\ea%x1867
\label{ex:3:x1867}
\gll Eema=ke waisow amis-ar-e-k. \textstyleEmphasizedVernacularWords{Os=ke} uuriw urup-emi{\dots}\\
Eema=\textsc{cf} 3s.ISOL knowledge-\textsc{inch}-\textsc{pa}-3s 3s.\textsc{fc}=\textsc{cf} morning rise-\textsc{ss}.\textsc{sim}\\
\glt`Only Eema knew. She got up in the morning and {\dots}'
\z

When a participant has been established as the topic, (s)he is referred to with a verb suffix only, or with a \textstyleAcronymallcaps{NP} if a better identification is needed. A pronoun is used mainly when the topic is re-activated after being inactive for a while (\sectref{sec:9.2.3}). The example \REF{ex:3:x1922} is from a text where a couple goes down to the husband's village and then returns to the wife's village. The wife's relatives, inactive as a topic for the span of five clauses, are re-assigned the topic status with the pronoun \textstyleStyleVernacularWordsItalic{wi} `they'. 

\ea%x1922
\label{ex:3:x1922}
\gll Or-op ik-ok nainiw urup-e-mik. Aria \textstyleEmphasizedVernacularWords{wi} samapora maneka fook-ap {\dots}\\
descend-\textsc{ss}.\textsc{seq} be-SS again ascend-\textsc{pa}-1/3p alright 3p.\textsc{unm} floor big split-\textsc{ss}.\textsc{seq} \\
\glt`They (=the couple) went down and after a while they came up again. Alright they (=the wife's relatives) split (wood for) a big floor and {\dots}'
\z

In commands (\sectref{sec:7.3}) the subject pronouns are more frequent than in statements.\footnote{As many as 39\% of commands in the text material have a pronoun subject, as against 6\% in statements.} The pronoun here is not a vocative; that would be separated from the rest of the clause by a pause, whereas a subject is not. The following is a fairly typical command:

\ea%x685
\label{ex:3:x685}
\gll Ni ikiw-eka! \\
2p go-\textsc{imp}.2p\\
\glt`Go (2p)!'
\z

This is an unusual feature cross-linguistically, as languages tend to drop the subject pronoun in imperative clauses \citep[80]{Givon1979}.\footnote{The relatively high frequency of subject pronouns in imperative clauses may not be a peculiarity of Mauwake only. The grammatical descriptions of Papuan languages often state that the subject pronoun is optional in these clauses, but give no information as to their actual frequency. Personal communication with other field linguists working on Papuan languages gives reason to suggest that an overt personal pronoun with the imperative may be more common than is generally assumed.} 

\section{Spatial deictics}\label{sec:3:6}
%\hypertarget{RefHeading19801935131865}
{}
This section brings together what are often called demonstrative pronouns and deictic locative adverbs. What is common to them is the spatial orientation based on the location of the speaker, as well as morphological similarity. The whole deictic system, which also includes personal and temporal deixis, is discussed briefly in 6.3. 

Deictics operate on the scale of proximity, making reference to something else on the basis of location \citep[57--58]{HallidayEtAl1976}. The relative proximity may be measured either from the speaker or from the speaker and addressee. Papuan languages manifest both these types as well as a combination of the two. Elevation and visibility may be additional parameters, so the demonstrative systems range from a simple and rather common two-term system to quite complicated ones \citep[75--77]{Foley1986}. Two-way distinctions are found in Siroi \citep[20]{Wells1979} and Golin \citep{Bunn1974}, three-way distinctions in Waskia \citep[59]{RossEtAl1978}%Paol
, Bine \citep{Saari1985} and Korafe \citep[65]{FarrEtAl1981}%Whitehead
. Usan has four basic deictics, but derivations extend the system into an elaborate one \citep[76--81]{Reesink1987}. \citet[38--39]{Murane1974} reports 19 locatives in Daga that are also used as demonstrative pronouns. 

\subsection{The basic spatial deixis in Mauwake}\label{sec:3:y:x}
%\hypertarget{RefHeading19821935131865}
{}
The main factors dividing the deictic space in Mauwake are the relative proximity to the speaker, and visibility. There are four deictic roots, one of them proximal and three distal. They are as follows: 

\begin{table}
\caption{Please provide a caption}
\label{} 
\begin{tabular}{lll}
\mytoprule
fa- &`here' (close to speaker, visible) &proximate\\
na- &`there' (away from the speaker; generic) &distal-1\\
eef- &`here/there' (rather close, usually visible) &distal-2\\
een- &`there' (far away, usually not visible) &distal-3\\
\mybottomrule
\end{tabular}
\end{table}


The proximal deictic \textstyleStyleVernacularWordsItalic{fa}- indicates close proximity to the speaker: prototypically the referent marked with \textstyleStyleVernacularWordsItalic{fain} `this' can be touched by the speaker, and \textstyleStyleVernacularWordsItalic{fan} `here' indicates the speaker's location or close proximity to it. The distal-1 deictic \textstyleStyleVernacularWordsItalic{na}\textit{-} indicates a distance that is out of touching distance to the speaker; the distance to the addressee is irrelevant. \textstyleStyleVernacularWordsItalic{Na}- is the most neutral and the least restricted of the three distal deictics, and its frequency is extremely high because of the various functions that the demonstrative \textstyleStyleVernacularWordsItalic{nain} has. On the other hand, the words formed with both the distal\nobreakdash-2 root \textstyleStyleVernacularWordsItalic{eef}\textit{\nobreakdash-} and the distal\nobreakdash-3 root \textstyleStyleVernacularWordsItalic{een}\nobreakdash-, although available, are rarely used. They may be employed when the pragmatic situation meets the semantic specification for their occurrence, and they are needed when more than one far deictic is called for. Often the distance is a relative matter, and the speaker has a subjective choice between the different deictics.

The deictic roots suffixed with -\textstyleStyleVernacularWordsItalic{in}, marking given information, are used as demonstratives. When the roots are suffixed with -\textstyleStyleVernacularWordsItalic{an} `locative', the words function as locative adverbs. The distribution of both these suffixes is very restricted: they are only attached to deictic or question word (\sectref{sec:3.7.1}) roots.

The deictic manner adverbs (\sectref{sec:3.6.4}) are also based on the same roots.

\subsection{Demonstratives}\label{sec:3:y:x}
%\hypertarget{RefHeading19841935131865}
{}
The four demonstratives in Mauwake are formed by one of the deictic roots plus the suffix -\textstyleStyleVernacularWordsItalic{in} indicating given information. 

In Mauwake the demonstratives are like the personal pronouns in that they can function as the sole head of a \textstyleAcronymallcaps{NP}. But they differ from the personal pronouns in that they do not have the case forms typical of the latter. In this respect the demonstratives are more like adjectives. Another feature that they share with adjectives is that they mainly function as modifiers in a \textstyleAcronymallcaps{NP}. But unlike the adjectives, which only occur alone in complement position (unless the \textstyleAcronymallcaps{NP} is elliptical), the demonstratives occur by themselves in several clause positions. 

The numeral modifiers are positioned between an adjective and a demonstrative in a \textstyleAcronymallcaps{NP} \REF{ex:3:x631}, but never between two adjectives \REF{ex:3:x632}.

\ea%x631
\label{ex:3:x631}
\gll koora maneka arow \textstyleEmphasizedVernacularWords{nain} \\
house big three that1\\
\glt`those three big houses'
\z

\ea%x632
\label{ex:3:x632}
\gll siowa sepa gelemuta erup \\
dog black small two\\
\glt`two small black dogs'
\z

There is a clear distinction in Mauwake between human and non-human reference, which shows in the choice of a pronoun vs. a demonstrative. A third person pronoun is not used for non-humans, whereas demonstratives in isolation\footnote{Demonstratives are common as \textit{modifiers} of NPs referring to humans.} are normally only used for non-humans. The only exception in my data is example \REF{ex:3:x633}; \textstyleStyleVernacularWordsItalic{nain} `that' would not be acceptable even here.

\ea%x633
\label{ex:3:x633}
\gll No{\footnotemark} \textstyleEmphasizedVernacularWords{fain} me nena niawi akena=ke. \\
2s.\textsc{unm} this not 2s.\textsc{gen} 2s/p.father true=\textsc{cf}\\
\glt`This is not your true father.'
\z

\footnotetext{\textit{No} `you' is an extra-clausal theme, not part of the subject.} 

Apart from the proximal demonstrative \textstyleStyleVernacularWordsItalic{fain} `this', the other demonstratives are not mutually exclusive. The distal-1 demonstrative \textstyleStyleVernacularWordsItalic{nain} `that' is the least restricted of the three, and it is extremely frequent, whereas both \textstyleStyleVernacularWordsItalic{eefin} `this/that' and \textstyleStyleVernacularWordsItalic{eenin} `that' are very rarely used. In \REF{ex:3:x1749} the distances of the two mountains fit the specifications for \textstyleStyleVernacularWordsItalic{eefin} and \textstyleStyleVernacularWordsItalic{eenin} , and more than one distal demonstrative is needed for contrastive purposes:

\ea%x1749
\label{ex:3:x1749}
\gll Ema \textstyleEmphasizedVernacularWords{eenin} fikera=ke aw-o-k, aria \textstyleEmphasizedVernacularWords{eefin} fikera=ke me aw-o-k.\\
mountain that3 kunai.grass=\textsc{cf} burn-\textsc{pa}-3s, alright that2 kunai.grass=\textsc{cf} not burn-\textsc{pa}-3s\\
\glt`The kunai grass on that mountain (far away, invisible) burned, but the grass on this/that one (somewhat closer) did not burn.'
\z

There is no number distinction in demonstratives. When they modify a [+human] noun, plurality is shown in the person/number marking of the verb and optionally by an additional personal pronoun.

\ea%x635
\label{ex:3:x635}
\gll (\textstyleEmphasizedVernacularWords{Wi}) takira \textstyleEmphasizedVernacularWords{fain=ke} niir-e-mik. \\
3p.\textsc{unm} boy this=\textsc{cf} play-\textsc{pa}-1/3p\\
\glt`It was these boys that played.'
\z

With [-human] nouns, a quantifier in the \textstyleAcronymallcaps{NP} may be used \REF{ex:3:x636}, or distributive suffix on the verb \REF{ex:3:x637} to indicate plurality, or the number may be left unspecified \REF{ex:3:x638}.

\ea%x636
\label{ex:3:x636}
\gll Mera \textstyleEmphasizedVernacularWords{arow} \textstyleEmphasizedVernacularWords{nain} aaw-e-m. \\
fish three that1 get-\textsc{pa}-1s\\
\glt`I caught those three fish.'
\z

\ea%x637
\label{ex:3:x637}
\gll Mera \textstyleEmphasizedVernacularWords{nain} aaw-\textstyleEmphasizedVernacularWords{omak}-e-m. \\
fish that1 get-\textsc{distr}.\textsc{pl}-\textsc{pa}-1s\\
\glt`I caught those (many) fish.'
\z

\ea%x638
\label{ex:3:x638}
\gll Amina \textstyleEmphasizedVernacularWords{fain} p-ekap-e-mik. \\
pot this Bpx-come-\textsc{pa}-1/3p\\
\glt`We brought this pot / these pots.'
\z

Besides the exophoric (text-external) deictic use described above, another common function for demonstratives cross-linguistically is endophoric, or text-internal anaphoric and cataphoric reference. The proximity in the case of demonstratives relates to the participants in the text, rather than the speech situation \citep[278]{Lyons1968}. 

Mauwake follows the typical pattern: the neutral distal demonstrative \textstyleStyleVernacularWordsItalic{nain} `that' is anaphoric: it only refers to the text preceding it, as in \REF{ex:3:x639}, where the example sentence comes after the description of fishing with a fish trap. The proximal \textstyleStyleVernacularWordsItalic{fain} `this'is cataphoric, referring to the text following it \REF{ex:3:x640}. The other two demonstratives, \textstyleStyleVernacularWordsItalic{eefin} and \textstyleStyleVernacularWordsItalic{eenin}, are not used for text-internal reference at all.

\ea%x639
\label{ex:3:x639}
\gll \textstyleEmphasizedVernacularWords{Nain} soo era=ke. \\
that1 fish.trap way=\textsc{cf}\\
\glt`That is the way (to catch fish) with a fish trap.'
\z

\ea%x640
\label{ex:3:x640}
\gll Mua arow \textstyleEmphasizedVernacularWords{fain}: Kuuten, Dogimaw, aria Olas {\dots} \\
man three this: Kuten, Dogimaw, alright Olas\\
\glt`These three men: Kuuten, Dogimaw and Olas {\dots}'
\z

The demonstrative \textstyleStyleVernacularWordsItalic{nain} `that' marks given/established information, and often has a similar function to a definite article (cf. Dryer 2007c:154). It has an important pragmatic function of marking topic continuity in Mauwake. A continuing [+human] topic, especially the main participant, is usually marked only by person/number inflection on the verb, whereas a minor participant or a [-human] established topic uses \textstyleAcronymallcaps{NP}s modified by \textstyleStyleVernacularWordsItalic{nain}.

Still another function for the demonstrative \textstyleStyleVernacularWordsItalic{nain} `that' is that of a nominaliser of otherwise finite verbal clauses (\sectref{sec:5.7.2}). A nominalized clause of this type may be a relative clause \REF{ex:3:x687} (\sectref{sec:8.3.1}), a complement clause \REF{ex:3:x689} (\sectref{sec:8.3.2}) or a temporal subordinate clause \REF{ex:3:x688} (\sectref{sec:8.3.3.1}).\footnote{All these clauses have a function that is consistent with the core meaning of `givenness' \citep{Haiman1978} or presupposition \citep{Reesink1987}.}

\ea%x687
\label{ex:3:x687}
\gll [Merena ifa keraw-a-k \textstyleEmphasizedVernacularWords{nain}]\textsubscript{RC} puuk-a-mik. \\
leg snake bite-\textsc{pa}-3s that1 cut-\textsc{pa}-1/3p\\
\glt`They cut the leg that the snake had bitten.'
\z

\ea%x689
\label{ex:3:x689}
\gll [Mukuna kerer-e-k \textstyleEmphasizedVernacularWords{nain}]\textsubscript{CC} i me paayar-e-mik. \\
fire start-\textsc{pa}-3s that1 1p.\textsc{unm} not understand-\textsc{pa}-1/3p\\
\glt`We didn't know that a fire had started.'
\z

\ea%x688
\label{ex:3:x688}
\gll [Goron-ep ora-i-ya \textstyleEmphasizedVernacularWords{nain},] maa muutitik iiwawun lalat-i-ya.\\
fall-\textsc{ss}.\textsc{seq} descend-Np-\textsc{pr}.3s that1 thing all.kinds altogether sweep-Np-3s\\
\glt`When it goes down, it sweeps everything with it.'
\z

The same demonstrative is also used as a strong adversative `but' \REF{ex:3:x690} (\sectref{sec:8.1.3}). In that function it is placed clause-initially rather than clause-finally.

\ea%x690
\label{ex:3:x690}
\gll Wiawi eliw naak-e-k, \textstyleEmphasizedVernacularWords{nain} me ikiw-o-k. \\
3s/p.father all.right say-\textsc{pa}-3s that1 not go-\textsc{pa}-3s\\
\glt`He said yes (lit: all right) to his father, but didn't go.'
\z

\subsection{Deictic locative adverbs} \label{sec:3:y:x}
%\hypertarget{RefHeading19861935131865}
{}
The undebatable locative adverbs in Mauwake are all deictic (\sectref{sec:3.9.1.1}). For each of the four deictic roots there are two corresponding locative adverbs. The first set contains the deictic root and the locative suffix -\textstyleStyleVernacularWordsItalic{an}. The homorganic vowels in the root and affix have merged into one. The second set is suffixed with the contrastive focus clitic -(\textstyleStyleVernacularWordsItalic{e})\textstyleStyleVernacularWordsItalic{ke}. When the clitic is added, the deictic adverb is in focus, but not necessarily contrastive. The morphophonological change that has taken place in the root is unusual: the vowel /a/ has assimilated with the initial /e/ of the contrastive focus clitic. 

\begin{table}
\caption{Please provide a caption}
\label{} 
\begin{tabular}{lll}
\mytoprule
Adv &Adv + CF&\\
\midrule
fa-an{\textgreater}fan &fa-eke{\textgreater}feeke &`here' (close to speaker, visible) \\
na-an{\textgreater}nan &na-eke{\textgreater}neeke &`there' (away from the speaker; generic)\\
eef-an &eef-eke &`here' (rather close, usually visible)\\
een-an &een-eke &`there' (far away, usually not visible)\\
\mybottomrule
\end{tabular}
\end{table}


The difference in the usage between the neutral and focused member of each pair is that the first is \textstyleEmphasizedWords{\textsc{only}} used with realis-type verb forms, i.e. past \REF{ex:3:}, \REF{ex:3:} and present tense \REF{ex:3:}, whereas the second one is \textstyleEmphasizedWords{\textsc{mainly}} used with future \REF{ex:3:}, imperative \REF{ex:3:}, and counterfactual \REF{ex:3:}, i.e. irrealis-type forms. Yet Mauwake does not differentiate between realis and irrealis in verbs, and a possible explanation here is that only locative adverbs that are in focus can make it into a future, imperative or counterfactual clause, whereas past or present clauses are less restrictive and use either focal or non-focal form. 

\ea%x463
\label{ex:3:x463}
\gll Owowa=pa \textstyleEmphasizedVernacularWords{fan} ik-emkun aasa maneka ekap-o-k. \\
village=\textsc{loc} here be-1s/p.\textsc{ds} canoe big come-\textsc{pa}-3s\\
\glt`As I was here in the village the big ship came.'
\z

\ea%x464
\label{ex:3:x464}
\gll Eliw \textstyleEmphasizedVernacularWords{feeke} soop-i-yen. \\
well here.\textsc{cf} bury-Np-\textsc{fu}.1p\\
\glt`We can bury him \textstyleEmphasizedWords{\textsc{here}}.'
\z

\ea%x1213
\label{ex:3:x1213}
\gll Yo fura belemuta \textstyleEmphasizedVernacularWords{eefan} piipu-a-m. \\
1s.\textsc{unm} knife small there2 leave-\textsc{pa}-1s\\
\glt`I left the small knife (somewhere) here.'
\z

\ea%x465
\label{ex:3:x465}
\gll Ni koora epa \textstyleEmphasizedVernacularWords{eefeke} ku-eka. \\
2p.\textsc{unm} house place there2.\textsc{cf} build-\textsc{imp}.2p\\
\glt`Build a/the house \textstyleEmphasizedWords{\textsc{over here}} in this place.'
\z

\ea%x1214
\label{ex:3:x1214}
\gll Wi aakisa fain manina \textstyleEmphasizedVernacularWords{eenan} on-i-mik. \\
3p.\textsc{unm} now this garden there3 make-Np-\textsc{pr}.1/3p\\
\glt`Nowadays they make the garden(s) there (far away).'
\z

\ea%x1573
\label{ex:3:x1573}
\gll Ni \textstyleEmphasizedVernacularWords{eeneke} ikiw-ep momor naap niir-eka. \\
2p there3.\textsc{cf} go-\textsc{ss}.\textsc{seq} foolish thus play-\textsc{imp}.2p\\
\glt`Go \textstyleEmphasizedWords{\textsc{there}} (out of my sight) and play your foolish game.'
\z

\ea%x466
\label{ex:3:x466}
\gll \textstyleEmphasizedVernacularWords{Neeke} ik-ek-a-k=na iwer(a) ififa=ke ifakim-ek-a-k. \\
there1.\textsc{cf} be-\textsc{cntf}-\textsc{pa}-3s=\textsc{tp} coconut dry=\textsc{cf} kill-\textsc{cntf}-\textsc{pa}-3s\\
\glt`If he had been \textstyleEmphasizedWords{\textsc{there}} a (falling) dry coconut would have killed him.'
\z

\ea%x1197
\label{ex:3:x1197}
\gll Soo nainiw muf-owa pun naap, aana=pa \textstyleEmphasizedVernacularWords{neeke} muf-i-mik. \\
trap again pull-\textsc{nmz} too thus rattan=\textsc{loc} there1.\textsc{cf} pull-Np-\textsc{pr}.1/3p\\
\glt`Pulling the trap again is also like that, we/they pull it \textstyleEmphasizedWords{\textsc{there}} by the rattan.'
\z

\ea%x1198
\label{ex:3:x1198}
\gll Malol=pa \textstyleEmphasizedVernacularWords{neeke} nainiw suuw-urup-i-ya. \\
open.sea there1.\textsc{cf} again push-ascend-Np-\textsc{pr}.3s\\
\glt`\textstyleEmphasizedWords{\textsc{There}} from the open sea it (= tsunami wave) again pushes up (to the coast).'
\z

In the following examples \textstyleStyleVernacularWordsItalic{neeke} and\textstyleStyleVernacularWordsItalic{ feeke} are used with past or present tense verbs and indicate a temporary rather than permanent location, but this is probably secondary, or related, to the adverbs being focal: there is less need to focus on a permanent location than on a temporary one. Note that in these clauses it is possible to have two constituents with contrastive focus marking.

\ea%x1146
\label{ex:3:x1146}
\gll Miiw(a) aasa fa-ow(a) mua=ke \textstyleEmphasizedVernacularWords{neeke} wia aaw-o-k. \\
land canoe drive-\textsc{nmz} man=\textsc{cf} there1.\textsc{cf} 3p.\textsc{acc} take-\textsc{pa}-3s\\
\glt`\textstyleEmphasizedWords{\textsc{There}} the truck driver picked them up.'
\z

\ea%x1199
\label{ex:3:x1199}
\gll Or-op \textstyleEmphasizedVernacularWords{neeke} ika-iwkin kokom-ar-e-k. \\
descend-\textsc{ss}.\textsc{seq} there1.\textsc{cf} be-2/3p.\textsc{ds} dark-\textsc{inch}-\textsc{pa}-3s\\
\glt`When they had gone down and were \textstyleEmphasizedWords{\textsc{there}} it became dark.'
\z

\ea%x1200
\label{ex:3:x1200}
\gll Nainiw mukuna mamaiya \textstyleEmphasizedVernacularWords{neeke} ikiw-o-k. \\
again fire close there1.\textsc{cf} go-\textsc{pa}-3s\\
\glt`Again he went \textstyleEmphasizedWords{\textsc{there}} close to the fire.'
\z

The following example is a comment from a man after he sees Japanese bombers in the sky:

\ea%x1572
\label{ex:3:x1572}
\gll Fa, Yaapan=ke \textstyleEmphasizedVernacularWords{feeke} ik-e-mik! \\
\textsc{intj} Japan=\textsc{cf} here.\textsc{cf} be-\textsc{pa}-1/3p\\
\glt`Damn, the Japanese are \textstyleEmphasizedWords{\textsc{here}}!'
\z

\subsection{Deictic manner adverbs}\label{sec:3:y:x}
%\hypertarget{RefHeading19881935131865}
{}
The four deictic manner adverbs are based on the deictic roots, but their derivation is less regular than that of either the demonstratives or the deictic locatives, due to the restriction that a geminate vowel is only possible in an initial syllable. Again, the proximate and especially the distal-1 adverbs are common but the others are very infrequent.

\begin{table}
\caption{Please provide a caption}
\label{} 
\begin{tabular}{lll}
\mytoprule
feenap &`like this' &proximate\\
naap &`like that, thus' &distal-1\\
eefenap &`like that (further away)' &distal-2\\
eenap &`like that (far away)' &distal-3\\
\mybottomrule
\end{tabular}
\end{table}


\ea%x701
\label{ex:3:x701}
\gll Ikiw-e-mik=na \textstyleEmphasizedVernacularWords{feenap} ma-em-ik-e-mik {\dots} \\
go-\textsc{pa}-1/3p=\textsc{tp} like.this say-\textsc{ss}.\textsc{sim}-be-\textsc{pa}-1/3p\\
\glt`They went and (unexpectedly) kept saying like this {\dots}'
\z

\ea%x702
\label{ex:3:x702}
\gll \textstyleEmphasizedVernacularWords{Naap} maak-iwkin \textstyleEmphasizedVernacularWords{naap} ik-ua. \\
thus tell-2/3.\textsc{ds} thus be-\textsc{pa}.3s\\
\glt`They told him like that, and he was like that.'
\z

In \REF{ex:3:x1857} there is a long temporal distance between the hearing and the recounting of the story, which is apparently reflected in the choice of the adverbial.

\ea%x1857
\label{ex:3:x1857}
\gll Iiriw auwa-ke ma-iwkin \textstyleEmphasizedVernacularWords{eefenap} miim-a-m. \\
earlier 1s/p.father=\textsc{cf} say-2/3p.\textsc{ds} thus2 hear-\textsc{pa}-1s\\
\glt`The fathers spoke (about this) long ago and I heard it like that.'
\z

In \REF{ex:3:x1858} there is both some temporal and a considerable locative distance between the original time and place of the quote and that of the rest of the example: 

\ea%x1858
\label{ex:3:x1858}
\gll ``Mua nain opora=pa wu-ami ifakim-e,'' \textstyleEmphasizedVernacularWords{eenap} efa maak-e-mik.\\
man that1 talk=\textsc{loc} put-\textsc{ss}.\textsc{sim} kill-\textsc{imp}.2s thus3 1s.\textsc{acc} tell-\textsc{pa}-1/3p\\
\glt` ``Accuse (lit: put to talk) that man and kill him,'' they told me like that.'
\z

Location verbs (\sectref{sec:3.8.4.4.3}) are also based on the deictic roots, but directional verbs (\sectref{sec:3.8.4.4.5}), which also participate in the spatial deictic system in Mauwake, have different roots. 

\section{Question words and indefinites}\label{sec:3:7}
%\hypertarget{RefHeading19901935131865}
{}
Most of the indefinites in Mauwake are also question words, hence the treatment of both in the same subsection.

\subsection{Question words}\label{sec:3:y:x}
%\hypertarget{RefHeading19921935131865}
{}
The question words are here grouped together because of their shared semantic features and their function and position in content questions, although on the basis of their syntactic function on clause level some are pronouns, others adjectives or adverbs. 

The majority of the question words have an initial morpheme \textstyleStyleVernacularWordsItalic{ka}-, which indicates a question and is below in the derivations given the gloss `what', although it is unrelated to the question word \textstyleStyleVernacularWordsItalic{mauwa} `what'. The morphemes that make up the question words in the list below are given in parentheses when they can be reasonably clearly established. 

The question words are:

\begin{table}
\caption{Please provide a caption}
\label{} 
\begin{tabular}{lll}
\mytoprule
iikamin &`when?'\footnote{\textstyleFootnoteBaseChar{\textit{Ama kamin}} `sun how much' is used when time measured by clock is inquired; \textstyleFootnoteBaseChar{\textit{iikamin}} is less specific.} &({{\textless}}iir-kamin `time-how.much')\\
kaakew(e) &`of what place?'&\\
kaan &`where' &({{\textless}}ka-an `what=\textsc{loc}')\\
kaaneke &`where?' &({{\textless}}ka-an-eke `what=\textsc{loc}=\textsc{cf}\textstyleAcronymallcaps{'})\\
kaanin &`which (of two)?' &({{\textless}}ka-an-in `what=\textsc{loc}-\textstyleAcronymallcaps{GIVEN'})\\
kain &`which?' &({{\textless}}ka-in `what-\textstyleAcronymallcaps{GIVEN'})\\
kamin &`how many?', `how much?'&\\
kamenap &`how?', `what {\dots} like?' &({\textless}kamin-naap `how.much-thus')\\
mauwa &`what?'&\\
moram &`why?'&\\
naarew(e) &`who?'&\\
kamenion &`(or) what/how?' &({{\textless}}kamin-yon `how.much-perhaps')\\
naap-i &`like that?'&\\
\mybottomrule
\end{tabular}
\end{table}


Both the words translated with\textit{} `which', \textstyleStyleVernacularWordsItalic{kain} and \textstyleStyleVernacularWordsItalic{kaanin}, have the suffix -\textstyleStyleVernacularWordsItalic{in} marking givenness. They are both morphologically and semantically related to the demonstratives \textstyleStyleVernacularWordsxiiptItalic{fain} `this' and \textstyleStyleVernacularWordsxiiptItalic{nain} `that' (\sectref{sec:3.6.2}). 

\textstyleStyleVernacularWordsItalic{Kaan} `where' is formed by the question root \textstyleStyleVernacularWordsItalic{ka}- and the same locative affix -\textstyleStyleVernacularWordsItalic{an} that is used in the deictic locative adverbs \textstyleStyleVernacularWordsItalic{fan} `here' and \textstyleStyleVernacularWordsItalic{nan} `there' (\sectref{sec:3.6.3}). The derivation with the contrastive focus marker \textstyleStyleVernacularWordsItalic{-(e)}\textstyleStyleVernacularWordsItalic{ke} is more frequently used than the non-focused form, possibly because the the other two most frequent question words, \textstyleStyleVernacularWordsItalic{mauwa} `what' and \textstyleStyleVernacularWordsItalic{naarewe} `who', so often take the contrastive focus clitic. 

\ea%x1852
\label{ex:3:x1852}
\gll Mua nain unuf-ami ma-i-kuan, {\textquotedbl}Mua nain \textstyleEmphasizedVernacularWords{kaan} ik-ua?{\textquotedbl} \\
man that1 call-\textsc{ss}.\textsc{sim} say-Np-\textsc{fu}.3p man that1 where be-\textsc{pa}.3s\\
\glt`They call the man's name and say, ``Where is that man?'' '
\z

\ea%x1854
\label{ex:3:x1854}
\gll Oo Sarak, no \textstyleEmphasizedVernacularWords{kaan=eke} ik-ok kerer-e-n a? \\
\textsc{intj} Sarak 2s.\textsc{unm} where=\textsc{cf} be-SS arrive-\textsc{pa}-2s \textsc{intj}\\
\glt`Oh Sarak, where have you been (lit: where were you and arrived)?'
\z

\textstyleStyleVernacularWordsItalic{Kaanin} `which of two' also shares the locative morpheme \textstyleStyleVernacularWordsItalic{an}- with \textstyleStyleVernacularWordsItalic{kaan}- `where' as well as \textstyleStyleVernacularWordsItalic{fan} `here' and \textstyleStyleVernacularWordsItalic{nan} `there', although in its present meaning it is not a locative question.

There is also a morphological relationship between \textstyleStyleVernacularWordsItalic{kamenap} `how/ what{\dots}like?' and \textstyleStyleVernacularWordsItalic{kamin} `how many/much?' and the deictic adverb \textstyleStyleVernacularWordsItalic{naap} `thus', but synchronically their semantic relationship is opaque. \textstyleStyleVernacularWordsItalic{Kamenion} `or what? / how is it?' has obviously developed from \textstyleStyleVernacularWordsxiiptItalic{kamin} `how many/much?' and the modal clitic \textstyleStyleVernacularWordsItalic{\nobreakdash-yon} `perhaps' (\sectref{sec:3.9.3}), but again, the relationship is not transparent any more.

The question words, except for \textstyleStyleVernacularWordsItalic{kamenion} and \textstyleStyleVernacularWordsItalic{naap-i}, occupy the same syntactic position and clausal function as the corresponding non-interrogative element would have:

\ea%x520
\label{ex:3:x520}
\gll Mua nain \textstyleEmphasizedVernacularWords{iikamin} ekap-o-k? \\
man that when come-\textsc{pa}-3s\\
\glt`When did that/the man come?'
\z

\ea%x647
\label{ex:3:x647}
\gll Mua nain \textstyleEmphasizedVernacularWords{unan} ekap-o-k. \\
man that yesterday come-\textsc{pa}-3s\\
\glt`That/the man came yesterday.'
\z

\ea%x521
\label{ex:3:x521}
\gll Maa \textstyleEmphasizedVernacularWords{mauwa} en-e-n? \\
thing/food what eat-\textsc{pa}-2s\\
\glt`What did you eat?'
\z

\ea%x648
\label{ex:3:x648}
\gll Maa \textstyleEmphasizedVernacularWords{oposia} en-e-m. \\
thing/food meat eat-\textsc{pa}-1s\\
\glt`I ate meat.'
\z

Neither number nor case is marked on the interrogative words themselves. If either marking is required, it is done through personal pronouns, but for [+human] \textstyleAcronymallcaps{NP}s only.

\ea%x522
\label{ex:3:x522}
\gll Mua \textstyleEmphasizedVernacularWords{naarew} \textstyleEmphasizedVernacularWords{wia} uruf-a-n? \\
man who 3p.\textsc{acc} see-\textsc{pa}-2s\\
\glt`Whom (pl) did you see?'
\z

\ea%x523
\label{ex:3:x523}
\gll \textstyleEmphasizedVernacularWords{Naarew} \textstyleEmphasizedVernacularWords{wiar} aaw-o-k? \\
who 3.\textsc{dat} get-\textsc{pa}-3s\\
\glt`Who did he get it from?'
\z

When an interrogative word is used as a subject, the contrastive focus marker \nobreakdash-\textstyleStyleVernacularWordsItalic{ke} is added. This is natural since it is the question word that is the focal element in questions. 

\ea%x524
\label{ex:3:x524}
\gll \textstyleEmphasizedVernacularWords{Mauwa}\textstyleEmphasizedVernacularWords{=ke} nefa aruf-a-k? \\
what=\textsc{cf} 2s.\textsc{acc} hit-\textsc{pa}-3s\\
\glt`What hit you?'
\z

\ea%x525
\label{ex:3:x525}
\gll Mua \textstyleEmphasizedVernacularWords{kain=ke} nomak-e-k? \\
man which=\textsc{cf} win-\textsc{pa}-3s\\
\glt`Which man won?'
\z

\ea%x526
\label{ex:3:x526}
\gll Masin \textstyleEmphasizedVernacularWords{kaanin=ke} samor-ar-e-k? \\
engine which.of.2=\textsc{cf} bad-\textsc{inch}-\textsc{pa}-3s\\
\glt`Which engine (of the two) broke?'
\z

\textstyleStyleVernacularWordsItalic{Naarew}\textstyleStyleVernacularWordsItalic{(e)} `who?' is only used for [+human] referents. When the contrastive focus maker -\textstyleStyleVernacularWordsItalic{ke} is suffixed to the question word, the last syllable is normally deleted. \textstyleStyleVernacularWordsItalic{Mauwa} `what', on the other hand, is used almost solely for [-human] nouns. The only natural expression with \textstyleStyleVernacularWordsItalic{mauwa} referring to humans that I have encountered is of the type \REF{ex:3:x649}. When a person's name is inquired, either \textstyleStyleVernacularWordsItalic{naarewe} \REF{ex:3:} or \textstyleStyleVernacularWordsItalic{kamenap} \REF{ex:3:} is used rather than \textstyleStyleVernacularWordsItalic{mauwa}.

\ea%x649
\label{ex:3:x649}
\gll Emeria nain no/nena \textstyleEmphasizedVernacularWords{mauwa=ke}? \\
woman that 1s.\textsc{unm}/1s.\textsc{gen} what=\textsc{cf}\\
\glt`What (relation) of yours is that woman?'
\z

\ea%x1855
\label{ex:3:x1855}
\gll O unuma \textstyleEmphasizedVernacularWords{naare=ke}? \\
3s.\textsc{unm} name who=\textsc{cf}\\
\glt`What is his/her name?'
\z

\textstyleStyleVernacularWordsItalic{Kaanin} `which of two?' is specified for number \REF{ex:3:x691}, but \textstyleStyleVernacularWordsItalic{kain} `which?' is not.

\ea%x691
\label{ex:3:x691}
\gll No \textstyleEmphasizedVernacularWords{kain} kookal-i-n? \\
2s.\textsc{unm} which like-Np-\textsc{pr}.2s\\
\glt`Which one (of two or many) do you like?'
\z

The locative question word \textstyleStyleVernacularWordsItalic{kaan}(\textstyleStyleVernacularWordsItalic{eke}) `where' is often used as a phrase by itself \REF{ex:3:}, \REF{ex:3:}. but it is also employed as a modifier of a locative noun phrase rather than \textstyleStyleVernacularWordsItalic{kain} or \textstyleStyleVernacularWordsItalic{kaanin}: 

\ea%x1853
\label{ex:3:x1853}
\gll [Epa ara \textstyleEmphasizedVernacularWords{kaan=eke}]\textsubscript{NP} ikiw-e-mik? \\
place section where=\textsc{cf} go-\textsc{pa}-1/3p\\
\glt`What/which area did they go to?'
\z

\textstyleStyleVernacularWordsItalic{Kamenap} is a question word both for manner `how?' \REF{ex:3:x527} and for adjectives `what {\dots} like?'. In the latter sense it usually modifies the noun \textstyleStyleVernacularWordsItalic{sira} `custom, kind' \REF{ex:3:x528}.

\ea%x527
\label{ex:3:x527}
\gll No \textstyleEmphasizedVernacularWords{kamenap} ik-o-n? \\
2s.\textsc{unm} how be-\textsc{pa}-2s\\
\glt`How are/were you?'
\z

\ea%x528
\label{ex:3:x528}
\gll O koora \textstyleEmphasizedVernacularWords{sira} \textstyleEmphasizedVernacularWords{kamenap} ku-a-k? \\
3s.\textsc{unm} house custom/kind what.like build-\textsc{pa}-3s\\
\glt`What kind of house did he build?'
\z

It is also used with the noun \textstyleStyleVernacularWordsItalic{unuma} `name' when the name of someone or something is inquired:

\ea%x650
\label{ex:3:x650}
\gll O unuma \textstyleEmphasizedVernacularWords{kamenap}? \\
3s.\textsc{unm} name what.like?\\
\glt`What is his/her name?'
\z

\ea%x651
\label{ex:3:x651}
\gll Nomokowa fain unuma \textstyleEmphasizedVernacularWords{kamenap}? \\
tree this name what.like\\
\glt`What is the name of this tree?'
\z

In example \REF{ex:3:} \textstyleStyleVernacularWordsItalic{kamenap} is interchangeable with \textstyleStyleVernacularWordsItalic{naare}(\textstyleStyleVernacularWordsItalic{we})-\textstyleStyleVernacularWordsItalic{ke} `who', but in \REF{ex:3:} it is not interchangeable with \textstyleStyleVernacularWordsItalic{mauwa}\textstyleStyleVernacularWordsItalic{-ke} `what'.

The interrogative \textstyleStyleVernacularWordsItalic{kamenion} forms a clause by itself and only occurs after the question clitic -\textstyleStyleVernacularWordsItalic{i} and/or the connective \textstyleStyleVernacularWordsItalic{e} `or'. 

\ea%x529
\label{ex:3:x529}
\gll Maa en-owa=ko p-ekap-e-mik=i \textstyleEmphasizedVernacularWords{kamenion}? \\
thing eat-\textsc{nmz}=\textsc{nf} Bpx-come-\textsc{pa}-1/3p=\textsc{qm} or.what\\
\glt`Did they bring food, or what (happened)?'
\z

The question word \textstyleStyleVernacularWordsItalic{naap-i} `like that?' is different from the other question words. It is formed by adding the question marker -\textstyleStyleVernacularWordsItalic{i} to the demonstrative \textstyleStyleVernacularWordsItalic{naap} `thus, like that', and it occurs by itself or sentence-finally after a statement, which often follows another question. It is mainly used in argumentation. 

\ea%x1194
\label{ex:3:x1194}
\gll Siiwa arow ikiw-eya maa en-owa perek-i-mik. \textstyleEmphasizedVernacularWords{Naap}\textstyleEmphasizedVernacularWords{=i}? \\
moon three go-2/3s.\textsc{ds} thing eat-\textsc{nmz} harvest-Np-1/3p thus=\textsc{qm}\\
\glt`After three months we'll harvest the food, right?'
\z

\ea%x1195
\label{ex:3:x1195}
\gll Feenap eliw ma-i-yen=i? Sira nain eliw marew, \textstyleEmphasizedVernacularWords{naap}\textstyleEmphasizedVernacularWords{=i}?\\
like.this well say-Np-\textsc{fu}.1p=\textsc{qm} custom that1 good none thus=\textsc{qm}\\
\glt`Should we say that that custom is not good -- is that what you are saying?'
\z

Questions are discussed in \sectref{sec:7.2}, which has more examples as well.

\subsection{Indefinites}\label{sec:3:y:x}
%\hypertarget{RefHeading19941935131865}
{}
Indefinites are sometimes classified as pronouns, although they often are not very pronoun-like; sometimes they are grouped together with quantifiers \citep[81]{HakulinenEtAl1979}%Karlsson
. By definition they lack definiteness which is typical of other pronouns \citep[376]{QuirkEtAl1985}. Also their status as \textstyleAcronymallcaps{NP} substitutes is questionable.

In Mauwake, the indefinites behave syntactically very much like quantifiers. The position of the indefinites in the \textstyleAcronymallcaps{NP} is after the adjective phrase and immediately preceding the demonstrative. They rarely co-occur with a quantifier phrase, but if they do, they follow the \textstyleAcronymallcaps{QP}.

The number of indefinites in Mauwake is very small. The last four in the list are actually question words (\sectref{sec:3.7.1}) that also function as indefinites:

\begin{table}
\caption{Please provide a caption}
\label{} 
\begin{tabular}{ll}
\mytoprule
oko &`a certain, (an)other'\\
papako &`some, other'\\
naarew(e) &`whoever, someone, one'\\
mauwa &`whatever, something'\\
kain &`whichever'\\
kaanin &`whichever (of two)'\\
\mybottomrule
\end{tabular}
\end{table}


\ea%x641
\label{ex:3:x641}
\gll Iiriw muuka \textstyleEmphasizedVernacularWords{oko} wiawi onak urera maa uup-e-mik.\\
long.ago boy other 3s/p.father 3s/p.mother evening food cook-\textsc{pa}-1/3p\\
\glt`Long ago, a certain boy's father and mother cooked food.'
\z

\ea%x642
\label{ex:3:x642}
\gll Ne wia, \textstyleEmphasizedVernacularWords{papako=ke} ma-e-mik, {\dots} \\
\textsc{add} no, some/other=\textsc{cf} say-\textsc{pa}-1/3p\\
\glt`But no, some/others said, {\dots}'
\z

The indefinite \textstyleStyleVernacularWordsItalic{oko} `a certain, (an)other' also has the meaning `otherwise' when it introduces an apprehensive clause (\sectref{sec:8.1.6}).

\ea%x741
\label{ex:3:x741}
\gll Gurun-owa epasia=pa miim-am-ika-i-kuan, \textstyleEmphasizedVernacularWords{oko} mua papako maa ik-em-ik-owa nain kawus wiar uruf-i-kuan.\\
rumble-\textsc{nmz} far=\textsc{loc} hear-\textsc{ss}.\textsc{sim}-be-Np-\textsc{fu}.3p other man some thing/food roast-\textsc{ss}.\textsc{sim}-be-\textsc{nmz} that1 smoke 3.\textsc{dat} see-Np-\textsc{fu}.3p\\
\glt`They (villagers) keep listening to the rumble from far away, otherwise/lest they (pilots) see the smoke from some men's/people's food-roasting fire.'
\z

Those question words (\sectref{sec:3.7.1}) that may function as indefinites behave similarly to question words as \textstyleAcronymallcaps{NP} constituents, but on the sentence level there are differences between them. The interrogatives occur either in a simple interrogative sentence or occasionally in a medial clause \REF{ex:3:x643}. The indefinites can occur in a medial clause \REF{ex:3:x644}, but they are more common in subordinate clauses, especially relative clauses \REF{ex:3:x645}. 

\ea%x643
\label{ex:3:x643}
\gll \textstyleEmphasizedVernacularWords{Naarew} wia far-ep ekap-o-n? \\
who 3p.\textsc{acc} call-\textsc{ss}.\textsc{seq} come-\textsc{pa}-2s\\
\glt`Who did you call, and then came?'\footnote{A more natural translation into English would be `Who did you call before you came?', but it would hide the fact that medial clauses are coordinate.}
\z

\ea%x644
\label{ex:3:x644}
\gll Masin \textstyleEmphasizedVernacularWords{kaanin=ke} samor-ar-eya oko fain=ke asip-i-non.\\
engine which.of.2=\textsc{cf} bad-\textsc{inch}-2/3s.\textsc{ds} other this=\textsc{cf} help-Np-\textsc{fu}.3s\\
\glt`Whichever engine breaks down, this other one will help/substitute.'\footnote{With question intonation it would mean: `Which engine\textsubscript{i} will this other one\textsubscript{j} help, if it\textsubscript{i} breaks down?'}
\z

\ea%x645
\label{ex:3:x645}
\gll Prais aaw-ep [\textstyleEmphasizedVernacularWords{uf-owa} \textstyleEmphasizedVernacularWords{kain=ke} nomak-e-k nain]\textsubscript{RC} wi-e-mik.\\
prize take-\textsc{ss}.\textsc{seq} dance-\textsc{nmz} which=\textsc{cf} win-\textsc{pa}-3s that1 give.them-\textsc{pa}-1/3p\\
\glt`They took the prize and, whichever dance won, they gave it (the prize) to them (the dancers).'
\z

The indefinite \textstyleStyleVernacularWordsItalic{mauwa} `what' is also used as a generic substitute for any [\nobreakdash-human] \textstyleAcronymallcaps{NP} that is left unmentioned because the name of the particular thing is not known or is temporarily forgotten, like \textstyleForeignWords{whatchamacallit} in English.

\ea%x646
\label{ex:3:x646}
\gll Mua nain \textstyleEmphasizedVernacularWords{mauwa} nain akim-a-k=na weetak, \textstyleEmphasizedVernacularWords{mauwa}  nain me or-o-k.\\
man that1 what that1 try-\textsc{pa}-3s=\textsc{tp} no, what that1 not descend-\textsc{pa}-3s\\
\glt`The man tried the thing (press button), but the thing (lift) didn't go down.'
\z

The locative question word \textstyleStyleVernacularWordsxiiptItalic{kaaneke} is also used as an indefinite locative adverb:

\ea%x1869
\label{ex:3:x1869}
\gll No \textstyleEmphasizedVernacularWords{kaaneke} ikiw-i-nan=na, yos pun nook-i-nen. \\
2s.\textsc{unm} where.\textsc{cf} go-Np-\textsc{fu}.2s=\textsc{tp} 1s.\textsc{fc} too follow.you-Np-\textsc{fu}.1s\\
\glt`Wherever you go, I will follow you.'
\z

\section{Verbs}\label{sec:3:8}
%\hypertarget{RefHeading19961935131865}
{}
\subsection{General discussion}\label{sec:3:y:x}
%\hypertarget{RefHeading19981935131865}
{}
\subsubsection{Definition}\label{sec:3:z:y:x}
%\hypertarget{RefHeading20001935131865}
{}
The verb category can be defined morphologically, syntactically, semantically and pragmatically. Of these, the first criterion is the most critical in Mauwake and covers the whole class; the others are less definitive, but help define a \textstyleEmphasizedWords{\textsc{prototypical}} verb.

According to the \textstyleEmphasizedWords{\textsc{morphological}}, or structural, criterion, a verb is a word that can be inflected for tense as well as the person and number of the subject. The derivational suffix categories of verbaliser, distributive and benefactive are not as useful in defining the class of verbs, as these can be used in the nominalized forms of verbs as well. \citet[190]{Anderson1985b} also adds aspect and mood into inherent verbal inflections, but in Mauwake aspect is coded syntactically (see verbal groups in \sectref{sec:3.8.5.1}), and modal categories either morphologically, syntactically or lexically.

\textstyleEmphasizedWords{\textsc{Syntactically}} a verb functions as the nucleus of a predication independently or as part of a verbal cluster (\sectref{sec:3.8.5}). Since single verbs and verbal clusters have such similar functions, the latter are described in the morphology chapter immediately after the verbs, and not in the chapter on phrase. Also, the term \textstyleEmphasizedWords{\textsc{verb}} is often used below as a generic term to cover both a single verb and a verbal cluster, unless specifically the verbal cluster is meant. The verb is the last element in a pragmatically neutral clause.

The verbal predicate is the only obligatory element in an intransitive clause. A transitive clause does require an object, but even it can often consist of a verb only, as the third person singular accusative pronoun, used for object, is zero \REF{ex:3:x177}. The directional verbs (\sectref{sec:3.8.4.4.5}) often co-occur with a goal, but when it is left implied the verb can be the only element \REF{ex:3:x178}. In a verbless clause the predicate is a noun, adjective, possessive pronoun or adverb. 

\ea%x177
\label{ex:3:x177}
\gll \textstyleEmphasizedVernacularWords{Aaw-e-m.} \\
get-\textsc{pa}-1s\\
\glt`I got it.'
\z

\ea%x178
\label{ex:3:x178}
\gll \textstyleEmphasizedVernacularWords{Urup-e-mik}. \\
go/come.up-\textsc{pa}-1/3p \\
\glt`We went/came up.' 
\z

The predicate verb selects the arguments in a predication. This argument selection can be used as an important basis for the division into different verb classes (\sectref{sec:3.8.4}).

\textstyleEmphasizedWords{\textsc{Semantically}}, according to \citet[64]{Givon}, a prototypical verb encodes ``{less time-stable experiences, primarily transitory states, events and actions}''. In Mauwake this lack of time-stability feature shows in the strong tendency to use inchoative verbs \REF{ex:3:x179} (\sectref{sec:3.8.2.2.2}) instead of adjectives to describe non-permanent states. 

\ea%x179
\label{ex:3:x179}
\gll \textstyleEmphasizedVernacularWords{supuk-ar-e}\textstyleEmphasizedVernacularWords{-k} vs. \textstyleEmphasizedVernacularWords{supuka} `wet' \\
wet-\textsc{inch}-\textsc{pa}-3s \\
\glt`(it) is wet' (lit: `has become wet') 
\z

But it is also possible to express less prototypical, time-stable states and events with verbs. In \citegen[66]{Frawley1992} words, ``{verbs {\dots} require temporal fixing''}, when compared with the ``{relative atemporality}'' of an entity. So the \textstyleEmphasizedWords{\textsc{relative temporality}} is the main defining factor for verbs, regardless of the time-stability.

\citet[726]{HopperEtAl1984} add a \textstyleEmphasizedWords{\textsc{discourse}} perspective to the definition of verbs by suggesting that ``{verbs which do not report discourse events fail to show the range of oppositions characteristic of those which do}'', and are therefore less prototypical. According to them, categoriality is only weakly associated with the root forms, and the discourse use determines how clearly the verbhood manifests itself (ibid. 747). Theirs is an important viewpoint for the study of language in general and of those languages in particular that have plenty of root forms that can be used for different word classes. But for Mauwake I assume the existence of rather discrete categories of noun and verb, which the root forms belong to, rather than just having {``a propensity or predisposition to become} \textstyleAcronymallcaps{\textup{N}}{'s or} \textstyleAcronymallcaps{\textup{V}}{'s''} (ibid. 747). The number of roots that can be used across categories without special derivational suffixes is small.

\subsubsection{General characteristics of verbs in Mauwake}\label{sec:3:z:y:x}
%\hypertarget{RefHeading20021935131865}
{}
Mauwake is a strongly verb-oriented language, and often a verb is the only element in the clause. In running text, there are roughly three words per clause, so approximately one word in three is a verb, as most of the clauses are verbal clauses.

The verb morphology is agglutinative; this shows mainly in the structure of the verbs. Suffixing is the basic strategy, but a few prefixes are used as well. Reduplication is of the prefixing type, with few exceptions. 

Although the verb morphology in Mauwake is quite extensive, for a Papuan language it is not very complex, and the patterns are quite transparent. The verb morphology marks features of the event itself: tense, mood, sequentiality vs. simultaneity of actions, but also features related to the participants in the clause: subject, beneficiary, and distributive indicating the number of \textstyleAcronymallcaps{S}, \textstyleAcronymallcaps{O} or \textstyleAcronymallcaps{REC}. Aspect is expressed through verbal groups (\sectref{sec:3.8.5.1.1}).

To enlarge its verb inventory, Mauwake uses serial verbs (\sectref{sec:3.8.5.1.2}) or adjunct\footnote{Adjunct is here used in the sense of ``a secondary element in a construction [,~which] may be removed without the structural identity of the rest of the construction being affected'' \citep[9]{Crystal1997}.} plus verb constructions (\sectref{sec:3.8.5.2}). The serial verbs are mostly formed by a productive process, whereas the adjunct plus verb constructions tend to be lexicalized forms. 

Some verbs have roots that are very similar to nouns. Especially in Austronesian languages the question arises whether these roots are originally nouns, verbs, or unspecified as to the grammatical category \citep[162--165]{Bugenhagen1995}. This question for Mauwake is discussed in the section on verb derivation (\sectref{sec:3.8.2}).

Mauwake has no passive voice. The subject demotion strategy is described in \sectref{sec:3.8.4.3.3}. 

There is a distinction in Mauwake between medial (\sectref{sec:3.8.3.5}) and final verbs (\sectref{sec:3.8.3.4}).\footnote{Sometimes they are also called dependent and independent verbs (e.g. {Foley 1986}:11).} This distinction is very important on both sentence and discourse levels. 

The verbs can be divided into two conjugation classes based on the past tense suffix vowel. Semantically these classes are arbitrary; the division is made on the basis of morphophonology and is discussed in \sectref{sec:2.3.3.3}. But the classification done according to transitivity (\sectref{sec:3.8.4.2}) and that based on semantic characteristics (\sectref{sec:3.8.4.4}) are more interesting grammatically and reveal more of the nature of the language. 

\subsubsection{Verb structure}\label{sec:3:z:y:x}
%\hypertarget{RefHeading20041935131865}
{}
A verb consists of a root optionally preceded by a derivational prefix and followed by various derivational and inflectional suffixes, as shown in the diagram below (\figref{fig:1}). Only tense and person/number suffixes are obligatory in a finite verb in the \textstyleEmphasizedWords{\textsc{indicative}} mood. The obligatory elements are bolded in the diagrams.

 
\begin{figure}
\setlength{\tabcolsep}{0pt}
\begin{tabular}{cccc}
 Derivation &  & Derivation & Inflection\\
$\overbrace{\hspace{1cm}}$ & & $\overbrace{\hspace{3.4cm}}$ &$\overbrace{\hspace{4cm}}$ \\
 prefix- & root &  -\textsc{inch-caus-distr-ben} & -\textsc{bnfy-cntf-\textbf{tns-prs/num}} \\
\multicolumn{3}{c}{$\underbrace{\hspace{6.2cm}}$}\\
\multicolumn{3}{c}{Stem}\\ 
\end{tabular}
 
\caption{Verb derivation and finite inflection (indicative)}
\label{fig:1}
\setlength{\tabcolsep}{6pt}
\end{figure}

\ea%x180
\label{ex:3:x180}
\gll Soomia wia \textstyleEmphasizedVernacularWords{amap-ep-om-i-ya.} \\
spoon 3p.\textsc{acc} Bpx-go-\textsc{ben}-Np-\textsc{pr}.3s\\
\glt`He takes spoons to them.'
\z

\ea%x181
\label{ex:3:x181}
\gll Iwera pun wiar \textstyleEmphasizedVernacularWords{aw-omak-e-k}. \\
coconut too 3.\textsc{dat} burn-\textsc{distr}/\textsc{pl}-\textsc{pa}-3s\\
\glt`Many of his coconut palms burned too.'
\z

\ea%x182
\label{ex:3:x182}
\gll Lawiliw akena \textstyleEmphasizedVernacularWords{um-ek-a-m.} \\
nearly very die-\textsc{cntf}-\textsc{pa}-1s \\
\glt`I very nearly died.'
\z

The \textstyleEmphasizedWords{\textsc{imperative}} verb structure is understandably different in that it cannot take counterfactual, tense or indicative person/number suffixes. Instead, an imperative person/number suffix needs to be attached as the final suffix of the verb.

Prefix -- \textbf{ROOT} -- \textstyleAcronymallcaps{INCH} -- \textstyleAcronymallcaps{CAUS} -- \textstyleAcronymallcaps{DISTR} -- \textstyleAcronymallcaps{BEN} -- \textstyleAcronymallcaps{BNFY} -- \textbf{IMP.PRS/NUM}

\ea%x183
\label{ex:3:x183}
\gll Ni \textstyleEmphasizedVernacularWords{ekap-omak-eka.} \\
2p.\textsc{unm} come-\textsc{distr}/\textsc{pl}-\textsc{imp}.2p\\
\glt`Come!' (said to several people together) 
\z

\ea%x184
\label{ex:3:x184}
\gll Muuka \textstyleEmphasizedVernacularWords{arim-ow-e.} \\
son grow-CAUS-\textsc{imp}.2s \\
\glt`Bring up the boy.' 
\z

\textstyleEmphasizedWords{\textsc{Medial}} verbs likewise can have only the medial suffix after the derivational suffixes, if there are any. The medial suffix distinguishes between sequentiality \REF{ex:3:} and simultaneity \REF{ex:3:} of the actions when the subject stays the same; with a different subject \REF{ex:3:} the actions are understood to be sequential, and simultaneity needs to be marked through continuous aspect form (\sectref{sec:3.8.5.1.1.2}).
 
SEQ 
 
 SS 
\todo{text disappeared} 
 ncing verbs can easily have a locative or instrument phrase, and the verb itself can take a benefactive suffix. A sentence like \REF{ex:3:} would be possible for instance when sending money to people travelling in the same vehicle as the addressee. 

\ea%x338
\label{ex:3:x338}
\gll Miiw-aasa=pa \textstyleEmphasizedVernacularWords{wi-om}-\textstyleEmphasizedVernacularWords{e}. \\ \\
\glt`They cut the meat in many pieces and cooked it.'
\z

\ea%x186
\label{ex:3:x186}
\gll Ewar=ke \textstyleEmphasizedVernacularWords{wuun-ow-ami} epia faker-a-k, mukuna. \\
wind=\textsc{cf} blow-CAUS-\textsc{ss}.\textsc{sim} firewood raise-\textsc{pa}-3s fire \\
\glt`The wind blew and raised the fire(wood), the fire.' 
\z

\ea%x187
\label{ex:3:x187}
\gll \textstyleEmphasizedVernacularWords{Kees-om-a-ya} en-ek. \\
spit-\textsc{ben}-\textsc{bnfy}2-2/3s.\textsc{ds} eat-\textsc{pa}-3s \\
\glt`He spat/regurgitated it for her and she ate.'
\z

\subsection{Verb derivatives}\label{sec:3:y:x}
%\hypertarget{RefHeading20061935131865}
{}
This section deals with derivational processes in which the end result is always a verb. Verbs can be derived from other word classes through two category-\textstyleEmphasizedWords{\textsc{changing}} strategies. In category-\textstyleEmphasizedWords{\textsc{maintaining} }derivations affixes are added to the verb root to change the semantics of the root. Among the latter, the semantic changes can be considerable especially in cases where the valence changes, whereas in category-changing derivations the semantic difference is not always so great \citep[83]{Bybee1985}. 

\subsubsection{Derivation vs. inflection}\label{sec:3:z:y:x}
%\hypertarget{RefHeading20081935131865}
{}
According to \citet[81]{Bybee1985}, ``{an inflectional morpheme {\dots} is a bound non-root morpheme whose appearance in a particular position is compulsory}.'' It is ``{required by syntax}''. In contrast, derivational affixes are non-obligatory \citep[191]{Greenberg1954}. In Mauwake, all the derivations are non-obligatory. Of the inflections, the beneficiary suffix and the counterfactual suffix as such are not required by syntax like the tense and person marking, but they have an interdependence relationship with other suffixes: the beneficiary suffix has to occur in a past tense or imperative form of a verb that also has the benefactive suffix, and the counterfactual suffix restricts the tense marking to past tense.

In Mauwake verb structure the derivational suffixes always precede the inflectional ones. This agrees with one of {Greenberg}'s universals: ``{If the derivation and inflection follow the root {\dots} the derivation is always between the root and the inflection}'' \
(\citeyear[93]{Greenberg1966}).

Inflectional suffixes in Mauwake form paradigms, even if in some cases the paradigms only have two members.

The greater syntagmatic freedom of derivational affixes \citep[128--129]{Malkiel1978} is shown in Mauwake by the fact that a verb with any of the derivations can be nominalized with the nominalizing suffix \nobreakdash-\textstyleStyleVernacularWordsItalic{owa}, whereas one with inflectional suffixes cannot.\footnote{It is possible to nominalize whole \textit{clauses} where the main verb has inflectional suffixes, by adding the demonstrative \textstyleFootnoteBaseChar{\textit{nain}} `that' after the clause, but this strategy is not available for individual verbs (\sectref{sec:5.7.2}).} This ability of verb stems with derivational suffixes to be nominalized is the main distinction between derivation and inflection in Mauwake. 

A special feature in Mauwake is the dividing point between the derivational and inflectional suffixes: the benefactive suffix is derivational, whereas the beneficiary suffix is inflectional. The latter can only be present when there are other verbal suffixes following, whereas the former can also be followed by a nominaliser suffix. Other differences between the two suffixes are described below (\sectref{sec:3.8.2.3.3}), (\sectref{sec:3.8.3.1}). In the following section, the derivational suffixes are introduced in the order that they occur following the verb root; the prefixes are discussed last.

There is clear iconicity in the linear ordering of the derivational suffixes: the closer the suffix is to the root, the more profound the change it effects on it. The verbalizing suffixes change the word class; the causative adds an argument; the distributive pluralizes an argument, and the benefactive adds a peripheral.

\begin{table}
\begin{tabular}{llllll}
\mytoprule
Prefix & Root & Verbaliser & Causative & Distributive & Benefactive\\
\midrule
p- & & -{\O} & -ow & -omak & -om\\
amap- & & -ar & & -urum & \\
aap- & & & & & \\
\textsc{rdp} & & & & & \\
\mybottomrule 
\end{tabular}
\todo[inline]{this table seem incomplete}
\caption{Verbal derivation}
\label{tab:10}
\end{table}


\subsubsection{Category-changing derivation: verb formation}\label{sec:3:z:y:x}
%\hypertarget{RefHeading20101935131865}
{}
There are two strategies in Mauwake whereby words from other word classes can be changed into verbs. Zero verb formation is less productive than the inchoative. Also the meanings of the verbs resulting from zero verb formation are in some cases more lexicalized, or less transparent, than the meanings of the verbs formed with the inchoative suffix. Often roots can be used for both the strategies, but not always: words like \textstyleStyleVernacularWordsItalic{amisa} `\textstyleFreeTranslationChar{knowledge'} and \textstyleStyleVernacularWordsItalic{ewur} `\textstyleFreeTranslationChar{quickly, fast'} only allow the inchoative suffix.

\paragraph{Zero verb formation}\label{sec:3:a:z:y:x}
%\hypertarget{RefHeading20121935131865}
{}
Mauwake has a number of verbs where the root is originally a noun, an adjective or an adverb, and the verb is formed without any overt morpheme to mark the category change. Hopper and \citet[745]{Thompson1984} remark that ``{languages often possess rather elaborate morphology whose sole function is to convert verbal roots into} \textstyleAcronymallcaps{N}{'s, but no morphology whose sole function is to convert nominal roots into} \textstyleAcronymallcaps{V}{'s}''. Zero verb formation is here understood, not as adding a zero morpheme, but as a lexical process (following \citealt[224]{Payne1997}). A noun, adjective or adverb is used as a root for the verb, and in this process it becomes a true verb, unlike nominalizations which are nouns but retain a lot of their verbal nature as well \citep[747]{HopperEtAl1984}%Thompson
.\footnote{\citet{HopperEtAl1984} propose that a word root is unspecified as to the grammatical category, and that discourse function assigns categoriality. This fits many Austronesian languages in which there are plenty of words where only the non-root morphology, or else syntactic behaviour, shows what class the word belongs to. Mauwake has relatively few forms like this and it is reasonable to assign words to specific word classes even without reference to discourse function.}

The resulting verb is usually transitive, with a few exceptions. The final vowel of a noun or an adjective, usually /a/, is deleted before the verbal inflection. 

\begin{table}
\caption{Verbs from nouns.}
\label{} 
\begin{tabular}{>{\itshape}ll>{\itshape}ll}
\mytoprule
akuwa &`knot' &akuw- &`knot/bind/tie with a knot'\\
anima &`blade' &anim- &`sharpen'\\
eneka &`tooth, flame' &enek- &`light (a fire)'\\
ilen &`sign' &ilen- &`recognise sign'\\
nanar &`story' &nanar- &`tell a story'\\
\mybottomrule
\end{tabular}

\end{table}

 

\begin{table}
\caption{Verbs from adjectives.}
\label{} 
\begin{tabular}{>{\itshape}ll>{\itshape}ll}
\mytoprule
dubila &`smooth' &dubil- &`smoothen'\\
enuma &`new' &enum- &`renew'\\
iiwa &`short' &iiw- &`shrink'\\
itita &`soft' &itit- &`smash'\\
kaken &`straight' &kaken- &`straighten'\\
maneka &`big' &manek- &`enlarge'\\
momora &`fool' &momor- &`confuse'\\
samora &`bad' &samor- &`destroy'\\
siina &`tight' &siin- &\textstyleTableEntryChar{`diminish' (intr.)}\footnote{Another intransitive verb can be derived from \textstyleFootnoteBaseChar{\textit{siina}} with the inchoative suffix: \textstyleFootnoteBaseChar{\textit{siin-ar}}- `become tight/narrow'.}\\
\mybottomrule
\end{tabular}

\end{table}



\begin{table}
\caption{Verbs from adverbs.} 
\label{} 
\begin{tabular}{>{\itshape}ll>{\itshape}ll}
\mytoprule
bilik &`mixed' &bilik- &`mix'\\
ikum &`illicitly' &ikum- &`speculate'\\
kerew &`strongly' &kerew- &`be angry at'\\
fan &`here' &fan- &`be/come here'\\
nan &`there' &nan- &`be/come here'\\
\mybottomrule
\end{tabular}

\end{table}

%\stepcounter{nx}{\thenx}x188) 
\ea%x188
\label{ex:3:x188}
\gll Yo aakisa inasina Rubaruba \textstyleEmphasizedVernacularWords{nanar-i-yem}.\\
1s.\textsc{unm} now spirit Rubaruba story-Np-\textsc{pr}.1s \\
\glt`Now I tell about spirit Rubaruba.'
\z

%\stepcounter{nx}{\thenx}x189) 
\ea%x189
\label{ex:3:x189}
\gll Aruf-ami me \textstyleEmphasizedVernacularWords{samor-eka}!\\
hit-\textsc{ss}.\textsc{sim} not bad-\textsc{imp}.2p \\
\glt`Don't hit/beat and destroy it.' 
\z

Semantically the resulting verb is usually very close to the word that serves as the root, but in a few instances like \REF{ex:3:x190} the semantic link is not very strong.

\ea%x190
\label{ex:3:x190}
\gll Nefa \textstyleEmphasizedVernacularWords{ikum-am-ika-iwkin} nan kerer-e-n. \\
2s.\textsc{acc} illicitly-\textsc{ss}.\textsc{sim}-be-2/3p.\textsc{ds} there appear-\textsc{pa}-2s \\
\glt`They were just speculating about you when you arrived.' 
\z

\paragraph{Inchoative suffix} \label{sec:3:a:z:y:x}
%\hypertarget{RefHeading20141935131865}
{}
The second verb formation process in Mauwake takes a noun, adjective or adverb root and adds an inchoative suffix -\textstyleStyleVernacularWordsItalic{ar} (\sectref{sec:2.3.3.4}) to form a new verb usually meaning `become n'.\footnote{The term `inchoative' is used for derivation; `inceptive' for aspect, following \citet[95]{Payne1997}.} Although in the majority of the cases a word from one of the other word classes is made into a verb, the basic meaning is inchoative rather than verbalizing, as the same suffix can also be added to a few verbs. The suffix has been grammaticalized from the verb \textstyleStyleVernacularWordsItalic{ar}\textstyleEmphasizedVernacularWords{-} `become', `enter into a state', and there are a few cases where it is difficult to decide with certainty which one it is. The differences between the full verb and the suffix are listed below. 

The full resultative verb \textstyleStyleVernacularWordsItalic{ar}\textstyleEmphasizedVernacularWords{-} `become' is more common with nouns, and the meaning of the verb is transparent \REF{ex:3:x191}. It is also used with numerals \REF{ex:3:x192}. Both words retain their word stress.

\ea%x191
\label{ex:3:x191}
\gll Arim-emi mu'a \textstyleEmphasizedVernacularWords{ar-'e-k}. \\
grow-\textsc{ss}.\textsc{sim} man become-\textsc{pa}-3s \\
\glt`He grew up and became man/adult.'
\z

\ea%x192
\label{ex:3:x192}
\gll Aruf-owa e\textstyleEmphasizedVernacularWords{'}repam \textstyleEmphasizedVernacularWords{ar-'e-m}. \\
hit-\textsc{nmz} four become-\textsc{pa}-1s \\
\glt`I hit it four times.' (Lit: `Hitting it I became four.')
\z

The inchoative suffix \textstyleEmphasizedVernacularWords{\nobreakdash-}\textstyleStyleVernacularWordsItalic{ar} can occur with nouns \REF{ex:3:x193}, but is more common with adjectives \REF{ex:3:x194} and adverbs \REF{ex:3:x195}, and can attach to a few verb roots too \REF{ex:3:x196}. Since the result is one word it only has one word stress. 

\ea%x193
\label{ex:3:x193}
\gll Yiena opaimika me baliwep \textstyleEmphasizedVernacularWords{a'mis-ar-e-mik}. \\
1p.\textsc{gen} talk not well knowledge-\textsc{inch}-\textsc{pa}-1/3p \\
\glt`They don't know our language well.'
\z

\ea%x194
\label{ex:3:x194}
\gll Miiw-aasa \textstyleEmphasizedVernacularWords{sa'mor-ar-ek}.{\footnotemark} \\
land-canoe bad-\textsc{inch}-\textsc{pa}-3s \\
\glt`The car broke.'
\z

\footnotetext{\textstyleFootnoteBaseChar{\textit{Miiw-aasa samor-a-k}} `He broke the car' would be a corresponding sentence with zero verbalization. }

\ea%x195
\label{ex:3:x195}
\gll Kau pun weeser-owa \textstyleEmphasizedVernacularWords{e'wur-ar-ek.} \\
cow too finish-\textsc{nmz} quickly-\textsc{inch}-\textsc{pa}-3s \\
\glt`The beef finished quickly too.'
\z

\ea%x196
\label{ex:3:x196}
\gll Mua \textstyleEmphasizedVernacularWords{i'men-ar-ep} opora pun \textstyleEmphasizedVernacularWords{i'men-ar-ek}. \\
man find-\textsc{inch}-\textsc{ss}.\textsc{seq} talk too find-\textsc{inch}-\textsc{pa}-3s\\
\glt`When man appeared, talk/language appeared too.'
\z

Verbs derived from adjectives are often used rather than adjectives in the predicative position \REF{ex:3:x82}, \REF{ex:3:x1764}. And instead of a modifying adjective, a whole relative clause with a verb derived from an adjective may be used \REF{ex:3:x83}. This happens especially when the property denoted by the adjective is not static. 

\ea%x82
\label{ex:3:x82}
\gll Sia nain senam \textstyleEmphasizedVernacularWords{pin(a)-ar-e-k}. \\
netbag that1 too.much heavy-\textsc{inch}-\textsc{pa}-3s\\
\glt`The netbag is/was (lit: became) very heavy.' 
\z

\ea%x1764
\label{ex:3:x1764}
\gll Muuka nain op-iya \textstyleEmphasizedVernacularWords{dubil}\textstyleEmphasizedVernacularWords{(a)-}\textstyleEmphasizedVernacularWords{al}\textstyleEmphasizedVernacularWords{-}\textstyleEmphasizedVernacularWords{e}\textstyleEmphasizedVernacularWords{-}\textstyleEmphasizedVernacularWords{k}. \\
boy that1 hold-2/3s.\textsc{ds} slippery-\textsc{inch}-\textsc{pa}-3s\\
\glt`When he\textsubscript{1} held the boy\textsubscript{2}, he\textsubscript{2} was slippery.'
\z

\ea%x83
\label{ex:3:x83}
\gll [Konima \textstyleEmphasizedVernacularWords{supuk(a)-ar-e-k} \textstyleEmphasizedVernacularWords{nain}] yasuw-e. \\
cloth wet-\textsc{inch}-\textsc{pa}-3s that1 wash-\textsc{imp}.2s\\
\glt`Wash the wet cloth.' (Lit: `Wash the cloth that has become wet.')
\z

The consonant /r/ in the suffix is lateralized into /l/ when the root has /l/ in the immediately preceding syllable. Lateralization takes place arbitrarily in a few other cases as well \REF{ex:3:}.

\ea%x197
\label{ex:3:x197}
\gll Yo damol(a)-\textstyleEmphasizedVernacularWords{al}-e-m oo. \\
1s.\textsc{unm} bad-\textsc{inch}-\textsc{pa}-1s oh \\
\glt`I feel terrible.' (Lit: `I'm destroyed/ruined.') 
\z

\ea%x198
\label{ex:3:x198}
\gll Epa dabel(a)-\textstyleEmphasizedVernacularWords{al}-ek. \\
place cold-\textsc{inch}-\textsc{pa}-3s \\
\glt`It is cold.'
\z

\ea%x199
\label{ex:3:x199}
\gll Opaimika efa masi(a)-\textstyleEmphasizedVernacularWords{al}-i-ya. \\
mouth 1s.\textsc{acc} bitter-\textsc{inch}-Np-\textsc{pr}.3s \\
\glt`It tastes bitter to me / in my mouth.'
\z

If two preceding syllables contain /l/, the consonant in the verbaliser is not lateralized.\footnote{This rule is very tentative, as \textstyleFootnoteBaseChar{\textit{ilelar}}- is the only example found so far.}

\ea%x200
\label{ex:3:x200}
\gll Aasa puuk-ap ilel(a)-\textstyleEmphasizedVernacularWords{ar}-i-ya. \\
canoe cut-\textsc{ss}.\textsc{seq} gouge-\textsc{inch}-Np-\textsc{pr}.3s \\
\glt`He has cut the canoe (length from a tree) and is gouging/carving it' 
\z

The suffix \textstyleStyleVernacularWordsItalic{ar}- often retains its original verbal meaning `become' when adjectives are made into verbs \REF{ex:3:x201}, but when the other word classes are used as the root the original meaning tends to become more opaque or get lost \REF{ex:3:x202}. 

\ea%x201
\label{ex:3:x201}
\gll \textstyleEmphasizedVernacularWords{Dubil}\textstyleEmphasizedVernacularWords{(a)-al-e-k.} \\
slippery/smooth-\textsc{inch}-\textsc{pa}-3s \\
\glt`It became slippery/smooth.' 
\z

\ea%x202
\label{ex:3:x202}
\gll No \textstyleEmphasizedVernacularWords{wadol}\textstyleEmphasizedVernacularWords{(a)-al-i-n.} \\
2s.\textsc{unm} lie-\textsc{inch}-Np-\textsc{pr}.2s \\
\glt`You are lying.' 
\z

Most of the verbs formed with the inchoative suffix are intransitive, but some are active, transitive verbs:

\ea%x203
\label{ex:3:x203}
\gll Muuka kuisow \textstyleEmphasizedVernacularWords{muuk}\textstyleEmphasizedVernacularWords{(a)-ar-e-k}. \\
son one son-\textsc{inch}-\textsc{pa}-3s \\
\glt`She gave birth to one son.'
\z

\ea%x204
\label{ex:3:x204}
\gll Epa \textstyleEmphasizedVernacularWords{mores-ar-ep} ikiw-o-k. \\
place like(ADV)-\textsc{inch}-\textsc{ss}.\textsc{seq} go-\textsc{pa}-3s \\
\glt`He made the place ready and went.'
\z

The inchoative verb formation is also used with verb loans from other languages, especially Tok Pisin.\footnote{The Tok Pisin loans often originally come from English.} Both of the loan words below also have a vernacular synonym.

\ea%x487
\label{ex:3:x487}
\gll Muuka wia \textstyleEmphasizedVernacularWords{was-ar-e-mik}.\textstyleParagraphChari{ (from Tok Pisin} \textstyleForeignWords{was} \textstyleParagraphChari{`look after')} \\
son 3p.\textsc{acc} look.after-\textsc{inch}-\textsc{pa}-1/3p\\
\glt`They were looking after the boys/children'
\z

\ea%x488
\label{ex:3:x488}
\gll \textstyleEmphasizedVernacularWords{Nading-ar-ep} uf-e-mik. (from Mala \textstyleForeignWords{nading} `decoration') \\
decoration-\textsc{inch}-\textsc{ss}.\textsc{seq} dance-\textsc{pa}-1/3p\\
\glt`We decorated ourselves and danced.'
\z

\subsubsection{Category-maintaining derivation: suffixes}\label{sec:3:z:y:x}
%\hypertarget{RefHeading20161935131865}
{}
\paragraph{Causative suffix}\label{sec:3:a:z:y:x}
%\hypertarget{RefHeading20181935131865}
{}
The causative suffix \nobreakdash-\textstyleStyleVernacularWordsItalic{ow} (or \nobreakdash-\textstyleStyleVernacularWordsItalic{aw}) transitivizes an intransitive verb \citep[2]{Peterson2007}: the clause gets a new subject, and the subject of the intransitive verb becomes the direct object. Usually it adds a causative meaning `cause someone to do something', or `cause something to happen'. The object of a causative construction has no control, or only minimal control, over the action or event indicated by the verb.

In many verbs there is free variation between \nobreakdash-\textstyleStyleVernacularWordsItalic{ow} and \nobreakdash-\textstyleStyleVernacularWordsItalic{aw}. Some verbs seem to prefer one or the other, but there is no clear pattern. There is also some dialectal and possibly age-based variation depending on the speaker. \nobreakdash-\textstyleStyleVernacularWordsItalic{ow} is taken here as the basic form, since it is the more common of the two, and because in ``double causatives'' it is always used at least as the first one. 

\begin{table}
\caption{Causative from} 
\todo[inline]{find better caption}
\begin{tabular}{>{\itshape}ll>{\itshape}ll}
\mytoprule
arim-ow- &`bring up / raise' &arim- &`grow'\\
in-aw- &`put to bed' &in- &`lie down'\\
bagiwir-ow &`cause to be angry' &bagiwir- &`be angry'\\
iimar-ow- &`make sg. stand up' &iimar- &`stand up'\\
imenar-ow- &`create/cause to appear' &imenar- &`appear'\\
waki-ow-aw- &`cause to stumble' &waki- &`stumble'\\
ook-ow- &`place alongside' &ook- &`follow'\\
\mybottomrule
\end{tabular}
\end{table}

Sometimes the causative suffix occurs reduplicated as a ``double causative'', but these still add only one argument. Many of the short directional verbs (\sectref{sec:3.8.4.4.5}) take a double causative instead of a single one. 

\ea%x205
\label{ex:3:x205}
\gll Eewua ir\textstyleEmphasizedVernacularWords{-ow-aw-}ap osaiwa ar-e-k. \\
wing climb-CAUS-CAUS-\textsc{ss}.\textsc{seq} bird.of.paradise become-\textsc{pa}-3s \\
\glt`She put the wing up (on herself) and became a bird of paradise.'
\z

A single or double causative can be added to the intransitive verb \textstyleStyleVernacularWordsItalic{reen}- `(become) dry' with the result of two different meanings, but both of these still only add one more argument: \textstyleStyleVernacularWordsItalic{reenow}- `dry (something)', \textstyleStyleVernacularWordsItalic{reenowaw}- `smoke (something)'.

The only two transitive verbs that have been found to take the causative are \textstyleStyleVernacularWordsItalic{mik}- `spear/hit' and \textstyleStyleVernacularWordsItalic{op}- `hold/grab', with the causative forms \textstyleStyleVernacularWordsItalic{mik-ow-aw}- `join (the ends of two long items)' and \textstyleStyleVernacularWordsItalic{op-aw}- `accuse falsely'. 

Verbs that do \textstyleEmphasizedWords{\textsc{not}} have any causative meaning include the following:

\begin{table}
\caption{Please provide a caption}
\label{} 
\begin{tabular}{lllll}
\mytoprule
aakun-ow- &`grumble (at)' &from: &aakun- &`speak'\\
baun-ow- &`bark (at)' & &baun- &`bark'\\
kirir-ow- &`shout (about)' & &kirir- &`shout'\\
op-aw- &`accuse falsely' & &op- &`hold'\\
\mybottomrule
\end{tabular} 

\end{table}

\ea%x991
\label{ex:3:x991}
\gll Mukuna kuuf-ap kirir-e-k. \\
fire see-\textsc{ss}.\textsc{seq} shout-\textsc{pa}-3s\\
\glt`She saw the fire and shouted.'
\z

\ea%x489
\label{ex:3:x489}
\gll Yiok-ami naap \textstyleEmphasizedVernacularWords{yia} \textstyleEmphasizedVernacularWords{kirir-ow-am-ik-ua.} \\
follow.us-\textsc{ss}.\textsc{sim} thus 1p.\textsc{acc} shout-CAUS-\textsc{ss}.\textsc{sim}-be-\textsc{pa}.3s\\
\glt`She was following us and shouting about us like that.'
\z

The causative as a valence-increasing device is discussed in \sectref{sec:3.8.4.3.1}. 

\paragraph{Distributive suffix}\label{sec:3:a:z:y:x}
%\hypertarget{RefHeading20201935131865}
{}
A distributive suffix pluralizes one of the verbal arguments. There are two distributive suffixes: \nobreakdash-\textstyleStyleVernacularWordsItalic{urum} `all' and \nobreakdash-\textstyleStyleVernacularWordsItalic{omak} `many'. They are fully productive in the whole verb class, as long as the semantics of the verb allows multiple arguments. 

The hierarchy of which argument the distributive applies to is as follows: if there is a recipient \REF{ex:3:x209} or beneficiary \REF{ex:3:x429}, the distributive applies to that; if there is no recipient or beneficiary but an object, the distributive applies to the object \REF{ex:3:x492}; and in case the clause has neither a recipient or beneficiary nor an object, the distributive applies to the subject \REF{ex:3:x491}. Since transitive verbs need an object, the subject can be pluralized with the distributive only when the verb is intransitive.

\ea
REC/BEN {{\textgreater}} O {{\textgreater}} S
\z

\ea%x209
\label{ex:3:x209}
\gll Mua teeria opaimika wia sesek-\textstyleEmphasizedVernacularWords{omak}-e-mik. \\
man family talk 3p.\textsc{acc} send-\textsc{distr}/\textsc{pl}-\textsc{pa}-1/3p \\
\glt`They sent word to (many members of) the man's family.'
\z

\ea%x429
\label{ex:3:x429}
\gll Wiena wiawi=ke amia wia keraw-om-\textstyleEmphasizedVernacularWords{omak}-e-mik. \\
3p.\textsc{gen} 3s/p.father=\textsc{cf} spear 3p.\textsc{acc} carve-\textsc{ben}-\textsc{distr}/\textsc{pl}-\textsc{pa}-1/3p\\
\glt`Their fathers carved spears for them (\textit{many} beneficiaries).'
\z

\ea%x492
\label{ex:3:x492}
\gll Emeria unowa fain nia aaw-\textstyleEmphasizedVernacularWords{urum}-i-kuan. \\
woman many this 2s.\textsc{acc} take-\textsc{distr}/\textsc{a}-Np-\textsc{fu}.3p\\
\glt`They will take all of you women.'
\z

\ea%x491
\label{ex:3:x491}
\gll Emeria teeria koka ikiw-\textstyleEmphasizedVernacularWords{urum}-e-mik. \\
woman group jungle go-\textsc{distr}/\textsc{a}-\textsc{pa}-1/3p\\
\glt`The whole group of women / all the women went to the jungle.'
\z

In verbal groups (\sectref{sec:3.8.5.1}) the distributive suffix usually attaches to the last verb root, but it can occasionally also attach to the first root, i.e. the main verb in a verb+\textstyleAcronymallcaps{AUX} combination \REF{ex:3:}.

\ea%x207
\label{ex:3:x207}
\gll Iinan aasa ikiw-emi paran-em-\textstyleEmphasizedVernacularWords{mi-omak-e-k}. \\
sky canoe go-\textsc{ss}.\textsc{sim} rumble-\textsc{ss}.\textsc{sim}-go.around-\textsc{distr}/\textsc{pl}-\textsc{pa}-3s\\
\glt `Many planes went rumbling around.'
\z

\ea%x490
\label{ex:3:x490}
\gll Iinan aasa fan or-om\textstyleEmphasizedVernacularWords{-ik-omak-eya} {\dots} \\
sky canoe here1 descend-\textsc{ss}.\textsc{sim}-be-\textsc{distr}/\textsc{pl}-2/3s.\textsc{ds}\\
\glt`When many planes were coming down here {\dots}'
\z

\ea%x208
\label{ex:3:x208}
\gll Wi ifa saarik \textstyleEmphasizedVernacularWords{in-urum-ep}-ik-e-mik. \\
3p.\textsc{unm} snake like sleep-\textsc{distr}/\textsc{a}-\textsc{ss}.\textsc{seq}-be-\textsc{pa}-1/3p \\
\glt`They all slept/lay like snakes.'
\z

Both suffixes can be attached to the same verb but it is rare. In that case \textstyleEmphasizedVernacularWords{\nobreakdash-}\textstyleStyleVernacularWordsItalic{urum} precedes \textstyleEmphasizedVernacularWords{\nobreakdash-}\textstyleStyleVernacularWordsItalic{omak}.

\ea%x206
\label{ex:3:x206}
\gll Wia ifakim\textstyleEmphasizedVernacularWords{-urum}-\textstyleEmphasizedVernacularWords{omak}-e-mik. \\
3p.\textsc{acc} kill-\textsc{distr}/\textsc{a}-\textsc{distr}/\textsc{pl}-\textsc{pa}-1/3p \\
\glt`They killed each and every one of them.' (There were many of those killed.) 
\z

\paragraph{Benefactive suffix} \label{sec:3:a:z:y:x}
%\hypertarget{RefHeading20221935131865}
{}
The benefactive suffix, indicating the fact that the action of the verb is done \textstyleEmphasizedWords{\textsc{for someone}}, for their benefit or detriment, is a borderline case among the derivations. It is the last one of the derivational suffixes, and the \textstyleEmphasizedWords{\textsc{beneficiary suffix}} (\sectref{sec:3.8.3.1}) following it and marking the person that the action is done for, is inflectional even if the two suffixes go together semantically. The position of the benefactive is not as stable as that of the other suffixes: it comes after the distributive when the beneficiary is first person singular \REF{ex:3:x210}, \REF{ex:3:x1925} but occurs preceding it with the other persons \REF{ex:3:x211}, \REF{ex:3:x1926}. 

\ea%x210
\label{ex:3:x210}
\gll Mua Maneka=ke maa maneka on-omak-\textstyleEmphasizedVernacularWords{om}-e-k. \\
Man Big=\textsc{cf} thing big do-\textsc{distr}/\textsc{pl}-\textsc{ben}-\textsc{bnfy}1.\textsc{pa}{\footnotemark} -3s \\
\glt`God did great things to/for me.'
\z
\footnotetext{The vowel of the beneficiary suffix deletes the vowel of the past tense suffix. The relationship between the beneficiary suffix and the suffix following it is discussed in detail in \sectref{sec:3.8.3.1}, and the medial suffix forms are discussed in \sectref{sec:3.8.3.5}.}

\ea%x1925
\label{ex:3:x1925}
\gll Buk aaw-omak-\textstyleEmphasizedVernacularWords{om}-e! \\
book get-\textsc{distr}/\textsc{pl}-\textsc{ben}-\textsc{bnfy}1.\textsc{imp}.2s\\
\glt`Get the books for me!'
\z

\ea%x211
\label{ex:3:x211}
\gll Buk aaw-\textstyleEmphasizedVernacularWords{om}-omak-e! \\
book get-\textsc{ben}-\textsc{distr}/\textsc{pl}-\textsc{imp}.2s \\
\glt`Get the books for him!'
\z

\ea%x1926
\label{ex:3:x1926}
\gll Wiena wiawi=ke amia wia keraw-\textstyleEmphasizedVernacularWords{om}-omak-e-mik. \\
3p.\textsc{gen} 3s/p.father=\textsc{cf} bow 3p.\textsc{acc} carve-\textsc{ben}-\textsc{distr}/\textsc{pl}-\textsc{pa}-1/3p \\
\glt`Their fathers carved bows for them.'
\z

In verbal groups the benefactive suffix is usually attached to the finite verb or auxiliary \REF{ex:3:x212} but can occasionally occur on the non-finite root \REF{ex:3:x213} or even on both of the two \REF{ex:3:x214}.

\ea%x212
\label{ex:3:x212}
\gll Iwera wia uruk-am-ik-\textstyleEmphasizedVernacularWords{om}-a-mik. \\
coconut 3p.\textsc{acc} drop-\textsc{ss}.\textsc{sim}-be-\textsc{ben}-\textsc{bnfy}2.\textsc{pa}-1/3p\\
\glt`We kept dropping coconuts for them.' 
\z

\ea%x213
\label{ex:3:x213}
\gll Maamuma wia p-ikiw-\textstyleEmphasizedVernacularWords{om}-ap-pu-ap {\dots} \\
money 3p.\textsc{acc} BPf-go-\textsc{ben}-\textsc{bnfy}2.\textsc{ss}.\textsc{seq}-\textsc{cmpl}-\textsc{ss}.\textsc{seq} \\
\glt`Having taken money to them, {\dots} '
\z

\ea%x214
\label{ex:3:x214}
\gll Moro mua wia wu-\textstyleEmphasizedVernacularWords{om}-am-ik-\textstyleEmphasizedVernacularWords{om}-a-mik. \\
Moro man 3p.\textsc{acc} put-\textsc{ben}-\textsc{bnfy}2.SS.SIM-be-\textsc{ben}-\textsc{bnfy}2.\textsc{pa}-1/3p\\
\glt`They put them (=carts) for the Moro men.' 
\z

The benefactive form does not always mean that something happens for someone's\textstyleEmphasizedWords{} \textstyleEmphasizedWords{\textsc{benefit}}. The benefactive may be strengthened with the adverb \textstyleStyleVernacularWordsItalic{orawin} `for the benefit' \REF{ex:3:x215}, which makes it unambiguous.

\ea%x215
\label{ex:3:x215}
\gll Iwera \textstyleEmphasizedVernacularWords{orawin} kais-\textstyleEmphasizedVernacularWords{om}-e-mik. \\
Coconut for.the.benefit husk-\textsc{ben}-\textsc{bnfy}1.\textsc{pa}-1/3p \\
\glt`They husked coconuts for me (for free).'
\z

By using a suffix completely unrelated to the verb `give', Mauwake shows itself different from all of those reasonably closely related languages that have grammatical descriptions available. A serial verb construction involving the verb `give' is a very common way of expressing benefactive in Papuan languages \citep[141]{Foley1986}. Waskia ({Ross and Paol 1978}:45) and Maia \citep[125]{Hardin2002} employ this strategy, and in Usan it is behind one of the two strategies: the benefactive verb form has been grammaticalized from a serial verb with the verb `give' ({Reesink 1987}:110--111). The other strategy for Usan is to use a postposition with the appropriate noun phrase (ibid. 154). Bargam is similar to it ({Hepner 2002}:65--66, 99), but Amele utilizes an indirect object clitic attached to the verb to express the beneficiary as well as other semantic relations \citep[167]{Roberts1987}. 

\subsubsection{Derivational prefixes}\label{sec:3:z:y:x}
%\hypertarget{RefHeading20241935131865}
{}
Although Mauwake is very strongly a suffixing language, it makes use of some derivational prefixes as well. Reduplication is the most common among these. 

\paragraph{Reduplication}\label{sec:3:a:z:y:x}
%\hypertarget{RefHeading20261935131865}
{}
The morphophonological aspect of reduplication was already described in \sectref{sec:2.3.3.2}. In \sectref{sec:6.4.2} reduplication is discussed as one of the many quantification strategies in Mauwake.

Reduplication in verbs is used in Mauwake to indicate continuity or iterativity of action and/or plurality of the resulting object. Mostly the reduplication is done only once, but especially motion verbs can have several identical reduplicative prefixes. 

In verbs of motion reduplication means continuity \REF{ex:3:x218}, and the passing of time may be shown by the number of reduplications \REF{ex:3:x216}.

\ea%x218
\label{ex:3:x218}
\gll \textstyleEmphasizedVernacularWords{Biri-birin-emi} wia akim-omak-e-mik. \\
\textsc{rdp}-fly-\textsc{ss}.\textsc{sim} 3p.\textsc{acc} try-\textsc{distr}/\textsc{pl}-\textsc{pa}-1/3p \\
\glt`They were flying and teasing them.' 
\z

\ea%x216
\label{ex:3:x216}
\gll Ne \textstyleEmphasizedVernacularWords{oro-oro-oro}-oro-mi \textstyleEmphasizedVernacularWords{oro-oro}-or-o-k, onoma.\\
and \textsc{rdp}-\textsc{rdp}-\textsc{rdp}-descend-\textsc{ss}.\textsc{sim} \textsc{rdp}-\textsc{rdp}-descend-\textsc{pa}-3s horizon.\\
\glt`And it went down and down and down all the way to the horizon.'
\z

In other intransitive verbs reduplication indicates either iterative action \REF{ex:3:x217} or occasionally continuity \REF{ex:3:x692}.

\ea%x217
\label{ex:3:x217}
\gll Nomokowa \textstyleEmphasizedVernacularWords{ku-ku-ep} or-om-ik-ua. \\
tree \textsc{rdp}-break-\textsc{ss}.\textsc{seq} descend-\textsc{ss}.\textsc{sim}-be-\textsc{pa}.3s\\
\glt`The timber (in a bridge) kept breaking and falling down.' 
\z

\ea%x692
\label{ex:3:x692}
\gll Epa \textstyleEmphasizedVernacularWords{wii-wiim-ik-ua}, {\dots} \\
place \textsc{rdp}-dawn-be-\textsc{pa}.3s\\
\glt`It was dawning, {\dots}'
\z

Both of these meanings fit in well with \citegen[319]{Moravcsik1978} description of the various meanings that reduplication in verbs can have. In transitive verbs reduplication indicates iterative action as well as the plurality of an inanimate object \REF{ex:3:x219}. The form is used especially when the action \textstyleEmphasizedWords{results} in a plural object \REF{ex:3:x220}.

\ea%x219
\label{ex:3:x219}
\gll Iinan aasa=ke maifa \textstyleEmphasizedVernacularWords{fu-fuurk-ikiw-o-k}. \\
sky canoe=\textsc{cf} paper \textsc{rdp}-throw-go-\textsc{pa}-3s \\
\glt`The plane went throwing paper slips down' 
\z

\ea%x220
\label{ex:3:x220}
\gll Oposia nain \textstyleEmphasizedVernacularWords{pu-puuk-ap} uup-e-mik. \\
meat that1 \textsc{rdp}-cut-\textsc{ss}.\textsc{seq} cook-\textsc{pa}-1/3p \\
\glt`We cut up the meat (into many pieces) and cooked it.'
\z

Usan differs from Mauwake in that it does not use reduplication very much in verbs, and never in main clause final verbs \citep[116]{Reesink1987}. Also, when reduplication is used to indicate duration or repetition the whole verb word is reduplicated (ibid. 117). In Bargam reduplication occurs but is not very productive. In transitive verbs reduplication indicates plurality of objects, in intransitive verbs plurality of subjects \citep[19]{Hepner2002}. In Maia ``{verb roots may be partially or completely reduplicated. Verb reduplication broadly indicates an augmented action which may include a greater, more massive, more intensified or very often repetitive form of the action}'' \citep[50]{Hardin2002}.

\paragraph{Bring-\textsc{pr}efixes}\label{sec:3:a:z:y:x}
%\hypertarget{RefHeading20281935131865}
{}
The prefixes in this group change the directional verbs (see \sectref{sec:3.8.4.4.5}) into transitive verbs with the meaning `bring' or `take'. \textstyleStyleVernacularWordsItalic{p}\nobreakdash- is a neutral prefix and by far the most common one \REF{ex:3:}, \textstyleStyleVernacularWordsItalic{amap}\nobreakdash- is used when something is brought out in the open, often with the meaning `bring forth'. Usually there is a clear goal, a person or a place, which may not be mentioned in the clause itself but occurs in an earlier one \REF{ex:3:}, or is understood from the context \REF{ex:3:}. If the goal is explicitly mentioned in the clause, the neutral prefix is used \REF{ex:3:}, \REF{ex:3:}. The prefix \textstyleStyleVernacularWordsItalic{aap}\nobreakdash- \REF{ex:3:} is very rare and I have been unable to establish whether it really differs from \textstyleStyleVernacularWordsItalic{amap}\nobreakdash- or whether it is just a matter of idiolectal use.\footnote{The bring\nobreakdash-\textsc{pr}efixes may have been grammaticalized from a medial verb construction involving the verb \textit{aaw}\nobreakdash- `take'. It is easy to see how \textit{aawep ekap}\nobreakdash- `take (and) come' could have developed into \textit{aapekap}\nobreakdash- `bring' and possibly also into \textit{pekap}-. Another possibility is that it is a result of a related process to that in Usan where the verb \textstyleFootnoteBaseChar{\textit{ba}} `take' has contracted into \textstyleFootnoteBaseChar{\textit{b}}\nobreakdash-, which has combined with verbs of motion and been lexicalized with the meaning of bringing or taking ({Reesink 1987}:144--145). The \textstyleFootnoteBaseChar{\textit{amap}}\nobreakdash-\textsc{pr}efix may have its origin in the expression \textstyleFootnoteBaseChar{\textit{ama-\textsc{pa}}} `in the sun', which implies `in the open'. There is also a very slight possibility that the \textit{p}\nobreakdash-\textsc{pr}efix might be an Austronesian loan, as p(V\textit{)\nobreakdash-} is a common causative or transitivizer prefix in Austronesian languages \citep[61]{Bugenhagen1995}. \textstyleFootnoteBaseChar{But all of this is just conjecture at this point.}}

\ea%x221
\label{ex:3:x221}
\gll Amina aaw-ep Liisa ame wia \textstyleEmphasizedVernacularWords{p-er}-om-a.\\
pot take/get-\textsc{ss}.\textsc{seq} Liisa others 3p.\textsc{acc} Bpx-go-\textsc{ben}-\textsc{bnfy}2.\textsc{imp}.2s\\
\glt`Get the pot and take it to Liisa and the others.' 
\z

\ea%x430
\label{ex:3:x430}
\gll Pita pensil wiar or-op ik-ua nain aaw-ep \textstyleEmphasizedVernacularWords{amap-ikiw}-om-aka.\\
Pita pencil 3.\textsc{dat} fall-\textsc{ss}.\textsc{seq} be-\textsc{pa}.3s that1 take-\textsc{ss}.\textsc{seq} Bpx-go-\textsc{ben}-\textsc{bnfy}2.\textsc{imp}.2p\\
\glt`Take to Pita his pencil that has dropped.'
\z

\ea%x222
\label{ex:3:x222}
\gll Wiipa oko \textstyleEmphasizedVernacularWords{amap-ora}-iwkin ma-e-k {\dots} \\
daughter other Bpx-descend-2/3p.\textsc{ds} say-\textsc{pa}-3s \\
\glt`When they took another daughter down (from the house out in the open), he said{\dots}' 
\z

\ea%x223
\label{ex:3:x223}
\gll Ni auwa maa \textstyleEmphasizedVernacularWords{p-urup}-om-aka. \\
2p.\textsc{unm} father food Bpx-ascend-\textsc{ben}-\textsc{bnfy}2.\textsc{imp}.2p \\
\glt`Take food (up) to father.'
\z

\ea%x224
\label{ex:3:x224}
\gll Iwera ir-ap erup op-ap \textstyleEmphasizedVernacularWords{aap-or}-e. \\
coconut go.up-\textsc{ss}.\textsc{seq} two grab-\textsc{ss}.\textsc{seq} Bpx-descend-\textsc{imp}.2s \\
\glt`Climb the coconut palm, grab two coconuts and bring them down.' 
\z

\subsection{Verb inflection} \label{sec:3:y:x}
%\hypertarget{RefHeading20301935131865}
{}
\tabref{tab:11} shows those inflectional suffixes for the Mauwake verbs that change with the person and/or number of the subject. All of these are discussed in more detail below.

\begin{table}
\begin{tabular}{l}
\mytoprule
  {bnfy}  {{cntf}}
  {tense}  {2}{l}{{pers./ number}}
  {{imperat.}}
  {{medial}}\\
     
non-\textsc{pa}st:
-i

past:
\textbf{-}e / -a  {pres}
  past    {same subject}\\
{1s}

  -e  -ek   {\textbf{-}yem}
  \textbf{-}m  {-u \textstyleTableEntryChar{(1d)}}

  {SEQUENTIAL: -ep/ap}\\

2s     {2}{l}{\textbf{-}n}
  {\textbf{-}e /-a}
  {SIMULTANEOUS: -emi/ami}

{3s}
  -a    {\textbf{-}ya}
  \textbf{-}k  {\textbf{-}inok}
  {DIFFERENT SUBJECT}

1p     {2}{l}{\textbf{-}mik}
  {\textbf{-}ikua}
  {2}{l}{\textbf{-}Vmkun (s \ p)}
  1\\

2p     {2}{l}{\textbf{-}man}
  {\textbf{-}eka\textbf{ /-}aka}
  {\textbf{-}eya (s)}
  {\textbf{-}iwkin (p)}
  2\\

3p     {2}{l}{\textbf{-}mik}
  \textbf{-}uk    3\\

     {2}{l}{FUTURE}
  {4}{l}\\

1s     {2}{l}{\textbf{-}nen}


2s     {2}{l}{\textbf{-}nan}


3s     {2}{l}{\textbf{-}non}


1p     {2}{l}{\textbf{-}yen}


2p     \textbf{-}o (Np)  {2}{l}{\textbf{-}wen}


3p      {2}{l}{\textbf{-}kuan}
\\
\mybottomrule 
\end{tabular}

\caption{Inflectional suffixes of Mauwake verbs}
\todo[inline]{This table looks already strange in the word version. Please provide a new table in word and pdf}
\label{tab:11}
\end{table}

\subsubsection{Beneficiary}\label{sec:3:z:y:x}
%\hypertarget{RefHeading20321935131865}
{}
The beneficiary suffix indicates the person the action is done for. Its position is directly after the benefactive suffix, or after the distributive suffix in those few cases where the benefactive comes before the distributive (\sectref{sec:3.8.2.3.3}). It is inflectional rather than derivational because 1) when it is used, nominalization is blocked and 2) it has a paradigm for different persons, even if the paradigm only consists of two members. 

The only two forms for the beneficiary are \textstyleEmphasizedVernacularWords{\nobreakdash-}\textstyleStyleVernacularWordsItalic{e} for first or second person singular \REF{ex:3:x225} and \textstyleEmphasizedVernacularWords{\nobreakdash-}\textstyleStyleVernacularWordsItalic{a} for all the other persons \REF{ex:3:x226}. The context often provides more person distinctions, as the plural requires accusative pronouns to precede the verb to indicate the beneficiary, like third person plural in \REF{ex:3:x227}. 

\ea%x225
\label{ex:3:x225}
\gll Wafur-om-\textstyleEmphasizedVernacularWords{e}! \\
throw-\textsc{ben}-\textsc{bnfy}1.\textsc{imp}.2s \\
\glt`Throw it to me!' 
\z

\ea%x226
\label{ex:3:x226}
\gll Marasin wu-om-\textstyleEmphasizedVernacularWords{a}-mik=na weetak. \\
medicine put-\textsc{ben}-\textsc{bnfy}2.\textsc{pa}-1/3p=\textsc{tp} no \\
\glt`They put medicine on him but no (it didn't help).'
\z

\ea%x227
\label{ex:3:x227}
\gll Na-iwkin \textstyleEmphasizedVernacularWords{wia} uf-om-\textstyleEmphasizedVernacularWords{a}-mik. \\
say-2/3p.\textsc{ds} 3p.\textsc{acc} dance-\textsc{ben}-\textsc{bnfy}2.\textsc{pa}-1/3p \\
\glt`They said so and we danced for them.'
\z

When the beneficiary suffix is followed by a vowel, a mid vowel is deleted adjacent to a low vowel \REF{ex:3:} and both a mid and a low vowel are deleted preceding a high vowel \REF{ex:3:}. In the latter case the person distinction gets neutralized in the singular \REF{ex:3:}, but not in the plural where the obligatory accusative pronouns maintain the distinction \REF{ex:3:}. The examples \REF{ex:3:}-\REF{ex:3:} below show how the beneficiary suffix affects the past tense suffix. In \REF{ex:3:} a sequence of two identical vowels is reduced to one vowel.

\ea%x228
\label{ex:3:x228}
\gll aaw-om-\textstyleEmphasizedVernacularWords{ak}-a-m {{\textless} aaw-om}-\textstyleEmphasizedVernacularWords{a-ek}-a-m \\
get-\textsc{ben}-\textsc{bnfy}2.CNTF-\textsc{pa}-1s \\
\glt`I would have gotten it for him' 
\z

\ea%x444
\label{ex:3:x444}
\gll aaw-om-\textstyleEmphasizedVernacularWords{i-non} {{\textless} aaw-om-}\textstyleEmphasizedVernacularWords{e-i-non}, aaw-om-\textstyleEmphasizedVernacularWords{a-i-non} \\
get-\textsc{ben}-\textsc{bnfy}.Np-\textsc{fu}.3s\\
\glt`he will get it for me/you/him/her'
\z

\ea%x1750
\label{ex:3:x1750}
\gll aaw-om-\textstyleEmphasizedVernacularWords{uk} {{\textless} aaw-om-}\textstyleEmphasizedVernacularWords{e-uk}, aaw-om-\textstyleEmphasizedVernacularWords{a-uk} \\
get-\textsc{ben}-\textsc{bnfy}.\textsc{imp}.3p\\
\glt`let them get it for me/you/him/her'
\z

\ea%x445
\label{ex:3:x445}
\gll Panewowa maa \textstyleEmphasizedVernacularWords{wia} p-ikiw-om-\textstyleEmphasizedVernacularWords{uk}. \\
old food 3p.\textsc{acc} BPx-go-\textsc{ben}-\textsc{bnfy}.\textsc{imp}.3p\\
\glt`Let them take food for the old people.'
\z

\ea%x494
\label{ex:3:x494}
\gll Uf-\textstyleEmphasizedVernacularWords{o}-k. \\
dance-\textsc{pa}-3s\\
\glt`He danced.'
\z

\ea%x495
\label{ex:3:x495}
\gll Uf-om-\textstyleEmphasizedVernacularWords{e}-k. \\
dance-\textsc{ben}-\textsc{bnfy}1.\textsc{pa}-3s\\
\glt`He danced for me/you.'
\z

\ea%x496
\label{ex:3:x496}
\gll Uf-om-\textstyleEmphasizedVernacularWords{a}-k. \\
dance-\textsc{ben}-\textsc{bnfy}2.\textsc{pa}-3s\\
\glt`He danced for him.'
\z

\subsubsection{Counterfactual}\label{sec:3:z:y:x}
%\hypertarget{RefHeading20341935131865}
{}
The counterfactual modality is the only modal distinction made in the verb morphology. It is an expression of the truth value of the statement: something could or would have happened, but did not, or something might be the case but for some reason is not. The counterfactual is marked by the suffix \nobreakdash-\textstyleStyleVernacularWordsItalic{ek} \REF{ex:3:x234} and is only used with the past tense suffix even if the verb refers to the present \REF{ex:3:x433} or future \REF{ex:3:x434} time\textstyleEmphasizedVernacularWords{\textmd{\textit{.}}} The counterfactual is used in both hypothetical and counterfactual conditional clauses (\sectref{sec:8.3.5}). 

\ea%x234
\label{ex:3:x234}
\gll Lawiliw akena waki-\textstyleEmphasizedVernacularWords{ek}-a-m. \\
nearly very fall-\textsc{cntf}-\textsc{pa}-1s \\
\glt`I very nearly fell.' 
\z

\ea%x433
\label{ex:3:x433}
\gll Yena aamun aakisa uruf-\textstyleEmphasizedVernacularWords{ek-a}-m=na kemel-\textstyleEmphasizedVernacularWords{ek-a}-m. \\
1s.\textsc{gen} yonger.brother now see-\textsc{cntf}-\textsc{pa}-1s=\textsc{tp} rejoice-\textsc{cntf}-\textsc{pa}-1s\\
\glt`If I saw my younger brother now, I would be happy.'
\z

\ea%x434
\label{ex:3:x434}
\gll Morauta fan ik-\textstyleEmphasizedVernacularWords{ek-a}-k=na uurika ikiw-ep maak-\textstyleEmphasizedVernacularWords{ek-a}-mik.\\
Morauta here be-\textsc{cntf}-\textsc{pa}-3s=\textsc{tp} tomorrow go-\textsc{ss}.\textsc{seq} tell-\textsc{cntf}-\textsc{pa}-1/3p\\
\glt`If Morauta were here, tomorrow we would go and tell him.'
\z

If there is a beneficiary suffix \textstyleStyleVernacularWordsItalic{-a} preceding the counterfactual, the vowel /e/ of the counterfactual suffix is deleted \REF{ex:3:x235}:

\ea%x235
\label{ex:3:x235}
\gll Maifa yia aaw-om-\textstyleEmphasizedVernacularWords{ak}-a-k=na{\dots} \\
paper 1p.\textsc{acc} get-\textsc{ben}-\textsc{bnfy}2.CNTF-\textsc{pa}-3s=\textsc{tp} \\
\glt`If he had gotten tickets for us{\dots}'
\z

\subsubsection{Mood}\label{sec:3:z:y:x}
%\hypertarget{RefHeading20361935131865}
{}
Mood in Mauwake is defined as a morphological category of the verb, relating to the pragmatic function of the sentence \citep[cf.][21]{Palmer1986}. Mauwake has a mixed tense-mood system, where the indicative present, past and future, and the imperative are in contrast. 

The mood distinctions only show in the finite verb. Same-subject medial verbs take the interpretation of their mood from the following finite verb, but different-subject medial verbs may be independent of the final verb as to their mood.

\paragraph{Indicative}\label{sec:3:a:z:y:x}
%\hypertarget{RefHeading20381935131865}
{}
The indicative is the neutral, morphologically unmarked mood. It is characterized by the tense distinctions between present, past and future, and the person/number distinctions of first, second, and third person in singular and plural. 

\ea%x693
\label{ex:3:x693}
\gll I me yia damol-a-mik. \\
1s.\textsc{unm} not 1s.\textsc{acc} harm-\textsc{pa}-1/3p\\
\glt`They didn't harm us.'
\z

\ea%x694
\label{ex:3:x694}
\gll Aria, iperowa opora wiar ook-i-yen. \\
alright, middle-aged talk 3.\textsc{dat} follow-Np-\textsc{fu}.1p\\
\glt`Alright, we'll follow the advice of the middle-aged men.'
\z

\paragraph{Imperative}\label{sec:3:a:z:y:x}
%\hypertarget{RefHeading20401935131865}
{}
The term imperative is used for ``mands'' \citep[745]{Lyons1977}\footnote{Lyons borrows the term from B.F. Skinner as a useful cover term, without subscribing to Skinner's behaviouristic position.} showing in the verbal morphology, regardless of person. In Mauwake the imperatives form a full paradigm (with the first person singular being replaced with the first person dual), and their syntactic behaviour is similar. So there is no valid reason to divide them into different categories such as imperatives, jussives and hortatives, just because semantically giving orders to oneself differs from giving orders to an addressee or to a third person.\footnote{For a discussion on this question, see \citet[109--111]{Palmer1986}.}

There are no tense distinctions in the imperative forms. The initial (or only) vowel in the second person imperatives is usually /e/, but in very few cases it is /a/.\footnote{The only verbs found with -\textstyleFootnoteBaseChar{\textit{a}} in the imperative are \textstyleFootnoteBaseChar{\textit{iw}}- `go', \textstyleFootnoteBaseChar{\textit{mik}}- `spear, hit', \textstyleFootnoteBaseChar{\textit{op}}- `hold' and \textstyleFootnoteBaseChar{\textit{pok}}- `sit'.}

\begin{table}
\begin{tabular}{ll}
\mytoprule
&person/number \\
\midrule 
-u & 1d\\
-e (-a) & 2s\\
-inok & 3s\\
-ikua & 1p\\
-eka (-aka) & 2p\\
-uk & 3p\\
\mybottomrule 
\end{tabular}
\caption{Imperative suffixes}
\label{tab:12}
\end{table}

\ea%x229
\label{ex:3:x229}
\gll Or-op mua nain uruf-\textstyleEmphasizedVernacularWords{e}. \\
descend-\textsc{ss}.\textsc{seq} man that1 see-\textsc{imp}.2s \\
\glt`Go down and see that man.'
\z

\ea%x1847
\label{ex:3:x1847}
\gll Ikoka amap-urup-eya op-\textstyleEmphasizedVernacularWords{ikua}. \\
later Bpx-ascend-2/3s.\textsc{ds} hold-\textsc{imp}.1p\\
\glt`Later when he comes up, let's hold/grab him.'
\z

\ea%x230
\label{ex:3:x230}
\gll Wi urup-ep mukuna nain umuk-\textstyleEmphasizedVernacularWords{uk}. \\
3p.\textsc{unm} ascend-\textsc{ss}.\textsc{seq} fire that1 extinguish-\textsc{imp}.3p\\
\glt`Let them go up and extinguish the fire.' 
\z

The imperative differs from the other moods in that it has a dual form in the first person but no singular: 

\ea%x446
\label{ex:3:x446}
\gll Aria, i owowa=ko or-\textstyleEmphasizedVernacularWords{u}. \\
alright, 1p.\textsc{unm} village=\textsc{nf} descend-\textsc{imp}.1d\\
\glt`Alright, let's (d.) go down to the village.'
\z

\ea%x1196
\label{ex:3:x1196}
\gll Yiena ikos akena iw-\textstyleEmphasizedVernacularWords{u}. \\
1p.\textsc{gen} two.together truly go-\textsc{imp}.1d\\
\glt`Let's just the two of us go together.'
\z

The initial (or only) vowel in the second person imperative forms is deleted after the beneficiary suffix \REF{ex:3:x432}-\REF{ex:3:x233}. 

\ea%x432
\label{ex:3:x432}
\gll Iwera ir-\textstyleEmphasizedVernacularWords{e.} \\
coconut ascend-\textsc{imp}.2s\\
\glt`Climb up the coconut palm (to get coconuts).'
\z

\ea%x431
\label{ex:3:x431}
\gll Iwera ir-om-\textstyleEmphasizedVernacularWords{e}. \\
coconut ascend-\textsc{ben}-\textsc{bnfy}1.\textsc{imp}.2s\\
\glt`Climb up the coconut palm for me.'
\z

\ea%x232
\label{ex:3:x232}
\gll Iwera ir-om-\textstyleEmphasizedVernacularWords{a.} \\
coconut ascend-\textsc{ben}-\textsc{bnfy}2.\textsc{imp}.2s \\
\glt`Climb the coconut for him.'
\z

\ea%x233
\label{ex:3:x233}
\gll Iwera \textstyleEmphasizedVernacularWords{yia} ir-om-\textstyleEmphasizedVernacularWords{a}ka. \\
coconut 1p.\textsc{acc} ascend-\textsc{ben}-\textsc{bnfy}2.\textsc{imp}.2p \\
\glt`Climb (plural) the coconut for us.'
\z

The semantics of the imperative and the functional aspects of commands are discussed in \sectref{sec:7.3}. On the use of subject pronouns with imperatives, see \sectref{sec:3.5.11}. The imperative forms of the verbs are also used in desiderative (\sectref{sec:8.3.2.1.3}) and conative (\sectref{sec:8.3.2.1.5}) constructions. 

\subsubsection{Tense and person/number in final verbs}\label{sec:3:z:y:x}
%\hypertarget{RefHeading20421935131865}
{}
Tense is a ``{grammaticalized expression of location in time}'' \citep[9]{Comrie1985}. Mauwake has a straightforward three-tense system in the finite verbs marking past, present and future time reference. The tense system is simple compared with most other Papuan languages, many of which have more than three genuine tense distinctions and/or interaction between tense and status\footnote{`Status' here refers to the distinction between realis and irrealis.} resulting in several ``tenses'' ({Foley 1986}:158--163). Of the most closely related languages well studied so far, Usan has five tenses \citep[98]{Reesink1987} out of which one, uncertain future/subjunctive, is semantically related to irrealis. Maia has a complete status system instead of a tense system, and temporal relations are inferred from the realis or irrealis status and the aspects \citep[55]{Hardin2002}. According to {Foley}, {``most Papuan languages are tense-dominated [rather than status-dominated]''} \citeyear[162]{Foley1986}. In Mauwake the status hardly plays any role at all.

Portmanteau morphemes of the tense and person/number markers are very common in Papuan languages, but having the two distinct from each other is not uncommon either \citep[137]{Foley1986}. The tense and person suffixes are separate morphemes in Mauwake, but have an interesting interplay with each other.

The speech event is taken as the reference point. The tense suffixes in themselves only distinguish between two tenses, past and non-\textsc{pa}st, and the further distinction between present and future is made by the person/number suffixes. The person/number suffixes, on the other hand, are the same in past and present tense except for the first and third person singular forms. 

\ea%x1029
\label{ex:3:x1029}
\gll Unan \textstyleEmphasizedVernacularWords{aakun-e-mik}, aakisa \textstyleEmphasizedVernacularWords{aakun-i-mik} ne uurika nainiw \textstyleEmphasizedVernacularWords{aakun-i-yen}.\\
yesterday talk-\textsc{pa}-1/3p now/today talk-Np-\textsc{pr}.1/3p \textsc{add} tomorrow again talk-Np-\textsc{fu}.1p\\
\glt`Yesterday we talked, now/today we talk and tomorrow we'll talk again.'
\z

The non-\textsc{pa}st marker in the second person plural future form is \textstyleEmphasizedVernacularWords{\nobreakdash-}\textstyleStyleVernacularWordsItalic{o} instead of \textstyleEmphasizedVernacularWords{\nobreakdash-}\textstyleStyleVernacularWordsItalic{i} possibly because of assimilation to the labial consonant /w/ in the person/number suffix. 

The verb conjugation classes determining the past tense suffix vowels are discussed in the section on morphophonology (\sectref{sec:2.3.3.3}). The beneficiary and the counterfactual suffixes influence the past tense suffix in the following way. After the counterfactual the past tense suffix is always \textstyleEmphasizedVernacularWords{\nobreakdash-}\textstyleStyleVernacularWordsItalic{a}. When the beneficiary suffix is present, the vowel of the past tense suffix is assimilated to it.

The following table presents the full paradigms for the tense and person/number suffixes.


\begin{table}
\resizebox{\textwidth}{!}{
\begin{tabular}{lllllll}
\mytoprule
     &  Non-past &  Present \& person  &  Non-past &  Future \& person & Past &  Person\\
\midrule
  1s & -i & -yem &  -i & -nen  & -a/E & -m\\
  2s & -i & -n &  -i & -nan &  -a/E & -n\\
  3s & -i & -ya  & -i & -non  & -a/E & -k\\
  1p & -i & -mik  & -i & -yen  & -a/E & -mik\\
  2p & -i & -man  & -o & -wen  & -a/E & -man\\
  3p & -i & -mik  & -i & -kuan  & -a/E & -mik\\
\mybottomrule  
\end{tabular}
}
\caption{Tense and person/number suffixes}
\label{tab:13}
\end{table}

The person/number marking in the verb distinguishes three persons in both singular and plural. There is no dual number, nor is there inclusive-exclusive distinction in the first person plural form. The plural is marked only for humans, spirits and important animals. The singular form is used for less important and small animals as well as all inanimates:

\ea%x236
\label{ex:3:x236}
\gll Waa muuka arow ekap-o-\textstyleEmphasizedVernacularWords{k}. \\
pig boy three come-\textsc{pa}-3s \\
\glt`Three piglets came.'
\z

Besides their primary meaning, the present and future tenses also have secondary meanings. The present tense form of the first or third person plural is used for generic or time-neutral statements \REF{ex:3:x1034}. For the habitual aspect in the present, the simple present tense \REF{ex:3:x1035} is an alternative to the full habitual aspect form.\footnote{Continuous aspect form is required for the past habitual (\sectref{sec:3.8.5.1.1.2}).} 

\ea%x1034
\label{ex:3:x1034}
\gll Ifa yia keraw-i-ya nain miira \textstyleEmphasizedVernacularWords{saawirin-i-mik}. \\
snake 1p.\textsc{acc} bite-Np-\textsc{pr}.3s that1 face become.round-Np-\textsc{pr}.1/3p\\
\glt `When a snake bites us, we become dizzy.'
\z

\ea%x1035
\label{ex:3:x1035}
\gll Nos=ke anane urema efar \textstyleEmphasizedVernacularWords{ikum-ar-i-n}. \\
2s.\textsc{fc}=\textsc{cf} always bandicoot 1s.\textsc{dat} illicitly-\textsc{inch}-Np-\textsc{pr}.2s\\
\glt`You always steal bandicoots from me.'
\z

The future tense in any person form is used for habitual or generic conditionals (\sectref{sec:8.3.5}). The example \REF{ex:3:} refers to a traditional custom and is generic, even if the first person form of the verb is used, and the first person pronoun as well.

\ea%x1640
\label{ex:3:x1640}
\gll Waaya \textstyleEmphasizedVernacularWords{ika-i-non}, waaya \textstyleEmphasizedVernacularWords{uup-i-nan}, naap. \\
pig be-Np-\textsc{fu}.3s pig cook-Np-\textsc{fu}.2s thus\\
\glt`If there is a pig, you will cook it - it is like that.'
\z

\ea%x1641
\label{ex:3:x1641}
\gll Ikoka yo \textstyleEmphasizedVernacularWords{um-i-nen}, muuka nain nainiw wiena \\
later 1s.\textsc{unm} die-Np-\textsc{fu}.1s son that1 again 3p.\textsc{gen} \textstyleEmphasizedVernacularWords{aaw-i-kuan}. take-Np-\textsc{fu}.3p\\
\glt`Later if I die (without paying the bride price) they will take the son back.'
\z

The second person singular form of the future tense has two other usages as well. It can be used when referring to generic or habitual situations, especially in process descriptions which can also be understood as instructions. For these, the first person plural form of the present tense is much more common, but often the two alternate. The following example describes work involved in harvesting taro roots (and the addressee that the story was told to, had no garden, so the speaker did not refer to her personally). 

\ea%x1038
\label{ex:3:x1038}
\gll Perek-ami en-ow(a) gelemuta \textstyleEmphasizedVernacularWords{on-i-nan}. \\
harvest-\textsc{ss}.\textsc{sim} eat-\textsc{nmz} little make-Np-\textsc{fu}.2s\\
\glt`When you harvest it you make a little feast.'
\z

It is also used for a command, or a statement of obligation:

\ea%x1039
\label{ex:3:x1039}
\gll Ikoka kuisow kuuma kuisow \textstyleEmphasizedVernacularWords{yi-i-nan}. \\
later one stick one give.me-Np-\textsc{fu}.2s\\
\glt`Very soon you have to give me 10 kina.' Or: `Give me 10 kina very soon.'
\z

\subsubsection{Medial verb marking}\label{sec:3:z:y:x}
%\hypertarget{RefHeading20441935131865}
{}
The distinction between medial and final verbs is common in Papuan languages \citep[11]{Foley1986}. Especially in the \textstyleAcronymallcaps{TNG} languages the medial verbs are ``{very common, universal over a wide area [and the] systems often highly complex}'' (\citealt[63]{Wurm1982}, also \citealt{Roberts1997}). The medial verbs typically lack the full tense and person/number marking of the finite verbs. Instead, they usually indicate whether the subject is the same as the subject of the following verb, and/or whether the action of the verb is simultaneous or sequential with the action of the following verb.\footnote{This is called the ``switch-reference system''. {The question whether} the system really tracks the topic (pragmatic subject) or the syntactic subject is discussed further in 8.2.3.} As for person reference, the verbs in the simplest systems only show whether the two subjects are the same or different, but in the most elaborate systems the subjects of both the medial verb and that of the following verb are shown in the medial verb, which is thus even more specific than the finite verb.\footnote{Usan makes a distinction between neutral and future medial verbs, and in both of these there is a division between same-subject and different-subject forms, but not between sequentiality and simultaneity {(Reesink 1987}:87--92). Maia only uses medial verbs when a clause has the same subject as the following clause; a distinction is made between simultaneous and sequential actions. When the following clause has a different subject, finite forms plus the contrast clitic \textstyleFootnoteBaseChar{\nobreakdash-(}\textstyleFootnoteBaseChar{\textit{d)i}} is used \citep[87]{Hardin2002}. Amele makes the basic distinction between the same-subject and different-subject medial verbs, and has simultaneous and sequential forms in both. But it also has different-subject simultaneous irrealis forms \citep[275]{Roberts1987}. Particularly the East New Guinea Highlands languages are known for marking the anticipatory subject in their medial verbs. See \citet[40--41]{Franklin1983} for a succinct list of switch-reference characteristics in Papuan languages, and \citet{Roberts1997} for a more comprehensive overview} 

In Mauwake the medial verb system is relatively simple. The suffixes indicate whether the subject of the medial verb stays the same in the following verb as well, and in the ``same subject following'' (\textstyleAcronymallcaps{SS}) verbs there is a further distinction between simultaneous and sequential action. The ``different subject following'' (\textstyleAcronymallcaps{DS}) verbs indicate sequential action; for simultaneous action one needs to use the continuous (\sectref{sec:3.8.5.1.1.2}) or stative aspect (\sectref{sec:3.8.5.1.1.3}). The \textstyleAcronymallcaps{DS} verbs also have some person marking but not as detailed as the finite verbs have.

The two sections below give a general outline of the person reference in medial clauses, but it is discussed in more detail in \sectref{sec:8.2.3}.

Typically, medial verbs have much fewer inflectional possibilities than finite verbs \citep[11]{Foley1986}. This is the case in Mauwake too: mood or tense and full person/number marking cannot be suffixed to the medial verbs. Derivational suffixes, on the other hand, can freely occur on the medial verbs. In Tail-Head linkage a new sentence begins with a medial verb copy of the finite verb that ended the previous sentence (\sectref{sec:8.2.3.5}). Often the derivational morphology of the two verbs is the same, and sometimes the medial verb has less derivation than the final verb; very rarely it has even \textstyleEmphasizedWords{\textsc{more}} \REF{ex:3:x237}:

\ea%x237
\label{ex:3:x237}
\gll Ikiwosa wiar pepekim-ep kaik-a-m. Kaik-\textstyleEmphasizedVernacularWords{om}-\textbf{a}p{\dots} \\
head 3.\textsc{dat} measure-\textsc{ss}.\textsc{seq} tie-\textsc{pa}-1s tie-\textsc{ben}-\textsc{bnfy}2.\textsc{ss}.\textsc{seq}\\
\glt`I measured her head and tied it (=headdress). I tied it for her and {\dots}'
\z

\paragraph{Same-subject marking}\label{sec:3:a:z:y:x}
%\hypertarget{RefHeading20461935131865}
{}
When the subject of the medial clause is the same as that of the following clause, the verb itself does not give any indication of the person and number of the subject, only that the same subject continues in the next clause.\footnote{For exceptions to this, see \sectref{sec:8.2.3} where the functional aspects of switch reference are discussed.} If the actions are sequential, i.e. the action indicated by the verb in the medial clause precedes that of the following clause, the suffix is \nobreakdash-\textstyleStyleVernacularWordsItalic{ap} or \nobreakdash-\textstyleStyleVernacularWordsItalic{ep} \REF{ex:3:x238} depending on the conjugation class (\sectref{sec:2.3.3.3}).\footnote{For the second conjugation class verb \textstyleFootnoteBaseChar{\textit{or}}\textit{-} `descend' the suffix is -\textstyleFootnoteBaseChar{\textit{op}}\textstyleEmphasizedVernacularWords{.}} 

\ea%x238
\label{ex:3:x238}
\gll Owowa ek-\textstyleEmphasizedVernacularWords{ap}, wailal-\textstyleEmphasizedVernacularWords{ep} akia ik-e-k. \\
village go-\textsc{ss}.\textsc{seq} be.hungry-\textsc{ss}.\textsc{seq} banana roast-\textsc{pa}-3s \\
\glt`He went to the village, was hungry and roasted bananas.' 
\z

If the verb has the beneficiary suffix \textstyleStyleVernacularWordsItalic{-a} or \textstyleStyleVernacularWordsItalic{-e} (\sectref{sec:3.8.3.1}), the vowel of the medial verb suffix gets assimilated to it \REF{ex:3:x1930}, \REF{ex:3:x1929}.

\ea%x1930
\label{ex:3:x1930}
\gll {\dots}eka=pa merena yasuw-om-\textstyleEmphasizedVernacularWords{e}p... (cf. yasuw-\textstyleEmphasizedVernacularWords{a}p) \\
water=\textsc{loc} foot wash-\textsc{ben}-\textsc{bnfy}.1-\textsc{ss}.\textsc{seq}\\
\glt`{\dots} she washed my feet in water (and) {\dots}'
\z

\ea%x1929
\label{ex:3:x1929}
\gll ...waaya nain uup-om-\textstyleEmphasizedVernacularWords{a}p samapora=pa wu-ap maak-e-mik... (cf. uup-\textstyleEmphasizedVernacularWords{e}p)\\
pig that1 cook-\textsc{ben}-\textsc{bnfy}2.\textsc{ss}.\textsc{seq} floor=\textsc{loc} put-\textsc{ss}.\textsc{seq} tell-\textsc{pa}-1/3p\\
\glt`{\dots} they cooked the pig for him, put it on the floor and told him, {\dots}'
\z

When the medial clause subject is the same as the subject in the following clause but the two actions are simultaneous, or at least overlapping, the suffix is \nobreakdash-\textstyleStyleVernacularWordsItalic{ami} or \nobreakdash-\textstyleStyleVernacularWordsItalic{emi} (or \nobreakdash-\textstyleStyleVernacularWordsItalic{omi}) according to the conjugation class of the verb. Even if the action of the medial verb may often be \textstyleEmphasizedWords{\textsc{interpreted}} as continuous \REF{ex:3:x239}, the suffix in itself only indicates simultaneity \REF{ex:3:x240}. Continuous aspect marking may be needed for clarity when continuity is in focus \REF{ex:3:x241}.

\ea%x239
\label{ex:3:x239}
\gll Wi sawur ir-\textstyleEmphasizedVernacularWords{ami} fan yiar pok-a-mik. \\
3p.\textsc{unm} spirit go-\textsc{ss}.\textsc{sim} here 1p.\textsc{dat} sit.down-\textsc{pa}-1/3p \\
\glt`As the spirits were going they sat down here with us.' 
\z

\ea%x240
\label{ex:3:x240}
\gll {\dots}ekap-\textstyleEmphasizedVernacularWords{emi} koora=pa yia wua-i-mik. \\
come-\textsc{ss}.\textsc{sim} house=\textsc{loc} 1p.\textsc{acc} put-Np-\textsc{pr}.1/3p \\
\glt`{\dots}coming (=upon arrival) they put us in the house.' 
\z

\ea%x241
\label{ex:3:x241}
\gll \textstyleEmphasizedVernacularWords{Soomar-em-ik}\textstyleEmphasizedVernacularWords{-ok} ifara oko uruf-a-k. \\
walk-\textsc{ss}.\textsc{sim}-be-SS vine other see-\textsc{pa}-3s \\
\glt`He was walking and saw another vine. '
\z

The verb \textstyleStyleVernacularWordsItalic{ik}- `be' is different from other verbs in that there is no differentiation between the simultaneous and sequential forms in the same-subject medial verb: in example \REF{ex:3:x242} the actions are simultaneous, in \REF{ex:3:x243} sequential. Also, the verb does not take either one of the normal same-subject suffixes.

\ea%x242
\label{ex:3:x242}
\gll Owowa=pa neeke \textstyleEmphasizedVernacularWords{ik-ok} mua maak-ek{\dots} \\
village=\textsc{loc} there.\textsc{cf} be-SS man tell-\textsc{pa}-3s \\
\glt`While they were there in the village she told her husband, {\dots}'
\z

\ea%x243
\label{ex:3:x243}
\gll No kaaneke \textstyleEmphasizedVernacularWords{ik-ok} kerer-e-n? \\
2s.\textsc{unm} where.\textsc{cf} be-SS appear-\textsc{pa}-2s \\
\glt`Where have you been and now come?' 
\z

\paragraph{Different-subject marking}\label{sec:3:a:z:y:x}
%\hypertarget{RefHeading20481935131865}
{}
If the subject of the medial clause is different from that of the following clause, the suffix of the \textstyleAcronymallcaps{DS} verb reflects this. There are some person/number distinctions in these suffixes, even though not as many as in the finite verbs. The first person singular and plural are distinguished from all the other forms; in the other persons the distinction is based on the number: second and third person singular share the same suffix, and second and third person plural likewise.\footnote{There is great variation in this area in Papuan languages. Some only have one form to indicate that the subject changes, others have partial or full differentiation according to the person, some even show the subject of the following clause in the medial verb.} 


\begin{table}\begin{tabular}{lll}
\mytoprule
 & Singular & Plural\\
\midrule 
1 & \multicolumn{2}{c}{-Vmkun}\\
2 & \multirow{2}{*}{-eya} & \multirow{2}{*}{-iwkin}\\
3 & & \\ 
\mybottomrule 
\end{tabular}
\caption{Suffixes marking a different subject in the following clause}
\label{tab:14}
\end{table}

\ea%x244
\label{ex:3:x244}
\gll Imen-ap maak-\textstyleEmphasizedVernacularWords{iwkin} o miim-o-k. \\
find-\textsc{ss}.\textsc{seq} tell-2/3p.\textsc{ds} 3s.\textsc{unm} precede-\textsc{pa}-3s \\
\glt`They found him and told him, and he went ahead of them.' 
\z

\ea%x245
\label{ex:3:x245}
\gll Mik-\textstyleEmphasizedVernacularWords{amkun} me um-o-k, wiowa onaiya ikiw-em-ik-\textstyleEmphasizedVernacularWords{eya} Olas=ke war-ek.\\
spear-1s/p.\textsc{ds} not die-\textsc{pa}-3s spear with go-\textsc{ss}.\textsc{sim}-be-2/3s.\textsc{ds} Olas=\textsc{cf} shoot-\textsc{pa}-3s\\
\glt`When I speared it, it didn't die, (but) as it was going with the spear Olas shot it.'
\z

The suffix is -\textstyleStyleVernacularWordsItalic{aya} instead of \nobreakdash-\textstyleStyleVernacularWordsItalic{eya} in a few short conjugation class 1 verbs (\sectref{sec:3.8.4.1}) \REF{ex:3:x493}\footnote{The vowel \textstyleFootnoteBaseChar{\textit{-a}} is somewhat more common in the Muaka dialect group where the 2/3s.\textsc{ds} marker is -\textit{era} instead of \nobreakdash-\textit{eya.}} and in those benefactive verbs where the first vowel of the suffix is assimilated to the preceding vowel of the beneficiary suffix \REF{ex:3:x695}. 

\ea%x493
\label{ex:3:x493}
\gll Iw-\textstyleEmphasizedVernacularWords{aya} nan miira saawirin-e-k. \\
enter-2/3s.\textsc{ds} there face become.round-\textsc{pa}-3s\\
\glt`As [the poison] entered [his liver], he became dizzy.'
\z

\ea%x695
\label{ex:3:x695}
\gll Aaya=ko yia aaw-om-\textstyleEmphasizedVernacularWords{aya} enim-i-yan. \\
sugarcane=\textsc{nf} 1p.\textsc{acc} get-\textsc{ben}-\textsc{bnfy}2.2/3s.\textsc{ds} eat-Np-\textsc{fu}.1p\\
\glt`Get us sugarcane and we'll eat it.'
\z

The different-subject marking \nobreakdash-\textstyleStyleVernacularWordsItalic{eya} is also used with some non-verbs. This seems to be uncommon in PNG languages: in \citeauthor{Roberts1997}' (1997:137) survey the very few examples where the switch-reference marking was on non-verbs these were pro-clausal substitutes like a demonstrative or vocative. In Mauwake the \textstyleAcronymallcaps{DS} suffix can be added to nouns \REF{ex:3:} or adjectives \REF{ex:3:}, or the negative adverbs \textstyleStyleVernacularWordsItalic{weetak} and \textstyleStyleVernacularWordsItalic{marew} \REF{ex:3:} functioning as predicates in verbless clauses. When it is added to words ending in -\textstyleStyleVernacularWordsItalic{a}, the first vowel in the suffix gets assimilated to this vowel \REF{ex:3:}.

\ea%x250
\label{ex:3:x250}
\gll Enakiwa-\textstyleEmphasizedVernacularWords{ya} me aaw-e-m. \\
half-2/3s.\textsc{ds} not take-\textsc{pa}-1s \\
\glt`There was (only) half (left), so I didn't take any/it.'
\z

\ea%x251
\label{ex:3:x251}
\gll Mauwow maneka-\textstyleEmphasizedVernacularWords{ya}=na yia maak-i-non. \\
work big-2/3s.\textsc{ds}=\textsc{tp} 1p.\textsc{acc} tell-Np-\textsc{fu}.3s \\
\glt`If the work is big, she will tell us.'
\z

\ea%x252
\label{ex:3:x252}
\gll Soomia marew-\textstyleEmphasizedVernacularWords{eya} amap-ep-om-a-m. \\
spoon none-2/3s.\textsc{ds} BPx-come-\textsc{ben}-\textsc{bnfy}2.\textsc{pa}-1s \\
\glt`She has/had no spoons (lit: there are/were no spoons) so I brought them to her.'
\z

When the different subject marking \nobreakdash-\textstyleStyleVernacularWordsItalic{eya} is added to the adverb \textstyleStyleVernacularWordsItalic{naap} `thus', the outcome is a consecutive connective `therefore, so' (\sectref{sec:3.11.2}).

\paragraph{Tense and medial verbs}\label{sec:3:a:z:y:x}
%\hypertarget{RefHeading20501935131865}
{}
The medial verbs have no tense marking, so the tense in a medial clause is interpreted in relation to that of the next finite clause. When the finite clause is in the past tense, both a simultaneous \REF{ex:3:x1025} and a sequential medial clause \REF{ex:3:x1023} are also understood to be in the past tense.

\ea%x1025
\label{ex:3:x1025}
\gll Iwera uruk-am-ika-iwkin wi \textstyleEmphasizedVernacularWords{ikiw-emi} \textstyleEmphasizedVernacularWords{aaw-em-ik-e-mik}.\\
coconut drop-\textsc{ss}.\textsc{sim}-be-2/3p.\textsc{ds} 3p.\textsc{unm} go-\textsc{ss}.\textsc{sim} take-\textsc{ss}.\textsc{sim}-be-\textsc{pa}-1/3p\\
\glt`They\textsubscript{i} kept dropping coconuts, and they\textsubscript{j} went and got them.'
\z

\ea%x1023
\label{ex:3:x1023}
\gll Owowa \textstyleEmphasizedVernacularWords{or-op,} \textstyleEmphasizedVernacularWords{wailal-ep}, akia \textstyleEmphasizedVernacularWords{ik-e-k}. \\
village descend-\textsc{ss}.\textsc{seq} be.hungry-\textsc{ss}.\textsc{seq} banana roast-\textsc{pa}-3s\\
\glt`He came down to the village, was hungry and roasted bananas.'
\z

Since a sequential verb indicates that the action takes place before another action, a sequential medial clause preceding a present tense final clause has to be interpreted to be in the past tense, whereas a simultaneous clause is interpreted to be in the present tense like the final verb.

\ea%x1024
\label{ex:3:x1024}
\gll Iperuma nain=ke mua \textstyleEmphasizedVernacularWords{puuk-ap} owora \textstyleEmphasizedVernacularWords{en-emi} afura \textstyleEmphasizedVernacularWords{buan-em-ika-i-ya}.\\
eel that1=\textsc{cf} man become-\textsc{ss}.\textsc{seq} betelnut eat-\textsc{ss}.\textsc{sim} lime.container knock-\textsc{ss}.\textsc{sim}-be-Np-\textsc{pr}.3s\\
\glt`The eel has become man, and is eating betelnut and knocking the lime container.'
\z

Both a sequential and a simultaneous medial clause preceding a future final clause are also understood as future clauses. The action in a sequential medial clause takes place before that in the final clause, but it is still in the future \REF{ex:3:x1026}. The action in a simultaneous clause is partly or fully overlapping with that in the final clause \REF{ex:3:x1027}.

\ea%x1026
\label{ex:3:x1026}
\gll Is=ke maa uup-emkun wi \textstyleEmphasizedVernacularWords{ekap-ep} \textstyleEmphasizedVernacularWords{enim-i-kuan}. \\
1p.\textsc{fc}=\textsc{cf} food cook-1s/p.\textsc{ds} 3p.\textsc{unm} come-\textsc{ss}.\textsc{seq} eat-Np-\textsc{fu}.3p\\
\glt`We'll cook the food and they'll come and eat it.' Or: `When we have cooked the food they will come and eat it.'
\z

\ea%x1027
\label{ex:3:x1027}
\gll Wi \textstyleEmphasizedVernacularWords{ir-ami} nia \textstyleEmphasizedVernacularWords{aaw-emi} efa \textstyleEmphasizedVernacularWords{ifakim-i-kuan}. \\
3p.\textsc{unm} come-\textsc{ss}.\textsc{sim} 2p.\textsc{acc} take-\textsc{ss}.\textsc{sim} 1s.\textsc{acc} kill-Np-\textsc{fu}.3p\\
\glt`They will come and take you and kill me.'
\z

The medial verb form cannot be used in the following example, because the first verb refers to time preceding the speech event and the second verb to time following it. Final verbs with different tenses have to be used, and in this case it is most natural to place the past tense verb in a relative clause: 

\ea%x1030
\label{ex:3:x1030}
\gll Mukuna kerer-e-k nain kamenap umuk-i-yan? \\
fire appear-\textsc{pa}-3s that1 how extinguish-Np-\textsc{fu}.1p\\
\glt`How shall we extinguish the fire that started?'
\z

The medial verbs acquire more absolute-relative tense character of ``past in the past'' \citep[65]{Comrie1985} in those cases where sequential medial clauses are either right-dislocated and placed after the final clause \REF{ex:3:x1031} or placed inside another medial clause \REF{ex:3:x1032}, or when there is a separate time expression referring to earlier time than that indicated by the final verb \REF{ex:3:x1033}. 

\ea%x1031
\label{ex:3:x1031}
\gll Rubaruba nain=ke ona emeria nain aaw-ep p-ikiw-o-k, \textstyleEmphasizedVernacularWords{iw-iwkin}.\\
Rubaruba that1=\textsc{cf} 3s.\textsc{gen} woman that1 take-\textsc{ss}.\textsc{seq} BPx-go-\textsc{pa}-3s give.him-2/3p.\textsc{ds}\\
\glt`That Rubaruba took his wife and took her (away), when they had given her to him.'
\z

\ea%x1032
\label{ex:3:x1032}
\gll Um-eya merena ere-erup [\textstyleEmphasizedVernacularWords{ifara} \textstyleEmphasizedVernacularWords{aaw-ep}] kaik-ap nabena suuw-ap akua aaw-ep or-o-m.\\
die-2/3s.\textsc{ds} leg \textsc{rdp}-two vine get-\textsc{ss}.\textsc{seq} tie-\textsc{ss}.\textsc{seq} carrying.pole push-\textsc{ss}.\textsc{seq} shoulder take-\textsc{ss}.\textsc{seq} descend-\textsc{pa}-1s\\
\glt `It died, and I tied its legs in pairs with a vine that I had gotten, and pushed it to the carrying pole and carried it down on my shoulder.'
\z

\ea%x1033
\label{ex:3:x1033}
\gll \textstyleEmphasizedVernacularWords{Iiriw} inasin mua nain=ke naap wia \textstyleEmphasizedVernacularWords{maak-eya} wi naap on-a-mik.\\
earlier spirit man that=\textsc{cf} thus 3p.\textsc{acc} tell-2/3p.\textsc{ds} 3p.\textsc{unm} thus do-\textsc{pa}-1/3p\\
\glt `The spirit man had earlier told them like that and they did so.'
\z

\subsection{Verb classes} \label{sec:3:y:x}
%\hypertarget{RefHeading20521935131865}
{}
Verbs can be divided into classes on the basis of various criteria. Conjugation classes based on morphological/inflectional criteria are usually arbitrary and unrelated to other parts of the grammar \citep[191]{Anderson1985b}. They are only touched on briefly in the next subsection. Transitivity as a basis of verb classes is discussed in \sectref{sec:3.8.4.2}, and valence-changing operations in \sectref{sec:3.8.4.3}. Verb classes based on semantic features are described in \sectref{sec:3.8.4.4}.

\subsubsection{Conjugation classes}\label{sec:3:z:y:x}
%\hypertarget{RefHeading20541935131865}
{}
In the Mauwake lexicon the verbs are divided into classes 1 and 2 depending on whether they have /a/ or /e\~{o}/ as the past tense suffix. There are morphophonological rules for deriving the past tense marking for most verbs (see \sectref{sec:2.3.3.3}), but since some of the rules are rather complicated, and because they do not cover a number of cases like \REF{ex:3:x253} and \REF{ex:3:x254} below, the division into two separate classes is maintained.

\ea%x253
\label{ex:3:x253}
\gll miim-\textstyleEmphasizedVernacularWords{a}-k \\
hear-\textsc{pa}-3s \\
\glt`he heard'
\z

\ea%x254
\label{ex:3:x254}
\gll miim-\textstyleEmphasizedVernacularWords{o}-k \\
precede-\textsc{pa}-3s \\
\glt`he went ahead'
\z

In Class 1, transitive verbs outnumber intransitive verbs over four times, but Class 2 is divided almost equally between transitive and intransitive verbs.\footnote{For this count, the verbs formed with the verbalizer \textstyleFootnoteBaseChar{\textit{--ar}} and the causative \textstyleFootnoteBaseChar{\textit{--ow}} were deleted from the total of 857 verbs, since both these suffixes influence the choice of the past tense vowel.}

\ea%x255
\label{ex:3:x255}
\gll puuk-\textstyleEmphasizedVernacularWords{a}-k vs. puk-\textstyleEmphasizedVernacularWords{o}-k \\
cut-\textsc{pa}-3s burst-\textsc{pa}-3s \\
\glt`he cut (it)' `it burst' 
\z

\ea%x256
\label{ex:3:x256}
\gll teek-\textstyleEmphasizedVernacularWords{a}-k vs. ten-\textstyleEmphasizedVernacularWords{e}-k \\
pluck-\textsc{pa}-3s collapse-\textsc{pa}-3s\\
\glt`he plucked (it)' `it collapsed' (also: `it broke away')
\z

\subsubsection{Verb classes based on transitivity}\label{sec:3:z:y:x}
%\hypertarget{RefHeading20561935131865}
{}
With the term \textstyleEmphasizedWords{\textsc{transitivity}} of a verb I refer to its \textstyleEmphasizedWords{\textsc{syntactic}} transitivity, i.e. ``{the number of overt morpho-syntactically coded arguments it takes}'' \citep[147]{VanValinEtAl1997}. 

Intransitive verbs in Mauwake only require a subject, whereas transitive verbs require a direct object as well. This definition differs slightly from that of \citet[397]{Crystal1997}, who defines as transitive verbs those that \textstyleEmphasizedWords{\textsc{can}} take a direct object, and as intransitive those that \textstyleEmphasizedWords{\textsc{cannot}} (emphasis mine). Crystal's definition works for Mauwake when considering prototypical patient/undergoer objects, but it fails in the cases where the syntactic object manifests other roles not required by the semantic structure of the verb.\footnote{The syntactic transitivity of a verb can differ from both its semantic and macrorole transitivity \citep{VanValinEtAl1997}.} 

In some languages verb roots can be neutral as to transitivity \citep[53]{Kittila2002}, but in Mauwake each verb has a basic transitivity value. Most verbs are either intransitive (\sectref{sec:3.8.4.2.1}) or transitive (\sectref{sec:3.8.4.2.2}). There are only a few ambitransitives (\sectref{sec:3.8.4.2.3}). Mauwake does not have a regular class of ditransitive verbs that would require two objects. Instead, some verbs that are transitive easily allow a second object. And in the small class of the object cross-referencing verbs (\sectref{sec:3.8.4.2.4}), in which the pronominal object is in the verb root, two of the verbs require a second object as well. 

The basic transitivity of a verb can be changed with valence-changing strategies (\sectref{sec:3.8.4.3}). Causative (\sectref{sec:3.8.2.3.1}) and benefactive morphology (\sectref{sec:3.8.2.3.3}) as well as possessor raising (\sectref{sec:5.3.2.3}) are processes that increase the number of syntactic objects in a clause.

\paragraph{Intransitive verbs}\label{sec:3:a:z:y:x}
%\hypertarget{RefHeading20581935131865}
{}
In Mauwake the class of basic, or ``ordinary'', intransitive verbs consists of a semantically very diverse group including involuntary processes \REF{ex:3:x266}, many motion verbs \REF{ex:3:x267}, and some bodily function verbs \REF{ex:3:x269}.

\ea%x266
\label{ex:3:x266}
\gll Fikera \textstyleEmphasizedVernacularWords{aw-o-k}. \\
kunai.grass burn-\textsc{pa}-3s \\
\glt`The kunai grass burned.'
\z

\ea%x267
\label{ex:3:x267}
\gll Kuuten ikos \textstyleEmphasizedVernacularWords{karu-e-mik}. \\
Kuuten with run-\textsc{pa}-1/3p \\
\glt`I ran with Kuuten.'
\z

\ea%x269
\label{ex:3:x269}
\gll Niir-emi \textstyleEmphasizedVernacularWords{pisi-e}\textstyleEmphasizedVernacularWords{-k}. \\
laugh-\textsc{ss}.\textsc{sim} fart-\textsc{pa}-3s \\
\glt`He laughed and farted.'
\z

Some experience verbs expressing physiological states are also regular intransitive verbs:

\ea%x1485
\label{ex:3:x1485}
\gll Maa enowa nopa-yiaw-ep \textstyleEmphasizedVernacularWords{wailal-ep} ma-e-mik, ``...''\\
food eat-\textsc{nmz} search-move.around-\textsc{ss}.\textsc{seq} get.hungry-\textsc{ss}.\textsc{seq} say-\textsc{pa}-1/3p\\
\glt`They searched around for food and got hungry and said, ``{\dots}'' '
\z

The verbs derived with the inchoative suffix \nobreakdash-\textstyleStyleVernacularWordsItalic{ar} (\sectref{sec:3.8.2.2.2}) are mostly intransitive, but a few of them are transitive \REF{ex:3:}. 

\ea%x271
\label{ex:3:x271}
\gll Uuw-ap uuw-ap \textstyleEmphasizedVernacularWords{lebum-ar-e-m}. \\
work-\textsc{ss}.\textsc{seq} work-\textsc{ss}.\textsc{seq} lazy-\textsc{inch}-\textsc{pa}-1s \\
\glt`I worked and worked and got tired.'
\z

\ea%x1486
\label{ex:3:x1486}
\gll Nan teeria \textstyleEmphasizedVernacularWords{manek-ar-e-k}, owowa pun manek-ar-e-k. \\
there family big-\textsc{inch}-\textsc{pa}-3s village also big-\textsc{inch}-\textsc{pa}-3s\\
\glt`The family grew big there, and the village grew big too.'
\z

\ea%x1836
\label{ex:3:x1836}
\gll Maa unowa oram me \textstyleEmphasizedVernacularWords{amis}\textstyleEmphasizedVernacularWords{-}\textstyleEmphasizedVernacularWords{ar}\textstyleEmphasizedVernacularWords{-}\textstyleEmphasizedVernacularWords{i}\textstyleEmphasizedVernacularWords{-}\textstyleEmphasizedVernacularWords{mik}, weetak. \\
thing many just not knowledge-\textsc{inch}-Np-\textsc{pr}.1/3p no\\
\glt`We don't just gain knowledge of many things (without learning them), no.'
\z

Climate expressions often use intransitive verbs. There is no separate class of verbs for climate expressions.\footnote{Climate expressions also use directional verbs (\textstyleFootnoteBaseChar{\textit{ipia oraiya} }`the rain descends'), inchoative verbs (\textstyleFootnoteBaseChar{\textit{kokomarek} }`it got dark') and transitive verbs (\textstyleFootnoteBaseChar{\textit{ama fookak } }`the sun split (tr.)').}

\ea%x270
\label{ex:3:x270}
\gll Aapereka \textbf{paran-em-ika-i-ya}. \\
cloud rumble-\textsc{ss}.\textsc{sim}-be-Np-\textsc{pr}.3s\\
\glt`It is thundering.'
\z

Intransitive clauses are discussed in 5.4.

\paragraph{Transitive verbs}\label{sec:3:a:z:y:x}
%\hypertarget{RefHeading20601935131865}
{}
Transitive verbs require a subject and an object. A [+human] object needs to be marked with an accusative pronoun (\sectref{sec:3.5.3}) regardless of the presence or absence of a separate object \textstyleAcronymallcaps{NP}. 

\ea%x294
\label{ex:3:x294}
\gll Yaapan wia ifakim-e-mik. \\
Japan 3p.\textsc{acc} kill-\textsc{pa}-1/3p \\
\glt`They killed the Japanese.'
\z

Besides the prototypical transitive verbs with an agent subject and a patient-of-change object \citep[96]{Givon1984} like \REF{ex:3:x295} and \REF{ex:3:x296}, also many verbs of perception \REF{ex:3:x297}, cognition \REF{ex:3:x298} and emotion \REF{ex:3:x299} are transitive.

\ea%x295
\label{ex:3:x295}
\gll Wiipa erup wia \textstyleEmphasizedVernacularWords{sesek-a-mik}. \\
girl two 3p.\textsc{acc} send-\textsc{pa}-1/3p \\
\glt`They sent the two girls.'
\z

\ea%x296
\label{ex:3:x296}
\gll Yo me efa \textstyleEmphasizedVernacularWords{enim-uk}. \\
1s.\textsc{unm} not 1.\textsc{acc} eat-\textsc{imp}.3p \\
\glt`Let them not eat me.'\footnote{This was said in a traditional story by a spirit that was able to change into a man or into an eel, which the people in the story were preparing to eat.}
\z

\ea%x297
\label{ex:3:x297}
\gll Nomokowa unowa aakisa wia \textstyleEmphasizedVernacularWords{uruf-i-n.} \\
2s/p.brother many now 3p.\textsc{acc} see-Np-\textsc{pr}.2s \\
\glt`Now you see many brothers of yours.
\z

\ea%x298
\label{ex:3:x298}
\gll Nefa \textstyleEmphasizedVernacularWords{amis-ar-ep} ma-i-yem. \\
2s.\textsc{acc} knowledge-\textsc{inch}-\textsc{ss}.\textsc{seq} say-Np-\textsc{pr}.1s \\
\glt`I am saying (this) because I know you.'
\z

\ea%x299
\label{ex:3:x299}
\gll Yena mua=ke efa \textstyleEmphasizedVernacularWords{kookal-ep} manin(a) uuw-owa efa asip-i-ya.\\
1s.\textsc{gen} man 1s.\textsc{acc} like-\textsc{ss}.\textsc{seq} garden work-\textsc{nmz} 1s.\textsc{acc} help-Np-\textsc{pr}.3s\\
\glt`My husband likes me and helps me in the garden.'
\z

If there is no other overt object available for a transitive verb, the maximally generic noun \textstyleStyleVernacularWordsItalic{maa} `thing'\footnote{The semantic area of \textstyleFootnoteBaseChar{\textit{maa}} is at least as wide that of its English equivalent `thing'. Because it is used so often with verbs denoting eating and preparing food, it has acquired a secondary meaning `food'.} is used as a dummy object. Compare the next two examples: in \REF{ex:3:} \textstyleStyleVernacularWordsItalic{maa} is added because of syntactic requirements, whereas in \REF{ex:3:} the lack of an overt object indicates a third person singular object.

\ea%x300
\label{ex:3:x300}
\gll (Yo) \textstyleEmphasizedVernacularWords{maa} uruf-i-yem. \\
I thing see-Np-\textsc{pr}.1s \\
\glt`I see.' (=I see something, or: I can see.) 
\z

\ea%x301
\label{ex:3:x301}
\gll (Yo) uruf-i-yem. \\
I see-Np-\textsc{pr}.1s \\
\glt`I see him/her/it.'
\z

\ea%x1825
\label{ex:3:x1825}
\gll Iir oko \textstyleEmphasizedVernacularWords{maa} enim-i-yem, iir oko \textstyleEmphasizedVernacularWords{maa} me enim-i-yem. \\
time other thing eat-Np-\textsc{pr}.1s time other thing not eat-Np-\textsc{pr}.1s\\
\glt`Sometimes I eat, sometimes I don't eat.'
\z

The language-specific characteristic of syntactic transitivity \citep[49--51]{Kittila2002} is illustrated by a number of verbs that are transitive in Mauwake but intransitive in English:

\begin{table}
\caption{Please provide a caption}
\label{} 
\begin{tabular}{ll}
\mytoprule
aner- &`aim (at), refer (to)'\\
ikum- &`wonder (about)'\\
kerew- &`be angry (at)'\\
\mybottomrule
\end{tabular}

\end{table}

\ea%x302
\label{ex:3:x302}
\gll Wi wia amukar-emi me nefa \textstyleEmphasizedVernacularWords{aner-a-m}. \\
3p.\textsc{unm} 3p.\textsc{acc} scold-\textsc{ss}.\textsc{sim} not 2s.\textsc{acc} refer.to-\textsc{pa}-1s \\
\glt`When I scolded them I didn't refer to you.'
\z

\ea%x303
\label{ex:3:x303}
\gll Nefa \textstyleEmphasizedVernacularWords{ikum-am-ika-iwkin} nan kerer-e-n. \\
2s.\textsc{acc} wonder.about-\textsc{ss}.\textsc{sim}-be-2/3p.\textsc{ds} there arrive-\textsc{pa}-2s\\
\glt`As they were wondering about you, you arrived there.'
\z

There are a a few verbs that require an undergoer object, but usually have recipient object as well. The verbs \textstyleStyleVernacularWordsItalic{ofakow}- `show, teach' and \textstyleStyleVernacularWordsItalic{maak}- `tell' are the most common of these.\footnote{The verb `send' is cross-linguistically typically ditransitive, but in Mauwake it requires the benefactive suffix in order to be able to take a second object.} 

\ea%x1838
\label{ex:3:x1838}
\gll Tunde urera Liisa ame=ke [\textstyleEmphasizedVernacularWords{epa}]\textsubscript{O} [\textstyleEmphasizedVernacularWords{yia}]\textsubscript{O} \textstyleEmphasizedVernacularWords{ofakowa-y}\textstyleEmphasizedVernacularWords{i}\textstyleEmphasizedVernacularWords{aw}\textstyleEmphasizedVernacularWords{-}\textstyleEmphasizedVernacularWords{e}\textstyleEmphasizedVernacularWords{-}\textstyleEmphasizedVernacularWords{mik}.\\
Tuesday afternoon Liisa ASSOC=\textsc{cf} place 1p.\textsc{acc} show-move.around-\textsc{pa}-1/3p\\
\glt`On Tuesday afternoon Liisa and the othes showed us around the place.'
\z

\ea%x943
\label{ex:3:x943}
\gll Nena panewowa pun [\textstyleEmphasizedVernacularWords{wadol} \textstyleEmphasizedVernacularWords{opora}]\textsubscript{O}  [\textstyleEmphasizedVernacularWords{yia}]\textsubscript{O} \textstyleEmphasizedVernacularWords{maak-i-n.} \\
2s.\textsc{gen} old also lie talk 1p.\textsc{acc} tell-Np-\textsc{pr}.2s\\
\glt`You yourself -- an old person too! -- tell us lies.'
\z

The verb \textstyleStyleVernacularWordsItalic{wu}- `put' requires both an undergoer object and a locative adverbial: 

\ea%x1837
\label{ex:3:x1837}
\gll [Sosora nain]\textsubscript{O} [pona-\textsc{pa}]\textsubscript{AdvP} wu-a-mik. \\
grass.skirt that1 riverbank put-\textsc{pa}-1/3p\\
\glt`They put those grass skirts on the riverbank.'
\z

The directional verbs (\sectref{sec:3.8.4.4.5}) could be treated as weakly transitive, in which case the goal \textstyleAcronymallcaps{NP}, which is never marked with the locative clitic -\textstyleStyleVernacularWordsxiiptItalic{pa}, could be a locative object. There are two main reasons against this analysis. When the goal of a directional verb is a personal pronoun, the dative case is used rather than the accusative:

\ea%x1870
\label{ex:3:x1870}
\gll {\dots}ona wiawi \textstyleEmphasizedVernacularWords{wiar} ikiw-o-k. \\
3s.\textsc{gen} 3s/p.father 3.\textsc{dat} go-\textsc{pa}-3s\\
\glt`{\dots}she went to her father.'
\z

Also, if the directional verbs were considered weakly transitive and the the goal a locative object, the locative adverb \textstyleStyleVernacularWordsxiiptItalic{nan} `there' in the following clauses would be treated as a locative adverb in \REF{ex:3:x1871} but as a locative object in \REF{ex:3:x1872}:

\ea%x1871
\label{ex:3:x1871}
\gll Kerer-ep \textstyleEmphasizedVernacularWords{nan} soomare-miaw-e-mik. \\
arrive-\textsc{ss}.\textsc{seq} there walk-move.around-\textsc{pa}-1/3p\\
\glt`They arrived and walked around there.'
\z

\ea%x1872
\label{ex:3:x1872}
\gll Or-op \textstyleEmphasizedVernacularWords{nan} ikiw-ep wia uruf-a-k. \\
descend-\textsc{ss}.\textsc{seq} there go-\textsc{ss}.\textsc{seq} 3p.\textsc{acc} see-\textsc{pa}-3s\\
\glt`He went down and went there and saw them.'
\z

Transitive clauses are discussed in \sectref{sec:5.3}.

\paragraph{Ambitransitive verbs}\label{sec:3:a:z:y:x}
%\hypertarget{RefHeading20621935131865}
{}
Although most verb roots in Mauwake are clearly transitive or intransitive, there are a few that are ambitransitive. Many of their English equivalents would be intransitive. Only the following roots have been found to be neutral with regard to transitivity. Of them \textstyleStyleVernacularWordsItalic{ofof}- and \textstyleStyleVernacularWordsItalic{taan}- are of the S=O type, where the intransitive subject is an undergoer; the others are of the S=A type, where the intransitive subject is an actor \citep[124]{Dixon2010b}.

\begin{table}
\caption{Please provide a caption}
\label{} 
\begin{tabular}{ll}
\mytoprule
ofof- &`shake'\\
taan- &`become full'; `fill (a place)'\\
karu- &`run'; `visit'\\
om(om)- &`cry'; `mourn (for)'\\
pepek er- &`be enough'; `suffice'\\
aakun- &`speak, talk'; `discuss'\\
\mybottomrule
\end{tabular}

\end{table}

\ea%x1827
\label{ex:3:x1827}
\gll Ifar(a) makena wulewul \textstyleEmphasizedVernacularWords{ofof-i-ya}. \\
vine fruit wulewul shake-Np-\textsc{pr}.3s\\
\glt`The vine fruit (called) \textstyleForeignWords{wulewul} shakes.'
\z

\ea%x1826
\label{ex:3:x1826}
\gll Maa-ofofona saarik \textstyleEmphasizedVernacularWords{wia} \textstyleEmphasizedVernacularWords{ofof-a-k}. \\
earthquake like 3p.\textsc{acc} shake-\textsc{pa}-3s\\
\glt`It shook them like an earthquake.'
\z

\ea%x304
\label{ex:3:x304}
\gll Ifa uruf-ap baurar-ep \textstyleEmphasizedVernacularWords{karu-or-o-mik}. \\
snake see-\textsc{ss}.\textsc{seq} flee-\textsc{ss}.\textsc{seq} run-descend-\textsc{pa}-1/3p \\
\glt`We saw a snake and fled and ran down (to the village).'
\z

\ea%x305
\label{ex:3:x305}
\gll Epasia=pa ik-omkun me \textstyleEmphasizedVernacularWords{efa} \textstyleEmphasizedVernacularWords{karu-e-mik.} \\
far=\textsc{loc} be-1s/p.\textsc{ds} NEG 1s.\textsc{acc} run/visit-\textsc{pa}-1/3p\\
\glt`When I lived far away, they didn't visit me.'
\z

\ea%x1059
\label{ex:3:x1059}
\gll En-em-ika-eya ona wiamun=ke uruf-ap \textstyleEmphasizedVernacularWords{om-o-k}.\\
eat-\textsc{ss}.\textsc{sim}-be-2/3s.\textsc{ds} 3s.\textsc{gen} younger.sibling=\textsc{cf} see-\textsc{ss}.\textsc{seq} cry-\textsc{pa}-3s\\
\glt`When he was eating it his younger sibling saw it/him and cried.'
\z

\ea%x1060
\label{ex:3:x1060}
\gll \textstyleEmphasizedVernacularWords{Efa} \textstyleEmphasizedVernacularWords{om-em-ik-eya} epa wiim-o-k. \\
1s.\textsc{acc} cry-\textsc{ss}.\textsc{sim}-be-2/3s.\textsc{ds} place dawn-\textsc{pa}-3s\\
\glt`While she was mourning for me it dawned.'
\z

The subject of the adjunct plus verb \textstyleStyleVernacularWordsItalic{pepek er}- `be enough, suffice' is typically inanimate, whereas the object, when there is one, is usually human. 

\ea%x1058
\label{ex:3:x1058}
\gll Kemuka nain \textstyleEmphasizedVernacularWords{pepek} \textstyleEmphasizedVernacularWords{er-eya} onak ona wiar puuk-a-k.\\
string that1 enough go-2/3s.\textsc{ds} 3s/p.mother 3s.\textsc{gen} 3.\textsc{dat} cut-\textsc{pa}-3s\\
\glt`When the string was (long) enough, their mother herself cut it.'
\z

\ea%x306
\label{ex:3:x306}
\gll \textstyleEmphasizedVernacularWords{Wia} \textstyleEmphasizedVernacularWords{pepek} \textstyleEmphasizedVernacularWords{er-a-k}. \\
3p.\textsc{acc} enough come/go-\textsc{pa}-3s \\
\glt`It was enough for them.'
\z

\paragraph{Object cross-referencing verbs}\label{sec:3:a:z:y:x}
%\hypertarget{RefHeading20641935131865}
{}
One feature very common to a small group of verbs in the Trans-New Guinea languages is that the ``{verb stem undergoes changes according to the person of the object or beneficiary}'' \citep[62]{Wurm1982}.\footnote{Wurm actually seems to be referring to \textit{recipient} rather than beneficiary, as `give' is the most common of these verbs, and the verb stem changes according to the recipient.} In Mauwake this group consists of only five members. 

I call these verbs object cross-referencing because, besides marking the subject with a suffix like all other verbs do, they also \textstyleEmphasizedWords{\textsc{obligatorily}} mark the object in the verb root. What has clearly been a prefix\footnote{Phonetically this prefix is closer to the unmarked pronouns than the accusative pronouns.} earlier has been grammaticalized as part of the verb itself: there is no neutral root that would not be linked to any particular person.\footnote{When a ``neutral'' form is required, the third person singular is used.} In this respect these verbs differ from all the other verbs in Mauwake. In the case of `give' the verb root \textstyleStyleVernacularWordsItalic{i}- has assimilated into the prefix, so currently the person marking of the recipient object is the only root that there is. Four of the object cross-referencing verbs are listed in \tabref{tab:3:crossreferencingverbs}.

\begin{table}
 \caption{Cross-referencing verbs}
\label{tab:3:crossreferencingverbs}
\resizebox{\textwidth}{!}{
\begin{tabular}{llll}
\mytoprule
`give' &  `feed' &   `follow'   &`shoot'   \\
\midrule
yi- `give me' &enak  `feed me' &yook-  `follow me' &enar-  `shoot me'\\
ni- `give you' &nenak-  `feed you' &nook-  `follow you' &nenar-  `shoot you'\\
iw- `give him' &onak-  `feed him' &ook-  `follow him' &war-  `shoot him'\\
yi- `give us' &yienak-  `feed us &yiok-  `follow us &yiar-  `shoot us'\\
ni- `give you' &nienak-  `feed you' &niok-  `follow you' &niar- `shoot you'\\
wi-  `give them' &wienak-  `feed them' &wiok-  `follow them' &wiar-  `shoot them'\\
\mybottomrule 
\end{tabular}
}
\end{table}


\ea%x334
\label{ex:3:x334}
\gll Maa fain me \textstyleEmphasizedVernacularWords{iw}-o-k. \\
thing this not give.him/her-\textsc{pa}-3s \\
\glt`He did not give this thing to him/her.'
\z

\ea%x335
\label{ex:3:x335}
\gll Waaya pun \textstyleEmphasizedVernacularWords{enak}-e-mik. \\
pig too feed.me-\textsc{pa}-1/3p \\
\glt`They also gave me pork to eat.'
\z

\ea%x336
\label{ex:3:x336}
\gll Amia=iya \textstyleEmphasizedVernacularWords{nenar}-e-mik=i? \\
bow=\textsc{com} shoot.you-\textsc{pa}-1/3p=\textsc{qm}\\
\glt`Did they shoot you with a gun?'
\z

The cross-referenced objects are semantically quite different. In the verbs \textstyleStyleVernacularWordsItalic{iw}- `give (him)' and \textstyleStyleVernacularWordsItalic{onak}- `feed (him)' it is the recipient,\footnote{\textstyleFootnoteBaseChar{\textit{onak-}} `feed (him)' requires a food term as the undergoer object, so a better translation, but longer, would be `give him (something) to eat'.} in \textstyleStyleVernacularWordsItalic{war}- `shoot (him)' and \textstyleStyleVernacularWordsItalic{ook}- `follow (him)' the undergoer. The verb \textstyleStyleVernacularWordsItalic{wionar}-\footnote{A possible origin for this is \textit{PRON+onaiya+ar}- `become together-with PRON' (Kwan, p.c.)} `hide among (them)' is a special case in two ways: the cross-referenced argument `among a group' is quite untypical as a verbal argument; and only plural forms of this verb can be used because of semantic restrictions. 

\begin{table}
\caption{Please provide a caption}
\label{} 
\begin{tabular}{ll}
\mytoprule
yionar- &`hide among us'\\
nionar- &`hide among you (pl)'\\
wionar- &`hide among them'\\
\mybottomrule
\end{tabular}

\end{table}

\ea%x337
\label{ex:3:x337}
\gll Wi \textstyleEmphasizedVernacularWords{wionar}-ep pok-ap ik-ua. \\
3p.\textsc{unm} hide.among.them-\textsc{ss}.\textsc{seq} sit.down-\textsc{ss}.\textsc{seq} be-\textsc{pa}.3s \\
\glt`He sat hiding among them.'
\z

Maia does not have any verbs behaving like this \citep{Hardin2002}, and \citeauthor{Hepner2002} only reports one for Bargam: \textstyleForeignWords{\nobreakdash-g} `give' (2002:87). Usan has three verbs involving a stem change of this kind: \textstyleForeignWords{ut\^ab} `give (him)', \textstyleForeignWords{w\^ab} `shoot' and \textstyleForeignWords{w\^aramb} `hit'\citep[44]{Reesink1987}. \textstyleForeignWords{Ut\^ab}, which coreferences the recipient, has quite strict co-occurrence restrictions with other arguments or even with peripheral elements in the same clause (ibid. 129--30). 

Unlike Usan, in Mauwake the clauses with object cross-referencing verbs can easily have a locative or instrument phrase, and the verb itself can take a benefactive suffix. A sentence like \REF{ex:3:x338} would be possible for instance when sending money to people travelling in the same vehicle as the addressee. 

\ea%x338
\label{ex:3:x338}
\gll Miiw-aasa=pa \textstyleEmphasizedVernacularWords{wi-om}-\textstyleEmphasizedVernacularWords{e}. \\
land-canoe=\textsc{loc} give.them-\textsc{ben}-\textsc{bnfy}1.\textsc{imp}.2s \\
\glt`Give it to them for me in the car.'
\z

\subsubsection{Valence changes}\label{sec:3:z:y:x}
%\hypertarget{RefHeading20661935131865}
{}
The term \textstyleEmphasizedWords{\textsc{valence}} refers to the number of arguments that have a grammatical relation with the verb. As was mentioned above, almost all of the verb roots in Mauwake have a basic valence of one or two: they take either a subject only (intransitive verbs \sectref{sec:3.8.4.2.1}) or a subject and an object (transitive verbs \sectref{sec:3.8.4.2.2}) as their arguments. There are some ways to change the valence of verbs, even if strategies like passivization and dative shift are not possible in Mauwake. The valence is increased, when an intransitive verb is made into transitive or a transitive verb into causative with the addition of an causative suffix, or when a benefactive suffix is added to a verb. There are no processes to decrease the syntactic valence of a verb. The \textstyleEmphasizedWords{\textsc{semantic}} valence is decreased when the object of a transitive verb is a reflexive or reciprocal pronoun, since the subject and object have the same referent(s). Subject demotion is another way to decrease the semantic valence. 

\paragraph{Causatives}\label{sec:3:a:z:y:x}
%\hypertarget{RefHeading20681935131865}
{}
The causative always increases the number of arguments a verb can take: the subject of an intransitive verb becomes the object of a transitive verb, and a new subject is added. The causative suffix -\textstyleStyleVernacularWordsItalic{ow} was described above in \sectref{sec:3.8.2.3.1}. In most cases the meaning of a causative is `to cause someone or something to do something'. The caused `doing' is usually \textstyleEmphasizedWords{\textsc{not} }agentive. 

\ea%x997
\label{ex:3:x997}
\gll Iwera nainiw kaken iimar-e-k. (Intransitive) \\
coconut again straight stand-\textsc{pa}-3s\\
\glt`The coconut palm stood straight again.'
\z

\ea%x998
\label{ex:3:x998}
\gll [Eka napia]\textsubscript{O} koor miira=pa iimar-\textstyleEmphasizedVernacularWords{ow}-a-mik. \\
water bamboo house face=\textsc{loc} stand-CAUS-\textsc{pa}-1/3p\\
\glt`We made the bamboo water containers stand in front of the house.'
\z

\ea%x992
\label{ex:3:x992}
\gll [Wiowa erup]\textsubscript{O} ar-\textstyleEmphasizedVernacularWords{ow}-amkun um-o-k. \\
spear two become-CAUS-1s/p.\textsc{ds} die-\textsc{pa}-3s\\
\glt`I speared it a second time and it (=the pig) died.'(Lit: `I caused a spear to become two and it died.')
\z

The mental state of being angry is expressed via a verb in Mauwake \REF{ex:3:x993}, and it can take the causative suffix \REF{ex:3:x994}.

\ea%x993
\label{ex:3:x993}
\gll Kema bagiwir-a-m. \\
liver be.angry-\textsc{pa}-1s\\
\glt`I was angry.'
\z

\ea%x994
\label{ex:3:x994}
\gll Yo kema [efa]\textsubscript{O} bagiwir-\textstyleEmphasizedVernacularWords{ow}-a-n, yaa! \\
1s.\textsc{unm} liver 1s.\textsc{acc} be.angry-CAUS-\textsc{pa}-2s \textsc{intj}\\
\glt`Boy, have you made me angry!'
\z

In some cases the causative suffix acts simply as a transitiviser. The subject in \REF{ex:3:} does not actually cause the children to grow. Also in this case the suffix increases the valency of the verb: \textstyleStyleVernacularWordsItalic{arim}- `grow' in \REF{ex:3:} is intransitive, but \textstyleStyleVernacularWordsItalic{arimow}- in \REF{ex:3:} is transitive and takes an object.

\ea%x995
\label{ex:3:x995}
\gll Aakisa arim-o-n, aakisa muew-o-n. \\
now grow-\textsc{pa}-2s now marry-\textsc{pa}-2s\\
\glt`Now you have grown, now you have married.'
\z

\ea%x996
\label{ex:3:x996}
\gll No nena maa fariar-ep [muuka nain]\textsubscript{O} arim-\textstyleEmphasizedVernacularWords{ow}-e.\\
2s.\textsc{unm} 2s.\textsc{gen} food abstain-\textsc{ss}.\textsc{seq} son that1 grow-CAUS-\textsc{imp}.2s\\
\glt`You yourself have to abstain from (certain) food(s) and bring the son up.'
\z

When the causative suffix is added to the intransitive verb \textstyleStyleVernacularWordsItalic{sail}- `(tell a) lie', its meaning changes into `lie to someone', `cheat'. 

\ea%x448
\label{ex:3:x448}
\gll Opor(a) makena ma-i-yem, me [nia]\textsubscript{O} sail-\textstyleEmphasizedVernacularWords{ow}-iyem. \\
talk true say-Np-\textsc{pr}.1s not 2p.\textsc{acc} lie-CAUS-\textsc{pr}.1s\\
\glt`I am telling the truth, I am not cheating you.'
\z

Bring-\textsc{pr}efixes (\sectref{sec:3.8.2.4.2}) are another causative strategy, used only with the directional verbs (\sectref{sec:3.8.4.4.5}) and a couple of other motion verbs. The subject of the verb causes the object to move by undertaking the transfer himself/herself.

\ea%x999
\label{ex:3:x999}
\gll Maa unowa ifer aasa=ke \textstyleEmphasizedVernacularWords{p}-urup-o-k. \\
thing many sea canoe=\textsc{cf} BPx-ascend-\textsc{pa}-3s\\
\glt`A lot of things were brought/taken up by ships.'
\z

\ea%x1001
\label{ex:3:x1001}
\gll O mua imen-ap=na feeke wia \textstyleEmphasizedVernacularWords{p}-ekap-eka. \\
3s.\textsc{unm} man find-\textsc{ss}.\textsc{seq}=\textsc{tp} here.\textsc{cf} 3p.\textsc{acc} BPx-come-\textsc{imp}.2p\\
\glt`If/when you find a/any man, bring them/him here.'
\z

\ea%x1000
\label{ex:3:x1000}
\gll Gomi kawus \textstyleEmphasizedVernacularWords{p}-irapar-i-ya. \\
east.wind smoke BPx-move.to.and.fro-Np-\textsc{pr}.3s\\
\glt`The east wind moves the smoke around.'
\z

Forming a causative from an agentive verb (\textstyleEmphasizedWords{\textsc{inducive causative}}, \citealt[112]{Talmy2007}) is not done morphologically with an affix but syntactically with a verbal construction involving the nominalized form of the main verb and \textstyleStyleVernacularWordsItalic{suuw}- `push' as the causative auxiliary (\sectref{sec:5.7.1}). 

\ea%x1003
\label{ex:3:x1003}
\gll O uruf-ap op-ap Yeesus nomokowa moke \textstyleEmphasizedVernacularWords{akua-aaw-om-owa} \textstyleEmphasizedVernacularWords{suuw-a-mik}.\\
3s.\textsc{unm} see-\textsc{ss}.\textsc{seq} hold-\textsc{ss}.\textsc{seq} Jesus tree slanting shoulder-take-\textsc{ben}-\textsc{nmz} push-\textsc{pa}-1/3p\\
\glt`They saw him and took hold of him, and made him carry Jesus' cross on his shoulder.'
\z

\ea%x1002
\label{ex:3:x1002}
\gll Sira enuma \textstyleEmphasizedVernacularWords{ook-owa} \textstyleEmphasizedVernacularWords{nia} \textstyleEmphasizedVernacularWords{suuw-i-mik}. \\
custom new follow-\textsc{nmz} 2p.\textsc{acc} push-Np-\textsc{pr}.1/3p\\
\glt`They make you follow new customs/ways.'
\z

In the following examples the three different causative strategies have been applied to the same verb \textstyleStyleVernacularWordsItalic{ikiw}- `go', and in all of them the patient is [+human]. In \REF{ex:3:x1016} and \REF{ex:3:x1829} the object of the causative verb has no influence on what happens to him/her, but in \REF{ex:3:x1873} the object of the inducive causative is active and becomes the actor of the verb resulting from the causation. 

\ea%x1016
\label{ex:3:x1016}
\gll Ipamsika mua=ke \textstyleEmphasizedVernacularWords{ikiw-ow}\textstyleEmphasizedVernacularWords{-a-k}. \\
nail man=\textsc{cf} go-CAUS-\textsc{pa}-3s \\
\glt`A sorcerer (lit: nail man) killed him (lit: caused him to go).'
\z

\ea%x1829
\label{ex:3:x1829}
\gll Kes tepak=pa wu-ap \textstyleEmphasizedVernacularWords{p-ikiw-e-mik}. \\
coffin inside=\textsc{loc} put-\textsc{ss}.\textsc{seq} Bpx-go-\textsc{pa}-1/3p\\
\glt`They put him inside the coffin and took him (away).'
\z

\ea%x1873
\label{ex:3:x1873}
\gll Yo mua oko \textstyleEmphasizedVernacularWords{ikiw-owa} \textstyleEmphasizedVernacularWords{suuw-amkun} ikiw-i-non. \\
1s.\textsc{unm} man other go-\textsc{nmz} push-1s/p.\textsc{ds} go-Np-\textsc{fu}.3s\\
\glt`I make a man go and he goes.'
\z

\paragraph{ Benefactive}\label{sec:3:a:z:y:x}
%\hypertarget{RefHeading20701935131865}
{}
The benefactive form of a verb (\sectref{sec:3.8.2.3.3}) is used when an action is done \textstyleEmphasizedWords{\textsc{for} }someone, for their benefit \REF{ex:3:x1004}, or in some cases for their detriment \REF{ex:3:x1005}. With the addition of the benefactive suffix to the verb, the beneficiary becomes an obligatory argument. The beneficiary is always animate, and usually human. 

\ea%x1004
\label{ex:3:x1004}
\gll Wi owow mua=ke wilkar wia muf-em-ik-\textstyleEmphasizedVernacularWords{om}-a-mik.\\
3p.\textsc{unm} village man=\textsc{cf} cart 3p.\textsc{acc} pull-\textsc{ss}.\textsc{sim}-be-\textsc{ben}-\textsc{bnfy}2.\textsc{pa}-1/3p\\
\glt`The village men kept pulling carts for them.'
\z

\ea%x1005
\label{ex:3:x1005}
\gll Epia wilin-owa uruf-ap bom yia fuurk-\textstyleEmphasizedVernacularWords{om}-i-kuan.\\
fire(wood) shine-\textsc{nmz} see-\textsc{ss}.\textsc{seq} bomb 1p.\textsc{acc} throw-\textsc{ben}-Np-\textsc{fu}.3p\\
\glt`When they see the light from the fire(s) they will throw bombs at us.'
\z

More than one valency-increasing strategy can be applied to a verb simultaneously. In both \REF{ex:3:x1007} and \REF{ex:3:x1008} the valency of the verb increases from one to three: besides the subject of the original verb, the derived verbs also have both an object and a beneficiary.

\ea%x1007
\label{ex:3:x1007}
\gll Koor poka iimar-\textstyleEmphasizedVernacularWords{ow}-\textstyleEmphasizedVernacularWords{om}-e. \\
house post stand.up-CAUS-\textsc{ben}-\textsc{bnfy}1.\textsc{imp}.2s\\
\glt`Stand up the house posts for me.'
\z

\ea%x1008
\label{ex:3:x1008}
\gll Ona soomia marew-eya \textstyleEmphasizedVernacularWords{amap}-ep-\textstyleEmphasizedVernacularWords{om}-a-m. \\
3s.\textsc{gen} spoon no(ne)-2/3s.\textsc{ds} BPx-come-\textsc{ben}-\textsc{bnfy}2.\textsc{pa}-1s\\
\glt`She has/had no spoons of her own, so I brought them for her.'
\z

\paragraph{Decreasing semantic valence}\label{sec:3:a:z:y:x}
%\hypertarget{RefHeading20721935131865}
{}
There are no morphological means in Mauwake for decreasing syntactic valence. A verb that is inherently reflexive, like \textstyleStyleVernacularWordsItalic{yaki}- `wash oneself', is intransitive. But the semantic valence of transitive verbs is decreased when they are made either reflexive or reciprocal. Syntactically the reflexive/reciprocal pronoun is an object, but the pronoun refers to the same referent(s) as the subject. 

\ea%x1834
\label{ex:3:x1834}
\gll Birin-ep nomokowa iinan akena ikiw-ep wame pipilim-ep aakun-em-ika-i-non.\\
fly-\textsc{ss}.\textsc{seq} tree top very go-\textsc{ss}.\textsc{seq} 3s.\textsc{refl} hide-\textsc{ss}.\textsc{seq} speak-\textsc{ss}.\textsc{sim}-be-Np-\textsc{fu}.3s\\
\glt`It will fly and hide (itself) in the very top of a tree and keep making noise.'
\z

\ea%x1835
\label{ex:3:x1835}
\gll Osaiwa aalbok ikos uf-owa na-ep ofa wiam if-e-mik.\\
bird.of.paradise black.cuckoo-shrike together dance-\textsc{nmz} say-\textsc{ss}.\textsc{seq} colour 3p.\textsc{refl} paint-\textsc{pa}-1/3p\\
\glt`A bird of paradise and a black cuckoo-shrike wanted to dance together and painted each other with colour.'
\z

A common valence-decreasing device in many languages is the passive voice, which demotes or deletes the subject. In Mauwake verbs there is no passive voice. The standard way of demoting the subject is to have the verb in third person plural form and leave the subject \textstyleAcronymallcaps{NP} unexpressed.\footnote{Cf. the English impersonal ``they'': \textit{They say it is going to be cold tomorrow}.} None of the arguments change their syntactic function. The example \REF{ex:3:} comes from a story where the main point was that the people responsible for the fire were never found, and it was not known if only one person was involved or many. 

\ea%x1009
\label{ex:3:x1009}
\gll Fikera ikum \textstyleEmphasizedVernacularWords{kuum-e-mik} nain ma-i-yem. \\
kunai.grass illicitly burn-\textsc{pa}-1/3p that1 say-Np-\textsc{pr}.1s\\
\glt`I tell about when the kunai grass was burned (by arson).'
\z

\ea%x1010
\label{ex:3:x1010}
\gll Nefa \textstyleEmphasizedVernacularWords{war-iwkin} naap ma-e. \\
2s.\textsc{acc} shoot-2/3p.\textsc{ds} thus say-\textsc{imp}.2s\\
\glt`If/when you are shot, then say like that.' (Or: `If they shoot you, then say like that.')
\z

Another strategy to demote the subject is to use the same-subject sequential form of the main verb and the auxiliary \textstyleStyleVernacularWordsItalic{ik}- `be' agreeing with the object of the verb. This can only be used when the end result is a state. 

\ea%x1011
\label{ex:3:x1011}
\gll Nomokowa puuk-ap ik-ua. \\
tree cut-\textsc{ss}.\textsc{seq} be-\textsc{pa}.3s\\
\glt`The tree is cut.'
\z

\subsubsection{Semantically based verb classes} \label{sec:3:z:y:x}
%\hypertarget{RefHeading20741935131865}
{}
Even though the following classification is based on semantic characteristics of the verbs, the verbs within the resulting groups tend to have similarities in their syntactic behaviour as well.

\paragraph{Stative/existential verb \textit{ik}-} \label{sec:3:a:z:y:x}
%\hypertarget{RefHeading20761935131865}
{}
The basic meaning of the stative verb \textstyleStyleVernacularWordsItalic{ik}(\textstyleStyleVernacularWordsItalic{a})- is `be'. The vowel /a/ gets deleted elsewhere except in the present tense and the medial different-subject non-first plural form; in the corresponding singular form the vowel may be optionally deleted \REF{ex:3:x257}. 

\ea%x257
\label{ex:3:x257}
\gll Nan mukuna=pa \textstyleEmphasizedVernacularWords{ik(a)-eya} o nan samor aaw-o-k. \\
there fire=\textsc{loc} be-2/3s.\textsc{ds} 3s.\textsc{unm} there badly get-\textsc{pa}-3s\\
\glt`They (=bananas) were there on the fire and he really got bad there.'
\z

Like intransitive verbs, it may form a complete clause by itself. Example \REF{ex:3:x1455} is from a situation where the speaker was in a plane for the first time, refused to eat and declined any help offered to him.

\ea%x1455
\label{ex:3:x1455}
\gll \textstyleEmphasizedVernacularWords{Ika-i-nen}. \\
be-Np-\textsc{fu}.1s\\
\glt`I will just be (like this).'
\z

Often it is used for `be/live (somewhere)', and in this use it naturally co-occurs with a locative adverbial:

\ea%x497
\label{ex:3:x497}
\gll I naap koora=pa \textstyleEmphasizedVernacularWords{ik-e-mik}. \\
1p.\textsc{unm} thus house=\textsc{loc} be-\textsc{pa}-1/3p\\
\glt`We were in the house like that.'
\z

Together with the dative pronouns \textstyleStyleVernacularWordsItalic{ik}- is used to form possessive constructions \REF{ex:3:x258} (\sectref{sec:3.5.5}, 5.5.2). 

\ea%x258
\label{ex:3:x258}
\gll Yo waaya arow \textstyleEmphasizedVernacularWords{efar} \textstyleEmphasizedVernacularWords{ik-ua.} \\
1s.\textsc{unm} pig three 1s.\textsc{dat} be-\textsc{pa}.3s \\
\glt`I have three pigs.'
\z

The function of \textstyleStyleVernacularWordsItalic{ik}- as a copular verb is very restricted. In equative or descriptive clauses it is normally not used in the present tense finite form, but in the past \REF{ex:3:x259} and future \REF{ex:3:x1070} tenses it is employed. It could be said, following \citet[92]{Givon1984}, that in these clauses its primary function is to be the carrier of the tense. 

\ea%x259
\label{ex:3:x259}
\gll Yo um-ep ik-owa saarik \textstyleEmphasizedVernacularWords{ik-e-m}. \\
1s.\textsc{unm} die-\textsc{ss}.\textsc{seq} be-\textsc{nmz} like be-\textsc{pa}-1s\\
\glt`I was like dead.'
\z

\ea%x1070
\label{ex:3:x1070}
\gll Ikoka maa marew, eliw manek=iw \textstyleEmphasizedVernacularWords{ika-i-nan}. \\
later thing none well big=\textsc{lim} be-Np-\textsc{fu}.2s\\
\glt`Later there will be no problem, you will just be very well.'
\z

In Mauwake it can also be used when the non-verbal predicate is understood to be transitory \REF{ex:3:x499} rather than stable over time:

\ea%x499
\label{ex:3:x499}
\gll No kamenap \textstyleEmphasizedVernacularWords{ika-i-n}? \\
2s.\textsc{unm} how be-Np-\textsc{pr}.2s\\
\glt`How are you?' 
\z

The verb \textstyleStyleVernacularWordsItalic{ik}- `be' is in a class of its own for several reasons. Its morphology is irregular, and so are the semantics of some of its morphology. In \REF{ex:3:} the past tense and the person/number marker in the third person singular form are merged into one portmanteau morpheme. An alternative form for the different-subject first person form \textstyleStyleVernacularWordsItalic{ikemkun} is \textstyleStyleVernacularWordsItalic{ikomkun} \REF{ex:3:}. The same-subject medial form is \textstyleStyleVernacularWordsItalic{ikok} \REF{ex:3:}, not *\textstyleStyleVernacularWordsItalic{ikep} and *\textstyleStyleVernacularWordsItalic{ikemi}\textstyleEmphasizedVernacularWords{\textmd{\textit{.}}}\footnote{\textstyleFootnoteBaseChar{\textit{ikep}} and \textstyleFootnoteBaseChar{\textit{ikemi}} are the same subject medial forms of the homophonous verb \textstyleFootnoteBaseChar{\textit{ik-}} `roast'.} There is no formal differentiation between a simultaneous \REF{ex:3:} and a sequential \REF{ex:3:} form in the same-subject medial verb. 

\ea%x1931
\label{ex:3:x1931}
\gll Siowa nain kakalt-am-\textstyleEmphasizedVernacularWords{ik}\textstyleEmphasizedVernacularWords{-}\textstyleEmphasizedVernacularWords{emkun} arim-o-k. \\
dog that1 look.after-\textsc{ss}.\textsc{sim}-be-1s/p.\textsc{ds} grow-\textsc{pa}-3s\\
\glt`I was looking after the dog and it grew.'
\z

\ea%x260
\label{ex:3:x260}
\gll Naap \textstyleEmphasizedVernacularWords{ik-ok} uruf-am-ika-iwkin wia. \\
thus be-SS see-\textsc{ss}.\textsc{sim}-be-2/3p.\textsc{ds} no\\
\glt`As he was/stayed like that they were watching him (but) no (=he didn't get better).' 
\z

\ea%x262
\label{ex:3:x262}
\gll Owowa ekap-o-k, amia mua=pa \textstyleEmphasizedVernacularWords{ik-ok}. \\
village come-\textsc{pa}-3s bow man=\textsc{loc} be-SS \\
\glt`He came to the village, having been in the police (force).'
\z

It also differs from ordinary intransitive verbs in that in a verb+auxiliary construction it cannot be the main verb, but can be used as the aspectual auxiliary \REF{ex:3:} (see also \sectref{sec:3.8.4.5}). But it is like other intransitive verbs in that it can take an causative suffix \REF{ex:3:}.\footnote{\citet[142]{Reesink1987} notes that in Usan the corresponding verb \textstyleFootnoteBaseChar{\textit{igo}} `be' cannot occur with the causative suffix. In Mauwake there is no similar restriction.} 

\ea%x261
\label{ex:3:x261}
\gll Nomokowa war-ep miiwa=pa \textstyleEmphasizedVernacularWords{ik-ow-a-mik.} \\
tree cut-\textsc{ss}.\textsc{seq} ground=\textsc{loc} be-CAUS-\textsc{pa}-1/3p\\
\glt`We cut trees and laid them on the ground' 
\z

The tense distinction is partly neutralized: the past tense form is used for past (\textstyleParagraphChari{\stepcounter{nx}{\thenx}}) and present (\textstyleParagraphChari{\stepcounter{nx}{\thenx}}). The present tense form is not very common and is mainly used for less time-stable situations \REF{ex:3:}, \REF{ex:3:}, or to replace the missing continuous aspect form (\textstyleParagraphChari{\stepcounter{nx}{\thenx}}). The verb \textstyleStyleVernacularWordsItalic{ik}- is used as the regular continuous aspect auxiliary (\sectref{sec:4.4.1}), and as a main verb \textstyleStyleVernacularWordsItalic{ik}- `be' cannot take this auxiliary. 

\ea%x263
\label{ex:3:x263}
\gll Yo unan koka=pa \textstyleEmphasizedVernacularWords{ik-e-m}. \\
1s.\textsc{unm} yesterday jungle=\textsc{loc} be-\textsc{pa}-1s \\
\glt`Yesterday I was in the jungle.' 
\z

\ea%x264
\label{ex:3:x264}
\gll Ni kaaneke \textstyleEmphasizedVernacularWords{ik-e-man} oo, ni ekap-omak-eka oo! \\
2p.\textsc{unm} where be-\textsc{pa}-2p oh 2p.\textsc{unm} come-DISTR-\textsc{imp}.2p oh \\
\glt`Wherever you are, come!'
\z

\ea%x1028
\label{ex:3:x1028}
\gll Mesa asia fiker gone=pa \textstyleEmphasizedVernacularWords{ika-i-ya} nain aaw-em-ik-e-m.\\
winged.bean wild kunai.grass middle=\textsc{loc} be-Np-\textsc{pr}.3s that1 take-\textsc{ss}.\textsc{sim}-be-\textsc{pa}-1s\\
\glt`I was picking wild winged bean that was (lit: is) in the middle of the kunai grass.'
\z

\ea%x265
\label{ex:3:x265}
\gll Yo nan \textstyleEmphasizedVernacularWords{ika-i-yem} nain yo nia asip-i-yem, {\dots} \\
1s.\textsc{unm} there be-Np-\textsc{pr}.1s that1 1s.\textsc{unm} 2p.\textsc{acc} help-Np-\textsc{pr}.1s\\
\glt`Now that I am living there I help you, {\dots}'
\z

The verb \textstyleStyleVernacularWordsItalic{ik}- mainly functions in intransitive clauses, but it is also needed as a copula for those cases where a non-verbal predicate is in a non-\textsc{pr}esent tense. 

\ea%x969
\label{ex:3:x969}
\gll O ikoka somek mua maneka \textstyleEmphasizedVernacularWords{ika-i-non}. \\
3p.\textsc{unm} later song man big be-Np-\textsc{fu}.3s\\
\glt`He will later be the headmaster.'
\z

An equative or descriptive medial clause requires \textstyleStyleVernacularWordsItalic{ik}- as a copula regardless of the tense of the final verb \REF{ex:3:x498}.

\ea%x498
\label{ex:3:x498}
\gll Koora naap \textstyleEmphasizedVernacularWords{ik-eya} uruf-i-mik. \\
house thus be-2/3s.\textsc{ds} see-Np-\textsc{pr}.1p\\
\glt`We see the house as it is like that.'
\z

\paragraph{Position-taking verbs}\label{sec:3:a:z:y:x}
%\hypertarget{RefHeading20781935131865}
{}
The three position-taking verbs are among the most frequently used verbs in Mauwake: \textstyleStyleVernacularWordsItalic{pok}- `sit down', \textstyleStyleVernacularWordsItalic{iimar}- `stand up' and \textstyleStyleVernacularWordsItalic{in}- `lie down/ fall asleep'. They are essentially punctiliar verbs with an inceptive meaning \REF{ex:3:x273}, but they are most typically used in the same-subject sequential form together with the auxiliary \textstyleStyleVernacularWordsItalic{ik}- (\sectref{sec:3.8.4.5}) to convey stative meaning: `sit' \REF{ex:3:x274}, `stand', and `lie/sleep'.

\ea%x273
\label{ex:3:x273}
\gll Kokom-ar-eya \textstyleEmphasizedVernacularWords{in-e-mik}. \\
darkness-\textsc{inch}-2/3s.\textsc{ds} lie.down-\textsc{pa}-1/3p \\
\glt`When it got dark we went to bed.'
\z

\ea%x274
\label{ex:3:x274}
\gll Ona koora=pa arew-ap \textstyleEmphasizedVernacularWords{pok-ap} \textstyleEmphasizedVernacularWords{ik-e-mik}. \\
3s.\textsc{gen} house=\textsc{loc} wait-\textsc{ss}.\textsc{seq} sit.down-\textsc{ss}.\textsc{seq} be-\textsc{pa}-1/3p\\
\glt`We sat and waited (lit: waited and sat) in his house.'
\z

The verb \textstyleStyleVernacularWordsItalic{pok}- is occasionally used without the auxiliary to mean `sit': 

\ea%x1824
\label{ex:3:x1824}
\gll Neek(e) \textstyleEmphasizedVernacularWords{pok-aka}. \\
there sit-\textsc{imp}.2p\\
\glt`Sit there/Keep sitting there.' (Commonly used as a conversational ``filler'' for people that are already sitting, when there is a lull in the conversation.)
\z

The continuous aspect form of the position-taking verbs is not used with progressive meaning, only with the meaning `habitual' (\sectref{sec:3.8.5.1.1.2}). 

\ea%x275
\label{ex:3:x275}
\gll Irak-ow epa=pa koka=pa \textstyleEmphasizedVernacularWords{in-em-ik-e-mik}. \\
fight-\textsc{nmz} time=\textsc{loc} jungle=\textsc{loc} lie.down-\textsc{ss}.\textsc{sim}-be-\textsc{pa}-1/3p \\
\glt`During the war we used to sleep in the jungle.'
\z

\paragraph{Location verbs}\label{sec:3:a:z:y:x}
%\hypertarget{RefHeading20801935131865}
{}
The two verbs that have been verbalized from the demonstrative adverbs \textstyleStyleVernacularWordsItalic{fan} `here' and \textstyleStyleVernacularWordsItalic{nan} `there' (\sectref{sec:3.8.2.2.1}), are very restricted in their use. The original meaning of the verbs must refer to arrival at some place, but since they are only used in the past tense, they currently tend to indicate presence at a place rather than movement.\footnote{This may indicate that the past tense used to encode perfectivity (Malcolm Ross, p.c.)} They can even be used with immobile objects \REF{ex:3:}. 

\ea%x1270
\label{ex:3:x1270}
\gll No ikiw-e, irak-owa maneka \textstyleEmphasizedVernacularWords{fan}\textstyleEmphasizedVernacularWords{-}\textstyleEmphasizedVernacularWords{e}\textstyleEmphasizedVernacularWords{-}\textstyleEmphasizedVernacularWords{k} a. \\
2s.\textsc{unm} go-\textsc{imp}.2s fight-\textsc{nmz} big here-\textsc{pa}-3s \textsc{intj}\\
\glt`Go (home), the big war is here.'
\z

\ea%x1271
\label{ex:3:x1271}
\gll Aakisa i \textstyleEmphasizedVernacularWords{fan-e-mik}. \\
Now 1p.\textsc{unm} here-\textsc{pa}-1/3p \\
\glt`Now we are / have come here.'
\z

\ea%x1272
\label{ex:3:x1272}
\gll No niawi akena \textstyleEmphasizedVernacularWords{nan-e-k}. \\
2s.\textsc{unm} 2s/p.father true there-\textsc{pa}-3s\\
\glt`Your real father is there.'
\z

\ea%x1276
\label{ex:3:x1276}
\gll Aa, o koora \textstyleEmphasizedVernacularWords{fan-e-k} a. \\
\textsc{intj} 3s.\textsc{unm} house here-\textsc{pa}-3s \textsc{intj}\\
\glt`Ah, his house is here.'
\z

\paragraph{Resultative verbs}\label{sec:3:a:z:y:x}
%\hypertarget{RefHeading20821935131865}
{}
The resultative verbs with the meaning `become' are another small group of intransitive verbs. Besides the semantic similarity they also share the syntactic characteristic that, in addition to the subject, they require another argument expressing the result of change. This other obligatory argument is a noun with the verbs \textstyleStyleVernacularWordsItalic{ar}- `become' and \textstyleStyleVernacularWordsItalic{puuk}- `change into',\footnote{This verb is homonymous with the transitive verb \textstyleFootnoteBaseChar{\textit{puuk-}} `cut'. They may be historically related, but synchronically the meanings are quite different.} and a colour adjective with the verb \textstyleStyleVernacularWordsItalic{kir}- `turn'. 

\ea%x276
\label{ex:3:x276}
\gll Takira arim-ep mua \textstyleEmphasizedVernacularWords{ar-e-k}. \\
boy grow-\textsc{ss}.\textsc{seq} man become-\textsc{pa}-3s \\
\glt`The boy grew and became a man.'
\z

\ea%x277
\label{ex:3:x277}
\gll Emeria nain afa \textstyleEmphasizedVernacularWords{ar-e-mik}. \\
woman that1 flyng.fox become-\textsc{pa}-1/3p \\
\glt`Those women became flying foxes.'
\z

\ea%x278
\label{ex:3:x278}
\gll Inasin mua ifa \textstyleEmphasizedVernacularWords{puuk-ap} solon-ep {\dots} \\
spirit man snake change.into-\textsc{ss}.\textsc{seq} glide-\textsc{ss}.\textsc{seq} \\
\glt`The spirit man changed into a snake, glided and {\dots}'
\z

\ea%x279
\label{ex:3:x279}
\gll Oona kia \textstyleEmphasizedVernacularWords{kir-em-ik-eya} uruf-ap ma-e-k {\dots} \\
bone white turn-\textsc{ss}.\textsc{sim}-be-2/3s.\textsc{ds} see-\textsc{ss}.\textsc{seq} say-\textsc{pa}-3s \\
\glt`She saw that the bones were turning white and said, {\dots}'
\z

The verb \textstyleStyleVernacularWordsItalic{ar}- is mostly used when the subject stays essentially the same but undergoes some change \REF{ex:3:}. However, it can also be used when the subject changes into something else \REF{ex:3:}. The verb \textstyleStyleVernacularWordsItalic{puuk}- is only used in the latter context \REF{ex:3:}, and it is always an intentional action. It is most common in traditional stories where spirits change into various inanimate things or animate beings. The verb \textstyleStyleVernacularWordsItalic{kir}- is used with most colour terms \REF{ex:3:}, but for `black' there is a separate verb formed with the inchoative suffix \nobreakdash-\textstyleStyleVernacularWordsItalic{ar} : \textstyleStyleVernacularWordsItalic{sepenar}-\footnote{This is related to the adjective \textstyleFootnoteBaseChar{\textit{sepa}} `black'.} `become black'. The inchoative suffix (\sectref{sec:3.8.2.2.2}) is the standard device used for verbalizing adjectives. 

\paragraph{Directional verbs}\label{sec:3:a:z:y:x}
%\hypertarget{RefHeading20841935131865}
{}
The verbs indicating coming and going are among the most frequent verbs in Mauwake. These verbs have the direction inherent in the verb root. Verbs of this kind are quite common among Papuan languages: in some languages the directional is an affix, in others it is part of the meaning of the root itself \citep[149]{Foley1986}; Mauwake is of the latter type. The directional verb group contains verbs that in many languages would be prototypically intransitive.

Most of these verbs can be translated into English as either `go' or `come', depending on the context. Since the elevation of the goal, the direction of the compass and the distance all influence the choice of the verb, and may conflict with each other, the speaker has some freedom of choice. Also, with regard to proximity, it is a very relative notion how close or far away something is.

\begin{table}
\begin{tabular}{ll}
\mytoprule
ikiw- &`go', `leave' (away from the deictic centre; generic)\\
iw- &`go' (away from the deictic centre)\footnote{In the Moro area \textstyleFootnoteBaseChar{\textit{iw-}} also has the meaning `enter': \textstyleFootnoteBaseChar{\textit{Marasin kema wiar iwak}} `The poison entered his liver.'}\\
ekap- &`come' (towards the deictic centre; generic)\\
urup- &`go/come up', `ascend' (uphill/away from sea)\\
or(a)- &`go/come down', `descend' (downhill/towards sea)\\
ek- &`go (close/east)'\\
ep- &`come (close/west)'\\
er- &`go (not close/west/downriver)'\\
ir- &`come/go (not close/east/upriver)', `climb'\\
\mybottomrule 
\end{tabular}
\caption{Directional verbs.}
\end{table}

\ea%x280
\label{ex:3:x280}
\gll Manina \textstyleEmphasizedVernacularWords{urup-ep} nan uuw-ap owowa \textstyleEmphasizedVernacularWords{or-o-k}. \\
garden ascend-\textsc{ss}.\textsc{seq} there work-\textsc{ss}.\textsc{seq} village descend-\textsc{pa}-3s\\
\glt`She went up to the garden, worked there and came down to the village.'
\z

\ea%x281
\label{ex:3:x281}
\gll Fofa \textstyleEmphasizedVernacularWords{er-ap} \textstyleEmphasizedVernacularWords{ir-i-mik}. \\
market go.\textsc{ss}.\textsc{seq} come-Np-\textsc{pr}.1/3p \\
\glt`We are coming back from the market.' (Lit: `We went west to the market and are coming east.')
\z

The deictic orientation of \textstyleStyleVernacularWordsItalic{ikiw}- `away from speaker/deictic centre' and \textstyleStyleVernacularWordsItalic{ekap}- `towards the speaker/deictic centre'is stricter in Mauwake than in many European languages where the deictic centre especially for `come' can vary considerably. The sentence \REF{ex:3:x274} is all right in Finnish regardless of the location of the speaker, but the corresponding sentence in Mauwake would be acceptable only if the speaker were in Tampere at the time of speaking.

\ea%x282
\label{ex:3:x282}
\gll Isois\"ani \textstyleForeignWords{tuli} Tampereelle vuonna 1912. (Finnish)\\ \\
\glt`My grandfather \textstyleEmphasizedWords{came} to Tampere in 1912.' 
\z

The equivalent of the English `come' in \REF{ex:3:x283} has to be `go' in Mauwake \REF{ex:3:x284}. This is discussed further in 6.3. 

\ea%x283
\label{ex:3:x283}
\gll I'll \textstyleEmphasizedWords{\textsc{come}} to your place tomorrow. \\
\\
\z

\ea%x284
\label{ex:3:x284}
\gll Uurika nefa uruf-owa \textstyleEmphasizedVernacularWords{ikiw-i-nen}. \\
tomorrow 2s.\textsc{acc} see-\textsc{nmz} go-Np-\textsc{fu}.1s \\
\glt`Tomorrow I'll go to see you.'
\z

When these verbs occur with a locative phrase containing the locative marker (\sectref{sec:3.12.4}), the phrase almost always refers to either source \REF{ex:3:}, or location/path \REF{ex:3:}. The goal is very seldom marked with the locative marker -\textstyleStyleVernacularWordsItalic{pa}; this happens when the goal is important mainly as the location of the next event \REF{ex:3:}. Also, in \REF{ex:3:} \textstyleStyleVernacularWordsItalic{mukuna} `fire' is an untypical goal for a directional verb.

\ea%x285
\label{ex:3:x285}
\gll \textstyleEmphasizedVernacularWords{Manina}\textstyleEmphasizedVernacularWords{=pa} \textstyleEmphasizedVernacularWords{ekap-ep} maa uup-e-mik. \\
garden=\textsc{loc} come-\textsc{ss}.\textsc{seq} food cook-\textsc{pa}-1/3p \\
\glt`We came from the garden and cooked food.'
\z

\ea%x447
\label{ex:3:x447}
\gll Iinan aasa \textstyleEmphasizedVernacularWords{iinan=pa} fan \textstyleEmphasizedVernacularWords{ekap-emi} {\dots} \\
sky canoe sky=\textsc{loc} here come-\textsc{ss}.\textsc{sim}\\
\glt`The airplane came here in the sky and{\dots}'
\z

\ea%x1878
\label{ex:3:x1878}
\gll Ne soran-emi \textstyleEmphasizedVernacularWords{epia} \textstyleEmphasizedVernacularWords{mukuna}\textstyleEmphasizedVernacularWords{=}\textstyleEmphasizedVernacularWords{pa} \textstyleEmphasizedVernacularWords{or}\textstyleEmphasizedVernacularWords{-}\textstyleEmphasizedVernacularWords{omi} aw-o-k.
\\
\textsc{add} get.startled-\textsc{ss}.\textsc{sim} firewood fire=\textsc{loc} descend-\textsc{ss}.\textsc{sim} burn-\textsc{pa}-3s\\
\glt`And he got startled and fell on the fire and burned himself.'
\z

The directional verbs differ from other verbs in Mauwake in that they can be transitivized with the `bring' prefixes \textstyleStyleVernacularWordsItalic{p}-, \textstyleStyleVernacularWordsItalic{amap}- and \textstyleStyleVernacularWordsItalic{aap}- (\sectref{sec:3.8.2.4.2}) to indicate either bringing or taking something somewhere.

\ea%x286
\label{ex:3:x286}
\gll Ona owowa \textstyleEmphasizedVernacularWords{p-ikiw-ep} soop-i-yan. \\
3s.\textsc{gen} village BPx-go-\textsc{ss}.\textsc{seq} bury-Np-\textsc{fu}.1p \\
\glt`We'll take him (=his body) in his village and bury him (there).'
\z

The causative suffix \nobreakdash-\textstyleStyleVernacularWordsItalic{ow}\textstyleEmphasizedWords{} (\sectref{sec:3.8.4.3.1})\textstyleEmphasizedWords{} can be added to the roots; when following a one-syllable root the suffix is often reduplicated, but the meaning is still the same as with a single causative suffix.

\ea%x435
\label{ex:3:x435}
\gll Purowa ir-\textstyleEmphasizedVernacularWords{ow}-(\textstyleEmphasizedVernacularWords{ow})-eya siin-ar-e-k. \\
armband go.up-CAUS-CAUS-2/3s.\textsc{ds} tight-\textsc{inch}-\textsc{pa}-3s\\
\glt`She pushed the armband up and it got tight.'
\z

The directional verbs are very frequent as the second root in serial verbs \REF{ex:3:x287} (\sectref{sec:3.8.5.1.2}) and as the main verb in verb plus auxiliary constructions \REF{ex:3:x288} (\sectref{sec:3.8.5.1.1}). Some of them also enter into adjunct plus verb constructions \REF{ex:3:x289} (\sectref{sec:3.8.5.2}). 

\ea%x287
\label{ex:3:x287}
\gll Wi Amerika ``epa eliwa'' nae-\textstyleEmphasizedVernacularWords{ekap}-e-mik. \\
3p.\textsc{unm} America time good say-come-\textsc{pa}-1/3p \\
\glt`The Americans came saying, ``peace''.'
\z

\ea%x288
\label{ex:3:x288}
\gll Wi Yaapan saa=iw \textstyleEmphasizedVernacularWords{ir}-am-ika-i-mik. \\
3p.\textsc{unm} Japan sand=\textsc{inst} go-\textsc{ss}.\textsc{sim}-be-Np-\textsc{pr}.1/3p \\
\glt`The Japanese are going along the beach.'
\z

\ea%x289
\label{ex:3:x289}
\gll Kemuka \textstyleEmphasizedVernacularWords{pepek} \textstyleEmphasizedVernacularWords{er-}eya puuk-a-k. \\
string enough go-2/3s.\textsc{ds} cut-\textsc{pa}-3s \\
\glt`When the string was (long) enough she cut it.'
\z

The meaning of the verbs \textstyleStyleVernacularWordsItalic{ekap}- `come' and \textstyleStyleVernacularWordsItalic{ikiw}- `go' can be metaphorically extended to time, to signal time spans. The former is used when the time span is extended from the past to the present \REF{ex:3:x290}, the latter is more common when the time extends from the present to the future \REF{ex:3:x437}, but it can also refer to the past \REF{ex:3:x291}.

\ea%x290
\label{ex:3:x290}
\gll Naap on-am-ik-e-mik, \textstyleEmphasizedVernacularWords{ekap-ep}  aakisa. \\
thus do-\textsc{ss}.\textsc{sim}-be-\textsc{pa}-1/3p come-\textsc{ss}.\textsc{seq} now\\
\glt`We have been doing like that (all the time) up until now.'
\z

\ea%x437
\label{ex:3:x437}
\gll No naap ik-ok \textstyleEmphasizedVernacularWords{iki(w-e)p} mokoma enuma iiwawun aakun-i-nan.\\
2s.\textsc{unm} thus be-SS go-\textsc{ss}.\textsc{seq} year new altogether talk-Np-\textsc{fu}.2s\\
\glt`You will be like that (long time) but next year you will talk.'
\z

\ea%x291
\label{ex:3:x291}
\gll Buren \textstyleEmphasizedVernacularWords{ife-iki}\textstyleEmphasizedVernacularWords{(}\textstyleEmphasizedVernacularWords{w-e}\textstyleEmphasizedVernacularWords{)}\textstyleEmphasizedVernacularWords{p} aakisa arim-o-n. \\
ceremonial.liquid rub-go-\textsc{ss}.\textsc{seq} now grow-\textsc{pa}-2s \\
\glt`You have kept rubbing the \textstyleEmphasizedWords{buren}\textit{} liquid on (for years), and now you have grown up.'
\z

On the fringe of directional verbs are \textstyleStyleVernacularWordsItalic{kerer}- `arrive', \textstyleStyleVernacularWordsItalic{yiaw}-/\textstyleStyleVernacularWordsItalic{miaw}- `walk/move around, wander' and \textstyleStyleVernacularWordsItalic{irapar}- `move back and forth (aimlessly)', which share some of their grammatical characteristics but not all of them. Of these three verbs, \textstyleStyleVernacularWordsItalic{kerer}- cannot be prefixed with the bring-\textsc{pr}efixes, but it mainly occurs with an unmarked goal instead of a locative phrase \REF{ex:3:x292}.

\ea%x292
\label{ex:3:x292}
\gll Emeria mua manina \textstyleEmphasizedVernacularWords{kerer-e-mik}. \\
woman man garden arrive-\textsc{pa}-1/3p \\
\glt`The people arrived in the garden.'
\z

With the other two a bring-\textsc{pr}efix is acceptable \REF{ex:3:x293}, but they do not take a goal/path argument. If a locative phrase occurs with them it requires a locative clitic \REF{ex:3:x436}.

\ea%x293
\label{ex:3:x293}
\gll Gomi kawus \textstyleEmphasizedVernacularWords{p-irapar-i-ya}. \\
east.wind smoke BPx-move.back.and.forth-Np-\textsc{pr}.3s \\
\glt`The east wind moves/blows the smoke around.'
\z

\ea%x436
\label{ex:3:x436}
\gll Soora=pa nan \textstyleEmphasizedVernacularWords{yiaw-e-mik}. \\
jungle=\textsc{loc} there walk.around-\textsc{pa}-1/3p \\
\glt`They walked around in the jungle.'
\z

\paragraph{Utterance verbs}\label{sec:3:a:z:y:x}
%\hypertarget{RefHeading20861935131865}
{}
Utterance verbs may be either intransitive \REF{ex:3:}, ambitransitive \REF{ex:3:}, \REF{ex:3:}, or transitive \REF{ex:3:}. They may be used to introduce a quote complement, but not to close it. They often occur with one of the `saying' verbs described below \REF{ex:3:}.

\ea%x309
\label{ex:3:x309}
\gll Takira niir-emi \textstyleEmphasizedVernacularWords{kirir-i-mik}. \\
boy play-\textsc{ss}.\textsc{sim} shout-Np-\textsc{pr}.1/3p \\
\glt`The boys are playing and shouting.'
\z

\ea%x310
\label{ex:3:x310}
\gll Wi iperowa=ke \textstyleEmphasizedVernacularWords{aakun-ep} ma-e-mik, ``{\dots}'' \\
3p.\textsc{unm} middle.aged=\textsc{cf} discuss-\textsc{ss}.\textsc{seq} say-\textsc{pa}-1/3p\\
\glt`The middle-aged men discussed (it) / talked and said, ``{\dots}'' '
\z

\ea%x1927
\label{ex:3:x1927}
\gll Maapora kamenap \textstyleEmphasizedVernacularWords{aakun-i-yan}? \\
feast how discuss-Np-\textsc{fu}.1p\\
\glt`How shall we discuss the feast?'
\z

\ea%x311
\label{ex:3:x311}
\gll Yena mua \textstyleEmphasizedVernacularWords{far-e-m}, ``Sarak oo, {\dots}'' \\
1s.\textsc{gen} man call-\textsc{pa}-1s Sarak oh \\
\glt`I called to my husband, ``Oh Sarak,{\dots}'' '
\z

The `\textstyleEmphasizedWords{\textsc{saying verbs}}' described in this section below include three, or four, verbs that between them divide the semantic area of `tell/say/speak/think'. They are frequently used as frame verbs in quote formulas, but they have other functions as well.

\begin{table}
\caption{Please provide a caption}
\label{} 
\begin{tabular}{ll}
\mytoprule
maak-/naak- &`tell'\\
ma- &`say/speak'\\
na- &`say/speak/think'\\
\mybottomrule
\end{tabular}

\end{table}

The verb \textstyleStyleVernacularWordsItalic{maak}- `tell' is used in the same two main senses as its English equivalent: telling someone \textstyleEmphasizedWords{\textsc{about}} something \REF{ex:3:x312} and telling someone \textstyleEmphasizedWords{\textsc{to do}} something \REF{ex:3:x313}. In direct quote formulas it is used mainly preceding a quote \REF{ex:3:x314}, not directly following it as a short closing formula. It is not used in indirect quotes at all.

\ea%x312
\label{ex:3:x312}
\gll Ne \textstyleEmphasizedVernacularWords{maak-e-mik}, ``Ifa yia keraw-i-ya nain, {\dots''} \\
and tell-\textsc{pa}-1/3p snake 1p.\textsc{acc} bite-Np-\textsc{pr}.3s that1\\
\glt`And they told him, ``When a snake bites us, {\dots}'' '
\z

\ea%x313
\label{ex:3:x313}
\gll Moma yia \textstyleEmphasizedVernacularWords{maak-i-mik}. \\
taro 1p.\textsc{acc} tell-Np-\textsc{pr}.1/3p \\
\glt`They are telling us (to get them) taro roots.'
\z

\ea%x314
\label{ex:3:x314}
\gll Efa\textstyleFreeTranslationChar{ } \textstyleEmphasizedVernacularWords{maak-ek}\textstyleFreeTranslationChar{, } ``Opora tep=pa wu-e.'' \\
\textstyleFreeTranslationChar{1s.\textsc{acc} tell-\textsc{pa}-3s talk tape.recorder=\textsc{loc} put-\textsc{imp}.2s}\\
\glt`She told me, ``Put the talk on a tape recorder.'' '
\z

When \textstyleStyleVernacularWordsItalic{maak}- closes a direct quote, it requires the manner adverb \textstyleStyleVernacularWordsItalic{naap} `thus' to precede it:

\ea%x315
\label{ex:3:x315}
\gll ``Aaw-ep p-ekap-eka,'' \textstyleEmphasizedVernacularWords{naap} yia \textstyleEmphasizedVernacularWords{maak-em-ik-e-mik}.\\
get-\textsc{ss}.\textsc{seq} BPx-come-\textsc{imp}.2p thus 1p.\textsc{acc} tell-\textsc{ss}.\textsc{sim}-be-\textsc{pa}-1/3p\\
\glt` ``Bring it'', they were telling us like that.'
\z

The default object for \textstyleStyleVernacularWordsItalic{maak}- is the addressee \REF{ex:3:} and a possible second object is the speech itself \REF{ex:3:}.

\ea%x316
\label{ex:3:x316}
\gll [Wadol opora]\textsubscript{O} [yia]\textsubscript{O} \textstyleEmphasizedVernacularWords{maak-i-n}. \\
lie talk 1p.\textsc{acc} tell-Np-\textsc{pr}.2s \\
\glt`You are telling us lies.'
\z

The status of the verb \textstyleStyleVernacularWordsItalic{naak}- is unclear. It is infrequent, and in natural texts only occurs in closing formulas \REF{ex:3:x317}. It may have developed as an analogy to the verb pair \textstyleStyleVernacularWordsItalic{ma}-/\textstyleStyleVernacularWordsItalic{na}-.

\ea%x317
\label{ex:3:x317}
\gll ``No bom fain=iw mera kuum-e,'' \textstyleEmphasizedVernacularWords{naak-e-mik}. \\
2s.\textsc{unm} bomb this=\textsc{inst} fish burn-\textsc{imp}.2s tell-\textsc{pa}-1/3p \\
\glt` ``Blast fish with this bomb,'' they told him.'
\z

With the verb \textstyleStyleVernacularWordsItalic{ma}- `say/speak/tell' the addressee is not in focus, and is hardly ever even mentioned. Instead, the verb requires either an object referring to the speech content \REF{ex:3:x318} or an adverb \textstyleStyleVernacularWordsItalic{naap} `thus' \REF{ex:3:x319} preceding the verb, or a quote complement following it \REF{ex:3:x320}. 

\ea%x318
\label{ex:3:x318}
\gll Yo yena yaaya ifa ku-o-k nain opora \textstyleEmphasizedVernacularWords{ma-i-yem.}\\
1s.\textsc{unm} 1s.\textsc{gen} 1s/p.uncle snake bite-\textsc{pa}-3s that1 talk say-Np-\textsc{pr}.1s\\
\glt`I am telling a story about my uncle that was bitten by a snake.'
\z

\ea%x319
\label{ex:3:x319}
\gll Momora, no naap me \textstyleEmphasizedVernacularWords{ma-e.} \\
Fool 2s.\textsc{unm} thus not say-\textsc{imp}.2s \\
\glt`Fool, don't say like that.'
\z

\ea%x320
\label{ex:3:x320}
\gll En-e-mik na{\footnotemark} \textstyleEmphasizedVernacularWords{ma-e-mik}, ``Eliwa, aara oposia saarik.'' \\
eat-\textsc{pa}-1/3p \textsc{add} say-\textsc{pa}-1/3p good hen meat like\\
\glt`They ate it and said, ``It is good, like chicken meat.'' '
\z

\footnotetext{Tok Pisin \textit{na} `and' is increasingly used instead of the vernacular additive connective \textit{ne}.}

Occasionally the verb can occur without any of the above objects:

\ea%x321
\label{ex:3:x321}
\gll Yena oram \textstyleEmphasizedVernacularWords{ma-i-yem}. \\
1s.\textsc{gen} just say-Np-\textsc{pr}.1s \\
\glt`I'm just speaking (without any reason ).'
\z

The difference between the verbs \textstyleStyleVernacularWordsItalic{maak}- and \textstyleStyleVernacularWordsItalic{ma}- in regard to the semantic role of a person object is shown clearly in the next example:

\ea%x322
\label{ex:3:x322}
\gll Naap \textstyleEmphasizedVernacularWords{yi}\textstyleEmphasizedVernacularWords{a} \textstyleEmphasizedVernacularWords{ma-i-}\textstyleEmphasizedVernacularWords{kuan} na-ep yo ariman  \textstyleEmphasizedVernacularWords{nefa} \textstyleEmphasizedVernacularWords{maak-i-yem}.
\\
thus 1p.\textsc{acc} say-Np-\textsc{fu}.3p think-\textsc{ss}.\textsc{seq} 1s.\textsc{unm} openly 2s.\textsc{acc} tell-Np-\textsc{pr}.1s\\
\glt`Thinking that they will \textstyleEmphasizedWords{\textsc{say}} like that \textstyleEmphasizedWords{\textsc{about us}} I'm openly \textstyleEmphasizedWords{\textsc{telling you}} (this).'
\z

The verb \textstyleStyleVernacularWordsItalic{na}- `say/speak/call/think' is the most interesting of the speech verbs. In quote formulas it is only used for closing the quote \REF{ex:3:x323}, with or without another utterance verb in an opening formula.

\ea%x323
\label{ex:3:x323}
\gll {\dots}\textstyleEmphasizedVernacularWords{ma-em-ik-e-mik}, ``Oo, {\dots}'' \textstyleEmphasizedVernacularWords{na-em-ik-e-mik}. \\
{\dots}say-\textsc{ss}.\textsc{sim}-be-\textsc{pa}-1/3p oh ... say-\textsc{ss}.\textsc{sim}-be-\textsc{pa}-1/3p \\
\glt`...they kept saying, ``Oh...'', they kept saying (like that).'
\z

\ea%x942
\label{ex:3:x942}
\gll Amerika fan ``Epa eliwa'' \textstyleEmphasizedVernacularWords{nae-ekap-e-mik}. \\
America here time good say-come-\textsc{pa}-1/3p\\
\glt`The Americans came saying ``peace''.'
\z

In a Tail-Head type construction (\sectref{sec:8.2.3.5}) it is often used as a generic verb to replace another utterance verb, when normally the first verb would be repeated.\footnote{Other types of verbs, when not repeated in a Tail-Head construction, are replaced with the generic verb \textstyleFootnoteBaseChar{\textit{on-}} `do'.} 

\ea%x324
\label{ex:3:x324}
\gll Wia \textstyleEmphasizedVernacularWords{maak-e-mik}, ``Yia uf-om-aka.'' \textstyleEmphasizedVernacularWords{Na-iwkin}{\dots}\\
3p.\textsc{acc} tell-\textsc{pa}-1/3p 1p.\textsc{acc} dance-\textsc{ben}-\textsc{bnfy}2.\textsc{imp}.2p say-2/3p.\textsc{ds}\\
\glt`They told them, ``Dance for us.'' When they said (that){\dots}'
\z

When \textstyleStyleVernacularWordsItalic{na}- replaces another utterance verb in that way, the replaced verb may influence what semantic argument becomes the object. In \REF{ex:3:x325} \textstyleStyleVernacularWordsItalic{maak}- requires the addressee of the verb as the default object, and in the following sentence with \textstyleStyleVernacularWordsItalic{na}- the same accusative pronoun \textstyleStyleVernacularWordsItalic{wia}\textit{ }still refers to the addressees, even if with \textstyleStyleVernacularWordsItalic{na}- it would normally refer to the people spoken about.

\ea%x325
\label{ex:3:x325}
\gll Ekap-emi \textstyleEmphasizedVernacularWords{wia} \textstyleEmphasizedVernacularWords{maak-e-mik}, ``Maa iiw-eka.'' \textstyleEmphasizedVernacularWords{Wia} \textstyleEmphasizedVernacularWords{na-iwkin} ma-e-mik, ...\\
come-\textsc{ss}.\textsc{sim} 3p.\textsc{acc} tell-\textsc{pa}-1/3p food dish.out-\textsc{imp}.2p 3p.\textsc{acc} say-2/3p.\textsc{ds} say-\textsc{pa}-1/3p\\
\glt`They\textsubscript{i} came and told them\textsubscript{j}, ``Dish out food.'' They\textsubscript{i} said to them\textsubscript{j} like that and they\textsubscript{j} said, {\dots}'
\z

The verb \textstyleStyleVernacularWordsItalic{na}- is also used in a somewhat different sense `call (by some name)'. In \REF{ex:3:x326} the speaker tells that the word used by the Japanese soldiers for `coconut' was \textstyleStyleVernacularWordsItalic{yasi}, a foreign word for her.\footnote{The verb \textit{unuf}- is used when the calling by name or giving a name is emphasized.}

\ea%x326
\label{ex:3:x326}
\gll Iwera ``yasi'' yia \textstyleEmphasizedVernacularWords{na-em-ik-e-mik}. \\
coconut yasi 1p.\textsc{acc} say-\textsc{ss}.\textsc{sim}-be-\textsc{pa}-1/3p\\
\glt`They kept calling coconut (by the name) ``yasi'' to us.' 
\z

The ``speaking'' expressed by \textstyleStyleVernacularWordsItalic{na}- can also be internal speech, i.e. thinking \REF{ex:3:x327}. This characteristic is quite common to speech verbs in Papuan languages. When the thinking \textstyleEmphasizedWords{\textsc{process}} itself is more in focus, an adjunct plus verb construction \textstyleStyleVernacularWordsItalic{kema} \textstyleStyleVernacularWordsItalic{suuw}- `think' (literally: `push the liver') is used.

\ea%x327
\label{ex:3:x327}
\gll Maa eliwa=ke \textstyleEmphasizedVernacularWords{na-ep} aaw-e-m. \\
thing good=\textsc{cf} say-\textsc{ss}.\textsc{seq} get-\textsc{pa}-1s \\
\glt`I thought it was a good thing and got it.'
\z

Related to the inner speech is another usage typical of verbs for `saying' in Papuan languages: to convey desire, intention or plan. For this function only the same subject sequential form\textit{} \textstyleStyleVernacularWordsxiiptItalic{naep} is used, and the verb that indicates the desired or intended action is in a preceding speech complement clause. This is discussed more fully in the section on complements of utterance verbs (\sectref{sec:8.3.2.1}). 

\ea%x328
\label{ex:3:x328}
\gll [Yo manina urup-i-nen] \textstyleEmphasizedVernacularWords{na-ep}. \\
1s.\textsc{unm} garden ascend-Np-\textsc{fu}.1s say-\textsc{ss}.\textsc{seq} \\
\glt`I want to go to the garden.'
\z

\ea%x329
\label{ex:3:x329}
\gll [Irak-u] \textstyleEmphasizedVernacularWords{na-ep} ikiw-e-mik. \\
fight-\textsc{imp}.1d say-\textsc{ss}.\textsc{seq} go-\textsc{pa}-1/3p\\
\glt`They went to fight.' (Lit: ` ``Let's fight'' they said/thought and went.')
\z

\ea%x1608
\label{ex:3:x1608}
\gll [Ununa owowa p-or-owa] \textstyleEmphasizedVernacularWords{na-ep} maa eno-wa maneka on-i-kuan.\\
slit.gong village Bpx-descend-\textsc{nmz} say-\textsc{ss}.\textsc{seq} food eat-\textsc{nmz} big make-Np-\textsc{fu}.3p\\
\glt`When they want to take the slit gong down to the village they make a big feast.'
\z

In this function \textstyleStyleVernacularWordsxiiptItalic{naep} is becoming less like a regular medial verb. It can occur in sentence-final position, without being right-dislocated \REF{ex:3:}. It usually does retain its word stress, but there is a tendency to un-stress and shorten it by dropping the vowel /a/ in speech \REF{ex:3:}. When the verb in the speech complement clause is in the counterfactual form, all that is sometimes left of \textstyleStyleVernacularWordsItalic{na-ep} is only the suffix, which is then added as a suffix to the other verb \REF{ex:3:}.

\ea%x1830
\label{ex:3:x1830}
\gll Ifana wu-am-ika-i-kuan, [unuma wia miim-u] \textstyleEmphasizedVernacularWords{n}\textstyleEmphasizedVernacularWords{-}\textstyleEmphasizedVernacularWords{ep}.\\
ear put-\textsc{ss}.\textsc{sim}-be-Np-\textsc{fu}.3p name 3p.\textsc{acc} hear-1d.\textsc{imp} say-\textsc{ss}.\textsc{seq}\\
\glt`They\textsubscript{i} are listening carefully (lit: putting their ear), wanting to hear their\textsubscript{j} names.'
\z

\ea%x1609
\label{ex:3:x1609}
\gll Yo aakisa nanar nain \textstyleEmphasizedVernacularWords{ma-ek-a-m-{\O}-ep}. \\
1s.\textsc{unm} now story that1 say-\textsc{cntf}-\textsc{pa}-1s-{\O}-\textsc{ss}.\textsc{seq}\\
\glt`Now I would like to tell that story.'
\z

The verb \textstyleStyleVernacularWordsItalic{na}- quite freely combines with sound words, and a number of these combinations have been lexicalized \REF{ex:3:}, \REF{ex:3:}. The onomatopoeic word has become part of the verb, and the vowel /a/ has been deleted from the verb in the process.

\ea%x330
\label{ex:3:x330}
\gll Oro-mi \textstyleEmphasizedVernacularWords{bulak} \textstyleEmphasizedVernacularWords{na-i-ya}\textstyleEmphasizedVernacularWords{\textmd{\textit{.}}} \\
drop-\textsc{ss}.\textsc{sim} plop say-Np-\textsc{pr}.3s \\
\glt`When it drops it says ``plop''.'
\z

\ea%x331
\label{ex:3:x331}
\gll Siowa \textstyleEmphasizedVernacularWords{baun-i-ya}. ({{\textless}} bau na-i-ya) \\
dog bark-Np-\textsc{pr}.3s ( bau say-Np-\textsc{pr}.3s)\\
\glt`The dog barks.'
\z

\ea%x332
\label{ex:3:x332}
\gll Ema \textstyleEmphasizedVernacularWords{buun-eya} mua erup um-e-mik. ({{\textless}} buu na-eya) \\
mountain erupt-2/3s.\textsc{ds} man two die-\textsc{pa}-1/3p ( buu say-2/3s.\textsc{ds})\\
\glt`The mountain (=volcano) erupted and two men died.'
\z

In fast speech \textstyleStyleVernacularWordsItalic{na}- is often reduced to \textstyleStyleVernacularWordsItalic{a}- when the verb follows a consonant-final word.

\ea%x333
\label{ex:3:x333}
\gll ``Uruf-a-mik'' \textstyleEmphasizedVernacularWords{a-e-k}. \\
see-\textsc{pa}-1/3p say-\textsc{pa}-3s \\
\glt` ``They saw it,'' he said.'
\z

The medial form \textstyleStyleVernacularWordsItalic{na-eya} is also used as resultative connective `so, therefore' (\sectref{sec:3.11.2}).

\ea%x500
\label{ex:3:x500}
\gll Iwera yia na-em-ik-e-mik. \textstyleEmphasizedVernacularWords{Naeya} iwera wia uruk-am-ik-om-a-mik. \\
coconut 1p.\textsc{acc} say-\textsc{ss}.\textsc{sim}-be-\textsc{pa}-1/3p So coconut 3p.\textsc{acc} drop-\textsc{ss}.\textsc{sim}-be-\textsc{ben}-\textsc{bnfy}2.\textsc{pa}-1/3p\\
\glt`They kept speaking to us about coconuts /asking us for coconuts. So we kept dropping coconuts for them.'
\z

\paragraph{Impersonal experience verbs}\label{sec:3:a:z:y:x}
%\hypertarget{RefHeading20881935131865}
{}
This very small group mainly consists of verbs indicating some kind of pain. They look like transitive verbs, but the syntactic subject is inanimate, usually a body part, and the human experiencer is the object. 

\begin{table}
\caption{Please provide a caption}
\label{} 
\begin{tabular}{ll}
\mytoprule
gilin- &`smart (v.)'\\
kokas- &`itch'\\
liilin- &`sting'\\
tiitin- &`hurt, ache (generic)'\\
tukun- &`throb'\\
sirir- &`ache'\\
\mybottomrule
\end{tabular}

\end{table}

\ea%x1013
\label{ex:3:x1013}
\gll Maara efa \textstyleEmphasizedVernacularWords{tiitin-i-ya}. \\
forehead 1s.\textsc{acc} hurt-Np-\textsc{pr}.3s\\
\glt`My head hurts.'/ `I have a headache.' (Lit: `It hurts my forehead.')
\z

\ea%x1014
\label{ex:3:x1014}
\gll Uuw-ap uuw-ap oona=ke efa \textstyleEmphasizedVernacularWords{sirir-i-ya}. \\
work-\textsc{ss}.\textsc{seq} work-\textsc{ss}.\textsc{seq} bone=\textsc{cf} 1s.\textsc{acc} ache-Np-\textsc{pr}.3s\\
\glt`I have worked and worked, and my bones ache.'
\z

Most of the experience verbs in Mauwake are adjunct plus verb constructions (\sectref{sec:3.8.5.2.1}); a few are ordinary intransitive verbs (\sectref{sec:3.8.4.2.1}). 

\subsubsection{Auxiliary verbs}\label{sec:3:z:y:x}
%\hypertarget{RefHeading20901935131865}
{}
The small group of auxiliary verbs in Mauwake consists of two ordinary verbs that have also grammaticalized as auxiliaries indicating aspect. In this function the lexical meaning of the verbs is somewhat bleached. The auxiliary is the last verb in a verbal group (\sectref{sec:3.8.5.1}). 

The paradigms of the auxiliaries are similar to those of main verbs. The auxiliary verbs are:

\begin{table}
\caption{Please provide a caption}
\label{} 
\begin{tabular}{lll}
\mytoprule
\textstyleAcronymallcaps{AUX:} &\textstyleAcronymallcaps{MEANING:} &\textstyleAcronymallcaps{MAIN VERB FORM:}\\
\midrule
ik- &`continuous' &\textstyleAcronymallcaps{SS.SIM}\\
 &`stative' &\textstyleAcronymallcaps{\textsc{ss}.\textsc{seq}}\\
pu- ({{\textless}}wu-) &`completive' &\textstyleAcronymallcaps{\textsc{ss}.\textsc{seq}}\\
\mybottomrule
\end{tabular}

\end{table}

The auxiliary \textstyleStyleVernacularWordsItalic{ik}- is very frequent and has several functions. When it is used with a main verb in the same-subject simultaneous form (\textstyleAcronymallcaps{SS.SIM}), it indicates continuous aspect, which can have either progressive \REF{ex:3:x339} or habitual \REF{ex:3:x340} meaning. For position-taking verbs (\sectref{sec:3.8.4.4.2}) and other semantically punctiliar verbs the habitual interpretation is the only possible one, but for other verbs the context is needed to determine the correct interpretation. 

\ea%x339
\label{ex:3:x339}
\gll Fikera aw-em-\textstyleEmphasizedVernacularWords{ik}-eya uruf-a-k. (progressive) \\
kunai.grass burn-\textsc{ss}.\textsc{sim}-be-2/3s.\textsc{ds} see-\textsc{pa}-3s \\
\glt`When the kunai grass was burning she saw it.' (Or: `She saw the kunai grass burning.')
\z

\ea%x340
\label{ex:3:x340}
\gll I yabuela aaw-ep {\dots} wi-em-\textstyleEmphasizedVernacularWords{ik}-e-mik. (habitual) \\
1p.\textsc{unm} papaya get-\textsc{ss}.\textsc{seq} {\dots} give.them-\textsc{ss}.\textsc{sim}-be-\textsc{pa}-1/3p \\
\glt`We kept getting papayas and {\dots} giving them to them.'
\z

When the main verb is in the same-subject sequential form (\textstyleAcronymallcaps{\textsc{ss}.\textsc{seq}}), the auxiliary \textstyleStyleVernacularWordsItalic{ik}- indicates stativity \REF{ex:3:x341}. With non-punctiliar verbs this form can often be translated into English with a past perfect \REF{ex:3:x342}.

\ea%x341
\label{ex:3:x341}
\gll Pok-ap-\textstyleEmphasizedVernacularWords{ik}-emkun epa wiim-o-k. (stative) \\
sit.down-\textsc{ss}.\textsc{seq}-be-1s/p.\textsc{ds} place dawn-\textsc{pa}-3s \\
\glt`As I was sitting it became light.'
\z

\ea%x342
\label{ex:3:x342}
\gll Ikiw-ep-\textstyleEmphasizedVernacularWords{ik}-eya ona emeria=ke ekap-o-k. (perfect) \\
go-\textsc{ss}.\textsc{seq}-be-2/3s.\textsc{ds} 3s.\textsc{gen} woman=\textsc{cf} come-\textsc{pa}-3s \\
\glt``After he was/had gone his wife came.'
\z

The auxiliary \textstyleStyleVernacularWordsItalic{pu}- `completive', is obviously derived from \textstyleStyleVernacularWordsItalic{wu}- `put'\footnote{`Put' is one of the verbs commonly used in Papuan languages to indicate completion \citep[145]{Foley1986}.} through assimilation with the final /p/ of the same-subject sequential form in the main verb preceding it. Synchronically, the Mauwake speakers do not recognise the origin of the auxiliary.

\ea%x343
\label{ex:3:x343}
\gll Maa en-ep-\textstyleEmphasizedVernacularWords{pu}-ap soomar-eka. \\
food eat-\textsc{ss}.\textsc{seq}-\textsc{cmpl}-\textsc{ss}.\textsc{seq} walk-\textsc{imp}.2p \\
\glt`Having finished eating you may go.' (Lit: `Eat the food and go'.)
\z

\ea%x501
\label{ex:3:x501}
\gll Nan efa wu-ap-\textstyleEmphasizedVernacularWords{pu}-ami o Ulingan ikiw-o-k. \\
there1 1s.\textsc{acc} put-\textsc{ss}.\textsc{seq}-\textsc{cmpl}-\textsc{ss}.\textsc{sim} 3s.\textsc{unm} Ulingan go-\textsc{pa}-3s\\
\glt`He left (lit: put) me there and went to Ulingan.'
\z

\subsection{Verbal clusters}\label{sec:3:y:x}
%\hypertarget{RefHeading20921935131865}
{}
The verbal clusters are described here under verb morphology, because they function as a unit very much like single verbs. There are two kinds of verbal clusters: verbal groups and adjunct plus verb constructions. The definition of a verbal group is from \citet[175]{Halliday1985}: ``a sequence of words in the primary class of verb''. 

\ea%x347
\label{ex:3:x347}
\gll Ifara \textstyleEmphasizedVernacularWords{mokak-ikiw-em-ik-ok} ifara oko uruf-a-k. \\
vine stare-go-\textsc{ss}.\textsc{sim}-be-SS vine other see-\textsc{pa}-3s\\
\glt`He kept looking for a vine and saw one vine.'
\z

Adjunct plus verb combinations\footnote{\citet[184]{Halliday1985} calls these ``phrasal verbs''.} contain a verb (or a verbal group) plus an element from another word class that is obligatory and contributes to the meaning of the verb. 

\ea%x348
\label{ex:3:x348}
\gll Owora efar \textstyleEmphasizedVernacularWords{ikum} \textstyleEmphasizedVernacularWords{aaw-iwkin} wia maak-e-m. \\
betelnut 1s.\textsc{dat} illicitly get-2/3p.\textsc{ds} 3p.\textsc{acc} tell-\textsc{pa}-1s\\
\glt`They stole my betelnut and I talked to them.'
\z

The status of a verb phrase in Mauwake is somewhat questionable. It is discussed in \sectref{sec:4.5}. 

\subsubsection{Verbal groups}\label{sec:3:z:y:x}
%\hypertarget{RefHeading20941935131865}
{}
A verbal group consists of two or more verbs that function grammatically and semantically as one unit. The semantic unity within the group varies between different types of verbal groups. 

The verbal groups containing a main verb plus auxiliary have developed by merging clauses as can still be seen from the verbs involved. But since they synchronically function as a unit very much like an individual verb they are treated on the word level. Features that identify them as one close-knit unit are as follows:

\begin{itemize}
\item Shared subject (and object, if relevant)
\item No non-verbal elements intervening between the parts
\item Scope of negation spans over the whole group
\item No coordinators are allowed between the parts
\item Phrasal intonation and pause structure, i.e. no pauses between the words.
\end{itemize}

Mauwake has two kinds of verbal groups. The verbs in the first group consist of a main verb and an aspectual auxiliary. The second group consists of serial verbs, where all the verb stems contribute to the semantic, rather than grammatical, meaning of the verb.

\paragraph{Main verb plus auxiliary: aspect}\label{sec:3:a:z:y:x}
%\hypertarget{RefHeading20961935131865}
{}
The importance of tense as a verbal category in Mauwake shows in its obligatory morphological marking, but aspect is a relatively important category as well. Aspects are `{different ways of viewing the internal temporal constituency of a situation}' \citep[3]{Comrie1976}. 

Aspect in Mauwake is expressed periphrastically, through verbal groups that have a main verb and an auxiliary. The main verb, which is in the medial form, largely gives the semantic content to the whole, and the auxiliary adds the grammatical meaning of aspect. In the continuous and stative aspects also the medial form of the \textstyleEmphasizedWords{\textsc{main}} verb contributes to the aspectual meaning. What distinguishes these constructions from medial clauses (\sectref{sec:8.2}) is that the two verbs function as a unit rather than individual verbs, and their phonological stress, intonation and pause pattern is that of a word or phrase rather than a medial clause. 

 As is typical of \textstyleAcronymallcaps{\textup{SOV}} languages, the auxiliary follows the main verb (\citealt[85]{Greenberg1966}; \citealt[90]{Dryer2007a}). The more common of the aspectual auxiliaries is \textstyleStyleVernacularWordsxiiptItalic{ik}- `be', which can combine with two different medial forms. The other aspectual auxiliary is \textstyleStyleVernacularWordsxiiptItalic{pu}- `completive' (\sectref{sec:3.8.4.5}). 

The neutral, aspectually unmarked verb form is used in Mauwake whenever the speaker chooses not to pay special attention to the internal structure of the situation. It could be claimed that this is a neutral perfective, since the situation is viewed as a whole, but that term would be confusing, as the neutral forms can also be used in clauses that are aspectually habitual \citep[cf.][239]{Payne1997}. The majority of the verb forms used in all kinds of texts in Mauwake are aspectually neutral.

The marked completive aspect is only used when completion of an action is stressed. The continuous aspect is used for both progressive and habitual actions, and the stative aspect for a state continuing over some time.

{\bfseries
%\hypertarget{RefHeading20981935131865}
{}
Completive aspect}

When the \textstyleEmphasizedWords{\textsc{completion}} of an action is in focus, the completive aspect is used. It is formed by a main verb in the same-subject sequential form, followed by the auxiliary \textstyleStyleVernacularWordsItalic{pu}- `completive' (\sectref{sec:3.8.4.5}).

\ea%x361
\label{ex:3:x361}
\gll Ifakim-ep nomokow ekeka=pa \textstyleEmphasizedVernacularWords{sererim-ep-pu-a-k}. \\
kill-\textsc{ss}.\textsc{seq} tree branch=\textsc{loc} hang-\textsc{ss}.\textsc{seq}-\textsc{cmpl}-\textsc{pa}-3s \\
\glt`He killed it and hung it on a tree branch.'
\z

The completive aspect verb is often used in a medial same-subject sequential form, which in itself only indicates sequentiality but often implies completion of the first action as well. 

\ea%x362
\label{ex:3:x362}
\gll \textstyleEmphasizedVernacularWords{Sererim-ep-pu-ap} owowa or-o-k. \\
hang-\textsc{ss}.\textsc{seq}-\textsc{cmpl}-\textsc{ss}.\textsc{seq} village descend-\textsc{pa}-3s\\
\glt`He hung it up and went/came down to the village.'
\z

\ea%x1041
\label{ex:3:x1041}
\gll Manina \textstyleEmphasizedVernacularWords{n}\textstyleEmphasizedVernacularWords{op-ap-pu-ap} nomokowa war-i-mik. \\
garden burn-\textsc{ss}.\textsc{seq}-\textsc{cmpl}-\textsc{ss}.\textsc{seq} tree cut-Np-\textsc{pr}.1/3p\\
\glt`We burn (the undergrowth for new) garden and (when it is done we) cut the trees.'
\z

\ea%x1042
\label{ex:3:x1042}
\gll Nomokowa \textstyleEmphasizedVernacularWords{war-ep-pu-ap} arew-i-mik. \\
tree cut-\textsc{ss}.\textsc{seq}-\textsc{cmpl}-\textsc{ss}.\textsc{seq} wait-Np-\textsc{pr}.1/3p\\
\glt`We cut the trees and wait.'
\z

But it is not uncommon either to have the completive aspect with a simultaneous action medial form, when the second action coincides with the completion of the first one:

\ea%x363
\label{ex:3:x363}
\gll Wia \textstyleEmphasizedVernacularWords{maak-ep-pu-ami} i ikiw-e-mik. \\
3p.\textsc{acc} tell-\textsc{ss}.\textsc{seq}-\textsc{cmpl}-\textsc{ss}.\textsc{sim} 1p.\textsc{unm} go-\textsc{pa}-1/3p\\
\glt`We told them and went.' 
\z

\ea%x1040
\label{ex:3:x1040}
\gll Aria yo nan efa \textstyleEmphasizedVernacularWords{wu-ap-pu-ami} o Ulingan ikiw-o-k.\\
alright 1s.\textsc{unm} there 1s.\textsc{acc} put-\textsc{ss}.\textsc{seq}-\textsc{cmpl}-\textsc{ss}.\textsc{sim} 3s.\textsc{unm} Ulingan go-\textsc{pa}-3s\\
\glt`Alright he put me there and he went to Ulingan.'
\z

\ea%x1043
\label{ex:3:x1043}
\gll Maa en-owa \textstyleEmphasizedVernacularWords{wakesim-ep-pu-ami} ikiw-o-k. \\
thing eat-\textsc{nmz} cover-\textsc{ss}.\textsc{seq}-\textsc{cmpl}-\textsc{ss}.\textsc{sim} go-\textsc{pa}-3s\\
\glt`Covering the food she left.'
\z

The completive aspect form is also used when \textstyleEmphasizedWords{\textsc{momentaneity}} of the action is emphasized:

\ea%x364
\label{ex:3:x364}
\gll \textstyleEmphasizedVernacularWords{En-ep-pu-ap} ikiw-e! \\
eat-\textsc{ss}.\textsc{seq}-\textsc{cmpl}-\textsc{ss}.\textsc{seq} go-\textsc{imp}.2s\\
\glt`Get done with your eating and go!'
\z

The origin of the auxiliary, the verb `put', shows in the fact that it cannot be used with non-controlled actions.\footnote{In general, control vs. non-control is not a prominent feature in the verb system in Mauwake, unlike many other Papuan languages (\citealt[127]{Foley1986}, \citealt[128]{Reesink1987}).}

\ea%x365
\label{ex:3:x365}
\gll *Waki-ep-pu-a-k \\
fall-\textsc{ss}.\textsc{seq}-\textsc{cmpl}-\textsc{pa}-3s\\
\glt
\z

In process descriptions a medial verb, followed by the verb \textstyleStyleVernacularWordsItalic{weeser}- `finish', which stresses the endpoint of the action, is used more than the completive aspect. This, however, is a case of clause chaining (\sectref{sec:8.2}), not a verbal group.

\ea%x366
\label{ex:3:x366}
\gll Uup-ep \textstyleEmphasizedVernacularWords{weeser-eya} wienak-e-m. \\
cook-\textsc{ss}.\textsc{seq} finish-2/3s.\textsc{ds} feed.them-\textsc{pa}-1s\\
\glt`I finished cooking it and fed it to them.' [Lit: `I cooked it and when it (=the cooking) was finished I fed it to them.']
\z

{\bfseries
%\hypertarget{RefHeading21001935131865}
{}
Continuous aspect: progressive and habitual}

Continuity, or duration, is the semantic component shared by the aspects called progressive and habitual in many languages: continuation of the same action or of repeated actions of the same kind \citep[26]{Comrie1986}. The continuous aspect form in Mauwake can have either progressive \REF{ex:3:}, \REF{ex:3:} or habitual \REF{ex:3:}, \REF{ex:3:} interpretation. The main verb is in the same-subject simultaneous medial form, but with the final /i/ deleted, and the auxiliary \textstyleStyleVernacularWordsItalic{ik}- `be' is inflected for tense and person/number \REF{ex:3:}. 

\ea%x349
\label{ex:3:x349}
\gll Maa \textstyleEmphasizedVernacularWords{en-em-ik-omkun} ama or-o-k. \\
food eat-\textsc{ss}.\textsc{sim}-be-1s/p.\textsc{ds} sun descend-\textsc{pa}-3s \\
\glt`As I was eating the sun went down.'
\z

\ea%x1044
\label{ex:3:x1044}
\gll Fikera \textstyleEmphasizedVernacularWords{aw-em-ik-eya} nain umuk-i-nen na-ep urup-o-k.\\
kunai.grass burn-\textsc{ss}.\textsc{sim}-be-2/3s.\textsc{ds} that1 extinguish-Np-\textsc{fu}.1s say-\textsc{ss}.\textsc{seq} ascend-\textsc{pa}-3s\\
\glt`The kunai grass was burning, and she went up in order to extinguish it.'
\z

\ea%x350
\label{ex:3:x350}
\gll Iwera=ke wia aruf-eya \textstyleEmphasizedVernacularWords{ma-em-ik-e-mik}, ``{\dots''} \\
coconut=\textsc{cf} 3p.\textsc{acc} hit-2/3s.\textsc{ds} say-\textsc{ss}.\textsc{sim}-be-\textsc{pa}-1/3p \\
\glt`When coconuts hit them, they kept saying, `` ...'' '
\z

\ea%x1045
\label{ex:3:x1045}
\gll Wi Yaapan naap kuisow=iw \textstyleEmphasizedVernacularWords{ekap-em-ik-e-mik}. \\
3p.\textsc{unm} Japan thus one=\textsc{inst} come-\textsc{ss}.\textsc{sim}-be-\textsc{pa}-1/3p\\
\glt`The Japanese kept coming like that, one by one.'
\z

For punctiliar verbs the habitual interpretation \REF{ex:3:} is the only one possible, whereas for non-punctiliar verbs both habitual and progressive interpretations are possible.

\ea%x351
\label{ex:3:x351}
\gll Koka=pa nan \textstyleEmphasizedVernacularWords{in-em-ik-e-mik.} \\
jungle=\textsc{loc} there lie.down-\textsc{ss}.\textsc{sim}-be-\textsc{pa}-1/3p\\
\glt`We kept sleeping in the jungle'
\z

\ea%x1932
\label{ex:3:x1932}
\gll Owowa oko wiam=iya \textbf{irak-em-ik-e-mik}. \\
village other 3p.\textsc{acc}=\textsc{com} fight-\textsc{ss}.\textsc{sim}-be-\textsc{pa}-1/3p\\
\glt`We were fighting (or: kept fighting repeatedly) with the other village.'
\z

 When the verbal group is in the medial form, the progressive interpretation \REF{ex:3:x353} is the more common:

\ea%x353
\label{ex:3:x353}
\gll Waaya \textstyleEmphasizedVernacularWords{urup-em-ik-eya}  mik-a-m. \\
pig ascend-\textsc{ss}.\textsc{sim}-be-2/3s.\textsc{ds} spear-\textsc{pa}-1s\\
\glt`As the pig was going/coming up I speared it.'
\z

Often the context provides the only clue as to whether the continuous aspect form should be interpreted as progressive or habitual. The example \REF{ex:3:x354} describes a situation where the villagers kept feeding the Japanese soldiers who asked them for food; the sentence \REF{ex:3:x355} is from a text describing a coconut plantation fire and its consequences.

\ea%x354
\label{ex:3:x354}
\gll Waaya yia na-iwkin waaya \textstyleEmphasizedVernacularWords{wienak-em-ik-e-mik}. \\
pig 1p.\textsc{acc} say-2/3p.\textsc{ds} pig feed.them-\textsc{ss}.\textsc{sim}-be-\textsc{pa}-1/3p\\
\glt`They asked us for pigs and we kept giving them pigs to eat.'
\z

\ea%x355
\label{ex:3:x355}
\gll Kawus \textstyleEmphasizedVernacularWords{ir-am-ik-eya} kuuf-a-k. \\
smoke rise-\textsc{ss}.\textsc{sim}-be-2/3s.\textsc{ds} see-\textsc{pa}-3s\\
\glt`The smoke was rising and she saw it.'
\z

Cross-linguistically the habitual aspect more commonly receives overt marking in the past tense than in the present \citep[154]{Cristofaro2006}. In Mauwake the continuous aspect can be used for habitual in any of the three tenses. The example \REF{ex:3:x1063} was said about particular work that the speaker was not involved in continuously; he used to do it time to time because of his position as need arose. The example \REF{ex:3:x1064} refers to a couple needing to keep visiting an ailing father. 

\ea%x1063
\label{ex:3:x1063}
\gll Yo anane maneka naap \textstyleEmphasizedVernacularWords{mauw-am-ika-i-yem}. \\
1s.\textsc{unm} always very thus work-\textsc{ss}.\textsc{sim}-be-Np-\textsc{pr}.1s\\
\glt`I always/forever keep working like that.'
\z

\ea%x1064
\label{ex:3:x1064}
\gll O me sariar-i-non-(na) neeke \textstyleEmphasizedVernacularWords{in-em-ika-i-kuan}. \\
3s.\textsc{unm} not get.well-Np-\textsc{fu}.3s-(TP) there.\textsc{cf} sleep-\textsc{ss}.\textsc{sim}-be-Np-\textsc{fu}.3p\\
\glt`If he doesn't get well, they will keep sleeping/staying \textit{there}.' 
\z

For a clause to have habitual interpretation it is not obligatory to use the continuous aspect form in the verb. For instance in process descriptions, which tell how something is habitually done, the unmarked, aspectually neutral present tense form is more common than the continuous aspect. Three of the four verbs in \REF{ex:3:x1049} are aspectually unmarked, although all the clauses have habitual interpretation, describing seclusion customs.

\ea%x1049
\label{ex:3:x1049}
\gll Moma ik-owa \textstyleEmphasizedVernacularWords{enim-i-mik}. Eka me \textstyleEmphasizedVernacularWords{enim-i-mik}, iwer eka me \textstyleEmphasizedVernacularWords{enim-i-mik}. Aaya muutiw \textstyleEmphasizedVernacularWords{en-em-ika-i-mik}.\\
taro roast-\textsc{nmz} eat-Np-\textsc{pr}.1/3p water not eat-Np-\textsc{pr}.1/3p coconut water not eat-Np-\textsc{pr}.1/3p sugarcane only eat-\textsc{ss}.\textsc{sim}-be-Np-\textsc{pr}.1/3p\\
\glt`We do not eat roasted taro. We do not drink water or coconut water. We only eat / keep eating sugarcane.'
\z

{\bfseries
%\hypertarget{RefHeading21021935131865}
{}
Stative aspect}

The same semantic component of continuity is also shared by the other aspect using the auxiliary \textstyleStyleVernacularWordsxiiptItalic{ik}- `be': this time it is a \textstyleEmphasizedWords{\textsc{state}} rather than activity that continues the same over time. In the stative aspect the auxiliary is combined with a main verb that is in the same-subject sequential form. This usage is most common with the position-taking verbs like \textstyleStyleVernacularWordsItalic{pok}- `sit down', \textstyleStyleVernacularWordsItalic{iimar}- `stand up' and \textstyleStyleVernacularWordsItalic{in}- `lie down/ fall asleep'.

\ea%x356
\label{ex:3:x356}
\gll \textstyleEmphasizedVernacularWords{Pok-ap-ik-omkun}  epa wiim-o-k. \\
sit.down-\textsc{ss}.\textsc{seq}-be-1s/p.\textsc{ds} place dawn-\textsc{pa}-3s \\
\glt`As we were sitting it dawned.'
\z

\ea%x1046
\label{ex:3:x1046}
\gll Yena koor miira=pa \textstyleEmphasizedVernacularWords{iimar-ep-ik-e-m}, {\dots} \\
1s.\textsc{gen} house face=\textsc{loc} stand.up-\textsc{ss}.\textsc{seq}-be-\textsc{pa}-1s\\
\glt`I was standing in front of my house, {\dots}'
\z

Other punctiliar verbs \REF{ex:3:x357}, as well as non-punctiliar verbs can be used in this aspect to indicate the state resulting from an action \REF{ex:3:x358}, or process \REF{ex:3:x1047}, but they are less frequent.

\ea%x357
\label{ex:3:x357}
\gll Ifakim-eya \textstyleEmphasizedVernacularWords{pu-ep-ik-eya}  om-em-ik-ua. \\
kill-2/3s.\textsc{ds} die-\textsc{ss}.\textsc{seq}-be-2/3s.\textsc{ds} cry-\textsc{ss}.\textsc{sim}-be-\textsc{pa}.3s\\
\glt`When she killed him and he was dead, she was crying.'
\z

\ea%x358
\label{ex:3:x358}
\gll \textstyleEmphasizedVernacularWords{Ikiw-ep-ik-eya} ona emeria=ke ekap-o-k. \\
go-\textsc{ss}.\textsc{seq}-be-2/3s.\textsc{ds} 3s.\textsc{gen} woman=\textsc{cf} come-\textsc{pa}-3s\\
\glt`While he was gone his wife came.'
\z

\ea%x1047
\label{ex:3:x1047}
\gll Ewar pun wuun-e-k ne epa \textstyleEmphasizedVernacularWords{reen-ep-ik-ua}. \\
west.wind too blow-\textsc{pa}-3s and place dry-\textsc{ss}.\textsc{seq}-be-\textsc{pa}.3s\\
\glt`The west wind blew, too, and the ground was dry.'
\z

In the example \REF{ex:3:} the continuous form indicates more active waiting process than is the case in \REF{ex:3:} with the stative aspect. In \REF{ex:3:} the people were getting impatient with the vehicle that should already have come to get them. The example \REF{ex:3:} is from a description of garden work, and part of the work process is the state of patiently waiting for the felled trees and undergrowth to dry. 

\ea%x359
\label{ex:3:x359}
\gll Arew\textstyleEmphasizedVernacularWords{-am-}ik-omkun ama ikur miiw-aasa kerer-ek. \\
wait-\textsc{ss}.\textsc{sim}-be-1s/p.\textsc{ds} sun five land-canoe arrive-\textsc{pa}-3s\\
\glt`As we were waiting the car arrived at five.'
\z

\ea%x360
\label{ex:3:x360}
\gll Nomokowa war-ep-pu-ap arew\textstyleEmphasizedVernacularWords{-ap-}ika-iwkin \\
tree cut-\textsc{ss}.\textsc{seq}-\textsc{cmpl}-\textsc{ss}.\textsc{seq} wait-\textsc{ss}.\textsc{seq}-be-2/3p.\textsc{ds}
reen-eya saama kuum-i-mik.
dry-2/3s.\textsc{ds} cleared.bush burn-Np-\textsc{pr}.1/3p\\
\glt`They cut the trees and while they are waiting it dries and then they burn the cleared bush.' 
\z

\paragraph{Serial verbs} \label{sec:3:a:z:y:x}
%\hypertarget{RefHeading21041935131865}
{}
Verbal groups called serial verbs are very common in Papuan languages \citep[116]{Foley1986}. Finding a cross-linguistic definition for serial verbs has proved to be an extremely hard task (\citealt[5]{Sebba1987}, \citealt[1]{Lord1993}). Instead of one definition covering all the possible serial verbs, \citet[19]{Crowley2002} suggests defining these verbs within ``{specific typological and linguogenetic groupings}'' for comparative purposes. 

For a working definition I borrow one given by \citet[28]{James1983} describing the serial verbs in Siane, another Papuan language: \\
``A serial verb construction consists of two or more verbs which occur in series with neither normal coordinating nor subordinating markers, which share at least some core argument (normally subject and/or object/goal), and which in some sense function together semantically like a single predication''. 

Typically, even if not obligatorily, one of the verbs in the series is finite and the other(s) more or less ``stripped-down''. In a verb-final language the finite verb is the last one in the series. After describing the serial verb construction in Mauwake I will discuss the question whether serial verbs are actually compound verbs, and the relationship of the serial verbs to main verb + auxiliary verbal groups and medial clauses.

In Mauwake a non-final verb in a serial construction consists of a bare root without any inflection at all. This restriction is tighter than those given for serial verbs in many other languages (\citealt[19]{Crowley2002}; \citealt[86--87]{Sebba1987}; \citealt[28]{James1983}). Each of the verbs in a serial construction contribute to the overall semantic meaning of the predicate. Even if the meaning is not exactly the same as the combination of the same verbs would have in a tight medial verb chain \citep[cf.][310]{Payne1997}, it does not get bleached either, like that of the auxiliaries.\footnote{Since a serial verb construction has only one main stress it is written as one word in the orthography, but the verb stems are separated by hyphens to make reading easier.}

\ea%x377
\label{ex:3:x377}
\gll Sama=pa \textstyleEmphasizedVernacularWords{oro-boon-ek}. \\
ladder=\textsc{loc} descend-get.loose-\textsc{pa}-3s\\
\glt`He fell from the ladder.'
\z

The last verb in a series is either a finite verb with tense and person/number inflection, or a medial verb. The arguments are shared by the whole verbal complex, even if they would originally have been associated with only one of the verbs \REF{ex:3:x378}. Also negation and obliques \REF{ex:3:x379} are shared. All this points to serial verbs being a nuclear-level phenomenon in Mauwake, rather than a core-level one \citep[189--193]{FoleyEtAl1984}. 

\ea%x378
\label{ex:3:x378}
\gll Yo Amerika wia \textstyleEmphasizedVernacularWords{akup-ikiw-i-yem}. \\
1s.\textsc{unm} America 3p.\textsc{acc} search-go-Np-\textsc{pr}.1s \\
\glt`I am going to look for the Americans. / I go searching the Americans.' 
\z

\ea%x379
\label{ex:3:x379}
\gll Neeke \textstyleEmphasizedVernacularWords{aw(e)-or-om-ik-eya} {\dots} \\
there.\textsc{cf} burn-descend-\textsc{ss}.\textsc{sim}-be-2/3s.\textsc{ds}\\
\glt`As it was burning (towards) down \textit{there}{\dots}'
\z

Semantically the verb combinations are of two types. In the more common one a directional or another motion verb follows another verb stem, giving the meaning of \textstyleEmphasizedWords{\textsc{movement}} to the whole \REF{ex:3:}-\REF{ex:3:}, and often the meaning of \textstyleEmphasizedWords{\textsc{directionality}} as well \REF{ex:3:}-\REF{ex:3:}.\footnote{Cross-linguistically motion and location verbs are very common in serial verbs \citep[9]{Lord1993}.} This is a productive process, as long as the verbs are semantically compatible.

\ea%x438
\label{ex:3:x438}
\gll Wia \textstyleEmphasizedVernacularWords{mokak-urup-o-k}, wia \textstyleEmphasizedVernacularWords{mokak-or-o-k}. \\
3p.\textsc{acc} stare-ascend-\textsc{pa}-3s 3p.\textsc{acc} stare-descend-\textsc{pa}-3s\\
\glt`He stared them up and down.'
\z

\ea%x381
\label{ex:3:x381}
\gll Aasa \textstyleEmphasizedVernacularWords{suuw-or-o-mik}. \\
canoe push-descend-\textsc{pa}-1/3p \\
\glt`We pushed the canoe down (towards the sea).'\footnote{Compare this with a medial construction: \textstyleFootnoteBaseChar{\textit{Aasa suuw-ap or-o-mik}} `We pushed the canoe and went down (to sea)'}
\z

If the first stem is also a motion verb, it indicates the \textstyleEmphasizedWords{\textsc{manner}} of movement:

\ea%x380
\label{ex:3:x380}
\gll Merena kir-ep \textstyleEmphasizedVernacularWords{segen-ikiw-o-k}. \\
foot turn-\textsc{ss}.\textsc{seq} limp-go-\textsc{pa}-3s \\
\glt`He twisted his foot and limped.'
\z

A motion verb in a serial construction can also indicate \textstyleEmphasizedWords{\textsc{temporal continuity}} over a long period of time. In \REF{ex:3:x439} the length of time is emphasized even more by the repetition of the motion verb.

\ea%x439
\label{ex:3:x439}
\gll \textstyleEmphasizedVernacularWords{Ife-iki}(w-e)\textstyleEmphasizedVernacularWords{p} \textstyleEmphasizedVernacularWords{iki}(w-e)\textstyleEmphasizedVernacularWords{p} aakisa arim-o-n. \\
rub-go-\textsc{ss}.\textsc{seq} go-\textsc{ss}.\textsc{seq} now grow-\textsc{pa}-3s\\
\glt`You kept rubbing it (over the years) and now you have grown up.'
\z

In the second type, any two verbs can, in principle, combine into a serial verb. But this process is less productive, and both the type and token frequency of this type is low when compared with the frequency of the first type. Usually, like in \REF{ex:3:x382} the meaning of the whole is transparent and can be inferred from the meanings of the component roots, but sometimes the semantics are more opaque \REF{ex:3:x383}.

\ea%x382
\label{ex:3:x382}
\gll Emera \textstyleEmphasizedVernacularWords{kue-puuk-ap} okaiwi siowa onak-e-k. \\
sago bite-cut-\textsc{ss}.\textsc{seq} other.side dog feed.him-\textsc{pa}-3s \\
\glt`He bit off half of the sago cake and fed it to the dog.'
\z

\ea%x383
\label{ex:3:x383}
\gll Aakun-emi \textstyleEmphasizedVernacularWords{mika-kof-a-m}. \\
speak-\textsc{ss}.\textsc{sim} spear-knock-\textsc{pa}-1s \\
\glt`I stumbled in my speech.'
\z

This type of serialization in Mauwake is very close to what \citet[1--5]{James1983} calls \textstyleEmphasizedWords{\textsc{lexical}} serialization. 

A special case among the roots forming serial verbs is \textstyleStyleVernacularWordsItalic{afur}- `do well'/`augmentative', which is not used as an independent verb, only as a second element in a serial verb structure.\footnote{See \citet[32]{James1983} for the use of a similar verb, \textstyleFootnoteBaseChar{\textit{ito,}} in Siane.}

\ea%x384
\label{ex:3:x384}
\gll Koora ku-owa \textstyleEmphasizedVernacularWords{amis-ar-afur-a-k}. \\
house build-\textsc{nmz} knowledge-\textsc{inch}-do.well-\textsc{pa}-3s \\
\glt`He really knew how to build a house.'
\z

It is quite possible even if not very common to form a three-root serial verb by combining the two types:

\ea%x385
\label{ex:3:x385}
\gll \textstyleEmphasizedVernacularWords{Mika-fien-ikiw-o-k}. \\
hit-push.aside-go-\textsc{pa}-3s \\
\glt`He went on countering (an attack).'
\z

It is far more common to have three verbs in a combination where an auxiliary is attached to a serial verb:

\ea%x386
\label{ex:3:x386}
\gll Naap \textstyleEmphasizedVernacularWords{amis-ar-ikiw-em-ik-o-wen}. \\
thus knowledge-\textsc{inch}-go-\textsc{ss}.\textsc{sim}-be-Np-\textsc{fu}.2p \\
\glt`That way you will gain more and more knowledge.'
\z

Combining four or more roots into one verbal group is more of a theoretical possibility than a practical reality. Examples are easy enough to obtain through elicitation, but very rare in non-elicited texts.

Mauwake does \textstyleEmphasizedWords{\textsc{not}} use serial verbs for a benefactive like many languages do \citep[174--80]{Sebba1987}; it utilizes benefactive morphology for that purpose (\sectref{sec:3.8.2.3.3}, 3.8.3.1). Neither is the serial verb structure used for aspect, as a verb plus auxiliary construction takes care of that. Another function often associated with serial verbs is that of instrument marking, but for that Mauwake uses either an ordinary switch-reference construction or an adverbial phrase (\sectref{sec:4.6.3}).

Distinguishing serial verbs from compound verbs on the one hand and medial clauses on the other is not a problem for Mauwake only, as serial verbs can behave very much like either \citep[17]{Crowley2002}. Crowley suggests the following continuum of gradually loosening syntactic juncture: verbal compounds {{\textgreater}} nuclear serial verbs {{\textgreater}} core serial verbs {{\textgreater}} clause chains {{\textgreater}} subordinate clauses {{\textgreater}} coordinate clauses (ibid. 18). In the following I will briefly discuss the relationship of serial verbs to adjunct plus verb constructions, to verbal groups consisting of a main verb plus auxiliary, and to medial clauses in Mauwake. 

The serial verbs in Mauwake show the following characteristics of compounding (cf. \citealt[69]{James1983} regarding Papuan languages). The first verb appears as a mere root (or as a stem, if it has undergone derivation); secondly, the verbs obligatorily share the same arguments; thirdly, the meaning of the whole may differ from the combined meanings of the parts. Furthermore, the stress and intonation contour of a serial verb is that of a single word rather than that of a phrase or a clause. There are two main reasons for calling them serial verbs. The first one is that especially the first type is productive. I also want to link them to a typologically widespread phenomenon instead of looking at them from a strictly language-specific point of view. In this I follow \citet[101]{Margetts1999}, who maintains that ``{the term `compound' does not by definition contradict an analysis as serialization}''. A similar position is also strongly defended by \citet[16]{Crowley2002} and by \citet[17]{Givon1991}.

Because of the tight restriction of ``root only'' for the first element in a serial verb in Mauwake, the main verb plus auxiliary combinations are left outside the group by definition. Another reason for this differential treatment is the fact that different processes seem to be going on in the two groups: grammaticalization in the main verb + \textstyleAcronymallcaps{AUX} group, lexicalization in the true serial verbs.\footnote{In some other languages main verb + AUX constructions are included among serial verbs (e.g. \citealt[29]{James1983}; \citealt[178]{Crowley2002}). \citet[174]{Farr1999} notes the ``staging'' aspects of the two constructions: in medial verbs the temporal relationship of the two verbs may be specified, but as ``the verbal constituents of SVCs [serial verb constructions] do not specify temporal borders or overlapping relationships, the events they represent can blend into a unit {\dots} and present the SVC is a complex but integrated event''.} 

In Mauwake the clause chaining is structurally midway between serialization and main clause coordination, and may consequently be used instead of either in some cases. The instrumental may in Mauwake be expressed by a `take-instrument-do' structure \REF{ex:3:x387} which in many serializing languages is a serial verb construction \citep[162--74]{Sebba1987}; but in Mauwake there is no good reason to call the structure anything other than a combination of a medial and final clause. This shows more clearly in example \REF{ex:3:x388}, which does not pass the rule for verbal groups: ``no non-verbal elements between the parts''. 

\ea%x387
\label{ex:3:x387}
\gll Fura \textstyleEmphasizedVernacularWords{aaw-ep} puuk-a-m. \\
knife take-\textsc{ss}.\textsc{seq} cut-\textsc{pa}-1s \\
\glt`I took a knife and cut it.' Or: `I cut it with a knife.'
\z

\ea%x388
\label{ex:3:x388}
\gll Burir aaw-ep nomokowa unowa war-e-mik. \\
axe take-\textsc{ss}.\textsc{seq} tree many fell-\textsc{pa}-1/3p \\
\glt`We took an axe and felled many trees.' Or: `We felled many trees with an axe.'
\z

For Mauwake, I propose the following continuum where the syntactic juncture gradually loosens: serial verb {{\textgreater}} verb + \textstyleAcronymallcaps{AUX} group {{\textgreater}} subordinate+main clause {{\textgreater}} clause chain {{\textgreater}} coordinate main clauses.

The borderline between serial verbs and medial verbs on the one hand, and between verb + \textstyleAcronymallcaps{AUX} groups and medial verbs on the other is not absolutely clear-cut. In \REF{ex:3:x389} the medial verb structure is used instead of a serial verb, even though the two actions are simultaneous, not sequential as indicated by the form of the medial verb.\footnote{Mauwake does not allow same subject simultaneous forms following each other except in a strictly coordinate structure where the verbs do not so much indicate simultaneity with each other as with the final verb.} 

\ea%x389
\label{ex:3:x389}
\gll Wi Malala=ke \textstyleEmphasizedVernacularWords{muf-ep} \textstyleEmphasizedVernacularWords{ekap-emi}{\dots} \\
3p.\textsc{unm} Malala=\textsc{tp} pull-\textsc{ss}.\textsc{seq} come-\textsc{ss}.\textsc{sim} \\
\glt`The Malala people came pulling it and{\dots}'
\z

Likewise, the four verbs in \REF{ex:3:x390} describe \textstyleEmphasizedWords{\textsc{one}} protracted action in spite of the sequential form in the medial verbs:

\ea%x390
\label{ex:3:x390}
\gll Ifa nain \textstyleEmphasizedVernacularWords{murar-ep} \textstyleEmphasizedVernacularWords{wiok-ap} \textstyleEmphasizedVernacularWords{ekap-ep} \textstyleEmphasizedVernacularWords{ekap-ep} owowa kerer-ek.\\
snake that1 follow-\textsc{ss}.\textsc{seq} follow.them-\textsc{ss}.\textsc{seq} come-\textsc{ss}.\textsc{seq} come-\textsc{ss}.\textsc{seq} village arrive-\textsc{pa}-3s\\
\glt`The snake kept following them and arrived in the village.'
\z

The main verb in verb plus \textstyleAcronymallcaps{AUX} combinations has to be in medial form. The only exception found is the continuous aspect form of the verb \textstyleStyleVernacularWordsItalic{wiaw}- `move around'. The mere root of this verb is used when it is the second verb in a serial structure which then takes an aspectual auxiliary:

\ea%x391
\label{ex:3:x391}
\gll Ifara mufe-\textstyleEmphasizedVernacularWords{wiaw}-ik-ok{\dots} \\
vine pull-move.around-be-\textsc{ss} \\
\glt`As he was pulling the vine around{\dots}'
\z

\subsubsection{Adjunct plus verb constructions} \label{sec:3:z:y:x}
%\hypertarget{RefHeading21061935131865}
{}
Papuan languages typically enlarge their verb inventories through adjunct plus verb combinations \citep[127]{Foley1986}. Foley only discusses nominal adjuncts, but adverbial adjuncts are commonly used in these structures as well. 

Mauwake is not nearly as productive in the use of the adjunct plus verb construction as many other Papuan languages. Some of them use almost exclusively generic verbs (\citealt[117]{Foley1986}; \citealt[309]{Roberts1987}; \citealt[145]{Whitehead2004}), whereas others employ a larger set of verbs \citep[62--66]{Farr1999} in these constructions.\footnote{Farr divides the nominals in these constructions into `complements' an `adjuncts'. Korafe does not seem to use adverbial adjuncts in these structures. } 

\paragraph{Nominal adjunct plus verb}\label{sec:3:a:z:y:x}
%\hypertarget{RefHeading21081935131865}
{}
The nominal adjuncts look like object \textstyleAcronymallcaps{NP}s, and the origin of at least some of them probably is in object \textstyleAcronymallcaps{NP}s, but currently there are syntactic and semantic differences between the two. An object \textstyleAcronymallcaps{NP} may be separated from the verb by the negator adverb \textstyleStyleVernacularWordsItalic{me} or by an accusative or a dative pronoun, but a nominal adjunct must immediately precede the verb. The meaning of the nominal adjunct plus verb construction often cannot be derived from the meanings of its constituent parts. 

\ea%x450
\label{ex:3:x450}
\gll Meta yia miim-ap yia \textstyleEmphasizedVernacularWords{miira} \textstyleEmphasizedVernacularWords{puuk-ekap-e-mik}. \\
fame 1p.\textsc{acc} hear-\textsc{ss}.\textsc{seq} 1p.\textsc{acc} face cut-come-\textsc{pa}-1/3p\\
\glt`They heard about us and came to greet us.'
\z

An object \textstyleAcronymallcaps{NP} only occurs with a transitive verb, but a nominal adjunct can also occur with an intransitive verb:

\ea%x451
\label{ex:3:x451}
\gll Uura or-op \textstyleEmphasizedVernacularWords{arua} \textstyleEmphasizedVernacularWords{karu-e-mik}. \\
night descend-\textsc{ss}.\textsc{seq} torch run-\textsc{pa}-1/3p\\
\glt`At night we went down to sea and fished with a torch.'
\z

Those nominal adjunct plus verb structures where the verb is transitive look like two-object clauses, and in a few cases behave like them syntactically. In \REF{ex:3:} the nominal adjunct \textstyleStyleVernacularWordsItalic{kema} `liver' is in its normal adjunct position, but in \REF{ex:3:} it is in object \textstyleAcronymallcaps{NP} position. The basic meanings of the two sentences are the same, but with a different prominence: \REF{ex:3:} encodes marked negative focus and \REF{ex:3:} verb focus. The clause \REF{ex:3:} with an initial theme pronoun \textstyleStyleVernacularWordsItalic{yo} `I' is pragmatically more neutral than the others except in cases where the initial pronoun receives extra stress. Note the intervening negator also in \REF{ex:3:}. 

\ea%x452
\label{ex:3:x452}
\gll Me efa \textstyleEmphasizedVernacularWords{kema} \textstyleEmphasizedVernacularWords{suuw-a-k}. \\
not 1s.\textsc{acc} liver push-\textsc{pa}-3s\\
\glt`He did \textstyleEmphasizedWords{\textsc{not}} think of me.'
\z

\ea%x453
\label{ex:3:x453}
\gll \textstyleEmphasizedVernacularWords{Kema} me efa \textstyleEmphasizedVernacularWords{suuw-a-k}. \\
liver not 1s.\textsc{acc} push-\textsc{pa}-3s\\
\glt`He didn't \textstyleEmphasizedWords{\textsc{think}} of me.'
\z

\ea%x1874
\label{ex:3:x1874}
\gll Yo me efa \textstyleEmphasizedVernacularWords{kema} \textstyleEmphasizedVernacularWords{suuw-a-k}. \\
1s.\textsc{unm} not 1s.\textsc{acc} liver push-\textsc{pa}-3s\\
\glt`He didn't think of me.'
\z

In cases where the adjunct only occurs with a certain verb it is difficult to give it a specified meaning apart from the verb. The same is true for verbs that do not occur independently, only with an adjunct.

\ea%x454
\label{ex:3:x454}
\gll \textstyleEmphasizedVernacularWords{Naruw} \textstyleEmphasizedVernacularWords{ir-a-mik}. \\
? ascend-\textsc{pa}-1/3p\\
\glt`They acted silly.'
\z

\ea%x455
\label{ex:3:x455}
\gll Naap \textstyleEmphasizedVernacularWords{kema} \textstyleEmphasizedVernacularWords{tuup-am-ika-i-ya}. \\
thus liver ?-\textsc{ss}.\textsc{sim}-be-Np-\textsc{pr}.3s\\
\glt`He is hoping so.'
\z

Most of the verbs in Mauwake indicating physiological or psychological states and cognition are nominal adjunct plus verb constructions. The verb takes the person marking from the experiencer. The following list gives only a small sample of these constructions, where the most common nominal is \textstyleStyleVernacularWordsItalic{kema} `liver'.\footnote{A good list of these is in \citet[47--63]{Kwan1989}, where she has described a large number of body image concepts formed with \textit{kema} from semantic point of view. For that study the syntactic characteristics of the structures were not relevant.} The second column provides a literal translation. A few more examples of these constructions are in the sentences \REF{ex:3:}-\REF{ex:3:}.

\begin{table}
\begin{tabular}{>{\it}lll}
\mytoprule
kema enekar- &liver catch.fire &`be thirsty'\\
kema kaalal- &liver float &`be enthusiastic'\\
kema korin- &liver get.stuck &`be confused'\\
kema peelal- &liver rot &`be grieved'\\
kema ten- &liver collapse &`be relieved'\\
eneka maayar- &tooth become.long &`be hungry for meat'\\
miira ikiw- &face go &`feel dizzy'\\
\mybottomrule 
\end{tabular}
\caption{Please provide a caption}
\todo[inline]{please provide a caption}
\label{tab:3:nominaladjunctplusverbconstrucion}
\end{table}


\ea%x1490
\label{ex:3:x1490}
\gll Uura \textstyleEmphasizedVernacularWords{uroma} \textstyleEmphasizedVernacularWords{ikiw-e-m}. \\
night stomach go-\textsc{pa}-1s\\
\glt`Last night I had diarrhea.'
\z

\ea%x1487
\label{ex:3:x1487}
\gll \textstyleEmphasizedVernacularWords{Kema} \textstyleEmphasizedVernacularWords{samor-ar-ep} maa me enim-i-yem. \\
liver spoil-\textsc{inch}-\textsc{ss}.\textsc{seq} food not eat-Np-\textsc{pr}.1s\\
\glt`I am sad and don't eat.'
\z

\ea%x1488
\label{ex:3:x1488}
\gll ...oko \textstyleEmphasizedVernacularWords{emina}  \textstyleEmphasizedVernacularWords{urur}\textstyleEmphasizedVernacularWords{-}\textstyleEmphasizedVernacularWords{ep} soomar-ikiw-i-kuan. \\
...other occiput drop-\textsc{ss}.\textsc{seq} walk-go-Np-\textsc{fu}.3p\\
\glt`{\dots}lest they feel ashamed and walk away.'
\z

\ea%x1489
\label{ex:3:x1489}
\gll Muuka gelemuta akena \textstyleEmphasizedVernacularWords{kema} me \textstyleEmphasizedVernacularWords{puk}\textstyleEmphasizedVernacularWords{-}\textstyleEmphasizedVernacularWords{e}\textstyleEmphasizedVernacularWords{-}\textstyleEmphasizedVernacularWords{mik}. \\
son small very liver not burst-\textsc{pa}-1/3p\\
\glt`Little boys/children do not think well (yet).'
\z

\paragraph{Adverbial adjunct plus verb}\label{sec:3:a:z:y:x}
%\hypertarget{RefHeading21101935131865}
{}
Adverbial adjuncts also have to precede the verb without any intervening words.

\ea%x456
\label{ex:3:x456}
\gll Maamuma efar \textstyleEmphasizedVernacularWords{ikum} \textstyleEmphasizedVernacularWords{aaw-e-mik}. \\
money 1s.\textsc{dat} illicitly get-\textsc{pa}-1/3p\\
\glt`They stole money from me.'
\z

\ea%x457
\label{ex:3:x457}
\gll Maa me efa \textstyleEmphasizedVernacularWords{pepek} \textstyleEmphasizedVernacularWords{er-a-k}. \\
food not 1s.\textsc{acc} enough go-\textsc{pa}-3s\\
\glt`The food wasn't enough for me.'
\z

Some of the adverbial adjuncts, like \textstyleStyleVernacularWordsItalic{ikum} `illicitly' \REF{ex:3:} and \textstyleStyleVernacularWordsItalic{pepek} `enough' \REF{ex:3:}, also function as independent adverbs, shown by an intervening pronoun \REF{ex:3:} and/or negator \REF{ex:3:}.

\ea%x458
\label{ex:3:x458}
\gll Yo oram \textstyleEmphasizedVernacularWords{ikum} efa wu-a-n. \\
1s.\textsc{unm} for.nothing illicitly 1s.\textsc{acc} put-\textsc{pa}-2s\\
\glt`You accused me for theft without grounds.'
\z

\ea%x459
\label{ex:3:x459}
\gll No \textstyleEmphasizedVernacularWords{pepek} me ma-e-n. \\
2s.\textsc{unm} enough not say-\textsc{pa}-2s\\
\glt`You didn't say right.'
\z

Other adjuncts like \textstyleStyleVernacularWordsItalic{ane} `together' and \textstyleStyleVernacularWordsItalic{anu} `apart', only combine with verbs to form verbal groups, and it is hard to give them an exact meaning; the glosses below are just approximations.

\ea%x460
\label{ex:3:x460}
\gll Apura \textstyleEmphasizedVernacularWords{ane} \textstyleEmphasizedVernacularWords{suuw-am-ika-iwkin} pok-ap ik-ok om-o-k.\\
widow together push-\textsc{ss}.\textsc{sim}-be-2/3p.\textsc{ds} sit.down-\textsc{ss}.\textsc{seq} be-SS cry-\textsc{pa}-3s\\
\glt`They were supporting the widow (sitting against her back) and she sat and wailed.'
\z

\ea%x461
\label{ex:3:x461}
\gll Opora \textstyleEmphasizedVernacularWords{anu} \textstyleEmphasizedVernacularWords{fien-owa} me pepek. \\
talk apart/aside brush.off-\textsc{nmz} not enough \\
\glt`He wasn't able to disregard the talk.'
\z

It was mentioned above that the meanings of the adjunct plus verb combinations are often idiomatic rather than analytically derivable from the meanings of the parts. But this is a somewhat dangerous statement for one to make who comes from outside the speech community. For example, how literally \textstyleStyleVernacularWordsItalic{kema} `liver', which figures very strongly in the adjunct plus verb constructions, is understood to be really involved in the emotional and cognitive processes would need to be established in a separate study.

\section{Adverbs}\label{sec:3:9}
%\hypertarget{RefHeading21121935131865}
{}
Adverbs in Mauwake are a heterogeneous class morphologically, syntactically and semantically. \citegen[20]{Schachter1985} definition of adverbs as words functioning ``{as modifiers of constituents other than nouns}'' is quite usable for Mauwake. Functionally the adverbs can be divided into four groups. The \textstyleEmphasizedWords{\textsc{material}} adverbs \citep{Ahlman1933}\footnote{Ahlman used the term in classifying adverbs in Finnish, and I find it useful in describing the adverbs in Mauwake as well, since the temporal, locative and manner adverbs share some characteristics which differentiate them from the other adverbs.} form the largest group, which contains the subgroups of locative, temporal and manner adverbs. The second group, that of \textstyleEmphasizedWords{\textsc{intensity}} adverbs,\footnote{In some grammars these form a class of their own, called ``intensifiers''. But that name is somewhat misleading as it may contain words like \textstyleFootnoteBaseChar{\textit{somewhat}} or \textstyleFootnoteBaseChar{\textit{hardly}} which do not intensify the meaning of the adjacent adjective or adverb.} consists of a small group of adverbs that function on phrase level and modify an adjective or adverb. \textstyleEmphasizedWords{\textsc{Sentential}} (or \textstyleEmphasizedWords{\textsc{modal}}) adverbs modify a whole sentence. The last group consists of the two \textstyleEmphasizedWords{\textsc{free}} adverbs \textstyleStyleVernacularWordsItalic{pun} `also' and \textstyleStyleVernacularWordsItalic{muutiw} `only'.

A material adverb may function as the head of an adverbial phrase. In this respect, however, adverbs differ from most other word classes: whereas the head of a \textstyleAcronymallcaps{NP} is usually a noun, that of a \textstyleAcronymallcaps{VP} a verb and an \textstyleAcronymallcaps{AP} an adjective, an adverbial phrase typically either consists of an adverb only, or does not contain an adverb word at all (\sectref{sec:4.6}.). The material and sentential adverbs may be modified by an intensity adverb, in particular by \textstyleStyleVernacularWordsItalic{akena} `very, truly' \REF{ex:3:x462}.

\ea%x462
\label{ex:3:x462}
\gll \textstyleEmphasizedVernacularWords{baliwep} \textstyleEmphasizedVernacularWords{akena} \\
well very\\
\glt`very well'
\z

The position of adverbs within a clause is also discussed under adverbial phrase (\sectref{sec:4.6}).

\subsection{Material adverbs}\label{sec:3:y:x}
%\hypertarget{RefHeading21141935131865}
{}
The material adverbs function as peripherals in a clause. They are divided into locative, temporal, and manner adverbs. The temporal and manner adverbs may be subdivided into deictic and non-deictic adverbs, and the locative adverbs are practically all deictic; in this they differ from the intensity and modal adverbs, which cannot be deictic.

\subsubsection{Locative adverbs}\label{sec:3:z:y:x}
%\hypertarget{RefHeading21161935131865}
{}
All the non-controversial locative adverbs are deictic, and they were discussed above in section on spatial deictics (\sectref{sec:3.6.3}). 

\ea%x1933
\label{ex:3:x1933}
\gll {\dots}mokoma kuisow naap \textstyleEmphasizedVernacularWords{fan} yiam=iya ik-e-mik. \\
year one thus here 1p.\textsc{refl}=\textsc{com} be-\textsc{pa}-1/3p\\
\glt`{\dots}for about a year they were here with us.'
\z

\ea%x1934
\label{ex:3:x1934}
\gll {\dots}mua owawiya \textstyleEmphasizedVernacularWords{neeke} ik-ok uruf-ap{\dots} kiiriw ep-i-kuan. \\
man with there.\textsc{cf} be-SS see-\textsc{ss}.\textsc{seq} again come-Np-\textsc{fu}.3p\\
\glt`{\dots}having been with her husband there and seeing [her father] they will come (back) again.'
\z

The words that are formed with a noun plus the locative clitic \nobreakdash-\textstyleStyleVernacularWordsxiiptItalic{pa} are treated as (adverbial) locative phrases, since they are expandable.

The words \textstyleStyleVernacularWordsxiiptItalic{mamaiya} `near, close' and \textstyleStyleVernacularWordsItalic{epasia} \footnote{\textit{Epasia} has probably developed from \textit{epa asia} `wild place'.} `far (away)' are actually locative nouns, but may be in the process of becoming adverbs. They optionally take the locative clitic \nobreakdash-\textstyleStyleVernacularWordsItalic{pa}, but its presence or absence causes no semantic difference. \textstyleStyleVernacularWordsItalic{Tiil} `edgewise, close' cannot take the locative clitic. Its use is quite restricted, and it might be more accurately classified as a manner adverb. 

\ea%x467
\label{ex:3:x467}
\gll \textstyleEmphasizedVernacularWords{Epasia} ikiw-em-ik-omkun yia far-e-k. \\
far go-\textsc{ss}.\textsc{sim}-be-1s/p.\textsc{ds} 1p.\textsc{acc} call-\textsc{pa}-3s\\
\glt`As we were (still) walking at a distance, he called us.'
\z

\ea%x469
\label{ex:3:x469}
\gll Fikera \textstyleEmphasizedVernacularWords{mamaiya=pa} nan pok-ap {\dots} \\
kunai.grass near=\textsc{loc} there sit-\textsc{ss}.\textsc{seq} \\
\glt`Having sat there near the kunai grass {\dots}'
\z

\ea%x1856
\label{ex:3:x1856}
\gll Mua oko=ke \textstyleEmphasizedVernacularWords{mamaiya} pok-a-k. \\
man other=\textsc{cf} near sit-\textsc{pa}-3s\\
\glt`Another man slept with her (lit: sat near).'
\z

\ea%x470
\label{ex:3:x470}
\gll Saapipia baliwep me wu-a-m, \textstyleEmphasizedVernacularWords{tiil} wu-a-m. \\
trap well not put-\textsc{pa}-1s on.edge put-\textsc{pa}-1s\\
\glt`I didn't put the trap well, I put it right on the edge (of the reef).'
\z

Locative expressions that in some other languages would be expressed through pre- or postpositions or adverbs are formed with locative phrases containing locative relational nouns in Mauwake. 

\ea%x468
\label{ex:3:x468}
\gll koor \textstyleEmphasizedVernacularWords{kuenuma}\textstyleEmphasizedVernacularWords{=pa} \\
house underside=\textsc{loc}\\
\glt`underneath (lit: in/on the underside of) the house'
\z

\subsubsection{Temporal adverbs}\label{sec:3:z:y:x}
%\hypertarget{RefHeading21181935131865}
{}
The temporal adverbs can be classified semantically as deictic or non-deictic. The meaning of the former is tied to the time of the utterance, whereas the meaning of the latter is independent of it. Both the deictic and non-deictic temporal adverbs are either specific or non-specific. This grouping is relevant on the syntactic level, as it influences the ordering of multiple temporal adverbials within a clause (\sectref{sec:4.6.2}).

\textstyleEmphasizedWords{\textsc{Deictic specific}} temporal adverbs refer to a certain day in relation to the time of the utterance.\footnote{The only exception to this in the data is \textstyleFootnoteBaseChar{\textit{uurika}}, which in the forms \textstyleFootnoteBaseChar{\textit{uurik ona}} (lit: `tomorrow place') and \textstyleFootnoteBaseChar{\textit{uurika naap nain}} (lit: `tomorrow thus that') means `the following day' and takes the time of the event as the deictic centre.} They are given in \tabref{tab:3:deicticsepecifictemporaladverbs}.

\begin{table}
 
\begin{tabular}{ll}
\mytoprule
aakisa\footnote{\textstyleFootnoteBaseChar{\textit{Aakisa}} `today, now' may be either specific or non-specific.} &`today'\\
unan &`yesterday'\\
erekema &`the day before yesterday'\\
uurika &`tomorrow'\\
ere &`the day after tomorrow'\\
arowona &`third day from today'\\
\mybottomrule 
\end{tabular}
\caption{Deictic temporal adverbs}
\label{tab:3:deicticsepecifictemporaladverbs}
\end{table}



\ea%x471
\label{ex:3:x471}
\gll \textstyleEmphasizedVernacularWords{Unan} nainiw yiam fiirim-e-mik. \\
Yesterday again 1p.\textsc{refl} gather-\textsc{pa}-1/3p\\
\glt`Yesterday we met again.'
\z

\ea%x472
\label{ex:3:x472}
\gll \textstyleEmphasizedVernacularWords{Uurika} emeria manina ikiw-ep en-owa nop-ap or-eka.\\
tomorrow woman garden go-\textsc{ss}.\textsc{seq} eat-\textsc{nmz} fetch-\textsc{ss}.\textsc{seq} descend-\textsc{imp}.2p\\
\glt`You women, go to the garden tomorrow and fetch food (and come) down.'
\z

The \textstyleEmphasizedWords{\textsc{deictic non-specific temporals}} (\tabref{tab:3:deicticnonspecifictemporals}) refer to a time that is related to the time of the utterance (or in some cases to the time of the event), but is not restricted to a certain day.

\begin{table}
\begin{tabular}{ll}
\mytoprule
aakisa &`now'\\
aakisa fain &`nowadays, now', literally: `now this'\\
aakisa fan &`just a while ago, just now (past)', literally: `now here'\\
aakisa kuisow &`right now, in a minute' (future), literally: `now one'\\
eewuar &`not yet'\\
iirakuma &`a few days ago'\\
iiriw &`already, earlier, long ago'\\
iiriwiw &`long time ago'\\
ikoka &`later'\\
ikoka kuisow &`right now' (future), literally: `later one'\\
uurik ona &`the following day', literally: `tomorrow place'\\
wiimar &`later, some other time'\\
\mybottomrule 
\end{tabular}
\caption{Deictic non-specific temporals.}
\label{tab:3:deicticnonspecifictemporals}
\end{table}

\ea%x473
\label{ex:3:x473}
\gll Aria, no \textstyleEmphasizedVernacularWords{aakisa} maa enim-e. \\
alright 2s.\textsc{unm} now thing/food eat-\textsc{imp}.2s\\
\glt`Alright, eat now.'
\z

\ea%x1215
\label{ex:3:x1215}
\gll \textstyleEmphasizedVernacularWords{Eewuar,}  eka me saanar-owa ik-ua. \\
not.yet water not dry-\textsc{nmz} be-\textsc{pa}.3s\\
\glt`Not yet, the water hadn't dried.'
\z

\ea%x474
\label{ex:3:x474}
\gll No emeria \textstyleEmphasizedVernacularWords{iiriw} sesek-a-mik. \\
2s.\textsc{unm} woman already send-\textsc{pa}-1/3p\\
\glt`We already sent your wife (away).'
\z

Both \textstyleStyleVernacularWordsItalic{ikoka} and \textstyleStyleVernacularWordsItalic{wiimar } mean `later', and they can occasionally be used interchangeably. \textstyleStyleVernacularWordsItalic{Ikoka} is the more common of the two, and has to be used when referring to a later time the same day. \textstyleStyleVernacularWordsItalic{Wiimar} always refers to a less specific time somewhere in the future, but the use of \textstyleStyleVernacularWordsItalic{ikoka} is spreading to cover that too. The sentence \REF{ex:3:} is from a wedding speech, and it was unlikely that the young couple would be fighting later the very same day.

\ea%x476
\label{ex:3:x476}
\gll \textstyleEmphasizedVernacularWords{Wiimar} ikiw-i-yan, \textstyleEmphasizedVernacularWords{ikoka} weetak. \\
later go-Np-\textsc{fu}.1p later no\\
\glt`We'll go some other time, not later today.'
\z

\ea%x477
\label{ex:3:x477}
\gll No \textstyleEmphasizedVernacularWords{ikoka} mua ikos irak-ep me efar kerer-e. \\
2s.\textsc{unm} later man with fight-\textsc{ss}.\textsc{seq} not 1s.\textsc{dat} arrive-\textsc{imp}.2s\\
\glt`Later when you fight with your husband, don't come to me.'
\z

\textstyleStyleVernacularWordsItalic{Aakisa} `now' can be modified to further specify the meaning, as the exact present moment is so short that a word referring to it only is practically useless. \textstyleStyleVernacularWordsItalic{Aakisa kuisow} (lit: `now one') refers to something that \textstyleEmphasizedWords{\textsc{will take place}} `just now', in a moment \REF{ex:3:x478}, \textstyleStyleVernacularWordsItalic{aakisa fan} (lit: `now here') refers to something that \textstyleEmphasizedWords{\textsc{has happened}} just now \REF{ex:3:x479} and \textstyleStyleVernacularWordsItalic{aakisa fain} (lit: `now this') compares the present situation with earlier times \REF{ex:3:x480}.

\ea%x478
\label{ex:3:x478}
\gll \textstyleEmphasizedVernacularWords{Aakisa} \textstyleEmphasizedVernacularWords{kuisow} on-e, ikoka weetak. \\
now one do-\textsc{imp}.2s later no\\
\glt`Do it right now, not later.'
\z

\ea%x479
\label{ex:3:x479}
\gll Muuna kirip-owa ma-e-mik nain \textstyleEmphasizedVernacularWords{aakisa} \textstyleEmphasizedVernacularWords{fan} kirip-a-mik. \\
debt return-\textsc{nmz} say-\textsc{pa}-1/3p that1 now here return-\textsc{pa}-1/3p\\
\glt`They (only) just now returned the debt they have talked about returning.'
\z

\ea%x480
\label{ex:3:x480}
\gll Iiriw miiw-aasa marew, \textstyleEmphasizedVernacularWords{aakisa} \textstyleEmphasizedVernacularWords{fain} miiw-aasa nepik akena. \\
earlier land-canoe none now this land-canoe crowd real\\
\glt`Earlier there were no cars, now(adays) there are lots of cars.'
\z

The interpretation of the \textstyleEmphasizedWords{\textsc{non-deictic}} temporals is not tied to the time of the utterance or to the time of the event (\tabref{tab:3:nondeicticspecific}). 

\begin{table}
\begin{tabular}{ll}
\mytoprule
uuriw &`morning'\\
amirika &`day(time), noon'\\
urera &`(late) afternoon'\\
uura &`evening/night'\\
uur gonegon\footnote{The word \textit{gonegon}, which I have not come across elsewhere, is a partial reduplication of the locative noun \textit{gone} `middle'. As a reduplication it is unusual in that the partial reduplication follows rather than precedes the root.} &`midnight'\\
epa wiiwim\footnote{This is a back-formation of the expression \textit{epa wii-wiim-ik-ua} [place \textsc{rdp}-dawn-be-\textsc{pa}.3s] `It is/was beginning to dawn'.} &`close to dawn'\\
\mybottomrule 
\end{tabular}
\caption{Non-deictic specific adverbs}
\label{tab:3:nondeicticspecific}
\end{table}

\ea%x698
\label{ex:3:x698}
\gll \textstyleEmphasizedVernacularWords{Amirika} ama kekan-eya uurar-i-mik. \\
day sun strong-2/3.\textsc{ds} rest-Np-\textsc{pr}.1/3p\\
\glt`During the day (or: at noon) when the sun is strong, we take a rest.'
\z

\ea%x699
\label{ex:3:x699}
\gll Yaapan=ke \textstyleEmphasizedVernacularWords{uura} ifera=pa nan pok-om-ow-a-mik. \\
Japan=\textsc{cf} evening/night sea=\textsc{loc} there sit-\textsc{ben}-CAUS-\textsc{pa}-1/3p\\
\glt`In the evening the Japanese made him sit in the sea.'
\z

The temporal adverbs in \tabref{tab:3:nondeicticnonspecific} are both \textstyleEmphasizedWords{\textsc{non-deictic}} and \textstyleEmphasizedWords{\textsc{non-specific}}:

\begin{table}
\begin{tabular}{ll}
\mytoprule
aawurun &`forever'\\
anane &`always', `every day'\\
ewur &`soon, quickly, fast'\\
ewursow &`soon, at once'\\
iir oko &`once upon a time, at some point' (lit: `(an)other time')\\
kiikir &`first'\\
kiiriw &`again'\\
mokomokoka &`first'\\
nainiw &`again' ({{\textless}}nain=iw)\\
muri\footnote{This is an Austronesian borrowing (M. Ross, p.c.). It also occurs in the verb \textit{murar-} `follow', which has grammaticalized from the adjunct plus verb compound \textit{muri ar-} `behind become'.} &`later, behind'\\
\mybottomrule 
\end{tabular}
\caption{Non-deictic non-specific advbers}
\label{tab:3:nondeicticnonspecific}
\end{table}

\ea%x502
\label{ex:3:x502}
\gll Yo \textstyleEmphasizedVernacularWords{anane} naap mauw-am-ika-i-yem. \\
I always thus work-\textsc{ss}.\textsc{sim}-be-Np-\textsc{pr}.1p\\
\glt`I always work like that.'
\z

\ea%x504
\label{ex:3:x504}
\gll Irak-owa maneka \textstyleEmphasizedVernacularWords{ewur} me imen-ar-e-k. \\
fight-\textsc{nmz} big quickly not find-\textsc{inch}-\textsc{pa}-3s\\
\glt`The big fight/war didn't start quickly.'
\z

\textstyleStyleVernacularWordsItalic{Kiiriw} and \textstyleStyleVernacularWordsItalic{nainiw} both mean `again', and they can be used interchangeably when referring to repeated action.

\ea%x697
\label{ex:3:x697}
\gll Ne \textstyleEmphasizedVernacularWords{nainiw} sande uura yiam fiirim-e-mik. \\
\textsc{add} again Sunday evening 1p.\textsc{refl} gather-\textsc{pa}-1/3p\\
\glt`And again on Sunday evening we gathered together.'
\z

\ea%x1762
\label{ex:3:x1762}
\gll Ne \textstyleEmphasizedVernacularWords{kiiriw} enuma on-am-ik-e-mik. \\
\textsc{add} again new make-\textsc{ss}.\textsc{sim}-be-\textsc{pa}-1/3p\\
\glt`And again they kept making a new one.'
\z

When some action or event results in a state that is the same or similar as before, even if the action itself is not repeated, only\textstyleStyleVernacularWordsItalic{} \textstyleStyleVernacularWordsItalic{kiiriw} can be used. Thus only \textstyleStyleVernacularWordsItalic{kiiriw} is possible in \REF{ex:3:x503}. \textstyleStyleVernacularWordsItalic{Kiiriw} indicates that Jesus is alive again, as he had been before, whereas \textstyleStyleVernacularWordsItalic{nainiw} would indicate that he had risen from the dead earlier too.

\ea%x503
\label{ex:3:x503}
\gll Yeesus \textstyleEmphasizedVernacularWords{kiiriw} iikir-a-k. \\
Jesus again rise-\textsc{pa}-3s\\
\glt`Jesus rose again (= rose from the dead).'
\z

Also, if the action is the same but the situation changes, \textstyleStyleVernacularWordsItalic{kiiriw} is used. The example \REF{ex:3:x1761} describes a situation where a grandmother first sent her younger grandchild, a girl, to listen to a sound. Later she sent the grandson for the same errand; the act of sending was repeated but the person who was sent changed:

\ea%x1761
\label{ex:3:x1761}
\gll \textstyleEmphasizedVernacularWords{Kiiriw} morena iperowa nain sesek-a-k. \\
again male older that1 send-\textsc{pa}-3s\\
\glt`Again she sent the elder male (grandchild).'
\z

Occasionally \textstyleStyleVernacularWordsItalic{kiiriw} and \textstyleStyleVernacularWordsItalic{nainiw} can be used together:

\ea%x700
\label{ex:3:x700}
\gll Ar-ep ik-eya aria \textstyleEmphasizedVernacularWords{kiiriw} mua nain \\
become-\textsc{ss}.\textsc{seq} be-2/3s.\textsc{ds} alright again man that1
\textstyleEmphasizedVernacularWords{nainiw} urup-o-k.
again ascend-\textsc{pa}-3s\\
\glt`When she had become like that, alright the man came up again.'
\z

\subsubsection{Manner adverbs}\label{sec:3:z:y:x}
%\hypertarget{RefHeading21201935131865}
{}
The manner adverbial phrase is often manifested by just an adverb word rather than a longer phrase. The same distinction between deictic and non-deictic adverbs that was made with the other material adverbs can be made with the manner adverbs as well. The description of the deictic manner adverbs is in 3.6.4.

\ea%x1935
\label{ex:3:x1935}
\gll ...maa oposia pun \textstyleEmphasizedVernacularWords{naap} sesek-a-mik. \\
thing meat also thus sell-\textsc{pa}-1/3p\\
\glt`{\dots}like that they also sold meat.'
\z

\ea%x1936
\label{ex:3:x1936}
\gll Soo nain \textstyleEmphasizedVernacularWords{feenap}: era erup ik-ua. \\
fishtrap that1 like.this way two be-\textsc{pa}.3s\\
\glt`The fishtrap (custom) is like this: there are two ways.'
\z

A few of the non-deictic manner adverbs have been derived from adjectives by the deletion of word-final /a/, but this process is not productive. \tabref{tab:3:nondeicticmanneradverbs} gives a list of some of the more common non-deictic manner adverbs.
 
\begin{table}
\begin{tabular}{ll}
\mytoprule
ariman &`openly, publicly'\\
baliwep/balisow &`well'\\
damol/samor &`badly, poorly' (from \textstyleStyleVernacularWordsItalic{damola/samora} `bad')\\
ewur/ewuriw &`quickly'\\
ikum &`illicitly'\\
kapi &`askew'\\
kaken/kakeniw &`straight, correctly'\\
kekelka &`quietly, gently'\\
kerew &`strongly'\\
kokot &`secretly'\\
momasia &`slowly' (cf. adjective \textstyleStyleVernacularWordsItalic{momasia} `slow')\\
momor &`indiscriminately', `foolishly' (from \textstyleStyleVernacularWordsItalic{momora} `foolish')\\
pepek &`correctly'\\
oram/moram &`without reason', `without doing anything'\footnote{This word is difficult to gloss in English; its meaning is close to that of Tok Pisin \textit{nating}.}\\
orawin &`for the benefit'\\
\mybottomrule 
\end{tabular}
\caption{Non-deictic manner adverbs.}
\label{tab:3:nondeicticmanneradverbs}
\end{table}

\ea%x704
\label{ex:3:x704}
\gll Naap yia ma-i-kuan na-ep yo \textstyleEmphasizedVernacularWords{ariman} nefa maak-i-yem.\\
thus 1p.\textsc{acc} say-Np-\textsc{fu}.3p say/think-\textsc{ss}.\textsc{seq} 1s.\textsc{unm} openly 2s.\textsc{acc} tell-Np-\textsc{pr}.1s\\
\glt`I am telling you this openly, thinking that they will say like that about us.'
\z

\ea%x505
\label{ex:3:x505}
\gll Opaimika \textstyleEmphasizedVernacularWords{baliwep} me wiar amis-ar-e-m. \\
talk well not 3.\textsc{dat} knowledge-\textsc{inch}-\textsc{pa}-1s\\
\glt`I don't/didn't know their language well.'
\z

\ea%x507
\label{ex:3:x507}
\gll Fikera \textstyleEmphasizedVernacularWords{ikum} kuum-e-mik nain ma-i-yem. \\
kunai.grass illicitly burn-\textsc{pa}-1/3p that1 say-Np-\textsc{pr}.1s\\
\glt`I tell about that when the kunai grass was burned by arson.'
\z

\ea%x506
\label{ex:3:x506}
\gll \textstyleEmphasizedVernacularWords{Samor} akena aruf-a-mik. \\
badly very hit-\textsc{pa}-1/3p\\
\glt`They beat him very badly.'
\z

\subsection{Intensity adverbs}\label{sec:3:y:x}
%\hypertarget{RefHeading21221935131865}
{}
Intensity adverbs are a small and heterogeneous group of adverbs that modify a verb, an adjective, a quantifier or another adverb. Some of them (\textstyleStyleVernacularWordsItalic{akena, maneka}) are also adjectives, some others (\textstyleStyleVernacularWordsItalic{lawisiw, iiwawun, wenup}) are non-numeral quantifiers (\sectref{sec:3.4.2}) with a second function as intensity adverbs. The distribution is different for each of the intensity adverbs (\tabref{tab:3:intensityadverbs}).

\begin{table}
\begin{tabular}{ll}
\mytoprule
akena &`very, really, truly'\\
iiwawun &`altogether'\\
kakeniw &`exactly'\\
lawisiw/lawiliw &`somewhat'\\
maneka &`very'\\
oram &`very, just'\\
pepek &`enough'\\
wenup &`very'\\
\mybottomrule 
\end{tabular}
\caption{Intensity adverbs}
\label{tab:3:intensityadverbs}
\end{table}

\ea%x508
\label{ex:3:x508}
\gll Moma fain eliw(a) \textstyleEmphasizedVernacularWords{oram}. \\
taro this good just/very \\
\glt`This taro is very good.'
\z

\ea%x510
\label{ex:3:x510}
\gll Koora nain maala \textstyleEmphasizedVernacularWords{pepek}. \\
house that long enough\\
\glt`That house is long enough.'
\z

\textstyleStyleVernacularWordsItalic{Akena} `really, truly' is more flexible than the other intensity adverbs in that it can modify a word belonging to almost any word class.

\ea%x706
\label{ex:3:x706}
\gll Eka mamaiya \textstyleEmphasizedVernacularWords{akena} i yoowa me aaw-i-yen \\
river near very 1p.\textsc{unm} hot not get-Np-\textsc{fu}.1p\\
\glt`Very near the river we'll not get hot.'
\z

\ea%x708
\label{ex:3:x708}
\gll Iikamin \textstyleEmphasizedVernacularWords{akena=ko} imen-ar-i-non? \\
when really=\textsc{nf} find-\textsc{inch}-Np-\textsc{fu}.3s\\
\glt`Exactly when is it going to appear?'
\z

\ea%x709
\label{ex:3:x709}
\gll Sira samora piipu-eka \textstyleEmphasizedVernacularWords{akena}. \\
habit bad leave-\textsc{imp}.2p really\\
\glt`You (pl) must really leave (your) bad habits.'
\z

\ea%x710
\label{ex:3:x710}
\gll Yiena ikos \textstyleEmphasizedVernacularWords{akena} iw-u. \\
1p.\textsc{gen} two.together really go-\textsc{imp}.1d\\
\glt`Lets's go \textstyleEmphasizedWords{\textsc{just}} the two of us together.'
\z

\ea%x1875
\label{ex:3:x1875}
\gll Weetak \textstyleEmphasizedVernacularWords{akena}, i=ko me kuum-e-mik. \\
no really, 1p.\textsc{unm}=\textsc{nf} not burn-\textsc{pa}-1/3p\\
\glt`\textstyleEmphasizedWords{\textsc{Really no}}, we did not burn it.'
\z

\textstyleStyleVernacularWordsItalic{Lawisiw} `somewhat' is different from the rest in that it precedes the expression it modifies, rather than following it.

\ea%x703
\label{ex:3:x703}
\gll Uuw-owa nain \textstyleEmphasizedVernacularWords{lawisiw}  yoowa. \\
work-\textsc{nmz} that1 somewhat hot/hard\\
\glt`That work is somewhat hard.'
\z

As an adjective \textstyleStyleVernacularWordsItalic{maneka} `big' is very common, but as an intensity adverb `very' it is very restricted in its distribution. \textstyleStyleVernacularWordsItalic{Maneka} cannot modify a verb, but it can intensify some non-numeral quantifiers like \textstyleStyleVernacularWordsItalic{unowa} `many' and \textstyleStyleVernacularWordsItalic{iiwawun} `altogether', as well as the temporal adverb \textstyleStyleVernacularWordsItalic{anane} `always'. 

\ea%x509
\label{ex:3:x509}
\gll Yo anane \textstyleEmphasizedVernacularWords{maneka} naap mauw-am-ika-i-yem. \\
I always very thus work-\textsc{ss}.\textsc{sim}-be-Np-\textsc{pr}.1s\\
\glt`I \textstyleEmphasizedWords{\textsc{always}} keep working like that.'
\z

\subsection{Modal adverbs}\label{sec:3:y:x}
%\hypertarget{RefHeading21241935131865}
{}
The two modal adverbs in Mauwake differ from each other not only semantically, but morphologically and syntactically as well. Modality of a predication is discussed in \sectref{sec:6.1}.

\textstyleStyleVernacularWordsItalic{Eliw} `all right, well'\footnote{The manner adverb `well' is \textstyleFootnoteBaseChar{\textit{baliwep} }\textstyleFootnoteBaseChar{(\sectref{sec:3.8.1.3})}.} is a deontic adverb and expresses permission or desirability: `it is all right/good that{\dots}'. It can often be translated with the auxiliary `may' in English. It follows the subject, if there is any, but precedes the other clause constituents \REF{ex:3:x514}. It may also be in the tail position after the clause, either following a clause that already has \textstyleStyleVernacularWordsItalic{eliw} in it \REF{ex:3:x515}, or by itself \REF{ex:3:x516}. 

\ea%x514
\label{ex:3:x514}
\gll Wie wi \textstyleEmphasizedVernacularWords{eliw} wiar op-i-kuan. \\
3s/p.uncle 3p.\textsc{unm} well 3.\textsc{dat} hold-Np-\textsc{fu}.3p\\
\glt`Her uncles may get (lit: hold) them (=clay pots) from her.' 
\z

\ea%x515
\label{ex:3:x515}
\gll \textstyleEmphasizedVernacularWords{Eliw} Kululu ma-e-man, \textstyleEmphasizedVernacularWords{eliw}. \\
well Kululu say-\textsc{pa}-2p well\\
\glt`It is all right that you mentioned Kululu, that is OK.'
\z

\ea%x516
\label{ex:3:x516}
\gll Nomokowa, nie owowa=pa fan pok-a-n, \textstyleEmphasizedVernacularWords{eliw}. \\
2s/p.brother 2s/p.uncle village=\textsc{loc} here sit-\textsc{pa}-2s well\\
\glt`It is good/OK that you settled here in your brother's and uncle's village.'
\z

An epistemic modal adverb is the clitic -\textstyleStyleVernacularWordsItalic{yon} (with an alternative form -\textstyleStyleVernacularWordsItalic{nion}), expressing hesitation or non-committal assumption: `perhaps', `maybe', `I suppose'. It is attached to the predicate, which usually is a verb but can also be non-verbal \REF{ex:3:}.

\ea%x517
\label{ex:3:x517}
\gll Maa me wu-om-a-mik=\textstyleEmphasizedVernacularWords{yon}. \\
thing/food not put-\textsc{ben}-\textsc{bnfy}2.\textsc{pa}-1/3p-perhaps\\
\glt`Perhaps they didn't put food (aside) for him.'
\z

\ea%x518
\label{ex:3:x518}
\gll Yo me efa ma-e-n=\textstyleEmphasizedVernacularWords{yon} aa? \\
1s.\textsc{unm} not 1s.\textsc{acc} say-\textsc{pa}-2s-perhaps aa\\
\glt`I suppose you weren't saying it about me?'
\z

\ea%x519
\label{ex:3:x519}
\gll Ni kema puk-owa marewa=ke=\textstyleEmphasizedVernacularWords{yon}! \\
2p.\textsc{unm} liver burst-\textsc{nmz} none=\textsc{cf}-perhaps\\
\glt`You must be crazy!' (Lit: `I suppose your liver hasn't burst (yet).')
\z

The question word \textstyleStyleVernacularWordsItalic{kamenion} `or what' is related to the modal adverb -\textstyleStyleVernacularWordsItalic{yon} (\sectref{sec:3.9.3}).

\subsection{Free adverbs}\label{sec:3:y:x}
%\hypertarget{RefHeading21261935131865}
{}
The adverbs \textstyleStyleVernacularWordsItalic{muut(a}\textstyleStyleVernacularWordsItalic{)}/\textstyleStyleVernacularWordsItalic{muutiw} `just/only' and \textstyleStyleVernacularWordsItalic{pun} `also, too' are called free adverbs, as they can move around quite freely and attach themselves to various elements in a clause. \textstyleStyleVernacularWordsItalic{Muutiw} is a combination of \textstyleStyleVernacularWordsItalic{muut(a)} and the limiter clitic \nobreakdash-\textstyleStyleVernacularWordsItalic{iw}, and it restricts restricts the scope of a preceding noun phrase or adverbial phrase. \textstyleStyleVernacularWordsItalic{Muut(a)} is used almost exclusively with noun phrases.

\ea%x747
\label{ex:3:x747}
\gll Aaya \textstyleEmphasizedVernacularWords{muutiw} en-em-ika-i-mik. \\
sugarcane only eat-\textsc{ss}.\textsc{sim}-be-Np-\textsc{pr}.1/3p\\
\glt`They are only eating sugarcane.' 
\z

\ea%x748
\label{ex:3:x748}
\gll Ofa sepa \textstyleEmphasizedVernacularWords{muutiw}  (if-o-k). \\
paint black only paint-\textsc{pa}-3s\\
\glt`He painted with only black paint.'
\z

\ea%x757
\label{ex:3:x757}
\gll Ewar wuun-i-ya nain \textstyleEmphasizedVernacularWords{muutiw} miim-i-nan. \\
wind blow-Np-\textsc{pr}.3s that only hear-Np-\textsc{fu}.2s\\
\glt`You will hear only the wind blowing.'
\z

\ea%x758
\label{ex:3:x758}
\gll Lotu koora Ulingan=pa \textstyleEmphasizedVernacularWords{muutiw} ik-ua=i? \\
worship house Ulingan=\textsc{loc} only be-\textsc{pa}.3s=\textsc{qm}\\
\glt`Is there a church only at Ulingan?'
\z

\ea%x806
\label{ex:3:x806}
\gll Aakisa \textstyleEmphasizedVernacularWords{muutiw} niir-i-mik. \\
today only play-Np-\textsc{pr}.1/3p\\
\glt`They play only today.'
\z

\ea%x807
\label{ex:3:x807}
\gll Eliw \textstyleEmphasizedVernacularWords{muutiw}. \\
well only\\
\glt`It's just all right.'
\z

\ea%x1820
\label{ex:3:x1820}
\gll Yo opora \textstyleEmphasizedVernacularWords{muut} naap. \\
1s.\textsc{unm} talk only thus\\
\glt`That's my talk.'
\z

\ea%x1821
\label{ex:3:x1821}
\gll Uf-owa erup \textstyleEmphasizedVernacularWords{muuta} naap uf-e-mik. \\
dance-\textsc{nmz} two only thus dance-\textsc{pa}-1/3p\\
\glt`We only danced two dances like that.'
\z

\textstyleStyleVernacularWordsItalic{Pun} `also' has even wider distribution than \textstyleStyleVernacularWordsItalic{muutiw}: it can occur following almost any element in a clause.\footnote{\textit{Pun} may be in the process of developing into a clitic. As a one-syllable word it it often has a weak stress, and some speakers also write it attached to the preceding word with a hyphen, the way clitics are written in the Mauwake orthography.} 

\ea%x749
\label{ex:3:x749}
\gll Ne waaya nain \textstyleEmphasizedVernacularWords{pun} afila marew, waaya asia \textstyleEmphasizedVernacularWords{pun.} \\
and pig that1 also grease no(ne) pig wild also\\
\glt`And that pig also didn't have fat, (as) it was a wild pig too.'
\z

\ea%x1937
\label{ex:3:x1937}
\gll Yos \textstyleEmphasizedVernacularWords{pun} wie opora nainiw ma-i-yem. \\
1s.\textsc{fc} too 3s/p.uncle talk again say-Np-\textsc{pr}.1s\\
\glt`I, too, will again give ``uncle-talk'' (=cultural instruction).'
\z

\ea%x750
\label{ex:3:x750}
\gll Ne \textstyleEmphasizedVernacularWords{pun} aakisa iperowa korokor or-owa sira iiriw wafur-a-mik.\\
and also now middle.aged initiation descend-\textsc{nmz} custom earlier throw-\textsc{pa}-1/3p\\
\glt`Also, now the middle-aged people have already rejected the initiation custom.'
\z

\ea%x751
\label{ex:3:x751}
\gll Iiriw \textstyleEmphasizedVernacularWords{pun} miiwa muuta nain irak-owa marew. \\
earlier also ground because.of that1 fight-\textsc{nmz} no(ne)\\
\glt`Earlier there were also no fights over ground' (or: `Earlier, too, there were no fights over ground.')
\z

\ea%x808
\label{ex:3:x808}
\gll Teeria maneka wadol opora mik-a-mik \textstyleEmphasizedVernacularWords{pun} naap, {\dots} \\
group big lie talk hit-\textsc{pa}-1/3p also thus\\
\glt`(When) the big group lied it was also like that, {\dots}'
\z

\section{Negators} \label{sec:3:10}
%\hypertarget{RefHeading21281935131865}
{}
Mauwake has four negators: \textstyleStyleVernacularWordsItalic{weetak}, \textstyleStyleVernacularWordsItalic{wia}, \textstyleStyleVernacularWordsItalic{me} and \textstyleStyleVernacularWordsItalic{marew}. They are morphologically free and syntactically heterogeneous, each one having its specific position. Of the four negators \textstyleStyleVernacularWordsItalic{me} is positioned before the negated element, while \textstyleStyleVernacularWordsItalic{marew} follows the negated element. \textstyleStyleVernacularWordsItalic{Weetak} and \textstyleStyleVernacularWordsItalic{wia} either form a complete utterance by themselves, or they are sentence-initial when used as negative interjections \REF{ex:3:} but clause-final when functioning as non-verbal predicates \REF{ex:3:}, and when replacing full clauses they take the position of the clause they replace \REF{ex:3:}, \REF{ex:3:}. 

\ea%x654
\label{ex:3:x654}
\gll Maamuma \textstyleEmphasizedVernacularWords{me} tuun-owa ik-e-mik. \\
money not count-\textsc{nmz} be-\textsc{pa}-1/3p\\
\glt`They haven't counted the money (yet).'
\z

\ea%x1112
\label{ex:3:x1112}
\gll Mukuna \textstyleEmphasizedVernacularWords{me} op-a, nefa kuum-i-non! \\
fire not touch-\textsc{imp}.2s 2s.\textsc{acc} burn-Np-\textsc{fu}.3s\\
\glt`Don't touch the fire, it will burn you!'
\z

\ea%x655
\label{ex:3:x655}
\gll I muuka \textstyleEmphasizedVernacularWords{marew}. \\
1p.\textsc{unm} son no(ne).\\
\glt`We have no son.'
\z

\ea%x707
\label{ex:3:x707}
\gll \textstyleEmphasizedVernacularWords{Wia}, me kookal-i-yem. \\
No not like-Np-\textsc{pr}.1s \\
\glt`No, I don't like it.'
\z

\ea%x1212
\label{ex:3:x1212}
\gll Yo uuw-owa oko \textstyleEmphasizedVernacularWords{weetak}. \\
1s.\textsc{unm} work-\textsc{nmz} other no\\
\glt`I have no other work.'
\z

\ea%x705
\label{ex:3:x705}
\gll Wafur-a-k na \textstyleEmphasizedVernacularWords{weetak}, ufer-a-k. \\
throw-\textsc{pa}-3s but no, miss-\textsc{pa}-3s\\
\glt`He threw it (a spear), but no (=he didn't succeed), he missed (the pig).'
\z

\ea%x1111
\label{ex:3:x1111}
\gll Akup-a-mik, akup-a-mik, \textstyleEmphasizedVernacularWords{wia}. \\
search-\textsc{pa}-1/3p search-\textsc{pa}-1/3p no\\
\glt`We searched and searched, but no (=we did not find it).'
\z

According to a rough generalization the most frequent negator \textstyleStyleVernacularWordsItalic{me} is basically a clause and constituent negator. It is also used to negate imperatives. \textstyleStyleVernacularWordsItalic{Weetak} and \textstyleStyleVernacularWordsItalic{wia} are negative interjections or predicates in verbless clauses, and \textstyleStyleVernacularWordsItalic{marew} can negate non-verbal predicates and occasionally noun phrase constituents. \textstyleStyleVernacularWordsItalic{Marew} often has the meaning `none at all'.

\textstyleStyleVernacularWordsItalic{Weetak}\textstyleStyleVernacularWordsItalic{} \REF{ex:3:}, \textstyleStyleVernacularWordsItalic{wia} and occasionally \textstyleStyleVernacularWordsItalic{marew}, may be intensified by a postposed intensity adverb \textstyleStyleVernacularWordsItalic{akena} `truly, very'. \textstyleStyleVernacularWordsItalic{Me} can only be intensified as a verbal negator, in which case \textstyleStyleVernacularWordsItalic{akena} comes after the verb rather than after the negator.

\ea%x652
\label{ex:3:x652}
\gll Ni niam erup kema\textbf{ marew akena}! \\
2p.\textsc{unm} 2p.\textsc{refl} two liver no(ne) really\\
\glt`The two of you have \textstyleEmphasizedWords{\textsc{really no}} sense at all!'
\z

\ea%x653
\label{ex:3:x653}
\gll \textstyleEmphasizedVernacularWords{Me} on-a-m \textstyleEmphasizedVernacularWords{akena}. \\
not do-\textsc{pa}-1s really\\
\glt`I \textstyleEmphasizedWords{\textsc{really didn't}} do it.'
\z

A fuller treatment of the negators is in \sectref{sec:6.2}, where negation as a functional category is discussed.\footnote{\citet{Berghall2006} gives a somewhat more comprehensive treatment of negation in Mauwake, but some of the analysis has changed since the writing of the article.}

\section{Connectives}\label{sec:3:11}
%\hypertarget{RefHeading21301935131865}
{}
The inventory of connectives in Mauwake is small. They are called connectives rather than conjunctions, because conjunctions are normally understood as a class of words, but in Mauwake a connective may be a word or a phrase. The term \textstyleEmphasizedWords{\textsc{conjunction}} is reserved for the conjunctive coordination construction (\sectref{sec:8.1.1}). Many of the connectives also have another primary function. 

The main division is into pragmatic and semantic connectives; all of them are coordinate. Subordination is discussed in \sectref{sec:8.3}. The connectives mostly operate on sentence level, joining clauses (\sectref{sec:8.1}). Almost all of the coordinators also conjoin sentences. Only the pragmatic connectives and the disjunctive connective \textstyleStyleVernacularWordsItalic{e} `or' are able to conjoin elements on the word and phrase levels as well. 

The most typical way of combining clauses is clause chaining through medial verbs, with no connective words at all (\sectref{sec:8.2}). When there are connectives, they are always placed between the two clauses. 

\subsection{Pragmatic connectives}\label{sec:3:y:x}
%\hypertarget{RefHeading21321935131865}
{}
Instead of clearly specifying the semantic relationship between the units they connect, like semantic connectives do, the pragmatic connectives signal a pragmatic relationship between them.\footnote{For this distinction on pragmatic and semantic connectives I am indebted to Stephen Levinsohn.} In \citegen[8]{Haspelmath2007} terms they are `medial [and] prepositive', meaning that they occur between the items they conjoin, and are linked more closely to the following constituent rather than the preceding one.

The connective \textstyleStyleVernacularWordsItalic{ne} `additive' only indicates that something is added to what has just been said. It can connect word and phrase level units (\sectref{sec:4.1.2}), but is mostly used between clauses (\sectref{sec:8.1}) and even sentences. It is semantically neutral. When it conjoins words \REF{ex:3:} or phrases \REF{ex:3:}, and often when it coordinates clauses \REF{ex:3:} or sentences \REF{ex:3:}, it can be translated into English with `and'. 

\ea%x711
\label{ex:3:x711}
\gll kumin, wutkekela \textstyleEmphasizedVernacularWords{ne} mera ... \\
hermit.crab calamari \textsc{add} fish\\
\glt`hermit crabs, calamari and fish {\dots}'
\z

\ea%x713
\label{ex:3:x713}
\gll Inawera sira unowa, \textstyleEmphasizedVernacularWords{ne} kemena unowa. \\
dream custom many \textsc{add} inside many\\
\glt`There are many kinds of dreams, and (they have) many meanings.'
\z

\ea%x714
\label{ex:3:x714}
\gll \textstyleEmphasizedVernacularWords{Ne} yo aakisa tep=pa ma-i-yem. \\
\textsc{add} 1s.\textsc{unm} now tape.recorder=\textsc{loc} say-Np-\textsc{pr}.1s\\
\glt`And now I say it to a tape recorder.'
\z

Words or phrases in lists are most commonly joined by juxtaposition only. If a connective is used, \textstyleStyleVernacularWordsItalic{ne} usually joins the last two \REF{ex:3:x1359} coordinands. It is also possible to place the connective(s) closer to the beginning of the list.

\ea%x1359
\label{ex:3:x1359}
\gll Sesa nain waaya erup arow \textstyleEmphasizedVernacularWords{ne} maamuma kuuma erepam ikur \textstyleEmphasizedVernacularWords{ne} manar kuisow, waa eneka, naap muuka sesenar-i-nen.\\
price that1 pig two three \textsc{add} money stick four five \textsc{add} forehead.ornament one pig tooth thus son buy-Np-\textsc{fu}.1s\\
\glt`(As for) the price, I will buy my son with two-three pigs and forty-fifty kina and a forehead ornament (and) pig's tusk(s), like that.'
\z

There is no emphatic coordinate connective of the type `both {\dots} and' in Mauwake. 

If the propositions connected by \textstyleStyleVernacularWordsItalic{ne} contrast with each other in some way, it may be interpreted as adversative and translated into English with `but'.\footnote{Many Papuan languages have a connective that is glossed `and/but'. I suspect it is an additive connective like \textit{ne}, which is only interpreted as either `and' or `but' according to the content of the clauses conjoined.} In these cases it is always a ``weak'' adversative in contrast to the demonstrative \textstyleStyleVernacularWordsItalic{nain} used in ``strong'' adversative clauses (\sectref{sec:8.1.3}).

\ea%x715
\label{ex:3:x715}
\gll Maa en-owa iw-e-mik, \textstyleEmphasizedVernacularWords{ne} rais weetak. \\
thing eat-\textsc{nmz} give.him-\textsc{pa}-1/3p \textsc{add} rice no\\
\glt`They gave him food, but not rice.'
\z

\ea%x716
\label{ex:3:x716}
\gll Wi me kuum-e-mik, \textstyleEmphasizedVernacularWords{ne} wi murar-owa=pa mukuna nain kerer-e-k. \\
3p.\textsc{unm} not burn-\textsc{pa}-1/3p \textsc{add} 3p.\textsc{unm} follow-\textsc{nmz}=\textsc{loc} fire that appear-\textsc{pa}-3s\\
\glt`They didn't burn it, but the fire started after them.'
\z

In a number of cases either neutral additive or contrastive interpretation is possible:

\ea%x1361
\label{ex:3:x1361}
\gll Wiam erup irak-ep puk-e-mik, aalbok=ke ifera or-o-k ne osaiwa=ke soor(a) asia ikiw-o-k.\\
3p.\textsc{refl} two fight-\textsc{ss}.\textsc{seq} disperse-\textsc{pa}-1/3p black.cuckoo-shrike=\textsc{cf} sea descend-\textsc{pa}-3s \textsc{add} bird.of.paradise=\textsc{cf} forest wild go-\textsc{pa}-3s\\
\glt`The two of them fought and went their separate ways, the black cuckoo-shrike went down to the coast and/but the bird of paradise went to the wild (rain)forest.'
\z

There are two discourse-marking pragmatic connectives, \textstyleStyleVernacularWordsItalic{aria} and \textstyleStyleVernacularWordsItalic{ne aria}. They both mark discontinuity in the text.

\textstyleStyleVernacularWordsItalic{Aria} `alright'\footnote{The translation reflects the Tok Pisin word \textit{orait}, which sometimes has a similar discourse function. \textit{Aria} occurs in many Madang languages, and the speakers of those languages tend to use \textit{aria} in Tok Pisin too.} usually comes sentence-initially, but can also or occur sentence-medially. Its main function is to indicate a break in the topic chain. In \REF{ex:3:x717} the topic changes from the snake to the man, and in \REF{ex:3:x718} from a health extension officer to a group of men:

\ea%x717
\label{ex:3:x717}
\gll Keraw-eya \textstyleEmphasizedVernacularWords{aria} nomokowa gelemuta puuk-ap ifa nain ifakim-o-k.\\
bite-2/3s.\textsc{ds} alright tree small cut-\textsc{ss}.\textsc{seq} snake that kill-\textsc{pa}-3s\\
\glt`It (=the snake) bit him, and he cut a small tree and killed the snake.'
\z

\ea%x718
\label{ex:3:x718}
\gll {\dotso miim-o-k. } \textstyleEmphasizedVernacularWords{Aria} wi kiiriw neeke {\dots} \\
{\dots}3s.\textsc{unm} precede-\textsc{pa}-3s. Alright 3p.\textsc{unm} again there.\textsc{cf} ...\\
\glt`{\dots} he went ahead. (When) they were \textstyleEmphasizedWords{\textsc{there}} again {\dots}'
\z

It often signals the beginning of a turn in a conversation \REF{ex:3:x721}, or beginning of a speech \REF{ex:3:x720}, again indicating a break with the preceding text. 

\ea%x721
\label{ex:3:x721}
\gll \textstyleEmphasizedVernacularWords{Aria} wiipa, i yia uruf-e. \\
alright daughter, 1p.\textsc{unm} 1p.\textsc{acc} see-\textsc{imp}.2s\\
\glt`Daughter, look at us.'
\z

\ea%x720
\label{ex:3:x720}
\gll \textstyleEmphasizedVernacularWords{Aria}, i owowa=ko urup-u. \\
alright, 1p.\textsc{unm} village=\textsc{nf} ascend-\textsc{imp}.1d\\
\glt`Alright, let's go back to the village.'
\z

Even if the topic stays the same, \textstyleStyleVernacularWordsItalic{aria} can be used, especially when there is a contrast between alternatives \REF{ex:3:x719}, or sometimes when an expected sequence of events is broken \REF{ex:3:x722}.

\ea%x719
\label{ex:3:x719}
\gll Mua maneka maamuma erup, \textstyleEmphasizedVernacularWords{aria} wi suule takira maamuma kuisow, naap omopora sesenar-e-mik.\\
man big money two alright 3p.\textsc{unm} school child money one thus door buy-\textsc{pa}-1/3p\\
\glt`The grown men paid two coins (=20 toea) for entrance, the schoolchildren one coin.'
\z

\ea%x722
\label{ex:3:x722}
\gll Wiawi onak urera maa uup-e-mik, \textstyleEmphasizedVernacularWords{aria} maa me wu-om-a-mik=yon.\\
3s/p.father 3s/p.mother evening food cook-\textsc{pa}-1/3p alright food not put-\textsc{ben}-\textsc{bnfy}2.\textsc{pa}-1/3p-perhaps\\
\glt`In the evening his parents cooked food, (but) perhaps they didn't put any food for him.'
\z

\textstyleStyleVernacularWordsItalic{Ne aria} `and alright' occurs less often than \textstyleStyleVernacularWordsItalic{aria}, and only sentence-initially. It marks major points of development in the plot of a story. 

\ea%x723
\label{ex:3:x723}
\gll Naap wia maak-e-mik. \textstyleEmphasizedVernacularWords{Ne} \textstyleEmphasizedVernacularWords{aria}, ifa nain murar-ep{\dots} \\
thus 3p.\textsc{acc} tell-\textsc{pa}-1/3p \textsc{add} alright snake that follow-\textsc{ss}.\textsc{seq}\\
\glt`They told them like that. Now, the snake followed them and {\dots}'
\z

Sometimes it also signals return to foreground text (i.e. main story line) after some backgrounded material. 

\subsection{Semantic connectives}\label{sec:3:y:x}
%\hypertarget{RefHeading21341935131865}
{}
The semantic connectives specify the relationship between two propositions. 

The disjunctive connective \textstyleStyleVernacularWordsItalic{e} `or' can connect not only propositions but words or phrases as well. It is used both for standard \REF{ex:3:x724} and interrogative \REF{ex:3:x725} disjunction\footnote{This terminology is from \citet{Haspelmath2007}.} (\sectref{sec:8.1.2}, 7.2.2). When there are two alternatives, the connective occurs between them. It is also common to have the question marker -\textstyleStyleVernacularWordsItalic{i} cliticized to the end of the first alternative, especially in questions, but also elsewhere.

\ea%x724
\label{ex:3:x724}
\gll ama arow naap, \textstyleEmphasizedVernacularWords{e} erepam naap, {\dots} \\
sun three thus or four thus \\
\glt`at about three o'clock, or at about four {\dots}'
\z

\ea%x725
\label{ex:3:x725}
\gll Emeria=ko efar uruf-a-man=\textstyleEmphasizedVernacularWords{i} \textstyleEmphasizedVernacularWords{e} weetak? \\
woman=\textsc{nf} 1s.\textsc{dat} see-\textsc{pa}-2p=\textsc{qm} or no\\
\glt`Did you see my wife or not?'
\z

When there are more alternatives than one and the question clitic is present, the connective may be left out altogether \REF{ex:3:x726}, or it may occur between the first two alternatives \REF{ex:3:x727}. 

\ea%x726
\label{ex:3:x726}
\gll maa oposia=i moma, emera, naap \\
thing meat=\textsc{qm} taro, sago, thus\\
\glt`meat, or taro, or sago, (things) like that'
\z

\ea%x727
\label{ex:3:x727}
\gll iwer eka=ki \textstyleEmphasizedVernacularWords{e} mauwa=ki, a episowa=ki, ufia=ki {\dots}\\
coconut water=\textsc{cf}.\textsc{qm} or what=\textsc{cf}.\textsc{qm} ah tobacco=\textsc{cf}.\textsc{qm}, betel.pepper=\textsc{cf}.\textsc{qm}\\
\glt`coconut juice or - ummm - tobacco, or betel pepper {\dots}'
\z

The following consecutive connectives marking effect or result\footnote{It is typical for Papuan languages to mark the effect/result clause rather than the cause/reason clause. For a Papuan language which has several connectives both for result and for reason, see \citet[267--273]{Farr1999}.} are used in sentences where the clauses have a consecutive, i.e. a cause-effect or reason-result relationship: \textstyleStyleVernacularWordsItalic{naapeya/naeya}, \textstyleStyleVernacularWordsItalic{neemi}, and \textstyleStyleVernacularWordsItalic{naap nain}. They can all be glossed with `therefore, (and) so'.

\textstyleStyleVernacularWordsItalic{Naapeya/naeya} is the most generic and frequently used of the four. \textstyleStyleVernacularWordsItalic{Naapeya} has developed from the manner adverb \textstyleStyleVernacularWordsItalic{naap} `thus' followed by the different-subject marker -\textstyleStyleVernacularWordsItalic{eya}\textstyleParagraphChari{ (\sectref{sec:3.8.3.5.2})};\footnote{Actually this connective in the coastal villages is \textit{naapera}, but because of the language committee's decision to use -\textit{eya} for the 2/3s.\textsc{ds} marker, this form is used here too.} the resulting meaning is `it being thus'. The origin of \textstyleStyleVernacularWordsItalic{naeya} is in the medial different-subject form of the verb \textstyleStyleVernacularWordsItalic{na}- `say, think'. The difference between the two is mainly dialectal, or areal: \textstyleStyleVernacularWordsItalic{naapeya} is used more on the coast, \textstyleStyleVernacularWordsItalic{naeya} in the inland. They are used for marking the effect or result clause in a consecutive sentence. 

\ea%x731
\label{ex:3:x731}
\gll I maamuma marew, \textstyleEmphasizedVernacularWords{naapeya} ifera=ko me sesenar-e-mik. \\
1p.\textsc{unm} money no(ne), so salt=\textsc{nf} not buy-\textsc{pa}-1/3p\\
\glt`We didn't have money, so we didn't buy salt.'
\z

\ea%x732
\label{ex:3:x732}
\gll Ben uuw-owa piipu-a-k. \textstyleEmphasizedVernacularWords{Naapeya} emina urur-ep me ekap-o-k.\\
Ben work-\textsc{nmz} left-\textsc{pa}-3s therefore occiput fall-\textsc{ss}.\textsc{seq} not come-\textsc{pa}-3s\\
\glt`Ben has left the work. Therefore he was ashamed to come.'
\z

\ea%x735
\label{ex:3:x735}
\gll Pika oona me kekan-ow-a-k. \textstyleEmphasizedVernacularWords{Naeya} uura ewar=ke teek-a-k.\\
wall bone not strong-CAUS-\textsc{pa}-3s therefore night wind=\textsc{cf} tear-\textsc{pa}-3s\\
\glt`He didn't strengthen the wall studs. So at night the wind tore it (the house) down.'
\z

\ea%x1413
\label{ex:3:x1413}
\gll I miiw-aasa=pa ekap-e-mik, \textstyleEmphasizedVernacularWords{naeya} o me yook-a-k.\\
1p.\textsc{unm} land-canoe=\textsc{loc} come-\textsc{pa}-1/3p therefore 3s.\textsc{unm} not follow.us-\textsc{pa}-3s\\
\glt`We came in a car, so he didn't follow/come with us.'
\z

The origin of \textstyleStyleVernacularWordsItalic{naeya} is so transparent that there are many cases where two different interpretations for \textstyleStyleVernacularWordsItalic{naeya} are acceptable \REF{ex:3:x734}. 

\ea%x734
\label{ex:3:x734}
\gll ``Yo koka=pa ik-e-m.'' \textstyleEmphasizedVernacularWords{Na-eya} Magerka=ke (ma-e-k){\dots} \\
1s.\textsc{unm} jungle=\textsc{loc} be-\textsc{pa}-1s say-2/3s.\textsc{ds} MacArthur (say-\textsc{pa}-3s)\\
\glt` ``I was in the jungle.'' He said that, and (or: So) MacArthur said, {\dots}'
\z

But in \REF{ex:3:x733} \textstyleStyleVernacularWordsItalic{naeya} clearly means `therefore' and cannot be interpreted as a medial verb, as the correct verb form in this case would be plural \textstyleStyleVernacularWordsItalic{naiwkin} `they said and{\dots}', not singular \textstyleStyleVernacularWordsItalic{naeya} `you/(s)he said and{\dots}'.

\ea%x733
\label{ex:3:x733}
\gll Iwera yia na-em-ik-e-mik. \textstyleEmphasizedVernacularWords{Naeya} iwera wia uruk-am-ik-om-a-mik.\\
coconut 1s.\textsc{acc} say-\textsc{ss}.\textsc{sim}-be-\textsc{pa}-1/3p So coconut 3p.\textsc{acc} drop-\textsc{ss}.\textsc{sim}-be-\textsc{ben}-\textsc{bnfy}2.\textsc{pa}-1/3p\\
\glt`They kept asking us for coconuts. So we kept dropping coconuts for them.'
\z

The originally dialectal difference may be developing into a semantic one. In the original text data from three decades ago there is no clear semantic distinction between the use of \textstyleStyleVernacularWordsItalic{naapeya} and \textstyleStyleVernacularWordsItalic{naeya}, but fairly recently when a group with members from different dialects, discussing language matters, produced consecutive clauses, nearly all of the sentences with \textstyleStyleVernacularWordsItalic{naapeya} were cases of cause-effect \REF{ex:3:x1414}, and all of the sentences with \textstyleStyleVernacularWordsItalic{naeya} were cases of reason and result \REF{ex:3:x1415}. 

\ea%x1414
\label{ex:3:x1414}
\gll I fiirim-owa=pa ik-emkun ama or-o-k, \textstyleEmphasizedVernacularWords{naapeya} epa kokom-ar-e-k.\\
1p.\textsc{unm} gather-\textsc{nmz}=\textsc{loc} be-1s/p.\textsc{ds} sun descend-\textsc{pa}-3s therefore place dark-\textsc{inch}-\textsc{pa}-3s\\
\glt`When we were in the meeting the sun went down, so it became dark.'
\z

\ea%x1415
\label{ex:3:x1415}
\gll I fiirim-owa=pa ik-emkun ama or-o-k, \textstyleEmphasizedVernacularWords{naeya} maa me wiar en-owa ikiw-o-k.\\
1p.\textsc{unm} gather-\textsc{nmz}=\textsc{loc} be-1s/p.\textsc{ds} sun descend-\textsc{pa}-3s therefore food not 3.\textsc{dat} eat-\textsc{nmz} go-\textsc{pa}-3s\\
\glt`When we were in the meeting the sun went down, so he went without eating the food.'
\z

\textstyleStyleVernacularWordsItalic{Naapeya} can also co-occur with the conjunctive coordinators \textstyleStyleVernacularWordsItalic{ne} or \textstyleStyleVernacularWordsItalic{aria}. In argumentation, \textstyleStyleVernacularWordsItalic{ne naapeya} or \textstyleStyleVernacularWordsItalic{aria naapeya} has to be used, when the reason is not confined to one clause but extends to a longer stretch of the discourse.

\ea%x1406
\label{ex:3:x1406}
\gll \textstyleEmphasizedVernacularWords{Aria} \textstyleEmphasizedVernacularWords{naapeya} niena soomar-owa ne aakun-owa pun sira yi-e-k nain kaken=iw ook-ap soomar-eka.\\
alright therefore 2p.\textsc{gen} walk-\textsc{nmz} \textsc{add} talk-\textsc{nmz} also custom give.us-\textsc{pa}-3s that1 straight=\textsc{lim} follow-\textsc{ss}.\textsc{seq} walk-\textsc{imp}.2p\\
\glt`So therefore, as concerns your walk and talk too, follow straight the behaviour that he gave us and walk that way.'
\z

\textstyleStyleVernacularWordsItalic{Neemi} is used only in reasoning. It requires some point of similarity between the antecedent and the result clause.

\ea%x736
\label{ex:3:x736}
\gll Teeria fain K10 wu-a-mik. \textstyleEmphasizedVernacularWords{Neemi} wi teeria nain pun K10 wu-a-mik.\\
group this K10 put-\textsc{pa}-1/3p therefore 3p.\textsc{unm} group that1 too K10 put-\textsc{pa}-1/3p\\
\glt`This group put down K10. Therefore that group put down K10, too.'
\z

\textstyleStyleVernacularWordsItalic{Naap nain} can be translated into English with `therefore', `in that case', `if so, then'. It is made up of the manner adverb \textstyleStyleVernacularWordsItalic{naap} `thus' and the distal demonstrative \textstyleStyleVernacularWordsItalic{nain} `that'. It is a strong connective, stressing the fact that the proposition following the connective is a logical conclusion from the preceding proposition.

\ea%x737
\label{ex:3:x737}
\gll Ni moma uup-i-man=i? \textstyleEmphasizedVernacularWords{Naap} \textstyleEmphasizedVernacularWords{nain} yo saa uup-i-nen.\\
2p.\textsc{unm} taro cook-Np-2p=\textsc{qm} thus that 1s.\textsc{unm} rice cook-Np-\textsc{fu}.1s\\
\glt`Are you cooking taro? In that case I'll cook rice.'
\z

It is much less common in Mauwake to mark the reason clause than the result clause with a connective. When the reason clause is emphasized, it is marked with the connective \textstyleStyleVernacularWordsItalic{moram} (\textstyleStyleVernacularWordsItalic{wia}) `because' and always follows the result clause rather than preceding it. The origin of the reason connective is in the question word \textstyleStyleVernacularWordsItalic{moram} `why?' and the negator \textstyleStyleVernacularWordsItalic{wia} `no(t)'.\footnote{\textit{Moram} as a reason connective is probably a calque on Tok Pisin \textit{bilong wanem} `why, because'. The negator, which does not influence the meaning of the connective, may have been added in Mauwake to help distinguish the connective from the question word.} The difference between \textstyleStyleVernacularWordsItalic{moram wia} and \textstyleStyleVernacularWordsItalic{moram} is that the former is mainly used across sentence boundary \REF{ex:3:x738}, and the latter within a sentence \REF{ex:3:x739}.

\ea%x738
\label{ex:3:x738}
\gll Maamuma senam aaw-e-mik. \textstyleEmphasizedVernacularWords{Moram} \textstyleEmphasizedVernacularWords{wia}, maa ele-eliwa sesek-a-mik. \\
money a.lot get-\textsc{pa}-1/3p why not thing \textsc{rdp}-good sell-\textsc{pa}-1/3\\
\glt`They got a lot of money. (That's) because they sold good things/foods.'
\z

\ea%x739
\label{ex:3:x739}
\gll Miiw-aasa muf-owa me ikiw-e-mik, \textstyleEmphasizedVernacularWords{moram} os=ke naap ar-eya.\\
land-canoe pull-\textsc{nmz} not go-\textsc{pa}-1/3p why 3s.\textsc{fc}=\textsc{cf} thus become-2/3s.\textsc{ds}\\
\glt`We didn't go to fetch a truck, because she had become like that (=died).'
\z

\section{Postpositions and clitics}\label{sec:3:12}
%\hypertarget{RefHeading21361935131865}
{}
Since Mauwake is an \textstyleAcronymallcaps{SOV} language, it is natural that it has postpositions rather than prepositions. But their number is small: besides the comitative postpositions there are only two others, one for comparison and one indicating reason.

Unlike the postpositions, which are both phonological and grammatical words, clitics are grammatical words that together with the preceding word form one phonological word. The stress assignment rule does not affect them: they are always unstressed. If there are any derivational and inflectional suffixes in the host word, the clitics are added after all of them (Dixon 2010a:221\nobreakdash-2).

The nominal clitics associate with noun phrases and attach themselves phonologically to the last element of the noun phrase. They mark either the case role or the pragmatic function of the \textstyleAcronymallcaps{NP}. The only sentential clitic is the question marker \nobreakdash-\textstyleStyleVernacularWordsItalic{i}. The modal clitic -\textstyleStyleVernacularWordsItalic{yon} `perhaps' was discussed above in 3.9.3.

The postpositions and clitics are discussed together because of their shared origin in some cases, causing similarity in form, and because some of them have similarities in function.

\subsection{Comitative clitic and postpositions}\label{sec:3:y:x}
%\hypertarget{RefHeading21381935131865}
{}
Accompaniment, or a\textstyleEmphasizedWords{\textsc{ comitative}} relationship may be expressed by one clitic or by five different postpositions, three of which are formed with the clitic. 

The comitative clitic is -\textstyleStyleVernacularWordsItalic{iya} `with, and, both {\dots} and'. The clitic may be attached to either of the two related \textstyleAcronymallcaps{NP}s, or to both.

\ea%x775
\label{ex:3:x775}
\gll Nan pok-ap-ik-e-mik, \textstyleEmphasizedVernacularWords{mua=iya} \textstyleEmphasizedVernacularWords{emeria}. \\
there sit-\textsc{ss}.\textsc{seq}-be-\textsc{pa}-1/3p man=\textsc{com} woman.\\
\glt`They were sitting there, (both) husband and wife.'
\z

\ea%x776
\label{ex:3:x776}
\gll \textstyleEmphasizedVernacularWords{Muuka} \textstyleEmphasizedVernacularWords{wiip=iya} kerer-e-mik. \\
son daughter=\textsc{com} appear-\textsc{pa}-1/3p\\
\glt`(Both) a son and a daughter appeared.'
\z

\ea%x777
\label{ex:3:x777}
\gll \textstyleEmphasizedVernacularWords{Bom=iya} \textstyleEmphasizedVernacularWords{kateres=iya,} \textstyleEmphasizedVernacularWords{bom=iya} \textstyleEmphasizedVernacularWords{kateres=iya} (fuurk-a-mik). \\
bomb=\textsc{com} cartridge=\textsc{com} bomb=\textsc{com} cartridge=\textsc{com} drop-\textsc{pa}-1/3p\\
\glt`They dropped (both) bombs and cartridges, (both) bombs and cartridges.'
\z

It combines with pronouns to form comitative pronouns (\sectref{sec:3.5.9}), and the word for `all', \textstyleStyleVernacularWordsItalic{unowiya}, is made up of \textstyleStyleVernacularWordsItalic{unowa} `many' plus the comitative clitic.

Occasionally the clitic can also be used to indicate instrument. 

\ea%x778
\label{ex:3:x778}
\gll Mauwa ar-e-n, \textstyleEmphasizedVernacularWords{amia=iya} nenar-e-mik=i? \\
what become-\textsc{pa}-2s bow=\textsc{com} shoot.you-\textsc{pa}-1/3p=\textsc{qm}\\
\glt`What happened to you, did they shoot you with a gun?'
\z

\textstyleStyleVernacularWordsItalic{Owawiya}\textstyleStyleVernacularWordsItalic{/owawik} `with, together with' is used only for humans; it can refer to two or more people. It can also occur by itself \REF{ex:3:}. The origin of the first part \textstyleStyleVernacularWordsItalic{owaw}- is unknown. The second part is either the comitative clitic -\textstyleStyleVernacularWordsItalic{iya} or the root of the existential verb \textstyleStyleVernacularWordsItalic{ik}\nobreakdash- `be', a reflection of an earlier construction \textstyleStyleVernacularWordsItalic{owawiya ik}\nobreakdash- `be together'. 

\ea%x821
\label{ex:3:x821}
\gll Yoli onak \textstyleEmphasizedVernacularWords{owawiya} efa amukar-e-mik. \\
Yoli 3s/p.mother with 1s.\textsc{acc} scold-\textsc{pa}-1/3p\\
\glt`Yoli and his mother scolded me.'
\z

\ea%x822
\label{ex:3:x822}
\gll \textstyleEmphasizedVernacularWords{Owawiya} feeke pok-ap ik-ok soomar-ek-eka. \\
with here.\textsc{cf} sit-\textsc{ss}.\textsc{seq} be-SS walk-go-\textsc{imp}.2p\\
\glt`(First) sit here with us and (then) go.'
\z

\ea%x1817
\label{ex:3:x1817}
\gll Iikir-ami onak \textstyleEmphasizedVernacularWords{owawik} soomar-e-mik. \\
get.up-\textsc{ss}.\textsc{sim} 3s/p.mother with walk-\textsc{pa}-1/3p\\
\glt`He got up and walked with his mother.'
\z

The postposition \textstyleStyleVernacularWordsItalic{onaiya/onaria}\textstyleStyleVernacularWordsItalic{/onaiyik} may be based on the third person singular genitive pronoun ona and the clitic -\textstyleStyleVernacularWordsItalic{iya}; \textstyleStyleVernacularWordsItalic{onaiyik} also includes the root of the verb \textstyleStyleVernacularWordsItalic{ik}\nobreakdash- `be'. This postposition is more generic and can be used with noun phrases referring to people \REF{ex:3:} but is also common when referring to things \REF{ex:3:}. When the relationship between the two noun phrases is unequal, \textstyleStyleVernacularWordsItalic{onaiya} may used, like in \REF{ex:3:}, where the other people carried the sick man. The subject marking on the verb is influenced by how active part the referents of the comitative \textstyleAcronymallcaps{NP} take in the action. When all the participants are active, the subject marking on the verb is plural. The speech in \REF{ex:3:} was directed towards the villagers who were instructed to stay away from the Japanese troops. 

\ea%x754
\label{ex:3:x754}
\gll Mua unowa \textstyleEmphasizedVernacularWords{onaiya} ikiw-e-mik. \\
man many with go-\textsc{pa}-1/3p\\
\glt`We went with many people.'
\z

\ea%x755
\label{ex:3:x755}
\gll Urom(a) \textstyleEmphasizedVernacularWords{onaiya} ik-ua. \\
stomach with be-\textsc{pa}.3s\\
\glt`She is/was pregnant.'
\z

\ea%x823
\label{ex:3:x823}
\gll Mua napuma \textstyleEmphasizedVernacularWords{onaiya} Medebur ek-a-mik. \\
man sick with Medebur go-\textsc{pa}-1/3p\\
\glt`They went to Medebur with the sick man '
\z

\ea%x1819
\label{ex:3:x1819}
\gll No ara sepa ara kia \textstyleEmphasizedVernacularWords{onaiyik} bilik ar-i-nan=na {\dots}\\
2s.\textsc{unm} trunk black trunk white together mixed become-Np-\textsc{fu}.2s=\textsc{tp}\\
\glt`If you, a black person, are together with the white people mixed with them {\dots}'
\z

Another comitative postposition that mainly refers to things is \textstyleStyleVernacularWordsItalic{feekiya} `with'. It originates from the combination of \textstyleStyleVernacularWordsItalic{feeke} `here' and -\textstyleStyleVernacularWordsItalic{iya} `comitative', but the meaning does not reflect the deictic origin of the initial part. In those rare cases when it is attached to a [+human] \textstyleAcronymallcaps{NP}, the referent of this \textstyleAcronymallcaps{NP} is subordinate to the referent of the other \textstyleAcronymallcaps{NP} and not in control, but still influencing the subject marking of the verb \REF{ex:3:x824}. 

\ea%x824
\label{ex:3:x824}
\gll Mokok urupa kaik-i-man nain \textstyleEmphasizedVernacularWords{feekiya} baurar-eka. \\
eye cup tie-Np-\textsc{pr}.2p that1 with flee-\textsc{imp}.2p\\
\glt`Flee with your ``eye cups'' (a singsing decoration) still on.'
\z

\ea%x825
\label{ex:3:x825}
\gll Wiamun gelemuta pun aaw-ep \textstyleEmphasizedVernacularWords{feekiya} ikiw-e-mik. \\
3s/p.brother small also take-\textsc{ss}.\textsc{seq} with go-\textsc{pa}-1/3p\\
\glt`He took his little brother too, and went with him.'
\z

\ea%x1818
\label{ex:3:x1818}
\gll Maa eliw akena nain aaw-ep \textstyleEmphasizedVernacularWords{feekiya} ikiw-o-k. \\
thing good very that1 take-\textsc{ss}.\textsc{seq} with go-\textsc{pa}-3s\\
\glt`He took the very good thing and went with it.'
\z

The dual comitative \textstyleStyleVernacularWordsItalic{ikos} `with, together (with)' can only be used when two human participants are referred to \REF{ex:3:x752}. It can also occur alone, without a preceding noun phrase, when the participants are known from the person/number suffix in the verb, or from the context \REF{ex:3:x753}. The parties are considered equally active, so the verb is always in the plural.

\ea%x752
\label{ex:3:x752}
\gll Wekera \textstyleEmphasizedVernacularWords{ikos} irak-e-mik. \\
3s/p.sister with fight-\textsc{pa}-1/3p\\
\glt`He fought with his sister.'
\z

\ea%x753
\label{ex:3:x753}
\gll \textstyleEmphasizedVernacularWords{Ikos} ikiw-i-yen. \\
with go-Np-\textsc{fu}.1p\\
\glt`Let's go together (just the two of us).'
\z

Associative \textstyleStyleVernacularWordsItalic{ame} `with others' is different from the comitative postpositions above in that only one of the parties is specified. The identity of `the others' is left unspecified.

\ea%x826
\label{ex:3:x826}
\gll Auwa \textstyleEmphasizedVernacularWords{ame} wia maak-eya res aaw-ep merena puuk-a-mik.\\
1s/p.father ASSOC 3p.\textsc{acc} tell-2/3s.\textsc{ds} razor take-\textsc{ss}.\textsc{seq} leg cut-\textsc{pa}-1/3p\\
\glt`He told my father and the others, and they took a razor and made a cut on his leg.'
\z

\ea%x827
\label{ex:3:x827}
\gll Kuuten \textstyleEmphasizedVernacularWords{ame}=ke miim-e-mik. \\
Kuuten ASSOC=\textsc{cf} precede-\textsc{pa}-1/3p\\
\glt`Kuuten with (some) others went ahead of them.'
\z

\subsection{Reason postposition \textit{muuta (nain)}}\label{sec:3:y:x}
%\hypertarget{RefHeading21401935131865}
{}
\textstyleStyleVernacularWordsItalic{Muuta}\textstyleStyleVernacularWordsItalic{ (nain)} `because of, for' gives a reason for an action, when the reason is expressed in a noun phrase rather than a full clause. It has developed from the adverb \textstyleStyleVernacularWordsItalic{muuta} `a little, only', and in some cases the meaning `only' is retained with the new function as well \REF{ex:3:x756}, \REF{ex:3:x1876}. The distal-1 demonstrative \textstyleStyleVernacularWordsItalic{nain} `that' is optional, and is left out especially when there is another demonstrative \textstyleStyleVernacularWordsItalic{nain} preceding \textstyleStyleVernacularWordsItalic{muuta} \REF{ex:3:x759}.

\ea%x756
\label{ex:3:x756}
\gll Iiriw miiwa \textstyleEmphasizedVernacularWords{muuta} \textstyleEmphasizedVernacularWords{nain} irak-owa marew, oram momor mauw-am-ik-e-mik.\\
earlier land for that1 fight-\textsc{nmz} no(ne) just indiscriminately work-\textsc{ss}.\textsc{sim}-be-\textsc{pa}-1/3p\\
\glt`Earlier there was no fighting for land, they just worked indiscriminately (on any land).'
\z

\ea%x1876
\label{ex:3:x1876}
\gll Yia amukar-owa \textstyleEmphasizedVernacularWords{muuta} \textstyleEmphasizedVernacularWords{nain} nan iiriw ifakim-e-mik. \\
1p.\textsc{acc} scold-\textsc{nmz} for that1 there earlier kill-\textsc{pa}-1/3p\\
\glt`We killed her earlier (only) because she scolded us (lit: {\dots}for her scolding of us).'
\z

\ea%x759
\label{ex:3:x759}
\gll Opora ara nain \textstyleEmphasizedVernacularWords{muuta} ifakim-u na-ep on-a-mik. \\
Talk section that1 for kill-\textsc{imp}.1d say-\textsc{ss}.\textsc{seq} do-\textsc{pa}-1/3p\\
\glt`(Only) because of that talk they tried to kill him.'
\z

\subsection{Comparison postposition \textit{saarik}}\label{sec:3:y:x}
%\hypertarget{RefHeading21421935131865}
{}
\textstyleStyleVernacularWordsItalic{Saarik} `like' occurs with noun phrases \REF{ex:3:x760} and with nominalized clauses \REF{ex:3:x761}. It indicates a point of similarity between two essentially \textstyleEmphasizedWords{different} things.

\ea%x760
\label{ex:3:x760}
\gll Mua eliwa \textstyleEmphasizedVernacularWords{saarik} aakun-e-k. \\
man good like speak-\textsc{pa}-3s\\
\glt`He spoke like a good man.'
\z

\ea%x761
\label{ex:3:x761}
\gll No sia on-owa \textstyleEmphasizedVernacularWords{saarik} magimal puuk-a-n. \\
2s.\textsc{unm} netbag make-\textsc{nmz} like vine.sp. cut-\textsc{pa}-2s\\
\glt`You cut \textstyleForeignWords{magimal} vine as if you were going to make a netbag.'
\z

For the functional category of comparison, see \sectref{sec:6.5}.

\subsection{Locative clitic -\textit{pa}}\label{sec:3:y:x}
%\hypertarget{RefHeading21441935131865}
{}
The locative clitic -\textstyleStyleVernacularWordsItalic{pa} mainly marks locative in noun phrases \REF{ex:3:x762}. The most common verb that it collocates with is \textstyleStyleVernacularWordsItalic{ik}- `be' \REF{ex:3:x1880}. When it occurs with the directional verbs (\sectref{sec:3.8.4.4.5}), it often indicates source \REF{ex:3:x1877}, but it can also be used for path \REF{ex:3:x770}, or for instrument in cases where it has a strong locative meaning as well \REF{ex:3:x771}. It is rarely used for goal \REF{ex:3:x1879}; this is possible in cases where the goal is the location for an event taking place immediately.

\ea%x762
\label{ex:3:x762}
\gll Pon piipa unowa=\textbf{pa} soomar-em-ik-eya mik-a-m. \\
turtle seaweed many=\textsc{loc} walk-\textsc{ss}.\textsc{sim}-be-2/3s.\textsc{ds} spear-\textsc{pa}-1s\\
\glt`The turtle was walking among the seaweeds and I speared it.'
\z

\ea%x1880
\label{ex:3:x1880}
\gll Ona owowa=\textstyleEmphasizedVernacularWords{pa} ik-eya epa wiim-o-k. \\
3s.\textsc{gen} village=\textsc{loc} be-2/3s.\textsc{ds} place dawn-\textsc{pa}-3s\\
\glt`When he was in his village it dawned.'
\z

\ea%x1877
\label{ex:3:x1877}
\gll Ifa maneka=ke iinan=\textstyleEmphasizedVernacularWords{pa} or-o-k. \\
snake big=\textsc{cf} on.top=\textsc{loc} descend-\textsc{pa}-3s\\
\glt`A big snake dropped from above.'
\z

\ea%x770
\label{ex:3:x770}
\gll Saa=\textstyleEmphasizedVernacularWords{pa} ir-am-ika-i-mik. \\
sand=\textsc{loc} come/go-\textsc{ss}.\textsc{sim}-be-Np-\textsc{pr}.1/3p\\
\glt`They are coming on/along the beach.'
\z

\ea%x771
\label{ex:3:x771}
\gll Miiw aasa=\textstyleEmphasizedVernacularWords{pa} ikiw-e-mik. \\
land canoe=\textsc{loc} go-\textsc{pa}-1/3p\\
\glt`We went by car/in a car.'
\z

\ea%x1879
\label{ex:3:x1879}
\gll Mua nain ... eka kapa\textstyleEmphasizedVernacularWords{=pa} ir-ap eka nain up-o-k.\\
man that1 ... river top=\textsc{loc} come/go-\textsc{ss}.\textsc{seq} river that1 block-\textsc{pa}-3s\\
\glt`The man went to the top/source of the river and blocked the river.'
\z

As temporal phrases locate an event in time, they also use the same locative clitic \REF{ex:3:x763}.

\ea%x763
\label{ex:3:x763}
\gll Fraide=\textstyleEmphasizedVernacularWords{pa} maapora puk-o-k, urera. \\
Friday=\textsc{loc} celebration burst-\textsc{pa}-3s, afternoon.\\
\glt`On Friday the celebration started, in the afternoon.'
\z

It can also be used with an essive meaning, when referring to people's jobs:

\ea%x765
\label{ex:3:x765}
\gll Yena mua owowa ekap-o-k, amia mua=\textstyleEmphasizedVernacularWords{pa} ik-ok. \\
1s.\textsc{gen} man village come-\textsc{pa}-3s bow man=\textsc{loc} be-SS \\
\glt`My husband came back to the village, having been a policeman.'
\z

The locative clitic has its origin in the word \textstyleStyleVernacularWordsItalic{epa} `place'; the transition vowel [e] can sometimes be heard between the clitic and its host, when the host word ends in a consonant \REF{ex:3:x764}.

\ea%x764
\label{ex:3:x764}
\gll Ne Sarak ikos Gawar=(\textstyleEmphasizedVernacularWords{e})\textstyleEmphasizedVernacularWords{pa} ik-emkun yia maak-e-mik {\dots} \\
\textsc{add} Sarak with Gawar=\textsc{loc} be-1s/p.\textsc{ds} 1p.\textsc{acc} tell-\textsc{pa}-1/3p\\
\glt`And as Sarak and I were in Gawar, they told us, {\dots} '
\z

\subsection{Instrumental clitic -\textit{iw}}\label{sec:3:y:x}
%\hypertarget{RefHeading21461935131865}
{}
The instrumental clitic -\textstyleStyleVernacularWordsItalic{iw}\textstyleStyleVernacularWordsItalic{} is used both for concrete \REF{ex:3:} and abstract \REF{ex:3:} instruments.\footnote{A less emphasized way to add an instrument is to use the chaining structure: `take instrument do something' \REF{ex:3:}, \REF{ex:3:}.} 

\ea%x766
\label{ex:3:x766}
\gll Nomokowa galua-galua nain=\textstyleEmphasizedVernacularWords{iw} biiris on-am-ik-e-mik \\
tree soft-soft that1=\textsc{inst} bridge make-\textsc{ss}.\textsc{sim}-be-\textsc{pa}-1/3p\\
\glt`They kept making bridges with soft timber.'
\z

\ea%x768
\label{ex:3:x768}
\gll ...wiena opaimik=\textstyleEmphasizedVernacularWords{iw} yia maak-em-ik-e-mik. \\
3p.\textsc{gen} mouth=\textsc{inst} 1p.\textsc{acc} tell-\textsc{ss}.\textsc{sim}-be-\textsc{pa}-1/3p\\
\glt`They talked to us in their language.'
\z

It is also used for path and has the meaning `along'; the verb indicates action that continues for some time.

\ea%x767
\label{ex:3:x767}
\gll Saa=\textstyleEmphasizedVernacularWords{iw} ir-am-ika-i-mik, ... \\
sand=\textsc{inst} ascend-\textsc{ss}.\textsc{sim}-be-Np-\textsc{pr}.3p\\
\glt`They are coming along the beach{\dots}'
\z

The difference between \REF{ex:3:} and \REF{ex:3:} above is that in \REF{ex:3:} the people were coming along the beach at least some of the way, and more specifically at the time of the speaking; whereas \REF{ex:3:} indicates that they travelled along the beach more or less the whole way.

The instrumental may also be utilized to indicate manner: 

\ea%x1881
\label{ex:3:x1881}
\gll Uurik ona naap=\textstyleEmphasizedVernacularWords{iw} iw-ap poka aaw-e-mik. \\
tomorrow place thus=\textsc{inst} go-\textsc{ss}.\textsc{seq} housepost get-\textsc{pa}-1/3p\\
\glt`The following day they went in the same way and got houseposts.'
\z

\ea%x773
\label{ex:3:x773}
\gll Karu-(o)w=\textstyleEmphasizedVernacularWords{iw} ekap-o-k. \\
run-\textsc{nmz}=\textsc{inst} come-\textsc{pa}-3s\\
\glt`He came running.'
\z

\ea%x1814
\label{ex:3:x1814}
\gll Ne ikoka maa marew eliw  manek=\textstyleEmphasizedVernacularWords{iw} ika-i-nan. \\
\textsc{add} later thing none well big=\textsc{inst} be-Np-\textsc{fu}.2s\\
\glt`And later you will have no problems, you will just be very well.'
\z

\ea%x774
\label{ex:3:x774}
\gll Waaya=ke anane wiar en-ow=\textstyleEmphasizedVernacularWords{iw} ika-i-ya. \\
pig=\textsc{cf} always 3.\textsc{dat} eat-\textsc{nmz}=\textsc{inst} be-Np-\textsc{pr}.3s\\
\glt`A pig stays eating their (taro) all the time.'
\z

Another usage is in those temporal phrases that refer to something taking place repeatedly at the same time: 

\ea%x1882
\label{ex:3:x1882}
\gll I amirik=\textstyleEmphasizedVernacularWords{iw} ... Gawar wiar ikiw-e-mik. \\
1p.\textsc{unm} daytime=\textsc{inst} {\dots} Gawar 3.\textsc{dat} go-\textsc{pa}-1/3p\\
\glt`In the daytime we always went to Gawar {\dots}'
\z

\subsection{Limiter -\textit{iw}}\label{sec:3:y:x}
%\hypertarget{RefHeading21481935131865}
{}
The limiter clitic -\textstyleStyleVernacularWordsItalic{iw} `only, just' is homophonous with the instrumental. The two probably are of common origin, but synchronically their meanings and positions in the word are distinct (\sectref{sec:3.12.9}). The limiter does not mark a case but restricts the applicability of the predication to the element that it is attached to. 

\ea%x769
\label{ex:3:x769}
\gll [Maa eka]\textsubscript{NP}=\textstyleEmphasizedVernacularWords{iw} en-ep en-ep lebum-ar-i-nan. \\
food water=\textsc{lim} eat-\textsc{ss}.\textsc{seq} eat-\textsc{ss}.\textsc{seq} lazy-\textsc{inch}-Np-\textsc{fu}.2s\\
\glt`When you keep eating only food cooked with water you become tired of it.'
\z

\ea%x772
\label{ex:3:x772}
\gll Mua=\textstyleEmphasizedVernacularWords{iw} pok-aka. \\
man=\textsc{lim} sit-\textsc{imp}.2p\\
\glt`Sit just among the men.'
\z

The limiter clitic may attach itself to genitive and focal pronouns (\sectref{sec:3.5.7}). 

The free adverb \textstyleStyleVernacularWordsItalic{muutiw} `only' is a combination of \textstyleStyleVernacularWordsItalic{muut(a)} `only' and the limiter clitic.

\subsection{Topic and focus markers}\label{sec:3:y:x}
%\hypertarget{RefHeading21501935131865}
{}
The topic and focus markers indicate the discourse function of the noun phrases that they are attached to.

\subsubsection{Topic markers}\label{sec:3:z:y:x}
%\hypertarget{RefHeading21521935131865}
{}
Of the two topic markers \textstyleStyleVernacularWordsItalic{ena} is fairly low in frequency, and the description given here is only tentative. It seems that \textstyleStyleVernacularWordsItalic{ena} as an independent word originally had a topic marking function, but later the topic clitic \textstyleStyleVernacularWordsxiiptItalic{-(e)na} developed from it and is now used for highlighted topic (\sectref{sec:9.1.2.4}) in main clauses. \textstyleStyleVernacularWordsItalic{Ena} still marks a topic, but only in relative clauses. It often has a specifying function as well: `the/that (particular one)'. 

\ea%x1681
\label{ex:3:x1681}
\gll [\textstyleEmphasizedVernacularWords{Mua} \textstyleEmphasizedVernacularWords{ena} ma-e-k nain] makena yos. \\
man SPEC say-\textsc{pa}-3s that1 true 1s.\textsc{fc}\\
\glt`The man that he talked about is I.'
\z

In long relative clauses, where it is attached to the \textstyleAcronymallcaps{RelNP}, it helps to distinguish it from all the other \textstyleAcronymallcaps{NP}s in the clause. 

\ea%x1815
\label{ex:3:x1815}
\gll [\textstyleEmphasizedVernacularWords{Mua} \textstyleEmphasizedVernacularWords{papako} \textstyleEmphasizedVernacularWords{ena} Australia=ke wia aaw-ep wiena feekiya yiaw-e-mik nain] me epa fan irak-owa uruf-a-mik {\dots}\\
man other SPEC Australia=\textsc{cf} 3p.\textsc{acc} take-\textsc{ss}.\textsc{seq} 3p.\textsc{gen} with walk.around-\textsc{pa}-1/3p that1 not place here fight-\textsc{nmz} see-\textsc{pa}-1/3p\\
\glt`Those other (particular) men whom the Australians took and with whom they walked around did not see the war here in this place {\dots}'
\z

\ea%x1683
\label{ex:3:x1683}
\gll [\textstyleEmphasizedVernacularWords{I} mua owowa=pa ik-ok \textstyleEmphasizedVernacularWords{ena} irakowa uruf-a-mik nain] nanar nain yo fan ma-i-yem.\\
1p.\textsc{unm} man village=\textsc{loc} be-SS SPEC fight-\textsc{nmz} see-\textsc{pa}-1/3p that1 story that1 1s.\textsc{unm} here say-Np-\textsc{pr}.1s\\
\glt`I am telling the story of us (particular) people who stayed in the village and saw the fighting.'
\z

If the head noun of the \textstyleAcronymallcaps{NP} is is recoverable from the context, it may be deleted, leaving behind only \textstyleStyleVernacularWordsItalic{ena}. In \REF{ex:3:x1682} the head noun \textstyleStyleVernacularWordsItalic{epira} `bowl(s)' has been omitted.

\ea%x1682
\label{ex:3:x1682}
\gll [Aakisa fan \textstyleEmphasizedVernacularWords{ena} maneka wu-a-mik nain] eliw, wie wi eliw wiar op-i-kuan.\\
now here SPEC big put-\textsc{pa}-1/3p that1 well 3s/p.uncle 3p.\textsc{unm} well 3.\textsc{dat} grab-Np-\textsc{fu}.3p\\
\glt`Those big (bowls) that we put just now, all right, the uncles may take those from them.'
\z

The following example, taken from Bible translation, has a highlighted topic marker -\textstyleStyleVernacularWordsItalic{na} on the sentence-initial topic \textstyleAcronymallcaps{NP,} which is part of the main clause\textstyleAcronymallcaps{,} and \textstyleStyleVernacularWordsItalic{ena} inside the relative clause:

\ea%x1816
\label{ex:3:x1816}
\gll Ni Samaria=na [\textstyleEmphasizedVernacularWords{o} \textstyleEmphasizedVernacularWords{ena} me baliwep amis-ar-e-man nain]\textsubscript{RC} lotu on-i-man.\\
2p.\textsc{unm} Samaria=\textsc{tp} 3s.\textsc{unm} SPEC not well knowledge-\textsc{inch} that1 worship do-Np-\textsc{pr}.2p\\
\glt`You Samaritans worship [the one that you do not know well].'
\z

Without \textstyleStyleVernacularWordsItalic{ena} in the relative clause the sentence would mean `You Samaritans do not know him well but (still) worship him.'

The more common topic clitic \textstyleStyleVernacularWordsItalic{-(e)na}\footnote{The clitic has mostly lost the phoneme /e/, but it can sometimes be heard when the host word ends in a voiceless consonant. } is used to highlight a changed topic, to which attention is drawn. The topic may have been introduced in the immediately preceding clause. The use of this device is infrequent in texts. It can often be glossed with `as for X'. Highlighted topics are discussed in \sectref{sec:9.1.2.4}.

Example \REF{ex:3:x779} is from a traditional story, where a man has gone hunting and the spirit of his lover comes to his home. When the wife sees her, she knows what the spirit woman has come to look for and comments:

\ea%x779
\label{ex:3:x779}
\gll Nena mua=\textstyleEmphasizedVernacularWords{na} urema osarena ikiw-o-k. \\
2s.\textsc{gen} man=\textsc{tp} bandicoot path go-\textsc{pa}-3s\\
\glt`(As for) your husband, he went to catch bandicoots.'
\z

In \REF{ex:3:x1751} the answer to the question reveals the identity of the person asked about; the topic marker may be used even in a short exchange like this but especially if the text continues to tell more about the topic. 

\ea%x1751
\label{ex:3:x1751}
\gll Mua nain naareke? Mua nain=\textbf{na} owow saria maneka=ke. \\
Man that1 who.\textsc{cf} man that1=\textsc{tp} village headman big=\textsc{cf}\\
\glt`Who is that man? -That man is the big village headman.'
\z

In \REF{ex:3:x780} the speaker changes the topic to the addressee after a discussion on something else:

\ea%x780
\label{ex:3:x780}
\gll Nos=\textstyleEmphasizedVernacularWords{na}? \\
2s.\textsc{fc}=\textsc{tp}\\
\glt`(So,) what about you?'
\z

An important function for the topic marker -\textstyleStyleVernacularWordsItalic{na} is to mark conditional clauses (\sectref{sec:8.3.5}). This is a common function for topic markers in Papuan languages (\citealt{Haiman1978}; \citealt{Reesink1983b}, \citeyear[242]{Reesink1987}; \citealt[203]{Foley1986}).

\ea%x744
\label{ex:3:x744}
\gll Opora wiar ika-i-ya=\textstyleEmphasizedVernacularWords{na} eliw urup-ep wia maak-uk.\\
talk 3.\textsc{dat} be-Np-\textsc{pr}.3s=\textsc{tp} well ascend-\textsc{ss}.\textsc{seq} 3p.\textsc{acc} tell-\textsc{imp}.3p\\
\glt`If they have something to say, they can get up and tell them.'
\z

\ea%x745
\label{ex:3:x745}
\gll O emeria aaw-owa kookal-ek-a-k=\textstyleEmphasizedVernacularWords{na} iw-ek-a-mik.\\
3s.\textsc{unm} woman get-\textsc{nmz} like-\textsc{cntf}-\textsc{pa}-3s=\textsc{tp} give.him-\textsc{cntf}-\textsc{pa}-1/3p\\
\glt`If he had liked to get a wife, they would have given him one.'
\z

It is also used in adversative subordinate clauses (\sectref{sec:8.3.4}) when the main clause expresses a frustrated effort or a cancelled expectation \REF{ex:3:x1399}, or surprise.

\ea%x1399
\label{ex:3:x1399}
\gll Ikiw-ep mukuna nain umuk-a-mik=\textstyleEmphasizedVernacularWords{na} me pepek. \\
go-\textsc{ss}.\textsc{seq} fire that1 extinguish-\textsc{pa}-1/3p=\textsc{tp} not enough/able\\
\glt`We went and (tried to) extinguish the fire, but couldn't.'
\z

\subsubsection{Focus clitics}\label{sec:3:z:y:x}
%\hypertarget{RefHeading21541935131865}
{}
There are two focus clitics, the contrastive focus marker -(\textstyleStyleVernacularWordsItalic{e})\textstyleStyleVernacularWordsItalic{ke} and the neutral focus marker -\textstyleStyleVernacularWordsItalic{ko}, which has developed from the indefinite \textstyleStyleVernacularWordsItalic{oko} `a certain, other'.\footnote{Most of this section is based on J\"arvinen (1988b:81--96).} The main candidate for the contrastive focus marker is the subject of a noun phrase \REF{ex:3:x781}. When the object is fronted as a theme (\sectref{sec:9.1}), the subject usually gets the contrastive focus marking to distinguish it from the object \REF{ex:3:x782}. 

\ea%x781
\label{ex:3:x781}
\gll Iiriw ifa marasin=\textstyleEmphasizedVernacularWords{ke} kekan-e-k. \\
earlier snake poison=\textsc{cf} be.strong-\textsc{pa}-3s\\
\glt`The snake poison had already taken effect.'
\z

\ea%x782
\label{ex:3:x782}
\gll Episowa ifa nain atua=\textstyleEmphasizedVernacularWords{ke} en-e-k. \\
tobacco leaf that1 worm=\textsc{cf} eat-\textsc{pa}-3s\\
\glt`The tobacco leaves were eaten by worms.'
\z

Another possible host is the non-verbal predicate of a verbless clause (\sectref{sec:5.6}).

\ea%x783
\label{ex:3:x783}
\gll Iperuma nain me enim-eka, inasin mua=\textstyleEmphasizedVernacularWords{ke}. \\
eel that1 not eat-\textsc{imp}.2p spirit man=\textsc{cf}\\
\glt`Do not eat that eel, it is a spirit man.'
\z

There are a few isolated cases where it occurs on some other contrasted element of a clause.

\ea%x784
\label{ex:3:x784}
\gll Amirika=\textstyleEmphasizedVernacularWords{ke} eliw ika-i-yem, uura=\textstyleEmphasizedVernacularWords{ke} napum-ar-i-yem. \\
noon=\textsc{cf} well be-Np-\textsc{pr}.1s night=\textsc{cf} sickness-\textsc{inch}-Np-\textsc{pr}.1s\\
\glt`At noon I'm well, at night I am sick.'
\z

Contrastive focus as a pragmatic device in a text is discussed in \sectref{sec:9.3.1}.

The neutral focus clitic \textstyleStyleVernacularWordsxiiptItalic{-ko} commonly occurs in irrealis-type clauses:\footnote{Because of this, it was called \textit{Irrealis focus clitic} in J\"arvinen (1988b).} questions, commands, negative clauses, or those with future tense. Unlike the contrastive focus marker, the neutral focus marker can be attached to almost any element of a clause except the final verb. 

\ea%x785
\label{ex:3:x785}
\gll Mukuna=\textstyleEmphasizedVernacularWords{ko} op-a-man=i? \\
fire=\textsc{nf} hold-\textsc{pa}-2p=\textsc{qm}\\
\glt`Did you hold any fire?'
\z

\ea%x786
\label{ex:3:x786}
\gll Mua nain=\textstyleEmphasizedVernacularWords{ko} onak-e! \\
man that1=\textsc{nf} give.to.eat-\textsc{imp}.2s\\
\glt`Give it to that man to eat!'
\z

\ea%x787
\label{ex:3:x787}
\gll Oposia en-e-man nain yo=\textstyleEmphasizedVernacularWords{ko} me uruf-a-m. \\
meat eat-\textsc{pa}-2p that1 1s.\textsc{unm}=\textsc{nf} not see-\textsc{pa}-1s\\
\glt`I didn't (even) see the meat that you ate.'
\z

\ea%x788
\label{ex:3:x788}
\gll Akim-ap=\textstyleEmphasizedVernacularWords{ko} uruf-i-yen. \\
try-\textsc{ss}.\textsc{seq}=\textsc{nf} see-Np-\textsc{fu}.1p\\
\glt`We'll try and see.'
\z

Neutral focus as a textual device is discussed further in \sectref{sec:9.3.2}.

\subsection{Question marker}\label{sec:3:y:x}
%\hypertarget{RefHeading21561935131865}
{}
The question marker -\textstyleStyleVernacularWordsItalic{i} is a sentential clitic, used to form polar questions, and it attaches itself to the clause-final verb or another clause-final element. Its relationship to the alternative connective \textstyleStyleVernacularWordsItalic{e} `or' (\sectref{sec:3.11.2}) is unclear; it is possible that \textstyleStyleVernacularWordsItalic{i} was originally an alternative connective but was also employed as a question marker and became so established in this function that a new alternative connective \textstyleStyleVernacularWordsItalic{e} developed. It is not uncommon in \textstyleAcronymallcaps{TNG} Madang languages that the question marker and the alternative connective are either the same or closely related.\footnote{In Usan and Amele the alternative connector and the question marker are the same (\citealt[293]{Reesink1987},\citealt[99]{Roberts1987}). Maia has \textit{-i} `QM' and \textit{e} `or' like Mauwake \citep[83,159]{Hardin2002}. Bargam has borrowed the Tok Pisin \textit{o} as an alternative connector but has retained the clitic \textit{-e} as the question marker \citep[53,122]{Hepner2002}. Kobon may use the alternative interrogative connector \textit{aka} `or' sentence-finally in leading polar questions, where the speaker expects the addressee to agree with the proposition. } 

\ea%x789
\label{ex:3:x789}
\gll Sira nain piipua-i-nan=\textstyleEmphasizedVernacularWords{i}? \\
habit that1 leave-Np-\textsc{fu}.2s=\textsc{qm}\\
\glt`Will you give up that habit?'
\z

\ea%x790
\label{ex:3:x790}
\gll Nobonob ikiw-e-man nain, owowa eliwa=\textstyleEmphasizedVernacularWords{i}? \\
Nobonob go-\textsc{pa}-2p that1 village good=\textsc{qm}\\
\glt`You went to Nobonob, is it a good village?'
\z

The question marker can also be used in statements, when two or more alternatives are given:

\ea%x791
\label{ex:3:x791}
\gll Mua kuisow manina erup=\textstyleEmphasizedVernacularWords{i} (e) arow=\textstyleEmphasizedVernacularWords{i} (e) naap. \\
man one garden two=\textsc{qm} (or) three=\textsc{qm} (or) thus\\
\glt`A man can have two gardens, or three, like that.'
\z

\subsection{Co-occurrence of the clitics}\label{sec:3:y:x}
%\hypertarget{RefHeading21581935131865}
{}
It is possible to have two or more clitics attached to the same word; only the topic marker does not allow other clitics with it. A case marking clitic, forming a constituent with the preceding noun phrase, is placed first. Next comes the focus clitic \REF{ex:3:}, with the scope over a phrase but not forming a constituent. The limiter \nobreakdash-\textstyleStyleVernacularWordsItalic{iw} may have a phrase with the focus marking in its scope, so it follows the focus marker. When the limiter follows another clitic, there is a transition consonant /r/ between the clitics.The modal clitic \nobreakdash-\textstyleStyleVernacularWordsItalic{yon} `perhaps' (\sectref{sec:3.9.3}) may have a scope over a whole predication, so its position is after the limiter. The sentential clitic, with a scope over the whole sentence, comes last. When the contrastive focus marker -\textstyleStyleVernacularWordsItalic{ke} and the question marker -\textstyleStyleVernacularWordsItalic{i} are adjacent they become a portmanteau clitic -\textstyleStyleVernacularWordsItalic{ki} \REF{ex:3:}.

\ea%x792
\label{ex:3:x792}
\gll Fura=\textstyleEmphasizedVernacularWords{iw}=\textstyleEmphasizedVernacularWords{ko} me puuk-a-mik. \\
knife=\textsc{inst}=\textsc{nf} not cut-\textsc{pa}-1/3p\\
\glt`They didn't cut it with a knife.'
\z

\ea%x795
\label{ex:3:x795}
\gll Fikera=\textstyleEmphasizedVernacularWords{pa}-r=\textstyleEmphasizedVernacularWords{iw} fiirim-eka. \\
kunai.grass=\textsc{loc}-{\O}=\textsc{lim} gather-\textsc{imp}.2p\\
\glt`Gather them right at the \textstyleForeignWords{kunai} grass.'
\z

\ea%x796
\label{ex:3:x796}
\gll Os=\textstyleEmphasizedVernacularWords{ke}-r=\textstyleEmphasizedVernacularWords{iw} maa en-emi ewur-ar-e-k. \\
3s.\textsc{fc}=\textsc{cf}-{\O}=\textsc{lim} food eat-\textsc{ss}.\textsc{sim} haste-\textsc{inch}-\textsc{pa}-3s\\
\glt`Only he rushed with his food.'
\z

\ea%x793
\label{ex:3:x793}
\gll Wi anim onoma pun o makena Krais=ke na-i-mik=\textstyleEmphasizedVernacularWords{yon}=\textstyleEmphasizedVernacularWords{i}?\\
3p.\textsc{unm} blade basis also 3s.\textsc{unm} truly Christ=\textsc{cf} say-Np-\textsc{pr}.1/3p=perhaps=\textsc{qm}\\
\glt`Are also the authorities perhaps saying that he truly is Christ?'
\z

\ea%x794
\label{ex:3:x794}
\gll Mua fain Saror muuka=\textstyleEmphasizedVernacularWords{ki}? \\
man this Saror son=\textsc{cf}.QM\\
\glt`Is this man Saror's son?'
\z

\section{Interjections}\label{sec:3:13}
%\hypertarget{RefHeading21601935131865}
{}
There are a lot of interjections in Mauwake; the following list is not even nearly exhaustive. The pronunciation of interjections may differ from that of other words: intonational variations are greater, and lengthening, even extreme lengthening, of the final vowel is common. Interjections are not part of the normal clause structure, and they usually occur either sentence-initially or finally. Some can also be placed between clauses in a coordinate sentence \REF{ex:3:}. The glosses given in \tabref{tab:3:interjections} for the interjections are just rough approximations. 

\begin{table}
\begin{tabular}{ll}
\mytoprule
a &impatience, also used as a filler\\
aa &`oh' emphasizes what has been said\\
ae &`yes' agreement\footnote{The negators \textit{weetak} and \textit{wia} `no' also function as interjections but are not listed here, because they have other functions as well (see \sectref{sec:3.10} and \sectref{sec:6.2}).} \\
aiyoo &distress, disapproval\\
arika\footnote{This is obviously an imperative derived from the discourse-marker \textit{aria} (\sectref{sec:3.9.1}), as it is only used with second person plural, whereas \textit{aria} is used with all other persons. Elsewhere \textit{aria} has no verb-like qualities.} &`OK, let's go' exhorting others to get going\\
awue &`wow any strong emotion, surprise\\
ee &delight\\
ei &`hey!' being surprised or startled\\
emawa &`sorry' expression of empathy, especially grief or pity\\
emawik &`excuse me' speech opening in a controversial situation\\
faa &disgust or astonishment\\
maa senam &`watch out!' grave warning (lit: `thing too.much')\\
oo &`o' calling someone\\
oo [ooɁ] &`yes' agreement\\
na &strengthens an imperative\\
nii? &`really? \textstyleDefinitionE{oh?' r}\textstyleEncyclopedicinfoE{esponse to hearing something surprising} \\
nom &\textstylefstandard{ `please' when repeating a request or command}\\
noma &\textstylefstandard{ `oh dear' } \\
sa, se &impatience, disapproval\\
se-ek &`wow great happiness\\
wiisak &`sorry' mild regret, for minor losses\\
yaa &impatience\\
yee &`oh' recognition, emphasis\\
yii &`eek', `oh' fear, sorrow\\
\mybottomrule 
\end{tabular}
\caption{Interjections}
\label{tab:3:interjections}
\end{table}


\ea%x797
\label{ex:3:x797}
\gll \textstyleEmphasizedVernacularWords{Aa}, kema=ko kir-ek-a-n \textstyleEmphasizedVernacularWords{aa}! \\
oh liver=\textsc{nf} turn-\textsc{cntf}-\textsc{pa}-2s oh\\
\glt`Oh, if only you had changed your ways (lit: turned your liver)!'
\z

\ea%x798
\label{ex:3:x798}
\gll \textstyleEmphasizedVernacularWords{Arika}, takira, yo yook-eka. \\
let's.go boy 1s.\textsc{unm} follow.me-\textsc{imp}.2p\\
\glt`Alright boys, follow me (and will get going).'
\z

\ea%x1225
\label{ex:3:x1225}
\gll Laman tapala wu-a-k, \textstyleEmphasizedVernacularWords{aiyoo}! \\
Laman hat put-\textsc{pa}-3s \textsc{intj}\\
\glt`My goodness, Laman put a hat on (and exposed himself and us to the fighter pilots)!'
\z

\ea%x799
\label{ex:3:x799}
\gll \textstyleEmphasizedVernacularWords{Emawa}, nena niawi um-o-k. \\
sorry 2s.\textsc{gen} 2s/p.father die-\textsc{pa}-3s\\
\glt`Sorry, your father is dead.'
\z

\ea%x800
\label{ex:3:x800}
\gll Naap ma-emi om-em-ika-i-nan, \textstyleEmphasizedVernacularWords{na}. \\
thus say-\textsc{ss}.\textsc{sim} cry-\textsc{ss}.\textsc{sim}-be-Np-\textsc{fu}.2s \textsc{intj}\\
\glt`You must cry saying like that.'
\z

\ea%x801
\label{ex:3:x801}
\gll Yo damol-al-e-m \textstyleEmphasizedVernacularWords{oo}, fiker fufa iw-a-m \textstyleEmphasizedVernacularWords{oo}. \\
1s.\textsc{unm} bad-\textsc{inch}-\textsc{pa}-1s oh kunai.grass old.grass enter-\textsc{pa}-1s oh\\
\glt` Oh, I'm in a bad way, I am hiding among the grass.'
\z

\ea%x802
\label{ex:3:x802}
\gll Ni kaaneke ik-e-man \textstyleEmphasizedVernacularWords{oo}, ni ekap-omak-eka oh\\
2p.\textsc{unm} where be-\textsc{pa}-2p oh 2p.\textsc{unm} come-\textsc{distr}/\textsc{pl}-\textsc{imp}.2p \textstyleEmphasizedVernacularWords{oo}!\\
\glt`O where are you? -- come!'
\z


\ea%x803
\label{ex:3:x803}
\gll \textstyleEmphasizedVernacularWords{Yii}, ifa=ke \textstyleEmphasizedVernacularWords{yee}! \\
Eek snake=\textsc{cf} \textsc{intj}\\
\glt`Eek, that's a snake oh!
\z
