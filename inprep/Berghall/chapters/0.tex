%0

%ISBN  978-952-10-6416-6  (PDF) \textbf{(NB. this is dissertation ISBN; manuscript has been edited since)}

\section*{Acknowledgements}

There are several people without whom this description of Mauwake grammar would not have become a reality, and whom I want to thank from my heart.

My colleague for 25 years and a close friend since 1977, Kwan Poh San shared the joys and burdens of life and work with me during the whole of the Mauwake project. We learnt the language together and analysed it together, and although I have written this grammar, she has also contributed significantly towards it.  There are many sections where some of the analysis was done by her and some by myself, but since we worked together it is sometimes hard to distinguish who did which part. Her oral command of the language is better than mine, and I have benefitted from her insights and comments during the writing process.  

The Mauwake people welcomed Kwan Poh San and myself to live with them in Moro village and the family of Leo Magidar adopted us as their daughters. The people built our house, and brought us food. They taught us their customs and shared their everyday lives with us for those over 20 years that we lived in Moro. Although we naturally had more contact with the people in Moro village, the inhabitants of the other Mauwake villages also showed their hospitality and friendship to us. For this I thank them all. 

Several people helped us with language learning and analysis. Saror Aduna first became the main language teacher and later the co-translator for the whole time we were working with the Mauwake people. Others who gave us texts include Kalina Sarak, Balthasar Saakawa, Kululu Sarak, Albert Kiramaten, Alois Amdara, Charles Matuwina, John Meldia, Aduna, Kedem Saror, Bang, Kuumu, Komori, Darawin, and Muandilam.The New Testament checking committee members Balthasar Saakawa, Lukas Miime, Charles Matuwina, Lawrence Alinaw and Leo Nimbulel helped us to understand the language better during our long checking sessions.

Without the encouragement of Professor Fred Karlsson I would have given up many times. He believed in writing descriptive grammars even when it was not fashionable in linguistics, and gave me his unwavering support during the many years that this grammar was in the making.

In my early years in Papua New Guinea, when I was not yet excited about grammar,  Bob Litteral and Ger Reesink encouraged me to take up a study program. The numerous discussions with many SIL-PNG colleagues, Cynthia Farr, Larry Lovell, Robert Bugenhagen, Dorothy James, John Roberts, Carl Whitehead, Eileen Gasaway, Ren\'e van den Berg, Catherine McGuckin and several others, helped me to gain a better understanding both of  PNG languages and of linguistics. Betty Keneqa, the librarian in the SIL linguistic library, was always helpful in locating material and making photocopies. The instruction received from the international linguistics consultants Thomas Payne and David Weber was inspiring and practical.

Others whose teaching, writings and/or personal interaction have shaped my thinking and writing include the late John Verhaar, Bernard Comrie, Talmy Giv\'on, Malcolm Ross and Andrew Pawley, and many others.  

The comments from the external readers of the dissertation, Malcolm Ross and Ger Reesink, helped me to clarify, modify and reorganise the text, and even re-analyse some of the data for the final version.

The financial assistance through scholarships from the Finnish Cultural Foundation, SIL International and SIL-PNG are gratefully acknowledged. My employer, the Finnish Evangelical Lutheran Mission, allowed some working time to be used for studies, and some of the writing was done as my work assignment. 

I thank my husband Jouko for his love and encouragement, and for being there to remind me that there is a lot more to life besides language work.

My greatest thanks go to God, who in his Word gives life, and love, and hope. 



\section*{Abbreviations and symbols}

\begin{tabular}{ll}
ACC & accusative nn, \\
ADD & additive connective \\
ADJ & adjective \\
ADV & adverb(ial) \\
AdvP & adverbial phrase \\
AP & adjective phrase \\
APP & apposition(al) \\
ASP & aspect \\
ASSOC & associative \\
AUX & auxiliary \\
BEN & benefactive \\
BNFY1 & beneficiary 1/2singular \\
BNFY2 & beneficiary non-1/2singular \\
BPx & bring-prefix \\
CAUS & causative \\
CC & complement clause \\
CF & contrastive focus \\
CL & clause \\
CNJ & connective \\
CNTF & counterfactual \\
COM & comitative \\
CMPL & completive aspect \\
CONT & continuous aspect \\
COORD & coordinate \\
CTV & complement-taking verb \\
\end{tabular}

\begin{tabular}{ll}
DAT & dative \\
DEM & demonstrative deictic \\
DISTR/A & distributive: ``all'' \\
DISTR/PL & distributive: ``many'' \\
d & dual \\
DS & different subject following \\
EXC & exclamation, interjection \\
FC & focal (pronoun) \\
FU & future tense \\
GEN & genitive \\
HAB & habitual \\
HN & head noun \\
IMP & imperative \\
INAL & inalienably possessed noun \\
INC & inceptive \\
INCH & inchoative \\
INSTR & instrument \\
INTJ & interjection \\
ISOL & isolative \\
LIM & limiter \\
LOC & locative \\
MAN & manner \\
NEG & negation \\
NF & neutral focus \\
NMZ & nominaliser \\
N & noun \\
\end{tabular}

\begin{tabular}{ll}
NP & noun phrase \\
Np & non-past \\
NVP & non-verbal predicate \\
O & object \\
p, & pl plural \\
P & phrase \\
PA & past tense \\
PAT & patient \\
POSS & possessive \\
PR & present tense \\
QM & question marker \\
QP & quantifier phrase \\
RC & relative clause \\
RDP & reduplication \\
REC & recipient \\
REFL & reflexive \\
s & singular \\
S & subject \\
SEQ & sequential action \\
SIM & simultaneous action \\
SPEC & specifier \\
SR & switch-reference \\
SS & same subject following \\
TH & theme \\
TNG & Trans-New Guinea \\
TP & topic \\
\end{tabular}

\begin{tabular}{ll}
T.P. & Tok Pisin \\
UNM & unmarked (pronoun) \\
v, & V verb \\
1 & first person \\
2 & second person \\
3 & third person \\
* & ungrammatical \\
? & questionable \\
/ & or \\
/ & / phonemic transcription \\
 {} [ & ] phonetic transcription \\
( & ) variant; optional \\
. & syllable break \\
- & morpheme break \\
= & clitic break \\
{\O} & zero morpheme \\
{{\textprimstress}} & primary stress \\
{{\textprimstress}}{{\textprimstress}} & secondary stress \\
\end{tabular}
