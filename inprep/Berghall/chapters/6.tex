%6

\chapter{Functional domains}
%\hypertarget{RefHeading22301935131865}

This chapter describes various functional systems that affect the clause or the sentence as a whole.  Most of them are touched upon in various other parts of the grammar where they are relevant, but here they are treated in a more systematic manner.

\section{Modality}
%\hypertarget{RefHeading22321935131865}

Modality, or mode -- expressing the speaker's attitude to a situation -- relates not just to the verb but to the whole proposition. Because of this it is typically not expressed via verbal inflection \citep[22]{Bybee1985}.  In Mauwake the counterfactual modality is manifested by a suffix on the verb 
\sectref{sec:3.8.3.2}); more often the modality is conveyed via various other strategies outlined below.

Many Papuan languages make a distinction between realis and irrealis mode,\footnote{\citet[158]{Foley1986} calls it \textit{status}.} and tense. \citet[162]{Foley1986} estimates that, on the whole, tense is more prominent than mode, but there are also languages like Hua \citep{Haiman1980} and Maia \citep{Hardin2002} which do not have tense as a verbal category at all, only mode. But in Mauwake the realis-irrealis dichotomy is not grammatically relevant. 

\subsection{Epistemic modality}
%\hypertarget{RefHeading22341935131865}

Epistemic modality has to do with certainty, probability and possibility: it ``relates to the speaker's {\dots} commitment to the probability that the situation is true'' \citep[234]{Payne1997}.  

The default  and unmarked mood in statements is indicative, when something is stated as a fact. If the speaker wants to strengthen the proposition more, the intensity adverb \textstyleStyleVernacularWordsItalic{akena} `truly, very' is added to the end of the statement.

\ea%x1050
\label{ex:6:x1050}
\gll Wi  owow  oko  oko  pun  wia  maake-miaw-i-yem \textstyleEmphasizedVernacularWords{akena}. \\
3p.\textsc{unm}  place  other  other  also  3p.\textsc{acc}  tell-wander-Np-\textsc{pr}.1s truly\\
\glt `I really walk around telling people in many other places too.'
\z
\todo[inline]{I wonder whether `Np' needs to be formatted to `\textsc{np}' in the glossing. If so, replace all the `-Np' will do.}

\ea%x1051
\label{ex:6:x1051}
\gll Wi  o  ook-owa  nain  me  pepek  \textstyleEmphasizedVernacularWords{akena}. \\
3p.\textsc{unm}  3s.\textsc{unm}  follow.him-\textsc{nmz}  that1  not  able  truly\\
\glt `They really are not able to follow him.'
\z

When the proposition is considered less than certain, either probable or just possible, the modal adverb clitic -\textstyleStyleVernacularWordsItalic{yon} `probably/perhaps/I think' (\sectref{sec:3.9.3}) is attached to the last word in the statement, usually either a verb or non-verbal predicate. An interjection can still follow the word with -\textstyleStyleVernacularWordsItalic{yon}.

\ea%x1052
\label{ex:6:x1052}
\gll Mua  Maneka=ke  lawisiw  wia  amukar-e-k=\textstyleEmphasizedVernacularWords{yon}. \\
man  big=\textsc{cf}  somewhat  3p.\textsc{acc}  scold-\textsc{pa}-3s-perhaps\\
\glt `Perhaps God reproached/punished them a little.'
\z

\ea%x1053
\label{ex:6:x1053}
\gll Nis  pun  kema  puk-owa  marewa=ke=\textstyleEmphasizedVernacularWords{yon}  aa! \\
2p.\textsc{unm}  also  liver  burst-\textsc{nmz}  none=\textsc{cf}-perhaps  \textsc{exc}\\
\glt `Ah, I think you don't have any sense at all (lit: your liver hasn't burst)!'
\z

\ea%x1071
\label{ex:6:x1071}
\gll Naap=\textstyleEmphasizedVernacularWords{yon}. \\
thus-perhaps\\
\glt `I think/suppose it is like that.'
\z

The counterfactual form of the verb (\sectref{sec:3.8.3.2}) is used in speculative statements where the situation mentioned in the proposition \textstyleEmphasizedWords{\textsc{did not}} happen, although it could have. 

\ea%x1054
\label{ex:6:x1054}
\gll Lawisiw  akena  um-\textstyleEmphasizedVernacularWords{ek}-a-m. \\
somewhat  very  die-\textsc{cntf}-\textsc{pa}-1s\\
\glt `I very nearly fell (but in reality didn't).'
\z

\ea%x1055
\label{ex:6:x1055}
\gll Yena  kookal-owa=pa  uuw-\textstyleEmphasizedVernacularWords{ek}-a-m=na  sesa  na-ek-a-m. \\
1s.\textsc{gen}  like-\textsc{nmz}=\textsc{loc}  work-\textsc{cntf}-\textsc{pa}-1s=\textsc{tp}  price  say-\textsc{cntf}-\textsc{pa}-1s\\
\glt `If I had worked on my own will, I would have required payment.'
\z

\ea%x1056
\label{ex:6:x1056}
\gll Naap  wiar  amis-ar-\textstyleEmphasizedVernacularWords{ek}-a-mik  oo! \\
thus  3.\textsc{dat}  knowledge-\textsc{inch}-\textsc{cntf}-\textsc{pa}-1/3p  \textsc{exc}\\
\glt `Oh, if only we had known that about him/them!'
\z

Abilitative is expressed by the adverb \textstyleStyleVernacularWordsItalic{pepek} `enough, correctly, able' as a non-verbal predicate. In affirmative statements the verb that the adverb refers to often occurs in the following clause:

\ea%x1089
\label{ex:6:x1089}
\gll No  \textstyleEmphasizedVernacularWords{pepek},  eliw  on-i-nan. \\
2s.\textsc{unm}  able  well  do-Np-\textsc{fu}.2s\\
\glt `You are able, you can do it.'
\z

The verb may also be in the same clause but in the nominalized form; this is more common in negative statements:

\ea%x1088
\label{ex:6:x1088}
\gll {\dots} mukuna  nain \textstyleEmphasizedVernacularWords{umuk-owa}  \textstyleEmphasizedVernacularWords{me} \textstyleEmphasizedVernacularWords{pepek}. \\
{\dots} fire that1  extinguish-\textsc{nmz} not  able\\
\glt `{\dots} (we were) not able to extinguish the fire.'
\z

Evidentials are an important feature in some Papuan languages, but Mauwake does not have them as a grammatical category.

\subsection{Deontic modality} \label{sec:6.1.2}
%\hypertarget{RefHeading22361935131865}

The deontic modality indicates obligation or permission. Deontic expressions can vary from a statement of a strong obligation to a polite request or to expressions of permission or denying permission. 

The syntactic strategy for expressing strong obligation is to use the nominalized form of the verb followed by the contrastive focus clitic, and optionally an appropriate form of the verb `be'. 

\ea%x1077
\label{ex:6:x1077}
\gll Yo  uurika  owow  maneka  \textstyleEmphasizedVernacularWords{ikiw-owa=ke}  \textstyleEmphasizedVernacularWords{(ik-ua)}. \\
1s.\textsc{unm}  tomorrow  village  big  go-\textsc{nmz}=\textsc{cf}  be-\textsc{pa}.3s\\
\glt `I have to go to town tomorrow.'
\z

A nominalized clause structure may may be interpreted to express obligation even without the contrastive focus clitic, and in a medial clause. A dative pronoun is added if clarification is needed to state who is obligated to do something. 

\ea%x1079
\label{ex:6:x1079}
\gll \textstyleEmphasizedVernacularWords{Ekap-owa}  \textstyleEmphasizedVernacularWords{efar}  \textstyleEmphasizedVernacularWords{ika-eya}  ekap-e-m. \\
come-\textsc{nmz}  1s.\textsc{dat}  be-2/3s.\textsc{ds}  come-\textsc{pa}-1s\\
\glt `I had to come, so I came.'
\z

A polite request can also take the form of a question.

\ea%x1163
\label{ex:6:x1163}
\gll No  maa  nain=ko  eliw  yi-i-nan=i? \\
2s.\textsc{unm}  thing  that=\textsc{nf}  well  give.me-Np-\textsc{fu}T.2s=\textsc{qm}\\
\glt `Will/would you give that to me (please)?'
\z

Permission is indicated by the adverb \textstyleStyleVernacularWordsItalic{eliw} `well/all right' placed before the verb, which is in the future form.

\ea%x1085
\label{ex:6:x1085}
\gll Yiena  miiwa  kuisow,  \textstyleEmphasizedVernacularWords{eliw}  feeke  soop-i-yen. \\
1p.\textsc{gen}  land  one  well  here.\textsc{cf}  bury-Np-\textsc{fu}.1p\\
\glt `Our land is one, we may bury him here.'
\z

\ea%x1086
\label{ex:6:x1086}
\gll \textstyleEmphasizedVernacularWords{Eliw}  ek-ap  fook-i-nan,  fook-ap  ep-i-nan. \\
well  come-\textsc{ss}.\textsc{seq}  split-Np-\textsc{fu}.2s  split-\textsc{ss}.\textsc{seq}  go-Np-\textsc{fu}.2s\\
\glt `You may come and split (coconuts), and having split them, go.'
\z

Prohibition or denial of permission is done with a negated nominalized form of a verb.

\ea%x1087
\label{ex:6:x1087}
\gll Manin  maneka  na-isow  nena  kookal-owa=pa  \textstyleEmphasizedVernacularWords{perek-owa} \textstyleEmphasizedVernacularWords{weetak.} \\
garden  big  2s.\textsc{isol}  2s.\textsc{gen}  like-\textsc{nmz}=\textsc{loc}  harvest-\textsc{nmz} no\\
\glt `The big garden you are not allowed to harvest by yourself according to your liking.'
\z

\ea%x1078
\label{ex:6:x1078}
\gll I  \textstyleEmphasizedVernacularWords{me}  sira  samora  \textstyleEmphasizedVernacularWords{on-owa=ke}, weetak.\\
1p.\textsc{unm}  not  custom  bad  do-\textsc{nmz}=\textsc{cf}  no\\
\glt `We must not do bad things.'
\z

In sentences expressing disobedience to a prohibition, it is particularly common to have the prohibition in a relative clause where the nominalized verb is negated with the verbal negator \textstyleStyleVernacularWordsItalic{me}. Here the contrastive focus clitic is not used. 

\ea%x1887
\label{ex:6:x1887}
\gll Maa=ko  [\textstyleEmphasizedVernacularWords{me}  \textstyleEmphasizedVernacularWords{on-owa}  nain]  nis=ke  on-i-man. \\
thing=\textsc{nf}  not  do-\textsc{nmz}  that1  2p.\textsc{fc}=\textsc{cf}  do-Np-\textsc{pr}.2p\\
\glt `You do things that must not be done.'
\z

\ea%x1888
\label{ex:6:x1888}
\gll Sabat  fofa=pa  [\textstyleEmphasizedVernacularWords{me}  \textstyleEmphasizedVernacularWords{uuw-owa}  nain]  emeria  nain  saliw-a-k. \\
sabbath  day=\textsc{loc}  not  work-\textsc{nmz}  that1  woman  that1  heal-\textsc{pa}-3s\\
\glt `He healed the woman on a Sabbath day when it was forbidden to work.'
\z

\section{Negation}\footnotemark{}
%\hypertarget{RefHeading22381935131865}

\footnotetext{The contents of this section is mostly based on \citet{BerghETAL2006}.}
Mauwake has more variety in negation than many other Papuan languages.
\todo[inline]{The placement of the footnote mark above is unusual to me. Please check if it is inteneded to be placed right next the to section title} 
There are four negators in Mauwake instead of only one or two: \textstyleStyleVernacularWordsItalic{me}, \textstyleStyleVernacularWordsItalic{weetak}, \textstyleStyleVernacularWordsItalic{wia} and \textstyleStyleVernacularWordsItalic{marew}, which have somewhat overlapping functions (\sectref{sec:3.10}). Negation can also express frustration or be used as a verb root with certain suffixes; its scope can vary from one constituent to a whole sentence; and it may be emphasized. Double negation results in cancellation of the negation rather than emphasizing it. 

\subsection{Clausal negation} \label{sec:6.2.1}
%\hypertarget{RefHeading22401935131865}

Verbal clauses are negated with the negator \textstyleStyleVernacularWordsItalic{me} `not', placed before the verb \REF{ex:6:x1090}, verbal group \REF{ex:6:x1091} or verb phrase \REF{ex:6:x1092}. 
\todo[inline]{Example ex:x1113 is not referenced. Is this intended?}
This type of negation, also called standard negation, is symmetric in Mauwake: the negative clause is similar to the corresponding affirmative clause apart from the presence of the negator \citep[61--67]{Miestamo2005}. The negation strategy is the same for transitive and intransitive, independent and dependent clauses, and for imperatives as well. 

\ea%x1090
\label{ex:6:x1090}
\gll I  iinan  aasa  \textstyleEmphasizedVernacularWords{me}  \textstyleEmphasizedVernacularWords{kuuf-a-mik}. \\
1p.\textsc{unm}  sky  canoe  not  see-\textsc{pa}-1/3p\\
\glt `We did not see the airplanes.'
\z

\ea%x1091
\label{ex:6:x1091}
\gll Yo  \textstyleEmphasizedVernacularWords{me}  \textstyleEmphasizedVernacularWords{keker}  \textstyleEmphasizedVernacularWords{op-a-m},  Kedem=ke  makena. \\
1s.\textsc{unm}  not  fear  hold-\textsc{pa}-1s,  Kedem=\textsc{cf}  true\\
\glt `I was not afraid, true, Kedem was.'
\z

\ea%x1092
\label{ex:6:x1092}
\gll Mua  \textstyleEmphasizedVernacularWords{me}  \textstyleEmphasizedVernacularWords{wia}  \textstyleEmphasizedVernacularWords{kuuf-a-mik},  \textstyleEmphasizedVernacularWords{me}  \textstyleEmphasizedVernacularWords{wia}  \textstyleEmphasizedVernacularWords{furew-a-mik}, ne  \textstyleEmphasizedVernacularWords{me}  \textstyleEmphasizedVernacularWords{wia}  \textstyleEmphasizedVernacularWords{imen-a-mik}.\\
man  not  3p.\textsc{acc}  see-\textsc{pa}-1/3p  not  3p.\textsc{acc}  sense-\textsc{pa}-1/3p and  not  3p.\textsc{acc}  find-\textsc{pa}-1/3p\\
\glt `We did not see, sense, or find the men.'
\z

\ea%x1113
\label{ex:6:x1113}
\gll Ni \textstyleEmphasizedVernacularWords{uf-ep=na}  maadara  \textstyleEmphasizedVernacularWords{me}  \textstyleEmphasizedVernacularWords{iirar-eka}. \\
2p.\textsc{unm}  dance-\textsc{ss}.\textsc{seq}=\textsc{nf}  forehead.ornament  not  remove-\textsc{imp}.2p\\
\glt `If/when you have danced, do not remove your forehead ornaments.'
\z

The non-verbal predicate in equative and descriptive clauses can be negated with any of the four negators. 

\ea%x1093
\label{ex:6:x1093}
\gll O  somek  mua  \textstyleEmphasizedVernacularWords{weetak/wia}. \\
3s.\textsc{unm}  song  man  no\\
\glt `He is not a teacher.'
\z

\ea%x1095
\label{ex:6:x1095}
\gll O  \textstyleEmphasizedVernacularWords{me}  somek  mua=ke. \\
3s.\textsc{unm}  not  song  man=\textsc{cf}\\
\glt `He is not a teacher.' 
\z

However, \textstyleStyleVernacularWordsItalic{marew} is possible in these clauses only if the predicate contains an adjective.

\ea%x1096
\label{ex:6:x1096}
\gll Awuliak  fain  afila  \textstyleEmphasizedVernacularWords{weetak/wia/marew}. \\
sweet.potato  this  sweet  no\\
\glt `This sweet potato is not sweet.'
\z

\ea%x1097
\label{ex:6:x1097}
\gll Awuliak  fain  \textstyleEmphasizedVernacularWords{me}  afila(=ke). \\
sweet.potato  this  not  sweet=\textsc{cf}\\
\glt `This sweet potato is not sweet.'
\z

When the possessive and existential clauses are negated with the verbal negator \textstyleStyleVernacularWordsItalic{me}, they are like other verbal clauses. But if any of the other negators is used, the negator replaces the verb and becomes a non-verbal predicate, so these clauses become verbless clauses (\sectref{sec:5.6.3}).

\ea%x1098
\label{ex:6:x1098}
\gll I  sira  naap  \textstyleEmphasizedVernacularWords{me}  \textstyleEmphasizedVernacularWords{yiar}  \textstyleEmphasizedVernacularWords{ik-ua}. \\
1p.\textsc{unm}  custom  thus  not  1p.\textsc{dat}  be-\textsc{pa}.3s\\
\glt `We do not have a custom like that.'
\z

\ea%x1094
\label{ex:6:x1094}
\gll Wi  Yaapan  emeria  \textstyleEmphasizedVernacularWords{weetak},  mua  manek=iw. \\
3p.\textsc{unm}  Japan  woman  no,  man  big=\textsc{lim}\\
\glt `The Japanese had no women/wives, (they were) only men.'
\z

\ea%x1099
\label{ex:6:x1099}
\gll Iiriw  miiwa  muuta  nain  irak-owa  \textstyleEmphasizedVernacularWords{me}  \textstyleEmphasizedVernacularWords{ik-ua.} \\
earlier  land  for  that1  fight-\textsc{nmz}  not  be-\textsc{pa}.3s\\
\glt `Earlier there was no fighting for land.'
\z

\ea%x1100
\label{ex:6:x1100}
\gll Iiriw  miiwa  muuta  nain  irak-owa  \textstyleEmphasizedVernacularWords{marew}, {\dots} \\
earlier  land  for  that1  fight-\textsc{nmz}  no(ne)\\
\glt `Earlier there was no fighting for land, ...'
\z

With so many possible alternatives, the speaker has a choice of repeating the same negator or using different ones when several items are negated. Either strategy is used by good language users.

\ea%x1128
\label{ex:6:x1128}
\gll I  muuka  \textstyleEmphasizedVernacularWords{marew}  a,  i  wiipa  \textstyleEmphasizedVernacularWords{marew}  a. \\
1p.\textsc{unm}  son  no(ne)  oh  1p.\textsc{unm}  daughter  no(ne)  oh\\
\glt `We have no son, and we have no daughter.'
\z

\ea%x1127
\label{ex:6:x1127}
\gll I  urupa  \textstyleEmphasizedVernacularWords{weetak},  i  soomia  \textstyleEmphasizedVernacularWords{wia},  i epira  \textstyleEmphasizedVernacularWords{marew.}\\
1p.\textsc{unm}  cup  no  1p.\textsc{unm}  spoon  no  1p.\textsc{unm} plate  no(ne)\\
\glt `We had no cups, we had no spoons, we had no plates.'
\z

In a few cases the choice of a negator indicates a difference in meaning. The example \REF{ex:6:x1129} is the Mauwake equivalent for the common Tok Pisin idiom \textstyleForeignWords{nogat tok } `I do not have anything against it'.

\ea%x1129
\label{ex:6:x1129}
\gll Yo  opora  \textstyleEmphasizedVernacularWords{weetak/wia}. \\
1p.\textsc{unm}  talk  no\\
\glt `I have no talk. (= I do not have anything to say.)'
\z

\ea%x1130
\label{ex:6:x1130}
\gll Yo  opora  \textstyleEmphasizedVernacularWords{marew}. \\
1p.\textsc{unm}  talk  no(ne)\\
\glt `I have no talk. (= It is OK / I do not have anything against it.)'
\z

The predicate function of the negators \textstyleStyleVernacularWordsItalic{weetak} and \textstyleStyleVernacularWordsItalic{marew} is also shown in the fact that they take a medial different-subject suffix, when the verbless negative possessive or existential-presentative clauses occur sentence-medially in a chaining structure. \textstyleStyleVernacularWordsItalic{Wia} cannot be suffixed with the medial verb suffix.

\ea%x1101
\label{ex:6:x1101}
\gll Maa  pela  \textstyleEmphasizedVernacularWords{marew-eya}  /  \textstyleEmphasizedVernacularWords{weetak-eya}  fofa  er-a-m. \\
thing  leaf  no(ne)-2/3s.\textsc{ds}  /  no-2/3s.\textsc{ds}  market  go-\textsc{pa}-1s\\
\glt `I had no greens and went to the market.'
\z

\subsection{Constituent negation} \label{sec:6.2.2}
%\hypertarget{RefHeading22421935131865}

Papuan languages typically do not have lexicalized constituent negation of the type `nothing', `nobody' etc., and even syntactic constituent negation may be lacking \citep[271--272]{Reesink1987}. But in Mauwake it is possible to negate various constituents within a clause, and, although very rarely, even inside a noun phrase. The basic constituent negator is \textstyleStyleVernacularWordsItalic{me} `not'. It precedes the negated element, which receives extra stress. Position of the negator, stress, and sometimes the neutral focus clitic all interact in constituent negation.

\ea%x1102
\label{ex:6:x1102}
\gll \textstyleEmphasizedVernacularWords{Me}  \textstyleEmphasizedVernacularWords{napuma=ke}  ifakim-o-k. \\
not  sickness  kill-\textsc{pa}-3s\\
\glt `It wasn't sickness that killed him.'
\z

\ea%x1103
\label{ex:6:x1103}
\gll Maa  oposia  \textstyleEmphasizedVernacularWords{me}  \textstyleEmphasizedVernacularWords{ewur}  enim-i-mik. \\
thing  meat  not  soon  eat-Np-\textsc{pr}.1/3p\\
\glt `Meat we will not eat soon (after spouse's death).'
\z

\ea%x1108
\label{ex:6:x1108}
\gll \textstyleEmphasizedVernacularWords{Me}  \textstyleEmphasizedVernacularWords{epa}  \textstyleEmphasizedVernacularWords{fan}  irak-owa  uruf-a-mik. \\
not  place  here  fight-\textsc{nmz}  see-\textsc{pa}-1/3p\\
\glt `It was not here that they saw the fighting.'
\z

\ea%x1104
\label{ex:6:x1104}
\gll Nepa  opaimika  \textstyleEmphasizedVernacularWords{me}  \textstyleEmphasizedVernacularWords{baliwep}  miim-a-mik. \\
bird  talk  not  well  hear-\textsc{pa}-1/3p\\
\glt `They did not hear (understand) Tok Pisin well.'
\z

\ea%x1105
\label{ex:6:x1105}
\gll \textstyleEmphasizedVernacularWords{Me}  \textstyleEmphasizedVernacularWords{nomokowa}  \textstyleEmphasizedVernacularWords{eliwa} aaw-e\textstyleEmphasizedVernacularWords{-} mik. \\
not  tree  good  take-\textsc{pa}-1/3p\\
\glt `It wasn't good trees that they took.'
\z

In clauses with \textstyleEmphasizedWords{\textsc{quantifiers}}, constituent negation has an important function disambiguating the meaning. If the subject or object noun phrase has a quantifier, the negation is done in different ways depending on whether the quantifier is in the scope of the negation or not. In \REF{ex:6:x1142} the noun phrase with the particular quantifier \textstyleStyleVernacularWordsItalic{kuisow} `one' is not in the scope of the negation, but in \REF{ex:6:x1143} it is. The neutral focus clitic is required to clarify the meaning; it can even be attached to some other constituent between the quantifier and the negator \REF{ex:6:x1147}.

\ea%x1142
\label{ex:6:x1142}
\gll Mua  \textstyleEmphasizedVernacularWords{kuisow}  \textstyleEmphasizedVernacularWords{me}  ekap-o-k. \\
man  one  not  come-\textsc{pa}-3s\\
\glt `One (particular) man did not come.'
\z

\ea%x1143
\label{ex:6:x1143}
\gll Mua  \textstyleEmphasizedVernacularWords{kuisow}=\textstyleEmphasizedVernacularWords{ko}  \textstyleEmphasizedVernacularWords{me}  ekap-o-k. \\
man  one=\textsc{nf}  not  come-\textsc{pa}-3s\\
\glt `Not (even) one man came.'
\z

\ea%x1147
\label{ex:6:x1147}
\gll Mua  \textstyleEmphasizedVernacularWords{kuisow}  owowa=pa=\textstyleEmphasizedVernacularWords{ko}  \textstyleEmphasizedVernacularWords{me}  ik-ua. \\
man  one  village=\textsc{loc}=\textsc{nf}  not  be-\textsc{pa}.3s\\
\glt `Not (even) one man stayed in the village.'
\z

The example \REF{ex:6:x1145} is ambiguous as to whether only one man did not go down or whether it is negated that only one man went. The first alternative is the more likely meaning, and if one wants to make sure to give the second meaning, the standard strategy for constituent negation \REF{ex:6:x1144} is used.

\ea%x1145
\label{ex:6:x1145}
\gll Mua  kuisow  muuta  \textstyleEmphasizedVernacularWords{me}  ekap-o-k. \\
man  one  only  not  come-\textsc{pa}-3s\\
\glt `Only one man did not come.' Or: `Not only one man came (but more).'
\z

\ea%x1144
\label{ex:6:x1144}
\gll \textstyleEmphasizedVernacularWords{Me}  mua  kuisow  (muuta)  ekap-o-k. \\
not  man  one  (only)  come-\textsc{pa}-3s\\
\glt `Not only one man came.'
\z

Similarly, with the universal quantifier \textstyleStyleVernacularWordsItalic{unowiya} `all' the scope of the negation may be ambiguous. The preferred interpretation for \REF{ex:6:x1148} is that the statement about not following God's talk refers to all people, thus \textstyleEmphasizedWords{\textsc{no one}} follows it; but it may also be understood that even if all the people do not follow it, some do.

\ea%x1148
\label{ex:6:x1148}
\gll Emeria  mua  \textstyleEmphasizedVernacularWords{unowiya}  Mua  Maneka  opora  \textstyleEmphasizedVernacularWords{me}  ook-i-mik. \\
woman  man  all  Man  Big  talk  not  follow-Np-\textsc{pr}.1/3p\\
\glt `All the people do not follow God's talk.'
\z

If the negator is in the constituent negation position, the statement is unambiguous. In this respect Mauwake behaves differently from Usan, which does not allow a constituent negation structure \citep[275--277]{Reesink1987}.

\ea%x1149
\label{ex:6:x1149}
\gll Nain  \textstyleEmphasizedVernacularWords{me}  mua  \textstyleEmphasizedVernacularWords{unowiya}  opora  wiar  op-i-mik. \\
but  not  man  all  talk  3.\textsc{dat}  hold-Np-\textsc{pr}.1/3p\\
\glt `But not all men/people believe in him (= some do).'
\z

Stress may also be employed to give a constituent negation interpretation to a negated clause. When the clausal stress is on the negator, the whole clause is negated \REF{ex:6:x669}. In order to negate the universal quantifier rather than the verb, the main stress needs to be on the quantifier \REF{ex:6:x671}. This type of negation is used in Usan as well (ibid. 277).

\ea%x669
\label{ex:6:x669}
\gll Mua \textstyleEmphasizedVernacularWords{unow=iya}  \textstyleEmphasizedVernacularWords{'me}  \textstyleEmphasizedVernacularWords{ikiw-e-mik.} \\
man  many=\textsc{com}  not  go-\textsc{pa}-1/3p\\
\glt `All the men \textit{didn't go} (=none of them went).'
\z

\ea%x671
\label{ex:6:x671}
\gll Mua  \textstyleEmphasizedVernacularWords{'unow=iya} \textstyleEmphasizedVernacularWords{me}  ikiw-e-mik. \\
man  many=\textsc{com}  not  go-\textsc{pa}-1/3p\\
\glt `\textit{All} the men didn't go (=only some went).'
\z

When a clause with the universal quantifier \textstyleStyleVernacularWordsItalic{unow onaiya} `all' is negated, it tends to be interpreted as a constituent negation of the quantifier, possibly because \textstyleStyleVernacularWordsxiiptItalic{unow onaiya} is a heavier structure than \textstyleStyleVernacularWordsItalic{unowiya} and as such more prominent \REF{ex:6:x668}.

\ea%x668
\label{ex:6:x668}
\gll \textstyleEmphasizedVernacularWords{Unow}  \textstyleEmphasizedVernacularWords{onaiya}  \textstyleEmphasizedVernacularWords{me}  ikiw-e-mik. \\
many  with  not  go-\textsc{pa}-1/3p\\
\glt `Not all of them went (= only some went)'
\z

The following example is not a case of \textstyleStyleVernacularWordsItalic{unowa} negated separately inside a \textstyleAcronymallcaps{NP}; instead, \textstyleStyleVernacularWordsItalic{mua} `man' is fronted as a theme (\sectref{sec:9.1}):

\ea%x1150
\label{ex:6:x1150}
\gll Mua  \textstyleEmphasizedVernacularWords{me}  \textstyleEmphasizedVernacularWords{unowa}  ekap-e-mik. \\
man  not  many  come-\textsc{pa}-1/3p\\
\glt `There were not many men that came.' Or: `As for men, not many came.'
\z

\textstyleStyleVernacularWordsItalic{Eliwa} `good' may be the only adjective that can be negated by itself inside a noun phrase. These structures are very rare and would need a more careful study. \REF{ex:6:x1106} may also be a combination of a non-verbal clause and a transitive clause where the object \textstyleAcronymallcaps{NP} only retains the adjective; the noun is deleted because it occurs in the previous clause.

\ea%x1106
\label{ex:6:x1106}
\gll Maa  en-owa  \textstyleEmphasizedVernacularWords{eliw(a)}  \textstyleEmphasizedVernacularWords{marew}  p-or-o-mik. \\
thing  eat-\textsc{nmz}  good  no(ne)  Bpx-descend-\textsc{pa}-1/3p\\
\glt `They brought down not-good food.'
\z

\ea%x1107
\label{ex:6:x1107}
\gll Biiris  \textstyleEmphasizedVernacularWords{me}  \textstyleEmphasizedVernacularWords{eliwa},  damo-damola=ko  on-a-mik. \\
bridge  not  good  \textsc{rdp}-bad=\textsc{nf}  make-\textsc{pa}-1/3p\\
\glt `They did not make good bridges (but) bad ones.' (Or: The bridges were not good, they made bad ones.) 
\z

The following looks like a constituent negation attached to the noun \textstyleStyleVernacularWordsItalic{mokoka} `eye(s)', but actually \textstyleStyleVernacularWordsItalic{me}  here is a clausal negator negating the whole idiomatic sentence of `keeping one's eyes shut' (i.e. being ignorant).

\ea%x1114
\label{ex:6:x1114}
\gll \textstyleEmphasizedVernacularWords{Me}  mokoka  op-ar-ep  ik-e. \\
not  eye(s)  closed-\textsc{caus}-\textsc{ss}.\textsc{seq}  be-\textsc{imp}.2s\\
\glt `Do not have your eyes closed.'
\z

Those cases of constituent negation where \textstyleStyleVernacularWordsItalic{me}  precedes a verb can be distinguished from clausal negation only in spoken language on the basis of extra stress on the verb.

\ea%x1110
\label{ex:6:x1110}
\gll Ni  iperuma  fain  \textstyleEmphasizedVernacularWords{me}  \textstyleEmphasizedVernacularWords{e{\textprimstress}nim-eka},  wafur-eka! \\
2p.\textsc{unm}  eel  this  not  eat-\textsc{imp}.2p  throw-\textsc{imp}.2p\\
\glt `\textit{Don't eat}  this eel, throw it away!'
\z

\subsection{Negative interjection} \label{sec:6.2.3}
%\hypertarget{RefHeading22441935131865}

A negative interjection is used as a one-word reply to a question or a statement. It stands as a complete sentence by itself or is preposed and syntactically independent of the rest of the sentence. Two of the negators are used as negative interjections: \textstyleStyleVernacularWordsItalic{weetak} and \textstyleStyleVernacularWordsItalic{wia}. They are synonymous and usually interchangeable, but in a few environments one or the other is preferred.

\ea%x1115
\label{ex:6:x1115}
\gll No  aaya  sesenar-e-n=i?  -\textstyleEmphasizedVernacularWords{Weetak/wia}  (me  sesenar-e-m). \\
2s.\textsc{unm}  sugar  buy-\textsc{pa}-2s=\textsc{qm}  -no  (not  buy-\textsc{pa}-1s)\\
\glt `Did you buy sugar?' --`No (I didn't).'
\z

\ea%x1116
\label{ex:6:x1116}
\gll Yomar  owora  efar  aaw-o-k. -\textstyleEmphasizedVernacularWords{Weetak/wia},  me  os=ke  aaw-o-k.\\
1s/p.cousin  betelnut  1s.\textsc{dat}  take-\textsc{pa}-3s -no  not  3s.\textsc{fc}=\textsc{cf}  take-\textsc{pa}-3s\\
\glt `My cousin took my betelnut. --No, it wasn't he who took it.'
\z

For the use of \textstyleStyleVernacularWordsItalic{weetak/wia} as an answer to a negative question, see \sectref{sec:7.2.7}.

\subsection{Other cases of negation} \label{sec:6.2.4}
%\hypertarget{RefHeading22461935131865}

When an affirmative clause is followed by a negative one, and the two only differ by the contrasted element, the whole clause apart from the contrasted element is replaced by \textstyleStyleVernacularWordsItalic{weetak} or \textstyleStyleVernacularWordsItalic{wia}. A full clause is possible instead of \textstyleStyleVernacularWordsItalic{weetak}/\textstyleStyleVernacularWordsItalic{wia}, but it is not as common.

\ea%x1119
\label{ex:6:x1119}
\gll Mua  bug  maala  nain=ke  mera  unowa  isak-i-non, mua  bug  iiwa  nain  \textstyleEmphasizedVernacularWords{weetak/wia}.\\
man  wind  long  that1=\textsc{cf}  fish  many  spear-Np-\textsc{fu}.3s man  wind  short  that1  no.\\
\glt `A man with long breath (=big lungs) will spear many fish, a man with short breath will not.'
\z

\ea%x1120
\label{ex:6:x1120}
\gll Mera  papako  unowa,  papako  \textstyleEmphasizedVernacularWords{weetak/wia}. \\
fish  some  many  some  no\\
\glt `Some fish there are many, some not.'
\z

Also when an affirmative question is followed by a negative alternative, \textstyleStyleVernacularWordsItalic{weetak} or \textstyleStyleVernacularWordsItalic{wia} is used. 

\ea%x1121
\label{ex:6:x1121}
\gll Sira  nain  piipua-i-nan=i  e  \textstyleEmphasizedVernacularWords{weetak}? \\
habit  that  leave-Np-\textsc{fu}.2s=\textsc{qm}  or  no\\
\glt `Will you stop that habit or not?'
\z

\ea%x1122
\label{ex:6:x1122}
\gll Yo  emeria=ko  efar  uruf-a-man=i  e  \textstyleEmphasizedVernacularWords{weetak}? \\
1s.\textsc{unm}  woman=\textsc{nf}  1s.\textsc{dat}  see-\textsc{pa}-2p=\textsc{qm}  or  no\\
\glt `Have you seen my wife or not?'
\z

If an action fails to have the expected result, again one of the two negative interjections is used either by itself or followed by a full clause. 

\ea%x1124
\label{ex:6:x1124}
\gll Marasin  wu-om-a-mik=na  \textstyleEmphasizedVernacularWords{weetak}. \\
medicine  put-\textsc{ben}-\textsc{bnfy}2.\textsc{pa}-1/3p=\textsc{tp}  no\\
\glt `They put medicine in him but no (=with no result).'
\z

\ea%x1126
\label{ex:6:x1126}
\gll Naap  ik-ok  uruf-am-ika-iwkin  \textstyleEmphasizedVernacularWords{wia}. \\
thus  be-\textsc{ss}  see-\textsc{ss}.\textsc{sim}-be-2/3p.\textsc{ds}  no\\
\glt `They were thus watching him (but) no (he did not revive).'
\z

\ea%x1123
\label{ex:6:x1123}
\gll I  unan  maa  en-e-mik  en-e-mik  \textstyleEmphasizedVernacularWords{wia}, ipoka  taan-ep  enakiwa  wu-a-mik. \\
1p.\textsc{unm}  yesterday  food  eat-\textsc{pa}-1/3p  eat-\textsc{pa}-1/3p  no stomach  become.full-\textsc{ss}.\textsc{seq}  half  put-\textsc{pa}-1/3p\\
\glt `Yesterday we ate and ate, (but) no (=we could not finish the food), our stomachs were full and we put half of it aside.'
\z

When the clause expressing frustration of an effort starts a new sentence and begins with the additive connective \textstyleStyleVernacularWordsItalic{ne} `and/but', the negator is always \textstyleStyleVernacularWordsItalic{wia}, and an explanatory clause follows.

\ea%x1125
\label{ex:6:x1125}
\gll \textstyleEmphasizedVernacularWords{Ne}  \textstyleEmphasizedVernacularWords{wia},  papako=ke  ma-e-mik,  ``Weetak,  moram owowa  p-ikiw-i-yan?'' \\
\textsc{add}  no  other=\textsc{cf}  say-\textsc{pa}-1/3p  no  why village  Bpx-go-Np-\textsc{fu}.1p\\
\glt `But no, others said, ``Why take him to the village?'' '
\z

Mauwake has two different kinds of double negation. In both cases the negation is cancelled and the result is affirmative, but not an emphatic affirmative. A negated verb or an inherently negative verb may occur with the clausal negator \textstyleStyleVernacularWordsItalic{me} `not':

\ea%x1131
\label{ex:6:x1131}
\gll Ona  muuka  \textstyleEmphasizedVernacularWords{me}  sesek-owa=ke  \textstyleEmphasizedVernacularWords{me}  ma-e-k. \\
3s.\textsc{gen}  son  not  send-\textsc{nmz}=\textsc{cf}  not  say-\textsc{pa}-3s\\
\glt `He did not say that he wouldn't send (lit: say about not sending) his son.'
\z

\ea%x1132
\label{ex:6:x1132}
\gll Maamuma  \textstyleEmphasizedVernacularWords{me}  \textstyleEmphasizedVernacularWords{marew-ar-e-mik}. \\
money  not  no(ne)-\textsc{inch}-\textsc{pa}-1/3p\\
\glt `We/They did not lack money.'
\z

In the second type of double negation a speaker's negative statement is challenged by another speaker. In this case a different negator is used to challenge the original negation: 

\ea%x1133
\label{ex:6:x1133}
\gll Yo  episowa  weetak.  -\textstyleEmphasizedVernacularWords{Weetak}  \textstyleEmphasizedVernacularWords{wia}. \\
1s.\textsc{unm}  tobacco  no.  -no  no\\
\glt `I have no tobacco.'  `Don't say you don't have any.'
\z

The negation can be emphasized with the intensity adverb \textstyleStyleVernacularWordsItalic{akena} `very':

\ea%x1134
\label{ex:6:x1134}
\gll \textstyleEmphasizedVernacularWords{Weetak}  \textstyleEmphasizedVernacularWords{akena},  i  me  kuum-e-mik. \\
no  very  1p.\textsc{unm}  not  burn-\textsc{pa}-1/3s\\
\glt `\textstyleEmphasizedWords{\textsc{No}}, we did not burn it.'
\z

\ea%x1135
\label{ex:6:x1135}
\gll I  \textstyleEmphasizedVernacularWords{me}  kuum-e-mik  \textstyleEmphasizedVernacularWords{akena}. \\
1p.\textsc{unm}  not  burn-\textsc{pa}-1/3p  very\\
\glt `We did \textstyleEmphasizedWords{\textsc{not}} burn it.'
\z

Another possible strategy for emphasizing a negative statement or command is to attach the neutral focus clitic -\textstyleStyleVernacularWordsItalic{ko} to the verbal negator \textstyleStyleVernacularWordsItalic{me} `not'. In \REF{ex:6:x1136} the neutral focus clitic appears twice, as the speaker wants both to emphasise the negation and to distance himself from the situation (without implying that someone else did see what he did not). 

\ea%x1152
\label{ex:6:x1152}
\gll I  \textstyleEmphasizedVernacularWords{me=ko}  miim-a-mik. \\
1p.\textsc{unm}  not=\textsc{nf}  hear-\textsc{pa}-1/3p\\
\glt `We did \textstyleEmphasizedWords{\textsc{not}} hear it.'
\z

\ea%x1136
\label{ex:6:x1136}
\gll Yo=ko  \textstyleEmphasizedVernacularWords{me=ko}  uruf-a-m. \\
1s.\textsc{unm}=\textsc{nf}  not=\textsc{nf}  see-\textsc{pa}-1s\\
\glt `\textstyleEmphasizedWords{I} did \textstyleEmphasizedWords{\textsc{not}} see it.'
\z

\ea%x1137
\label{ex:6:x1137}
\gll \textstyleEmphasizedVernacularWords{Me}\textstyleEmphasizedVernacularWords{=ko}  niir-e  sa,  kae  napum-ar-e-k. \\
not=\textsc{nf}  laugh-\textsc{imp}.2s  \textsc{intj}  grandfather  sick-\textsc{inch}-\textsc{pa}-3s\\
\glt `Do \textstyleEmphasizedWords{\textsc{not}} laugh, grandfather is sick.'
\z

Negative spreading is fairly common in languages that have a medial verb system. The negation can spread forwards or backwards, or both, depending on the language. In Mauwake both forward \REF{ex:6:x1138} and backward \REF{ex:6:x1140} spreading is possible across medial clause boundaries, but only with the same-subject medial verbs.\footnote{In Usan, the negation of a final clause can spread backwards even with a different subject medial verb  \citep[282]{Reesink1987}, but Hua, like Mauwake, requires a same subject medial verb \citep[408]{Haiman1980}.} The spreading is not common, but it is more acceptable if the verbs form a logical sequence, an ``expectancy chain''.

\ea%x1138
\label{ex:6:x1138}
\gll Nain  yo  \textstyleEmphasizedVernacularWords{me}  \textstyleEmphasizedVernacularWords{ep-ap}  \textstyleEmphasizedVernacularWords{nefa}  \textstyleEmphasizedVernacularWords{aaw-e-m}. \\
but  1s.\textsc{unm}  not  come-\textsc{ss}.\textsc{seq}  2s.\textsc{acc}  get-\textsc{pa}-1s\\
\glt `But I did not come and get you.'
\z

\ea%x1140
\label{ex:6:x1140}
\gll Nainiw  \textstyleEmphasizedVernacularWords{ekap-ep}  \textstyleEmphasizedVernacularWords{maa}  \textstyleEmphasizedVernacularWords{me}  \textstyleEmphasizedVernacularWords{sesek-a-mik}. \\
again  come-\textsc{ss}.\textsc{seq}  food  not  sell-\textsc{pa}-1/3p\\
\glt `They did not come back and sell food again.'
\z

But negative spreading is not automatic; even with a same-subject medial verb two clauses \textstyleParagraphChari{can} have different polarity \REF{ex:6:x1156}, \REF{ex:6:x1163}. If the speaker wants to avoid ambiguity, finite clauses can be used when the polarity is different \REF{ex:6:x1153}.

\ea%x1156
\label{ex:6:x1156}
\gll Nepa  opaimika  \textstyleEmphasizedVernacularWords{me}  \textstyleEmphasizedVernacularWords{baliwep}  \textstyleEmphasizedVernacularWords{amis-ar-ep}  wiena opaimik=iw  yia  maak-em-ik-e-mik.\\
bird  talk  not  well  knowledge-\textsc{inch}-\textsc{ss}.\textsc{seq}  3p.\textsc{gen} talk=\textsc{inst}  1p.\textsc{acc}  tell-\textsc{ss}.\textsc{sim}-be-\textsc{pa}-1/3p\\
\glt `They did not know Tok Pisin well and talked to us in their own language.'
\z

\ea%x1763
\label{ex:6:x1763}
\gll Mua  lebuma  \textstyleEmphasizedVernacularWords{me}  \textstyleEmphasizedVernacularWords{arim-ep}  takira  ik-ok  emeria wia  aaw-i-mik.\\
man  lazy  not  grow-\textsc{ss}.\textsc{seq}  young  be-\textsc{ss}.\textsc{sim}  woman 3s.\textsc{acc}  take-Np-\textsc{pr}.1/3p\\
\glt `Lazy men, not having grown and (still) being young, take wives.'
\z

\ea%x1153
\label{ex:6:x1153}
\gll Nainiw  ekap-e-mik,  nain  maa  \textstyleEmphasizedVernacularWords{me}  sesek-a-mik. \\
again  come-\textsc{pa}-1/3p  that1  food  not  sell-\textsc{pa}-1/3p\\
\glt `They came back again, but did not sell any food.'
\z

If the context is not clear enough, the negator can be repeated for each negated verb in a medial verb construction. In \REF{ex:6:x1139}, if only the first verb is negated, the sentence could mean that many people do not know the person but follow him nevertheless; whereas if only the second verb is negated, the sentence might be taken to mean that many people do know the person but do not follow him. 

\ea%x1139
\label{ex:6:x1139}
\gll Mua  unowa  o  \textstyleEmphasizedVernacularWords{me}  amis-ar-ep \textstyleEmphasizedVernacularWords{me}  ook-i-mik.\\
man  many  3s.\textsc{unm}  not  knowledge-\textsc{inch}-\textsc{ss}.\textsc{seq} not  follow-Np-\textsc{pr}.3p\\
\glt `Many people do not know him and do not follow him.'
\z

Different-subject marking blocks negative spreading in both directions. Thus in \REF{ex:6:x1141} the polarity changes with each new clause:

\ea%x1141
\label{ex:6:x1141}
\gll Soomar-em-ika-iwkin \textstyleEmphasizedVernacularWords{me}  \textstyleEmphasizedVernacularWords{wia}  \textstyleEmphasizedVernacularWords{far-eya} nefa  ma-i-kuan, {\dots}\\
walk-\textsc{ss}.\textsc{sim}-be-2/3p.\textsc{ds}  not  3p.\textsc{acc}  call-2/3s.\textsc{ds} 2s.\textsc{acc}  say-Np-\textsc{fu}.3p\\
\glt `When they walk past, and you do not call them, they will say about you that {\dots}'
\z

Negative transportation from a complement clause to a main clause does not take place in Mauwake.\footnote{This is true of Amele as well \citep[44]{Roberts1987}, but Usan allows it \citep[278--280]{Reesink1987}.} 

\section{Deixis} 
%\hypertarget{RefHeading22481935131865}

Different parts in the grammar interact to produce the deictic system, the spatio-temporal and personal orientation related to the speech situation or another situation specified in the text. The default deictic centre is the speaker, the speaker's location and the present time.  

\subsection{Person deixis} \label{sec:6.3.1}
%\hypertarget{RefHeading22501935131865}

Only the first and second person are inherently deictic, as they get their whole meaning, apart from the number, from the speech situation. The person marking is done by pronouns (\sectref{sec:3.5}) and by person/number suffixes on the verbs (\sectref{sec:3.8.3.4}, \sectref{sec:3.8.3.5}). The special status of the first person as against both the second and third persons shows in the imperative and the switch-reference marking. In the imperative the dual number is only possible in the first person (\sectref{sec:3.8.3.2.2}\todo{please check}): 

\ea%x1262
\label{ex:6:x1262}
\gll Aria,  i  owowa=ko  urup-\textstyleEmphasizedVernacularWords{u}.  Auwa  aite wia  karu-i-yan,  owowa=pa  wia  uruf-\textstyleEmphasizedVernacularWords{u}.\\
alright  1p.\textsc{unm}  village=\textsc{nf}  ascend-\textsc{imp}.1d  1s/p.father  1s/p.mother 3p.\textsc{acc}  visit-Np-\textsc{fu}.1p  village=\textsc{loc}  3p.\textsc{acc}  see-\textsc{imp}.1d\\
\glt `Alright, let's (dl) go up to the village. We'll visit father and mother, let's (dl) see them in the village.'
\z

In the different-subject medial verbs the first person singular and plural share the same suffix, whereas the second and third persons are grouped together and the distinction is made according to number, between singular and plural (\sectref{sec:3.8.3.4.2}\todo{please check}).

\ea%x1263
\label{ex:6:x1263}
\gll I  ikoka  urup-ep  nia  \textstyleEmphasizedVernacularWords{maak-omkun} \textstyleEmphasizedVernacularWords{ora-iwkin,}  aria  owawiya  feeke  pok-ap  ik-ok  {\dots}\\
1p.\textsc{unm}  later  ascend-\textsc{ss}.\textsc{seq}  2p.\textsc{acc}  tell-1s/p.\textsc{ds} descend-2/3p.\textsc{ds}  alright  together  here.\textsc{cf}  sit-\textsc{ss}.\textsc{seq}  be-\textsc{ss}\\
\glt `Later when we come up and tell you and (then) you come down and we sit down together here and {\dots}'
\z

Even though the first and second person pronouns are already deictic in themselves, their unmarked plural forms can both co-occur with the proximate demonstrative \textstyleStyleVernacularWordsItalic{fain} 'this', 
\todo[inline]{There is a black space in the word ``demonstrative'' above when it's split across lines} 
and the second person also with the distal demonstrative \textstyleStyleVernacularWordsItalic{nain} `that'. As only one of the people referred to by these plural forms typically is a speech act participant and the others may or may not be present, the addition of the demonstrative  makes it clear that all the people referred to are present in the situation: 

\ea%x1269
\label{ex:6:x1269}
\gll Ikoka  Yaapan=ke  ekap-emi  \textstyleEmphasizedVernacularWords{ni}  emeria  unowa  \textstyleEmphasizedVernacularWords{fain} nia  aaw-urum-i-kuan.\\
later  Japan=\textsc{cf}  come-\textsc{ss}.\textsc{sim}  2p.\textsc{unm}  woman  many  this 2p.\textsc{acc}  take-\textsc{distr}/\textsc{a}-Np-\textsc{fu}.3p\\
\glt `Later the Japanese will come and take all of you many women [here in this village].'
\z

Mauwake has no separate system of social deixis, as there are no honorifics, nor are there special pronouns used for particular kin or social groups or the like.

Emotional deixis, associating the speaker with the topic of conversation or distancing him from it, is a possible use for demonstratives in Papuan languages and worldwide (\citealt[72--78]{FarrEtAl1982}, \citealt[347--355]{Lakoff1974}). In Mauwake that possibility is not utilized: the demonstratives are neutral in this respect.

\subsection{Locative deixis} \label{sec:6.3.2}
%\hypertarget{RefHeading22521935131865}

Locative deixis, which relates the location to the speech act participants, utilizes several different word classes. The proximate demonstrative \textstyleStyleVernacularWordsItalic{fain} `this' (\sectref{sec:3.6.2}) and the corresponding locative adverb \textstyleStyleVernacularWordsItalic{fan} `here'(\sectref{sec:3.6.3})  are truly deictic, as their meaning is based on the location of the speaker. The distal-1 demonstrative \textstyleStyleVernacularWordsItalic{nain} `that' and the adverb \textstyleStyleVernacularWordsItalic{nan} `there' are more neutral, and the less common distal-2 and -3 deictics have other defining features besides the distance to the speaker. 

\ea%x1273
\label{ex:6:x1273}
\gll Ep-ap  owora  \textstyleEmphasizedVernacularWords{fain}  aaw-ep  enim-eka,  iwer(a)  eka \textstyleEmphasizedVernacularWords{fain}  enim-eka. \\
come-\textsc{ss}.\textsc{seq}  betelnut  this  take-\textsc{ss}.\textsc{seq}  eat-\textsc{imp}.2p  coconut  water this  eat-\textsc{imp}.2p\\
\glt `Come and take this betelnut and eat it, (and) drink this coconut water.'
\z

\ea%x1274
\label{ex:6:x1274}
\gll Yo  wia  wiim-urup-ep  \textstyleEmphasizedVernacularWords{fan}  wia  wu-ap kiiriw  iw-a-m.\\
1s.\textsc{unm}  3p.\textsc{acc}  escort-ascend-\textsc{ss}.\textsc{seq}  here  3p.\textsc{acc}  put-\textsc{ss}.\textsc{seq} again  go-\textsc{pa}-1s\\
\glt `I escorted them up here and went (back) again.'
\z

In the location verbs \textstyleStyleVernacularWordsItalic{fan}- `arrive/be here' and \textstyleStyleVernacularWordsItalic{nan}- `arrive/be there' (\sectref{sec:3.8.4.4.3}) the deictic goal forms the verb root. 

\ea%x1275
\label{ex:6:x1275}
\gll Auwa  afura  \textstyleEmphasizedVernacularWords{fan-e-k}  a,  no=ko  wiar akim-ap=ko  uruf-e.\\
1s/p.father  lime  here-\textsc{pa}-3s  \textsc{intj}  2s.\textsc{unm}=\textsc{nf}  3.\textsc{dat} try-\textsc{ss}.\textsc{seq}=\textsc{nf}  see-\textsc{imp}.2s\\
\glt `Ah, father's lime is here, you try it and see.'
\z

In the directional verbs (\sectref{sec:3.8.4.4.5}) as well as the related bring-verbs (\sectref{sec:3.8.2.4.2}) the verb root gives indication as to the direction of the movement. Only those directional verbs where the direction is clearly related to the speaker are deictic. The second person is not a possible alternative deictic centre for the verb \textstyleStyleVernacularWordsItalic{ekap}- `come'.  

\ea%x1278
\label{ex:6:x1278}
\gll Uurika  nefar  \textstyleEmphasizedVernacularWords{ikiw-i-nen}. \\
tomorrow  2s.\textsc{dat}  go-Np-\textsc{fu}.1s \\
\glt `Tomorrow I'll come to you.' (Lit: `{\dots}I'll go (from my present place) ...')
\z

\ea%x1279
\label{ex:6:x1279}
\gll Mua  imen-ap=na  feeke  wia  \textstyleEmphasizedVernacularWords{p-ekap-eka}. \\
man  find-\textsc{ss}.\textsc{seq}=\textsc{tp}  here.\textsc{cf}  3p.\textsc{acc}  \textsc{bp}x-come-\textsc{imp}.2p\\
\glt `If you find the men, bring them here.'
\z
\todo[inline]{a previous occurrence of \textsc{bp}x is written as Bpx. Please confirm which one to use.}

Although the prototypical deictic centre is close proximity to the speaker, it may be extended to quite a large area. In \REF{ex:6:x1892} where the coming of the Japanese troops is described, it covers the whole North Coast of the New Guinea island: 

\ea%x1892
\label{ex:6:x1892}
\gll Ne  \textstyleEmphasizedVernacularWords{ekap-ep}  Numbia=pa  nan  urup-e-mik. \\
\textsc{add}  come-\textsc{ss}.\textsc{seq}  Numbia=\textsc{loc}  there  ascend-\textsc{pa}-1/3p\\
\glt `And they came and landed at Numbia.'
\z

In narratives it is more typical that the verbs \textstyleStyleVernacularWordsItalic{ekap}- `come' and \textstyleStyleVernacularWordsItalic{ikiw}- `go', as well as the related verbs for `bring' and `take', get their deictic centre from the main character, not the narrator, since the narrator often is not even a participant in the story. 

\ea%x1277
\label{ex:6:x1277}
\gll Sawur  emeria  nain  ikiw-eya  o  iikir-ami  owowa ekap-o-k. \\
spirit  woman  that  go-2/3s.\textsc{ds}  3s.\textsc{unm}  get.up-\textsc{ss}.\textsc{sim}  village come-\textsc{pa}-3s\\
\glt `When the spirit woman went (away), he came to the/his village.'
\z

\subsection{Temporal deixis}
%\hypertarget{RefHeading22541935131865}

Temporal deixis relates time to the speech act, or alternatively to the time of a specific event. Tense marking (\sectref{sec:3.8.3.4}) is the most important device for this in Mauwake, as tense is an obligatory category in verbs.\footnote{In some Papuan languages tense markers and demonstratives are morphologically related (Cindi Farr, p.c.), but this is not the case in Mauwake.} The present tense marks the default deictic centre, the past tense refers to the time before that point, and the future tense to the time after it. The example \REF{ex:3:x1029} 
\todo[inline]{We need to find out the previous use of example ex:x1893. - it is used as ex:3:x1029 at section 3.8.3.4 -- Felix}
is repeated here as \REF{ex:6:x1893}:

\ea%x1893
\label{ex:6:x1893}
\gll Unan  \textstyleEmphasizedVernacularWords{aakun-e-mik},  aakisa  \textstyleEmphasizedVernacularWords{aakun-i-mik}  ne uurika  nainiw  \textstyleEmphasizedVernacularWords{aakun-i-yen}.\\
yesterday  talk-\textsc{pa}-1/3p  now/today  talk-Np-\textsc{pr}.1/3p  \textsc{add} tomorrow  again  talk-Np-\textsc{fu}.1p\\
\glt `Yesterday we talked, now/today we talk and tomorrow we'll talk again.'
\z

Papuan languages in general favour presenting a narrative in strictly chronological order, so a relative tense, where the deictic centre is shifted either to the past or to the future, is not utilized widely. This is true of Mauwake as well. When a shift to the past is needed, it can be done by right-dislocating a medial clause after a past-tense marked final clause: 

\ea%x1268
\label{ex:6:x1268}
\gll Wilkar  wia  muf-em-ik-om-a-mik,  \textstyleEmphasizedVernacularWords{mua} \textstyleEmphasizedVernacularWords{kui-kuisow}  \textstyleEmphasizedVernacularWords{wia}  \textstyleEmphasizedVernacularWords{maak-iwkin}.\\
cart  3p.\textsc{acc}  pull-\textsc{ss}.\textsc{sim}-be-\textsc{ben}-\textsc{bnfy}2.\textsc{pa}-1/3p  man \textsc{rdp}-one  3p.\textsc{acc}  tell-2/3p.\textsc{ds} \\
\glt `They\textsubscript{1} pulled carts for them\textsubscript{2}, after they\textsubscript{2} had told the men\textsubscript{1} one by one.'
\z

The same-subject sequential forms of the directional verbs \textstyleStyleVernacularWordsItalic{ekap}- `come' and \textstyleStyleVernacularWordsItalic{ikiw}- `go' also have temporal deictic use, the former referring to time extending to the present moment, the latter mainly to time from the present moment onwards. The examples \REF{ex:3:x290} and \REF{ex:3:x437} 
\todo[inline]{We need to find out the previous use of example ex:x11941 and ex:x1942. Done -- Felix}
are repeated below as \REF{ex:6:x1941} and \REF{ex:6:x1942}.

\ea%x1941
\label{ex:6:x1941}
\gll Naap  on-am-ik-e-mik,  \textbf{ekap-ep} aakisa. \\
thus  do-\textsc{ss}.\textsc{sim}-be-\textsc{pa}-1/3p  come-\textsc{ss}.\textsc{seq}  now\\
\glt `We have been doing like that (all the time) up until now.'
\z

\ea%x1942
\label{ex:6:x1942}
\gll No  naap  ik-ok  \textbf{iki(w-e)p}  mokoma  enuma  iiwawun aakun-i-nan.\\
2s.\textsc{unm}  thus  be-\textsc{ss}  go-\textsc{ss}.\textsc{seq}  year  new  altogether talk-Np-\textsc{fu}.2s\\
\glt `You will be like that (long time) but next year you will talk.'
\z

The two groups of deictic temporal adverbs (\sectref{sec:3.9.1.2}) behave differently as to what the deictic centre is. The specific temporal adverbs, which refer to a certain day in relation to the utterance, always take the time of the speech act as their deictic centre. 

\ea%x1889
\label{ex:6:x1889}
\gll {\dots}i  \textbf{uurika}  ora-i-yan,  ifera  un-owa ora-i-yan.\\
{\dots}1p.\textsc{unm}  tomorrow  descend-Np-\textsc{fu}.1p  sea(water)  fetch-\textsc{nmz} descend-Np-\textsc{fu}.1p\\
\glt `{\dots}we will go down tomorrow, we will go down to fetch sea water.'
\z

The non-specific temporals normally do this too: 

\ea%x1890
\label{ex:6:x1890}
\gll Nain  \textstyleEmphasizedVernacularWords{iiriw}  me  kerer-e-k,  \textstyleEmphasizedVernacularWords{aakisa} \textstyleEmphasizedVernacularWords{fan}  {\O}. \\
that1  earlier  not  appear  now  here \\
\glt `That didn't appear ealier/long ago but just now (lit: now here).'
\z

But their time reference may also be relative, with the time of the event taken as the deictic centre. This is especially true of \textstyleStyleVernacularWordsItalic{aakisa} `now', which is used for perspectivization.\footnote{The ``WAS-NOW paradox'' occurs in ``free indirect style'' when ``[t]he deictic centre of the utterance is the writer/narrator, but certain deictic elements are relativized to give the impression of direct access to the character's mental states: these include temporal and spatial expressions such as \textit{now, here, today} {\dots} but not tense or person.'' \citep{MushinEtAl2000}.} The temporal adverbs in the following two examples, \textstyleStyleVernacularWordsItalic{aakisa} `now' and \textstyleStyleVernacularWordsItalic{aakisa fan} `just now', do not refer externally to the time close to the speech event; instead, they are text-internal perspectivization devices to highlight the importance of the event to the main characters in the text. \REF{ex:6:x475} is from an old traditional story and \REF{ex:6:x1891} tells about events that took place over four decades before the recording. 

\ea%x475
\label{ex:6:x475}
\gll Nain  or-op  ``buu''  (na-e-k),  \textstyleEmphasizedVernacularWords{aakisa}  eka  saanar-\textstyleEmphasizedVernacularWords{e}-k. \\
that1  fall-\textsc{ss}.\textsc{seq}  buu  say-\textsc{pa}-3s  now  water  dry-\textsc{pa}-3s\\
\glt `It fell with a thud (and they knew that) now the water had dried up.'
\z

\ea%x1891
\label{ex:6:x1891}
\gll Ekap-ep,  ekap-ep,  \textstyleEmphasizedVernacularWords{aakisa} \textstyleEmphasizedVernacularWords{fan}  unowa  Wewak=pa nan  urup-\textstyleEmphasizedVernacularWords{e}-mik.\\
come-\textsc{ss}.\textsc{seq}  come-\textsc{ss}.\textsc{seq}  now  here  many  Wewak=\textsc{loc} there  ascend-\textsc{pa}-1/3p \\
\glt `They came and came, and just now many came up there in Wewak.'
\z

For the deictic shift that takes place in indirect speech, see \sectref{sec:8.3.2.1.2}.

\section{Quantification}
%\hypertarget{RefHeading22561935131865}

Nouns are not inflected for number in Mauwake, and in the whole noun phrase the number may be left unspecified \REF{ex:6:x1284}. The verbs are marked for either singular or plural, but the plural form can be used also for unspecified number \REF{ex:6:x1285}. The pronouns must be either singular or plural. Besides these two obligatory number marking devices the language has several other means for quantification.

\ea%x1284
\label{ex:6:x1284}
\gll Waaya  kiikir=iw  uruf-i-mik,  owowa=pa. \\
pig  first=\textsc{inst}  see-Np-\textsc{pr}.1/3p  village=\textsc{loc} \\
\glt `First they look at the pig(s) in the village.'
\z

\ea%x1285
\label{ex:6:x1285}
\gll Nain  pun  sira  naap=iw,  mua=ko  me  kerer-e-mik. \\
that1  too  custom  thus=\textsc{inst}  man=\textsc{nf}  not  appear-\textsc{pa}-1/3p\\
\glt `That was like that too, the (guilty) person/people did not appear.'
\z

\subsection{Quantification in the noun phrase}
%\hypertarget{RefHeading22581935131865}

Numerals (\sectref{sec:3.4.1}) are used when the exact number is relevant, non-numeral quantifiers (\sectref{sec:3.4.2}) are used elsewhere.  

\ea%x1286
\label{ex:6:x1286}
\gll \textstyleEmphasizedVernacularWords{Masin} \textstyleEmphasizedVernacularWords{erup}  nainiw  wu-owa  epa  ik-ua. \\
engine  two  again  put-\textsc{nmz}  place  be-\textsc{pa}.3s\\
\glt `There is a place for putting two more engines.'
\z

\ea%x1308
\label{ex:6:x1308}
\gll \textstyleEmphasizedVernacularWords{Waa}  \textstyleEmphasizedVernacularWords{muuka} \textstyleEmphasizedVernacularWords{arow}  ekap-o-k. \\
pig  son  three  come-\textsc{pa}-1s\\
\glt `Three piglets came.'
\z

\ea%x1287
\label{ex:6:x1287}
\gll \textstyleEmphasizedVernacularWords{Emeria}  \textstyleEmphasizedVernacularWords{unow=iya}  ikiw-ep  eka  nain  imar-e-mik. \\
woman  many=\textsc{com}  go-\textsc{ss}.\textsc{seq}  river  that1  catch.fish-\textsc{pa}-1/3p\\
\glt `All the women went and fished at the river.'
\z

The third person plural unmarked pronoun functions as a pluraliser both in an ordinary \textstyleAcronymallcaps{NP} and with place names when the population of the place is referred to (\sectref{sec:4.1.1}). 

\ea%x1288
\label{ex:6:x1288}
\gll Nain  \textstyleEmphasizedVernacularWords{wi} \textstyleEmphasizedVernacularWords{mua}  sira=ke,  emeria  soop-owa  sira. \\
that1  3p.\textsc{unm}  man  custom=\textsc{cf}  woman  bury-\textsc{nmz}  custom\\
\glt `That is the men's custom, the custom of burying wife/wives.'
\z

\ea%x1289
\label{ex:6:x1289}
\gll Irak-owa  weeser-eya  aria  \textstyleEmphasizedVernacularWords{wi}  \textstyleEmphasizedVernacularWords{Simbine} baurar-e-mik.\\
fight-\textsc{nmz}  finish-2/3s.\textsc{ds}  alright  3p.\textsc{unm}  Simbine flee-\textsc{pa}-1/3p\\
\glt `When the fighting was finished, alright the Simbine people fled.'
\z

Even without the pluralizing pronoun the word for, or a name of, a village may occasionally, as a subject of a clause, refer to the population and thus be interpreted as plural. In the following example this shows in the plural marking in the verb. 

\ea%x1307
\label{ex:6:x1307}
\gll Ne  owowa  oko  nain=ke  maak-e-mik,  {\dots} \\
\textsc{add}  village  other  that1=\textsc{cf}  tell-\textsc{pa}-1/3p\\
\glt `And (the people of) that other village told him, {\dots}''
\z

Reduplication is another pluralizing device used in the \textstyleAcronymallcaps{NP}. Only a small group of nouns can undergo reduplication (\sectref{sec:3.2.6.2}), but in adjectives it is somewhat more common (\sectref{sec:3.3}).

\ea%x1290
\label{ex:6:x1290}
\gll Waaya  pa-ep  \textstyleEmphasizedVernacularWords{kio-kiowa}  naap  uup-e-mik. \\
pig  butcher-\textsc{ss}.\textsc{seq}  \textsc{rdp}-piece  thus  cook-\textsc{pa}-1/3p\\
\glt `We butchered the pig and cooked the pieces like that.'
\z

\ea%x1291
\label{ex:6:x1291}
\gll Owow(a)  saria=ke  kiikir  perek-i-mik, \textstyleEmphasizedVernacularWords{mua}  \textstyleEmphasizedVernacularWords{or-oram}  fain  weetak.\\
village  headman=\textsc{cf}  first  pull.out-Np-\textsc{pr}.1/3p man  \textsc{rdp}-insignificant  this  no\\
\glt `The village headmen harvest it first, not common people like this/us.'
\z

Comitative noun phrases (\sectref{sec:4.1.3}) are used to indicate duality or plurality. 

\ea%x1292
\label{ex:6:x1292}
\gll \textstyleEmphasizedVernacularWords{(Yo/I)}  \textstyleEmphasizedVernacularWords{auwa}  \textstyleEmphasizedVernacularWords{ikos}  fan  ik-e-mik. \\
1s/1p.\textsc{unm}  1s/p.father  together.with  here  be-\textsc{pa}-1/3p\\
\glt `I and my father are here.'
\z

\ea%x1293
\label{ex:6:x1293}
\gll No  ikoka  \textstyleEmphasizedVernacularWords{mua}  \textstyleEmphasizedVernacularWords{owawiya}  irak-ep  me  efar kerer-e.\\
2s.\textsc{unm}  later  man  with  fight-\textsc{ss}.\textsc{seq}  not  1s.\textsc{dat} appear-\textsc{imp}.2s\\
\glt `Later when you and your husband fight, don't come to me.'
\z

\ea%x1294
\label{ex:6:x1294}
\gll Ne  \textstyleEmphasizedVernacularWords{bom=iya}  \textstyleEmphasizedVernacularWords{kateres=iya}  \textstyleEmphasizedVernacularWords{bom=iya}  \textstyleEmphasizedVernacularWords{kateres=iya} {\O}\textstyleEmphasizedVernacularWords{.} \\
and  bomb=\textsc{com}  cartridge=\textsc{com}  bomb=\textsc{com}  cartridge=\textsc{com}\\
\glt `And bombs and cartridges, bombs and cartridges (kept dropping).'
\z

\ea%x1295
\label{ex:6:x1295}
\gll \textstyleEmphasizedVernacularWords{Pauli}  \textstyleEmphasizedVernacularWords{ame}  era=pa  wia  uruf-ap  {\dots} \\
Pauli  \textsc{assoc}  road=\textsc{loc}  3p.\textsc{acc}  see-\textsc{ss}.\textsc{seq}\\
\glt `I saw Pauli and the others on the road, and {\dots}'
\z

Personal pronouns have to mark the number,\footnote{Except for third person dative pronoun, which is \textit{wiar} for both singular and plural.} but in cases where the number is unknown or unspecified, plural is used.

\ea%x1309
\label{ex:6:x1309}
\gll Ikiw-ep  mua  \textstyleEmphasizedVernacularWords{wia}  uruf-a-k  na  weetak,  mua=ko  me \textstyleEmphasizedVernacularWords{wia}  furew-a-k. \\
go-\textsc{ss}.\textsc{seq}  man  3p.\textsc{acc}  see-\textsc{pa}-3s  but  no  man=\textsc{nf}  not 3p.\textsc{acc}  sense-\textsc{pa}-3s\\
\glt `She went and looked for anyone/people but no, she did not sense (there was) anyone (there).'
\z

\subsection{Quantification devices in the verbs} \label{sec:6.4.2}
%\hypertarget{RefHeading22601935131865}

The person/number suffix in the finite verbs (\sectref{sec:3.8.3.4}) is the most frequently used device to indicate quantification: it shows whether the subject is singular or plural. Often the person/number suffix in the verb is the only element in a clause overtly showing number. 

\ea%x1296
\label{ex:6:x1296}
\gll Mauw-am-ik-ok  ik-ok  mauw-owa  weeser-eya urera  ekap-e-\textstyleEmphasizedVernacularWords{mik}.\\
work-\textsc{ss}.\textsc{sim}-be-\textsc{ss}  be-\textsc{ss}  work-\textsc{nmz}  finish-2/3s.\textsc{ds} afternoon  come-\textsc{pa}-1/3p\\
\glt `They came and landed there at Numbia.'
\z

But if the subject noun is [-human], even the person/number suffix may not indicate the number, since plural marking is only used for humans and occasionally for large animals, and only very rarely for inanimates. In the following example, the verbs in both sentences refer to airplanes, but because the action in the first sentence is attributed to the soldiers inside the planes, the finite verb is in plural form. 

\ea%x1283
\label{ex:6:x1283}
\gll Amerika  irak-ow(a)  iinan  aasa  ekap-ep  Ulingan  nan  bom \textstyleEmphasizedVernacularWords{fu-fuurk-ikiw-e-mik}.  Iinan=iw  iinan=iw  wu-ami  feenap Wewak  kame  naap  \textstyleEmphasizedVernacularWords{ikiw-o-k}.\\
America  fight-\textsc{nmz}  sky  canoe  come-\textsc{ss}.\textsc{seq}  Ulingan  there  bomb \textsc{rdp}-drop-go-\textsc{pa}-1/3p  sky=\textsc{inst}  sky=\textsc{inst}  put-\textsc{ss}.\textsc{sim}  like.this Wewak  side  thus  go-\textsc{pa}-3s\\
\glt `American fighter planes came and went on dropping bombs there in Ulingan. They were really high up and went like this to Wewak.'
\z

Reduplication is more common in verbs than in nouns or adjectives (\sectref{sec:3.8.2.4.1}). In transitive verbs the reduplication indicates plurality of the resulting object. 

\ea%x1298
\label{ex:6:x1298}
\gll Kau  nain  pa-ep,  gele-gelemuti-tik \textstyleEmphasizedVernacularWords{pu-puuk}\textstyleEmphasizedVernacularWords{-}\textstyleEmphasizedVernacularWords{ap}  uup-e-mik.\\
cow  that1  butcher-\textsc{ss}.\textsc{seq}  \textsc{rdp}-small-\textsc{rdp} \textsc{rdp}-cut-\textsc{ss}.\textsc{seq}  cook-\textsc{pa}-1/3\\
\glt `They butchered the cow and cut it into small pieces and cooked it/them.'
\z

\ea%x1297
\label{ex:6:x1297}
\gll Aruf-irapar-emi  meren(a)  suuw-owa  wiar \textstyleEmphasizedVernacularWords{pere-perek-a-mik.}\\
hit-to.and.fro-\textsc{ss}.\textsc{sim}  leg  pull-\textsc{nmz}  3.\textsc{dat} \textsc{rdp}-tear-\textsc{pa}-1/3p\\
\glt `They hit him all over and tore his trousers to pieces.'
\z

Both the distributive suffixes (\sectref{sec:3.8.2.3.2}) mark plurality; the argument that the marking pluralizes depends on the type of verb. 

\ea%x1300
\label{ex:6:x1300}
\gll Iinan  aasa  fan  \textstyleEmphasizedVernacularWords{or-om-ik-omak-i-ya}. \\
sky  canoe  here  descend-\textsc{ss}.\textsc{sim}-be-\textsc{distr}/\textsc{pl}-Np-\textsc{pr}.3s\\
\glt `Many planes are descending here.'
\z

\ea%x1299
\label{ex:6:x1299}
\gll Koora  pun  ariwa=ke  kuum-eya  \textstyleEmphasizedVernacularWords{aw-omak-e-k}. \\
house  also  arrow=\textsc{cf}  burn-2/3s.\textsc{ds}  burn-\textsc{distr}/\textsc{pl}-\textsc{pa}-3s\\
\glt `Also many houses burned down when the ammunition burned them.'
\z

\ea%x1301
\label{ex:6:x1301}
\gll Owowa  wia  \textstyleEmphasizedVernacularWords{wi-urum-e-p}  naap  ikiw-i-kuan. \\
village  3p.\textsc{acc}  give.them-\textsc{distr}/\textsc{a}-\textsc{ss}.\textsc{seq}  thus  go-Np-\textsc{fu}.3p\\
\glt `They give villages to all of them and then they go like that.' (Certain villages are designated for certain people to go to.)
\z

\ea%x1302
\label{ex:6:x1302}
\gll O  iiriw  maa  bala  wiar  \textstyleEmphasizedVernacularWords{aaw-urum-ep} ona  mia=pa-r=iw  wu-a-k.\\
3s.\textsc{unm}  earlier  thing  ornament  3.\textsc{acc}  get-\textsc{distr}/\textsc{a}-\textsc{ss}.\textsc{seq} 3s.\textsc{gen}  body=\textsc{loc}-{\O}=\textsc{lim}  put-\textsc{pa}-3s\\
\glt `Earlier he had received ornaments from all of them and (now) he put them on his own body only.'
\z

In the object cross-referencing verbs (\sectref{sec:3.8.4.2.4}) the root shows singularity or plurality of the object that is cross-referenced.

\ea%x1303
\label{ex:6:x1303}
\gll Iperowa  opora  \textstyleEmphasizedVernacularWords{wiok-i-yan}. \\
middle.aged  talk  follow.them-Np-\textsc{fu}.1p\\
\glt `We'll follow/obey the talk of the middle-aged men.'
\z

\ea%x1304
\label{ex:6:x1304}
\gll Maa  eneka  kes  mane-maneka  oram  \textstyleEmphasizedVernacularWords{iw-e-mik}. \\
thing  tooth  case  \textsc{rdp}-big  just  give.him-\textsc{pa}-1/3p\\
\glt `They gave him big meat (tin) cases for free.'
\z

When a numeral follows a nominalized verb form and precedes the resultative verb \textstyleStyleVernacularWordsItalic{ar}- `become' (\sectref{sec:3.8.4.4.4}), that indicates how many times an action was performed. 

\ea%x1305
\label{ex:6:x1305}
\gll Ewar  maneka  \textstyleEmphasizedVernacularWords{muf-owa}  \textstyleEmphasizedVernacularWords{erup}  \textstyleEmphasizedVernacularWords{ar-e}. \\
wind  big  pull-\textsc{nmz}  two  become-\textsc{imp}.2s\\
\glt `Breathe deeply twice.'
\z

\ea%x1306
\label{ex:6:x1306}
\gll Kiikir  iinan=pa  \textstyleEmphasizedVernacularWords{akim-owa}  \textstyleEmphasizedVernacularWords{arow} \textstyleEmphasizedVernacularWords{ar-e-mik}. \\
first  top=\textsc{loc}  try-\textsc{nmz}  three  become-\textsc{pa}-1/3p\\
\glt `First they tried it three times on top.'
\z

\section{Comparison}
%\hypertarget{RefHeading22621935131865}

\subsection{Comparison of inequality: comparative constructions}
%\hypertarget{RefHeading22641935131865}

As the inventory of adjectives is typically small in Papuan languages (\citealt[268]{Haiman1980}, \citealt[63]{Reesink1987}, \citealt[105--107]{MacDonald1990}), it is no surprise that regular morphological or syntactic forms to express comparative and superlative are rare, or even non-existent. In Mauwake comparison can be expressed in various ways, but there are no specific forms that could be called comparative or superlative. Since the overall frequency of comparative constructions is very low, it is not possible here to call any of them the preferred strategy.

One way to express comparison is to conjoin two structurally similar clauses, where the adjective in the first one functioning as the non-verbal predicate is unintensified, but in the second clause it has an intensifier. The first clause contains the standard of comparison. 

\ea%x1336
\label{ex:6:x1336}
\gll Poka  fain  \textstyleEmphasizedVernacularWords{maala},  ne  oko  \textstyleEmphasizedVernacularWords{maala}  \textstyleEmphasizedVernacularWords{akena}. \\
stilt  this  long  \textsc{add}  other  long  very\\
\glt `This stilt is long but the other one is longer (lit: very long).'
\z

Although the clauses usually are descriptive, as above, they do not have to be. In the following example the locative noun \textstyleStyleVernacularWordsItalic{iinan} `top', functioning like an adjective here, modifies the head noun in both the clauses.  

\ea%x1324
\label{ex:6:x1324}
\gll Ema  \textstyleEmphasizedVernacularWords{iinan} urup-e-m,  ne  no  ema \textstyleEmphasizedVernacularWords{iinan}  \textstyleEmphasizedVernacularWords{akena}  urup-o-n.\\
mountain  top  ascend-\textsc{pa}-1s  \textsc{add}  2s.\textsc{unm}  mountain top  very  ascend-\textsc{pa}-2s\\
\glt `I climbed a high mountain, but you climbed a higher (lit: very high) mountain.'
\z

Another way is to use adjectives that are antonymous. As a comparative structure this is problematic in that it is arbitrary to call the subject of one clause the standard and the subject of the other the object of comparison. 

\ea%x1325
\label{ex:6:x1325}
\gll Waaya  nain  \textstyleEmphasizedVernacularWords{gelemuta},  oko  nain  \textstyleEmphasizedVernacularWords{maneka}. \\
pig  that1  small  other  that1  big\\
\glt `That pig is smaller than the other one.' Or: `That other pig is bigger than that one.' (Lit: That pig is small, the other one is big.)'
\z

The same caveat applies to the following structure, where the adjective is negated for comparison:

\ea%x1326

\label{ex:6:x1326}
\gll Auwa  uuw-owa  \textstyleEmphasizedVernacularWords{eliwa},  mua  oko  fain  \textstyleEmphasizedVernacularWords{wia}. \\
1s/p.father  work-\textsc{nmz}  good  man  other  this  no\\
\glt `My father's work is better than this other man's. (Lit: My father's work is good, this other man's is not.) '
\z

According to \citet{Stassen2008} this \textstyleEmphasizedWords{\textsc{Conjoined Comparative}} strategy, exemplified above, is prevalent in  Australia and New Guinea. But the sample of New Guinean languages used for the generalization is very small (and includes both Austronesian and Papuan languages), and I suggest that at least for \textstyleAcronymallcaps{TNG} languages the \textstyleEmphasizedWords{\textsc{Exceed Comparative}}, the strategy represented in that sample only by Amele \citep[134--135]{Roberts1987}, is a possible alternative and may actually be as common as, or more common than, the Conjoined Comparative strategy.\footnote{My opinion is mainly based on the experience of working with national translators. When searching for translation equivalents for comparison forms, they often start with the Conjoined Comparative pattern, but very soon after realising that they do not have to stay within the limits of stative clauses only or stick to the adjective class, many actually tend to prefer the Exceed Comparative as the more natural and accurate expression for comparison.  \citet[68]{Reesink1987} mentions both of these mechanisms for Usan. }   There are two clauses in this pattern too: one may be equative and contain an adjective, the other is a transitive clause containing the verb \textstyleStyleVernacularWordsItalic{nomak}- `exceed/surpass' as the predicate and the standard of comparison as the object. The order of the two clauses is free.

\ea%x1327
\label{ex:6:x1327}
\gll Maa  mane-maneka,  maa  fain  \textstyleEmphasizedVernacularWords{nomak-ep}  ik-ua. \\
thing  \textsc{rdp}-big  thing  this  surpass-\textsc{ss}.\textsc{seq}  be-\textsc{pa}.3s\\
\glt `They are big things, greater than these.'
\z

\ea%x1328
\label{ex:6:x1328}
\gll No  yiena  nembesir  \textstyleEmphasizedVernacularWords{nomak-ep}  maneka ar-ek-a-m  na-ep=i?\\
2s.\textsc{unm}  1p.\textsc{gen}  ancestor  surpass-\textsc{ss}.\textsc{seq}  big become-\textsc{cntf}-\textsc{pa}-1s  say-\textsc{ss}.\textsc{seq}=\textsc{qm}\\
\glt `Do you want to become greater than our ancestors?'
\z

\ea%x1333
\label{ex:6:x1333}
\gll Nomokowa  kakawa  fain  iiwa,  oko  \textstyleEmphasizedVernacularWords{nomak-ep}  puuk-a-m. \\
tree  strip  this  short  other  surpass-\textsc{ss}.\textsc{seq}  cut-\textsc{pa}-1s.\\
\glt `This piece of timber is short, I cut the other one longer.'
\z

A transitive clause with \textstyleStyleVernacularWordsItalic{nomak}- is also used, when a noun rather than an adjective describes the characteristic under comparison. 

\ea%x1329
\label{ex:6:x1329}
\gll O  kekan-owa=ke  yo  kekan-owa  efar \textstyleEmphasizedVernacularWords{nomak-e-k}.\\
3s.\textsc{unm}  be.strong-\textsc{nmz}=\textsc{cf}  1s.\textsc{unm}  be.strong-\textsc{nmz}  1s.\textsc{dat} surpass-\textsc{pa}-3s\\
\glt `He is stronger than I. (Lit: His strength surpasses my strength.)'
\z

\ea%x1894
\label{ex:6:x1894}
\gll Mua  oko=ke  ikiwosa/amisa  efar  \textstyleEmphasizedVernacularWords{nomak-e-k}. \\
man  other=\textsc{cf}  head/knowledge  1s.\textsc{dat}  surpass-\textsc{pa}-3s\\
\glt `Someone else is more intelligent than I. (Lit: {\dots}surpasses my head/knowledge.)'
\z

In the following example, \textstyleStyleVernacularWordsItalic{nomak}- is employed to compare arrival times:

\ea%x1895
\label{ex:6:x1895}
\gll ...wia  \textstyleEmphasizedVernacularWords{nomak-ep}  me  miim-ep  ... urup-i-yen,  weetak.\\
3p.\textsc{acc}  surpass-\textsc{ss}.\textsc{seq}  not  precede-\textsc{ss}.\textsc{seq}  {\dots} ascend-Np-\textsc{fu}.1p  no\\
\glt `{\dots} we'll not go up earlier than they, no.'
\z

For superlatives, the quantifier \textstyleStyleVernacularWordsItalic{unowiya} `all' may be used in the object \textstyleAcronymallcaps{NP}.

\ea%x1330
\label{ex:6:x1330}
\gll No  unuma  nain  mua  \textstyleEmphasizedVernacularWords{unow=iya}  wia  \textstyleEmphasizedVernacularWords{nomakek}. \\
2s.\textsc{unm}  name  that1  man  many=\textsc{com}  3p.\textsc{acc}  surpass-\textsc{pa}-3s\\
\glt `You are the most important of all people.' (Lit: `Your name surpasses all people.')
\z

In the following example, the two comparison strategies are employed in the same sentence, and the intensifier \textstyleStyleVernacularWordsItalic{akena} `very' indicates superlative:

\ea%x1337
\label{ex:6:x1337}
\gll Poka  fain  \textstyleEmphasizedVernacularWords{maala},  nain  \textstyleEmphasizedVernacularWords{nomak-e-k},  ne  oko  nain  \textstyleEmphasizedVernacularWords{maala} \textstyleEmphasizedVernacularWords{akena}.\\
stilt  this  long  that1  surpass-\textsc{pa}-3s  \textsc{add}  other  that1  long very\\
\glt `This stilt is longer than that one, and/but that other one is the longest.'
\z

When there is a difference between things that are compared but the difference is not graded, the phrase \textstyleStyleVernacularWordsItalic{sira oko} 'different (lit: another kind)' is used to modify the noun.

\ea%x1334
\label{ex:6:x1334}
\gll Iwakara  \textstyleEmphasizedVernacularWords{sira} \textstyleEmphasizedVernacularWords{oko}  miim-ap  baurar-e-mik. \\
neck  kind  other  hear-\textsc{ss}.\textsc{seq}  flee-\textsc{pa}-1/3p\\
\glt `They heard a different voice and ran away.'
\z

\ea%x1335
\label{ex:6:x1335}
\gll Takira  opor(a)  \textstyleEmphasizedVernacularWords{sira}  \textstyleEmphasizedVernacularWords{oko}=ko  me  wia  maak-e. \\
youngster  talk  kind  other=\textsc{nf}  not  3p.\textsc{acc}  tell-\textsc{imp}.2s\\
\glt `Don't tell different things to the youngsters (from what you are supposed to tell them).'
\z

\subsection{Comparison of similarity: equative constructions}\footnotemark{}
%\hypertarget{RefHeading22661935131865}

\footnotetext{ The term ``equative construction'' is not to be confused with equative clauses discussed in 5.6.1.}\todo[inline]{please check}
A possible outcome of comparison is that the compared items, or actions, are identical or similar rather than different. Mauwake has several ways of expressing similarity. 

For an equivalent of  `as \textstyleAcronymallcaps{\textsc{adj}} as' structure, the intensity adverb \textstyleStyleVernacularWordsItalic{pepek} 'enough' is used, often together with another intensifier.

\ea%x1331
\label{ex:6:x1331}
\gll No  merena  \textstyleEmphasizedVernacularWords{maneka}  yo  merena  \textstyleEmphasizedVernacularWords{iiwawun}  \textstyleEmphasizedVernacularWords{pepek}. \\
2s.\textsc{unm}  foot  big  1s.\textsc{unm}  foot  altogether  enough\\
\glt `Your feet are big, just as big as my feet.' Or: `Your big feet are just as big as mine.'
\z

\ea%x1332
\label{ex:6:x1332}
\gll Urauwa  maala  Moro  owowa  \textstyleEmphasizedVernacularWords{maala}  \textstyleEmphasizedVernacularWords{pepek}  \textstyleEmphasizedVernacularWords{akena}. \\
hole  long  Moro  village  long  enough  very\\
\glt `The hole (is) as deep as Moro village is long.'
\z

The two most common words used in equative constructions are the deictic manner adverb \textstyleStyleVernacularWordsItalic{naap} `thus, like that' (\sectref{sec:3.9.1.3}) and the postposition \textstyleStyleVernacularWordsItalic{saarik} `like' (\sectref{sec:3.12.3}). \textstyleStyleVernacularWordsItalic{Naap} is used to compare things that are essentially the same, even identical.

\ea%x1338
\label{ex:6:x1338}
\gll Auwa  mia  maneka,  muuka  pun  \textstyleEmphasizedVernacularWords{naap}. \\
1s/p.father  body  big  son  also  like.that\\
\glt `The father is big, (and) the son is like that too.'
\z

\ea%x1339
\label{ex:6:x1339}
\gll Muuka  nain  (ona)  wiawi  \textstyleEmphasizedVernacularWords{naap}. \\
boy  that1  3s/p.\textsc{gen}  father  like.that\\
\glt `The boy/son is like his father.'
\z

\ea%x1348
\label{ex:6:x1348}
\gll I  maa  en-owa  \textstyleEmphasizedVernacularWords{naap}  nain  yienak-e. \\
1p.\textsc{unm}  food  eat-\textsc{nmz}  like.that  that1  feed.us-\textsc{imp}.2s\\
\glt `Give us food like that.'
\z

Also the corresponding proximal manner adverb \textstyleStyleVernacularWordsItalic{feenap} 'like this' is used occasionally:

\ea%x1347
\label{ex:6:x1347}
\gll Uura  \textstyleEmphasizedVernacularWords{feenap}  nain,  wi  wilkar  nain  muf-e-mik. \\
night  like.this  that1  3p.\textsc{unm}  cart  that1  pull-\textsc{pa}-1/3p\\
\glt `On nights like this they pulled the carts.'
\z

The postposition \textstyleStyleVernacularWordsItalic{saarik} `like' expresses some similarity between two essentially different things. The actual point of similarity may be expressed explicitly \REF{ex:6:x1341} or left implied \REF{ex:6:x1340}.

\ea%x1341
\label{ex:6:x1341}
\gll Pon  oposia  eliwa,  aara  oposia  \textstyleEmphasizedVernacularWords{saarik}. \\
turtle  meat  good  chicken  meat  like\\
\glt `Turtle meat is good, like chicken meat.'
\z

\ea%x1340
\label{ex:6:x1340}
\gll Mera  iperuma  ifa  \textstyleEmphasizedVernacularWords{saarik}. \\
fish  eel  snake  like\\
\glt `An eel is like a snake.'
\z

The similarity may not be a particular quality, expressable with an adjective. In the following example it is the number of different things that is compared.

\ea%x1342
\label{ex:6:x1342}
\gll Ulingan  fa=na  iinan  aasa  nepa  \textstyleEmphasizedVernacularWords{saarik,}  unow(a)  akena. \\
Ulingan  \textsc{intj}=\textsc{tp}  sky  canoe  bird  like  many  very\\
\glt `Ulingan -- wow -- the airplanes were like birds, there were lots of them.'
\z

When \textstyleStyleVernacularWordsItalic{saarik} is postposed after a nominalized verb, it indicates pretension. This is a case of a similarity of action, but not ``the real thing''.

\ea%x1343
\label{ex:6:x1343}
\gll Moram  era  \textstyleEmphasizedVernacularWords{paayar-owa}  \textstyleEmphasizedVernacularWords{saarik}  fan  yia p-or-o-n?\\
why  road  understand-\textsc{nmz}  like  here  1p.\textsc{acc} \textsc{bp}x-descend-\textsc{pa}-2s\\
\glt `Why did you bring us down here as if you knew the road?'
\z

\ea%x1344
\label{ex:6:x1344}
\gll O  Menamura  \textstyleEmphasizedVernacularWords{or-owa}  \textstyleEmphasizedVernacularWords{saarik}  iwera  fook-a-k. \\
3s.\textsc{unm}  Manam  descend-\textsc{nmz}  like  coconut  split-\textsc{pa}-3s\\
\glt `He split coconuts (for copra), as if he were going to Manam.'
\z

In other cases it may not indicate pretension but a false or ungrounded expectation: 

\ea%x1345
\label{ex:6:x1345}
\gll Yo  \textstyleEmphasizedVernacularWords{efa}  \textstyleEmphasizedVernacularWords{sesenar-owa}  \textstyleEmphasizedVernacularWords{saarik}  oram  maneka uuw-owa  yoowa  on-a-m. \\
1s.\textsc{unm}  1s.\textsc{acc}  buy-\textsc{nmz}  like  for.nothing  big work-\textsc{nmz}  hot  do-\textsc{pa}-1s\\
\glt `I worked hard for nothing, as if they would pay me for it (lit: buy me).'
\z

The phrase \textstyleStyleVernacularWordsItalic{nainiw} \textstyleEmphasizedVernacularWords{akena} `exactly like' is reserved for the cases of striking similarity: 

\ea%x1346
\label{ex:6:x1346}
\gll Wiipa  nain  onak  miikapura  \textstyleEmphasizedVernacularWords{nainiw}  \textstyleEmphasizedVernacularWords{akena}. \\
girl  that1  3s/p.mother  face  again  very\\
\glt `The girl's face is exactly like her mother's.'
\z

