%9

\chapter{Theme, topic and focus}
\hypertarget{RefHeading23761935131865}{}
Three features of textual prominence, the pragmatic functions theme, topic and focus, are discussed in this chapter.  All of them play an important role in Mauwake, and they show up in morphology and/or syntax. They are not mutually exclusive: a clausal constituent may have more than one pragmatic function. 

Theme, topic and focus have been defined in linguistic literature in several different and sometimes conflicting ways, so they need a definition of how they are used here. 

The definitions of topic are mainly divided along two questions: whether the topic needs to be an entity -- more specifically an argument -- or not, and whether it functions on clause or discourse level, or both. 

One classic definition describes the topic as ``the entity about which something is said, whereas the further statement made about this entity is the comment'' (Crystal 1997, see also Dik 1978:19). It treats the topic as a clause-level function and an entity, but does not specify whether this entity needs to be an argument of the verb or not.  Chafe's (1976:50) well-known definition also discusses the topic on the sentence level only. Any constituent may be a topic, in fact it need not even be an entity: ``the topic sets a spatial, temporal or individual framework within which the main predication holds'' (see also Li and Thompson 1976:461). Haiman's (1978) analysis of conditionals as topics is based on this definition, as the protasis in the conditional clauses provides the presupposition for the assertion in the apodosis.

The definitions above do not touch upon topic continuity, which is an important  object of study for those linguists who consider topic mainly from discourse point of view (Giv\'on 1976, 1983a, 1990).  In this case the topic function can only be assigned to an argument of a clause. Also \citet[340]{Dixon2010a} defines topic as ``an argument which occurs in a succession of clauses in a discourse and binds them together''.\footnote{In some other approaches this discourse topic has also been called \textit{theme} or \textit{global topic}.} A single sentence can be said to have a topic only if the sentence constitutes at least a clause chain or a paragraph (Giv\'on 1990:902). 

In the following, \textstyleEmphasizedWords{\textsc{topic}} is understood in the sense that Giv\'on advocates, whereas the term \textstyleEmphasizedWords{\textsc{theme}} is used to refer to Chafe's ``topic'', for which a sentence-initial position is crucial.\footnote{Considering ``topicality'' in Chafe's sense, Mauwake is basically a subject-prominent language (Li \& Thompson 1976). It has the following characteristics: surface coding for the subject as the first argument and as the argument that governs verb agreement; scarcity of ``double subject'' constructions, even though they are possible; the subject controls co-referential constituent deletion; there are constraints on the ``topic'' constituent; and the frequency of topic-comment clauses is low. But Mauwake shares the following  features with topic-prominent languages: there is no passivization, neither are there any empty or dummy subjects.} What \citet[19]{Dik1978} calls a theme is here called a \textstyleEmphasizedWords{\textsc{left-dislocated theme}}.

Mauwake is a \textstyleAcronymallcaps{SOV} language and the default topic is also the syntactic subject and the semantic agent/actor, and yet the first \textstyleAcronymallcaps{NP} in a clause often is not the topic. This is because once the topic has been established, it is normally only marked by verbal suffixes, and the clause-initial position is taken by another constituent.  

\section{Theme}
\hypertarget{RefHeading23781935131865}{}
The position as the leftmost non-verb constituent in a sentence defines the theme in Mauwake. It may be an argument or a peripheral. When a sentence -- or the first clause in a multi-clause sentence -- consists of a verb only, there is no theme in the sense adopted here. When the theme is an argument, it introduces what the sentence is about (\stepcounter{nx}{\thenx}). When it is not an argument but a peripheral, it provides a circumstancial setting for the sentence, most commonly a locative or temporal setting (\stepcounter{nx}{\thenx}). A theme forms one intonation contour with the rest of the clause. 

\ea%x1908
\label{ex:x1908}
\gll \textstyleEmphasizedVernacularWords{Wi}  \textstyleEmphasizedVernacularWords{owow  mua=ke}  wilkar  wia  \\
      \\
\glt
\z

3p.UNM  village  man=CF  cart  3p.ACC  

muf-em-ik-om-a-mik.

pull-SS.SIM-be-BEN-BNFY2.PA-1/3p

`The village men pulled carts for them.'

\ea%x1698
\label{ex:x1698}
\gll Ne  \textstyleEmphasizedVernacularWords{fraide=pa}  maapora  puk-o-k,  urera.  \\
      \\
\glt
\z

ADD  Friday=LOC  party  burst-PA-3s  afternoon

`And on Friday the party started, in the afternoon.'

When the theme coincides with the subject/topic (\stepcounter{nx}{\thenx}), the clause has the default word order. But when another argument is the theme, it takes the initial position, and if there is also a subject NP in the same clause, it follows the theme  NP: 

\ea%x1473
\label{ex:x1473}
\gll [\textstyleEmphasizedVernacularWords{I  yar}]\textsubscript{O}  [i]\textsubscript{S}  uruf-am-ik-omkun \\
      \\
\glt
\z

1p.UNM  1s/p.brother-in-law  1p.UNM  see-SS.SIM-be-1s/p.DS

o  koora=pa  pok-ap  ik-ua.

3s.UNM  house=LOC  sit.down-SS.SEQ  be-PA.3s

`Our brother-in-law, as we are seeing him, is sitting in his house.' (Lit: `Our brother-in-law we are seeing and he is sitting in his house.')

The sentence (\stepcounter{nx}{\thenx}) is from the middle of a description about the arrival of the Japanese troops, and the goods that they brought are only mentioned in this one sentence, so the theme is neither the subject nor the topic. 

\ea%x1701
\label{ex:x1701}
\gll [\textstyleEmphasizedVernacularWords{Maa}  \textstyleEmphasizedVernacularWords{unowa}]\textsubscript{O}  ifer  aasa=ke  p-urup-eya  \\
      \\
\glt
\z

thing  many  sea  canoe=CF  Bpx-ascend-2/3s.DS

miiw-aasa=ke  fan  p-ir-am-ik-ua.

land-canoe=CF  here  Bpx-come-SS.SIM-be-PA.3s

`The many things were brought up (to the coast) by ships and brought here by trucks.'

In a text about a school party, dancing is first mentioned in a final verb, and then the dance becomes the theme for the following sentence:

\ea%x1702
\label{ex:x1702}
\gll Naap  ik-ok  wi  Saramun=ke  wiisa  uf-e-mik. \\
      \\
\glt
\z

thus  be-SS  3p.UNM  Saramun=CF  dance.name  dance-PA-1/3p

[\textbf{Uf-owa  eliwa}]\textsubscript{O}  i  wiar  uruf-a-mik.

dance-NMZ  good  1p.UNM  3.DAT  see-PA-1/3p

`Then the Saramun people danced \textit{wiisa}. It was a good dance we saw from them.'

Very commonly the subject only shows in the verbal suffixation, and the theme position is taken either by an object -- which may or may not be a topic -- or by an adverbial phrase. In (\stepcounter{nx}{\thenx}) the theme \textstyleStyleVernacularWordsItalic{auwa ame} `father and the others' becomes a topic that continues for the next five clauses, whereas the themes of (\stepcounter{nx}{\thenx}) and (\stepcounter{nx}{\thenx}) do not become topics and are not mentioned any more. (In the free translation is not often possible to reflect the theme naturally.)

\ea%x1909
\label{ex:x1909}
\gll \textstyleEmphasizedVernacularWords{Auwa}  \textstyleEmphasizedVernacularWords{ame}  wia  maak-eya  res  aaw-ep  \\
      \\
\glt
\z

1s/p.father  ASSOC  3p.ACC  tell-2/3s.DS  razor  take-SS.SEQ  

merena  ifa  keraw-a-k  nain  puuk-a-mik.

leg  snake  bite-PA-3s  that1  cut-PA-1/3p

`He told my father and the others, and they took  a razor and made a cut into the leg that the snake had bitten.'

\ea%x1910
\label{ex:x1910}
\gll \textstyleEmphasizedVernacularWords{Maa}  \textstyleEmphasizedVernacularWords{en-owa}  nopa-yiaw-ep  wailal-ep  \\
      \\
\glt
\z

thing  eat-NMZ  search-move.around-SS.SEQ  hunger-SS.SEQ  

naap  ma-e-mik...

thus  say-PA-1/3p

`They searched around for food and were hungry and said like that{\dots}'

\ea%x1911
\label{ex:x1911}
\gll \textstyleEmphasizedVernacularWords{Emeria}  naap  wia  aruf-i-nen  na-ep  on-a-k. \\
      \\
\glt
\z

woman  thus  3p.ACC  hit-Np-FU.1s  say/think-SS.SEQ  do-PA-3s

`He tried to hit the women like that.'

In a tail-head linkage construction (\sectref{sec:8.2.3.5}) the final verb of a sentence is repeated in the beginning of the following sentence, but in a medial form. An argument (\stepcounter{nx}{\thenx}), (\stepcounter{nx}{\thenx}), or occasionally a peripheral (\stepcounter{nx}{\thenx}), (\stepcounter{nx}{\thenx}), from the final clause may be picked as the theme of the new sentence.

\ea%x1912
\label{ex:x1912}
\gll Owowa  or-op,  wailal-ep  akia  ik-e-k.  \\
      \\
\glt
\z

village  descend-SS.SEQ  hunger-SS.SEQ  banana  roast-PA-3s  

\textstyleEmphasizedVernacularWords{Akia}  ik-ep  en-em-ik-ok{\dots}

banana  roast-SS.SEQ  eat-SS.SIM-be-SS

`He came down to the village, was hungry and roasted bananas. He roasted bananas and was eating them and {\dots}'

\ea%x1913
\label{ex:x1913}
\gll Aria,  wi  kiiriw  neeke  {\O}  miiw-aasa  um-o-k.  \\
      \\
\glt
\z

alright  3p.UNM  again  there.CF  {\O}  land-canoe  die-PA-3s

\textstyleEmphasizedVernacularWords{Miiw-a}\textstyleEmphasizedVernacularWords{asa}  um-eya  miiw-aasa  nain  on-am-ika-iwkin...

land-canoe  die-2/3s.DS  land-canoe  that1  do-SS.SIM-be-2/3p.DS

`Alright, again when they (were) there the truck broke down (lit: died). The truck broke down, and while they were working on the truck{\dots}'

  (\stepcounter{nx}{\thenx}x1914)  Yaki-ep  weeser-eya  owowa  urup-e-mik.

bathe-SS.SEQ  finish-2/3s.DS  village  ascend-PA-1/3p  

\textstyleEmphasizedVernacularWords{Owowa}  urup-ep  o  koora  ikiw-o-k.

village  ascend-SS.SEQ  3s.UNM  house  go-PA-3s

`They bathed and when it was finished they came up to the village. They came to the village and he went into the house.'

\ea%x1915
\label{ex:x1915}
\gll ...siowa  wiawi  nain=ke  alu-owa  miim-ap  \\
      \\
\glt
\z

dog  3s/p.father  that1=CF  make.noise-NMZ  hear-SS.SEQ  

karu-(o)w=iw  ekap-o-k.  \textstyleEmphasizedVernacularWords{Karu-(o)w=iw}  ekap-ep

run-NMZ=INST  come-PA-3s  run-NMZ=INST  come-SS.SEQ  

uruf-a-k=na {\dots}

see-PA-3s=TP

`The dog's owner heard the noise and came running. He came running and saw{\dots}'

It is more common to have a tail-head linkage where only the final verb is repeated; when the speaker repeats an argument or a peripheral as well, there is a reason for it: to give it prominence as the theme in the new sentence. 

When the theme position is taken by a temporal or locative adverbial phrase, it normally gives a setting for the the whole sentence:  

\ea%x1699
\label{ex:x1699}
\gll \textstyleEmphasizedVernacularWords{Eka  mamaiya  akena}  i  yoowa  me  aaw-i-yen. \\
      \\
\glt
\z

river  close  very  1p.UNM  hot  not  get-Np-FU.1p

`Very close to the river we won't get hot.'

\ea%x1916
\label{ex:x1916}
\gll \textstyleEmphasizedVernacularWords{Ikoka  kuisow}  miiw-aasa=ke  karu-eya  ku-ku-ep  \\
      \\
\glt
\z

later  one  land-canoe=CF  run-2/3s.DS  RDP-break-SS.SEQ  

or-om-ik-ua.

descend-SS.SIM-be-PA.3s

`Straight away when the trucks ran (over them) they kept breaking and falling down.'

But in the following example the first temporal phrase is a setting for only the first clause, and the final clause has another temporal phrase. Also, in the second sentence the object is both a new topic (\sectref{sec:9.2.1}) and fronted as the theme before the temporal adverbial. In neutral constituent order a temporal adverbial precedes the object. 

\ea%x1703
\label{ex:x1703}
\gll \textstyleEmphasizedVernacularWords{Uura  feenap  nain}  i  me  in-em-ik-e-mik, \\
      \\
\glt
\z

night  like.this  that1  1p.UNM  not  sleep-SS.SIM-be-PA-1/3p

amirika  maa  me  en-em-ik-e-mik.

day  food  not  eat-SS.SIM-be-PA-1/3p

\textstyleEmphasizedVernacularWords{Maa}  uura  uup-ep  en-em-ik-e-mik.

food  night  cook-SS.SEQ  eat-SS.SIM-be-PA-1/3p

`On nights like this we did not sleep, in the daytime we did not eat food. The food we used to cook and eat at night.'

Other adverbial phrases may also be used in the theme position to provide a circumstantial setting. In particular the deictic manner adverbial \textstyleStyleVernacularWordsItalic{naap} `thus, like that' is relatively common.

\ea%x1917
\label{ex:x1917}
\gll \textstyleEmphasizedVernacularWords{Naap}  maak-iwkin  naap  ik-ua.  \textstyleEmphasizedVernacularWords{Naap}  ik-ok  \\
      \\
\glt
\z

thus  tell-2/3p.DS  thus  be-PA.3s  thus  be-SS  

uruf-am-ika-iwkin  wia.

see-SS.SIM-be-2/3p.DS  no

`Like that they told him and like that he was. Like that he was and they watched him, but no (he did not get better).'

\ea%x1918
\label{ex:x1918}
\gll \textstyleEmphasizedVernacularWords{Wiena}  \textstyleEmphasizedVernacularWords{merena  ne  wapen=iw}  era  akup-ami  owowa  \\
      \\
\glt
\z

3p.GEN  foot  ADD  hand=INST  road  search-SS.SIM  village  

ikiw-e-mik.

go-PA-1/3p

`With their feet and hands they searched the road and went to the village.'

A sentence-initial adverbial phrase that is syntactically outside the clause and also has its own slightly rising intonation contour on the last syllable is here called a \textstyleEmphasizedWords{\textsc{left-dislocated theme}}. In the written text it is separated from the rest of the clause by a comma.This clause-external pragmatic function is called theme by \citet[19]{Dik1978}. He defines its function as ``specif[ying] the universe of discourse with respect to which the subsequent predication is presented as relevant''.

 In the following example, the left-dislocated theme consists of a relative clause where the antecedent noun \textstyleStyleVernacularWordsItalic{soo} `fish trap' has been deleted:

\ea%x1704
\label{ex:x1704}
\gll Aria  [\textstyleEmphasizedVernacularWords{{\O}}  \textstyleEmphasizedVernacularWords{malol=pa  ifemak-i-mik  nain}]\textsubscript{RC},  aana  \\
      \\
\glt
\z

alright  {\O}  deep.sea=LOC  press-Np-PR.1/3p  that1  cane  

puuk-i-mik  ...

cut-Np-PR.1/3p

`Alright, as for those (=fishtraps) that we let down in the deep sea, we cut cane...'

There can be more than one dislocated theme for the same clause. In (\stepcounter{nx}{\thenx}) there are two dislocated themes -- a temporal and a locative phrase -- plus a clause-internal theme \textstyleStyleVernacularWordsItalic{moma} `taro', which is the syntactic object of the clause:

\ea%x1700
\label{ex:x1700}
\gll \textstyleEmphasizedVernacularWords{Iiriw},  \textstyleEmphasizedVernacularWords{owow(a)  oko  mua  manina},  moma    \\
      \\
\glt
\z

earlier  village  other  man  garden  taro    

waaya=ke  anane  wiar  en-ow(a)=iw  ika-i-ya.

pig=CF  always  3.DAT  eat-NMZ=INST  be-Np-PR.3s

`Earlier, (in) the garden of a man from another village, his taro was always being eaten by a pig.'

\section{Topic} 
\hypertarget{RefHeading23801935131865}{}
Giv\'on (1976:152) posited a universal topicality hierarchy, which shows features affecting the likelihood of \textstyleAcronymallcaps{NP}s becoming discourse topics: 

a.  human {{\textgreater}} non-human

b.  definite {{\textgreater}} indefinite

c.  more involved participant {{\textgreater}} less involved participant

d.  1\textsuperscript{st} person {{\textgreater}} 2\textsuperscript{nd} person {{\textgreater}} 3\textsuperscript{rd} person

This hierarchy can be observed in Mauwake as well: the prototypical topic refers to a referent that is human and definite, and if the first person is involved in the text, it is often the topic. And the most involved participant, the grammatical subject, is typically also the pragmatic topic. 

The following three sections discuss how a new topic is introduced and maintained in a narrative text, and how it is brought back after it has been absent for a while. 

\subsection{Introducing a new topic}
\hypertarget{RefHeading23821935131865}{}
Even when a new topic is introduced for the first time, it is often definite,\footnote{Definiteness is not an obligatory category in Mauwake. A NP may be marked as definite or indefinite when this  feature is considered important enough, but often it is left unspecified.} identifiable to the addressee: a personal pronoun, a proper name or a relationship term, or a noun phrase. 

\ea%x1663
\label{ex:x1663}
\gll \textstyleEmphasizedVernacularWords{I}  me  amis-ar-em-ik-omkun  iinan  aasa  \\
      \\
\glt
\z

1p.UNM  not  knowledge-INCH-SS.SIM-be-1s/p.DS  sky  canoe

iinan=pa  fan  ekap-emi  ...

sky=LOC  here  come-SS.SIM

`We were not aware (that anything would happen) and planes came here on the sky and {\dots}'

\ea%x1664
\label{ex:x1664}
\gll \textstyleEmphasizedVernacularWords{Muakura=ke}  ma-e-k,  ``  {\dots''} \\
      \\
\glt
\z

Muakura=CF  say-PA-3s

`Muakura said, `` ... '' '

\ea%x1665
\label{ex:x1665}
\gll Aria  \textstyleEmphasizedVernacularWords{yena  mua } pun  ...  iirar-iwkin  owowa  ekap-o-k. \\
      \\
\glt
\z

alright  1s.GEN  man  also  ...  remove-2/3p.DS  village  come-PA-3s

`Alright they also dismissed my husband {\dots} and he came to the village.'

\ea%x1667
\label{ex:x1667}
\gll Iiriw  \textstyleEmphasizedVernacularWords{wi}  \textstyleEmphasizedVernacularWords{mua  iperowa=ke}  feenap  \\
      \\
\glt
\z

earlier  3p.UNM  man  middle.aged=CF  like.this

ma-em-ik-e-mik,  emeria=ke  osaiwa  ar-e-mik.

say-SS.SIM-be-PA-1/3p  woman=CF  bird.of.paradise  become-PA-1/3p

`Earlier the elders kept telling this story that women had changed into birds of paradise.'

When a potential topic is introduced it is indefinite -- the addressee is not expected to be able to identify it -- and one of the following strategies is used. The new topic may  first be an object in a clause before becoming the subject in the following clause. (\stepcounter{nx}{\thenx}) is repeated here: 

\ea%x1668
\label{ex:x1668}
\gll Uura  feenap  nain  i  me  in-em-ik-e-mik, \\
      \\
\glt
\z

night  like.this  that1  1p.UNM  not  sleep-SS.SIM-be-PA-1/3p

amirika  \textstyleEmphasizedVernacularWords{maa}  me  en-em-ik-e-mik.

noon  food  not  eat-SS.SIM-be-PA-1/3p

\textstyleEmphasizedVernacularWords{Maa}  uura  uup-ep  en-em-ik-e-mik.

food  night  cook-SS.SEQ  eat-SS.SIM.be-PA-1/3p

`On nights like this we did not sleep, at noon we did not eat food. Food we used to cook and eat at night.'

Most commonly, the new topic is already a subject in the clause where it is introduced, and the \textstyleAcronymallcaps{NP} is either modified with the indefinite \textstyleStyleVernacularWordsItalic{oko} `other' or marked by the neutral focus clitic \textit{-}\textstyleStyleVernacularWordsItalic{ko}, which also has its origin in \textstyleStyleVernacularWordsItalic{oko}. Occasionally both of them are used on the same \textstyleAcronymallcaps{NP} (\stepcounter{nx}{\thenx}). 

\ea%x1669
\label{ex:x1669}
\gll Iiriw  Malala  suule  maneka  \textstyleEmphasizedVernacularWords{uuw-owa  mua  oko}  unuma  Kila. \\
      \\
\glt
\z

earlier  Malala  school  big  work-NMZ  man  other  name  Kila

`Earlier there was a workman at the big Malala school whose name was Kila.'

\ea%x1670
\label{ex:x1670}
\gll \textstyleEmphasizedVernacularWords{Emer(a)}  \textstyleEmphasizedVernacularWords{en-ow(a)  mua=ko}  emeria  fan  aaw-o-k. \\
      \\
\glt
\z

sago  eat-NMZ  man=NF  woman  here  take-PA-3s

`A Sepik man married a wife here.'

\ea%x1671
\label{ex:x1671}
\gll Iiriw  akena  \textstyleEmphasizedVernacularWords{mua  oko=ko}  fura  aaw-ep  koka  iw-a-k. \\
      \\
\glt
\z

earlier  very  man  other=NF  knife  take-SS.SEQ  jungle  go-PA-3s

`Long ago a man took a knife and went into the jungle.'

The indefinite \textstyleAcronymallcaps{NP} may even have contrastive focus marking:

\ea%x1666
\label{ex:x1666}
\gll Pika  ifara  mufe-wiaw-ik-ok  \textstyleEmphasizedVernacularWords{ifa  maneka=ke}  siowa  \\
      \\
\glt
\z

fence  vine  pull-move.around-be-SS  snake  big=CF  dog

wiar  aaw-o-k.

3.DAT  take-PA-3s

`As he was pulling around vines for the fence, a big snake grabbed his dog.'

Existential clauses, which Giv\'on (1990:741) mentions as one of the major strategies for introducing important topics, are possible but not very commonly employed for this function in Mauwake. Note that the neutral focus clitic -\textstyleStyleVernacularWordsItalic{ko} is also present.

\ea%x1672
\label{ex:x1672}
\gll \textstyleEmphasizedVernacularWords{Iiriw}  \textstyleEmphasizedVernacularWords{mua  iperowa=ko}  nan  Wakoruma  owowa=pa  ik-ua. \\
      \\
\glt
\z

earlier  man  middle.aged=NF  there  Wakoruma  village=LOC  be-PA.3s

`Earlier there was a middle-aged man in Wakoruma village.'

\subsection{Maintaining an established topic} 
\hypertarget{RefHeading23841935131865}{}
When a potential topic, which is indefinite when first introduced, becomes established the next mention is often made with a \textstyleAcronymallcaps{NP} marked as definite by the distal-1 demonstrative \textstyleStyleVernacularWordsItalic{nain} `that'. The sentence  following (\stepcounter{nx}{\thenx}) in the text is (\stepcounter{nx}{\thenx}). 

\ea%x1673
\label{ex:x1673}
\gll \textstyleEmphasizedVernacularWords{Mua  nain}  emeria  ne  muuka  wiipa  marew. \\
      \\
\glt
\z

man  that1  woman  ADD  son  daughter  no(ne).

`The man had no wife or children.'

Another possibility is the mere subject marking on the verb: sentence (\stepcounter{nx}{\thenx}) below is continuation to the sentence (\stepcounter{nx}{\thenx}) above. 

\ea%x1674
\label{ex:x1674}
\gll Ne  manina  ikiw-o-\textstyleEmphasizedVernacularWords{k}. \\
      \\
\glt
\z

ADD  garden  go-PA-3s.

`And he went to the garden.'

When the topic is already definite when introduced, it is possible to make a second mention with a personal pronoun. (This whole story is Text 2 in Appendix 2.) 

\ea%x1919
\label{ex:x1919}
\gll Yena  yaiya  Tup  ifa  ku-o-k  nain  opaimika  \\
      \\
\glt
\z

1s.GEN  1s/p.uncle  Tup  snake  bite-PA-3s  that1  talk  

ma-i-yem.  Ae,  \textstyleEmphasizedVernacularWords{o}  fiker(a)  gone  urup-o-k.

say-Np-PR.1s  yes  3s.UNM  kunai.grass  middle  ascend-PA-3s

`I tell a story about that when my uncle Tup was bitten by a snake. Yes, he went up to the middle of the \textit{kunai} grass (area).'

In (\stepcounter{nx}{\thenx}) the personal pronoun \textstyleStyleVernacularWordsItalic{wi}  `they' in the second sentence refers to the ``weria-relatives'' (see \sectref{sec:1.3.6}), introduced as the object in the preceding sentence; a mere verbal suffix would indicate continuation with the old topic, i.e. those who sent the message. 

\ea%x1868
\label{ex:x1868}
\gll Wiena  mua  weria  ...  opaimika  wia  \\
      \\
\glt
\z

3p.GEN  man  planting.stick  ...  talk  3p.ACC

sesek-omak-e-mik.  Ne  \textstyleEmphasizedVernacularWords{wi}  ekap-e-mik.

send-DISTR/PL-PA-1/3p  ADD  3p.UNM  come-PA-1/3p

`They\textsubscript{i} sent word to their\textsubscript{i} many weria-relatives\textsubscript{j}. And they\textsubscript{j} came.' 

More commonly the topic, once established, is maintained as a continuing topic without an overt \textstyleAcronymallcaps{NP} or a pronoun, only via subject marking on the verb. This minimal marking conforms to Giv\'on's (1983a:67) claim that the heaviness of the topic marking is in inverse relation to topic continuity/predictability. The following example is a section of a text where \textstyleStyleVernacularWordsItalic{sawur} `spirits', introduced earlier, think that there is a boy on the bed they are carrying, but the boy has already escaped. The reference to the spirits is only made by medial and final verb suffixes. 

\ea%x1675
\label{ex:x1675}
\gll Ne  aria,  samapora  oram  akua  aaw-\textstyleEmphasizedVernacularWords{ep}  ikiw-e-\textstyleEmphasizedVernacularWords{mik}. \\
      \\
\glt
\z

ADD  alright,  bed  just  shoulder  take-SS.SEQ  go-PA-1/3p

Ikiw-\textstyleEmphasizedVernacularWords{ep}  wiena  owowa=pa  uruf-a-\textstyleEmphasizedVernacularWords{mik}=na  weetak,

go-SS.SEQ  3p.GEN  village=LOC  see-PA-1/3p=TP  no

samapora  muutiw  akua  aaw-e-\textstyleEmphasizedVernacularWords{mik}.

bed  only  shoulder  take-PA-1/3p

Aria  nainiw  kir-e-\textstyleEmphasizedVernacularWords{mik}.  Kir-\textstyleEmphasizedVernacularWords{ep}  ekap-em-ika-\textstyleEmphasizedVernacularWords{iwkin}

alright  again  turn-PA-1/3p  turn-SS.SEQ  come-SS.SIM-be-2/3p.DS

epa  wiim-o-k.

place  dawn-PA-3s

`Alright, they carried just the bed and went. They went and in their village they looked but (to their surprise) they only carried the bed. Alright they turned back again. They turned and as they were coming, it dawned.'

The switch-reference system (\sectref{sec:3.8.3.4}, 8.2.3) together with the person/number marking in the finite verbs can easily keep track of two active topics alternating with each other. 

\ea%x1676
\label{ex:x1676}
\gll O  iiwawun  samor  aaw-o-k.  Ne  nan  ik-e-\textstyleEmphasizedVernacularWords{mik}.  \\
      \\
\glt
\z

3s.UNM  altogether  badly  get-PA-3s  ADD  there  be-PA-1/3p

Nan  ik-\textstyleEmphasizedVernacularWords{ok}  ik-\textstyleEmphasizedVernacularWords{ok}  neeke  pu-o-\textstyleEmphasizedVernacularWords{k}.  Neeke  pu-\textstyleEmphasizedVernacularWords{eya}  oram

there  be-SS  be-SS  there.CF  die-PA-3s.  there.CF  die-2/3s.DS  just

akua  aaw-e-\textstyleEmphasizedVernacularWords{mik}.

shoulder  take-PA-1/3p

`He got really bad. They stayed and stayed there and he died there. He died there and they just carried him on their shoulders.'

In the example above one of the topics is in the singular, the other in the plural. But in (\stepcounter{nx}{\thenx}), repeated below as (\stepcounter{nx}{\thenx}), there are two third person singular topics alternating, with only the verbal marking to indicate who is doing what:

\ea%x1920
\label{ex:x1920}
\gll Ifakim-\textbf{eya}  \textstyleEmphasizedVernacularWords{\textmd{pu-ep-ik}}\textstyleEmphasizedVernacularWords{-eya } om-em-ik-\textbf{ua}. \\
      \\
\glt
\z

kill-2/3s.DS  die-SS.SEQ-be-2/3s.DS  cry-SS.SIM-be-PA.3s

`When she killed him and he was dead, she was crying.'

\subsection{Re-activating an earlier topic}
\hypertarget{RefHeading23861935131865}{}
When a topic has not been active for some time in the text, there are two main strategies to re-activate it. A personal pronoun is mainly used for the major participants. For the third person singular pronoun this is the most common usage (\sectref{sec:3.5.11}). The following example is from a text where the main participant, a man, has been killed by his spirit lover. For several sentences the topic position is taken by the spirit woman and her parents, but in the sentence (\stepcounter{nx}{\thenx}) the man, as a re-activated topic, gets up and goes to his village.

\ea%x1921
\label{ex:x1921}
\gll Epa  wiim-eya  sawur  emeria  nain  ikiw-eya  \\
      \\
\glt
\z

place  dawn-2/3s.DS  spirit  woman  that1  go-2/3s.DS  

\textstyleEmphasizedVernacularWords{o}  iikir-ami  owowa  ekap-o-k.

3s.UNM  get.up-SS.SIM  village  come-PA-3s

`It dawned and the spirit woman went, and he got up and came to the village.'

The sentence (\stepcounter{nx}{\thenx}) re-activates the topic, a grandmother and two grandchildren, after a gap of five clauses:

\ea%x1923
\label{ex:x1923}
\gll Iwera  mekemkar-ep  or-eya  \textstyleEmphasizedVernacularWords{wi}  pikin-ep  \\
      \\
\glt
\z

coconut  bend-SS.SEQ  descend-2/3s.DS  3p.UNM  jump-SS.SEQ

miiwa  or-o-mik.

ground  descend-PA-1/3p

`The coconut tree bent down and they jumped down to the ground.'

In the following stretch the health officer, who is one of the main participants in this section of the story, is mentioned as an object NP, and after two clauses he becomes the topic for just one clause. Afterwards the men accompanying the sick man again resume as the topic. 

\ea%x1924
\label{ex:x1924}
\gll ...ikemika  kaik-owa  mua  nain  nop-a-mik,  imen-ap  \\
      \\
\glt
\z

wound  tie-NMZ  man  that1  search-PA-1/3p  find-SS.SEQ  

maak-iwkin  \textstyleEmphasizedVernacularWords{o}  miim-o-k.  Aria,  \textstyleEmphasizedVernacularWords{wi}  kiiriw  neeke

tell-2/3p.DS  3s.UNM  precede-PA-3s  alright  3p.UNM  again  there.CF

{\O}  miiw-aasa  um-o-k.

{\O}  land-canoe  die-PA-3s

`{\dots} they searched for the health officer, and when they found him and told him, he went ahead of them (to the aidpost). Alright, again when they were there the truck broke down.'

A full noun phrase is used for reactivating major participants when there are several of them and pronouns are not adequate for disambiguating between them. A \textstyleAcronymallcaps{NP} is always used when minor participants and props\footnote{Participants are typically human and active in a narrative, props are non-human and inactive. }  are brought back to the stage. 

The following example is an extract from a story about the speaker's uncle, who was introduced with a kinship term at the very beginning of the story and only referred to by a verbal suffix or an occasional pronoun afterwards. An  important prop, snake poison, which becomes a topic for a short stretch, is referred to by a full \textstyleAcronymallcaps{NP}.

\ea%x1677
\label{ex:x1677}
\gll Akia  ik-ep  en-em-ik-ok  \textstyleEmphasizedVernacularWords{ifa  marasin } \\
      \\
\glt
\z

banana  roast-SS.SEQ  eat-SS.SIM.be-SS  snake  poison  

\textstyleEmphasizedVernacularWords{nain=ke}  kema  wiar  iw-a-k.

that1=CF  liver  3.DAT  go-PA-3s

Iw-aya  nan  miira  saawirin-e-k.

go-2/3s.DS  there  face  become.round-PA-3s

Ne  auwa  ame  wia  maak-eya  res  aaw-ep

ADD  1s/p.father  ASS  3p.ACC  tell-2/3s.DS  razor  take-SS.SEQ

merena  ...  nain  puuk-a-mik.  Puuk-ap  marasin

leg  {\dots}  that1  cut-PA-1/3p  cut-SS.SEQ  medicine

wu-om-a-mik.  Marasin  wu-om-a-mik=na

put-BEN-BNFY2.PA-1/3p  medicine  put-BEN-BNFY2.PA-1/3p=TP  

weetak.  Iiriw  \textstyleEmphasizedVernacularWords{ifa  marasin=ke}  kekan-e-k.

no  earlier  snake  poison=CF  be.strong-PA-3s  

Ne  akia  ik-e-k  nain  me  en-e-k.  Nan

ADD  banana  roast-PA-3s  that1  not  eat-PA-3s  there  

mukuna=pa  ik-eya \textstyleEmphasizedVernacularWords{} o  nan  samor  aaw-o-k.

fire=LOC  be-2/3s.DS  3s.UNM  there  badly  get-PA-3s

`He roasted bananas and when he was eating them the snake poison entered his liver. It entered and he felt dizzy there. And when he told my father and others, they took a razor and made a cut into the leg{\dots} They made a cut and put medicine into it. They put medicine but no (it didn't help). The snake poison was already strong. And he didn't eat the bananas that he roasted. They were there on the fire and he really got bad there.'

\subsection{Highlighted topic}
\hypertarget{RefHeading23881935131865}{}
The topic marker -(\textstyleStyleVernacularWordsItalic{e})\textstyleStyleVernacularWordsItalic{na} (\sectref{sec:3.12.7.1}) is only used with a new topic that the speaker wants to highlight. The constituent may have been briefly mentioned in an earlier clause (\stepcounter{nx}{\thenx}), (\stepcounter{nx}{\thenx}), or it is known from the outset as a future topic (\stepcounter{nx}{\thenx}), and now it is specified as the topic for the following section of text. 

\ea%x1680
\label{ex:x1680}
\gll Mauw-owa  kamenap  nain  on-a-man?  \textstyleEmphasizedVernacularWords{Mauw-owa}\textstyleEmphasizedVernacularWords{=na} \\
      \\
\glt
\z

work-NMZ  what.like  that1  do-PA-2p  work-NMZ=TP

sira  yia  nokar-e-mik,  yiena  kae  sira

custom  1p.ACC  ask-PA-1/3p  1p.GEN  1s/p.grandfather  custom

nain.

that1

`What kind of work did you do? - The work (was such that) they they asked us about customs, our ancestral customs.'

\ea%x1678
\label{ex:x1678}
\gll ...Filip  uruf-ap  maak-e-k,  ``Ikos  ikiw-u.''  \textstyleEmphasizedVernacularWords{Filip=na} \\
      \\
\glt
\z

...Filip  see-SS.SEQ  tell-PA-3s  together  go-IMP.1d  Filip=TP

ona  owowa  Suaru,  {\dots}

3s.GEN  village  Suaru

`He saw Filip and told him, ``We'll go together.'' Filip is/was from Suaru, {\dots}'

\ea%x1679
\label{ex:x1679}
\gll Yo  efa  aaw-eya  i  owawiya  ik-omkun  \\
      \\
\glt
\z

1s.UNM  1s.ACC  take-2/3s.DS  1p.UNM  together  be-1s/p.DS

\textstyleEmphasizedVernacularWords{Yaapan}\textstyleEmphasizedVernacularWords{=ena}  Wewak  kame=pa  nan  ir-a-mik.

Japan=TP  Wewak  side=LOC  there  come-PA-1/3p

`He married me, and while we were together the Japanese came from the Wewak side.'

The highlighted topic is more frequent in conversations than in narratives. Since the first and second persons are more topical than the third person, they may even get marked as a topic when the discussion itself is about something else. The second person singular pronoun with the topic marker, \textstyleStyleVernacularWordsItalic{nos-na,} has acquired the meaning somewhat like `you know' in English (\stepcounter{nx}{\thenx}).

\ea%x1684
\label{ex:x1684}
\gll \textstyleEmphasizedVernacularWords{Nos=na}? \\
      \\
\glt
\z

2s.FC=TP

`What about you?' or: `What do you want?' or: `Where are you going?'

\ea%x1685
\label{ex:x1685}
\gll Emar,  \textstyleEmphasizedVernacularWords{yos=na}  amina=ke  weetak. \\
      \\
\glt
\z

friend  1s.FC=TP  saucepan=CF  no

`Friend, I do not have a saucepan.'

\ea%x1686
\label{ex:x1686}
\gll \textstyleEmphasizedVernacularWords{Is}\textstyleEmphasizedVernacularWords{=na}  yoo,  takira  fain  ifa=ke  ku-eya  akua  \\
      \\
\glt
\z

1p.FC=TP  INTJ  boy  this  snake=CF  bite-2/3s.DS  shoulder

aaw-ep  ekap-em-ika-i-mik  yoo!

take-SS.SEQ  come-SS.SIM-be-Np-PR.1/3p  INTJ

`We - this boy was bitten by a snake and we are coming carrying him on our shoulders.'

\ea%x1687
\label{ex:x1687}
\gll Mauwa  ar-e-n,  amia=iya  nenar-e-mik=i?  \\
      \\
\glt
\z

what  become-PA-2s  bow=COM  shoot.you-PA-1/3p=QM

-Wia,  \textstyleEmphasizedVernacularWords{nos=na},  yo  fiker  fufa  iw-ap  nefa

No  2s.FC=TP  1s.UNM  kunai.grass  base  go-SS.SEQ  2s.ACC

far-i-yem.

call-Np-PR.1s

`What happened to you, did they shoot you with a gun? -No, you know, I went inside the \textit{kunai} grass and am calling you.'

\section{Focus constructions}\footnotemark{}
\hypertarget{RefHeading23901935131865}{}
\footnotetext{ This section is based on my earlier paper (J\"arvinen 1988b).}
The focus discussed here refers to special prominence given to some constituent in a clause \citep[174]{Dixon2010a}. In Mauwake it is not the same as new information, as the focused element can be either new or given information. And it cannot be contrasted with topic, as the topic may receive focus marking as well. 

The main discussion concentrates on the two focus clitics, but syntactic and phonological focusing devices are also briefly touched on.

Since focus refers to special prominence, it is possible to have clauses and sentences without any focus marking; in fact, most of the clauses do not have any. But it is also possible to have more than one focused item in a clause. 

\subsection{Contrastive focus}
\hypertarget{RefHeading23921935131865}{}
\citet{Chafe1976} lists the following factors as necessary for focus of contrast: the knowledge that someone did something, a set of possible candidates in the addressee's mind, and the assertion as to which of these candidates is the correct one. Thus there can be no contrast if the number of candidates is either unlimited or one. A contrastive sentence often contradicts the addressee's expectation, but this is not crucial; what is essential is that there is a set of possible candidates in the addressee's mind. Also, as  \citet[348]{Linde1979} remarks, marking contrast is not obligatory. Even if there are more candidates than one in the addressee's mind, it is still the speaker who decides whether to make something overtly contrastive or not.  

The contrastive focus marker in Mauwake is -\textstyleStyleVernacularWordsItalic{ke}, attached as a clitic to a noun phrase (\sectref{sec:3.12.7.2}). It can also follow a temporal or location word, although this is rare. The domain of the contrastive focus is one constituent. It is used when the \textstyleAcronymallcaps{NP} is in focus and is contrasted with something else. 

\ea%x1688
\label{ex:x1688}
\gll \textstyleEmphasizedVernacularWords{Os=ke}  ikiw-o-k. \\
      \\
\glt
\z

3s.FC=CF  go-PA-3s

`It was he who went.'

\ea%x1689
\label{ex:x1689}
\gll \textstyleEmphasizedVernacularWords{Mua  bug  maala  nain}\textstyleEmphasizedVernacularWords{=ke}  mera  unowa  isak-i-non, \\
      \\
\glt
\z

man  wind  long  that1=CF  fish  many  spear-Np-FI.3s

mua  bug  iiwa  nain  weetak.

man  wind  short  that1  no

`A man with big lungs will spear many fish, a man with small lungs no. 

Sometimes the use or non-use of the contrastive marker makes a difference in the interpretation of the meaning of a word. In (\stepcounter{nx}{\thenx}) \textstyleStyleVernacularWordsItalic{maneka} `big' refers to the size of the man as a neutral quality, but in (\stepcounter{nx}{\thenx}) either his big size is contrasted with the size of other people, or he is set apart as one of a limited set of big men, i.e. chiefs. 

\ea%x1690
\label{ex:x1690}
\gll Mua  nain  maneka. \\
      \\
\glt
\z

man  that1  big

`That man is big.'

\ea%x1691
\label{ex:x1691}
\gll Mua  nain  \textstyleEmphasizedVernacularWords{maneka=ke}. \\
      \\
\glt
\z

man  that1  big=CF

`That man is \textstyleStyleVernacularWordsItalic{\textsc{big}}\textsc{'} or: `That man is a big man/chief.'

The main placement for the contrastive focus marking is the subject (\stepcounter{nx}{\thenx}),\footnote{Waskia has an identical morpheme \textit{ke}, labeled as a subject marker, which is very similar in function \citep[36]{RossEtAl1978}%Paol
.} (\stepcounter{nx}{\thenx}), or the non-verbal predicate of a verbless clause (\stepcounter{nx}{\thenx}). In a few cases some other constituent is marked: a contrasted object or adverbial phrase.

\ea%x1692
\label{ex:x1692}
\gll Ne  \textstyleEmphasizedVernacularWords{erepam  nain=ke}  wiena  skul  stua  on-a-mik. \\
      \\
\glt
\z

ADD  four  that1=CF  3p.GEN  school  store  make-PA-1/3p

`And the fourth one they made into their school store.'

\ea%x1697
\label{ex:x1697}
\gll Maa  en-owa  iw-e-mik,  \textstyleEmphasizedVernacularWords{rais=ke}  weetak. \\
      \\
\glt
\z

food  eat-NMZ  give.him-PA-1/3p  rice=CF  no

`They gave him food (root crops), but not rice.'

\ea%x1693
\label{ex:x1693}
\gll \textstyleEmphasizedVernacularWords{Amirika}\textstyleEmphasizedVernacularWords{=ke}  eliw  ika-i-yem,  \textstyleEmphasizedVernacularWords{uura=ke}  napum-ar-i-yem. \\
      \\
\glt
\z

noon=CF  well  be-Np-PR.1s  night=CF  sick-INCH-Np-PR.1s

`At noon I am well, at night I am sick.'

\ea%x1694
\label{ex:x1694}
\gll \textstyleEmphasizedVernacularWords{Amiten}\textstyleEmphasizedVernacularWords{=ke}  ikiw-i-yem,  \textstyleEmphasizedVernacularWords{Susure=ke}  me  ikiw-i-yem. \\
      \\
\glt
\z

Amiten=CF  go-Np-PR.1s  Susure=CF  not  go-Np-PR.1s

`I go to Amiten but I don't go to Susure.'

The contrastive focus marker is also used for subject disambiguation. When an object of a clause is fronted as the theme, the constituent order changes from \textstyleAcronymallcaps{SOV} to \textstyleAcronymallcaps{OSV}. If both the subject and object are in the third person and are realized as overt NPs, this creates a potential ambiguity as to which one is which argument. The contrastive focus marker, added to the subject, may first have been used to disambiguate clauses like (\stepcounter{nx}{\thenx}) and later spread as an optional marking even to clauses where the verbal suffix distinguishes the two arguments and which would not need this extra marking (\stepcounter{nx}{\thenx}). Without the contrastive focus marking the example (\stepcounter{nx}{\thenx}) would mean `and my younger sibling got/took an arrow/a bullet'.

\ea%x1695
\label{ex:x1695}
\gll Ne  yena  aamun  \textstyleEmphasizedVernacularWords{ariwa=ke}  aaw-o-k. \\
      \\
\glt
\z

ADD  1s.GEN  1s/p.younger.sibling  arrow=CF  get-PA-3s

`And my younger sibling was killed by a bullet (lit: arrow).'

\ea%x1696
\label{ex:x1696}
\gll Fofa=pa  maa  mauwa  on-i-mik  nain  (\textstyleEmphasizedVernacularWords{yos=ke})  \\
      \\
\glt
\z

market=LOC  thing  what  do-Np-PR.1/3p  that1  1s.FC=CF

ma-i-yem.

say-Np-PR.1s

`I tell what we do at the market.' 

Although the contrastive focus marking is very common when both the subject and object in an \textstyleAcronymallcaps{OSV} clause are \textstyleAcronymallcaps{NP}s, it is not obligatory. See (\stepcounter{nx}{\thenx}) for an example. 

Also in other situations where there is ambiguity about the subject, the \textstyleAcronymallcaps{CF} marker is used. If (\stepcounter{nx}{\thenx}) did not have \textstyleAcronymallcaps{CF} marking, Pita would be interpreted as the possessor of the betelnuts and the meaning would be `He stole Pita's betelnuts from me'.

\ea%x1705
\label{ex:x1705}
\gll \textstyleEmphasizedVernacularWords{Pita}\textstyleEmphasizedVernacularWords{=ke}  owora  efar  ikum  aaw-eya  {\dots} \\
      \\
\glt
\z

Pita=CF  betelnut  1s.DAT  illicitly  take-2/3s.DS

`Pita stole my betelnuts, and {\dots}'

In (\stepcounter{nx}{\thenx}) the second \textstyleAcronymallcaps{CF} is necessary, because otherwise `this other one' would be interpreted as an object; the real object in the second clause clause is marked by zero.

\ea%x1706
\label{ex:x1706}
\gll Ikoka  \textstyleEmphasizedVernacularWords{masin  kaanin=ke}  samor-ar-eya  \textstyleEmphasizedVernacularWords{oko  fain=ke} \\
      \\
\glt
\z

later  engine  which=CF  bad-INCH-2/3s.DS  other  this=CF

asip-i-non.

help-Np-FU.3s

`Later when any of the engines breaks this other one will help it.'

A clause can only have one constituent with contrastive focus. In a verbless clause  either the subject or the non-verbal predicate can be marked with it, but not both.

\ea%x1707
\label{ex:x1707}
\gll Yo  \textstyleEmphasizedVernacularWords{owow(a)  saria=ke}. \\
      \\
\glt
\z

1s.UNM  village  headman=CF

`I am the village headman.'

\ea%x1708
\label{ex:x1708}
\gll Wia,  \textstyleEmphasizedVernacularWords{yos=ke}  owow  saria  ika-i-yem. \\
      \\
\glt
\z

no  1s.FC=CF  village  headman  be-Np-PR.1s

`No, \textit{I} am the village headman.'

Contrastive focus can be assigned to a constituent regardless of whether it is given or new, definite or indefinite. Non-verbal predicates with \textstyleAcronymallcaps{\textup{CF}} are mostly new information, whereas for subjects neither givenness nor definiteness matters.

\ea%x1709
\label{ex:x1709}
\gll Iperuma  nain  me  enim-eka,  \textstyleEmphasizedVernacularWords{inasin  mua=ke}. \\
      \\
\glt
\z

eel  that1  not  eat-IMP.2p  spirit  man=CF

`Don't eat the eel, it is a spirit man.'

\ea%x1710
\label{ex:x1710}
\gll Iir  oko  \textstyleEmphasizedVernacularWords{mua  oko=ke}  koora  ku-ek-a-m  na-ep \\
      \\
\glt
\z

time  other  man  other=CF  house  build-CNTF-PA-1s  say-SS.SEQ

maakara  war-ep  {\dots}

timber  cut-SS.SEQ

`Another time a man wanted to build a house and he cut timber and {\dots}'

\ea%x1711
\label{ex:x1711}
\gll \textstyleEmphasizedVernacularWords{Aaya}  \textstyleEmphasizedVernacularWords{nain=ke}  ifa  puuk-a-k. \\
      \\
\glt
\z

sugarcane  that1=CF  snake  change.into-PA-3s

`The sugarcane changed into a snake.'

In non-polar questions it is the questioned element that is in focus. This is reflected in the question words and in the answers. When the question word is the subject or the non-verbal predicate, it takes the \textstyleAcronymallcaps{CF} clitic, and the corresponding constituent in the answer usually gets the focus marking as well.

\ea%x1714
\label{ex:x1714}
\gll \textstyleEmphasizedVernacularWords{Mua  naareke}  nefa  maak-e-k? \\
      \\
\glt
\z

man  who.CF  2s.ACC  tell-PA-3s

`Who told you?'

\ea%x1715
\label{ex:x1715}
\gll \textstyleEmphasizedVernacularWords{Mua}\textstyleEmphasizedVernacularWords{=ke}  me  efa  maak-e-mik,  yena  mokok=iw  uruf-a-m. \\
      \\
\glt
\z

man=CF  not  1s.ACC  tell-PA-1/3p  1s.GEN  eye=INST  see-PA-1s

`It wasn't people that told me, I saw it with my own eyes. '

\ea%x1712
\label{ex:x1712}
\gll Maa  nain  \textstyleEmphasizedVernacularWords{mauwa=ke}? \\
      \\
\glt
\z

thing  that1  what=CF

`What is that thing?'

\ea%x1713
\label{ex:x1713}
\gll Maa  nain  \textstyleEmphasizedVernacularWords{posa=ke}. \\
      \\
\glt
\z

thing  that1  turban.shell

`That thing is a turban shell.'

The contrastive focus clitic and the question clitic -\textstyleStyleVernacularWordsItalic{i} merge into -\textstyleStyleVernacularWordsItalic{ki} when both are used with the same constituent. This happens when the non-verbal predicate of a verbless clause is questioned, in alternative questions, and sometimes in alternative statements.

\ea%x1716
\label{ex:x1716}
\gll Emeria  fain  \textstyleEmphasizedVernacularWords{Eema=ki}? \\
      \\
\glt
\z

woman  this  Eema=CF.QM

`Is this woman Eema?'

\ea%x1718
\label{ex:x1718}
\gll Emeria  fain  \textbf{Eema=ki } e  \textbf{emeria  oko=ke?} \\
      \\
\glt
\z

woman  this  Eema=CF.QM  or  woman  other=CF

`Is this woman Eema or another woman?'

\ea%x1717
\label{ex:x1717}
\gll \textstyleEmphasizedVernacularWords{Iwer(a)  eka}\textstyleEmphasizedVernacularWords{=ki}  e  \textstyleEmphasizedVernacularWords{mauwa=ki},  \textstyleEmphasizedVernacularWords{owora=ki},  \\
      \\
\glt
\z

coconut  water=CF.QM  or  what=CF.QM  betelnut=CF.QM

\textstyleEmphasizedVernacularWords{episowa}\textstyleEmphasizedVernacularWords{=ki}  ika-i-non  ...

tobacco=CF.QM  be-Np-FU.3s

`If there is coconut water, or something else, betelnut, or tobacco {\dots}'

\subsection{Neutral focus} 
\hypertarget{RefHeading23941935131865}{}
The neutral focus clitic (\sectref{sec:3.12.7.2}) most commonly occurs in irrealis-type clauses, i.e. questions, commands, negated clauses or those with future tense, hence its original name in J\"arvinen (1988b). But it is also used in a some realis-type clauses. Although the clitic has probably developed from the indefinite \textstyleStyleVernacularWordsxiiptItalic{oko} `a (certain), (an)other', it is added to definite noun phrases as well. 

\ea%x1719
\label{ex:x1719}
\gll \textstyleEmphasizedVernacularWords{Aaya}\textstyleEmphasizedVernacularWords{=ko}  niar  ik-ua=i? \\
      \\
\glt
\z

sugar=NF  2p.DAT  be-PA.3s=QM

`Do you have (any) sugar?'

\ea%x1720
\label{ex:x1720}
\gll Ikiw-ep  \textstyleEmphasizedVernacularWords{maa  en-owa=ko}  nop-aka. \\
      \\
\glt
\z

go-SS.SEQ  food  eat-NMZ=NF  fetch-IMP.2s

`Go and fetch (some) food.'

\ea%x1721
\label{ex:x1721}
\gll \textstyleEmphasizedVernacularWords{Owowa}  \textstyleEmphasizedVernacularWords{oko=ko}  me  uf-e-mik. \\
      \\
\glt
\z

village  other=NF  not  dance-PA-1/3p

`Other villages did not dance.'

\ea%x1722
\label{ex:x1722}
\gll Yo  aakisa  \textstyleEmphasizedVernacularWords{opaimika=ko}  ma-i-nen. \\
      \\
\glt
\z

1s.UNM  now  talk/story=NF  say-Np-FU.1s

`Now I will tell a story.'

In most of those few instances where the \textstyleAcronymallcaps{NF} clitic marks a constituent in a clearly realis-type clause, that constituent is a new, indefinite \textstyleAcronymallcaps{NP} introduced as a subject:

\ea%x1733
\label{ex:x1733}
\gll ...\textstyleEmphasizedVernacularWords{emer  en-ow  mua=ko}  eka  en-ep  momor-ar-ep  {\dots} \\
      \\
\glt
\z

sago  eat-NMZ  man=NF  water  eat-SS.SEQ  fool-INCH-SS.SEQ

`{\dots} a Sepik man had drunk beer and became drunk and {\dots}'

\ea%x1732
\label{ex:x1732}
\gll Nan  iimar-ep  ika-eya  \textstyleEmphasizedVernacularWords{urema=ko}  ekap-eya \\
      \\
\glt
\z

there  stand.up-SS.SEQ  be-2/3s.DS  bandicoot=NF  come-2/3s.DS

miim-a-k.

hear-PA-3s

`He was standing there and he heard a bandicoot coming.' (Lit: `{\dots}a bandicoot came and he heard it').

Any constituent in a clause can be marked as focused with the neutral focus clitic. The subject and object have been exemplified above, but a recipient (\stepcounter{nx}{\thenx}), adverbial (\stepcounter{nx}{\thenx}), comitative (\stepcounter{nx}{\thenx}), (\stepcounter{nx}{\thenx}) and instrument (\stepcounter{nx}{\thenx}) are also possible:

\ea%x1723
\label{ex:x1723}
\gll \textstyleEmphasizedVernacularWords{Mua}  \textstyleEmphasizedVernacularWords{nain=ko } onak-e. \\
      \\
\glt
\z

man  that1=NF  feed.him-IMP.1s

`Give it to that man to eat.'

\ea%x1724
\label{ex:x1724}
\gll Miiw-aasa  \textstyleEmphasizedVernacularWords{era=pa=ko}  me  yiar  samor-ar-e-k. \\
      \\
\glt
\z

land-canoe  road=LOC=NF  not  1p.DAT  bad-INCH-PA-3s

`Our truck did not break on the road.'

\ea%x1730
\label{ex:x1730}
\gll Ne  \textstyleEmphasizedVernacularWords{samor  akena=ko}  aruf-a-mik. \\
      \\
\glt
\z

ADD  badly  very=NF  hit-PA-1/3p

`And they beat him \textsc{very badly}.'

\ea%x1725
\label{ex:x1725}
\gll \textstyleEmphasizedVernacularWords{Ikos}\textstyleEmphasizedVernacularWords{=ko}  niir-u. \\
      \\
\glt
\z

together=NF  play-IMP.1d

`Let's play \textsc{together}.'

\ea%x1726
\label{ex:x1726}
\gll \textstyleEmphasizedVernacularWords{Fura}\textstyleEmphasizedVernacularWords{=iw=ko}  me  puuk-a-mik. \\
      \\
\glt
\z

knife=INST=NF  not  cut-PA-1/3p

`They didn't cut it \textsc{with a knife}\textsc{.}'

In a sentence the final, fully inflected verbs are already on the basis of their position more prominent than other verbs, and they cannot receive focus marking. But the medial verbs may be given extra prominence with the neutral focus clitic:

\ea%x1727
\label{ex:x1727}
\gll Amerika  kerer-e-mik  na-i-ya,  \textstyleEmphasizedVernacularWords{ikiw-ep=ko} \\
      \\
\glt
\z

America  appear-PA-1/3p  say-Np-PR.3s  go-SS.SEQ=NF

wia  uruf-ik-ua.

3p.ACC  see-be-PA.3s

`He says that the Americans have arrived, let's \textsc{go} and see them.'

Even the verbal negation particle \textstyleStyleVernacularWordsItalic{me} `not' can be marked with the \textstyleAcronymallcaps{NF} clitic, in which case the focus is on negating the whole proposition.

\ea%x1728
\label{ex:x1728}
\gll Takira  \textstyleEmphasizedVernacularWords{me=ko}  wia  aruf-a-mik. \\
      \\
\glt
\z

boy  not=NF  3p.ACC  hit-PA-1/3p

`It is \textsc{not} the case that we hit the boys.'

A clause can only have one contrastive focus,\footnote{The clauses with a locative adverb \textit{neeke} or \textit{feeke} (\sectref{sec:3.6.3}) are an exception.}  but negations and especially polite requests may contain two or even three constituents marked with the neutral focus:

\ea%x1729
\label{ex:x1729}
\gll \textstyleEmphasizedVernacularWords{No=ko  era=ko  imen-ap=ko}  yia  asip-e. \\
      \\
\glt
\z

2s.UNM=NF  way=NF  find-SS.SEQ=NF  1p.ACC  help-IMP.2s

`If you find a way, please help us.' Or: `Please find a way to help us.' 

Especially in spoken language, it is possible to reduplicate the \textstyleAcronymallcaps{NF} clitic in a word for extra prominence:

\ea%x1731
\label{ex:x1731}
\gll Wi  kema  ma-e-mik,  ``\textstyleEmphasizedVernacularWords{O=ko=ko}  amukar-ek-a-n   \\
      \\
\glt
\z

3p.UNM  liver  say-PA-1/3p  3s.UNM=NF=NF  scold-CNTF-PA-2s  

nom.  Moram  me  amukar-e-n?''

please  why  not  scold-PA-2s

 `They said in their hearts, ``C'mon, you should have scolded \textstyleEmphasizedWords{\textsc{him}}. Why didn't you scold him?'' '

The two focus markers are not mutually exclusive, and consequently they can co-occur in one clause:

\ea%x1737
\label{ex:x1737}
\gll \textstyleEmphasizedVernacularWords{Yos=ke  maa  nain=ko}  me  aaw-e-m. \\
      \\
\glt
\z

1s.FC=CF  thing  that1=NF  not  take-PA-1s

`It wasn't I who took that thing.'

The constituent with focus marking retains the same position in a clause that it has in a non-focused clause. When a personal pronoun in some other case than nominative receives neutral focus, an unmarked pronoun is added as a pronoun copy and marked with the \textstyleAcronymallcaps{NF} clitic:

\ea%x1743
\label{ex:x1743}
\gll Mua  nain  \textstyleEmphasizedVernacularWords{i=ko}  me  \textstyleEmphasizedVernacularWords{yia}  far-e-k. \\
      \\
\glt
\z

man  that1  1p.UNM=NF  not  1p.ACC  call-PA-3s

`The man didn't call \textsc{us}.'

When this pronoun gets also fronted as a theme, it is the pronoun copy that is fronted:

\ea%x1744
\label{ex:x1744}
\gll \textstyleEmphasizedVernacularWords{I}\textstyleEmphasizedVernacularWords{=ko}  mua  nain-(ke)  me  \textstyleEmphasizedVernacularWords{yia}  far-e-k. \\
      \\
\glt
\z

1p.UNM=NF  man  that1-(CF)  not  1p.ACC  call-PA-3s

`\textsc{Us}  the man didn't call.'

Exactly what kind of prominence the \textstyleAcronymallcaps{NF} clitic marks is difficult to pin down, and more research is needed on that. Of the following three examples, (\stepcounter{nx}{\thenx}) is a low-prominence clause with no item marked for extra prominence, in (\stepcounter{nx}{\thenx}) `I' is contrasted with other people, and in (\stepcounter{nx}{\thenx}) the prominence is neutral: the speaker emphasizes that (s)he didn't see, but there is no implied contrast. 

\ea%x1734
\label{ex:x1734}
\gll Yo  me  uruf-a-m. \\
      \\
\glt
\z

1s.UNM  not  see-PA-1s

`I didn't see it.'

\ea%x1735
\label{ex:x1735}
\gll \textstyleEmphasizedVernacularWords{Yos}\textstyleEmphasizedVernacularWords{=ke}  me  uruf-a-m. \\
      \\
\glt
\z

1s.FC=CF  not  see-PA-1s

`It wasn't I who saw it (but someone else).'

\ea%x1736
\label{ex:x1736}
\gll \textstyleEmphasizedVernacularWords{Yo}\textstyleEmphasizedVernacularWords{=ko}  me  uruf-a-m. \\
      \\
\glt
\z

1s.UNM=NF  not  see-PA-1s

`\textit{I}  didn't see it (regardless of whether anyone else did or not)'

Introduction of an indefinite topic has already been mentioned as one of the functions of neutral focus. In questions and requests the focus marking indicates politeness. And especially in many negated clauses with \textstyleAcronymallcaps{NF} marking there is a sense of distancing oneself from the situation. 

\ea%x1738
\label{ex:x1738}
\gll I  \textstyleEmphasizedVernacularWords{mua=ko}  me  wia  furew-a-mik  ne  yiena  pun  \\
      \\
\glt
\z

1s.UNM  man=NF  not  3p.ACC  sense-PA-1/3p  ADD  1p.GEN  also

\textstyleEmphasizedVernacularWords{mukuna}\textstyleEmphasizedVernacularWords{=ko}  me  op-a-mik.

fire=NF  not  hold-PA-1/3p

`We didn't sense any people (there) and we ourselves didn't carry fire either.'

\subsection{Other focusing devices}
\hypertarget{RefHeading23961935131865}{}
Cross-linguistically possibly the most common focusing device is stress. Stress in Mauwake is not only a word-level feature (\sectref{sec:2.1.3.1}); it can be employed to give prominence to a word or phrase in a clause. Default, or neutral, clause stress is on the verb or the non-verbal predicate. An extra heavy stress is used for contrastive focus especially for those constituents that seldom or never take \textstyleAcronymallcaps{CF} marking:  

\ea%x1739
\label{ex:x1739}
\gll \textstyleEmphasizedVernacularWords{O}\textstyleEmphasizedVernacularWords{{{\textprimstress}}}\textstyleEmphasizedVernacularWords{wowa=pa}  emeria  unowa  wia  maak-e-mik. \\
      \\
\glt
\z

village=LOC  woman  many  3p.ACC  tell-PA-1/3p

`\textsc{In the village} they told it to many women.'

\ea%x1740
\label{ex:x1740}
\gll Owowa=pa  \textstyleEmphasizedVernacularWords{e}\textstyleEmphasizedVernacularWords{{{\textprimstress}}}\textstyleEmphasizedVernacularWords{meria}  unowa  wia  maak-e-mik. \\
      \\
\glt
\z

village=LOC  woman  many  3p.ACC  tell-PA-1/3p

`In the village they told it to many \textstyleEmphasizedWords{\textsc{women}}.'

\ea%x1741
\label{ex:x1741}
\gll Owowa=pa  emeria  \textstyleEmphasizedVernacularWords{u}\textstyleEmphasizedVernacularWords{{{\textprimstress}}}\textstyleEmphasizedVernacularWords{nowa}  wia  maak-e-mik. \\
      \\
\glt
\z

village=LOC  woman  many  3p.ACC  tell-PA-1/3p

`In the village they told it to \textstyleEmphasizedWords{\textsc{many}} women.'

\ea%x1742
\label{ex:x1742}
\gll Owowa=pa  emeria  unowa  wia  \textstyleEmphasizedVernacularWords{{{\textprimstress}}}\textstyleEmphasizedVernacularWords{maak-e-mik}. \\
      \\
\glt
\z

village=LOC  woman  many  3p.ACC  tell-PA-1/3p

`In the village they \textstyleEmphasizedWords{\textsc{told}} it to many women (instead of hiding it from them).'

Note that in (\stepcounter{nx}{\thenx}) it is only the loudness/intensity in the stressed word that distinguishes it from the neutral clausal stress.

Right-dislocation is often called a topicalizing device, but in Mauwake it can't be that, since only a few of the right-dislocated constituents are topics (\stepcounter{nx}{\thenx}). 

\ea%x1745
\label{ex:x1745}
\gll Maa  nain  aaw-ep  iima=pa  wu-om-ap  \\
      \\
\glt
\z

thing  that1  take-SS.SEQ  chest=LOC  put-BEN-BNFY2.SS.SEQ

om-em-ik-ua,  \textstyleEmphasizedVernacularWords{sawur} \textstyleEmphasizedVernacularWords{} \textstyleEmphasizedVernacularWords{emeria} \textstyleEmphasizedVernacularWords{} \textstyleEmphasizedVernacularWords{nain}\textstyleEmphasizedVernacularWords{=ke}.

cry-SS.SIM-be-3s.PA  spirit  woman  that1=CF

`She took the thing and put it on his chest, the spirit woman (did).'

Most of the right-dislocated elements are not topics. Right-dislocation seems to be a focusing device of a special kind: the speaker decides that some constituent needs clarification or more prominence than it received, and adds it as an afterthought after the clause.

\ea%x1746
\label{ex:x1746}
\gll Saapara=pa  nan  suusa  iw-e-mik,  \textstyleEmphasizedVernacularWords{wiena  ifa} \\
      \\
\glt
\z

Saapara=LOC  there  needle  give.him-PA-1/3p  3p.GEN  snake

\textstyleEmphasizedVernacularWords{suusa  nain}.

needle  that1

`There in Saapara they gave him an injection, their snake (antivenene) injection.'

\ea%x1747
\label{ex:x1747}
\gll Aaya  puuk-ap  iimar-ep  ik-ua,  \\
      \\
\glt
\z

sugarcane  change.into-SS.SEQ  stand.up-SS.SEQ  be-PA.3s  

\textstyleEmphasizedVernacularWords{manin(a)} \textstyleEmphasizedVernacularWords{} \textstyleEmphasizedVernacularWords{afua=pa}.

garden  old(garden)=LOC

`It had changed into a sugarcane and was standing in the old garden.'

\ea%x1748
\label{ex:x1748}
\gll Ne  fraide-pa  maapora  puk-o-k,  \textstyleEmphasizedVernacularWords{urera}. \\
      \\
\glt
\z

ADD  Friday  party  burst-PA-3s  afternoon

`And on Friday the party started, in the afternoon.'


