
\chapter{Abbreviations in glosses\footnotemark{}}
\footnotetext{ The abbreviations are compatible with (i.e. are a superset of) the list of standard abbreviations included in the Leipzig Glossing Rules  (\url{http://www.eva.mpg.de/lingua/files/morpheme.html}).\par }0

\todo[inline]{change all this to smallcaps, and change all occurrences of these abbreviations  the main chapters to smallcaps as well}
1  first person  NEG  negation

2  second person  NOM  nominative

3  third person  OBL  oblique

ACC  accusative  PART  partitive (case)

ALL  allative (case)  PARTART  partitive article

AN  animate  PASS  passive

ANT  anterior  PL  plural

ART  article  POSS  possessive

CMPR  comparative  PP  perfect participle

CS  construct state  PDA  preproprial definite article

DAT  dative  PIA  postadjectival indefinite article

DEF  definite (article)  PRAG  pragmatic particle

DEM  demonstrative  PROG  progressive

DU  dual  PRS  present

F  feminine  PST  past

GEN  genitive  Q  question particle/marker

IMP  imperative  REFL  reflexive

INDF  indefinite (article)  REL  relative (pronoun)

INF  infinitive  SBJ  subject

INFM  infinitive marker  SG  singular

IPFV  imperfective  SUP  supine

M  masculine  SUPERL  superlative

N  neuter  WK  weak form of adjective


\chapter{Abbreviations for provinces and dialect areas}
\todo[inline]{Apparently this will be changed globally as the intended audience can not be assumed to be familiar with those places}
Be  Dalabergslagsmål  Os  Ovansiljan

Bl  Blekinge  Pm  \textit{Pitemål}

Bo  Bohuslän  SI  \textit{Särna-Idremål}

COb  Central Ostrobothnian  Sk  Skåne

Dl  Dalsland  Sm  Småland

Es  Estonian Swedish vernaculars   SOb  Southern Ostrobothnian

  including Gammalsvenskby,  SVb  Southern Westrobothnian

  (Ukraine)    (“sydvästerbottniska”)

Go  Gotland  Sö  Södermanland

Hd  Härjedalian  Up  Uppland

Hl  Halland  Vd  Västerdalarna

Hä  Helsingian (“hälsingska”)  Vg  Västergötland

Jm  Jamtska (“jämtska”)  Vl  Västmanland

Kx  \textit{Kalixmål}  Vm  Värmland

Ll  \textit{Lulemål}  Åb  Åbolandic

Md  Medelpadian (“medelpadska”)  Ål  Ålandic

Nm  Northern Settler dialect area  Åm  Angermannian

NOb  Northern Ostrobothnian    (“ångermanländska”)

Ns  Nedansiljan  ÅV  Angermannian-Westrobothnian

NVb  Northern Westrobothnian    transitional area

  (“nordvästerbottniska”)    (“övergångsmål”)

Ny  Nylandic  Ög  Östergötland

Nä  Närke  Öl  Öland

