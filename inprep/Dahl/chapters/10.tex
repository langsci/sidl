
\chapter{Common symbols in vernacular examples}
\label{bkm:Ref224104485}

â – a very fronted [a] or [æ]

(upper-case) Ö, ô, [275?], \textsc{8  }\textsc{–}\textsc{ }a central schwa-like vowel with somewhat varying quality

[277?] - [u]

(upper-case) L, [1E37?], [26D?], [19A?] , \textbf{l }– a voiced retroflex flap (according to Swedish terminology \textit{tonande kakuminal lateral} or in everyday language \textit{tjockt l} ‘thick \textit{l}’)

(upper-case) N – a retroflex \textit{n}

$\lambda $, hl – an unvoiced \textit{l } (usually historically derived from \textit{sl})

´  – marks an “acute” pitch accent (also referred to as “Accent 1”)

{\textasciigrave} – marks a “grave” pitch accent (also referred to as “Accent 2”)

Doubling of vowels (\textit{aa}) is often used to denote a “circumflex” accent, but in Finland Swedish vernaculars instead means that the vowel is long.
