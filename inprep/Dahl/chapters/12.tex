
\chapter[Abbreviations for provinces and dialect areas]{\rmfamily Abbreviations for provinces and dialect areas}

Be  Dalabergslagsmål  Os  Ovansiljan

Bl  Blekinge  Pm  \textit{Pitemål}

Bo  Bohuslän  SI  \textit{Särna-Idremål}

COb  Central Ostrobothnian  Sk  Skåne

Dl  Dalsland  Sm  Småland

Es  Estonian Swedish vernaculars   SOb  Southern Ostrobothnian

  including Gammalsvenskby,  SVb  Southern Westrobothnian

  (Ukraine)    (“sydvästerbottniska”)

Go  Gotland  Sö  Södermanland

Hd  Härjedalian  Up  Uppland

Hl  Halland  Vd  Västerdalarna

Hä  Helsingian (“hälsingska”)  Vg  Västergötland

Jm  Jamtska (“jämtska”)  Vl  Västmanland

Kx  \textit{Kalixmål}  Vm  Värmland

Ll  \textit{Lulemål}  Åb  Åbolandic

Md  Medelpadian (“medelpadska”)  Ål  Ålandic

Nm  Northern Settler dialect area  Åm  Angermannian

NOb  Northern Ostrobothnian    (“ångermanländska”)

Ns  Nedansiljan  ÅV  Angermannian-Westrobothnian

NVb  Northern Westrobothnian    transitional area

  (“nordvästerbottniska”)    (“övergångsmål”)

Ny  Nylandic  Ög  Östergötland

Nä  Närke  Öl  Öland
