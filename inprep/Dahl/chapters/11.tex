
\chapter[Abbreviations in glosses]{Abbreviations in glosses\footnotemark{}}
\footnotetext{ The abbreviations are compatible with (i.e. are a superset of) the list of standard abbreviations included in the Leipzig Glossing Rules  (\url{http://www.eva.mpg.de/lingua/files/morpheme.html}).\par }

1  first person  NEG  negation

2  second person  NOM  nominative

3  third person  OBL  oblique

ACC  accusative  PART  partitive (case)

ALL  allative (case)  PARTART  partitive article

AN  animate  PASS  passive

ANT  anterior  PL  plural

ART  article  POSS  possessive

CMPR  comparative  PP  perfect participle

CS  construct state  PDA  preproprial definite article

DAT  dative  PIA  postadjectival indefinite article

DEF  definite (article)  PRAG  pragmatic particle

DEM  demonstrative  PROG  progressive

DU  dual  PRS  present

F  feminine  PST  past

GEN  genitive  Q  question particle/marker

IMP  imperative  REFL  reflexive

INDF  indefinite (article)  REL  relative (pronoun)

INF  infinitive  SBJ  subject

INFM  infinitive marker  SG  singular

IPFV  imperfective  SUP  supine

M  masculine  SUPERL  superlative

N  neuter  WK  weak form of adjective
