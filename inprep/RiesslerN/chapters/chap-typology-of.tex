%\documentclass{LSP/langsci}
%
%\usepackage{localmetadata}
%\usepackage{localpackages}
%\usepackage{localhyphenation}
%\usepackage{localcommands}
%\usepackage{authorindex}
%\bibliography{localbibliography}
%
%\begin{document}

\chapter[Typology of attribution marking]{A typology of adjective attribution marking devices} \label{ontology}

In the present chapter, different types of adjective attribution marking devices attested in natural languages will be described and systematized with a special focus on their typologization according to the morphology of attributive adjectives.

The term \emph{adjective attribution marking} will be used to refer to a grammatical operation relating an adjectival modifier to its noun head. \emph{Attribution marking device} will be used to subsume both overt and covert grammatical operations which license the syntactic relation of attribution. 

\section[Juxtaposition]{Syntactic attribution marking: juxtaposition} \label{juxtaposition}

Juxtaposition can be defined as an unmarked sequence of phrase constituents in which one constituent is syntactically subordinated to the other. It  has to be distinguished from \emph{apposition}. The latter term is usually used to denote an appositive construction of two noun phrases, as in \textit{Alma, meine Tochter} ‘Alma, my daughter’ or \textit{Iva, die jüngere Tochter} ‘Iva, the younger daughter’ where neither constituent is syntactically subordinated. See also the short discussion in Section \ref{apposition}. Juxtaposition is thus characterized by adjacency of noun phrase constituents alone. There is no construction marker present. Consider the following Komi-Zyrian examples where neither agreement markers or any other additional morphemes are present. The attributive adjective in (\ref{komi juxtap}) is represented by its pure stem form. It does not inflect for any of the categories marked on the head noun.\footnote{Beside \textsc{number}, these categories include \textsc{case} and \textsc{possession} in Komi-Zyrian.}
%%%
\ea
\langinfo{Komi-Zyrian}{Uralic}{\citep{lytkin1966a}}\\
\ea
\gll 	bur 	mort\\
		good	person\\
\glt		‘good person’
\ex
\gll 	bur	mort-jas\\
		good	person-\textsc{pl}\\
\glt		‘good people’
\z
\z
%%%
Juxtaposition constitutes a very widespread attribution marking device cross-linguistically. Among the northern Eurasian languages, juxtaposition occurs as the default attribution marking device in several families, among others in Mongolic, Turkic and Uralic. Whereas juxtaposition constitutes the default type even in the proto-stages in these language groups, the occurrence of juxtaposition in several other languages results from a relatively recent linguistic change in which the original agreement marking on adjectives was lost.

Defining juxtaposition as a “device” for marking attribution might, however, be questionable. Given the definition that attribution is licensed by the sequence of constituents alone, i.e.~that an adnominal modifier and a head noun occur next to each other in the syntactic structure, juxtaposition resembles a “non-marking” rather than a marking device. In English, for instance, one could also argue that the non-occurrence of the copula \textit{is\fshyp{}are} is relevant to the marking of attribution. In order to use an adjective as predicate in English (\textit{the man \textbf{is} good, the men \textbf{are} good}), the copula is obligatory. However, word order may be relevant, too. In English, again, juxtaposed attributive adjectives precede the noun as a rule, whereas predicative adjectives follow it. 

Word order can in fact be crucial in languages were both adjective attribution and predication are marked simply through adjacency of noun and adjective but with reversed word order, as for example, in Ainu or Kalmyk. 
%%%
\ea
\langinfo{Ainu (Shizunai)}{isolate}{\citealt{refsing1986}}\\
\ea {\rm Attribution: adjective-noun order}\\
\gll	pirka cep\\
	be\_good fish\\
\glt	‘a fine fish’
\ex {\rm Predication: noun-adjective order}\\
\gll	cep pirka\\
	fish be\_good\\
\glt	‘the fish is fine’
\z
\z
%%%
\ea
\langinfo{Kalmyk}{Mongolic}{\citealt{jachontova1997}}\\
\ea {\rm Attribution: adjective-noun order}\\
\gll	čyɣan časun\\
	white snow\\
\glt	‘white snow’
\ex {\rm Predication: noun-adjective order}\\
\gll	časun čyɣan\\
	snow white\\
\glt	‘the snow is white’
\z
\z
%%%
The only difference between attribution and predication of adjectives in Ainu\footnote{Note that there are no true adjectives in Ainu. Property words are stative verbs in this language (see also section \ref{ainu synchr}).} and Kalmyk is in word order. 
		
\section[Incorporation]{Covert morpho-syntactic construct marking:\\adjective incorporation} \label{attr incorporation}

Similarly to juxtaposition, \emph{adjective incorporation} is characterized by adjacency of phrase constituents. There is no additional morpheme present in this type of noun phrase either. The syntactic relation of attribution is, however, marked by a syntactic composition of modifier and head noun. This type can thus be characterized as covertly marked operation.
%%%
\ea  \label{bondska compound}
\langinfo{Västerbotten-Swedish}{Indoeuropean}{\citealt{larsson1929}}\\
\ea
\gll 	stor-båt-en\\	
	big-boat-\textsc{def:m.sg}\\
\glt	‘the big boat’
\ex
\gll 	stor-hus-et\\
	big-house-\textsc{def:n.sg}\\
\glt	‘the big house’
\z
\z
Since adjective incorporation in northern North-Germanic dialects is syntactically and semantically distinguishable from derivational compounding it is often referred to as \emph{Adjective-Noun-Incorporation} (for instance by \citealt{sandstrom-etal2003}; \citealt[119–124]{dahl2007} or \citealt[61]{julien2005}).

\paragraph{Phonological vs. syntactic compounds} In Västerbotten-Swedish (as well as in other North-Germanic varieties where adjective-noun compounds occur), accent patterns clearly indicate that adjectives are morpho-phonologically compounded \citep[cf.][]{dahl2003}. Non-compounded monosyllabic stems, such as \textit{tré}, ‘tree’, \textit{bǻt} ‘boat’, \textit{bǻt-er} ‘boats’, \textit{bǻt-er-na} ‘the boats’, have an acute accent (marked with ´ in the examples) as a rule and whether or not they are equipped with inflectional affixes. Bisylabic stems, including compounds, by contrast have pitch accent on the stem (marked with ´ ` in the examples). Compare \textit{tré-bå̀t-en} ‘the wooden boat’ or \textit{stór-bå̀t-en} with the noun phrase \textit{bǻt-en mín} ‘my boat’, where both the noun and the (non-compounded) possessive pronoun have acute accent.

Phonological composition, however, cannot be sufficient evidence for syntactic compounding (i.e.~incorporation). Phrase internal phonological or prosodic processes at the juncture of adjectives and nouns (as, e.g., the accent pattern described above) seem to be very common in languages. Such processes can perhaps prove morpho-phonological composition. For the present typology, however, adjective incorporation is defined purely syntactically as a noun phrase where the attributive adjective occurs obligatorily as a (syntactically) bound morpheme. To prove syntactic boundedness one has to show that the adjective cannot occur unbound. In Västerbotten-Swedish (and other North-Swedish dialects), for instance, the adjective stem cannot occur unbound unless alternative morpho-syntactic marking is applied. Using the adjective ‘big’ in Västerbotten-Swedish in a headless noun phrase results in a construction in which the adjective is marked for agreement and is obligatorily followed by en article serving as a dummy head.\footnote{This is true, however, only with the indefinite adjective. The definite adjective, by contrast, does not need a dummy head but is unbound (and equipped with the definite marker): {\it	stor-en} [big(\textsc{m}) \textsc{def:m.sg}] ‘the big one (masculine)’, \textit{stor-et} [big(\textsc{m}) \textsc{def:n.sg}] ‘the big one (neuter)’.}
%%%
\ea \label{bondska headless}
\langinfo{Västerbotten-Swedish}{Indoeuropean}{\citealt{larsson1929}}\\
\ea
\gll 	en stor en\\	
	\textsc{indef:m} big(\textsc{m}) \textsc{art:indef:m.sg}\\
\ex
\gll 	ett stor-t ett\\	
	\textsc{indef:n} big:\textsc{n} \textsc{art:indef:n.sg}\\
\glt	‘a big one’
\z
\z
%%%
If evidence for syntactic incorporation cannot be found compounded adjectives can only by described as a special case of juxtaposition. But interestingly, if the described test of syntactic boundedness is applied, then English falls in the category of incorporating languages as a result. In English too, attributive adjectives can only occur bound to a head. This head is either lexical or, similar to Västerbotten-Swedish indefinite noun phrases, an obligatory article as dummy head.\footnote{Applying the same test, it turnes out that English incorporates even other modifiers of nouns, such as possessive pronouns: \textit{give me her book} – \textit{give me her-s}.}

Whether or not English is coded as an incorporating language, adjective incorporation seems to constitute a minor type of attribution marking. Among languages of the northern Eurasian area, however, this type is attested in geographically quite distinct languages: besides the peripheral North-Germanic dialects, it is also found in Adyge and in Chukchi, Kamchatkan and in Eskimo-Aleut languages (see the respective sections of Part \ref{part synchr}).

\section[Agreement]{Morpho-semantico-syntactic attribution\\marking: agreement}

\emph{Agreement} (aka \emph{concord}) is a common type of overt attribution marking device. Agreement is commonly understood as a systematic covariance between a semantic or formal property of one element and a formal property of another \cite[610]{steele1978}. In other words, agreement can be defined as the spread of semantic or morphological properties across constituents of a syntactic phrase. The agreement properties (or \emph{agreement features}) spread from “trigger constituents”\footnote{In other terms, the trigger of agreement can be called \emph{controller}, cf.~\citealt{corbett2006}.} and are formally, i.e.~morphologically expressed on “target constituents”.

The primary syntactic function of agreement is to relate phrase constituents to each other. Agreement thus serves the formal licensing of dependency in the given phrase. As compared to construct marking, however, the licensing of dependency by means of agreement is more the indirect result of morphological copying of agreement features across phrase constituents.

In principle, agreement features can be triggered by both syntactic heads and syntactic dependents, as will be shown in the following sections. Based on where the agreement features originate, the terms \emph{head-driven} and \emph{dependent-driven agreement}, first proposed by Balthasar Bickel and Johanna Nichols in 2001 \citep[published as][]{bickel-etal2007}, will be used in the following.

\subsection{Head-driven agreement} \label{head-driven agreement}

Typical morpho-syntactic agreement features triggered by syntactic heads are \textsc{gender, number} and \textsc{case}, as in Lower Sorbian.
%%%
\ea \label{sorbian agr}
\langinfo{Lower Sorbian}{Indoeuropean}{\citealt{janas1976}}\\
\ea
\gll	dobr-y cłowjek\\
	good-\textsc{sg:m} person(\textsc{m})\\
\glt	‘a good person’
\ex
\gll	dobr-e cłowjek-y\\
	good-\textsc{pl} person-\textsc{pl}\\
\glt	‘good people’
\ex \label{ap case gov}
\gll	k dobr-emu cłowjek-oju\\
	to good-\textsc{sg:m:dat} person-\textsc{sg:m:dat}\\
\glt	‘to a good person’
\z
\z 
%%%
Note, however, that Kibort (\citeyear{kibort2008a}; following \citealt[133–135]{corbett2006}) does not list \textsc{case} as a prototypical agreement feature. In Kibort's and Corbett's view, the matching of a case value on the noun phrase head and its adjectival (or other) modifier(s) does not count as “canonical agreement” but is simultaneously imposed on the noun phrase constituents as the result of government by a syntactic element outside the noun phrase. Consider the Lower Sorbian example (\ref{ap case gov}) in which both the adjective ‘good’ and the noun ‘person’ are marked with the dative case suffix.

The question is whether the case value in such examples is imposed on both noun phrase constituents through government (in example \ref{ap case gov} by the preposition \textit{k} ‘to’) as argued by Kibort and Corbett, or if the dative case on the modifying adjective is imposed by its head by means of agreement, similar to gender and number agreement which are also imposed by the head noun. Adopting Mel'čuk's (\citeyear[329, 337]{melcuk1993}) dependency view of syntax instead of Corbett's (\citeyear[133]{corbett2006}) “constituency”, the dependent constituent in the adposition phrase is a noun phrase. The dependent constituent in the noun phrase, again, is an adjective phrase (i.e.~the attributive adjective) which depends on the noun head of the phrase and inherits its case marking. In this view, the morpho-syntactic mechanisms of assigning a head's morphological features to dependent constituents are similar for case and other agreement categories (like gender and number). Consider (\ref{ap case gov}) ‘to a good person’ in Lower Sorbian.
%%%
\ea
\langinfo{Lower Sorbian}{Indoeuropean}{\citealt{janas1976}}\\
$[$ $_{AdP}$ k $[$ $_{NP}$ dobremu$_{agr}$ cłowjekoju$_{gender:number:case}$ $]$ $]$
\z
%%%
Another possible agreement feature beside \textsc{gender, number} and \textsc{case} is \textsc{species}, typical values of which are \textsc{definite} and \textsc{indefinite}. Consider, for instance, the agreement paradigm of adjectives in Icelandic (Table \ref{icelandic agr}) in which indefinite and definite forms are distinguished.

Cross-linguistically, head-driven agreement seems to be a wide-spread attribution marking device across the world's language families. The actual morphological appearance of agreement marking, however, is highly diverse across languages and depends on several parameters.

One such parameter concerns the form of the agreement marking morphemes in comparison to the morphemes marking the respective values on the head noun. In fact, adjective agreement paradigms in many languages are different from the respective inflectional paradigms of nouns. This is true, for instance, for Slavic and Germanic languages, as mentioned, but also for other Indoeuropean languages. 
%%%
\begin{table}
  \begin{tabular}{l l | l l l | l l l}
    \lsptoprule
		&		&\textsc{m.sg}&\textsc{f.sg}&\textsc{n.sg}&\textsc{m.pl}&\textsc{f.pl}&\textsc{n.pl}\\
		\midrule
		&\textsc{nom}	&–ur		&–Ø	&–t		&–ir		&–ar		&–Ø \\
\textsc{indef}	&\textsc{acc}	&–an		&–a	&–t		&–a		&–ar		&–Ø \\
		&\textsc{dat}	&–um		&–ri	&–u		& –um&–um&–um\\
		&\textsc{gen}	&–s		&–rar	&–s		& –ra&–ra&–ra\\
		\midrule
		&\textsc{nom}	&–i		&–a		&–a		&\multicolumn{3}{c}{–u}\\
\textsc{def}	&\textsc{acc}	&–a		&–u		&–a		&\multicolumn{3}{c}{–u}\\
		&\textsc{dat}	&–a		&–u		&–a		&\multicolumn{3}{c}{–u}\\
		&\textsc{gen}	&–a		&–u		&–a		&\multicolumn{3}{c}{–u}\\
\lspbottomrule
\end{tabular}
\caption[Adjective paradigm for \textsc{Icelandic}]{Adjective declension paradigm for \langinfo{Icelandic}{Indoeuropean}{\citealt{kress1982}}} \label{icelandic agr}
\end{table}
%%%
In other languages, however, inflectional suffixes might simply reoccur on the modifier, as in Finnish.
%%%
\begin{exe}
\ex 
\langinfo{Finnish}{Uralic}{own knowledge} \label{finnish agr.}
\begin{xlist}
\ex
\gll 	iso-t		talo-t\\
	large-\textsc{pl}	house-\textsc{pl}\\
\glt	‘large houses’
\ex
\gll 	iso-i-ssa		talo-i-ssa\\
	large-\textsc{pl}-\textsc{iness} house-\textsc{pl}-\textsc{iness}\\
\glt	‘in large houses’
\end{xlist}
\end{exe}
%%%
Adjectives and nouns in Finnish (and in most other Uralic languages) differ in syntactic function rather than in morphological properties. Consequently, adjectives and nouns in Finnish exhibit similar inflectional paradigms. Probably, such a weak distinction between adjectival and nominal inflections was also true for proto-Indoeuropean (cf.~\citealt[80]{comrie1998}, \citealt[139]{kuriaki2007}). But the declensions of both adjectives and nouns in Indoeuropean languages have undergone radical changes and have become clearly distinct from each other. This is evident, e.g., in the Lower Sorbian example (\ref{sorbian agr}) on page \pageref{sorbian agr} where the adjective suffix \textit{-emu} and the noun suffix \textit{-oju} both mark the dative masculine singular.

Head-driven agreement marking also deviates across languages in respect to the inventory of morphological categories involved. Many languages exhibit head-driven agreement paradigms which exclude certain inherent or assigned morphological categories of the head noun, as in Finnish, where nouns inflect for \textsc{number}, \textsc{case} and \textsc{possession}. The latter feature, however, never spreads through the noun phrase.
%%%
\begin{exe}
\ex
\langinfo{Finnish}{Uralic}{own knowledge}
\begin{xlist}
\ex
\gll 	iso		talo-ni\\
	large	house-\textsc{poss:1sg}\\
\glt	‘my large house’
\ex
\gll 	*iso-ni	talo-ni\\
	large-\textsc{poss:1sg}	house-\textsc{poss:1sg}\\
\end{xlist}
\end{exe}
%%%
Finally, agreement paradigms can be “defect” in the sense that certain agreement categories do not show up on all members of the paradigm. In Danish, for example, gender as an agreement feature is marked on the attributive adjective only in indefinite noun phrases. In noun phrases marked for definite species, the attributive adjective is marked with an invariable definite agreement suffix. Consider examples (\ref{danish agr ex}) and Table \ref{danish agr paradigm} with the respective paradigm in Section \ref{n-germanic synchr}.
%%%
\begin{exe}
\ex 
\langinfo{Danish}{Indoeuropean}{own knowledge} \label{danish agr ex}
\begin{xlist}
\ex
\gll en \textbf{stor} mand\\
	\textsc{indef.com} big.\textsc{utr} man(\textsc{utr})\\
\glt	‘a tall man’
\ex
\gll ett \textbf{stor-t} hus\\
	\textsc{indef.n} big-\textsc{n} house(\textsc{n})\\
\glt	‘a large house’
\ex	
\gll den \textbf{stor-e} mand\\
	\textsc{def.com} big-\textsc{def} man(\textsc{utr})\\
\glt	‘the tall man’
\ex
\gll det \textbf{stor-e} hus\\
	\textsc{def.n} big-\textsc{def} house(\textsc{n})\\
\glt	‘the large house’
\end{xlist}
\end{exe}
%%%
An extreme case of a defective agreement paradigm is found in Chechen where adjectives only partially agree with the head noun and show only one single case distinction between nominative versus all other cases, as in the (incomplete) paradigm (\ref{chechen-defective}).\footnote{A similar defective agreement paradigm with only one case distinction is found in Ingush, cf.~Section \ref{ingush synchr}.}
%%%
\ea \label{chechen-defective}
\langinfo{Chechen}{Nakh-Daghestanian}{\citealt[29]{nichols1994a}}
\ea dika\textsuperscript{n} stag\textsuperscript{3} {\rm ‘good person’}	\jambox{\rm \textsc{nom:sg}}
\ex dikaču stega\textsuperscript{n} 							\jambox{\rm \textsc{gen:sg}}
\ex dikaču stagana 										\jambox{\rm \textsc{dat:sg}}
\ex dikaču staga 										\jambox{\rm \textsc{erg:sg}}
\ex dikaču stagie										\jambox{\rm \textsc{all:sg}}
\ex dika\textsuperscript{n} na:x								\jambox{\rm \textsc{nom:pl}}
\ex dikaču ne:xa\textsuperscript{n}							\jambox{\rm \textsc{gen:pl}}
\zl

\subsection{Dependent-driven agreement}

In many languages spoken inside and outside the northern Eurasian area, head-driven agreement is attested as a device for licensing attributive modification. The reverse agreement type, \emph{dependent-driven agreement}, is also wide-spread among the world's languages. Among the languages of my sample, however, dependent-driven agreement marking is attested only as a device for the licensing of (possessor) noun attributes. An example of a language with dependent-driven agreement marking in possessive noun phrases is Oroch.
%%%
\begin{exe}
\ex
\langinfo{Oroch}{Tungusic}{\citealt[3]{malchukov2000}} \label{oroch dep-driven agr.}
\gll 	nia	d'uu-ni\\
	man	house-\textsc{poss:3sg}\\
\glt	‘a man's house’
\end{exe}
%%%
The possessed noun ‘house’ in example (\ref{oroch dep-driven agr.}) obligatorily agrees with the \textsc{3sg} possessor ‘man’. This type of dependent-driven agreement is usually called \emph{possessor agreement}.\footnote{Another commonly used term is \emph{cross reference marking}.}

\subsubsection{Modifier-headed possessor agreement} \label{ModheadAgr}

The term \emph{modifier-headed possessor agreement} is derived from \emph{modifier-headed agreement} introduced in \citet{AUTOTYP-NP}. It is a subtype of dependent-driven agreement characterized by reverse semantic and syntactic dependency relations between attribute and head. 

Structurally similar to example (\ref{oroch dep-driven agr.}), Oroch also exhibits dependent-driven agreement marking by means of possessive affixes on attributive adjectives.

\begin{exe}
\ex
\langinfo{Oroch}{Tungusic}{\citealt[3]{malchukov2000}} \label{oroch mod-headed agr}
\begin{xlist}
\ex
\gll 	nia	aja-ni\\
	man	good-\textsc{poss:3sg}\\
\glt	‘a GOOD man’
\ex 
\gll nia-sa aja-ti\\	
	man-\textsc{pl} good-\textsc{poss:3pl}\\
\glt	‘GOOD men’
\end{xlist}
\end{exe}
%%%
In the Oroch example, the semantic head of the noun phrase ‘man’ is syntactically “degraded” to the (dependent) possessor function, and the semantic dependent is “upgraded” to the function of the syntactic head of the phrase, i.e.~the possessed. According to \citet[3]{malchukov2000}, the expression still has an attributive reading: ‘a man, a property of whom is “to be good”’, rather than a possessive one: *“a man's goodness”. Thus, the semantic attribute is rendered as the head (i.e.~the possessed) and the semantic head of the possessive noun phrase takes the slot of the dependent (i.e.~the possessor).

Whereas modifier-headed possessive agreement constitutes a marked structure in Oroch, it can be the universal type of attributive marking on adjectives in other languages. This kind of adjective attribution marking device is not very common in the northern Eurasian area under investigation, but it is pervasive, for instance, in Oceanic languages (cf.~\citealt{ross1998}). In Saliba, for example, attributive adjectives as a rule are marked by means of 3\textsuperscript{rd} person possessive suffixes.
%%
\begin{exe}
\ex
\langinfo{Saliba}{Austronesian}{\citealt{mosel1994}} \label{saliba poss-agr}
\begin{xlist}
\ex
\gll 	sine natu-na\\
 	woman child-\textsc{poss:3sg}\\
\glt ‘a woman's child / the child of the woman’
\ex
\gll 	sine-o natu-di\\
	woman-\textsc{pl} child-\textsc{poss:3pl}\\
\glt	‘women's children / the children of the women’%(check this example: child or children)
\end{xlist}
\end{exe}
%%%
In Saliba, possessor nouns are licensed as modifiers in a noun phrase by means of (dependent-driven) possessor agreement on the head noun. Similar to the marked noun phrase in Oroch (\ref{oroch mod-headed agr}), attributive adjectives are marked by means of modifier-headed possessor agreement.
%%%
\begin{exe}
\ex
\langinfo{Saliba}{Austronesian}{\citealt{mosel1994}} \label{saliba mod-headed agr}
\begin{xlist}
\ex
\gll 	mwaedo gagili-na\\
 eel small-\textsc{poss:3sg}\\
\glt ‘a small eel’
\ex
\gll 	mwaedo gagili-di\\
	eel small-\textsc{poss:3pl}\\
\glt ‘small eels’
\end{xlist}
\end{exe}
%%%
The adjectival attribute ‘small’ in example (\ref{saliba mod-headed agr}) occurs in a possessive-like construction (similar to \ref{saliba poss-agr}) where the adjective takes the slot of the possessed and is subsequently marked with a possessive agreement suffix.\footnote{An alternative account of noun phrase structure in Saliba could claim that the verbal attribute is marked by head-driven agreement, analyzing the suffixes \textit{-na} and \textit{-di} as singular and plural markers, respectively. This analysis is obviously underlying the descriptions of Saliba (e.g.~\citealt{mosel1994}, \citealt{margetts1999}), which leave the homophony of \textit{-na} \textsc{poss:3sg} and \textit{-di} \textsc{poss:3pl} with \textit{-na} \textsc{sg} and \textit{-di} \textsc{pl} undiscussed .} %?? [ON BASIS OF WHAT] I would rather make the claim, that attributive adjectives in Saliba occur in headstand noun phrases and are marked by means of modifier-headed possessor agreement.
Unlike in Oroch, however, modifier-headed possessor agreement is the default type of attributive connection of adjectives in Saliba.

\section[Construct marking]{Overt morpho-syntactic construct marking:\\attributive state marking}

Due to a lack of better terminology the feature \textsc{state} was earlier defined as assigned through \emph{syntactic government} (in Section \ref{crit eval}). Unlike the common notion of \emph{government}, which requires a trigger inside the phrase, true syntactic government considered in this study has no other trigger than the syntactic construction as such.

In order to avoid the misleading term \emph{government}, all overtly marked attribution devices with the exclusive function of licensing the syntactic relation between constituents of a noun phrase are defined here as \emph{attributive state marking}. “Overtly marked” means that (at least one) additional attribution marking morpheme is present in the noun phrase.

\emph{Attributive state} is adopted from “Construct state” or “Status constructus” which are commonly used in syntactic descriptions of languages exhibiting head-marking {state} (e.g.~Persian). Since construct state marking morphemes may occur on different loci inside the noun phrase, \emph{attributive state} will be used as superordinate term, subsuming the subtypes with the following loci of their respective attributive markers:\footnote{Other logically possible loci of attributive state markers would result from simultaneous marking on head- and\fshyp{}or on dependent+floating. I am, however, not aware of any language exhibiting such noun phrase types.}
%%%
\begin{itemize}
\settowidth\jamwidth{(floating construct)}
\item on-head \jambox{(construct)}
\item on-dependent \jambox{(anti-construct)}
\item neither on-head nor on-dependent \jambox{(floating construct)}
\item simultaneously on-head and on-dependent \jambox{(double construct)}
\end{itemize}
%%%
Among the northern Eurasian languages considered in the present study, only the first two types of attributive state marking, i.e.~head-marking state and dependent marking state, are attested as devices for licensing attributive adjectives. These two types are dealt with in more detail below in Sections \ref{head-marking state} and \ref{dep-marking state}.

\subsection{Head-marking attributive state } \label{head-marking state}

The attributive construction in Persian, commonly known as \emph{Ezafe} (or \emph{Izafe}), illustrates a typical case of head-marked attributive state.
%%%
\begin{exe}
\ex \label{persian constr state}
\langinfo{Persian}{Indoeuropean}{\citealt{mahootian1997}}\\
\gll xane-ye bozorg\\
	house-\textsc{attr} big\\
\glt 	‘a large house’
\end{exe}
%%%
The only function of the attributive suffix \textit{-(y)e}\footnote{The allomorph \textit{-e} appears ofter consonants.} on the noun ‘house’ is to show that “I am a noun phrase and I have a dependent.”\footnote{The attributive construct state marking in Persian is polyfunctional in the sense that its function is not restricted to the licensing of adjectives as modifier in a noun phrase, but also of noun attributes, adpositional phrases and infinitives.} The traditional term for the morphological value given by the head-marking attribution device in Persian is \emph{construct state} (or \emph{status constructus}). What is meant hereby is that the noun displays different “states” depending on the presence of a modifier in the noun phrase.

Obligatory attribution marking by means of an Ezafe-construction is also characteristic for other Iranian languages. In Kurmanji, a variety of Kurdish spoken in the northern Eurasian area, the Ezafe-formative is not an invariable suffix – unlike the cognate suffix \textit{-(y)e} in Persian – but also indicates morphological values of \textsc{number} (\textsc{sg\fshyp{}pl}), \textsc{gender} (\textsc{m\fshyp{}f}) and \textsc{species} (\textsc{def\fshyp{}indef}). Consider example (\ref{ez kirmanji ex}) and the paradigm in Table \ref{ez kirmanji paradigm}. 
%%%
\begin{exe}
\ex
\langinfo{Kirmanji}{Indoeuropean}{\citealt{ortmann2002b}} \label{ez kirmanji ex}%??CHECK source
\begin{xlist}
\ex
\gll	kur-\^e mezin\\
	boy-\textsc{attr:def.m.sg} big\\
\glt	‘the tall boy’
\ex	
\gll	ke\c{c}-a ba\c{c}\\
	girl-\textsc{attr:def.f.sg} nice\\
\glt	‘the nice girl’
\ex	
\gll	kur-\^en / ke\c{c}-\^en ba\c{c}\\
	boy-\textsc{attr:def.pl} {} girl-\textsc{attr:def.pl} nice\\
\glt	‘the nice boys / girls’ %\cite{ortmann}%glossen
\end{xlist}
\end{exe}
%%%
\begin{table}
\begin{tabular}{l | ccc}
\lsptoprule
		&\textsc{m.sg}	&\textsc{f.sg}		&\textsc{pl}\\
\midrule
\textsc{def}	&-(y)\^{e}	&-(y)a			&-(y)\^{e}n\\
\textsc{indef}	&-î		&-e				&\\
\lspbottomrule
\end{tabular}
\caption[Paradigm of the Ezafe in \textsc{Kurmanji}]{Paradigm of the Ezafe in \textsc{Kurmanji} \citep{schroder2002}} 
\label{ez kirmanji paradigm}
\end{table}
%%%
Note that the values of true morphological features (\textsc{number, gender, species}) of the noun are combined with the morpho-syntactic feature \textsc{attributive} in the differentiated forms of the Ezafe in Kirmanji. But agreement is not involved here because gender, number and species marking is not triggered within the noun phrase but is inherited to the head noun morpho-semantically.

\subsection{Dependent marking attributive state}\label{dep-marking state}

\subsubsection{Anti-construct state} 
In some languages there is an attributive construction corresponding to the Iranian Ezafe, which however does not mark the head but the adjectival dependent for “state” (i.e., indicating the availability of a head in the present noun phrase). This type of marking occurs, for instance in Saamic languages.
%%%
\begin{exe}
\ex
\langinfo{Kildin Saami}{Uralic}{own knowledge}
\begin{xlist}
\ex Predicative state \label{kildin pred.adj.}
\gll Tedt 	pērrht l{ī} ēll.\\
	\textsc{dem} house \textsc{cop} high\\
\glt	‘This house is high.’
\ex Attributive state
\begin{xlist}
\ex	\label{kildin attr.adj.sg}
\gll Tedt	l{ī} 	ēl'l'\textbf{-es'} 		pērrht.\\
	\textsc{dem} \textsc{cop}	high-\textsc{attr}	house\\
\glt	‘This is a high house.’
\ex	\label{kildin attr.adj.pl}
\gll Tegk 	liev 	ēl'l'\textbf{-es'}		pērht.\\	
	\textsc{dem}	\textsc{cop}	high-\textsc{attr} house$\backslash$\textsc{pl}\\
\glt	‘These are high houses.’
\end{xlist}
\end{xlist}
\end{exe}
%%%
Whereas the predicatively used adjective ‘high’ is represented by its pure stem form (\ref{kildin pred.adj.}), it is marked with the attributive suffix \textit{-es'} if used as modifier (\ref{kildin attr.adj.sg}+\ref{kildin attr.adj.pl}). Attributive marking on adjectives in Kildin and other Saamic languages is highly irregular due to the strong tendency to merge predicative and attributive adjective forms. Other adjective marking devices also occur. The default type in most Saamic languages, however, is that attributive adjectives exhibit an attributive inflection (\citealt{riesler2006b}; see also below section \ref{saami synchr}).

The attribution marker in Saamic is invariable, i.e.~the adjective does not show agreement with its head noun. The host of the Saamic attributive suffix is the adjective. Its only function is to specify the syntactic relation between head noun and adjectival modifier (“my host is dependent in the present syntactic structure”). Since the construction in Saamic constitutes dependent marking opposite to the Persian construct state, it can be labeled \emph{anti-construct}.\footnote{The term was introduced during Bickel's and Nichols' earlier work on the AUTOTYP Noun Phrase Structure Database, cf.~\citet[2, elsewhere]{bickel-etal2002}, \citet{AUTOTYP-NP}.} 

Anti-construct state marking seems not uncommon cross-linguistically, even if Saamic and the Iranian language Northern Talysh (cf.~Section \ref{talysh synchr}) provide the only examples of European languages with anti-construct state marking on adjectives. Note that typological descriptions and grammars use quite different terms for anti-construct state markers, such as \emph{attributive particle}, \emph{relator}, \emph{associative marker}, \emph{linker}, etc. If anti-construct marks the attribution of possessor nouns (besides adjectives) it is also often called \emph{attributive case} or \emph{genitive}.

\paragraph{Possessive case marking} 
From a purely syntactic point of view, possessive case marking is similar to anti-construct state marking. Both are syntactically governed dependent marking devices. In fact, anti-construct state marking of adjectives is sometimes described as “genitive” if the device is polyfunctional and marks possessor nouns as well.\footnote{Even other construct marking devices, such as the linker in Tagalog (\ref{tagalog linker}) or the construct state marker in Persian (\ref{persian constr state}) are often described as “genitives” because they mark possession. Unlike prototypical genitives, however, the construct markers in Tagalog and Persian do not constitute dependent marking devices.} Rather than extending the terminological domain of possessive case marking to adnominal modifiers beyond noun possessors, the term \emph{possessive case} (or \emph{possessor case}) will be used here only for describing a special subtype of anti-construct state. Whereas the latter is a purely morpho-syntactic device, possessive case additionally specifies a semantic relation (i.e.~possession).

\subsubsection{Anti-construct state agreement marking} \label{anti-constr agr}
Construct state markers such as the linker in Tagalog, the head-marking construct state marker \textit{-(y)e} in Persian, or the dependent marking anti-construct state marker \textit{-es'} in Kildin Saami are proper construct state markers in the sense that they are exclusively used as licensee of an attributive syntactic relation between modifying and modified constituents in the noun phrase. The respective formatives thus have morphologically unalterable shapes.

In other languages, however, certain adnominal modifiers marked for anti-construct state may additionally be the target of either head- or dependent-driven agreement. Such combined agreement and construct marking devices should consequently be characterized as simultaneously marking the syntactic and the morphological relation between the noun modifier and the modified noun. 

This subtype of anti-construct state marking, characterized by (adjectival or other) adnominal modifiers being marked simultaneously for anti-construct state and for head-driven agreement, will be labelled \emph{anti-construct state agreement marking} in the following.\footnote{The extended label \emph{\textbf{head-driven} anti-construct state agreement marking} seems obsolete because the agreement is self-evidently triggered by the head noun in this type.} %Aber warum zähle ich es zu Konstruktionsmarkierung und nicht als Untertyp von Kongruenz?

A typical example of a language with anti-construct state agreement marking is Russian.
%%
\begin{exe}
\ex
\langinfo{Russian}{Indoeuropean}{own knowledge} \label{ru-anti}
\begin{xlist}
\ex	{\rm Anti-construct state agreement}
\begin{xlist}
\ex
\gll 	vysok\textbf{-ij} 			dom\\
	high-\textsc{attr:m.nom} house(\textsc{m})\\
\glt	 ‘(a/the) high house’
\ex
\gll 	vysok\textbf{-aja} 			bašn'a\\
	high-\textsc{attr:f.nom} tower(\textsc{f})\\
\glt	 ‘(a/the) high tower’
\end{xlist}
\ex {\rm Predicative agreement}
\begin{xlist}
\ex 
\gll 	\.etot 	dom	vysok\\
	\textsc{dem:m} house(\textsc{m}) 	high:\textsc{m}\\
\glt	 ‘this house is high’
\ex
\gll 	\.eta 	bašn'a	vysoka\\
	\textsc{dem:f} tower(\textsc{f}) 	high:\textsc{f}\\
\glt	 ‘this house is high’
\end{xlist}
\end{xlist}
\end{exe}
%%%
In Russian, attributive as well as predicative adjectives show agreement in \textsc{gender} and \textsc{number}. Attributive adjectives agree additionally in \textsc{case}. The agreement suffixes of the attributive and predicative paradigms, however, have different shapes; consider Table \ref{Russian adj agr paradigm}.

Traditionally, the two inflection paradigms of the adjective in Russian have been contrasted to each other as “short” and “long” forms. These terms, however, describe the form rather than the function of the different agreement inflections and are thus less useful for the classification of the Russian noun phrase type from a morpho-syntactic typological perspective. The “long” adjectives of Russian do not simply belong to a different declension paradigm as compared to their “short” counterparts. The formal distinction between the two adjective declensions is connected to attribution marking. Whereas the predicative (“short”) forms show “pure” agreement, the agreement suffixes on attributive adjectives mark agreement and the attributive state of the adjective simultaneously.

Historically, the attributive adjective inflection consists of two morphemes: a pronominal stem plus the original “short” agreement suffix.\footnote{In the forms for nominative (cf.~Table \ref{Russian adj agr paradigm}) the two morphemes for \textsc{attr} and \textsc{gender/number/case} are still separable. In the remaining cases, however, they are merged into one portmanteau suffix.} Synchronically, the attributive adjective suffixes in Russian are thus best analyzed as portmanteau suffixes marking anti-construct and head-driven agreement simultaneously.

One could argue against the analysis of the “long” adjective declension in Russian as attributive state marking saying that “long form adjectives” also occur in predicative position. In fact, the semantic difference between the use of “short” versus “long” forms in adjective predication in Russian could be described as an opposition between temporal and permanent properties denoted by the adjective.
Nonetheless, the marking of the predicative adjective is rather irrelevant here. What is crucial, however, is the use of the “long” forms, which occur in attributive position as a rule. The “short” (i.e.~predicative) form cannot occur in attributive position.

Furthermore, it could even be argued that “long” form adjectives in predicative position are instances of adjective attribution marking rather than of adjective predication. This is the case if one analyses the “long form adjectives” as headless noun phrases in an appositive construction, as the “long” predicative form in (\ref{ru-pred-long}) denoting a permanent property apposed to the “short” predicative form in (\ref{ru-pred-short}) denoting a temporal property.\footnote{Russian examples of morphologically differentiated predicative adjectives also often reflect an opposition in the subject's denotative status. The “short” form is used for denoting reference to a class of objects: \textit{krasavicy \textbf{kaprizn-y}} [capricious-\textsc{pred:agr}] ‘beautiful women are capricious’), the “long” form is used for denoting reference to an individual: \textit{oni \textbf{kaprizn-ye}} [capricious-\textsc{attr:agr}] ‘they are capricious’ (or ‘they are (the) capricious ones’, e.g.~two sisters known from the discourse) (cf.~\citealt[210 Footnote 76]{mendoza2004}).}
%%%
\begin{exe}
\ex
\langinfo{Russian}{Indoeuropean}{own knowledge}
\begin{xlist}
\ex  {\rm “short” predicative adjective}\\
\gll on bolen\\
	3\textsc{sg} ill:\textsc{pred:m}\\
\glt	 ‘he is ill’\label{ru-pred-short}
\ex  {\rm “long” predicative adjective}\\
\gll on bol'nyj\\
	3\textsc{sg} ill:\textsc{attr:m}\\
\glt	 ‘he is a sick one (i.e.~he is mentally sick)’\label{ru-pred-long}
\end{xlist}
\end{exe}
%%%
The origin of anti-construct state agreement marking in Russian is dealt with in Chapter \ref{slavic diachr}. It is worth mentioning that remnants of an Old Slavic anti-construct adjective inflection are found in other modern Slavic languages as well, especially in the South-Slavic languages Slovenian and Serbian where the “long” adjective forms occur in definite noun phrases (cf.~Section \ref{slovenian synchr}).

Similar to South-Slavic but much more regular is the occurrence of a cognate anti-construct adjective inflection in the Baltic languages Latvian and Lithuanian.
%%%
\begin{exe}
\ex
\langinfo{Latvian}{Indoeuropean}{\citealt[115]{dahl2007}}
\begin{xlist}
\ex	\label{latvian indef}
\gll liel-a māja\\
	big-\textsc{f.nom.sg} house(\textsc{f})\\
\glt	‘a large house’
\ex	\label{latvian def}
\gll liel-ā māja\\
	big-\textsc{attr:f.nom.sg} house(\textsc{f})\\
\glt	‘the large house’
\end{xlist}
\end{exe}
%%%
Unlike in Russian where attributive adjectives are marked with the anti-construct state agreement suffixes as a rule, the use of the cognate attributive forms in the Baltic languages is usually described as depending on the referential status of the head noun. Whereas the “short form” agreement suffix is used with adjectives modifying indefinite nouns (\ref{latvian indef}), the attributive adjective in definite noun phrases is obligatorily marked with the “long form” agreement suffix (\ref{latvian def}).

The anti-construct state agreement marking suffixes in the Baltic languages is often described as a definiteness marker. Note, however, that the definite noun never exhibits definite marking itself. If no attributive adjective is present the definite noun remains unmarked. The analysis of the “long form” agreement suffix in Baltic as definite marker would thus presuppose the assumption that the definite marker is selective and shows up only on attributive adjectives. 

Markers which are selective according to their host's parts-of-speech membership are indeed attested.\footnote{Consider, e.g., the two allomorphs of the definite marker in Danish \textit{hus\textbf{-et}} [house-\textsc{def.n}] ‘the house’, \textit{\textbf{det} store hus} [\textsc{def.n} big\textsc{.def.n} house] ‘the large house’. The suffix \textit{-et} \textsc{def.n.} attaches to bare nouns, whereas the free form \textit{det} \textsc{def.n} attaches to noun phrases with adjective modifiers, cf.~also Table \ref{danish defallomorph}.} The Latvian and Lithuanian examples, however, could be compared to selective marking in other languages only if one assumes a zero-allomorph of the definiteness marker attaching to non-modified definite nouns.
%%%
\begin{exe}
\ex
\langinfo{Latvian}{Indoeuropean}{\citealt[115]{dahl2007}}
\begin{xlist}
\ex
\gll 	māja\\
	house\\
\glt	‘a house’
\ex	
\gll 	māja\textbf{-?Ø}\\
	house-\textsc{def}\\
\glt	‘the house’
\ex		
\gll 	liel\textbf{-ā} māja\\
	big-\textsc{def:f.nom.sg} house(\textsc{f})\\
\glt	‘the large house’
\end{xlist}
\end{exe}
%%%
\citet[31]{melcuk1998} introduced the term \emph{displaced category} (Russian \emph{smeščennaja kategorija}) for the type of marking found in Baltic. It has also been argued by Dahl (\citeyear[149–152]{dahl2003}; see also \citealt[115]{dahl2007}) that definite noun phrases often show special behavior in languages depending on whether or not they exhibit attributive adjectives (or other modifiers).\footnote{\citet[150]{dahl2003} compares the “long form” adjectives in the Baltic languages with attributive articles in Romance languages (such as in Latin \textit{Babylon illa magna}) and Yiddish, among others. A structural and even historical connection is indeed plausible, as will be shown in Part \ref{part diachr} of this study, especially in Section \ref{ie diachr}.}

An alternative analysis is preferred here: Since the “long form” agreement suffix only attaches to attributive adjectives, the formative could well be analyzed as an anti-construct state agreement marker (similar to Russian) which is, however, restricted to occurring in semantically definite noun phrases. 

Several examples of languages are attested where the occurrence of different noun phrase types is restricted to certain subsets of noun phrase constituents. In the case of the Latvian example given above (and similar to Lithuanian) attributive adjectives are marked differently depending on the referential status of the whole phrase. The choice between the head-driven agreement versus the anti-construct state agreement type would thus be constrained by the semantically defined subsets of the noun head (i.e.~indefinite versus definite). 

As a consequence of the suggested analysis of the “long form” agreement suffixes in Baltic as anti-construct state agreement markers, Latvian and Lithuanian could be described as lacking definiteness as morphological category. In fact, several authors have questioned the existence of morphologized definite marking at least in Lithuanian, where the occurrence of the anti-construct state agreement suffix is clearly not restricted to definite noun phrases (cf.~\citealt{wissemann1958} cit. \citealt[181–182]{kramsky1972}). \citet[37]{trost1966} argues that permanent versus non-permanent properties are marked rather than definite versus indefinite, for example (Lithuanian) \textit{aukštoji mokyla} ‘college (lit. ‘high school’)’.\footnote{For Latvian, however, \citet[38]{trost1966} accepts the analyses of the “long” suffix as definite marker because it occurs regularly after possessive pronouns.}

In Chapter \ref{slavic diachr}, diachronic arguments will be presented in favor of the assumption that a morphological feature \textsc{species} (with the values \textsc{definite} /~\textsc{indefinite}) was not present in Baltic languages, at least until the most recent stages in their language history. The anti-construct state agreement inflection is clearly older than the morphologization of definiteness in Baltic (and similarly in certain Slavic languages). In older stages of Baltic (and Slavic) the “long” adjective inflection was connected to attributive rather than to definiteness marking. To a certain extent, this holds true for the modern Baltic languages Latvian and Lithuanian.

Thus, in the ontology presented here anti-construct state agreement marking in Baltic belongs to the same noun phrase type as the one described for Russian (cf.~example \ref{ru-anti} on page \pageref{ru-anti}). This analysis seems justified regardless of the question as to whether the device constitutes the default type of adjective attribution marking (as in Russian) or is restricted to a given semantically restricted subset of the head noun (as in Latvian and Lithuanian).

Also in German (similar to the other West-Germanic languages, except English), attributive and predicative adjectives are morpho-syntactically differentiated. Whereas attributive adjectives show head-driven agreement, predicative adjectives are used in an invariable form. Given the definition of dependent marking attributive state which was applied here (see also Chapter \ref{syntax-morphology-interface}), German thus exhibits a similar type of obligatory anti-construct state agreement marking as Russian. Note, however, that the adjective inflection suffixes in German are merged to a relatively high degree: Only the five single forms \textit{-e, -en, -em, -er, -es} are formally distinguished. 

What is even more interesting in German is the fact that the agreement feature \textsc{species} exhibits a third value for which a grammatical label is hard to find. Whereas indefinite agreement shows up on adjectives in semantically indefinite noun phrases (formally marked by the indefinite marker \textit{ein} in Table \ref{german agr}) and definite agreement on adjectives occurs in semantically definite noun phrases (formally marked by the definite marker \textit{der} in Table \ref{german agr}), the “third species“ agreement forms show up in semantically indefinite or definite noun phrases marked, for instance, by possessive pronouns and the indefinite pronoun \textit{kein} ‘no(t any)’. Whereas the “third species“ agreement forms are similar to the indefinite forms in singular, they are similar to the definite forms in plural. Accordingly, three species values thus have to be distinguished in the morphological paradigm.

It is worth mentioning that adjectives which are simultaneously marked for attributive state (i.e.~anti-construct) and head-driven agreement are also attested in languages outside the northern Eurasian area. Similar to Russian, adjectives in Endo, a Nilotic language of Kenya, require different agreement suffixes depending on their use as modifiers of a noun or as predicates.
%%%
\begin{landscape}
\begin{table}
\begin{footnotesize}
\begin{tabular}{|ll|lll|lll|lll|lll|}
\lsptoprule
&&\multicolumn{3}{|c|}{\textsc{m.sg}}&\multicolumn{3}{|c|}{\textsc{f.sg}}&\multicolumn{3}{|c|}{\textsc{n.sg}}&\multicolumn{3}{|c|}{pl}\\
\midrule
&\textsc{nom}&(ein)&gut\textbf{-er}&(Mann)&(ein-e)&gut\textbf{-e}&(Frau)&(ein)&gut\textbf{-es}&(Kind)&&gut\textbf{-e}&(Leute)\\
\textsc{indef}&\textsc{gen}&(ein-es)&gut\textbf{-en}&(Mannes)&(ein-er)&gut\textbf{-en}&(Frau)&(ein-es)&gut\textbf{-en}&(Kind-es)&&gut\textbf{-er}&(Leute)\\
&\textsc{dat}&(ein-em)&gut\textbf{-en}&(Mann)&(ein-er)&gut\textbf{-en}&(Frau)&(ein-em)&gut\textbf{-en}&(Kind)&&gut\textbf{-en}&(Leuten)\\
&\textsc{acc}&(ein-en)&gut\textbf{-en}&(Mann)&(ein-e)&gut\textbf{-e}&(Frau)&(ein)&gut\textbf{-es}&(Kind)&&gut\textbf{-e}&(Leute)\\
\midrule
&\textsc{nom}&(der)&gut\textbf{-e}&(Mann)&(die)&gut\textbf{-e}&(Frau)&(das)&gut\textbf{-e}&(Kind)&(die)&gut\textbf{-en}&(Leute)\\
\textsc{def}&\textsc{gen}&(des)&gut\textbf{-en}&(Mannes)&(der)&gut\textbf{-en}&(Frau)&(des)&gut\textbf{-en}&(Kind-es)&(der)&gut\textbf{-en}&(Leute)\\
&\textsc{dat}&(dem)&gut\textbf{-en}&(Mann)&(der)&gut\textbf{-en}&(Frau)&(dem)&gut\textbf{-en}&(Kind)&(den)&gut\textbf{-en}&(Leuten)\\
&\textsc{acc}&(den)&gut\textbf{-en}&(Mann)&(die)&gut\textbf{-e}&(Frau)&(das)&gut\textbf{-e}&(Kind)&(die)&gut\textbf{-en}&(Leute)\\
\midrule
&\textsc{nom}&(mein)&gut\textbf{-er}&(Mann)&(meine)&gut\textbf{-e}&(Frau)&(mein)&gut\textbf{-es}&(Kind)&(meine)&gut\textbf{-en}&(Leute)\\
\textsc{in/def}&\textsc{gen}&(meines)&gut\textbf{-en}&(Mannes)&(meiner)&gut\textbf{-en}&(Frau)&(meines)&gut\textbf{-en}&(Kind-es)&(meiner)&gut\textbf{-en}&(Leute)\\
&\textsc{dat}&(meinem)&gut\textbf{-en}&(Mann)&(meiner)&gut\textbf{-en}&(Frau)&(meinem)&gut\textbf{-en}&(Kind)&(meinen)&gut\textbf{-en}&(Leuten)\\		
&\textsc{acc}&(meinen)&gut\textbf{-en}&(Mann)&(meine)&gut\textbf{-e}&(Frau)&(mein)&gut\textbf{-es}&(Kind)&(meine)&gut\textbf{-en}&(Leute)\\	
\lspbottomrule
\end{tabular}
\end{footnotesize}
\caption[Adjective paradigm for \textsc{German}]{Agreement paradigm for the \textsc{German} adjective ‘good’ (‘good man’ \textsc{m}, ‘good woman’ \textsc{f}, ‘good child’ \textsc{n}, ‘good people’ \textsc{pl})} \label{german agr}
\end{table}
\end{landscape}
%%%
\begin{exe}
\ex
\langinfo{Endo}{Nilotic}{\citealt[65]{zwarts2003}}
\begin{xlist}
\ex
\gll 	karaam 	inyeentee\\
	good(\textsc{sg}) \textsc{3sg}\\	
\glt	‘S/he is good.’
\ex	
\gll 	laakwa 	nyaa 		karaam\\
	child 	\textsc{attr:sg} 	good(\textsc{sg})\\	
\glt	‘a good child’
\ex	
\gll 	karaam-a 	akwaaneek\\
	good-\textsc{pred:pl} 	\textsc{3pl}\\
\glt	‘They are good.’
\ex	
\gll 	piich 	chaa 		karaam-een\\
	people 	\textsc{attr:pl} 	good-\textsc{attr:pl}\\
\glt	‘good people’
\end{xlist}
\end{exe}
%%%
The example illustrates that adjectives in Endo show agreement in number. The singular is unmarked and the plural is marked by the suffix \textit{-a} for predicative adjectives and by \textit{-een} for attributive adjectives.\footnote{Unlike in Russian, however, there is a second attributive marker present in Endo, an attributive article \textit{nyaa} \textsc{attr:sg}, \textit{chaa} \textsc{attr:pl}. The noun phrase type would thus better be characterized as a combination of attributive article+anti-construct state agreement, hence “double agreement”.}

\subsubsection{Attributive nominalization} \label{attr nmlz}
%%%
Nominalization is often understood very broadly as a word-class changing morphological operation deriving nouns from other syntactic classes. This definition stresses the lexical-semantic side of nominalization. But the term is sometimes also used for a syntactic operation in which a verbal (single or complex) constituent, like a verb, a verb phrase, a sentence, or a portion of a sentence (including a verb) is converted into a nominal (single or complex) constituent \citep[575]{li-etal1981}. In this latter sense, nominalization is a means of licensing nominal constituency.

Mandarin Chinese illustrates a language in which syntactic nominalization is a highly polyfunctional device for the licensing of different modifying phrase constituents (cf.~\citealt[575–593]{li-etal1981}; see also example \ref{multi mand} in Chapter \ref{polyfunctionality}). Adjectives in Mandarin are used in attributive position (\ref{mandarin attr}), in predicative position (\ref{mandarin pred}) and when used as adverbial modifiers (\ref{mandarin adv}).
%%%
\begin{exe}
\ex
\langinfo{Mandarin Chinese}{Sinotibetan}{\citealt{li-etal1981}}
\begin{xlist}
\ex	{\rm Adjectival attribute}\\
\gll	$[_{NP}$ xīn 		\textbf{de}$]$ 	shū\\
	{} new	 	\textsc{nmlz}  	book\\
\glt	‘new book’\label{mandarin attr}
\ex	{\rm Adjectival predicate}\\
\gll	wǒ-de shū shì $[_{NP}$ xīn \textbf{de}$]$\\
	\textsc{1sg-nmlz} book \textsc{cop} {} new \textsc{nmlz}\\
\glt	‘My book is new (i.e.~a new one).’\label{mandarin pred}
\ex	{\rm Adjectival adverb}\\
\gll	wǒ $[_{NP}$ yánli\textbf{-de}$]$ zébèi tā le\\
	\textsc{1sg} {} stern\textsc{-nmlz} reproach \textsc{3sg} \textsc{crs}\\
\glt	‘I sternly (i.e.~as a stern one) reproached him/her.’ \label{mandarin adv}
\end{xlist}
\end{exe}
%%%
Interestingly, nominal constituents can also be nominalized, i.e.~they can be syntactically licensed as constituents in larger syntactic units. In some languages, such syntactic licensing is obligatory for certain types of nominals. The respective markers (i.e.~nominalizers of nominals) are labelled with quite different terms, such as, for instance, “articles”, “noun phrase articles” or “noun (phrase) markers” (cf., e.g., \citealt[152]{dryer2007}, \citealt[95, elsewhere]{rijkhoff2002}). Prototypical examples of such markers come from Oceanic languages where noun phrases contain an obligatory nominalizer deriving from a demonstrative. 

Due to lack of a conventionalized terminological distinction, “nominalization” is here used for denoting the purely syntactic operation by which a noun or noun phrase is marked as a syntactic constituent by making it syntactically more complex, i.e.~by projecting a full noun phrase. This use of the term \emph{nominalization} is also consistent with the fact that “nominal” is most often used as a homonym for “noun phrase” rather than for “noun”. “Substantivation”, on the other hand, will be used for the purely morpho-semantic (derivational) process yielding a noun (substantive). Whereas substantivation belongs to the spheres of morpho-semantics and lexicon, nominalization belongs to syntax: Nominalizers function exclusively for the licensing of noun phrases as constituents in larger syntactic units. 

\emph{Attributive nominalization} has already been discussed as “appositive modification” in Section \ref{apposition}. Attributive nominalization is a special subtype of dependent marking construct state. Similar to the latter, attributive nominalization represents a covert dependent-marking morpho-syntactic device and is triggered either by purely syntactic government (as, e.g., anti-construct state marking in Kildin Saami, see section \ref{dep-marking state}) or by syntactic government in combination with head-driven agreement (as, e.g., anti-construct state agreement marking in Russian, see section \ref{anti-constr agr}). The special distinguishing characteristic of attributive nominalization lies in the syntactic structure: Whereas true anti-construct state markers attach directly to the dependent constituent (as, e.g., the respective inflectional suffixes in Kildin Saami or Russian), attributive nominalizers attach to an intermediate dependent phrasal constituent between the head noun and the modifier.

Epithet-constructions with attributive articles in Germanic languages illustrate a prototypical case of attributive nominalization by means of an article.\footnote{The examples are from \citet[179–180]{himmelmann1997}. Note that attributive nominalization in German is restricted to noun phrases with proper names as heads. This restriction is, however, irrelevant to the following argumentation.}
%%%
\ea  \label{german epithet}
\langinfo{German}{Indoeuropean}{own knowledge}\\
Friedrich der Große 			\jambox{\rm ‘Frederick the Great’}
\z
%%%
Following \citet[180]{himmelmann1997}, the syntactic structure of this example can be described as follows:
%%%
\ea	$[$$_{NP}$ Friedrich $[$ $_{NP'}$ $_{ART}$der $_{A}$Große $]]$
\z
%%%
The intermediate phrasal constituent between the noun phrase (NP) and the adjective is labeled as NP', leaving open the rather theoretical question about what constitutes the syntactic head of this phrasal projection.\footnote{“Article phrase” (similar to “Determiner phrase” in X-bar syntax) would imply the nominalizer (in this case the article \textit{der}) is the head.}

Note that the attributive marker \textit{der} in example \ref{german epithet} is homophone with the definite marker \textit{der} but clearly has a different function in this construction. For instance, the attributive marker \textit{der} cannot be exchanged with a possessive or a demonstrative pronoun and is thus not a marker of definiteness. The proper noun \textit{Friedrich}, on the other hand, can be further determined by means of a demonstrative (\textit{\textbf{jener} Friedrich der Große} ‘that Frederick the Great’) or a possessive pronoun (\textit{\textbf{unser} Friedrich der Große} ‘our Frederick the Great’). In fact, (in-)definiteness marking of the whole noun phrase does not affect the attributive nominalizer, consider the following example:
%%%
\begin{exe}
\ex
\langinfo{German}{Indoeuropean}{own knowledge}
\begin{xlist}
\ex	Irgendein $[$Friedrich der Große$]$$_{\textit{indef.nom}}$ soll das gesagt haben.
\ex	Dieser $[$Friedrich der Große$]$$_{\textit{def.nom}}$ soll das gesagt haben. 
\ex	Ich sehe mir irgendeinen $[$Friedrich den Großen$]$$_{\textit{indef.acc}}$ an.
\ex	Ich sehe mir diesen $[$Friedrich den Großen$]$$_{\textit{def.acc}}$ an.
\end{xlist}
\end{exe}
%%%
The attributive adjective forms a complex constituent together with the article. This complex constituent is subordinated to the noun phrase head (i.e.~the proper name \textit{Friedrich}) whom it modifies. Agreement in gender\fshyp{}number\fshyp{}case is triggered (through agreement) by the head noun (\textit{Friedrich})%??
%The definite and case marking of the adjective \textit{groß-e} \textsc{nom.m}\fshyp{}\textit{groß-en} \textsc{acc.m} and of the article \textit{der} \textsc{nom.m}\fshyp{}\textit{den} \textsc{acc.m} , which is a proper noun and consequently definite. It ). 
The agreement pattern in the German epithet-construction also show that the nominalizer \textit{der} has not only to be distinguished from the homophone definite marker but also from the relativizer \textit{der}. Consider the following examples (cf.~also \citealt[181]{himmelmann1997}).
%%%
\ea \label{article versus rel}
\langinfo{German}{Indoeuropean}{own knowledge}
\ea[*]	 {ein Jagdhund Friedrichs der Große}
\ex[]	{ein Jagdhund Friedrichs des Großen}
\ex[]	{die Jagdhunde Friedrichs, der seine Sommerresidenz in Potsdam hatte}
\ex[]	{die Jagdhunde Friedrichs, den man auch den Alten Fritz nannte}
\z
\z
%%%
According to Lehmann (\citeyear[230–231]{lehmann1984}; cf.~also \citealt[181]{himmelmann1997}) true relative pronouns represent the syntactic head for the predicate of the embedded clause. The syntactic function of the relative pronoun is determined by the predicate, but it is independent from the syntactic function of the head noun. Consequently, the relativizer \textit{der} (similar to the adjective \textit{groß}) in example (\ref{article versus rel}) agrees only in gender and number with the head noun \textit{Friedrich}. Case is alloted according to the function of \textit{der} as argument in the embedded clause. This is different from the syntactic function of the attributive nominalizer \textit{der}. The nominalizer does agree in case with the head noun. The article's syntactic function is thus dependent of the head noun's function in the superordinate construction.

\subsubsection{Attributive articles} \label{attr art}
%%%
Attributive nominalizers similar to \textit{der} in German epithet-constructions will be labeled \emph{attributive articles} in the following. Attributive articles are similar to anti-construct state agreement markers in that they mark the syntactic relation of attribution and agreement simultaneously. Prototypically, attributive articles are grammatical words and hence syntactic constituents on their own. In the case of the German attributive article \textit{der}, the constituency of the marker becomes evident in the fact that both the adjective and the article are the target of head-driven agreement.

Even though “article” is often used for many different types of grammatical markers, this term (<Latin \emph{artus\fshyp{}articulus} ‘joint, small connecting part’) originally referred to the metaphor of a joint between the constituents in a noun phrase, hence a true attribution marker. Interestingly, \citet[83]{dryer1989a} and \citet{rijkhoff2002} distinguish two types of “articles”: (1) words indicating (in-) definiteness (or some related discourse notion) and (2) words serving as a noun phrase marker “in the sense that noun phrases in that language [\dots] typically occur with one of the words in question” \citep[285]{rijkhoff2002}. Attributive articles could nicely be subsumed under type (2) “Noun phrase marker” if the definition would be extended:  “a marker which occurs with noun phrases \textbf{and\fshyp{}or phrasal dependent constituents of noun phrases}”.

The term \emph{attributive article} used here matches Himmelmann's (\citeyear{himmelmann1997}) \emph{Gelenkartikel} ‘linking article’, which in turn is borrowed from Gamillscheg's (\citeyear{gamillscheg1937}) description of the “linking function” (\emph{Gelenksfunktion}) of articles in different Indoeuropean languages.\footnote{In Himmelmann's \citeyear{himmelmann1997} terminology, however, the attributive or linking article is a subtyp of a class of grammatical words (which he calls “operators”), which are labeled \emph{articles}. Other subtypes of this class are definite, indefinite and other types of (non-attributive) grammatical markers.} 

Even though the use of the term \emph{article} by Indoeuropeanists is often applied in grammatical descriptions of different languages and even in theoretical linguistic studies, the present study prefers to use \emph{article} only for denoting an attributive marker. On the basis of examples from Greek (with the so-called repeated article) and from Latin (with the so-called linking demonstrative), \citet[48]{gamillscheg1937} characterizes the attributive article as exhibiting “a disjunctive and linking function simultaneously”\footnote{“[\dots] zugleich trennende und verbindende Funktion [\dots]”} by marking the adjective as “physically independent.”\footnote{“[\dots] physisch selbständig [\dots]”} The articles \textit{ille} in Latin and \textit{tó} in Greek thus have different functions than the homophone demonstratives\fshyp{}definite markers in that the article nominalizes an adnominal constituent in order to function as attribute of a certain kind. The homophone demonstrative\fshyp{}definite marker on the other hand, marks the whole noun phrase for certain values of the feature \textsc{species}.

While the use of attributive articles in German, English and several other Indoeuropean languages is restricted to epithet-constructions, a similar construction with an attributive article occurs much more unrestrictedly in Yiddish.
%%%
\begin{exe}
\ex
\langinfo{Yiddish}{Indoeuropean}{\cite{jacobs-etal1994}}\label{yiddish attr appos}
\begin{xlist}
\ex 
\gll 	di grin-e oyg-n\\
	\textsc{def.pl}	green-\textsc{def.pl} eye-\textsc{pl}\\
\glt	‘the green eyes’\label{yiddish agr a}
\ex
\gll 	di oyg-n di grin-e\\
	\textsc{def.pl} eye-\textsc{pl} \textsc{attr.def.pl} green-\textsc{def.pl}\\
\glt	‘the GREEN eyes’\label{yiddish attr defarticle}
\ex	
\gll	'n grin-et oyge\\
	\textsc{indef.n} green-\textsc{indef.n} eye(\textsc{n})\\
\glt	‘a green eye’\label{yiddish agr b}
\ex	
\gll	'n oyge 'n grin-et\\
	\textsc{indef.n} eye(\textsc{n}) \textsc{attr.indef.n} green-\textsc{indef.n}\\
\glt	‘a GREEN eye’\label{yiddish attr indefarticle}
\end{xlist}
\end{exe}
%%%
In the default attributive construction in Yiddish, the adjective precedes the noun which also triggers agreement on the adjective (\ref{yiddish agr a}+\ref{yiddish agr b}). In an emphatic construction and postponed to the head noun, however, the attributive adjective is marked with an article (\ref{yiddish attr defarticle}+\ref{yiddish attr indefarticle}) \citep[342–347]{plank2003}.

Yiddish thus shows that attributive articles can have a much broader use than for example in German. But even in Yiddish the use of the attributive article is subject to restrictions. In this case, the restriction is of a semantic nature and is due to the referential status of the adjective. In order to occur in an attributive nominalization construction the adjective must be in contrastive focus.

A similar rule applies to Greek, where the so-called repeated article also occurs in contrastive-focus constructions. 
%??PERHAPS LATER: Unlike in Yiddish, where the article is used only with modifying adjectives, the article is polyfunctional in Greek and does mark modifying nouns (genitives) and...
%%
\begin{exe}
\ex
\langinfo{(Modern) Greek}{Indoeuropean}{\citealt{ruge1986}} \label{greek noun phrase}
\begin{xlist}
\ex
\gll 	i kondés fústes\\
	\textsc{def} short skirts\\
\glt	‘the short skirts’
\ex 
\gll 	i fústes i kondés\\
	\textsc{def} skirts \textsc{attr} short\\
\glt	‘the SHORT skirts’
\end{xlist}
\end{exe}
%%%
Note that the the two phrases in the attributive apposition constructions (i.e.~attributive nominalization) of German (Section \ref{attr nmlz}), Yiddish (\ref{yiddish attr appos}) and Greek (\ref{greek noun phrase}) cannot be re-arranged unless the whole construction yields a different reading. In the case of the epithet-construction in German, re-arrangement of adjective and noun would result in a simple noun phrase with an attributive adjective which is, however, no longer an epithet. Re-arrangement of the constructions in Yiddish and Greek would result in true noun phrase appositions.

\paragraph{Attributive articles as subtype of attributive nominalizers}
%%
Attributive articles have been characterized as grammatical words and agreement targets. In accordance with the common practice of labelling an unchangeable, non-bound grammatical marker “particle”, the attributive nominalizer \textit{the} in English (epen\-thet constructions) would fall into this category because it is not an agreement target.\footnote{Consider also Himmelmann's (\citeyear{himmelmann1997}) “Gelenkartikel” versus “Gelenkpartikel”.}

In the present survey, however, there are only a few examples of languages with attributive, non-article nominalizers attested, among them Ket (cf.~Section \ref{yeniseian synchr}) and Dungan (cf.~Section \ref{sinotibetan synchr}) where the respective markers seem to constitute affixes rather than particles.

In the present ontology, attributive articles are defined as a subclass of attributive nominalizers. Whereas attributive nominalizers are construct markers (belonging to pure morpho-syntax), articles have an additional semantic component because they undergo agreement. %Their word-hood (inflected or uninflected grammatical word, affix, etc.) is irrelevant.

\paragraph{D-Elements which are not nominalizers}
In the previous section, attributive articles and other attributive nominalizers have been described and attributive nominalizers have been characterized as a special subtype of anti-construct state markers which attaches to an intermediate dependent phrasal constituent between the head noun and the modifier.

Somewhat similarly, Himmelmann \citeyear{himmelmann1997} describes attributive articles and other attributive nominalizers as D(eterminer) elements between head and attribute\footnote{“D(eterminer)-Element zwischen Kopf und Attribut”}. Illustrating attributive nominalization with examples from several languages, the author shows that these markers prototypically originate from adnominally grammaticalized local deictic pronouns used as functional heads of nominalizer phrases. Himmelmann does not, however, clearly distinguish between synchronic and diachronic evidence and considers both attributive nominalizers (such as the “repeated article” in Greek), agreement markers (such as the so-called “adjective article” in Albanian and even linkers (as in Tagalog) as D-elements.

The linker in Tagalog is not an article (not even an attributive nominalizer) according to the present ontology of attribution marking devices because the marker is floating, with a locus neither on-dependent or on-head, and it does not project a noun phrase (cf.~Section \ref{linker} in Part \ref{part typ}). Examples of agreement marking “D-Elements” come from Swedish and Albanian.
%%
\begin{exe}
\ex
\begin{xlist}
\ex 
\langinfo{Swedish}{Indoeuropean}{own knowledge}\\
\gll	\textbf{den} goda vännen\\
	\textbf{\textsc{attr:def.sg.utr}} good:\textsc{def.sg.com} friend:\textsc{def.sg.com}\\
\ex 
\langinfo{Albanian}{Indoeuropean}{\citealt[166–167]{himmelmann1997}}\\
\gll	shoku \textbf{i} mirë\\
	friend:\textsc{def:nom.sg.m} \textbf{\textsc{nmlz:nom.sg.m}} good:\textsc{nom.sg.m}\\
\glt	‘the good friend’
\end{xlist}
\end{exe}
%%%
Whereas the agreement marking “D-Element” in Albanian is perhaps a nominalizer, the markers in Swedish (and other languages) might simply be construct-state agreement markers from a purely synchronic point of view because they do not occur in attributive apposition constructions, i.e.~they do not project noun phrases (cf.~Sections \ref{albanian synchr} for Albanian and \ref{swedish synchr} for Swedish). From a diachronic point of view, however, these markers clearly originate from absolutely similar attributive nominalizers. Consequently, the grammaticalization path suggested by Himmelmann \citeyear{himmelmann1997} can even be extended with an additional stage: from “D-elements” to attributive articles (or other attributive nominalizers) to construct-state markers, as will be shown in the diachronic part \ref{part diachr}.

From a purely synchronic point of view, however, the different types of \emph{anti-construct state agreement} and \emph{attributive article} might not always be easily distinguishable from each other or from \emph{head-driven agreement}. The first two often include some “article notion” (sometimes connected to definiteness or other referential values), and all three types include agreement marking. “Pure” agreement marking, however, cannot include the feature \textsc{state} (construct marking). A simple test is whether or not attributive adjectives show different agreement marking than predicative adjectives. If they do, as, e.g.~in Russian, construct marking is involved. If construct marking undergoes agreement and additionally projects a full noun phrase, as, e.g.~the article in Germanic epithet constructions, than the type of marking is best characterized as attributive article.

\subsection{Head+dependent marking attributive state}
This combined type refers to state marking which has two loci: on-head and on-dependent simultaneously. A language spoken outside the northern Eurasian area which gives an example of this noun phrase type is the Toreva dialect of Hopi.
%%%
\begin{exe}
\ex 
\langinfo{Hopi, Toreva}{Uto-Aztecan}{\citealt{whorf1946}}
\begin{xlist}
\ex	\label{hopi adjective}
\gll caˑva\\
	is\_short\\
\ex	
\gll pọ̀yo\\
	knive\\
\ex	\label{hopi adj attribution}
\gll caˑv vọ̀yo\\
 	is\_short\textbackslash\textsc{attr} knive\textbackslash\textsc{attr}\\
\glt	‘a short knive’
\end{xlist}
\end{exe}
%!!in meinen Notizen von Whorf fehlt die pred. Form des Adj.
%%%
According to \citet[178]{whorf1946} both the adjective modifier (which is a stative verb in Hopi) and the noun head alter their phonological shapes regarding whether or not they are used in predication or as constituents in a noun phrase. Consider the noun phrase in example (\ref{hopi adj attribution}) where the modifier \textit{caˑva} ‘is short’ occurs with a shortened stem form (compared to \ref{hopi adjective}) and the noun is marked by means of lenition of the word-initial consonant (\textit{pọ̀yo} ‘knive’ versus \textit{vọ̀yo} [knive\textbackslash\textsc{attr}]).

The noun phrase type in Hopi is thus best analyzed as attributive state marking in which both the noun head and the adjective dependent are construct marked. Note, however, that in contrast to the other mentioned examples of different types of state markers, the respective formatives in the noun phrase of Hopi are non-concatenative morphemes represented by stem alternations.

Double (head+dependent) construct state marking is also attested as adjective attribution marking device in one language of northern Eurasia. In Northern Saami, two adjectives meaning ‘little’ govern diminutive marking on the head noun. Noun phrases with these two adjectives are ungrammatical if diminutive marking on the noun is missing.
%%%
\ea
\langinfo{Northern Saami}{Uralic}{own knowledge}\\
\ea[]	{
	{\rm Diminutive derivation}\\
\gll	guolli / guolá-š / guolá-ža-t\\
	fish {} fish-\textsc{dim} {} fish-\textsc{dim}-\textsc{pl}\\
\glt	‘fish’ / ‘little fish’ / ‘little fishes’\label{dim n}
	}
\ex[]	{
	{\rm Anti-construct state marking (‘big’)\footnote{State marking of ‘big’ is non-concatenative and affects the quantity of the stem consonants and the quality and quantity of the stem-final vowel, cf.~the same adjective inflected for predicative state (agreement): \textit{guolli/guoláš lea \textbf{stuoris}} [\textsc{pred:sg}] ‘the fish/little fish is big’; \textit{guolit/guolážat leat \textbf{stuorrát}} [\textsc{pred:pl}] ‘the fishes/little fishes are big’.}}\\
\gll	\textbf{stuorra} guolli / guoli-t / guolá-š / guolá-ža-t\\
	big:\textsc{attr} fish {} fish-\textsc{pl}  {} fish-\textsc{dim} {} fish-\textsc{dim}-\textsc{pl}\\
\glt	‘big fish’ / ‘big fishes’ / ‘big little-fish’ / ‘big little-fishes’\label{dim np}
	}
\ex[]	{
	{\rm Double-construct state marking (‘little’)\footnote{State marking of ‘little’ is non-concatenative and affects the quantity of the stem consonants and the quality and quantity of the stem-final vowel, cf.~the same adjective inflected for predicative state (agreement): \textit{guolli/guolá-š lea \textbf{unnni}} [\textsc{pred:sg}] ‘the fish/little fish is little’; \textit{guolit/guolážat leat \textbf{unni}} [\textsc{pred:pl}] ‘the fishes/little fishes are small’.}}\\
\gll	\textbf{unna} guolá\textbf{-š} / guolá\textbf{-ža}-t\\
	smal:\textsc{attr} fish-\textsc{dim} {} fish-\textsc{dim}-\textsc{pl}\\
\glt	‘small fish’ / ‘small fishes’\label{dim state}
	}
\ex[*]{	
\gll	\textbf{unna} guolli / guoli-t\\
	small:\textsc{attr} fish {} fish-\textsc{pl}\\
	}
	\z
\z
%%%
Diminutive is a derivational category in Northern Saami. Normally it is assigned semantically to the noun and thus belongs to the morphological features, as in (\ref{dim n}+\ref{dim np}). However, diminutive can in fact also be a morpho-syntactic feature in Northern Saami, namely when it is obligatorily governed by one of the two attributive adjectives \textit{unna} or \textit{uhca} ‘little, small (attr.)’, as in (\ref{dim state}). However marginal these examples seem to be, diminutive is assigned syntactically on the head by the dependent and thus also belongs to the morpho-syntactic features in Northern Saami.

\subsection{Neutral attributive state} \label{linker}
%%
The term \emph{neutral marking} was introduced by Nichols (\citeyear{nichols1986}) in her typology of head marking versus dependent marking grammar. \emph{Neutral marking} refers to a marker's locus neither on-head nor on-dependent. This means that the marker floats in the noun phrase depending on the actual order of constituents. A floating state marker occurs, for instance, in Tagalog.
%%
\begin{exe} 
\ex 
\langinfo{Tagalog}{Austronesian}{\citealt{rubin1994}} \label{tagalog linker}
\begin{xlist}
\ex 	{\rm Predication}\\
\gll Maganda ang bahay.\\
	beautiful \textsc{top} house\\
\glt	‘The house is beautiful.’\label{tagalog adj predication}
\ex	{\rm Attribution (adjective-noun)}\\
\gll maganda\textbf{-ng} bahay\\
	beautiful-\textsc{attr} house\\
\glt	‘beautiful house’\label{tagalog adj attribution AN}
\ex	{\rm Attribution (noun-adjective)}\\
\gll bahay \textbf{na} maganda\\
	house \textsc{attr} beautiful\\
\glt	‘beautiful house’ \label{tagalog adj attribution NA}
\end{xlist}
\end{exe}
%%%
In the Tagalog noun phrase, the combination of noun and modifier is licens\-ed by the attributive state marker \textit{-ng}.\footnote{After consonants the allomorph \textit{na} is used.} The marker occurs with attributive adjectives (\ref{tagalog adj attribution AN} and \ref{tagalog adj attribution NA}) but not with predicative ones (\ref{tagalog adj predication}).\footnote{The state marker in Tagalog is polyfunctional in the sense that it also marks attribution of demonstratives, numerals and other modifiers \cite[160–161]{himmelmann1997}. See also below Chapter \ref{polyfunctionality}.}

The two types of adjective attribution in Tagalog (\ref{tagalog adj attribution AN} and \ref{tagalog adj attribution NA}) are distinguished from each other only by word order of the head noun and the modifying adjective. The attribution marker follows the first constituent, regardless of whether this is the modifier or the noun. The attribution marker in Tagalog behaves thus like a second-position clitic (\citealt[65]{nichols1986}; see also \citealt[160, 162]{himmelmann1997}).

In the typology presented here only a floating state marker, i.e.~an overt state marker which behaves neutrally with regard to its locus and is neither head- nor dependent marking, is considered to be a true \emph{linker}. The occurrence of such an attribution marking device is not attested among the northern Eurasian languages investigated for the present study. However, since \emph{linkers} and \emph{articles} (but even other attribution marking devices) are sometimes not clearly distinguished in terminology (see below Section \ref{attr art}), it seems rather relevant to characterize this noun phrase type here.

%\end{document}